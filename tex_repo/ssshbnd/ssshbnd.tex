%%%%%%%%%%%%%%%%%%%%%%%%%%%%%%%%%%%%%%%%%%%%%%%
%
%    Self-Similar shear bands, Existence, Numerics, Asymptotics
%
%                                                      by
%
%                                       Min-Gi Lee   
%
%                                          version Sep 2016
%
%
%%%%%%%%%%%%%%%%%%%%%%%%%%%%%%%%%%%%%%%%%%%%%%%
\documentclass[a4paper,11pt]{article}

\usepackage[margin=2cm]{geometry}
\usepackage{setspace}
%\onehalfspacing
\doublespacing
%\usepackage{authblk}
\usepackage{amsmath}
\usepackage{amssymb}
\usepackage{amsthm}

\usepackage[notcite,notref]{showkeys}

% \usepackage{psfrag}
\usepackage{graphicx,subfigure}
\usepackage{color}
\def\red{\color{red}}
\def\blue{\color{blue}}
%\usepackage{verbatim}
% \usepackage{alltt}
%\usepackage{kotex}

\usepackage{enumerate}

%%%%%%%%%%%%%% MY DEFINITIONS %%%%%%%%%%%%%%%%%%%%%%%%%%%

\def\tr{\,\textrm{tr}\,}
\def\div{\,\textrm{div}\,}
\def\sgn{\,\textrm{sgn}\,}

\def\th{\tilde{h}}
\def\tx{\tilde{x}}
\def\tk{\tilde{\kappa}}

\def\tg{{\tilde{\gamma}}}
\def\tv{{\tilde{v}}}
\def\tth{{\tilde{\theta}}}
\def\ts{{\tilde{\tau}}}
\def\tu{{\tilde{u}}}

\def\dtg{\dot{\tilde{\gamma}}}
\def\dtv{\dot{\tilde{v}}}
\def\dtth{\dot{\tilde{\theta}}}
\def\dts{\dot{\tilde{\tau}}}
\def\dtu{\dot{\tilde{u}}}

\def\dpp{\dot{p}}
\def\dqq{\dot{q}}
\def\drr{\dot{r}}
\def\dss{\dot{s}}

\def\ta{{\tilde{a}}}
\def\tb{{\tilde{b}}}
\def\tc{{\tilde{c}}}
\def\td{{\tilde{d}}}




\def\bx{\bar{x}}
\def\bm{\bar{\mathbf{m}}}
\def\K{\mathcal{K}}
\def\E{\mathcal{E}}
\def\del{\partial}
\def\eps{\varepsilon}

\newcommand{\tcr}{\textcolor{red}}
\newcommand{\tcb}{\textcolor{blue}}

\newcommand{\ubar}[1]{\text{\b{$#1$}}}
\newtheorem{theorem}{Theorem}
\newtheorem{lemma}{Lemma}[section]
\newtheorem{proposition}{Proposition}[section]
%\newtheorem{definition}{Definition}[section]
\newtheorem{remark}{Remark}[section]

%%%%%%%%%%%%%%%%%%%%%%%%%%%%%%%%%%%%%%%%%%%%%%%%%%%%%%%%%%
\begin{document}
\title{Note for the Self-similar shear bands problem}
\author{Min-Gi Lee\footnotemark[1]}
% \and Athanasios Tzavaras\footnotemark[1]\  \footnotemark[3]  \footnotemark[4]}
\date{}

\maketitle
\renewcommand{\thefootnote}{\fnsymbol{footnote}}
% \footnotetext[1]{Computer, Electrical and Mathematical Sciences \& Engineering Division, King Abdullah University of Science and Technology (KAUST), Thuwal, Saudi Arabia}
% \footnotetext[2]{Department of Mathematics and Applied Mathematics, University of Crete, Heraklion, Greece}
% \footnotetext[3]{Institute of Applied and Computational Mathematics, FORTH, Heraklion, Greece}
% \footnotetext[4]{Corresponding author : \texttt{athanasios.tzavaras@kaust.edu.sa}}
%\footnotetext[4]{Research supported by the King Abdullah University of Science and Technology (KAUST) }
\renewcommand{\thefootnote}{\arabic{footnote}}


\maketitle

\tableofcontents
% \begin{abstract}
% abstract
% \end{abstract}

\section{The model description}
We consider a 1-d shear deformation of a material whose material law of stress depends on 1) temperature, 2) strain, 3) strain rate. The motion is described by following field variables,
\begin{equation} \label{eq:vars}
\begin{aligned}
 \gamma(t,x) &: \text{strain}\\
 u(t,x)=\gamma_t &: \text{strain rate}\\
 v(t,x) &: \text{vertical velocity}\\
 \theta(t,x) &: \text{temperature}\\
 \tau(t,x) &: \text{stress}
\end{aligned}
\end{equation}
The material exhibits 1) temperature-softening, 2) strain-hardening, 3) rate-hardening. we denote the shear stress
$$ \tau = \tau(\theta,\gamma,u). $$
and study a model
\begin{equation}
 \tau = \theta^{-\alpha}\gamma^m u^n. \label{eq:stresslaw}
\end{equation}

A few forehand perspectives :
\begin{enumerate}
 \item The regime where $-\alpha+m+n <0$ will exhibit the localization, whereas the regime $-\alpha+m+n > 0$ will exhibit stabilization. {\blue Can we rigorously study the linearize stability to the uniform shearing solution?}
 \item The uniform shearing solution will appear as one of the self-similar solution by a specific $\lambda$ that is negative.
\end{enumerate}
\subsection{A system of conservation laws}
For the field variables \eqref{eq:vars}, equations describing the deformation are given by
\begin{align}
 \gamma_t &= u, \quad \text{(kinematic compatibility)} 	\label{eq:g}\\
 v_t &= \tau_x, \quad \text{(momentum conservation)} 	\label{eq:v}\\
 \theta_t &= \tau u \quad \text{(energy conservation)}	\label{eq:th}\\
 \tau &=\theta^{-\alpha}\gamma^m u^n.			\label{eq:tau}
\end{align}
\subsection{Scale invariance property of the system}
The system \eqref{eq:g}-\eqref{eq:tau} admits a scale invariance property. Suppose $(\gamma,u,v,\theta,\tau)$ is a solution. Then a rescaled version of it 
\begin{align*}
 \gamma_\rho(t,x) &= \rho^a\gamma(\rho^{-1}t,\rho^\lambda x), &
 v_\rho(t,x) &= \rho^bv(\rho^{-1}t,\rho^\lambda x),\\
 \theta_\rho(t,x) &= \rho^c\theta(\rho^{-1}t,\rho^\lambda x) &
 \tau_\rho(t,x) &= \rho^d\tau(\rho^{-1}t,\rho^\lambda x),\\
 u_\rho(t,x) &= \rho^{b+\lambda}\gamma(\rho^{-1}t,\rho^\lambda x). 
\end{align*}

Calculations :
\begin{align*}
 \text{let} \;f(t,x) = \rho^k F(\rho^{-1}t,\rho^\lambda x), \\
 \partial_t f(t,x) = \rho^k \partial_{t'} F(\rho^{-1}t,\rho^\lambda x) \rho^{-1} = \rho^{k-1} \partial_{t'}F, \\
 \partial_x f(t,x) = \rho^k \partial_{x'} F(\rho^{-1}t,\rho^\lambda x) \rho^\lambda = \rho^{k+\lambda} \partial_{x'}F.
\end{align*}
Relations for invariance :
\begin{align*}
 a-1 = b+\lambda, \quad b-1 = d+\lambda, \quad c-1 = d+b+\lambda, \quad d = -\alpha c + m a + n (b+\lambda)
\end{align*}
From above, we reach to exponents
\begin{align*}
 D & = 1+2\alpha-m-n,\\
 a&= \frac{2+2\alpha-n}{D} + \frac{2+2\alpha}{D}\lambda =: a_0 + a_1 \lambda, & b&=\frac{1+m}{D} + \frac{1+m+n}{D}\lambda =: b_0 + b_1\lambda,\\
 c&=\frac{2(1+m)}{D} + \frac{2(1+m+n)}{D}\lambda =: c_0 + c_1\lambda, & d&=\frac{-2\alpha + 2m +n}{D} + \frac{-2\alpha+2m+2n}{D}\lambda =: d_0 + d_1\lambda
\end{align*}
for each $\lambda \in \mathbb{R}$. For localization, $\lambda>0$. But uniform shearing solution takes a negative $\lambda$.

\begin{align*}
 &a_0-1 = b_0,  \quad b_0-1=d_0,  \quad c_0-1=d_0 +b_0, \quad c_1-1=d_1+b_1, \quad d_0 = -\alpha c_0 + ma_0 +nb_0,\\
 &a_1-1=b_1, \quad b_1-1=d_1, \quad c_1-1=d_1+b_1,  \quad d_1 = -\alpha c_1 + ma_1 +n(b_1+1).
\end{align*}


\subsection{Self-Similar variables}
We try the solutions of type, i.e. put $\rho =t$ in the rescaling,
\begin{align*}
 \gamma(t,x) &= t^a\Gamma(t^\lambda x),\\
 v(t,x) &= t^b V(t^\lambda x),\\
 \theta(t,x) &= t^c \Theta(t^\lambda x),\\
 \tau(t,x) &= t^d \Sigma(t^\lambda x),\\
 u(t,x) &= t^{b+\lambda} U(t^\lambda x)
\end{align*}
and set $\xi = t^\lambda x$.

Calculations:
\begin{align*}
 &\text{Suppose } \; f(t,x) = t^k F(t^\lambda x),\\
 &\partial_t f = k t^{k-1} F + t^k F' \lambda t^{\lambda-1} x = t^{k-1} (kF + \lambda\xi F'),\\
 &\partial_x f = t^k F' t^\lambda = t^{k+\lambda} F',
\end{align*}

At \eqref{eq:g}-\eqref{eq:tau}:

\begin{align*}
 t^{a-1}(a \Gamma(\xi) + \lambda \xi \Gamma'(\xi)) &= t^{b+ \lambda} U(\xi),\\
 t^{b-1}(b V(\xi) + \lambda \xi V'(\xi)) &= t^{d+ \lambda} \Sigma'(\xi)\\
 t^{c-1}(c \Theta(\xi) + \lambda \xi \Theta'(\xi))&=t^{b+d+\lambda} \Sigma U(\xi),\\
 t^d\Sigma(\xi) &= t^{-\alpha c +ma +n(b+ \lambda)} \Theta(\xi)^{-\alpha} \Gamma(\xi)^m U(\xi)^n,\\
 t^{b+\lambda}V'(\xi)&=t^{b+\lambda}U(\xi)
\end{align*}
\begin{equation}
\begin{aligned}
 a \Gamma(\xi) + \lambda \xi \Gamma'(\xi) &= U(\xi),\\
 b V(\xi) + \lambda \xi V'(\xi) &= \Sigma'(\xi)\\
 c \Theta(\xi) + \lambda \xi \Theta'(\xi)&=\Sigma(\xi) U(\xi),\\
 \Sigma(\xi) &= \Theta(\xi)^{-\alpha} \Gamma(\xi)^m U(\xi)^n,\\
 V'(\xi)&=U(\xi)
\end{aligned} \label{eq:ss-odes}
\end{equation}

\begin{equation}
\begin{aligned}
 a \Gamma(\xi) + \lambda \xi \Gamma'(\xi) &= U(\xi),\\
 (b+\lambda) U(\xi) + \lambda \xi U'(\xi) &= \Sigma^{''}(\xi)\\
 c \Theta(\xi) + \lambda \xi \Theta'(\xi)&=\Sigma(\xi) U(\xi),\\
 \Sigma(\xi) &= \Theta(\xi)^{-\alpha} \Gamma(\xi)^m U(\xi)^n,\\
 V'(\xi)&=U(\xi)
\end{aligned} \label{eq:ss-odes2}
\end{equation}

\subsection{de-singularization}
% {\red This part cannot be done : constants are inconsistent
% We find the following monomials constitute the special solution.
% \begin{equation} \label{eq:monomials}
% \begin{aligned}
%  \Gamma(\xi)&=\xi^{-a_1}\Gamma_0, & \Theta(\xi)&=\xi^{-c_1}\Theta_0,\\
%  \Sigma(\xi)&=\xi^{-d_1}\Sigma_0, & U(\xi)&=\xi^{-b_1-1}U_0, \\
%  V(\xi)&=\int_{\xi_0}^\xi U(\xi')\, d\xi',
% \end{aligned}
% \end{equation}
% where the constants $\big(\Gamma_0, U_0, \Theta_0, \Sigma_0, U_0\big)$ is in relations
% \begin{align*}
%  \Gamma_0 &= \frac{U_0}{a_0}, \quad \Theta_0 = a_0^{-\frac{m}{1+\alpha}} c_0^{-\frac{1}{1+\alpha}} U_0^{\frac{1+m+n}{1+\alpha}}, \quad \Sigma_0=a_0^{-\frac{m}{1+\alpha}} c_0^{\frac{\alpha}{1+\alpha}} U_0^{-\frac{\alpha-m-n}{1+\alpha}} \\
% %  U_0&=\Big(b_1\,d_1\,a_0^{-\frac{m}{1+\alpha}} b_0^{-1}c_0^{\frac{\alpha}{1+\alpha}}\Big)^{\frac{1+\alpha}{D}}.
% \end{align*}
% }

These arise from the consideration of the scale invariance property of the system \eqref{eq:ss-odes}: Provided $\big(\Gamma(\xi), V(\xi), \Theta(\xi), \Sigma(\xi), U(\xi)\big)$ is a solution of it, then the scaled version $\big(\Gamma_\rho(\xi), V_\rho(\xi), \Theta_\rho(\xi), \Sigma_\rho(\xi), U_\rho(\xi)\big)$, where
\begin{align*}
 \Gamma_\rho(\xi)&=\rho^{a_1}\Gamma(\rho\xi), & V_\rho(\xi)&=\rho^{b_1}V(\rho\xi), & \Theta_\rho(\xi)&=\rho^{c_1}\Theta(\rho\xi),\\
 \Sigma_\rho(\xi)&=\rho^{d_1}\Sigma(\rho\xi), & U_\rho(\xi)&=\rho^{b_1+1}U(\rho\xi),
\end{align*}
again is a solution. From that we find \eqref{eq:monomials} by setting $\rho=\xi^{-1}$.


Introduce new field variables 
\begin{equation}
\begin{aligned}
 \Gamma(\xi) &= \xi^\ta \tg(\xi),\\
 V(\xi)&=\xi^\tb \tv(\xi),\\
 \Theta(\xi)&=\xi^\tc \tth(\xi),\\
 \Sigma(\xi)&=\xi^\td \ts(\xi),\\
 U(\xi)&=\xi^{\tb-1} \tu(\xi). 
\end{aligned}
\end{equation}
Then at \eqref{eq:ss-odes}:
\begin{align*}
 \xi^\ta\Big( a\tg + \lambda \ta \tg + \lambda\xi\tg'\Big) &=\xi^{\tb-1} \tu,\\
 \xi^\tb\Big( b\tv + \lambda \tb \tv + \lambda\xi\tv'\Big) &=\xi^{\td-1} \Big(\td\ts + \xi\ts'\Big),\\
 \xi^\tc\Big( c\tth+ \lambda \tc \tth+ \lambda\xi\tth'\Big)&=\xi^{\td+\tb-1} \ts\tu,\\
 \xi^\td\ts &= \xi^{-\alpha \tc +m\ta +n(\tb-1)} \tth^{-\alpha} \tg^m \tu^n,\\
 \xi^{\tb-1}\Big(\tb\tv + \xi \tv'\Big)&= \xi^{\tb-1} \tu.
\end{align*}

$\ta, \tb, \tc, \td$ such that
\begin{align*}
 &\ta=\tb-1, \quad \tb=\td-1, \quad \tc=\td+\tb-1,\quad \td = -\alpha \tc + m\ta +n(\tb-1) \\
 \Longrightarrow \quad&\ta = -a_1, \quad \td = -d_1, \quad \tc = -c_1, \quad \tb=-b_1.
\end{align*}

\begin{equation}
 \begin{aligned}
  a_0\tg + \lambda\xi\tg' &=\tu,\\
  b_0\tv + \lambda\xi\tv' &=-d_1 \ts + \xi\ts',\\
  c_0\tth+ \lambda\xi\tth'&=\ts\tu,\\
  \ts &=\tth^{-\alpha}\tg^m\tu^n,\\
  -b_1\tv+\xi\tv' &= \tu.
 \end{aligned}
\end{equation}

Introduce the new independent variable $\eta = \log\xi$.
\section{$(p,q,r,s)$-system derivation}

\begin{enumerate}
 \item The equation \eqref{eq:th} can be rewritten in the form
 $$ \Big(\frac{1}{1+\alpha} \tth^{1+\alpha}\Big)_t = \frac{\tu^n}{\tg^n} \Big(\frac{1}{1+m+n} \tg^{1+m+n}\Big)_t$$
 and we expect, at least for the self-similar solutions, that
 $$ \frac{ \frac{1}{1+\alpha} \tth^{1+\alpha} }{ \frac{1}{1+m+n} \tg^{1+m+n} }  = 1 + \mathcal{O}(n) \overset{put}{=} s^n, \quad \text{for some $s$}. $$
 \item If so, we can define the variable 
 $$r = \Big(\Big(\frac{1+m+n}{1+\alpha}\Big)^{\frac{\alpha}{(1+\alpha)}}\tau \gamma^{\frac{\alpha-m-n}{1+\alpha}}\Big)^{\frac{1}{n}} = \frac{\tu}{\tg}\,s^{-\frac{\alpha}{1+\alpha}} \sim \mathcal{O}(1) $$
 and when $\alpha=0$, it reduces to $\frac{\tu}{\tg}$. The advantage of this definition to the $\frac{\tu}{\tg}$ is that the latter expression couples to the $\tth$ too whereas $r$ here does not. 
\end{enumerate}

Auxiliary calculations:
\begin{align*}
 \frac{\dtg}{\tg} &= \frac{1}{\lambda }\Big(\frac{\tu}{\tg}-a_0\Big),\\
 \frac{\dts}{\ts} &= d_1+ b_0\frac{\tv}{\ts} + \lambda \frac{\dtv}{\ts} = d_1+ b_0\frac{\tv}{\ts} + \lambda \Big(b_1 + \frac{\tu}{\tv}\Big)\frac{\tv}{\ts} = d_1 + b\frac{\tv}{\ts} + \lambda\frac{\tu}{\tv}\frac{\tv}{\ts} ,\\
 \frac{\dtth}{\tth}&=\frac{1}{\lambda }\Big(\frac{\ts\tu}{\tth}-c_0\Big),\\
 0&=\frac{\dts}{\ts} +\alpha \frac{\dtth}{\tth} - m \frac{\dtg}{\tg} - n\frac{\dtu}{\tu},\\
 \frac{\dtv}{\tv}&= b_1 +\frac{\tu}{\tv}
\end{align*}

Define 
\begin{equation}\label{eq:pqrsdef}
 \begin{aligned}
  p &= \frac{\tg}{\ts}, & q&=b \frac{\tv}{\ts},\\
  r &= \Big(\Big(\frac{1+m+n}{1+\alpha}\Big)^{\frac{\alpha}{(1+\alpha)}}\tau \gamma^{\frac{\alpha-m-n}{1+\alpha}}\Big)^{\frac{1}{n}} , & s&=z^{\frac{1}{n}}, \quad z=\frac{ \frac{1}{1+\alpha} \tth^{1+\alpha} }{ \frac{1}{1+m+n} \tg^{1+m+n} }.
 \end{aligned}
\end{equation}

We have
\begin{align*}
 \frac{\dpp}{p}&=\frac{\dtg}{\tg} - \frac{\dts}{\ts}& &=\left[\frac{1}{\lambda }\Big(\frac{\tu}{\tg}-a_0\Big)\right] & &-\left[d_1 + b\frac{\tv}{\ts} + \lambda\frac{\tu}{\tv}\frac{\tv}{\ts}\right]\\
 \frac{\dqq}{q}&=\frac{\dtv}{\tv} - \frac{\dts}{\ts}& &=\left[b_1 +\frac{\tu}{\tv}\right] & &-\left[d_1 + b\frac{\tv}{\ts} + \lambda\frac{\tu}{\tv}\frac{\tv}{\ts}\right]\\
 n\frac{\drr}{r}&=\frac{\dts}{\ts} +\frac{\alpha-m-m}{1+\alpha} \frac{\dtg}{\tg}& &=\left[\frac{\alpha-m-n}{\lambda(1+\alpha) }\Big(\frac{\tu}{\tg}-a_0\Big)\right]& &+\left[d_1 + b\frac{\tv}{\ts} + \lambda\frac{\tu}{\tv}\frac{\tv}{\ts}\right]\\
 \frac{\dot{z}}{z} &= (1+\alpha)\frac{\dtth}{\tth} - (1+m+n)\frac{\dtg}{\tg} & &=\left[\frac{-1-m-n}{\lambda }\Big(\frac{\tu}{\tg}-a_0\Big)\right] & &+ \left[\frac{1+\alpha}{\lambda }\Big(\frac{\ts\tu}{\tth}-c_0\Big)\right].%\\
%  \dot{s} &=\frac{\partial s}{\partial (z-1)} \dot{z} &&= \frac{1+m}{\lambda}\frac{\partial s}{\partial (z-1)} \bigg\{z\big[-r - \frac{n}{D}\Big]+ ru^n\bigg\}.
\end{align*}
Noticing that $\displaystyle a_0(1+m+n)-c_0(1+\alpha)=n$ and that
\begin{align*}
 \frac{\ts\tu}{\tth} = \frac{\tg^{1+m+n}}{\tth^{1+\alpha}}\Big(\frac{\tu}{\tg}\Big)^{1+n} = \frac{1+m}{1+\alpha} \frac{1}{z}\,\Big(\frac{\tu}{\tg}\Big)^{1+n} =  \frac{1+m+n}{1+\alpha} \frac{1}{s^n}\,\Big(rs^{\frac{\alpha}{1+\alpha}}\Big)^{1+n}
 =\frac{1+m+n}{1+\alpha}\Big( r^{1+n} s^{\frac{\alpha-n}{1+\alpha}}\Big),
\end{align*}
\begin{align*}
n\frac{\dot{s}}{s} = \frac{\dot{z}}{z} = \frac{1+m+n}{\lambda}\Big(r^{1+n}s^{\frac{\alpha-n}{1+\alpha}} - rs^{\frac{\alpha}{1+\alpha}} \Big) + \frac{n}{\lambda} = \frac{1+m+n}{\lambda}\,rs^{\frac{\alpha}{1+\alpha}}\Big(r^{n}s^{\frac{-n}{1+\alpha}} - 1 \Big) + \frac{n}{\lambda}
\end{align*}

$(p,q,r,s)$-system:
\begin{align*}
 \frac{\dpp}{p}&=\left[\frac{1}{\lambda }\Big(\frac{\tu}{\tg}-a_0\Big)\right] & &-\left[d_1 + b\frac{\tv}{\ts} + \lambda\frac{\tu}{\tv}\frac{\tv}{\ts}\right]\\
 \frac{\dqq}{q}&=\left[b_1 +\frac{\tu}{\tv}\right] & &-\left[d_1 + b\frac{\tv}{\ts} + \lambda\frac{\tu}{\tv}\frac{\tv}{\ts}\right]\\
 n\frac{\drr}{r}&=\left[\frac{\alpha-m-n}{\lambda(1+\alpha) }\Big(\frac{\tu}{\tg}-a_0\Big)\right]& &+\left[d_1 + b\frac{\tv}{\ts} + \lambda\frac{\tu}{\tv}\frac{\tv}{\ts}\right]\\
 \frac{\dot{s}}{s} &= \frac{1+m+n}{\lambda}\,rs^{\frac{\alpha}{1+\alpha}}\Big(\frac{r^{n}s^{\frac{-n}{1+\alpha}} - 1}{n} \Big) + \frac{1}{\lambda}
\end{align*}

\begin{remark}
 \begin{enumerate}
  \item 
  \item Note that $s$ is not a fast variable. Even though $\displaystyle\frac{ \frac{1}{1+\alpha} \tth^{1+\alpha} }{ \frac{1}{1+m+n} \tg^{1+m+n} }$ relaxes to the manifold $1 + \mathcal{O}(n)$, the relaxation time is not of $\mathcal{O}(\frac{1}{n})$ but is of $\mathcal{O}(1)$.
  
 \end{enumerate}
\end{remark}



$(p,q,r,s)$-system:

\begin{equation}
\begin{aligned}
  {\dpp}&=p\bigg\{\Big[\frac{1}{\lambda }\Big(r-a_0\Big)\Big] -\Big[d_1 + q + \lambda p r\Big]\bigg\}\\
  {\dqq}&=q\bigg\{b_1-d_1 + \lambda p r\bigg\} +bpr,\\
 n{\drr}&=r\bigg\{\left[\frac{\alpha-m-n}{\lambda(1+\alpha) }\Big(\frac{\tu}{\tg}-a_0\Big)\right]& &+\left[d_1 + b\frac{\tv}{\ts} + \lambda\frac{\tu}{\tv}\frac{\tv}{\ts}\right]\bigg\}\\
 n\frac{\dot{s}}{s} &= \frac{1+m+n}{\lambda}\Big(r^{1+n}s^{\frac{\alpha-n}{1+\alpha}} - r \Big) + \frac{n}{\lambda}
\end{aligned}
\end{equation}

\begin{remark}
 In the case of the variable $r$, assuming $\tg, \ts,\tth \sim \mathcal{O}(1)$, the exponent $n$ is natural and the relaxation time scale $\drr \sim \mathcal{O}(\frac{1}{n})$. For the case of $\displaystyle\left(\frac{ \frac{1}{1+\alpha}\tth^{1+\alpha}}{ \frac{1}{1+m}\tg^{1+m} } -1 \right)$, there is no preferred relaxation time scale. For the time being, we do not specify the function $s(n,z-1)$ in the upcoming calculations.
\end{remark}
\begin{align*}
 \dot{s} =\frac{\partial s}{\partial (z-1)} \dot{z}
 &= \frac{1+m}{\lambda}\frac{\partial s}{\partial (z-1)} z\bigg\{-r - \frac{n}{D}+ \frac{r}{z}u^n\bigg\}\\
 &=\frac{1+m}{\lambda}\frac{\partial s}{\partial (z-1)} \bigg\{r(u^n-1) +r(1-z) -n\frac{z}{D}\bigg\}.
\end{align*}
We choose $s(n,z)$ such that
\begin{enumerate}
 \item As $(n,z-1) \rightarrow (0,0)$, $s(n,z-1) \rightarrow 0$, 
 \item For fixed $n$, the map $z \mapsto s$ is invertible, and for inverse $z=z(n,s)\rightarrow 1$ as $(n,s) \rightarrow (0,0)$.
\end{enumerate}
We set 
$$ s(n,z-1) = \frac{(z-1)^{\frac{1}{n}}}{n}, \quad \text{or} \quad z= 1+ns^n$$
and look for solutions of this form. Then we have


\begin{align*}
 \frac{\ts\tu}{\tth} &= \frac{1+m}{1+\alpha} \frac{r}{z}\,u^n = \frac{1+m}{1+\alpha} r + n\frac{1+m}{1+\alpha} \frac{r}{1+ns^n}\Big(\frac{u^n-1}{n}-s^n\Big),%\\
%  \frac{\tu}{\tv}\frac{\tv}{\ts}&=\frac{\ts}{\tv} \frac{\tg}{\ts} \frac{\tu}{\tg} \frac{\tv}{\ts} = pr.
\end{align*}





\begin{equation}
\begin{aligned}
  {\dpp}&=p\bigg\{\Big[\frac{1}{\lambda }\Big(r-a_0\Big)\Big] -\Big[d_1 + q + \lambda p r\Big]\bigg\}\\
  {\dqq}&=q\bigg\{b_1-d_1 + \lambda p r\bigg\} +bpr,\\
 n{\drr}&=r\bigg\{\Big[\frac{-m-n}{\lambda }\Big(r-a_0\Big)\Big]+\Big[d_1 + q + \lambda p r\Big]+\Big[\frac{\alpha}{\lambda }\Big(\frac{1+m}{1+\alpha}r-c_0\Big)\Big] + n\Big[\frac{\alpha}{\lambda }\frac{1+m}{1+\alpha} \frac{r}{1+ns^n}\Big(\frac{u^n-1}{n}-s^n\Big)\Big]\bigg\}\\
 n\dot{s}&=\frac{1+m}{\lambda}s^{1-n} \left\{r\frac{u^n-1}{n} - rs^n -\frac{1+ns^n}{D}\right\}.
\end{aligned}
\end{equation}

\section{Normally hyperbolic invariant manifold}
\section{Equilibrium points, Linear structure}
\subsection{Characterization of the heteroclinic orbit : why and how}
\subsection{Asymptotic behavior of self-similar variables in $\xi$}
\section{A $k$-parameter family of shear banding solutions}
\subsection{Asymptotic behavior of field variables in $t$ and $x$}
\section{Existence via Geometric theory of singular perturbation}



\end{document}