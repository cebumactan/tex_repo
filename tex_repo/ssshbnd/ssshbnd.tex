%%%%%%%%%%%%%%%%%%%%%%%%%%%%%%%%%%%%%%%%%%%%%%%
%
%    Self-Similar shear bands, Existence, Numerics, Asymptotics
%
%                                                      by
%
%                                       Min-Gi Lee   
%
%                                          version Sep 2016
%
%
%%%%%%%%%%%%%%%%%%%%%%%%%%%%%%%%%%%%%%%%%%%%%%%
\documentclass[a4paper,11pt]{article}

\usepackage[margin=3cm]{geometry}
\usepackage{setspace}
%\onehalfspacing
\doublespacing
%\usepackage{authblk}
\usepackage{amsmath}
\usepackage{amssymb}
\usepackage{amsthm}

\usepackage[notcite,notref]{showkeys}

% \usepackage{psfrag}
\usepackage{graphicx,subfigure}
\usepackage{color}
\def\red{\color{red}}
\def\blue{\color{blue}}
%\usepackage{verbatim}
% \usepackage{alltt}
%\usepackage{kotex}

\usepackage{enumerate}

%%%%%%%%%%%%%% MY DEFINITIONS %%%%%%%%%%%%%%%%%%%%%%%%%%%

\def\tr{\,\textrm{tr}\,}
\def\div{\,\textrm{div}\,}
\def\sgn{\,\textrm{sgn}\,}

\def\th{\tilde{h}}
\def\tx{\tilde{x}}
\def\tk{\tilde{\kappa}}

\def\tg{{\tilde{\gamma}}}
\def\tv{{\tilde{v}}}
\def\tth{{\tilde{\theta}}}
\def\ts{{\tilde{\tau}}}
\def\tu{{\tilde{u}}}

\def\ta{{\tilde{a}}}
\def\tb{{\tilde{b}}}
\def\tc{{\tilde{c}}}
\def\td{{\tilde{d}}}




\def\bx{\bar{x}}
\def\bm{\bar{\mathbf{m}}}
\def\K{\mathcal{K}}
\def\E{\mathcal{E}}
\def\del{\partial}
\def\eps{\varepsilon}

\newcommand{\tcr}{\textcolor{red}}
\newcommand{\tcb}{\textcolor{blue}}

\newcommand{\ubar}[1]{\text{\b{$#1$}}}
\newtheorem{theorem}{Theorem}
\newtheorem{lemma}{Lemma}[section]
\newtheorem{proposition}{Proposition}[section]
%\newtheorem{definition}{Definition}[section]
\newtheorem{remark}{Remark}[section]

%%%%%%%%%%%%%%%%%%%%%%%%%%%%%%%%%%%%%%%%%%%%%%%%%%%%%%%%%%
\begin{document}
\title{Note for the Self-similar shear bands problem}
\author{Min-Gi Lee\footnotemark[1]}
% \and Athanasios Tzavaras\footnotemark[1]\  \footnotemark[3]  \footnotemark[4]}
\date{}

\maketitle
\renewcommand{\thefootnote}{\fnsymbol{footnote}}
% \footnotetext[1]{Computer, Electrical and Mathematical Sciences \& Engineering Division, King Abdullah University of Science and Technology (KAUST), Thuwal, Saudi Arabia}
% \footnotetext[2]{Department of Mathematics and Applied Mathematics, University of Crete, Heraklion, Greece}
% \footnotetext[3]{Institute of Applied and Computational Mathematics, FORTH, Heraklion, Greece}
% \footnotetext[4]{Corresponding author : \texttt{athanasios.tzavaras@kaust.edu.sa}}
%\footnotetext[4]{Research supported by the King Abdullah University of Science and Technology (KAUST) }
\renewcommand{\thefootnote}{\arabic{footnote}}


\maketitle

\tableofcontents
% \begin{abstract}
% abstract
% \end{abstract}

\section{The model description}
We consider a 1-d shear deformation of a material whose material law of stress depends on 1) temperature, 2) strain, 3) strain rate. The motion is described by following field variables,
\begin{equation} \label{eq:vars}
\begin{aligned}
 \gamma(t,x) &: \text{strain}\\
 u(t,x)=\gamma_t &: \text{strain rate}\\
 v(t,x) &: \text{vertical velocity}\\
 \theta(t,x) &: \text{temperature}\\
 \tau(t,x) &: \text{stress}
\end{aligned}
\end{equation}
The material exhibits 1) temperature-softening, 2) strain-hardening, 3) rate-hardening. we denote the shear stress
$$ \tau = \tau(\theta,\gamma,u). $$
and study a model
\begin{equation}
 \tau = \theta^{-\alpha}\gamma^m u^n. \label{eq:stresslaw}
\end{equation}

A few forehand perspectives :
\begin{enumerate}
 \item The regime where $-\alpha+m+n <0$ will exhibit the localization, whereas the regime $-\alpha+m+n > 0$ will exhibit stabilization. {\blue Can we rigorously study the linearize stability to the uniform shearing solution?}
 \item The uniform shearing solution will appear as one of the self-similar solution by a specific $\lambda$ that is negative.
\end{enumerate}
\subsection{A system of conservation laws}
For the field variables \eqref{eq:vars}, equations describing the deformation are given by
\begin{align}
 \gamma_t &= u, \quad \text{(kinematic compatibility)} 	\label{eq:g}\\
 v_t &= \tau_x, \quad \text{(momentum conservation)} 	\label{eq:v}\\
 \theta_t &= \tau u \quad \text{(energy conservation)}	\label{eq:th}\\
 \tau &=\theta^{-\alpha}\gamma^m u^n.			\label{eq:tau}
\end{align}
\subsection{Scale invariance property of the system}
The system \eqref{eq:g}-\eqref{eq:tau} admits a scale invariance property. Suppose $(\gamma,u,v,\theta,\tau)$ is a solution. Then a rescaled version of it 
\begin{align*}
 \gamma_\rho(t,x) &= \rho^a\gamma(\rho^{-1}t,\rho^\lambda x), &
 v_\rho(t,x) &= \rho^bv(\rho^{-1}t,\rho^\lambda x),\\
 \theta_\rho(t,x) &= \rho^c\theta(\rho^{-1}t,\rho^\lambda x) &
 \tau_\rho(t,x) &= \rho^d\tau(\rho^{-1}t,\rho^\lambda x),\\
 u_\rho(t,x) &= \rho^{b+\lambda}\gamma(\rho^{-1}t,\rho^\lambda x). 
\end{align*}

Calculations :
\begin{align*}
 \text{let} \;f(t,x) = \rho^k F(\rho^{-1}t,\rho^\lambda x), \\
 \partial_t f(t,x) = \rho^k \partial_{t'} F(\rho^{-1}t,\rho^\lambda x) \rho^{-1} = \rho^{k-1} \partial_{t'}F, \\
 \partial_x f(t,x) = \rho^k \partial_{x'} F(\rho^{-1}t,\rho^\lambda x) \rho^\lambda = \rho^{k+\lambda} \partial_{x'}F.
\end{align*}
Relations for invariance :
\begin{align*}
 a-1 = b+\lambda, \quad b-1 = d+\lambda, \quad c-1 = d+b+\lambda, \quad d = -\alpha c + m a + n (b+\lambda)
\end{align*}
From above, we reach to exponents
\begin{align*}
 D & = 1+2\alpha-m-n,\\
 a&= \frac{2+2\alpha-n}{D} + \frac{2+2\alpha}{D}\lambda =: a_0 + a_1 \lambda, & b&=\frac{1+m}{D} + \frac{1+m+n}{D}\lambda =: b_0 + b_1\lambda,\\
 c&=\frac{2(1+m)}{D} + \frac{2(1+m+n)}{D}\lambda =: c_0 + c_1\lambda, & d&=\frac{-2\alpha + 2m +n}{D} + \frac{-2\alpha+2m+2n}{D}\lambda =: d_0 + d_1\lambda
\end{align*}
for each $\lambda \in \mathbb{R}$. For localization, $\lambda>0$. But uniform shearing solution takes a negative $\lambda$.
\subsection{Self-Similar variables}
We try the solutions of type, i.e. put $\rho =t$ in the rescaling,
\begin{align*}
 \gamma(t,x) &= t^a\Gamma(t^\lambda x),\\
 v(t,x) &= t^b V(t^\lambda x),\\
 \theta(t,x) &= t^c \Theta(t^\lambda x),\\
 \tau(t,x) &= t^d \Sigma(t^\lambda x),\\
 u(t,x) &= t^{b+\lambda} U(t^\lambda x)
\end{align*}
and set $\xi = t^\lambda x$.

Calculations:
\begin{align*}
 &\text{Suppose } \; f(t,x) = t^k F(t^\lambda x),\\
 &\partial_t f = k t^{k-1} F + t^k F' \lambda t^{\lambda-1} x = t^{k-1} (kF + \lambda\xi F'),\\
 &\partial_x f = t^k F' t^\lambda = t^{k+\lambda} F',
\end{align*}

At \eqref{eq:g}-\eqref{eq:tau}:

\begin{align*}
 t^{a-1}(a \Gamma(\xi) + \lambda \xi \Gamma'(\xi)) &= t^{b+ \lambda} U(\xi),\\
 t^{b-1}(b V(\xi) + \lambda \xi V'(\xi)) &= t^{d+ \lambda} \Sigma'(\xi)\\
 t^{c-1}(d \Theta(\xi) + \lambda \xi \Theta'(\xi))&=t^{b+d+\lambda} \Sigma U(\xi),\\
 t^d\Sigma(\xi) &= t^{-\alpha c +ma +n(b+ \lambda)} \Theta(\xi)^{-\alpha} \Gamma(\xi)^m U(\xi)^n,\\
 t^{b+\lambda}V'(\xi)&=t^{b+\lambda}U(\xi)
\end{align*}
\begin{equation}
\begin{aligned}
 a \Gamma(\xi) + \lambda \xi \Gamma'(\xi) &= U(\xi),\\
 b V(\xi) + \lambda \xi V'(\xi) &= \Sigma'(\xi)\\
 d \Theta(\xi) + \lambda \xi \Theta'(\xi)&=\Sigma(\xi) U(\xi),\\
 \Sigma(\xi) &= \Theta(\xi)^{-\alpha} \Gamma(\xi)^m U(\xi)^n,\\
 V'(\xi)&=U(\xi)
\end{aligned} \label{eq:ss-odes}
\end{equation}
\subsection{de-singularization}
Introduce new field variables 
\begin{equation}
\begin{aligned}
 \Gamma(\xi) &= \xi^\ta \tg(\xi),\\
 V(\xi)&=\xi^\tb \tv(\xi),\\
 \Theta(\xi)&=\xi^\tc \tth(\xi),\\
 \Sigma(\xi)&=\xi^\td \ts(\xi),\\
 U(\xi)&=\xi^{\tb-1} \tu(\xi). 
\end{aligned}
\end{equation}
Then at \eqref{eq:ss-odes}:
\begin{align*}
 \xi^\ta\Big( a\tg + \lambda \ta \tg + \lambda\xi\tg'\Big) &=\xi^{\tb-1} \tu,\\
 \xi^\tb\Big( b\tv + \lambda \tb \tv + \lambda\xi\tv'\Big) &=\xi^{\td-1} \Big(\td\ts + \xi\ts'\Big),\\
 \xi^\tc\Big( c\tth+ \lambda \tc \tth+ \lambda\xi\tth'\Big)&=\xi^{\td+\tb-1} \ts\tu,\\
 \xi^\td\ts &= \xi^{-\alpha \tc +m\ta +n(\tb-1)} \tth^{-\alpha} \tg^m \tu^n,\\
 \xi^{\tb-1}\Big(\tb\tv + \xi \tv'\Big)&= \xi^{\tb-1} \tu.
\end{align*}

$\ta, \tb, \tc, \td$ such that
\begin{align*}
 &\ta=\tb-1, \quad \tb=\td-1, \quad \tc=\td+\tb-1,\quad \td = -\alpha \tc + m\ta +n(\tb-1) \\
 \Longrightarrow \quad&\ta = -a_1, \quad \td = -d_1, \quad \tc = -c_1, \quad \tb=-b_1.
\end{align*}

\begin{equation}
 \begin{aligned}
  a_0\tg + \lambda\xi\tg' &=\tu,\\
  b_0\tv + \lambda\xi\tv' &=-d_1 \sigma + \xi\ts',\\
  c_0\tth+ \lambda\xi\tth'&=\ts\tu,\\
  \ts &=\tth^{-\alpha}\tg^m\tu^n,\\
  -b_1\tv+\xi\tv' &= \tu.
 \end{aligned}
\end{equation}

Introduce the new independent variable $\eta = \log\xi$.
\section{$(p,q,r,s)$-system derivation}
\section{Normally hyperbolic invariant manifold}
\section{Equilibrium points, Linear structure}
\subsection{Characterization of the heteroclinic orbit : why and how}
\subsection{Asymptotic behavior of self-similar variables in $\xi$}
\section{A $k$-parameter family of shear banding solutions}
\subsection{Asymptotic behavior of field variables in $t$ and $x$}
\section{Existence via Geometric theory of singular perturbation}



\end{document}