\documentclass[a4paper,12pt]{article}
\usepackage[margin=3cm]{geometry}
\usepackage{setspace}
%\onehalfspacing
\doublespacing
%\usepackage{authblk}
\usepackage{amsmath}
\usepackage{amssymb}
\usepackage{amsthm}

\usepackage[notcite,notref]{showkeys}

\usepackage{psfrag}
\usepackage{graphicx,subfigure}
\usepackage{color}
\def\red{\color{red}}
\def\blue{\color{blue}}
%\usepackage{verbatim}
\usepackage{alltt}
%\usepackage{kotex}

\usepackage{enumerate}

%%%%%%%%%%%%%% MY DEFINITIONS %%%%%%%%%%%%%%%%%%%%%%%%%%%

\def\tr{\,\textrm{tr}\,}
\def\div{\,\textrm{div}\,}
\def\sgn{\,\textrm{sgn}\,}

\def\th{\tilde{h}}
\def\tk{\tilde{\kappa}}
\def\bx{\bar{x}}
\def\bm{\bar{\mathbf{m}}}
\def\K{\mathcal{K}}
\def\E{\mathcal{E}}
\def\del{\partial}
\def\eps{\varepsilon}

\newcommand{\tcr}{\textcolor{red}}
\newcommand{\tcb}{\textcolor{blue}}

\newcommand{\ubar}[1]{\text{\b{$#1$}}}
\newtheorem{theorem}{Theorem}
\newtheorem{lemma}{Lemma}[section]
\newtheorem{proposition}{Proposition}[section]
%\newtheorem{definition}{Definition}[section]
\newtheorem{remark}{Remark}[section]

%%%%%%%%%%%%%%%%%%%%%%%%%%%%%%%%%%%%%%%%%%%%%%%%%%%%%%%%%%
% \usepackage[utf8]{inputenc}

%opening
\title{Second gradient theory, Kinetic model and Thermally driven flow.}
\author{Min-Gi Lee}

\begin{document}

\maketitle

\begin{abstract}

\end{abstract}

This note is, first of all for myself to organize what I read and thought but also for the discussions. I am currently looking into the works of
\begin{itemize}
 \item Yoshio Sone, Kinetic Theory and Fluid Dynamics (2002)
 \item Maxwell(1879), On stresses in rarefied gases arising from inequalities of temperature
 \item Dunn \& Serrin(1984), On the thermomechanics of interstitial working
 \item Giesselmann, Lattanzio, Tzavaras, Relative energy for the korteweg theory and related hamiltonian flows in gas dynamics
 \item Kim, Lee, Slemrod(2015), Thermal creep of a rarefied gas on the basis of non-linear Korteweg-theory
 \item Slemrod(2012), Chapman-Enskog $\Rightarrow$ Viscosity-Capillarity
 \item Karlin \& Gorban(2014), Hilbert's 6th problem : exact and approximate hydrodynamic manifolds for kinetic equations.
\end{itemize}

The current purpose of this note is to address three points.
\begin{enumerate}
 \item I wanted to clarify issues on the signs of terms in the stress come out of the {\it second gradient theory} or {\it kinetic considerations} (Section 5)
 \item One obvious difference between two stress tensors ($T^M$ from kinetic consideration, $T^K$ of korteweg theory) is that the former has the term on $D^2\theta$, $\theta$ the temperature, whereas the latter does not. I want to address that this matters to some extent and clarify the differences and consequences. (Section 6,7)
 \item I certainly can say that in 1D Euler-Korteweg model, Korteweg stress does carry the thermal effect in the correct direction in the presence of the temperature gradient. Its mechanism is to penalize the density gradient and what the korteweg tensor does is to moderate the density gradient and apparently it can be interpreted as the thermal effect. (Section 7)
\end{enumerate}



\tableofcontents

\section{Terminology}
I discriminate the term Thermal creep from thermal stress flow. In a narrow sense, thermal creep is associated to the surface-gas interaction on which the gradient of temperature presences. From hydrodynamic perspective, thermal creep could be modeled only by more elaborate slip boundary conditions. %, we may think of a model without any (thermal) additional stress contribution other than the pressure and the viscosity but with more elaborate boundary conditions that include the slip due to temperature gradient.

On the other hand, the thermal stress is the additional contribution in the stress term in the hydrodynamic model, and Sone refers to the flow purely due to this contribution as the  {\bf thermal stress flow}.

Phenomenologically, we might not able to know where to attribute, however.

Maxwell also distinguished these two, he used the terms normal stress and the tangential stress to denote the stress contribution that are in the normal to the surface boundary. If the surface has uniform temperature, then there will be no tangential stress contribution on the boundary.

\section{Notations}
\begin{itemize}
 \item {{{ $\rho, \theta , \mathbf{v}, T, T^v, T^k$ }}} : density, temperature, velocity, stress tensor, viscous stress tensor, Korteweg stress tensor
 \item $X$ : material coordinate,
 \item $x$ : spatial coordinate,
 \item $x(X,t)$ : motion field, {{{$F_{ij} = \frac{\partial x_i}{\partial X_j}$}}} : deformation gradient, $\tau = det(F) $ : specific volume.
\end{itemize}

\section{Work of Sone}
Idea of Sone : Sone were not concerned with the question of Chapman-Enskog but were concerned with how come we really is able to say some calculations are up to 0th, 1st, 2nd, 3rd ... order of Knudsen number no matter how the approximating system looks like. He refined the asymptotic cases according to the Mach number Ma, and the knudsen number Kn, and named each different expansion by, 1) G-expansion (Grad-Hilbert) 2) S-expansion 3) SB-expansion 4) H-expansion (Hilbert). One crucial thing we bare in mind from his calculation is that-- equations here are pde, and we need some size estimations of the derivatives of the field variables along with the size estimations of fields themselves (w.r.t. Ma and Kn) -- he proceeded the calculation under the size assumption that the differential of a field variable is of comparable size to the field variable. This means he won't probe the short wave for that field. This issue becomes critical in the context of bobylev instability of Burnett and Super-Burnett.

\begin{enumerate}
 \item[Regime 1] Asymptotics around Global Maxwellian 
 \begin{itemize}
  \item Perturbation is assumed to be much smaller than $Kn$ in the leading order and this case corresponds to $Ma, Kn \rightarrow 0$, $Ma \ll Kn$.
  \item He used Linearized Boltzmann equation around the global maxwellian.
  \item He named this expansion as Grad-Hilbert expansion.
 \end{itemize}
 
  \item[Regime 2] Asymptotics around Global Maxwellian
 \begin{itemize}
  \item This case corresponds to $Ma \sim Kn \rightarrow 0$
  \item He used the non-linear Boltzmann equation but from the $Q(f,f)$ terms the linearized part is extracted out explicitly and the residual non-linearity is considered.
  \item He named this expansion as S-expansion
 \end{itemize}
 
  \item[Regime 3] Asymptotics around Local Maxwellian 
 \begin{itemize}
  \item This case also corresponds to $Ma \sim Kn \rightarrow 0$, but the temperature of the Local Maxwellian depends on $x$, the velocity is a perturbation of the uniform velocity.
  \item He used the non-linear Boltzmann equation of the original form.
  \item He named this expansion as SB-expansion
 \end{itemize}
 
  \item[Regime 4] Asymptotics around Local Maxwellian 
 \begin{itemize}
  \item This case corresponds to $Ma \sim O(1)$, $Kn \rightarrow 0$. Both of the temperature and the velocity in the Local maxwellian depend on $x$.
  \item He used the non-linear Boltzmann equation of the original form.
  \item He named this expansion as Hilbert expansion.
 \end{itemize} 

\end{enumerate}

\subsection{remarks of Sone's work}
\begin{enumerate}
 \item In each of expansions, the thermal stress contribution appears in different ways and at different orders. For example, from 1) thermal stress and the slip boundary conditions appear in higher order, from 3) thermal stress appear in the leading order.
 \item Thermal stress flow, or thermal creep flow can occur at different orders in different regimes. For example, from 1) these flow can only be seen at higher order but from 3) flows can be seen from the leading order.
 \item This book is for the steady boundary value problem.
\end{enumerate}


\section{Work of Maxwell}
Maxwell's paper can be summarized as his two assertions and discussions of four different phenomena. What he did is the {\bf Chapman-Enskog} expansion truncated at higher order(Burnett order).

He asserted

 (1) in momentum equation $ \rho \frac{d\mathbf{v}}{dt} = div{T} $,

{{{$$ T = -p \delta_{ij} + T^v + T^M$$}}} and the additional contribution {{{$$T^M = -3\frac{\mu^2}{\rho\theta} \big( D^2 \theta + \frac{1}{2}\Delta \theta\big).$$}}}

 (2) On the solid boundary where temperature gradient exists, the velocity slips and the tangential velocity $u$ is given by

{{{ $$u = G(\rho,\theta) \left[ \frac{\partial u}{\partial x_n} - \frac{3}{2} \frac{\mu}{\rho\theta} \frac{\partial^2\theta}{\partial x_n\partial x_t}\right]  + \frac{3}{4} \frac{\mu}{\rho\theta} \frac{\partial\theta}{\partial x_t},$$ }}}
where $x_n$ and $x_t$ are the normal and the tangential components at the surface, $G(\rho,\theta)$ is a kind of a constitutive relation.
when $G(\rho,\theta)=0$, there is only the tangential temperature gradient contribution.

Maxwell succeeded to explain  the four phenomena with this new contribution. {\it The sign of the contribution was correct to explain the direction of thermal stress in the four phenomena.}  Concerning the bobylev instability at Burnett order, why it was successful despite of the bobylev instability at the Burnett order will be discussed later when I talk about the work of Karlin and Gorban.

\subsection{four phenomena appeared in the paper}
Four phenomena Maxwell discussed are more or less related to the crookes radiometer and after this paper, it seems the there were agreement among the researcher.

\begin{enumerate}
 \item[(A)] temperature on the body is higher (resp. lower) than the ambient temperature. Maxwell said {{{$\theta_{xx}$}}} on $x$-axis has the fixed sign and then there will be an excessive (resp. less) stress on the axis between the bodies, so two bodies repel each other. (resp. attract)
 \begin{figure}[ht]
  \centering
  \includegraphics[height=5cm]{Maxwell_A.png}
 \end{figure}
 
 \item[(B)] Cup-shaped body on which temperature is constant. temperature and their derivatives at $p$ is comparable to those with spherical body is placed. On the while, the temperature and their derivatives at $q$ is comparable to those when $q$ is the inside point of the spherical body. This difference makes the stress field non-trivial, and there is the flow (I forgot the direction). 
 \begin{figure}[ht]
  \centering
  \includegraphics[height=5cm]{Maxwell_B.png}
 \end{figure} 
 Indeed, Crookes came with the cup-shaped radiometer at those times. Even though temperature on both side of the vane is constant, (differently from the standard radiometer setting), this apparatus rotates under the light.
 \begin{figure}[ht]
  \centering
  \includegraphics[height=5cm]{cup_radiometer.png}
 \end{figure} 
 
 \item[(C)] Capillary tube on whose boundary temperature gradient presences.

 \begin{figure}[ht]
  \centering
  \includegraphics[height=5cm]{Maxwell_C.png}
 \end{figure} 
 It is in the appendix. Here, he approximately calculated and showed that
 \begin{itemize}
  \item If there is no pressure gradient, then net flow is from colder site to the hotter site.
  \item If there is no net flow, then the pressure at hotter site is higher than that of colder site.
 \end{itemize}
It is notable that Maxwell actually performed the approximate calculation of the influence on the pressure-driven flow(poisseulle flow) of the new stress even when the viscosity effect is taken account. Throughout his paper he is saying that, after long time, temperature relaxes to equilibrium (or become harmonic) and if the temperature at the boundary of the domain doesn't have the gradient, then the contribution of {{{$T^M$}}} in the momentum equation vanishes and also pressure becomes constant, which is consistent with the formula of {{{$T^M$}}} above. Speaking of this point, Maxwell pointed out that, when $T^M$ does not contribute and there is no slip at the boundary then there is no way to have the thermal flow in the steady state, insisting that in the thermal creep flow, the slip boundary condition is everything. I'll endeavor this point again later when we consider the Korteweg's stress tensor, not the Maxwell's stress tensor.

 \item[(D)] the standard crookes radiometer
  \begin{figure}[ht]
  \centering
  \includegraphics[height=5cm]{2dgeometry.png}
 \end{figure} 

\end{enumerate}

\subsection{remarks of Maxwell's work and discussions}
\begin{enumerate}
 \item All 4 phenomena cannot occur in Navier-Stokes model with no-slip condition.
 \item Maxwell separated issues on assertion (1) on thermal stress and assertion (2) on slip boundary condition. [A] and [B] can be explained without the assertion (2).
 \item There are two ways of demonstration of the presence of the additional contribution in the stress. To make things easier, let's suppose the {\bf temperature is maintained externally} (this amounts to say we ignore the energy equation). The demonstrations depend on what variable other than the temperature we have in our control. Two extremes are found:
 \begin{enumerate}
  \item[(a)] If the {\bf pressure is in our control}, then easiest case is the iso-baric case where the pressure is constant. Then the thermal stress exerts so that there is a flow {\bf from the colder site to the hotter site}.
  \item[(b)] If the {\bf velocity is in our control}, then easiest case is the no flow case where the velocity is $0$. Then the thermal stress exerts so that there is a pressure gradient and {\bf the pressure at the hotter site is greater than that at the colder site.}
 \end{enumerate}

The reasoning is as follows: Suppose there is any additional contribution in the stress term. Then, if we consider the pressure-driven flow (say, the poisseuille flow), the flow behavior is for sure influenced. For simplicity, let us ignore the viscosity for the moment. If no pressure gradient is applied, then for the classical model, there shouldn't be the acceleration. The presence of additional contribution can be demonstrated by showing, there is a non-trivial acceleration even when the pressure is constant. 

The transposition statement of this is : If we don't see any acceleration, then for classical model, pressure has to be constant and the presence of non-trivial pressure gradient implies the presence of the additional contribution in stress.

Then the question, that whether the Korteweg's capillarity theory can also explain the thermal flow or not, is at the {\it direction} of the pressure gradient. The observations of thermally driven flow indicates, the gas flows from colder site to the hotter site. This implies that in the circumstance the flow is absent, the thermal stress is against the pressure gradient or, the pressure at hotter site should be higher. Therefore, in the case (b), the question that the Korteweg theory predicts the thermally driven flow correctly or not boils down to checking whether the pressure increases as temperature increases, and whether the pressure decreases as temperature decreases.

In reality, we may be in the mixed situation where less pressure gradient exists but still is in same direction with the temperature gradient, and less acceleration from the colder site to the hotter site.
\end{enumerate}

\section{Work of Dunn and Serrin}
I didn't summarize the work of Dunn and Serrin. Hoever, the most critical message from this paper in regard to our study is that,
\begin{itemize}
 \item From the free energy function that depends on higher gradient of motion and temperature, this paper derives the stress tensor.
 \item Korteweg type stress is inconsistent with the Clausius-Duhem inequality, so the authors appended the additional energy flux, the interstitial working.
 \item At the beginning, the stress tensor could have included the terms of $D^2\theta$ because the free energy function depended on higher gradient of the temperature. However, due to the Clausius-Duhem inequality, this possibility is excluded in the course of the stress derivation.
 \item I need to know whether this exclusion is unavoidable or we could have modified theory by making more efforts, as same as they appended the interstitial working term to make the Korteweg tensor be consistent to the Clausius-Duhem inequality. I will read this paper more carefully so that I can answer to this question !
\end{itemize}

\section{Work of Karlin and Gorban, Marshall}
Karlin and Gorban explain the chapman-enskog expansion through the idea of relaxation to the slow manifold, regarding the Boltzmann equation as a dynamical system with multiple time scales. In a Boltzmann or other kinetic equation,

{{{$$ \partial_t f + v \cdot \partial_x f = \frac{Q(f,f)}{\epsilon}. $$}}}

\begin{itemize}
 \item From the dynamical system perspective, we treat the field as the field indexed by $v$, $f(v;t,x)=f^v$,
$$ \partial_t f^{v} + Lf^{v} = \frac{Q(f^v,f^v)}{\epsilon}, $$
and view the equation as an infinite dimensional dynamical system. Assuming the suitable integrability conditions, $\{f^v\}_{v\in \mathbb{R}}$ can be expanded to the equivalent infinite number of moments, $(M^0, M^1_{i}, M^2_{ij}, \cdots)$.
 \item Presence of $\epsilon$ in the denominator of r-h-s indicates that the system has the fast-slow structure. To be precise, the nonlinearity $Q(f,f)$ has the kernel, the local Maxwellians, and the set of local Maxwellians is put as the template manifold around which the orbits possibly relax.
 \item  The fact that integrals of $Q(f,f)$, $vQ(f,f)$, $|v|^2 Q(f,f)$ vanish, i.e. the 0th moment, first moments, and one second moment vanish indicates that these $5$ variables are not the fast variables but the slow variables. They are $\rho$, $\mathbf{v}$, and $\theta$, the density, velocity, and temperature.
 \item The question is the existence of the {\bf hydrodynamic invariant manifold}, which is around the set of local maxwellians, and on which all other fast components of $f^v$ are determined by the values of the first $5$ moments. 
 
 Note that these moments are in fact fields, with another independent variable $x$. Determinacy here is of functional, for example for a higher moment $M$ that
 $$M = M[\rho,\mathbf{v},\theta]$$
 is a functional dependency upon the $5$ functions that depend on $x$. If, in addition this dependency has a certain {\it locality} restriction in $x$, then we would be able to convert the functional expression into the expression of differential operator
 $$M(x;\epsilon)= \mathcal{M}(\rho(x),D\rho(x),D^2\rho(x), \cdots, \mathbf{v}(x), D\mathbf{v}(x), D^2\mathbf{v}(x), \cdots, \theta(x), D\theta(x), D^2\theta(x), \cdots).$$
 
 The hypothetical picture is that if the field is placed off the manifold then it undergoes the fast relaxation to the hydrodynamic manifold, and the dynamics afterwards is on the manifold, which is the intermediate asymptotics. 
 \item For {\it a given kinetic model}, if we can prove what is said above, then the intermediate asymptotic specify {\it a hydrodynamic model} and we are interested in how the model looks like, or how the stress tensor in the model looks like. 
 
 One important point is that this specific hydrodynamic model, which is parametrized by $\epsilon$, would be of the {\it regularly perturbed system}. Whether or not the $M(x;\epsilon)$ has an expansion in $\epsilon$ does not spoil this fact.
\end{itemize}

In the below, I focus on the term $\theta_{xx}$ and its sign and the above issue on regular perturbation with respect to the $\epsilon$.


\subsection{Linearized Grad 13 moments or 10 moments system around the global maxwellian}
This is a brief summary of the paper. The Grad 13 moments or 10 moments system is considered as a given kinetic model, which is parametrized by $\epsilon$ the Knudsen number. The system is linearized around the constant state of $(\rho_0,\mathbf{v}_0,\theta_0)$. From perspective of the Sone's work, this corresponds to his first expansion where Boltzmann equation is linearized around the global maxwellian. There, the interpretation was such that the perturbation size is much less than the Knudsen number. Thus one might be skeptical on the further inference of the results in this line.

\begin{enumerate}
 \item For this linearized system, the hydrodynamic invariant manifold exists, in particular regardlessly of $\epsilon$. However, the expansion in $\epsilon$ fails : high order chapman-enskog exhibit the bobylev instability.
 
 \item The ultimate form of the stress in terms of lower moments in fourier space is
 $$\sigma(k) = - k^2 B(k^2) p(k) - i k A(k^2)v(k),$$
 where $p = \rho + \theta$, (the linearized pressure) and {{{$A(k^2)$}}} and {{{$B(k^2)$}}} are two functions of {{{$k^2$}}} that are negative. So, it is a non-local operator on $p$ and $v$. 
 \item Importantly, this stress does not suffer from the bobylev instability. In the course, the rescale has been done, so in order to invoke $\epsilon$, we need to transform back by replacing $k \mapsto \epsilon k$. 
 \item The functions $A(k^2)$ and $B(k^2)$ are low-pass filters. 
%  
%  Approximately, {{{ $$ A(k^2) \sim -\frac{4}{3 + 9k^2}, \qquad B(k^2) \sim -\frac{4}{3 + 5k^2}. $$ }}}
%  If we invoke $\epsilon$,

Approximately,
{{{ $$ A(k^2) \sim -\frac{4}{3 + 9\epsilon^2 k^2}, \qquad B(k^2) \sim -\frac{4}{3 + 5\epsilon^2 k^2}. $$ }}}
It decays to $0$ as $k^2 \rightarrow \infty$ with tail behavior of $O(\frac{1}{k^2})$.

 \item Marshall presented the entropy dissipation equality for the 10-moments system in the form
$$ \frac{1}{2} \partial_t \int_{-\infty}^{\infty} \frac{3}{5} p(k)^2 + u(k)^2 \; dk + \frac{1}{2} \int_{-\infty}^{\infty} -\frac{3}{5}k^2 B(k^2)p(k)^2 \; dk = \int_{-\infty}^{\infty} k^2 A(k^2) u(k)^2 \; dk, $$
which is consistence in the sense that $A(k^2)$ and $B(k^2)$ are negative. The first term is the classical free energy and the kinetic energy and the second contribution can be interpreted as the {\it non-local capillary} energy.

 \item One can give the heuristic explanation to connect above {\it non-local} to the {\it local} Korteweg type theory. Marshall explained in the paper but I want to refine it as below.  
 \begin{enumerate}
 \item The leading order of $-\frac{3}{5}k^2 B(k^2) \sim -\frac{3}{5}B(0) k^2$
 
 This is the second derivativek, i.e., {\it local} stress with the {\it correct sign}. However the reason why it has the correct sign is rather delicate because in the 13-moments system, it would have a different sign; the theme is that we do not truncate at the finite order of $\epsilon$.
 
 \item The reason why the above local leading order has the correct sign is that the bobylev instability of the 10-moments system begins at the Super-burnett order, not from the Burnett order, as can be seen in the paper of Karlin and Gorban. 
 
 \item Without the truncation at the leading order, still we can give the formal explanation to reach the {\it local} version of the stress.
 
 \item For simplicity, let's ignore constants and examine the expression 
 $$b(k^2) = \frac{-1}{1+\epsilon^2 k^2}.$$ A few leading orders of $-k^2b(k^2)$ will be
 $$ k^2(1 - \epsilon^2k^2),$$
 where we see the bobylev instability from the second term.
 \item The expression $-k^2b(k^2)$ is expected to be {\it regularly perturbed}. We can be quite concrete on this. $b(k^2)$ has the inverse fourier transform, which is 
$$ b(\epsilon,x) = \frac{1}{2\epsilon} e^{-\frac{|x|}{\epsilon}} $$
and this kernel has the integral $1$ in entire space. Equivalent way to say that is 
\begin{align*}
 \sigma_{capillary} &= \Delta q, \quad q = (Id - \epsilon^2 \Delta) p, \quad (q(k) = 1+\epsilon^2k^2 p(k)), \quad \text{or}\\
 \sigma_{capillary} &= \Delta(Id - \epsilon^2 \Delta)^{-1} p.
\end{align*}
I remember those expression from Marshall's talk in regard to the failure of truncation of $(I-\epsilon^2 \Delta)^{-1}$. 

\item I did not give any proof, but the above expression formally should converges to $\Delta p$ in {\it a good mode of convergence}.
\item It is true that we would have another $\epsilon$ in front of $\sigma_{capillary}$, so the asymptotic limit as $\epsilon \rightarrow 0$ could be even more trivial. This amounts to say we are considering the limit of multiplication of two limits. One is overall size diminishing, and the other is the convergence to the {\it local} operator. My opinion is that both limits behave well ({\it regularly perturbed}) and the limit of the multiplication behaves well. The $\epsilon$ that determines the overall size of the $\sigma_{capillary}$ indeed has to be examined with the non-dimensional quantities that specify the problem. Thus, this $\epsilon$ has to be separately determined with respect to the problem specification.
\end{enumerate}
\item Now, we come the the local stress $p_{xx} = \rho_{xx} + \theta_{xx}$ that would appear in the limit $\epsilon \rightarrow 0$. Here we have the contribution $\theta_{xx}$ that does not appear in the Korteweg stress tensor. I address this issue because we shortly mentioned that this has incorrect sign in the introduction section of the previous thermal creep paper. After spending some times, I became to think that it has the correct sign and it has the same sign as the Maxwell's tensor(not surprizingly). I think this was the reason why Maxwell succeeded to explain phenomena in his paper. As to the bobylev instability, first of all the instability is the consequence of the eigenvalue analysis for entire linear system rather than the consequence of one term. Besides, it is for the high-frequency limit for a fixed $\epsilon$, the frequency $k$ and the knudesen number $\epsilon$. The instability happens when $k \rightarrow \infty$ for a fixed $\epsilon$. If we do not probe the high frequency functions it is more or less okay; recall that Sone either did not take those high frequency functions into account by assuming derivatives of functions are of comparable size to the functions themselves.

\item Regarding to the non-local to local issue, it is worth to look at the approximation Euler-Korteweg model (in Relative Energy paper (2.53)), which was motivated by the numerical concerns.

\item From the form of the kernel 
$$ b(\epsilon,x) = \frac{1}{2\epsilon} e^{-\frac{|x|}{\epsilon}}, $$
it is quite clear why the expansion in $\epsilon$ fails.
\end{enumerate}

\section{Hamiltonian Formalism of Euler-Kortweg Model}
Euler-Kortweg model is from the second gradient theory. It has interesting hamiltonian structure. I intentionally did not include the details here but I am studying it in the other direction. We pick an Euler-Korweg model for {\it Isothermal process}.
\begin{equation} \label{Euler-Kortweg}
\begin{aligned}
 &\dot{\rho} + \rho \div_x \mathbf{v} = 0 \\
 &\rho \dot{v} = \div_x \left[\left(-\big(\rho h^\prime(\rho)- h(\rho)\big) -\frac{1}{2} \Big(\big(\rho\kappa^\prime(\rho)-\kappa(\rho)\big)|\nabla_x\rho|^2\Big) + \rho\div_x\big(\kappa(\rho)\nabla_x \rho\big)\right)\delta_{ij}\right.\\
 &\left. -\Big(\kappa(\rho)\nabla_x\rho \otimes \nabla_x \rho\Big)\right].
\end{aligned}
\end{equation}

It is associated with the free energy
\begin{equation}
 \mathcal{E}[\rho] = \int h(\rho) \;+\; \frac{1}{2}\kappa(\rho)\, |\nabla_x\rho|^2 \:dx.
\end{equation}
and we write the hamiltonian in the Lagrangian coordinate $X$,
\begin{equation}
 H[x, \mathbf{m}] = \int \frac{1}{2\rho_0(X)} | \mathbf{m} |^2 \;+\; \left[\frac{h(\rho)}{\rho} \;+\; \frac{1}{2}\frac{\kappa(\rho)}{\rho} \, |\nabla_x \rho|^2\right] \, \rho_0(X) \:dX,
\end{equation}
where $\rho_0$ is such that $\rho_0dX = \rho dx$, with the Lagrangian coordinate $X$, $\mathbf{m}(X,t)=\rho_0(X) \mathbf{v}$, the momentum density w.r.t the Lagrangian volume form $dX$. Then we have the canonical equations
\begin{align}
 \langle \dot{x}, \psi \rangle &= \langle \frac{\delta H}{\delta \mathbf{m}}, \psi \rangle, \label{eq:xdot}\\
 \langle \dot{\mathbf{m}}, \phi \rangle &= \langle -\frac{\delta H}{\delta \mathbf{x}}, \phi \rangle, \label{eq:mdot}
\end{align} 
which gives the \eqref{Euler-Kortweg}
\section{Work of Kareem and Min-Gi}
In the earlier section, I addressed that 
\begin{enumerate}
 \item the stress out of the kinetic consideration has the term $\theta_{xx}$ that the Kortweg stress does not have.
 \item the term has the correct sign; Maxwell's explanations were successful.
\end{enumerate}
In this section, we study 1D Euler-Kortweg model, making comparison to the Maxwell's model. We attempt to verify the following statements hold for both of 1D Euler-Korteweg model and the Maxwell's model.
 \begin{itemize}
  \item If there is no pressure gradient, then net acceleration is from colder site to the hotter site.
  \item If there is no net acceleration, then the pressure at hotter site is higher than that of colder site.
 \end{itemize}

However, the way they are so has differences, so I will emphasize
\begin{itemize}
 \item the technical differences to arrive at above statements. 
 \item the situation that two models would predict different.
\end{itemize}


\subsection{Consistency of Kortweg model to the Maxwell model: two special cases}
We consider the radial symmetric 1-D problem, where the temperature $\theta(x)$ is externally maintained.
 \begin{figure}[ht]
  \centering
  \includegraphics[height=5cm]{1Dmodel.png}
 \end{figure}
This corresponds to the phenomenon [A] that maxwell discussed where two hotter (resp. colder) bodies are placed symmetrically at left side and the right side. Note that this problem does not have the surface-gas interaction on which temperature exhibits gradient, i.e., this is about the stress tensor, but might not be about the thermal creep. This problem is far simpler than the radiometer problem.

We consider the equations of motion,
$$ \rho \frac{d\mathbf{v}}{dt} = div(T), \quad \text{with}$$

Maxwell : {{{ $T = -p \delta_{ij} + T^v + T^M(\rho,\theta)$ }}}

Korteweg : {{{ $T= -p \delta_{ij} + T^v + T^K(\rho,\theta)$, }}}
respectively. $T^v$ is the viscous stress.
\begin{align*}
T^M &= -3\frac{\mu^2}{\rho\theta} \big( D^2 \theta + \frac{1}{2}\Delta \theta\big),\\
T^K &= \Big[-\frac{1}{2} \big(\rho c_\rho (\rho,\theta) + c(\rho,\theta) |\nabla \rho|^2\big) + \nabla \cdot \big(\rho c(\rho,\theta) \nabla \rho\big) \Big] \delta_{ij} - c(\rho,\theta) \nabla \rho \otimes \nabla\rho,
\end{align*}
where $c(\rho,\theta)$ is the capillary coefficient.

We verify the statements we gave in the above with two models. Before stating the results here we find the technical difference. In 1D case, if the capillary coefficient {{{ $c(\rho,\theta) = c_0 \rho^k \theta^\ell$}}}, then the
the {{{$T^K$}}} has the following expression  $$ T^k = \frac{c_0}{\alpha} \rho^{\alpha+1} \Big[ \theta^{\ell}(x) (\rho^{\alpha})^\prime \Big]^\prime=:A_\theta[\rho], \quad \alpha = \frac{k+1}{2}, $$ which is a non-linear elliptic operator on $\rho$ for externally given $\theta(x)>0$.
Having observed that, the presence of $D^2\theta$ in the maxwell's tensor and the absence of it in the Korteweg's stress tensor make the difference in the following way.
\begin{enumerate}
 \item In the Maxwell's model, no matter what the boundary conditions are, the sign of $T^M$ is the {\it local property} that depends only on the local temperature distribution.
 \item In the Korteweg's model, the boundary conditions crucially determine the sign; Only after we solve the boundary value problem we can tell the sign of $T^K$. It is kinematically determined and depends on function values on every other point. It is thus the {\it non-local property}, 
\end{enumerate}

\subsubsection{Iso-baric case : temperature and pressure are in control}
Now, we first consider the {\it Iso-baric} case, from which we verify the first statement of
 \begin{itemize}
  \item If there is no pressure gradient, then net acceleration is from colder site to the hotter site.
  \item If there is no net acceleration, then the pressure at hotter site is higher than that of colder site.
 \end{itemize}

Suppose we have the pressure in our control, and consider the isobaric case $\rho = \frac{1}{\theta}$. Then formula {{{$T^K=A_\theta[\rho]$}}} amounts to say that $T^k > 0$
\begin{align*}
 \begin{cases}
  \text{$\theta^{\ell-\alpha}$ is concave} & \text{when $\ell-\alpha>0$, }\\
  \text{$\theta^{\ell-\alpha}$ is convex}  & \text{when $\ell-\alpha<0$, }
 \end{cases}
\end{align*}
assuming $\alpha>0$.
Compare this to that $T^M >0$ if $\theta$ is concave and $T^M <0$ if $\theta$ is convex. 

In isobaric case, it does behave similarly, in a reasonable range of parameters to that of Maxwell. Again, the sign of Maxwell stress tensor should be as it is and not wrong. 

\subsubsection{No acceleration : temperature and velocity are in control}
Now, we verify the second statement of 
 \begin{itemize}
  \item If there is no pressure gradient, then net acceleration is from colder site to the hotter site.
  \item If there is no net acceleration, then the pressure at hotter site is higher than that of colder site.
 \end{itemize}
In this case, by the Gallilean symmetry, we put the velocity $v=0$. The advantage of this demonstration is that, we are only left with the equation of the balance in the total stress 
\begin{align*}
 div(T) = 0
\end{align*}
In 1D case,
\begin{align*}
 \partial_x(\rho\theta - T^k) = 0, \quad \text{or} \quad - A_\theta[\rho] + \theta\rho  = T^{total}_0
\end{align*}
Now the ellipticity of the operator plays the crucial role here. We set up the boundary conditions as below.
\begin{enumerate}
 \item Because of the radial symmetry, we look for the $\rho$ that is even function of $x$. We solve the problem in $[0,L]$.
 \item We set $\rho'(0)=0$.
 \item We parametrize $\rho(0)=\rho_0$.
 \item We parametrize the total stress $T^{total}$.
\end{enumerate}
As addressed earlier, the sign of the Korteweg stress is determined after we solve the boundary value problem and how $\rho$ re-distributes out is non-locally determined depending on the parameters and the external function $\theta(x)$ critically. But here we have too many freedom because we cannot fix $\rho_0$ and $T^{total}$ and the slop or the convexity of $\theta(x)$ are quite arbitrarily small or large. Nevertheless, we could observe a certain consistency from the numerical calculations of various instances.

\begin{remark}
From numerical calculations we observed the following: Let $[0,L]$ be the domain of interest, and $\theta(x)$ be the fixed temperature distribution that is monotone increasing (resp. decreasing) in $x$. Choose $\rho_0>0$ and $T^{total}$. Then, for this choice of parameters, either the solution breaks down, i.e. $\rho(x_0)<0$ at $x_0<L$ or the pressure is monotone increasing (resp. decreasing) in $x$, i.e., the pressure at hotter place is higher.
\end{remark}
We do not have the proof, but this might not be trivial, because we tested $T^{total}$ that is positive and negative.
\begin{figure}
 	\centering
    \subfigure[{\scriptsize pressure increases as temperature increases}] {
	   \includegraphics[width=7cm]{lorentzdist-inc.pdf} \label{fig:1}
    } \hfill
    \subfigure[{\scriptsize pressure decreases as temperature decreases}] {
	\includegraphics[width=7cm]{parabola-dec.pdf} \label{fig:2}
    }
	\caption{Numerical calculation of the pressure behavior influenced by temperature gradient.}
\end{figure}

For the numerical calculations, radially symmetric temperature profiles that are monotone in the radial variable $x \in [0,L]$ were tested, 
$$ \theta(x) = A+ \frac{B}{1+Cx^2}, \quad \text{or} \quad \theta(x) = Dx^2 + E. $$

\subsection{Discussion}
We examined the two examples. In the first example, we imagine the artificial situation that the Korteweg stress tensor is turned on at $t=t_0$ before which the gas is isobaric with respect to the given temperature $\theta(x)$. At the moment $t=t_0$, the direction of the acceleration is {\it consistent} and is {\it locally determined} depending on the temperature distribution. In the second example, we considered the situation that we assign the nontrivial temperature in a bounded region and wait long time until we do not see any flow motion. When the gas is at this state, the direction of the pressure gradient is {\it consistent} but is {\it non-locally determined} according to the boundary conditions and $\theta(x)$.

Therefore, the Korteweg model in 1D in those two extreme cases behaves {\it consistently} to the Maxwell's model. The mechanism it provides is rather different however. In the Korteweg model, it can be interpreted that, from energy perspective, it penalizes the density gradient, which in turn apparently gives the temperature effect in consistent way. 

If one considers the initial-value problem with same boundary conditions in a bounded region then the Maxwell's model and the Korteweg's model possibly converges to the similar steady states. However, it is doubtful that the intermediate flow of them are so. We perhaps can make an example that 
$|\theta_{xx}|$ is so high that it dominates the $\nabla p=\nabla (\rho\theta)$ and makes flow reverse against the pressure gradient. This is possible in the Maxwell's model. It is not clear how Korteweg model can implement such a flow.

Besides, as we discussed in the section of Dunn and Serrin, we could have the $D^2\theta$ term in the stress of the second gradient theory and we have not explored this possibility yet. 

Overall impression and perspective I have is that in the isothermal phenomena, it is strongly suggestive that the Korteweg theory in a certain sense agrees to the theory from the Kinetic model. For the phenomena where temperature plays the crucial role, it seems we have many possibilities that we have not yet reviewed.

\subsubsection{Isothermal capillary problem vs. thermal-stress flow problem}
It could have an interesting relation to the Isothermal capillary problem in 1D with the gravitational acceleration in $z$ because there we have
$$\rho(z)\theta_0+z - T^K[\rho(z)] = T^{total}.$$

\end{document}
