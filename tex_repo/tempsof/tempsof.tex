%%%%%%%%%%%%%%%%%%%%%%%%%%%%%%%%%%%%%%%%%%%%%%%
%
%    Self-Similar shear bands, Existence, Numerics, Asymptotics
%
%                                                      by
%
%                                       Min-Gi Lee   
%
%                                          version Sep 2016
%
%
%%%%%%%%%%%%%%%%%%%%%%%%%%%%%%%%%%%%%%%%%%%%%%%
\documentclass[a4paper,11pt]{article}

\usepackage[margin=2cm]{geometry}
\usepackage{setspace}
%\onehalfspacing
\doublespacing
%\usepackage{authblk}
\usepackage{amsmath}
\usepackage{amssymb}
\usepackage{amsthm}

\usepackage[notcite,notref]{showkeys}

% \usepackage{psfrag}
\usepackage{graphicx,subfigure}
\usepackage{color}
\def\red{\color{red}}
\def\blue{\color{blue}}
%\usepackage{verbatim}
% \usepackage{alltt}
%\usepackage{kotex}

\usepackage{enumerate}

%%%%%%%%%%%%%% MY DEFINITIONS %%%%%%%%%%%%%%%%%%%%%%%%%%%

\def\tr{\,\textrm{tr}\,}
\def\div{\,\textrm{div}\,}
\def\sgn{\,\textrm{sgn}\,}

\def\th{\tilde{h}}
\def\tx{\tilde{x}}
\def\tk{\tilde{\kappa}}

\def\tg{{\tilde{\gamma}}}
\def\tv{{\tilde{v}}}
\def\tth{{\tilde{\theta}}}
\def\ts{{\tilde{\sigma}}}
\def\tu{{\tilde{u}}}
\def\tph{{\tilde{\varphi}}}

\def\dtg{{\dot{\tilde{\gamma}}}}
\def\dtv{{\dot{\tilde{v}}}}
\def\dtth{{\dot{\tilde{\theta}}}}
\def\dts{{\dot{\tilde{\sigma}}}}
\def\dtu{{\dot{\tilde{u}}}}
\def\dtph{{\dot{\tilde{\varphi}}}}

\def\dpp{\dot{p}}
\def\dqq{\dot{q}}
\def\drr{\dot{r}}
\def\dss{\dot{s}}

\def\ta{{\tilde{a}}}
\def\tb{{\tilde{b}}}
\def\tc{{\tilde{c}}}
\def\td{{\tilde{d}}}




\def\bx{\bar{x}}
\def\bm{\bar{\mathbf{m}}}
\def\K{\mathcal{K}}
\def\E{\mathcal{E}}
\def\del{\partial}
\def\eps{\varepsilon}

\newcommand{\tcr}{\textcolor{red}}
\newcommand{\tcb}{\textcolor{blue}}

\newcommand{\ubar}[1]{\text{\b{$#1$}}}
\newtheorem{theorem}{Theorem}
\newtheorem{lemma}{Lemma}[section]
\newtheorem{proposition}{Proposition}[section]
%\newtheorem{definition}{Definition}[section]
\newtheorem{remark}{Remark}[section]

%%%%%%%%%%%%%%%%%%%%%%%%%%%%%%%%%%%%%%%%%%%%%%%%%%%%%%%%%%
\begin{document}
\title{Note for the Self-similar shear bands problem}
\author{Min-Gi Lee\footnotemark[1]}
% \and Athanasios Tzavaras\footnotemark[1]\  \footnotemark[3]  \footnotemark[4]}
\date{}

\maketitle
\renewcommand{\thefootnote}{\fnsymbol{footnote}}
% \footnotetext[1]{Computer, Electrical and Mathematical Sciences \& Engineering Division, King Abdullah University of Science and Technology (KAUST), Thuwal, Saudi Arabia}
% \footnotetext[2]{Department of Mathematics and Applied Mathematics, University of Crete, Heraklion, Greece}
% \footnotetext[3]{Institute of Applied and Computational Mathematics, FORTH, Heraklion, Greece}
% \footnotetext[4]{Corresponding author : \texttt{athanasios.tzavaras@kaust.edu.sa}}
%\footnotetext[4]{Research supported by the King Abdullah University of Science and Technology (KAUST) }
\renewcommand{\thefootnote}{\arabic{footnote}}


\maketitle

\tableofcontents
% \begin{abstract}
% abstract
% \end{abstract}

\section{The model description}
We consider a 1-d shear deformation of a material whose material law of stress depends on 1) temperature, 2) strain, 3) strain rate. The motion is described by following field variables,
\begin{equation} \label{eq:vars}
\begin{aligned}
 \gamma(t,x) &: \text{strain}\\
 u(t,x)=\gamma_t &: \text{strain rate}\\
 v(t,x) &: \text{vertical velocity}\\
 \theta(t,x) &: \text{temperature}\\
 \tau(t,x) &: \text{stress}
\end{aligned}
\end{equation}
The material exhibits 1) temperature-softening, 2) strain-hardening, 3) rate-hardening. we denote the shear stress
$$ \tau = \tau(\theta,\gamma,u). $$
and study a model
\begin{equation}
 \tau = \theta^{-\alpha}\gamma^m u^n. \label{eq:stresslaw}
\end{equation}

A few forehand perspectives :
\begin{enumerate}
 \item The regime where $-\alpha+m+n <0$ will exhibit the localization, whereas the regime $-\alpha+m+n > 0$ will exhibit stabilization. {\blue Can we rigorously study the linearize stability to the uniform shearing solution?}
 \item The uniform shearing solution will appear as one of the self-similar solution by a specific $\lambda$ that is negative.
\end{enumerate}
\subsection{A system of conservation laws}
For the field variables \eqref{eq:vars}, equations describing the deformation are given by
\begin{align}
 \gamma_t &= u, \quad \text{(kinematic compatibility)} 	\label{eq:g}\\
 v_t &= \tau_x, \quad \text{(momentum conservation)} 	\label{eq:v}\\
 \theta_t &= \tau u \quad \text{(energy conservation)}	\label{eq:th}\\
 \tau &=\theta^{-\alpha}\gamma^m u^n.			\label{eq:tau}
\end{align}
\subsection{Scale invariance property of the system}
The system \eqref{eq:g}-\eqref{eq:tau} admits a scale invariance property. Suppose $(\gamma,u,v,\theta,\tau)$ is a solution. Then a rescaled version of it 
\begin{align*}
 \gamma_\rho(t,x) &= \rho^a\gamma(\rho^{-1}t,\rho^\lambda x), &
 v_\rho(t,x) &= \rho^bv(\rho^{-1}t,\rho^\lambda x),\\
 \theta_\rho(t,x) &= \rho^c\theta(\rho^{-1}t,\rho^\lambda x) &
 \tau_\rho(t,x) &= \rho^d\tau(\rho^{-1}t,\rho^\lambda x),\\
 u_\rho(t,x) &= \rho^{b+\lambda}\gamma(\rho^{-1}t,\rho^\lambda x). 
\end{align*}

Calculations :
\begin{align*}
 \text{let} \;f(t,x) = \rho^k F(\rho^{-1}t,\rho^\lambda x), \\
 \partial_t f(t,x) = \rho^k \partial_{t'} F(\rho^{-1}t,\rho^\lambda x) \rho^{-1} = \rho^{k-1} \partial_{t'}F, \\
 \partial_x f(t,x) = \rho^k \partial_{x'} F(\rho^{-1}t,\rho^\lambda x) \rho^\lambda = \rho^{k+\lambda} \partial_{x'}F.
\end{align*}
Relations for invariance :
\begin{align*}
 a-1 = b+\lambda, \quad b-1 = d+\lambda, \quad c-1 = d+b+\lambda, \quad d = -\alpha c + m a + n (b+\lambda)
\end{align*}
From above, we reach to exponents
\begin{align*}
 D & = 1+2\alpha-m-n,\\
 a&= \frac{2+2\alpha-n}{D} + \frac{2+2\alpha}{D}\lambda =: a_0 + a_1 \lambda, & b&=\frac{1+m}{D} + \frac{1+m+n}{D}\lambda =: b_0 + b_1\lambda,\\
 c&=\frac{2(1+m)}{D} + \frac{2(1+m+n)}{D}\lambda =: c_0 + c_1\lambda, & d&=\frac{-2\alpha + 2m +n}{D} + \frac{-2\alpha+2m+2n}{D}\lambda =: d_0 + d_1\lambda
\end{align*}
for each $\lambda \in \mathbb{R}$. For localization, $\lambda>0$. But uniform shearing solution takes a negative $\lambda$.
\subsection{Self-Similar variables}
We try the solutions of type, i.e. put $\rho =t$ in the rescaling,
\begin{align*}
 \gamma(t,x) &= t^a\Gamma(t^\lambda x),\\
 v(t,x) &= t^b V(t^\lambda x),\\
 \theta(t,x) &= t^c \Theta(t^\lambda x),\\
 \tau(t,x) &= t^d \Sigma(t^\lambda x),\\
 u(t,x) &= t^{b+\lambda} U(t^\lambda x)
\end{align*}
and set $\xi = t^\lambda x$.

Calculations:
\begin{align*}
 &\text{Suppose } \; f(t,x) = t^k F(t^\lambda x),\\
 &\partial_t f = k t^{k-1} F + t^k F' \lambda t^{\lambda-1} x = t^{k-1} (kF + \lambda\xi F'),\\
 &\partial_x f = t^k F' t^\lambda = t^{k+\lambda} F',
\end{align*}

At \eqref{eq:g}-\eqref{eq:tau}:

\begin{align*}
 t^{a-1}(a \Gamma(\xi) + \lambda \xi \Gamma'(\xi)) &= t^{b+ \lambda} U(\xi),\\
 t^{b-1}(b V(\xi) + \lambda \xi V'(\xi)) &= t^{d+ \lambda} \Sigma'(\xi)\\
 t^{c-1}(c \Theta(\xi) + \lambda \xi \Theta'(\xi))&=t^{b+d+\lambda} \Sigma U(\xi),\\
 t^d\Sigma(\xi) &= t^{-\alpha c +ma +n(b+ \lambda)} \Theta(\xi)^{-\alpha} \Gamma(\xi)^m U(\xi)^n,\\
 t^{b+\lambda}V'(\xi)&=t^{b+\lambda}U(\xi)
\end{align*}
\begin{equation}
\begin{aligned}
 a \Gamma(\xi) + \lambda \xi \Gamma'(\xi) &= U(\xi),\\
 b V(\xi) + \lambda \xi V'(\xi) &= \Sigma'(\xi)\\
 d \Theta(\xi) + \lambda \xi \Theta'(\xi)&=\Sigma(\xi) U(\xi),\\
 \Sigma(\xi) &= \Theta(\xi)^{-\alpha} \Gamma(\xi)^m U(\xi)^n,\\
 V'(\xi)&=U(\xi)
\end{aligned} \label{eq:ss-odes}
\end{equation}
\subsection{de-singularization}
Introduce new field variables 
\begin{equation}
\begin{aligned}
 \Gamma(\xi) &= \xi^\ta \tg(\xi),\\
 V(\xi)&=\xi^\tb \tv(\xi),\\
 \Theta(\xi)&=\xi^\tc \tth(\xi),\\
 \Sigma(\xi)&=\xi^\td \ts(\xi),\\
 U(\xi)&=\xi^{\tb-1} \tu(\xi). 
\end{aligned}
\end{equation}
Then at \eqref{eq:ss-odes}:
\begin{align*}
 \xi^\ta\Big( a\tg + \lambda \ta \tg + \lambda\xi\tg'\Big) &=\xi^{\tb-1} \tu,\\
 \xi^\tb\Big( b\tv + \lambda \tb \tv + \lambda\xi\tv'\Big) &=\xi^{\td-1} \Big(\td\ts + \xi\ts'\Big),\\
 \xi^\tc\Big( c\tth+ \lambda \tc \tth+ \lambda\xi\tth'\Big)&=\xi^{\td+\tb-1} \ts\tu,\\
 \xi^\td\ts &= \xi^{-\alpha \tc +m\ta +n(\tb-1)} \tth^{-\alpha} \tg^m \tu^n,\\
 \xi^{\tb-1}\Big(\tb\tv + \xi \tv'\Big)&= \xi^{\tb-1} \tu.
\end{align*}

$\ta, \tb, \tc, \td$ such that
\begin{align*}
 &\ta=\tb-1, \quad \tb=\td-1, \quad \tc=\td+\tb-1,\quad \td = -\alpha \tc + m\ta +n(\tb-1) \\
 \Longrightarrow \quad&\ta = -a_1, \quad \td = -d_1, \quad \tc = -c_1, \quad \tb=-b_1.
\end{align*}

\begin{equation} \label{eq:tildesys}
 \begin{aligned}
  a_0\tg + \lambda\xi\tg' &=\tu,\\
  b_0\tv + \lambda\xi\tv' &=-d_1 \ts + \xi\ts',\\
  c_0\tth+ \lambda\xi\tth'&=\ts\tu,\\
  \ts &=\tth^{-\alpha}\tg^m\tu^n,\\
  -b_1\tv+\xi\tv' &= \tu.
 \end{aligned}
\end{equation}

Introduce the new independent variable $\eta = \log\xi$.

\section{temperature softening model}
We first focus on the problem where 
$$ \tau = \tau(\theta,u) = \theta^{-\alpha}u^n.$$
Then the first equation \eqref{eq:g} drops out from the system, and we focus on the system
\begin{equation} \label{eq:orisys}
 \begin{aligned}
  v_t &= \tau_x,\\
  \theta_t &= \tau u,\\
  \tau &=\theta^{-\alpha}u^n.
 \end{aligned}
\end{equation}


\section{$(p,q,r)$-system derivation}

\begin{enumerate}
 \item The temperature equation in \eqref{eq:tildesys}, \eqref{eq:orisys} can be rewritten in the form
 $$ \Big(\frac{1}{1+\alpha} \tth^{1+\alpha}\Big)_t = u^{1+n}, \quad c_0\tth^{1+\alpha} + \frac{\lambda}{1+\alpha} \xi\big({\tth^{1+\alpha}}\big)' = \tu^{1+n}. $$
 If we write the system in the form below, with $\tph = \theta^{1+\alpha}$, we see the similarity in the structure of the problem to the problem we had before for strain-softening model. i.e.,
 \begin{align*}
  c_0\tph + \frac{ \lambda }{1+\alpha} \dtph &= u^{1+n},\\
  b_0\tv + \lambda \dtv &= -d_1\ts + \dts,\\
  \dts &= \tph^{-\frac{\alpha}{1+\alpha}} u^n.
 \end{align*}
 Observe that the variable that is responsible for the softening is now $\tph$, which used to be the $\gamma$ in the strain-softening model. Notice also that the exponent $-\frac{\alpha}{1+\alpha} >-1$ always holds for any $\alpha > 0$, which came as the constraint in the strain-softening model. Therefore, we expect the existence of self-similar shear band solutions for the parameter $\alpha>0$, $-\alpha+n<0$.
\end{enumerate}

Auxiliary calculations:
\begin{align*}
 \frac{\dts}{\ts} &= d_1+ b_0\frac{\tv}{\ts} + \lambda \frac{\dtv}{\ts} = d_1+ b_0\frac{\tv}{\ts} + \lambda \Big(b_1 + \frac{\tu}{\tv}\Big)\frac{\tv}{\ts} = d_1 + b\frac{\tv}{\ts} + \lambda\frac{\tu}{\tv}\frac{\tv}{\ts} ,\\
 \frac{\dtth}{\tth}&=\frac{1}{\lambda }\Big(\frac{\ts\tu}{\tth}-c_0\Big),\\
 0&=\frac{\dts}{\ts} +\alpha \frac{\dtth}{\tth} - m \frac{\dtg}{\tg} - n\frac{\dtu}{\tu},\\
 \frac{\dtv}{\tv}&= b_1 +\frac{\tu}{\tv}
\end{align*}

Based on this observations, we define the variables
\begin{equation}\label{eq:pqrsdef}
 \begin{aligned}
  p = \frac{\tth^{\,\frac{1+\alpha}{1+n}}}{\ts}, \quad q=b \frac{\tv}{\ts},  \quad r = \frac{u}{\tth^{\,\frac{1+\alpha}{1+n}}}
 \end{aligned}
\end{equation} 

We have
\begin{align*}
 \frac{\dpp}{p}&=\frac{1+\alpha}{1+n}\,\frac{\dtth}{\tth} - \frac{\dts}{\ts}& &=\Big[\frac{1+\alpha}{1+n}\,\frac{1}{\lambda }\Big(\frac{\ts\tu}{\tth}-c_0\Big)\Big] & &-\left[d_1 + b\frac{\tv}{\ts} + \lambda\frac{\tu}{\tv}\frac{\tv}{\ts}\right]\\
 \frac{\dqq}{q}&=\frac{\dtv}{\tv} - \frac{\dts}{\ts}& &=\left[b_1 +\frac{\tu}{\tv}\right] & &-\left[d_1 + b\frac{\tv}{\ts} + \lambda\frac{\tu}{\tv}\frac{\tv}{\ts}\right]\\
 n\frac{\drr}{r}&=n\frac{\dtu}{\tu} -n\frac{1+\alpha}{1+n}\,\frac{\dtth}{\tth}= \frac{\dts}{\ts} + \Big(\alpha-n\frac{1+\alpha}{1+n}\Big)\,\frac{\dtth}{\tth} & &=\left[\frac{\alpha-n}{\lambda(1+n)}\Big(\frac{\ts\tu}{\tth}-c_0\Big)\right]& &+\left[d_1 + b\frac{\tv}{\ts} + \lambda\frac{\tu}{\tv}\frac{\tv}{\ts}\right]
\end{align*}
Noticing that
\begin{align*}
 \frac{\ts\tu}{\tth} = r^{1+n}, \quad \frac{\tu}{\tv} = \frac{\ts}{\tv} \frac{\tth^{\,\frac{1+\alpha}{1+n}}}{\ts} \frac{u}{\tth^{\,\frac{1+\alpha}{1+n}}} = \frac{bpr}{q}, \quad \frac{\tu}{\tv} \frac{\tv}{\ts} = pr
\end{align*}
we derive $(p,q,r)$-system:
\begin{align*}
 \frac{\dpp}{p}&=\Big[\frac{1+\alpha}{1+n}\,\frac{1}{\lambda }\Big(r^{1+n}-c_0\Big)\Big] -\Big[d_1 + q + \lambda pr\Big]\\
 \frac{\dqq}{q}&=\Big[b_1 +\frac{bpr}{q}\Big] -\Big[d_1 + q + \lambda pr\Big]\\
 n\frac{\drr}{r}&=\Big[\frac{\alpha-n}{\lambda(1+n)}\Big(r^{1+n}-c_0\Big)\Big]+\Big[d_1 + q + \lambda pr\Big]
\end{align*}

\section{Equilibrium points, Linear structure}
$(p,q,r)$-system possesses four equilibrium points and the exceptional equilibriums that we are not interested in. They are
\begin{align*}
  (0,0,0), \quad \Big(0,0,\big(c_0-d_1\frac{1+n}{\alpha-n}\lambda\big)^{\frac{1}{1+n}}\Big), \quad \Big(0,1,\big(c_0-(d_1+1)\frac{1+n}{\alpha-n}\lambda\big)^{\frac{1}{1+n}}\Big), \quad (0,1,0).\\
 (0,0,0), \quad \Big(0,1,\big(\frac{2}{D} + \frac{2(1+n)}{D} \lambda\big)^{\frac{1}{1+n}}\Big), \quad \Big(0,0,\big(\frac{2}{D} -\frac{(1+n)^2}{D(\alpha-n)} \lambda\big)^{\frac{1}{1+n}}\Big), \quad (0,1,0).
\end{align*}

\noindent
{\bf Exceptional cases}
\medskip

Exceptional equilibriums are,
\begin{align*}
 r=0, \quad q=0, \quad \lambda = \frac{1+\alpha}{1+n} \frac{c_0}{-d_1} = \frac{1+\alpha}{1+n} \frac{1}{\alpha-n},\\
 r=0, \quad q=1, \quad \lambda = \frac{1+\alpha}{1+n} \frac{c_0}{-d_1-1}= \frac{1+\alpha}{1+n} \frac{2}{2(\alpha-n)-1},\\
 p = \frac{d_1b_1}{b_0} c_0^{-\frac{1}{1+n}}, \quad q=-\frac{d_1 b}{b_0}, \quad r=c_0^{\frac{1}{1+n}}.
\end{align*}



\section{Normally hyperbolic invariant manifold}
The critical manifold $r=h(p,q,n=0)$.
\begin{equation}
 r=h(p,q,n=0) = \frac{ \frac{\alpha c_0}{\lambda} - d_1 -q }{ \frac{\alpha}{\lambda} + \lambda p}
\end{equation}
This is a smooth function of $p\ge0$ and $q$.

The system in {\it fast scale} with the independent variable $\tilde{\eta} = \eta/n$ is

\begin{align} 
 p^\prime &=np\Big( \Big[\frac{1+\alpha}{1+n}\,\frac{1}{\lambda }\Big(r^{1+n}-c_0\Big)\Big] -\Big[d_1 + q + \lambda pr\Big]\Big), \nonumber \\
 q^\prime &=nq\Big(\Big[b_1 +\frac{bpr}{q}\Big] -\Big[d_1 + q + \lambda pr\Big]\Big), \tag*{($\tilde{P}$)\textsuperscript{$\lambda,m,n$}}\label{eq:pqr_fast} \\%%\textsuperscript{$\lambda,m,n$}}%\\
 r^\prime&=r\Big( \Big[\frac{\alpha-n}{\lambda(1+n)}\Big(r^{1+n}-c_0\Big)\Big]+\Big[d_1 + q + \lambda pr\Big]\Big)\nonumber \\
 &=:f^{\lambda,m,n}(p,q,r), \nonumber
\end{align}

\begin{align} 
 p^\prime &=0, \nonumber \\
 q^\prime &=0, \tag*{($\tilde{P}$)\textsuperscript{$\lambda,m,0$}}\label{eq:pqr_fast} \\%%\textsuperscript{$\lambda,m,n$}}%\\
 r^\prime&=r\Big( \Big[\frac{\alpha}{\lambda}\Big(r-c_0\Big)\Big]+\Big[d_1 + q + \lambda pr\Big]\Big)\nonumber \\
 &=:f^{\lambda,m,0}(p,q,r), \nonumber
\end{align}

\begin{lemma} \label{lem:normal_hyper}
 $G^{\lambda,m,0}$ is a normally hyperbolic invariant manifold with respect to the system $(\tilde{P})^{ \lambda,m,0}$.
\end{lemma}
\begin{proof}
We linearize the system $(\tilde{P})^{ \lambda,m,0}$ around $G^{\lambda,m,0}$, and show that $0$ is the eigenvalue with a multiplicity of exactly $2$. Let the perturbations of $p$, $q$, and $r$ be $P$, $Q$, and $R$, respectively. After discarding  terms higher than the first order, we obtain
\begin{align*}
 \begin{pmatrix} {P}^\prime\\ {Q}^\prime \\ {R}^\prime \end{pmatrix} =
 \begin{pmatrix} 0 & 0& 0\\ 0 & 0 & 0\\ \lambda (h^{\lambda,\alpha,0})^2 & h^{\lambda,\alpha,0} & ( \frac{\alpha}{ \lambda} + \lambda p )h^{\lambda,\alpha,0} \end{pmatrix} \begin{pmatrix} {P}\\ {Q} \\ {R} \end{pmatrix}.
\end{align*}
The coefficient matrix has eigenvalues of $0$ and $( \frac{m}{ \lambda} + \lambda p )h^{\lambda,m,0}$. Since we take $h^{\lambda,m,0}$ away from zero and $p \ge 0$, the latter eigenvalue is strictly greater than zero. Thus, $0$ is an eigenvalue with multiplicity $2$. %The claim that the graph $G^{\lambda,m,0}$ is a normally hyperbolic invariant manifold with respect to $(\tilde{*})^{ \lambda,m,0}$ has been shown.
\end{proof}



\begin{align*}
 \frac{\dpp}{p}&=\Big[\frac{1+\alpha}{1+n}\,\frac{1}{\lambda }\Big(r^{1+n}-c_0\Big)\Big] -\Big[d_1 + q + \lambda pr\Big]\\
 \frac{\dqq}{q}&=\Big[b_1 +\frac{bpr}{q}\Big] -\Big[d_1 + q + \lambda pr\Big]\\
 n\frac{\drr}{r}&=\Big[\frac{\alpha-n}{\lambda(1+n)}\Big(r^{1+n}-c_0\Big)\Big]+\Big[d_1 + q + \lambda pr\Big]
\end{align*}

\noindent where we denoted $(\cdot)^\prime = \frac{d}{d\tilde{\eta}}(\cdot)$. 
The right-hand side of the equation on $r$ is denoted by $f^{\lambda,m,n}(p,q,r)$. We specify the {\it critical manifold} $G^{\lambda,m,0}$ in the below that is a compact subset of $\{(p,q,r)\;|\; f^{\lambda,m,0}(p,q,r)=0\}$. The latter set consists of the equilibria of the system $\tilde{(P)}^{\lambda,m,0}$. 

In the region $r>0$, one solves the algebraic equation $f^{\lambda,m,0}(p,q,r)=0$,
\begin{equation}\label{eq:hn0}
 r=h(p,q,n=0) = \frac{ \frac{\alpha c_0}{\lambda} - d_1 -q }{ \frac{\alpha}{\lambda} + \lambda p}
\end{equation} 
from which we notice that its level lines are straight lines; after rearranging, 
\begin{equation}
 q + \lambda \underbar{r}p =  -\frac{\alpha}{\lambda} \big( \underbar{r} - c_0\big)-d_1, \quad \text{for  $h^{\lambda,m,0}(p,q)=\underbar{r}$.} \label{eq:level}
\end{equation}

\begin{equation}
 q + \lambda \underbar{r}p =  \frac{\alpha}{\lambda} \Big( \big(\frac{2}{D} + \frac{2}{D}\lambda\big) - \underbar{r} \Big)
\end{equation}

In view of \eqref{eq:level}, the contour lines in the $pq$-plane sweep out the first quadrant from the origin. See Figure \ref{fig:contour}. More precisely, the contour line passes the origin when $\underbar{r}=a^{ \lambda,m,0}$ at the same time as the level line passes the equilibrium $M_0^{ \lambda,m,0}$. As $\underbar{r}$ decreases, the contour line intersects the $p$ and $q$ axes and becomes steeper. When $\underbar{r}$ reaches $c^{ \lambda,m,0}$, the level line passes $M_1^{ \lambda,m,0}$. $\underbar{r}$ then further decreases until the level line immerses in the $r=0$ plane. Note that the inequality \eqref{eq:a4} implies $c^{ \lambda,m,0}>0$. 


We let $T$ be the triangle enclosed by the $p$-axis, $q$-axis and the one contour line of \eqref{eq:level} with $0<\underbar{r} < c^{\lambda,m,0}$. We let $D$ be an open set in the vicinity of the triangle $T$. The critical manifold for each $\lambda$ and $m$ is defined by 
\begin{equation}
 G^{\lambda,m,0} = \{(p,q,r) \in \bar{D} \;|\; r=h^{\lambda,m,0}(p,q)\}.
\end{equation}




% \subsubsection{Flow on the critical manifold : the case $m=1$}
% 
% The marginal case $m=1$ provides closer detail. 
% By substituting $h^{\lambda,1,0}(p,q)$ in place of $r$, the system is explicitly solved and
% %we can solve the system explicitly and the whole critical graph is completely characterized. 
% the general solution on the graph is a family of parabolae $p=kq^2$ and $r=h^{\lambda,1,0}(p,q)$. This includes the two extremes $p=0$ and $q=0$, where $k$ takes $0$ and $\infty$ respectively. See Figure \ref{fig:hn0m1}. We focus on discussing two points: 1) In an effort to apprehend the flow of the rest of cases, we remark a few features for this marginal case, which in turn persist under the perturbation; and 2) we report features that do not persist too. These features do not play any role in our study, but this bifurcation is described here for clarity.
% 
% We address the first point. Look at $M_0^{ \lambda,1,0}$ in Figure \ref{fig:hn0m1_b} surrounded by a family of parabolae in the neighborhood. Our interested direction $\vec{X}_{02}$ and the other $\vec{X}_{01}$ are annotated near $M_0^{ \lambda,1,0}$ by a dotted arrow. The family of parabolae is manifesting the fact that orbit curves meet $M_0^{ \lambda,1,0}$ tangentially to $\vec{X}_{01}$; one exception is the degenerate straight line that emanates in $\vec{X}_{02}$, which is depicted as the green one in Figure \ref{fig:hn0m1}, the target orbit. Another observation from the $pq$-plane is that the flow in the first quadrant far away from the origin is {\it inwards}. More precisely, as illustrated in Figure \ref{fig:hn0m1_b}, whenever $0<\underbar{r} < 1 = c^{\lambda,1,0}$ the flow on the contour line $\underbar{r} = h^{\lambda,1,0}$ is inwards. We make use of this observation in the proof of Section \ref{sec:proof_proof}.
% 
% Now, we describe the bifurcation of this marginal case. The crucial difference is that $M_1^{\lambda,1,0}$ is replaced by a line of equilibria $h^{\lambda,1,0}(p,q) = c^{\lambda,1,0}=1$, which is the red line in Figure \ref{fig:hn0m1}. As a result, each of the parabolae emanated from $M_0^{\lambda,1,0}$ lands at a point among these equilibria. $\vec{X}_{02}$ is immersed on $q=0$ plane distinctively from all other cases and the target orbit in particular lands at the $q$-intercept of the line of equilibria. To compare this observation to the statement of Theorem \ref{thm:1}, the target orbit does not connect $M_0^{ \lambda,1,0}$ to $M_1^{ \lambda,1,0}$ but to this $q$-intercept. This observation does not spoil our proof in Section \ref{sec:proof_proof} because we assert the persistence of the critical manifold not the target orbit.

\subsection{Characterization of the heteroclinic orbit : why and how}
\subsection{Asymptotic behavior of self-similar variables in $\xi$}
\section{A $k$-parameter family of shear banding solutions}
\subsection{Asymptotic behavior of field variables in $t$ and $x$}
\section{Existence via Geometric theory of singular perturbation}



\end{document}