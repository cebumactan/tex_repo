\documentclass[10pt, letterpaper]{article}
\usepackage[utf8]{inputenc}
\usepackage{amsmath}
\usepackage{mathtools}
\usepackage{mathrsfs} %\mathscr{}
%\usepackage{enumitem}
\usepackage{amsfonts}
\usepackage{amssymb}
\usepackage{amsthm}
\usepackage{indentfirst}
\usepackage{graphicx}
% Color packages
\usepackage{xcolor}
\usepackage{color, soul}
\def\red{\color{red}}
\def\blue{\color{blue}}

% Show labels
\usepackage[notref,notcite]{showkeys}

% % Korean Language
% \usepackage{kotex}

\graphicspath{{./images/}}

\usepackage{stackengine}
\stackMath
\allowdisplaybreaks

\newcommand{\aint}[0]{\displaystyle\stackinset{c}{0.015in}{c}{0.015in}{-\mkern4mu}{\displaystyle\int}}

\def\Orb{{\textsf{Orbit}}\,}
\def\S{{\textsf{S}}}
\def\E{{\mathcal{E}}}
\def\H{{\mathcal{H}}}
\def\K{{\mathcal{K}}}
\def\adm{{\mathcal{A}(I\times X)}}
\def\Lip{{\textsf{Lip}}}
\def\drr{{d\rho_0(x)d\rho_0(x')}}
\def\dr{{d\rho_0(x)}}
\def\bdot{{\boldsymbol{\cdot}}}
\def\kk{{L}}

\def\tr{\,\textrm{tr}\,}
\def\div{\,\textrm{div}\,}
\def\sgn{\,\textrm{sgn}\,}

\def\th{\tilde{h}}
\def\tx{\tilde{x}}
\def\tk{\tilde{\kappa}}
\def\bx{\bar{x}}
\def\bm{\bar{\mathbf{m}}}
% \def\K{\mathcal{K}}
% \def\E{\mathcal{E}}
% \def\Ha{\mathcal{H}}
\def\del{\partial}
\def\eps{\varepsilon}
\def\Td{\mathbb{T}^d}




\newtheorem{theorem}{Theorem}
\theoremstyle{definition}
\newtheorem{definition}[subsubsection]{Definition}
\newtheorem{example}[subsubsection]{Example}
\newtheorem{lemma}[subsubsection]{Lemma}
\newtheorem{corollary}[subsubsection]{Corollary}
\newtheorem{proposition}[subsubsection]{Proposition}
\theoremstyle{remark}
\newtheorem{remark}[subsubsection]{Remark}


\author{}
\date{}

\title{Hamiltonian structure and relative energy of Euler-Poisson model in Lagrangian coordinate system}

\begin{document}

\maketitle
\tableofcontents

\section{Introduction}

\section{Hamiltonian flow and relative entropy}



\section{Euler-Korteweg-Poisson system as a formal hamiltonian flow}

\section{Hamiltonian flow and relative entropy}
\subsection{Euler-Poisson system}
% 
% We consider the 1D Euler-Poisson system
% \begin{equation}
%     \begin{aligned}\frac{\partial \rho}{\partial t}+\operatorname{div}(\rho u) &=0 \\\rho\left(\frac{\partial u}{\partial t}+(u \cdot \nabla) u\right) &=-\nabla p(\rho)-\rho \nabla c \\-\Delta c+\beta c &=\rho-\langle\rho\rangle.
%     \end{aligned}
% \end{equation}
% In this system, the energy functional is
% \begin{equation}
%     \mathcal{E}=\int h(\rho)+\frac{1}{2} \rho c d x=\int h(\rho) d x+\frac{1}{2} \iint \rho(x) \mathcal{K}(x-y) \rho(y) d x d y,
% \end{equation}
% and $c(x)$ is such that 
% \begin{equation}
%     \begin{gathered}
%         c(x)=(\mathcal{K} * \rho)(x)=\int \mathcal{K}(x-y) \rho(y) d y, \quad \text { in particular } \\
%         -\Delta_{x} c+\beta c=\gamma \rho-\langle\rho\rangle, \quad \text { where }\langle\rho\rangle=\int \rho d x.
%         \end{gathered}
% \end{equation}
% The kernel is singular at $x = 0$ and does not belong to $C^1$. Especially when $\beta = 0$, $\mathcal{K}(x)=\gamma \Phi(x)+\frac{|x|^{2}}{2}$, where $\Phi(x)$ is the fundamental solution of the Laplace's equation, or $-\Delta_{x} \Phi(x)=\delta(x)$.
% 
% Now we write the Hamiltonian in the Lagrangian coordinate with respect to the motion field $\eta(x, t)$ and the momentum field ${m}(x,t)$.
% \begin{equation}
%     \begin{aligned}
%         H[\eta, {m}] &=\int \frac{|{m}|^{2}}{2 \rho_{0}(x)}+\tilde{h}(\rho) \rho_{0}(x) d x+\frac{1}{2} \iint \rho_{0}(x) \mathcal{K}(\eta(x)-\eta(x')) \rho_{0}(x') d x d x' \\
%         &=: H_{0}+H_{1}
%         \end{aligned}
% \end{equation}
% TO DO: The variational structure for the first integral.
% 
% We are left to investigate the variational structure of the last double integral. For the moment, we assume $\mathcal{K}(x)$ is $C^1$. If so,
% \begin{equation}
%     \begin{aligned}
%         \left\langle-\frac{\delta H_{1}}{\delta \eta}, \phi\right\rangle &=-\left.\frac{d}{d \epsilon}\right|_{\epsilon=0} \frac{1}{2} \iint \rho_{0}(x) \mathcal{K}(\eta(x)+\epsilon \phi(x)-\eta(x')-\epsilon \phi(x')) \rho_{0}(x') d x d x' \\
%         &=-\frac{1}{2} \iint \rho_{0}(x) D \mathcal{K}(\eta(x)-\eta(x')) \cdot(\phi(x)-\phi(x')) \rho_{0}(x') d x d x' \\
%         &=-\frac{1}{2} \int \rho_{0}(x) \phi(x) \cdot \int D \mathcal{K}(\eta(x)-\eta(x')) \rho_{0}(x') d x' d x \\
%         &+\frac{1}{2} \int \rho_{0}(x') \phi(x') \cdot \int D \mathcal{K}(\eta(x)-\eta(x')) \rho_{0}(x) d x d x'.
%         \end{aligned}
% \end{equation}
% Now, if we can assume
% \begin{equation}
%     \mathcal{K}(z) \text { is even function of } z \text { and } D \mathcal{K}(z) \text { is an odd function of } z,
% \end{equation}
% which is the case for the kernel of type $\mathcal{K}(z)=\mathcal{K}(|z|)$ then above expression is
% \begin{equation}
%     \left\langle-\frac{\delta H_{1}}{\delta \eta}, \phi\right\rangle=-\int \rho_{0}(x) \phi(x) \cdot \int D \mathcal{K}(\eta(x)-\eta(x')) \rho_{0}(x') d x' d x
% \end{equation}
% Set the Eulerian independent variable $y = \eta(x)$, or $dy = \det \frac{\partial \eta}{\partial x}dx = \frac{\rho_0(x)}{\rho(y)}dx $ so that in Eulerian coordinate 
% \begin{equation}
%     \begin{aligned}
%         \left\langle-\frac{\delta H_{1}}{\delta \eta}, \phi\right\rangle &=-\int \rho \phi(\eta^{-1}(y)) \cdot \int D \mathcal{K}(y-y') \rho(y) d y d y'=-\int \rho(y) \phi(\eta^{-1}(y)) \cdot(D \mathcal{K} * \rho)(y) d y \\
%         &=-\int \rho(y) \phi(\eta^{-1}(y)) \cdot \nabla_{y}(c) d y=\int\left[-\left(\nabla_{y} c\right) \rho_{0}(x)\right] \cdot \phi d x
%         \end{aligned}
% \end{equation}
% deriving the new contribution to the equation of the motion,
% \begin{equation}
%     \dot{\mathbf{v}}=-\frac{1}{\rho} \nabla_{y} p(\rho)-\nabla_{y} c.
% \end{equation}
% When the kernel is $\operatorname{not} C^{1}$, we should make sense the integral and the derivation accordingly. For the kernel $\mathcal{K}$ such that $\mathcal{K} \in L_{l o c}^{1}\left(\mathbb{R}^{d}\right)$ and $D \mathcal{K} \in L_{l o c}^{1}\left(\mathbb{R}^{d}\right)$, which is the case for the fundamental solution $\Phi(x)$, and for the density $\rho(x) \in L^{\infty}\left(\mathbb{R}^{d}\right)$, we can make senses the integral and the derivation for the $C_{c}^{1}\left(\mathbb{R}^{d}\right)$ test function $\phi$. We will elaborate this later.
% 
% \subsection{Formal relative entropy identity}
% Here, we proceed to formally calculate the relative entropy identity for two smooth solutions for the given Hamiltonian system
% \begin{equation} \label{Ham}
%     \dot{\eta}=\frac{\delta H}{\delta {m}}, \quad \dot{{m}}=-\frac{\delta H}{\delta \eta}
% \end{equation}
% with
% \begin{equation}
%     H[\eta, {m}]=\int_{\Omega_{0}} \frac{|{m}|^{2}}{\rho_{0}} d x+\mathcal{E}[\eta].
% \end{equation}
% 
% Let $(\eta, {m})$ and $(\bar{\eta}, \bar{{m}})$ be the two smooth solutions of \eqref{Ham}. We define the relative energy as
% \begin{equation} \label{relHam}
%     H(\eta, {m} \mid \bar{\eta}, \bar{{m}})=H(\eta, {m})-H(\bar{\eta}, \bar{{m}})-\left\langle\frac{\delta H}{\delta \eta}(\bar{\eta}, \bar{{m}}), \eta-\bar{\eta}\right\rangle-\left\langle\frac{\delta H}{\delta {m}}(\bar{\eta}, \bar{{m}}), {m}-\bar{{m}}\right\rangle.
% \end{equation}
% We calculate the time derivative of relative energy. For simplicity we denote $H=H(\eta, {m})$ and $\bar{H} = H(\bar{\eta}, \bar{{m}})$.
% \begin{equation*}
%     \begin{aligned}
%         \frac{d}{dt}H[\eta, m | \bar{\eta}, \bar{m}] &= \frac{\delta K}{\delta {m}}\dot{{m}} - \frac{\delta \bar{K}}{\delta {m}}\dot{\bar{{m}}} - \frac{\delta^2 \bar{K}}{\delta {m}^2} \dot{\bar{{m}}}({m}-\bar{m}) - \frac{\delta \bar{K}}{\delta {m}}(\dot{{m}} - \dot{\bar{{m}}}) \\
%         & \quad + \frac{\delta \mathcal{E}}{\delta \eta}\dot\eta - \frac{\delta \mathcal{\bar{E}}}{\delta \eta}\dot{\bar{\eta}} - \frac{\delta ^2 \bar{\mathcal{E}}}{\delta \eta ^2}\dot{\bar{\eta}}(\eta - \bar{\eta}) - \frac{\delta \bar{\mathcal{E}}}{\delta \eta}(\dot{\eta} - \dot{\bar{\eta}}) \\
%         &=\dot{\bar{{m}}}\left(\frac{\delta K}{\delta m} - \frac{\delta \bar{K}}{\delta m} - \frac{\delta^2 \bar{K}}{\delta m^2}({m} - \bar{{m}})\right) \\
%         & \quad \quad + \frac{\delta K}{\delta m} \dot{{m}} - \frac{\delta {K}}{\delta m}\dot{\bar{{m}}}  - \frac{\delta \bar{K}}{\delta {m}}\dot{{m}}+ \frac{\delta\bar{K}}{\delta {m}}\dot{\bar{{m}}} \\
%         & \quad +\dot{\bar{\mathbf{\eta}}}\left(\frac{\delta \mathcal{E}}{\delta \eta} - \frac{\delta \bar{\mathcal{E}}}{\delta \eta} - \frac{\delta^2 \bar{\mathcal{E}}}{\delta \eta^2}(\mathbf{\eta} - \bar{\mathbf{\eta}})\right) \\
%         & \quad \quad + \frac{\delta \mathcal{E}}{\delta \eta} \dot{\mathbf{\eta}} - \frac{\delta {\mathcal{E}}}{\delta \eta}\dot{\bar{\mathbf{\eta}}}  - \frac{\delta \bar{\mathcal{E}}}{\delta \mathbf{\eta}}\dot{\mathbf{\eta}}+ \frac{\delta\bar{\mathcal{E}}}{\delta \mathbf{\eta}}\dot{\bar{\mathbf{\eta}}}.
%     \end{aligned}
% \end{equation*}
% From the symmetry of Hamiltonian structure,
% \begin{equation*}
%     \frac{\delta K}{\delta m} \dot{{m}} - \frac{\delta {K}}{\delta m}\dot{\bar{{m}}}  - \frac{\delta \bar{K}}{\delta {m}}\dot{{m}}+ \frac{\delta\bar{K}}{\delta {m}}\dot{\bar{{m}}} + \frac{\delta \mathcal{E}}{\delta \eta} \dot{\mathbf{\eta}} - \frac{\delta {\mathcal{E}}}{\delta \eta}\dot{\bar{\mathbf{\eta}}}  - \frac{\delta \bar{\mathcal{E}}}{\delta \mathbf{\eta}}\dot{\mathbf{\eta}}+ \frac{\delta\bar{\mathcal{E}}}{\delta \mathbf{\eta}}\dot{\bar{\mathbf{\eta}}} = 0.
% \end{equation*}
% Hence
% \begin{equation*}
%     \begin{aligned}
%         \frac{d}{dt}H[\eta, m | \bar{\eta}, \bar{m}] &=
%         \dot{\bar{{m}}}\left(\frac{\delta K}{\delta m} - \frac{\delta \bar{K}}{\delta m} - \frac{\delta^2 \bar{K}}{\delta m^2}({m} - \bar{{m}})\right) \\
%         & \quad +\dot{\bar{\mathbf{\eta}}}\left(\frac{\delta \mathcal{E}}{\delta \eta} - \frac{\delta \bar{\mathcal{E}}}{\delta \eta} - \frac{\delta^2 \bar{\mathcal{E}}}{\delta \eta^2}(\mathbf{\eta} - \bar{\mathbf{\eta}})\right) \\
%         &= \dot{\bar{\mathbf{\eta}}}\left(\frac{\delta \mathcal{E}}{\delta \eta} - \frac{\delta \bar{\mathcal{E}}}{\delta \eta} - \frac{\delta^2 \bar{\mathcal{E}}}{\delta \eta^2}(\mathbf{\eta} - \bar{\mathbf{\eta}})\right),
%     \end{aligned}
% \end{equation*}
% exploiting the following facts:
% \begin{itemize}
%     \item $\frac{\delta^{2} \bar{H}}{\delta \eta \delta {m}}=0$,
%     \item $\frac{\delta H}{\delta {m}}=\frac{{m}}{\rho_{0}(x)}=\mathbf{v}, \quad \frac{\delta^{2} H}{\delta {m}^{2}}=\frac{1}{\rho_{0}(x)}$.
% \end{itemize}
% The identity reduces to 
% \begin{equation} \label{dotrel}
%     \frac{d}{d t} H[\eta, {m} \mid \bar{\eta}, \bar{{m}}]=\left\langle\bar{\mathbf{v}}, \frac{\delta H}{\delta \eta}-\frac{\delta \bar{H}}{\delta \eta}-\frac{\delta^{2} \bar{H}}{\delta \eta^{2}}(\eta-\bar{\eta})\right\rangle .
% \end{equation}
% Interpretation is such that $\bar{\mathbf{v}}$ and $(\eta-\bar{\eta})$ are tested over functionals in the expression.
% \begin{remark}
%     The above formula can be interpreted as the relative kinetic energy has no contribution to the evolution of the relative Hamiltonian.
% \end{remark}
% 
% \subsection{Relative energy for the pressureless Euler-Poisson system in 1D}
% Here, we are standing on the previous work of Carrillo, Choi and Zatorska \cite{Carrillo2016}. As was explained in their paper, this problem goes well with the Lagrangian description in the following sense:
% \begin{itemize}
%     \item The initial data is compactly supported in $\Omega(t=0)$ and is so (or turns out to be so) at all $t>0$. In Eulerian coordinate system we may treat this problem either
%     1) as a free boundary problem with evolving domain $\Omega(t)$, or 2) as a problem in whole domain but with the vacuum region where $\rho=0$, either of which is not easy to handle.
%     \item Due to pressureless nature we use the relative energy formula \eqref{relHam} in Lagrangian description, and \eqref{relHam} has some advantages in manipulating terms.
% \end{itemize}
% As was in the free energy density $h(\rho)$, the convexity of the kernel $\mathcal{K}(x)$ in the energy formula is crucial in developing the relative energy theory.
% 
% We divide the kernel into two parts, $\mathcal{K}(x)=\gamma \Phi(x)+\frac{|x|^{2}}{2}, \gamma>0$, where we set $\gamma \Phi(x)=-|x|$ for $1 d,-\log (|x|)$ for $2 d,|x|^{2-d}$ for higher dimensions. The latter term is convex whereas the first term is neither convex nor concave, except for the $1 d$ case, where $-|x|$ is a concave function.
% 
% \subsubsection*{Relative energy formula}
% In the kernel $\mathcal{K}(x)=-|x|+\frac{|x|^{2}}{2}$ of $1 d$ Euler-Poisson, it is neither convex nor concave. The convex contribution in our sign convention would explain the attracting mechanism between particles and the concave contribution would explain the repelling mechanism. In general, we may consider $-\frac{|x|^{p}}{p}+\frac{|x|^{q}}{q}, p, q>0 .$ Due to the presence of both convex and concave contributions, the relative energy would not have a definite sign in general.
% However, the linear one $-|x|$, together with the crucial one dimensional feature that
% \begin{equation} \label{obser}
%     \operatorname{det} F=F=\frac{\partial x}{\partial X}>0 \Longrightarrow \operatorname{sgn}(X-Y)=\operatorname{sgn}(x(X, t)-x(Y, t)), \quad \forall t>0, \forall(X, Y)
% \end{equation} 
% turns out not to contribute any in the relative energy expression. Thus the relative energy happens to have the definite sign.
% 
% Now we proceed to calculate the relative energy. Let $K$ be the kinetic energy and $\mathcal{E}$ be the rest of the energy so that $H=K+\mathcal{E}$.
% 
% \subsubsection*{Relative kinetic energy}
% \begin{equation*}
%     \begin{aligned}
%         K(\eta, {m} \mid \bar{\eta}, \bar{{m}}) &=K(\eta, {m})-K(\bar{\eta}, \bar{{m}})-\left\langle\frac{\delta K}{\delta \eta}(\bar{\eta}, \bar{{m}}), \eta-\bar{\eta}\right\rangle-\left\langle\frac{\delta K}{\delta {m}}(\bar{\eta}, \bar{{m}}), {m}-\bar{{m}}\right\rangle \\
%         &=\int \frac{1}{2 \rho_{0}(x)}\left(|{m}|^{2}-|\bar{{m}}|^{2}-2 \bar{{m}} \cdot({m}-\bar{{m}})\right) d x=\int \frac{|{m}-\bar{{m}}|^{2}}{2 \rho_{0}(x)} d x
%         \end{aligned}
% \end{equation*}
% 
% \subsubsection*{Relative potential energy}
% \begin{equation*}
%     \mathcal{E}(\eta, {m} \mid \bar{\eta}, \bar{{m}})=\mathcal{E}(\eta, {m})-\mathcal{E}(\bar{\eta}, \bar{{m}})-\left\langle\frac{\delta \mathcal{E}}{\delta \eta}(\bar{\eta}, \bar{{m}}), \eta-\bar{\eta}\right\rangle-\left\langle\frac{\delta \mathcal{E}}{\delta {m}}(\bar{\eta}, \bar{{m}}), {m}-\bar{{m}}\right\rangle
% \end{equation*}
% Let us divide $\mathcal{K}(x)=\mathcal{K}_{1}(x)+\mathcal{K}_{2}(x)$, where $\mathcal{K}_{1}(x)=-|x|$ and $\mathcal{K}_{2}(x)=\frac{|x|^{2}}{2}$. The contribution from the first part is
% \begin{equation*}
%     \begin{aligned}
%         &\frac{1}{2} \iint \rho_{0}(x)[-|\eta(x)-\eta(x')|+|\bar{\eta}(x)-\bar{\eta}(x')| \\
%         &\quad \quad +\operatorname{sgn}(\bar{\eta}(x)-\bar{\eta}(x'))(\eta(x)-\eta(x')-\bar{\eta}(x)+\bar{\eta}(x'))] \rho_{0}(x') d x' d x \\
%         &=\frac{1}{2} \iint \rho_{0}(x)(\eta(x)-\eta(x'))\left(\operatorname{sgn}(\bar{\eta}(x)-\bar{\eta}(x'))-\operatorname{sgn}(\eta(x)-\eta(x))) \rho_{0}(x') d x' d x\right. \\
%         &=0
%         \end{aligned}
% \end{equation*}
% because of the \eqref{obser}. The contribution from the attractive part $\mathcal{K}_2(x)$ is 
% \begin{equation*}
%     \begin{aligned}
%         &\frac{1}{2} \iint \rho_{0}(x)\left[\frac{|\eta(x)-\eta(x')|^{2}}{2}-\frac{|\bar{\eta}(x)-\bar{\eta}(x')|^{2}}{2}\right. \\
%         &-(\bar{\eta}(x)-\bar{\eta}(x'))(\eta(x)-\eta(x')-\bar{\eta}(x)+\bar{\eta}(x'))] \rho_{0}(x') d x' d x \\
%         &=\frac{1}{4} \iint \rho_{0}(x)(\eta(x)-\eta(x')-\bar{\eta}(x)+\bar{\eta}(x'))^{2} \rho_{0}(x') d x' d x
%         \end{aligned}
% \end{equation*}
% Hence
% \begin{equation} \label{eq:relH}
%     H(\eta, {m} \mid \bar{\eta}, \bar{{m}})=\int \frac{|{m}-\bar{{m}}|^{2}}{2 \rho_{0}(x)} d x + \frac{1}{4} \iint \rho_{0}(x)(\eta(x)-\eta(x')-\bar{\eta}(x)+\bar{\eta}(x'))^{2} \rho_{0}(x') d x' d x
% \end{equation}
% 
% \subsubsection*{Time evolution of the relative energy}
% We have two observations,
% \begin{enumerate}
%     \item The time derivative of the relative energy is solely determined by the first and second variation of the Hamiltonian with respect to the $\eta(x, t)$. This implies, in EulerPoisson $1 d$, that the variation of the relative kinetic energy does not contribute to the time derivative. This is due to the fact that the kinetic energy does not depend on $\eta$ (and is quadratic in ${m}$) and the potential energy does not depend on $m$.
%     \item In addition, the concave part of the kernel $-|x|$ does not play any role in the relative energy, so the time derivative of the relative energy is solely determined by the variations of the part of the potential energy that is associated only to the convex kernel $\frac{x^{2}}{2}$. This kernel is smooth enough.
% \end{enumerate}
% From above observations, the time derivative of the relative energy is calculated, from the formula \eqref{dotrel},
% \begin{equation}\small
%     \begin{aligned}
%         &\frac{d}{d t} H(x, {m} \mid \bar{x}, \bar{{m}})=\frac{1}{2} \iint \rho_{0}(X)(x(X)-x(Y))(\bar{v}(X)-\bar{v}(Y)) \rho_{0}(Y) d X d Y \\
%         &\quad-\frac{1}{2} \iint \rho_{0}(X)(\bar{x}(X)-\bar{x}(Y))(\bar{v}(X)-\bar{v}(Y)) \rho_{0}(Y) d X d Y \\
%         &-\lim _{\epsilon \rightarrow 0} \frac{1}{\epsilon} \frac{1}{2} \iint \rho_{0}(X)[(\bar{x}(X)-\bar{x}(Y)+\epsilon(x(X)-x(Y)-\bar{x}(X)+\bar{x}(Y)))-(\bar{x}(X)-\bar{x}(Y))](\bar{v}(X)-\bar{v}(Y)) \rho_{0}(Y) d X d Y \\
%     &=0 .
%         \end{aligned}
% \end{equation}
% In conclusion, if $(x, {m})$ and $(\bar{x}, \bar{{m}})$ are two smooth solutions, then the relative energy is conserved. If $(x, {m})$ is a weak solution with decreasing energy and $(\bar{x}, \bar{{m}})$ is a smooth solution, then the relative energy decreases as time proceeds.
% 
% 

\section{Euler-Poisson system in $1$ space dimension} \label{sec:EP}

Let $X=Y=\mathbb{R}$, which respectively serves as the referential and spatial coordinate system. We define a class of regular position maps on $X$ taking values in $Y$, 
\begin{align*}
 \hat{\textsf{S}}_0(X)&:= \left\{\eta : X \rightarrow Y ~|~ 
\text{$\eta$ is injective, surjective, increasing,} \right.\\
&~~~~~~~~~~~~~~~~~~~~~~\left.\text{$\eta$ and $\eta^{-1}$ is Lipschitz.}\:\right\} \\
\intertext{and a class of motion maps over an open interval $I$}
 \hat{{\Orb}_0}(I\times X)&:= \left\{\eta : I\times X \rightarrow Y ~|~  \text{for all $t \in I$, \:$\eta(t,\cdot) \in \hat{\textsf{S}}_0(X)$} \: \right\}.
\end{align*}
For $\hat{\textsf{S}}_0(X)$ we currently do not consider any topology. One can verify that $\hat{\textsf{S}}_0(X)$ is a cone. 

Fix a measure $\rho_0$ on $X$ that is the mass distribution in the referential frame. %While $\rho_0$ with a natural referential frame might be quite wild, in this paper we will not be reluctant on imposing suitable assumptions on $\rho_0$ whenever we want.  
Suppose $\E : \hat{\textsf{S}}_0(X) \rightarrow \mathbb{R}$ is a given interaction energy functional. It is typical that $\E[\eta]$ is an integral with respect to the measure $\rho_0$. Given those, we say a map $\eta \in \hat{\textsf{S}}_0(X)$ is of finite interaction energy if $\E[\eta]$ is bounded. If a motion $\eta$ is such that for each $t\in I$, $\partial_t \eta(t,\cdot) \in L^1_{\rho_0}(X) \cap L^2_{\rho_0}(X)$, then we say $\eta$ is of finite momentum and kinetic energy. 

%For a motion $\eta \in \mathcal{A}(I\times X)$ and for $t\in I$ we let $\rho_0\partial_t\eta(t,\cdot)=: {m}$ the momentum. We then define 
\subsection{Energy of Euler-Poisson problem}
We take $\rho_0$ to be the lebesgue measure restricted to an open subset of $X$. 
% In this section, we $\rho_0$ (as a density) is in $L^\infty(X)$ and has a positive lower bound on its support.
In this section, we consider an energy functional on $\hat{\textsf{S}}_0(X)$ of the form
\begin{equation}
 \E = \E_{r} + \E_{a} + \E_p. \label{E}
\end{equation}
$\E_r$ and $\E_a$ are non-local interaction energies of double integral form. For some fixed $\eta_* \in S_0[X]$, 
\begin{equation} \label{Era}
    \begin{aligned}
        \E_r[\eta] &=\frac{1}{2} \iint \Big(\mathcal{K}_r\big(\eta(x)-\eta(x')\big) -  {\mathcal{K}_r\big(\eta_*(x)-\eta_*(x')\big)}\Big) d\rho_{0}(x)d\rho_0(x'), \\ 
        \E_a[\eta] &=\frac{1}{2} \iint \Big(\mathcal{K}_a\big(\eta(x)-\eta(x')\big) -  {\mathcal{K}_a\big(\eta_*(x)-\eta_*(x')\big)}\Big) d\rho_{0}(x)d\rho_0(x'). 
        \end{aligned}
\end{equation}
Kernels $K_r$ and $K_a$ are of power laws,
\begin{equation} \label{kernellaw}
    {K}_r(y) = -\frac{|y|^p}{p} \quad(-1< p \le 1, p\ne 0) \quad \text{ and }\quad {K}_a(y) = \frac{|y|^q}{q} \quad(q \ge 1).
\end{equation}
They are chosen so that ${K}_r$ and ${K}_a$ are radially symmetric, convex in the radial variable, ${K}_r$ is radially decreasing, and ${K}_a$ is radially increasing. We explain in the following section the reason for the choices and also for the subtraction by $K_r\big(\eta_*(x)-\eta_*(x')\big)$ and $K_a\big(\eta_*(x)-\eta_*(x')\big)$ repectively in the integrands. %We could typically choose $\eta_*(x)=x$ 

% {\red To have the finite interaction energy especially for the power type kernels in \eqref{kernellaw}, we find it convenient to formulate a sufficient condition for $\eta \in S_0(X)$ as in the below.}
% 
% prescribed asymptotic behaviors
% We consider an inequaility on $\eta \in \S_0(X)$
% \begin{equation} \label{suff} \tag{S}
%  J_r(\eta):=\iint_{X \times X} |x-x'|^{r-1}\Big| \big(\eta(x)-\eta(x')\big) -\big(\eta_*(x)-\eta_*(x')\big)\Big| d\rho_0(x)d\rho_0(x') < \infty
% \end{equation}
% for a fixed $r>-1$, and impose conditions
% \begin{equation}
%  J_p(\eta) < M_1, \quad J_q(\eta) < M_2 \label{M1M2} 
% \end{equation}
% 
% We additionally assume that   
% \begin{equation} \label{M}
%  \int_X \big|\big(\eta - \eta_{*}\big)_{x} \big|\:\phi \:dx \le M\|\phi\|_\infty, \quad \text{for all $\phi \in C_b(X)$.} 
% \end{equation}
% 
% 
Nextly, we define energy that gives rise to the pressure. We adopt one in isentropic ideal gas. For $\gamma>1$, we define $\E_p$ as
\begin{align} \label{Ep}
 \E_{p}[\eta] = \int_X \left(\left(\frac{\eta_x}{\rho_0}\right)^{1-\gamma}  - \left(\frac{\eta_{*x}}{\rho_0}\right)^{1-\gamma}\right)\: d\rho_0(x).%for a given $\psi(\cdot)$ convex.}
\end{align}
For $\eta\in \hat{\textsf{S}}_0(X)$, we associate a $\rho_0$-a.e. well-defined function $\tau = \frac{\eta_x}{\rho_0}\ge 0$, which we call the specific volume function.
% Subtraction by $\left(\frac{\rho_0}{\eta_{*x}}\right)^{\gamma-1}$ is again explained in the following section. 

% For our study where $\eta$ is the state variable, %and the specific volume $\frac{\eta_x}{\rho_0}$, instead of the density $\rho = \frac{\rho_0}{\eta_x}\circ \eta^{-1}$
% $\eta$ being a map with finite $\E_p$ has nothing to do with the growth of $|\eta_x|$. We additionally assume that   
% \begin{equation} \label{M}
%  \int_X \big|\big(\eta - \eta_{*}\big)_{x} \big|\:\phi \:dx \le M\|\phi\|_\infty, \quad \text{for all $\phi \in C_b(X)$.} 
% \end{equation}

Finally, we define the class of admissible motions $\mathcal{A}(I\times X)$ as follows. A position map $\eta \in \hat{\textsf{S}}_0(X)$ is admissible if 
\begin{enumerate}
 \item $\left|K_r\big(\eta(x) -\eta(x')\big)-K_r\big(\eta_*(x) -\eta_*(x')\big)\right|$\\ and $\left|K_a\big(\eta(x) -\eta(x')\big)-K_a\big(\eta_*(x) -\eta_*(x')\big)\right|$ are integrable with respect to the measure $\rho_0\times \rho_0$,
 \item $\left|\left(\frac{\eta_x}{\rho_0}\right)^{1-\gamma}  - \left(\frac{\eta_{*x}}{\rho_0}\right)^{1-\gamma}\right|$ is integralble with repect to the measure $\rho_0$,
%  \item \eqref{M} holds.
\end{enumerate}
We define $\mathcal{A}(I\times X)$ to be a collection of $\eta \in \hat{\Orb}_0(I\times X)$ such that for all $t\in I$, $\dot\eta(t,\cdot) \in L^1_{\rho_0}(X)\cap L^2_{\rho_0}(X)$, and $\eta(t,\cdot)$ is admissible.  %, $J_p[\eta(t,\cdot)] < \infty$, and $J_q[\eta(t,\cdot)] < \infty$. %If we let the hamiltonian
% $$\H[\eta, {m}] = \K[{m}] + \E[\eta],$$
% where ${m}=\rho_0\dot\eta$, and $\displaystyle\K[{m}] = \int_X \frac{|{m}|^2}{2\rho_0} \:dx$, then for given an admissible motion, the hamiltonian is finite for all $t\in I$.

\subsubsection{Non-local energy and convexity of kernel}
In this section, we give further explanations on the form of our interaction energy. We begin discussions on the non-local energies $\E_r$ and $\E_a$ and kernels $K_r$ and $K_a$. In our study we will only consider a kernel that is radially symmetric. If this interaction energy is employed in a Gradient flow, then ${K}$ that is radially decreasing is called the repulsive kernel, and ${K}$ that is radially increasing is called the attractive kernel. Although we are studying the Hamiltonian flow, we adopt this terminology. 

For the relative energies we will calculate in Section \ref{sec:rel} to be nonnegative, convexity of kernels is required. In the below, we reveal the non-necessity of kernels being convex, but the necessity of kernels being convex in radial variable. This is due to the monotonicity of $\eta\in \hat{\textsf{S}}_0(X)$.

Under the assumptions that the integrand in $\E_r$ is in $L^1_{\rho_0\otimes\rho_0}(X^2)$ and that $\rho_0$ is absolutely continuous to the Lebesgue measure, we note that the line $\left\{(x,x') ~|~ x=x'\right\}$ is $\rho_0\otimes\rho_0$-negligible, and we can write the integral
\begin{align*}
 &\frac{1}{2} \iint_{x\ge x'} \Big({K}_r\big(\eta(x)-\eta(x')\big) -  {{K}_r\big(\eta_*(x)-\eta_*(x')\big)}\Big) d\rho_{0}(x)d\rho_0(x')\\
 + \: &\frac{1}{2} \iint_{x\le x'} \Big({K}_r\big(\eta(x)-\eta(x')\big) -  {{K}_r\big(\eta_*(x)-\eta_*(x')\big)}\Big) d\rho_{0}(x)d\rho_0(x').
\end{align*}
Note that $\eta(x)-\eta(x')\ge 0$ in the first integral  and $\eta(x) - \eta(x')\le 0$ in the second integral due to the monotonicity of $\eta\in \hat{\textsf{S}}_0(X)$. But the second integral can be written with $z=x'$, $z'=x$
\begin{align*}
%  &\frac{1}{2} \iint_{x<x'} \Big({K}_r\big(\eta(x)-\eta(x')\big) -  {{K}_r\big(\eta_*(x)-\eta_*(x')\big)}\Big) d\rho_{0}(x)d\rho_0(x') \\
%  = \: &\frac{1}{2} \iint_{z>z'} \Big({K}_r\big(\eta(z')-\eta(z)\big) -  {{K}_r\big(\eta_*(z')-\eta_*(z)\big)}\Big) d\rho_{0}(z)d\rho_0(z') \\
&\frac{1}{2} \iint_{z\ge z'} \Big({K}_r\big(\eta(z)-\eta(z')\big) -  {{K}_r\big(\eta_*(z)-\eta_*(z')\big)}\Big) d\rho_{0}(z)d\rho_0(z')
\end{align*}
because the kernel is radially symmetric. Therefore the energy can be calculated only with contribution
$$\iint_{x\ge x'} \Big({K}_r\big(\eta(x)-\eta(x')\big) -  {{K}_r\big(\eta_*(x)-\eta_*(x')\big)}\Big) d\rho_{0}(x)d\rho_0(x')$$
where $\eta(x)-\eta(x')\ge 0$. In this way, convexity of $K$ in radial variable only matters. This calculation by no means generalizes to higher dimensions. 

In conclusion, the repulsive kernels including
$$ |y|^{p} \quad (p<0), \quad -\log(y), \quad -|y|^p \quad (0\le p \le 1)$$ 
and the attractive kernels including
$$ |y|^q \quad (q\ge 1)$$
can be adopted. We remark that one can consider wider range of kernels for the repulsive ones than the attractive ones. Importantly, the attractive singular kernel of type $-|y|^q \quad (q<0)$ cannot be included since it is radially concave. In McCann, where the displacement convexity is introduced for various energies in multi-dimensions, and the competition of pressure and attractive interaction is studied, the same restriction is found for the attractive kernel. If $p\le -1$, then $|y|^{-p}$ is not locally integrable and we do not consider this case make much senses.

The reason for the subtraction by $K\big(\eta_*(x) - \eta_*(x'))$ is to make integral finite, i.e., $|\eta(x)-\eta(x')|^q$ can be large if $|x-x'|$ is large. We remark that this subtraction does not affect in calculations of 
$$ \frac{d}{ds} \E(\eta +s\phi) \Big|_{s=0}$$
for some suitable test function $\phi$. Also, one notices that in a bounded region of $X \times X$,
$|\eta(x) - \eta(x')|^p$ and $|\eta(x) - \eta(x')|^q$ are always integrable, so the $\eta$ being of finite energy has nothing to do with the behaviors of $\eta_*$ in a bounded region. Thus being finite energy affects the behavior of $\eta(x)$ as $|x| \rightarrow \infty$. 

% {\blue Some explanations ...
% Since $\eta$ and $\eta_*$ are both Lipschitz, and $\displaystyle\iint_{|x|,|x'|\le c} |x-x'|^p \: d\rho_0(x)d\rho_0(x')$ is always finite,  \eqref{suff} does not impose any restriction for the behaviors of $(\eta-\eta_*)(z)$ for $|z|$ finite. \eqref{suff} forces the behaviors of $(\eta-\eta_*)(z)$ as $|z| \rightarrow \infty$ so that the integral is finite.
% 
% 
% % 
% % For our study, we fix the energy as follows. We consider a power law in \eqref{Ep} 
% % \begin{equation}
% % \psi(\tau) = \tau^{1-\gamma}, \quad \text{for $\gamma>1$,} \label{psilaw}
% % \end{equation}
% % and kernels 
% % We fix 
% % \begin{equation} \label{energy}
% %  \E = \E_p + \E_r + \E_a
% % \end{equation}
% % with $\psi$, $K_r$, and $K_a$ in \eqref{psilaw}-\eqref{kernellaw}.
% % 
% % 
% % 
% % 
% % a fixed kernel $\mathcal{K} : Y \rightarrow \mathbb{R}$ and a fixed $\eta_* \in \textsf{S}(X)$
% % \begin{equation}
% % \frac{1}{2} \iint \Big(\mathcal{K}\big(\eta(x)-\eta(x')\big) -  {\mathcal{K}\big(\eta_*(x)-\eta_*(x')\big)}\Big) d\rho_{0}(x)d\rho_0(x'). \label{dblint}
% % \end{equation}
% 
% 
% 
% Suppose $\eta$ and $\bar\eta$ are in $\S[X]$, or $\phi:=\eta - \bar\eta \in C^{0,1}(X)$ and it satisfies
% \begin{equation}
%  I_p[\phi]=\iint |x-x'|^{p-1}~| \phi(x) - \phi(x')| \:d\rho_0(x)d\rho_0(x') < \infty. \label{phicond}
% \end{equation}
% 
% 
% In the below, we will define a suitable regularity class of a motion $\eta \in \Orb(I\times X)$ where the relative interaction is finitely defined. When this is the case, the positivity of the corresponding relative interaction energy matters, and the positivity is implied by the convexity of the kernel $\mathcal{K}: Y \rightarrow \mathbb{R}$. We remark that we can in fact enlarge the class of kernels to the one that is convex in the radial variable, which is due to the strict monotonicity of $\eta\in \textsf{S}(X)$.
% 
% Under the assumptions that the integrand in \eqref{dblint} is in $L^1_{\rho_0\otimes\rho_0}(X^2)$ and that $\rho_0$ is absolutely continuous to the Lebesgue measure, we note that the line $\left\{(x,x') ~|~ x=x'\right\}$ is $\rho_0\otimes\rho_0$-negligible, and we can write the integral
% \begin{align*}
%  &\frac{1}{2} \iint_{x>x'} \Big(\mathcal{K}\big(\eta(x)-\eta(x')\big) -  {\mathcal{K}\big(\eta_*(x)-\eta_*(x')\big)}\Big) d\rho_{0}(x)d\rho_0(x')\\
%  + \: &\frac{1}{2} \iint_{x<x'} \Big(\mathcal{K}\big(\eta(x)-\eta(x')\big) -  {\mathcal{K}\big(\eta_*(x)-\eta_*(x')\big)}\Big) d\rho_{0}(x)d\rho_0(x').
% \end{align*}
% Note that $\eta(x)-\eta(x')>0$ in the first integral  and $\eta(x) - \eta(x')<0$ in the second integral that is due to the strict monotonicity of $\eta\in \textsf{S}(X)$. But the second integral can be written with $z=x'$, $z'=x$
% \begin{align*}
% %  &\frac{1}{2} \iint_{x<x'} \Big(\mathcal{K}\big(\eta(x)-\eta(x')\big) -  {\mathcal{K}\big(\eta_*(x)-\eta_*(x')\big)}\Big) d\rho_{0}(x)d\rho_0(x') \\
% %  = \: &\frac{1}{2} \iint_{z>z'} \Big(\mathcal{K}\big(\eta(z')-\eta(z)\big) -  {\mathcal{K}\big(\eta_*(z')-\eta_*(z)\big)}\Big) d\rho_{0}(z)d\rho_0(z') \\
% &\frac{1}{2} \iint_{z>z'} \Big(\mathcal{K}\big(\eta(z)-\eta(z')\big) -  {\mathcal{K}\big(\eta_*(z)-\eta_*(z')\big)}\Big) d\rho_{0}(z)d\rho_0(z')
% \end{align*}
% because the kernel is radially symmetric. Therefore the energy can be calculated only with contribution
% $$\iint_{x>x'} \Big(\mathcal{K}\big(\eta(x)-\eta(x')\big) -  {\mathcal{K}\big(\eta_*(x)-\eta_*(x')\big)}\Big) d\rho_{0}(x)d\rho_0(x')$$
% where $\eta(x)-\eta(x')>0$. In this way, convexity of $K$ in radial variable only matters. This calculation by no means generalizes to higher dimensions. 
% 
% In conclusion, the repulsive kernels including
% $$ |y|^{p} \quad (p<0), \quad -\log(y), \quad -|y|^p \quad (0\le p \le 1)$$ 
% can be studied, and the attractive kernels including
% $$ |y|^q \quad (q\ge 1)$$
% can be studied. We remark that one can consider wider range of kernels for the repulsive ones than the attractive ones. Importantly, the attractive singular kernel of type $-|y|^q \quad (q<0)$ cannot be included since it is radially concave. In McCann, where the displacement convexity is introduced for various energies in multi-dimensions, and the competition of pressure and attractive interaction is studied, the same restriction is found for the attractive kernel.
% 
% To have the finite interaction energy especially for the power type kernel, we find it convenient to formulate a sufficient condition for $\eta \in S(X)$ as in the below. Here $\rho_0$ is absolutely continuous to the Lebesgue measure, and its density is in $L^\infty(X)$. We consider $\eta \in S(X)$ satisfying
% \begin{equation} \label{suff} \tag{S}
%  \iint |x-x'|^{p-1}\Big| \big(\eta(x)-\eta(x')\big) -\big(\eta_*(x)-\eta_*(x')\big)\Big| d\rho_0(x)d\rho_0(x') < \infty
% \end{equation}
% for a fixed $p >-1$. Since $\eta$ and $\eta_*$ are both Lipschitz, and $\displaystyle\iint_{|x|,|x'|\le c} |x-x'|^p \: d\rho_0(x)d\rho_0(x')$ is always finite,  \eqref{suff} does not impose any restriction for the behaviors of $(\eta-\eta_*)(z)$ for $|z|$ finite. \eqref{suff} forces the behaviors of $(\eta-\eta_*)(z)$ as $|z| \rightarrow \infty$ so that the integral is finite.
% 
% One may consider instead a condition $\displaystyle \iint |x-x'|^{p-1}|\eta(x)-\eta(x')| d\rho_0(x)d\rho_0(x') < \infty$, which implies, by bi-Lipschitz property, $\displaystyle \iint |x-x'|^{p}d\rho_0(x)d\rho_0(x') < \infty$. Thus the condition is rather imposed on the measure $\rho_0$ than on $\eta$, and we see that $\rho_0\equiv 1$ cannot fulfill the condition. With or without the terms involving $\eta_*$, the variational caulation
% $$ \frac{d}{ds} \E(\eta +s\phi) \Big|_{s=0}$$
% will not be affected. For these reasons, energies in the form \eqref{dblint} allows wider range of measure $\rho_0$, and imposes the asymptotic condition on $\eta$.
% }
% 

\subsection{Weak energy solution and strong solution for hamiltonian flow}
In this section we define the weak and strong flows for the formal equalities
\begin{align*}
 \dot{\eta} &= ~~\frac{\delta \H}{\delta m}\\% = \frac{\delta \K}{\delta m} = v,\\
 \dot{m} &= -\frac{\delta \H}{\delta \eta}.% = -\frac{\delta \E}{\delta \eta}.
\end{align*}
for the hamiltonian
\begin{equation}\label{H}
\begin{aligned} 
 &\H[\eta,m] = \K[m] + \E[\eta], \quad \K[m] = \int_X \frac{m^2}{2\rho_0} \: dx, \\
 &\text{$\E$ is given by \eqref{E},\eqref{Era},\eqref{Ep}.}
\end{aligned}
\end{equation}
% We append one additional equation
% $$ \dot{\eta}_x = \partial_x v \left( = \partial_x \frac{\delta \H}{\delta m}\right).$$

Suppose $\eta$ is admissible. Since $\eta$ is assumed to be bilipschitz and so is the fixed map $\eta_*$, for $p$ and $q$ in \eqref{kernellaw} and $\gamma$ in \eqref{Ep}
% \begin{align*}
%  \infty &> \left|\iint_{X \times X} K_r([\eta]) - K_r([\eta_*]) \: \drr\right| \\
%  &\ge C\left| \iint_{X\times X} \int_0^1 \big|[\lambda \eta + (1-\lambda)\eta_*]\big|^{p-1} \big|[\eta-\eta_*]\big| \: d\lambda \drr \right|\\
%  &\ge C\big(\Lip(\eta^{-1})\big)\iint_{X\times X} |x-x'|^{p-1} \big| [\eta - \eta_*] \big| \:\drr,\\
%  \infty &> \left|\iint_{X \times X} K_a([\eta]) - K_a([\eta_*]) \: \drr\right| \\
%  &= \iint_{X\times X} \int_0^1 |[\lambda \eta + (1-\lambda)\eta_*]|^{q-1} [\eta-\eta_*] \: d\lambda \drr \\
%  &\ge C\big(\Lip(\eta)\big) \iint_{X\times X} |x-x'|^{q-1} \big| [\eta - \eta_*] \big| \:\drr.
% \end{align*}
there exist constants $E_1$ and $E_3$ depending on $\Lip(\eta^{-1})$, and $E_2$ depending on $\Lip(\eta)$ such that 
\begin{equation} \label{suff}
\begin{align*}
 &\iint_{X \times X} |x-x'|^{p-1}\Big| \big(\eta(x)-\eta(x')\big) -\big(\eta_*(x)-\eta_*(x')\big)\Big| d\rho_0(x)d\rho_0(x') < E_1,\\
 &\iint_{X \times X} |x-x'|^{q-1}\Big| \big(\eta(x)-\eta(x')\big) -\big(\eta_*(x)-\eta_*(x')\big)\Big| d\rho_0(x)d\rho_0(x') < E_2,\\
&\int_X \Big|\eta_x(x) - \eta_{*x}(x)\Big| \: dx < E_3.
\end{align*}
\end{equation}
Based on these observations, to have a weak formulation of a hamiltonian flow, we consider a function $\phi\in C^{0,1}(X)$ such that %with for any $t$
% Suppose $\eta$ and $\bar\eta$ are in $\mathcal{A}(I\times X)$, or $\phi:=\eta(t,\cdot) - \bar\eta(t,\cdot) \in C^{0,1}(X)$  satisfies
\begin{equation}
 J_r[\phi]=\iint |x-x'|^{r-1}~| \phi(x) - \phi(x')| \:d\rho_0(x)d\rho_0(x') < \infty \quad \text{for $r=p$ and $r=q$} \label{phicond}
\end{equation}
and $\phi_x$ is integrable.
For such a $\phi$, we define a {\it virtual work} as the directional derivative
$$ \phi \mapsto -\frac{d}{ds} \E(\eta + s\phi) \Big|_{s=0}.$$
% We will denote the value by $\ell[\eta](\phi)$.
\begin{proposition} \label{continuity}
 Suppose $\E$ is given by the formula \eqref{H}. Then the functional $\ell[\eta] : \phi \mapsto -\frac{d}{ds} \E(\eta + s\phi) \Big|_{s=0}$ is continuous on 
 $$\displaystyle\mathcal{C}=\left\{\phi \in C^{0,1}(X) ~|~ J_p[\phi] < \infty, \quad J_q[\phi] < \infty, \quad \text{$\phi_x$ is integrable}\right\}$$ and has the representation
 \begin{equation} \label{ellrep}
 \begin{aligned}
  \ell[\eta](\phi)&=-\int_X (1-\gamma) \left(\frac{\eta_x}{\rho_0}\right)^{-\gamma}\left(\frac{\phi_x}{\rho_0}\right) \:d\rho_0(x)\\
  &-\iint_{X\times X} K_r'\Big(\eta(x) - \eta(x')\Big)\Big(\phi(x)-\phi(x')\Big)\: d\rho_0(x)d\rho_0(x')\\
  &-\iint_{X\times X} K_a'\Big(\eta(x) - \eta(x')\Big)\Big(\phi(x)-\phi(x')\Big)\: d\rho_0(x)d\rho_0(x').
 \end{aligned}
 \end{equation}
\end{proposition}
\begin{proof}
Since $\eta^{-1}$ and $\phi$ is Lipshitz, $\tilde \phi(y) := \phi\circ \eta^{-1}(y)$ is Lipschiz. We can take $s_0>0$ so small that $s\Big|\frac{\phi_x}{\eta_x}\Big| \le \frac{1}{2}$ for $s\in(0,s_0)$.


change change change
% $\big(1 + s\frac{\phi_x}{\eta_x}\big) = 1 + s \tilde\phi_y(\eta(x)) \ge \frac{1}{2}$ for $s\in(0,s_0)$. 
Therefore,
\begin{align*}
 &\left|\frac{1}{s}\int_X \Big(\frac{\eta_x + s\phi_x}{\rho_0}\Big)^{1-\gamma} -\Big(\frac{\eta_x}{\rho_0}\Big)^{1-\gamma} \: \dr\right|\\
 &= \left|\int_X \Big(\frac{\eta_x}{\rho_0}\Big)^{1-\gamma} \frac{1}{s}\int_0^s(1-\gamma)\Big(1 + \lambda\frac{\phi_x}{\eta_x}\Big)^{-\gamma}\left(\frac{\phi_x}{\eta_x}\right)\:d\lambda \: \dr\right|\\
 &\le \left(\frac{1}{2}\right)^{-\gamma}(1-\gamma)\int_X \Big(\frac{\eta_x}{\rho_0}\Big)^{1-\gamma} \left|\frac{\phi_x}{\eta_x}\right| \: \dr
\end{align*}
is bounded uniformly of $s\in(0,s_0)$ because $\phi_x \in L^1(X)$. Then
 \begin{align*}
  -\lim_{s \rightarrow 0+} \frac{1}{s} \int_X \Big(\frac{\eta_x + s\phi_x}{\rho_0}\Big)^{1-\gamma} -\Big(\frac{\eta_x}{\rho_0}\Big)^{1-\gamma} \: \dr  = -(1-\gamma)\int_X \left(\frac{\eta_x}{\rho_0}\right)^{-\gamma}\left(\frac{\phi_x}{\rho_0}\right) \: \dr.
 \end{align*}
For non-local parts, 
\begin{align*}
  &\frac{1}{s}\iint_{X\times X} K_r\Big( (\eta + s\phi)(x) - (\eta + s\phi)(x')\Big)- K_r\Big(\eta(x) - \eta(x')\Big) \: d\rho_0(x)d\rho_0(x') \\
  = &\iint_{X\times X}\frac{1}{s}\int_0^s K_r'\Big( (\eta + \tau\phi)(x) - (\eta + \tau\phi)(x')\Big)\Big(\phi(x)-\phi(x')\Big)\: d\rho_0(x)d\rho_0(x')\\
  \le &\iint_{X\times X} |x-x'|^{p-1} |\phi(x)-\phi(x')| \: d\rho_0(x)d\rho_0(x') 
\end{align*}
is bounded uniformly of $s$ because of \eqref{suff}. Taking limit $s \rightarrow 0+$ gives the result. We do the same calculation for $K_a$. Continuity follows easily from the representation. % from the linearity.
\end{proof}

\begin{definition} Let $\eta \in \mathcal{A}(I \times X)$ and define ${v}=\dot\eta$, ${m}= \rho_0 {v}$. We say $\eta$ is a dissipative weak hamiltonian flow for $\H$  of \eqref{H} on an open interval $I$ if for all $\phi\in C_c^{0,1}(I \times X)$,
    \begin{equation} \label{wksol}
        \begin{aligned}
         -&\int_I\int_{X}^{}{m}\dot{\phi}(t,x) \: dxdt = \int_I \ell{[\eta(t,\cdot)]}\big(\phi(t,\cdot)\big) \: dt,%\\
%          &\int_I\int_X \eta_x \dot\varphi(t,x) -\dot\eta \varphi_x(t,x) \:dxdt = 0
        \end{aligned}
    \end{equation}
%     \begin{equation} \label{wksol}
%         \begin{aligned}
%         & -\int_{0}^{\infty}\int_{X}^{}{m}\dot{\phi}(t,x)dxdt\\
%         &\quad = -\int_0^\infty \int_X (1-\gamma)(\eta_x)^{-\gamma}\phi_x\: dx \\ & \quad\quad-  \int_{0}^{\infty}\iint_{X \times X}\operatorname{D}\mathcal{K}_r\big(\eta(t,x)-\eta(t,x')\big)\big(\phi(t,x)-\phi(t,x')\big) \:d\rho_0(x)d\rho_0(x')dt \\
%         &\quad\quad-  \int_{0}^{\infty}\iint_{X \times X}\operatorname{D}\mathcal{K}_a\big(\eta(t,x)-\eta(t,x')\big)\big(\phi(t,x)-\phi(t,x')\big) \:d\rho_0(x)d\rho_0(x')dt.
%         \end{aligned}
%     \end{equation}
and it satisfies
    \begin{equation} \label{dissip}
        \begin{aligned}
        -\int_I \H[\eta, {m}]\dot{\theta}(t)dt \leq 0,
        \end{aligned}
    \end{equation}
    for any non-negative $\theta\in C^{0,1}_c(I)$. 
We say a disspative weak hamiltonian flow $\bar\eta \in \mathcal{A}(I\times X)$ is a strong hamiltonian flow for $\H$ on $I$ if 
\begin{enumerate}
 \item $\ddot{\bar\eta}$, $\dot{\bar{\eta}}_x$, $\bar\eta_{xx}$ are bounded $\rho_0$-a.e., and $J_p[{\dot{\bar\eta}}]$ and $J_q[{\dot{\bar\eta}}]$ are finite, $\dot{\bar{\eta}}_x \in L^1(X)$, and
 \begin{equation}
  \int_X \dot{\bar{{m}}} \phi(x) \: dx = \ell[\bar\eta](\phi) \quad \text{for all $\phi\in C_c^{0,1}(X)$, and for a.e. $t\in I$.}
 \end{equation}
 \item \eqref{dissip} holds with equality.
\end{enumerate}
\end{definition}
% We will also use terms weak energy solution and strong solution for the respective flows.

\begin{remark} \label{wktest}
 Seen from Proposition \ref{continuity}, the equality in \eqref{wksol} holds for any $\phi=(\eta - \tilde\eta)\theta$ for $\theta\in C_0^{0,1}(I)$, and $\eta$, $\tilde\eta$ in $\mathcal{A}(I\times X)$.%, where it holds that $\phi(t,\cdot) \in C^{0,1}(X)$ with $J_p[\phi(t,\cdot)]<\infty$, $J_q[\phi(t,\cdot)]<\infty$,  and $\dot\phi(t,\cdot) \in L^1_{\rho_0}(X) \cap L^2_{\rho_0}(X)$ for all $t \in I$.
\end{remark}

\subsection{Relative energies} \label{sec:rel}
Let $\eta$ and $\bar\eta$ be weak energy and strong flow respectively. Relative energies in the below is defined on functions $\eta(t,\cdot)$, $\bar\eta(t,\cdot)$, $m(t,\cdot)$, and $\bar m (t,\cdot)$ for each time $t \in I$.
\begin{align} \small
    \K[{m} | \bar{m}] 
    %&:=K[{m}] - K[\bar{m}]-\left\langle \frac{\delta K}{\delta{m}}(\bar{m}), {m}-\bar{m} \right\rangle \nonumber \\
     &\ =\int_X \frac{|{m}|^2}{2\rho_0}- \frac{|\bar{m}|^2}{2\rho_0} - \frac{\bar{m}}{\rho_0}({m}-\bar{m})\:dx \nonumber\\ 
     &\ = \int_X \frac{|{m} - {\bar{m}|}^2}{2\rho_0}dx  = \int_X \frac{\rho_0 |{v} - {\bar{v}}|^2}{2} \: dx, \label{expK}\\
    \mathcal{E}_p[\eta | \bar{\eta}] 
    %&:= \mathcal{E}_p[\eta] - \mathcal{E}_p[\bar{\eta}] - \left\langle \frac{\delta \mathcal{E}_p}{\delta \eta}(\bar{\eta}), \eta - \bar{\eta} \right\rangle   \nonumber  \\
     & \  =\int_{X}^{}\psi(\tau) - \psi(\bar\tau) - \psi'(\bar\tau)\left( \tau - \bar{\tau} \right)d\rho_0(x) \label{exp1}\\
     & \ = \int_X\psi(\tau |\bar{\tau} ) d\rho_0(x), \quad \text{where $\psi(\tau)=\tau^{1-\gamma}$,}\nonumber \\
    \mathcal{E}_r[\eta | \bar{\eta}] 
    %&:=  \mathcal{E}_r[\eta] - \mathcal{E}_r[\bar{\eta}] - \left\langle \frac{\delta \mathcal{E}_r}{\delta \eta}, \eta - \bar{\eta}\right\rangle \nonumber  \\ 
    & \  =\iint_{X \times X}{K}_r\big(\eta(x)-\eta(x')\big) - {K}_r\big(\bar{\eta}(x)-\bar{\eta}(x')\big) \nonumber\\ 
    & \ \quad - {K}_r'\big(\bar{\eta}(x) - \bar{\eta}(x')\big)\big( \eta(x) - \eta(x') - \bar{\eta}(x) + \bar{\eta}(x') \big) \: d\rho_0(x)d\rho_0(x') \label{exp2}\\
    & \ = \iint_{X \times X} \mathcal{K}_r(\eta | \bar{\eta}) \: d\rho_0(x)d\rho_0(x'), \nonumber\\
    \mathcal{E}_a[\eta | \bar{\eta}] 
    %&:=  \mathcal{E}_a[\eta] - \mathcal{E}_a[\bar{\eta}] - \left\langle \frac{\delta \mathcal{E}_a}{\delta \eta}, \eta - \bar{\eta}\right\rangle  \nonumber \\ 
    & \ =\iint_{X \times X}{K}_a\big(\eta(x)-\eta(x')\big) - {K}_a\big(\bar{\eta}(x)-\bar{\eta}(x')\big) \nonumber \label{exp3}\\ 
    & \ \quad - {K}_a'\big(\bar{\eta}(x) - \bar{\eta}(x')\big)\big( \eta(x) - \eta(x') - \bar{\eta}(x) + \bar{\eta}(x') \big) \: d\rho_0(x)d\rho_0(x') \\
    & \ = \iint_{X \times X}{K}_a\big(\eta | \bar{\eta}\big) \: d\rho_0(x)d\rho_0(x'), \nonumber
\end{align}
where we have introduced the abbreviations of the integrands. We also introduce a notation $[u](x,x'):=u(x)-u(x')$ for a function $u$ on $X$ for simplicity. The relative Hamiltonian is
\begin{equation} \label{relHam}
    \begin{aligned}
    H[\eta, {m}| \bar\eta, \bar{m}]&:= H[\eta, {m}] - H[\bar{\eta}, \bar{m}] +\ell[\bar\eta](\eta-\bar\eta) - \int_X \frac{\bar{m}}{\rho_0}({m} - \bar{m}) \: dx\\
    &=K[{m} | \bar{m}] + \mathcal{E}_p[\eta | \bar{\eta}] + \mathcal{E}_r[\eta | \bar{\eta}] + \mathcal{E}_a[\eta | \bar{\eta}].
    \end{aligned}
\end{equation}

Now we estimate relative energies calculating its upper and lower bounds. As the weak energy solution $\eta \in \mathcal{A}(I\times X)$ is bilipschitz, $||D\eta||_\infty$ and $||D\eta^{-1}||_\infty$ are available, and avoiding from using them would be void efforts.  Nevertheless, we present estimates without them whenenver possible.
\begin{proposition}[Relative energies are finite] Let $\eta$ and $\bar\eta$ be weak and strong solution respectively. Then the expression \eqref{exp1}, \eqref{exp2}, and \eqref{exp3} are well-defined and finite.
\end{proposition} \label{ubnd}
\begin{proof}
$\E_p(\eta) - \E_p(\bar\eta)$, $\E_r(\eta) - \E_r(\bar\eta)$, and $\E_a(\eta)- \E_a(\bar\eta)$ are finite. As $(\eta_x - \bar\eta_x)$ is integrable seen by \eqref{suff},
\begin{align*}
 \left|\int_X \left(\frac{\rho_0}{\bar\eta_x}\right)^{\gamma} \left(\frac{\eta_x - \bar\eta_x}{\rho_0}\right)\:\dr \right| \le C|\Lip(\bar\eta^{-1}(t,\cdot))|^\gamma < \infty.
\end{align*}
Also,  
\begin{align*}
 &\left|K_r'\big([\bar\eta](x,x')\big)[\eta-\bar\eta](x,x')\right| \\
 &= \big|[\bar\eta]\big|^{p-1} \big|[\eta-\bar\eta]\big|
 \le ~ \Lip(\bar\eta^{-1}(t,\cdot))^{1-p} |x-x'|^{p-1}\big|[\eta-\eta_*]\big| + \big|[\bar\eta-\eta_*]\big|. 
 \end{align*}
Similarly we have
$$\left|K'_a([\bar\eta])[\eta-\bar\eta]\right| \le \Lip(\bar\eta(t,\cdot))^{q-1}\big|[\eta-\eta_*]\big| + \big|[\bar\eta-\eta_*]\big|.$$
By \eqref{suff}, the double integrals of them are finite. 
\end{proof}

\begin{theorem}\label{thm1} For a weak solution $\eta$, a strong solution $\bar\eta$, and for all $t\in I$, there exist positive constants $c_1$, $c_2$, $c_3$, and $c_4$ that satisfies the followings. The constants depends only on $p$, $q$, $\gamma$, $\Lip(\bar\eta)$, $\Lip(\bar\eta^{-1})$, $E_1$, $E_2$, $E_3$ in \eqref{suff}, and choice $\beta$ below.
\begin{enumerate}
 \item For a $\beta \in(0,1)$ and $r(\beta):=\frac{\gamma}{1-\beta}>1$,
 \begin{equation}
 \E_p[\eta|\bar\eta] \ge c_1 \left(\int_{\tau \le 3\Lip(\bar\eta)} (\tau - \bar\tau)^2 \:\dr + \left(\int_{\tau > 3\Lip(\bar\eta)} |\tau - \bar\tau|^\beta \: \dr\right)^{r(\beta)}\right). \label{L} 
 \end{equation}
 \item If $p\in (0,1)$,
 \begin{equation} \label{Lp1}
  \begin{aligned}
   &\E_r[\eta|\bar\eta] \ge c_2 \left(\iint_{\frac{\eta(x)-\eta(x')}{x-x'} \le 3\Lip(\bar\eta)}\big|[\eta - \bar\eta]\big|^2 |x-x'|^{p-2}\:d\rho_0(x)d\rho_0(x')\right. \\
 & ~~\quad \quad \quad\quad+\left.\iint_{\frac{\eta(x)-\eta(x')}{x-x'} > 3\Lip(\bar\eta)}\big|[\eta - \bar\eta]\big|^p d\rho_0(x)d\rho_0(x')\right).
  \end{aligned}
 \end{equation}
 \item {\blue If $p=1$, $\E_r[\eta | \bar\eta] = 0$. If $q=1$, $\E_a[\eta | \bar\eta] = 0$.}
 \item If $q \in (1,2)$,
       \begin{equation} \label{Lq1}
       \begin{aligned}
           &\E_r[\eta|\bar\eta] \ge c_3 \left(\iint_{\frac{\eta(x)-\eta(x')}{x-x'} \le 3\Lip(\bar\eta)}\big|[\eta - \bar\eta]\big|^2 |x-x'|^{q-2}\:d\rho_0(x)d\rho_0(x')\right. \\
           & ~~\quad \quad \quad\quad+\left.\iint_{\frac{\eta(x)-\eta(x')}{x-x'} > 3\Lip(\bar\eta)}\big|[\eta - \bar\eta]\big|^q d\rho_0(x)d\rho_0(x')\right).
       \end{aligned}
       \end{equation}
 \item If $q\ge 2$,
       \begin{equation}
        \E_a[\eta|\bar\eta] \ge c_4 \iint_{X \times X} \big|[\eta - \bar\eta]\big|^q d\rho_0(x)d\rho_0(x'). \label{Lq2}
       \end{equation}
\end{enumerate}
There exists a constant $c_5>0$ additionally depending on $\Lip(\eta^{-1})$ and choice $\beta'\in(-1,0)$ that satisfies the inequality below.
\begin{enumerate}
 \item[5.] If $p\in(-1,0)$, for a $\beta' \in(-p,1)$ and $r'(\beta'):=\frac{1-p}{1-p-\beta'}>1$
 \begin{equation}\label{Lp2}
  \begin{aligned}
   &\E_r[\eta|\bar\eta] \ge c_5 \left(\iint_{\frac{\eta(x)-\eta(x')}{x-x'} \le 3\Lip(\bar\eta)}\big|[\eta - \bar\eta]\big|^2 |x-x'|^{p-2}\:d\rho_0(x)d\rho_0(x')\right. \\
 & ~~ \quad+\left.\left(\iint_{\frac{\eta(x)-\eta(x')}{x-x'} > 3\Lip(\bar\eta)}\big|[\eta - \bar\eta]\big|^{\beta'+p}|x-x'|^{-\beta'} d\rho_0(x)d\rho_0(x')\right)^{r'(\beta')}\right).
  \end{aligned}
 \end{equation}
\end{enumerate}

% 
% \begin{align}
% &\\
% &\text{If $p\in(0,1)$,}\nonumber\\
% &\\
% &\text{and if $p\in (-1,0)$, for a $\beta' \in(-p,1)$, and $r'(\beta')=\frac{1-p}{1-p-\beta'}>1$} \nonumber \\
% &end{align}
\end{theorem}
\begin{proof}
We first estimate $\int_\Omega \psi(\eta|\bar\eta) \:dx$, where $\Omega$ is an open set such that $\rho_0(x) = \chi_\Omega(x)$. We write
\begin{align*}
 \psi(\eta|\bar\eta) = (1-\gamma)(-\gamma)\int_0^1\int_0^s (\lambda \tau + (1-\lambda) \bar\tau)^{-1-\gamma} (\tau - \bar\tau)^2 \:d\lambda ds.
\end{align*}
Note that $\tau_\lambda:=\lambda \tau + (1-\lambda)\bar\tau$ is nonnegative. If $\tau \le 3\Lip(\bar\eta)$, then $\bar\tau \le \Lip(\bar\eta)$ and $\tau_\lambda \le 3\Lip(\bar\eta)$ and thus
\begin{align*}
 \psi(\eta|\bar\eta) \ge C_1 (\tau - \bar\tau)^2.
\end{align*}
In case $\tau > 3\Lip(\bar\eta)$, then $\tau\ge\bar\tau$ and $\tau\ge \tau_\lambda$, and thus
\begin{align*}
 \tau_\lambda^{-1-\gamma}(\tau-\bar\tau)^2 
% = \left(\frac{\tau-\bar\tau}{\tau_\lambda}\right)^{1+\gamma} (\tau-\bar\tau)^{1-\gamma} 
\ge \left(\frac{\tau-\bar\tau}{\tau}\right)^{1+\gamma} (\tau-\bar\tau)^{1-\gamma} \ge \left( \frac{1}{2} \right)^{1+\gamma} (\tau-\bar\tau)^{1-\gamma}.
\end{align*}
On the integral of the last expression, we use the reverse H\"older inequaility, (Adams)
\begin{align*}
 \int_{\Omega \cap \{\tau>3\Lip(\bar\eta)\}} (\tau-\bar\tau)^{1-\gamma} \:dx 
 &\ge \frac{\left(\int_{\Omega \cap \{\tau>3\Lip(\bar\eta)\}} (\tau-\bar\tau)^{\delta s} \:dx \right)^{\frac{1}{s}}}{\left(\int_{\Omega \cap \{\tau>3\Lip(\bar\eta)\}} (\tau-\bar\tau)^{(1-\gamma-\delta) \frac{s}{s-1}} \:dx \right)^{\frac{1-s}{s}}}\\
 & \ge \frac{\left(\int_{\Omega \cap \{\tau>3\Lip(\bar\eta)\}} (\tau-\bar\tau)^{\delta s} \:dx \right)^{\frac{1}{s}}}{\left(\int_{\Omega} (\tau-\bar\tau)^{(1-\gamma-\delta) \frac{s}{s-1}} \:dx \right)^{\frac{1-s}{s}}},
\end{align*}
where for given $0<\beta <1$, choices of $s$ and $\delta$ were 
$$s = \frac{1-\beta}{\gamma} \in (0,1), \quad \delta = \frac{\beta}{s}>0,$$
which implies $\frac{(1-\gamma-\delta)s}{s-1} = 1$. Then by \eqref{M} the denominator is bounded by a constant depending on the choice, $M$, and $\gamma$. The first assertion then follows.

 
Nextly, we estimate the double integral of $K_r[\eta|\bar\eta]$. We introduce for each $(x,x')$, $x\ne x'$
$$ a(x,x') = \frac{\eta(x) - \eta(x')}{x-x'} \ge0, \quad \bar{a}(x,x') = \frac{\bar\eta(x) - \bar\eta(x')}{x-x'} \ge0.$$
Then, for $x\ne x'$
\begin{align*}
K_r[\eta|\bar\eta] &=\frac{-1}{p}\left(a^p - \bar a^p - p\bar a^{p-1}(a-\bar a)\right)|x-x'|^p \\
&= (1-p) \int_0^1 \int_0^s (a-\bar a)^2(a_\lambda)^{p-2} \: d\lambda ds\, |x-x'|^p.
\end{align*}
Similar calculations above give that
\begin{align*}
%  &\frac{-1}{p}\left(a^p - \bar a^p - \bar a^{p-1}\right) = (1-p) \int_0^1 \int_0^s (a-\bar a)^2 (a_\lambda)^{p-2} \: d\lambda ds,\\
 \iint_{X\times X} K_r[\eta|\bar\eta]\:\drr
 &\ge C_2 \iint_{a \le 3\Lip(\bar\eta)} (a-\bar a)^2 |x-x'|^{p} \: \drr\\
 & + C_3 \iint_{a > 3\Lip(\bar\eta)} (a-\bar a)^p |x-x'|^p \: \drr.
\end{align*}
Therefore the case $0<p<1$ is done. If $-1<p<0$, we estimate the second integral in the last expression. 
\begin{align*}
&\iint_{a > 3\Lip(\bar\eta)} \big| [\eta - \bar\eta]\big|^p \: \drr\\
\ge &\frac{\left(\iint_{\Omega^2\cap \{ a >3\Lip(\bar\eta)\}} \left(\big|[\eta - \bar\eta]\big|^\delta |x-x'|^{-(p-\delta)(p-1)}\right)^s \:dx \right)^{\frac{1}{s}}}{\left(\iint_{\Omega^2 \cap \{a>3\Lip(\bar\eta)\}} \left(\big|[\eta - \bar\eta]\big|^{p-\delta}|x-x'|^{(p-\delta)(p-1)}\right)^{ \frac{s}{s-1}} \: \drr \right)^{\frac{1-s}{s}}},
\end{align*}
where for given $-p<\beta' <1$, choices of $s$ and $\delta$ were
$$ s = \frac{1-p-\beta'}{1-p} \in (0,1), \quad \delta  = \frac{\beta' + p}{s} >0,$$%\delta = p + \frac{\beta'}{1-p-\beta'} >0,$$
which implies that
$$ \frac{(p-\delta)s}{s-1} = 1, \quad (p-\delta)(p-1)s = \beta'.$$%, \quad \delta s = \beta' + p >0.$$
Denominator is then bounded by a constant depending on the choice, $M_1$, and $p$.

Finally, we estimate the double integral of $K_a[\eta|\bar\eta]$. In case $q \in [1,2)$, similar estimate in the case $p\in(0,1)$ gives the result. In case $q \ge 2$, we estimate
\begin{align*}
%  \frac{1}{q}\left(a^p - \bar a^p - \bar a^{p-1}\right) &= (q-1) \int_0^1 \int_0^s (a-\bar a)^2(a_\lambda)^{q-2} \: d\lambda ds,\\
 (a-\bar a)^2 (a_\lambda)^{q-2} &= |a-\bar a|^q \left(\frac{a_\lambda}{|a-\bar a|}\right)^{q-2} \ge |a-\bar a|^q\left\{
 \begin{aligned}
  1-\lambda \quad \text{if $\bar a \ge a$,}\\
  \lambda \quad \text{if $\bar a < a$}.
 \end{aligned}\right.
\end{align*}
By taking $\displaystyle c_2:= (q-1)\min\left\{ \int_0^1\int_0^s \lambda^{q-2} \:d\lambda ds, \int_0^1\int_0^s (1-\lambda)^{q-2} \:d\lambda ds\right\} >0$, the last assertion follows.
\end{proof}
It is straightforward to have the following corollary from Theorem \ref{thm1}.
\begin{corollary} It holds that
 \begin{enumerate}
  \item $K[m|\bar m](t_0) = 0$ implies that $\dot{\eta}(t_0,\cdot) = \dot{\bar\eta}(t_0,\cdot)$ $\rho_0$-a.e.,
  \item $\E[\eta|\bar \eta](t_0)=0$ implies that $\eta(t_0,\cdot) = \bar\eta(t_0,\cdot) + C$ for any constant $C$ $\rho_0$-a.e..
 \end{enumerate}
\end{corollary}    
\begin{theorem}[Relative energy inequality] \label{ineqthm}
    Let $\eta$ and $\bar\eta$ be respectively a weak and strong solution in $\mathcal{A}(I\times X)$ with $I=(0,T)$. Then $\H[\eta,m|\bar\eta, \bar{m}](t)$ continuously extends to the endpoints and 
    \begin{equation} \label{relineq}
%         \begin{aligned}
%             - \int_{t_1}^{t_2} \H[\eta, m|\bar{\eta}, \bar{m}](t) \dot\theta(t) \:dt & \leq  \int_{t_1}^{t_2} \theta \int_X \bar{{v}}_x \psi'(\eta | \bar{\eta})\rho_0dxdt   \\
%             &  + \int_{t_1}^{t_2} \theta \iint_{X \times X} [\bar{{v}}](x, x'){K}_a'(\eta | \bar\eta)d\rho_0(x)d\rho_0(x') dt \\
%             &  + \int_{t_1}^{t_2} \theta \iint_{X \times X} [\bar{{v}}](x, x'){K}_r'(\eta | \bar\eta)d\rho_0(x)d\rho_0(x') dt, 
%         \end{aligned} 
        \begin{aligned}
            \left.\H[\eta, m|\bar{\eta}, \bar{m}]\right| (T) - \left.\H[\eta, m| \bar{\eta}, \bar{m}]\right| (0) & \leq  \int_0^T \int_X \bar{{v}}_x \psi'(\tau | \bar{\tau})dxdt  \\
            &  + \int_{0}^{T} \iint_{X \times X} [\bar{{v}}](x, x'){K}_a'(\eta | \bar\eta)d\rho_0(x)d\rho_0(x') dt \\
            &  + \int_{0}^{T} \iint_{X \times X} [\bar{{v}}](x, x'){K}_r'(\eta | \bar\eta)d\rho_0(x)d\rho_0(x') dt, 
        \end{aligned}
    \end{equation}
    where $[\bar{{v}}](x, x'):= \bar{{v}}(x) - \bar{{v}}(x')$.
\end{theorem}
Before we prove the Theorem \ref{ineqthm}, we establish the following estimates.
\begin{proposition} \label{pp}
 Let $\phi$ be a Lipshitz function. Then there exist positive constants $C_1$, $C_2$, and $C_3$ satisfying inequailities below. The constants depend on $p$, $q$, $\gamma$, $\Lip(\phi)$, $\Lip(\bar\eta^{-1})$, and $\Lip(\eta^{-1})$. 
 \begin{align}
  \Big|\phi_x\Big(\psi'(\tau) - \psi'(\bar\tau) - \psi''(\bar\tau)(\tau - \bar\tau)\Big)\Big|  &\le C_1 \psi(\tau|\bar\tau), \label{e1}\\
  \Big|[\phi]\Big(K_r'([\eta]) - K_r'([\bar\eta]) - K_r''([\bar\eta])[\eta-\bar\eta]\Big)\Big| &\le C_2 K_r(\eta|\bar\eta), \label{e2}\\
   \Big|[\phi]\Big(K_a'([\eta]) - K_a'([\bar\eta]) - K_a''([\bar\eta])[\eta-\bar\eta]\Big)\Big| &\le C_3 K_a(\eta|\bar\eta) \label{e3}. 
 \end{align}
\end{proposition}
\begin{proof}
 \begin{align*}
  &\phi_x\psi'(\tau|\bar\tau)=\phi_x \Big(\psi'(\tau) - \psi'(\bar\tau) - \psi''(\bar\tau)(\tau - \bar\tau)\Big)\\
  &= \phi_x(\tau - \bar\tau)^2\int_0^1 \int_0^s \psi'''(\lambda \tau + (1-\lambda)\bar\tau)\: d\lambda ds\\
  &= (\tau - \bar\tau)^2(-\gamma+1)(-\gamma)(-\gamma-1) \int_0^1\int_0^s (\tau^\lambda)^{-\gamma-1}~\frac{\phi_x}{\tau^\lambda} \: d\lambda ds,
  %(\lambda \eta_x + (1-\lambda)\bar\eta_x)^{-\gamma-1} \frac{\phi_x}{\lambda \eta_x + (1-\lambda)\bar\eta_x} \: d\lambda ds.
 \end{align*}
 where $\tau^\lambda = \lambda \tau + (1-\eta) \bar\tau$. We have that
\begin{align*}
 \left| \frac{\phi_x}{\lambda \eta_x + (1-\eta) \bar\eta_x} \right| \le \max \left\{ \left| \frac{\phi_x}{\eta_x}\right|, \left| \frac{\phi_x}{\bar\eta_x}\right|\right\} \le \Lip(\phi) \big( \Lip(\eta^{-1}) + \Lip(\bar\eta^{-1})\big)
\end{align*}
and \eqref{e1} follows. Similarly,
\begin{align*}
 &[\phi]K_r'(\eta | \bar\eta)=
 [\phi]\Big(K_r'([\eta]) - K_r'([\bar\eta]) - K_r''([\bar\eta])[\eta-\bar\eta]\Big)\\
 &=-(p-1)(p-2)[\eta-\bar\eta]^2\int_0^1 \int_0^s \Big|[\eta^\lambda]\Big|^{p-2} \frac{[\phi]}{[\eta^\lambda]} \:d\lambda ds,
\end{align*}
and 
$$\left|\frac{[\phi]}{[\lambda \eta + (1-\lambda)\bar\eta]} \right| \le  \Lip(\phi) \big( \Lip(\eta^{-1}) + \Lip(\bar\eta^{-1})\big),$$
and \eqref{e2} follows. \eqref{e3} follows similarly.
\end{proof}


% 
% \begin{remark}
%  We will see later that $\H[\eta,m|\bar\eta, \bar{m}](t)$ continuously extends to the endpoints of interval.
% \end{remark}
\begin{proof}[proof of Theorem \ref{ineqthm}]
From \eqref{relHam}, and energy inequaility \eqref{dissip},
\begin{equation} \label{firsteqn}
 \begin{aligned}
        -\int_I &\H[\eta, {m} | \bar{\eta}, {\bar{m}}] \dot\theta(t)\:dt \\
        &= -\int_I \bigg(\H[\eta, {m}] - \H[\bar{\eta}, {\bar{m}}]-  \left\langle \frac{\delta H}{\delta \eta}(\bar{\eta}), \eta - \bar\eta\right\rangle - \left\langle \frac{\delta H}{\delta {m}}(\bar{m}),{m} - \bar{m}\right\rangle \bigg)\dot\theta(t) \: dt\\
        &\le -\int_I \bigg(\left\langle -\frac{\delta\E}{\delta \eta}(\bar{\eta}), \eta - \bar\eta\right\rangle - \left\langle \frac{\delta \K}{\delta {m}}(\bar{m}),{m} - \bar{m}\right\rangle \bigg)\dot\theta(t) \: dt\\
%         &= -\int_I \bigg(\left\langle -\frac{\delta\E_p}{\delta \eta}(\bar{\eta}), \eta - \bar\eta\right\rangle + \left\langle -\frac{\delta\E_r}{\delta \eta}(\bar{\eta}), \eta - \bar\eta\right\rangle + \left\langle -\frac{\delta\E_a}{\delta \eta}(\bar{\eta}), \eta - \bar\eta\right\rangle\bigg)\dot\theta(t) \:dt\\
%         &+ \int_I \left\langle \frac{\delta \K}{\delta {m}}(\bar{m}),{m} - \bar{m}\right\rangle \dot\theta(t) \: dt.\\
%         &-\int_I \left\langle -\frac{\delta\E}{\delta \eta}(\bar{\eta}), \eta - \bar\eta\right\rangle\dot\theta(t) \: dt \\
%         &= -\int_I \iint_{X^2} K\\
%         &\le  \int_I \bigg(-\K[m|\bar m]- \E_p[\eta|\bar\eta] - \E_r[\eta|\bar\eta]- \E_a[\eta|\bar\eta]\bigg)\dot\theta(t) \: dt\\
%         &\int_I\left\langle \frac{\delta H}{\delta \eta}(\bar{\eta}), \eta - \bar\eta\right\rangle \dot{\theta}d\tau + \int_0^\infty \left\langle \frac{\delta H}{\delta {m}}(\bar{m}),{m} - \bar{m}\right\rangle \dot{\theta}d\tau \\
%         & = H[\eta, {m}]\theta|_{\tau=0} - H[\bar{\eta}, \bar{m}]\theta|_{\tau=0} \\
        & =\int_I\int_X \psi'(\bar\tau)(\tau - \bar\tau)\dot\theta \: \dr dt \\
        & + \int_I\iint_{X \times X} D\mathcal{K}\big([\bar{\eta}](x, x')\big)\big([\eta](x, x') - [\bar{\eta}](x, x')\big)\dot\theta \: \drr dt\\
        &  + \int_I\int_X ({m}-\bar{{m}})\bar{v}\dot\theta dxdt.
\end{aligned}
\end{equation}
Since $\bar\eta$ is a strong solution, we can write
\begin{align*}
        \int_I & \int_{X} \psi'(\bar\tau)(\tau - \bar\tau) \dot\theta \:\dr dt\\
        & = \int_I\int_X \Big( \big(\theta\psi'(\bar\tau)\big)^\bdot -  \theta\psi''(\bar\tau)\frac{\bar{v}_x}{\rho_0}\Big)(\tau - \bar\tau) \:\dr dt, \\
        \intertext{where the integrands are bounded by $C\left|\frac{\bar{\eta}_x}{\rho_0}\right|^{-\gamma}\frac{|\eta_x - \bar\eta_x|}{\rho_0}$,}
        \int_I & \iint_{X\times X} D\mathcal{K}\big([\bar{\eta}]\big)[\eta-\bar{\eta}]\dot\theta \: \drr dt,\\
        & = \int_I \iint_{X\times X} \Big(\big(\theta D\K([\bar\eta]) \big)^{\boldsymbol{\cdot}} - \theta D^2\K([\bar\eta])[\bar v]\Big)[\eta - \bar\eta] \:\drr dt,\\
        \intertext{where the integrands are bounded by $C(|x-x'|^{p-1} + |x-x'|^{q-1})|[\eta - \bar\eta]|$,} 
        \int_I&\int_X ({m}-\bar{{m}})\bar{v}\dot\theta dxdt\\
        & = \int_I\int_X \Big(\big(\theta \bar{v}\big)^\bdot - \theta \dot{\bar{v}}\Big)({m}-\bar{{m}}) \: dxdt.
\end{align*}
Consider an approximation $(\eta^\epsilon)_{\epsilon>0}$ such that for each $t\in I$ in $\eta^\epsilon(t,\cdot)$, $\dot{\eta}^\epsilon(t,\cdot)$, and $\dot{\eta}_x^\epsilon(t,\cdot)$ are in $C_c^\infty(X)$ and repectively converges $\rho_0$-a.e. to $\eta$, $\dot{\eta}$, $\dot{\eta}_x$ as $\epsilon \rightarrow 0$. Consider similarly $(\bar\eta^\epsilon)_{\epsilon>0}$.
We have then
\begin{equation}\label{secondeqn}
\begin{aligned}
 \int_I\int_X -\theta \dot{\bar{v}}(m-\bar m) \:dx dt &= \lim_{\epsilon \rightarrow 0}  \int_I\int_X -\theta \dot{\bar{m}}(\dot\eta^\epsilon -\dot{\bar\eta}^\epsilon) \:dx dt\\
 &= \lim_{\epsilon \rightarrow 0}\int_I\int_X \psi'(\bar\tau)\theta\left(\frac{\dot{\eta}^\epsilon_x - \dot{\bar\eta}^\epsilon_x}{\rho_0}\right) \: \dr dt \\
 &+ \lim_{\epsilon \rightarrow 0}\int_I\iint_{X\times X} DK([\bar\eta])\theta  [\dot{\eta}^\epsilon-\dot{\bar\eta}^\epsilon] \:\drr dt\\
 &=\lim_{\epsilon \rightarrow 0}-\int_I\int_X \big(\theta\psi'(\bar\tau)\big)^\bdot \left(\frac{{\eta}^\epsilon_x - {\bar\eta}^\epsilon_x}{\rho_0}\right) \:\dr dt \\
 &+\lim_{\epsilon \rightarrow 0}-\int_I\iint_{X\times X} \big(\theta DK([\bar\eta])\big)^\bdot [\eta^\epsilon-\bar\eta^\epsilon] \:\drr dt \\
 &= -\int_I\int_X \big(\theta\psi'(\bar\tau)\big)^\bdot \left(\frac{{\eta}_x - {\bar\eta}_x}{\rho_0}\right) \:\dr dt \\
 &-\int_I\iint_{X\times X} \big(\theta DK([\bar\eta])\big)^\bdot [\eta-\bar\eta] \:\drr dt.
\end{aligned}
\end{equation}
% We have cancelation
% \begin{align*}
%  \int_I\int_X -\theta \dot{\bar{v}}(m-\bar m) \:dx dt &+ \int_I\int_X \big(\theta\psi'(\bar\eta_x)\big)^\bdot (\eta_x - \bar\eta_x) \:\dr dt \\
%  &+\int_I\iint_{X\times X} \big(\theta DK([\bar\eta])\big)^\bdot [\eta-\bar\eta] \:\drr dt = 0
% \end{align*}
% since by \eqref{wksol} and by integration by parts
% \begin{align*}
%  \int_I\int_X \big(\theta\psi'(\bar\eta_x)\big)^\bdot (\eta_x - \bar\eta_x) \:\dr dt &= \int_I\int_X \big(\theta\psi'(\bar\eta_x)\big)_x (\dot\eta - \dot{\bar\eta}) \:\dr dt,\\
%  \int_I\iint_{X\times X} \big(\theta DK([\bar\eta])\big)^\bdot [\eta-\bar\eta] \:\drr dt &= \int_I\iint_{X\times X} -\big(\theta DK([\bar\eta])\big) [\dot\eta-\dot{\bar\eta}] \:\drr dt.
% \end{align*}
After cancelation, remaining terms are
\begin{align*}
 &\int_I\int_X \big(\theta \bar{v}\big)^\bdot ({m}-\bar{{m}}) \:dxdt\\
 &-\int_I\theta \int_X \left(\frac{\bar{v}_x}{\rho_0}\right)\psi''(\bar\tau) (\tau - \bar\tau)\: \dr dt
 -\int_I\iint_{X\times X}\theta D^2 K([\bar\eta])[\eta-\bar\eta] \:\drr dt\\
 &= \int_I \theta \int_X \left(\frac{\bar{v}_x}{\rho_0}\right) \Big( \psi'(\tau) - \psi'(\bar\tau) - \psi''(\bar\tau)(\tau - \bar\tau)\Big) \: \dr dt\\
 &+ \int_I \theta \int_{X\times X} [\bar v] \Big( DK([\eta]) - DK([\bar\eta]) - D^2K([\bar\eta])[\eta-\bar\eta]\Big) \: \drr dt.
\end{align*}

Considering estimates \eqref{e1}-\eqref{e3} and Proposition \ref{ubnd}, $t \mapsto \H[\eta,m|\bar\eta,\bar{m}](t)$ is absolutely continuous, and $\H[\eta,m|\bar\eta,\bar{m}](t)$ is extended to endpoints of interval $I$. To have \eqref{relineq}, we choose 
\begin{equation} \label{theta}
    \theta_\epsilon(t) = 
    \begin{cases}
        \frac{t}{\epsilon} & 0 \leq t < \epsilon, \\
        1 & \epsilon \le t \le T-\epsilon,\\
%         -\frac{1}{\epsilon}(\tau-t-\epsilon) & t \leq \tau < t+\epsilon \\
        -\frac{1}{\epsilon}(T-t) & T-\epsilon \le t \le T,
    \end{cases}
\end{equation}
and take $\epsilon \rightarrow 0$.
\end{proof}

By Theorem \ref{ineqthm} and estimates \eqref{e1}-\eqref{e3}, there is $C_0 = \max\{C_1,C_2,C_3\}>0$ such that for $t \in [t_1, t_2] \subset I$,
    \begin{equation} \label{e*}
        \begin{aligned}
            \left.H[\eta, {m} | \bar{\eta}, \bar{{m}}]\right|(t) \leq \left.H[\eta, {m} | \bar{\eta}, \bar{{m}}]\right|(t_1) + C_0\int_{t_1}^{t_2} \mathcal{E}[\eta | \bar{\eta}]dt
        \end{aligned}
    \end{equation}

   
\subsection{Applications of relative energy inequality}
In this section, we make use of Theorem \ref{thm1} and Theorem \ref{ineqthm} to deduce several results.
\subsubsection{Weak-strong uniqueness}


\begin{theorem}[Weak-strong uniqueness]
    Suppose $\eta$ and $\bar\eta$ are weak and strong flow in $\mathcal{A}((0,T)\times X)$ with
    $\eta(0,x) = \bar\eta(0,x)$, $\dot{\eta}(0,x) = \dot{\bar\eta}(0,x)$ for $\rho_0$-a.e. $x$. Then $\eta(t,x) = \bar\eta(t,x)$ for all $t\in(0,T)$ and $\rho_0$-a.e. $x$. 
\end{theorem}
\begin{proof}
 Assumptions on initial data imply that $\H[\eta,m | \bar\eta,\bar{m}](0)=0$. Since $\H[\eta,m|\bar\eta,\bar{m}]\ge \E[\eta|\bar\eta]$, estimate \eqref{e*} with $\H[\eta,m|\bar\eta,\bar{m}]$ in place of $\E[\eta|\bar\eta]$ in the integrand gives that $\H[\eta,m|\bar\eta,\bar{m}](t)=0$ for all $t \in I$. In particular, $\K[m|\bar{m}](t)=0$ implies that $\dot\eta(t,x) = \dot{\bar\eta}(t,x)$ for $\rho_0$-a.e. $x$. By integrating it with the initial data, we conclude also that $\eta(t,x) = \bar\eta(t,x)$ for all $t\in I$ and $\rho_0$ a.e. $x$.
\end{proof}

\subsubsection{Uniform stability of a rarefaction}
\begin{theorem}[Uniform stability]
    Assume that $\bar{{v}}$ is monotonically increasing and let $\displaystyle\ell(t):=\min_{x>x'} \left\{ \frac{\bar{v}(x) - \bar{v}(x')}{x-x'}\right\}\ge 0$. If $0 < p < 1$ and $1 < q < 2$, then there exists a constant $c_0>0$ such that for $L(t) = \frac{c_0\ell(t)}{\Lip(\eta(t,\cdot)) + \Lip(\bar\eta(t,\cdot))}$ following decay estimate holds:
\begin{align} \label{decay}
 \H[\eta,m|\bar\eta,\bar{m}](t) + \int_{0}^t L(s)\E(\eta|\bar\eta)(s) \:ds \le \H[\eta,m|\bar\eta,\bar{m}](0).  
\end{align}
%     \begin{equation}
%         \left.H[\eta, {m}| \bar{\eta}, \bar{{m}}]\right|_{\tau = t} \leq \left.H[\eta, {m}| \bar{\eta}, \bar{{m}}]\right|_{\tau=0}.
%     \end{equation}    
\end{theorem}
\begin{remark}
$L(t)$ is bounded above by
$\displaystyle\frac{c_0 \ell(t)}{\Lip(\bar\eta(t))} \le \frac{c_0\ell(t)}{ \ell(0) + \int_0^t \ell(s)\:ds}$
and thus $\E[\eta|\bar\eta]$ does not necessarily decay to $0$ as $t \rightarrow \infty$.
\end{remark}


\begin{proof}
Recall from Proposition \ref{pp} that 
\begin{equation*}
 \bar{v}_x\psi'(\eta_x | \bar\eta_x) = -\gamma(\gamma^2-1) (\eta_x - \bar\eta_x)^2 \int_0^1\int_0^s (\eta^\lambda_x)^{-\gamma-1}~\frac{\bar{v}_x}{\eta^\lambda_x} \: d\lambda ds.
\end{equation*}
From monotonicity of $\bar{v}$, $\eta$, and $\bar\eta$ we have that for each $t\in I$
\begin{align*}
\frac{\bar{v}_x}{\lambda\eta_x + (1-\lambda)\bar\eta_x} \ge \frac{\ell(t)}{\Lip(\eta(t,\cdot)) + \Lip(\bar\eta(t,\cdot))}
\end{align*}
We also have that
\begin{equation} \label{relK}
\begin{aligned}
[\bar v] K_r'(\eta|\bar\eta)  &= -(p-1)(p-2)[\eta-\bar\eta]^2\int_0^1 \int_0^s \Big|[\eta^\lambda]\Big|^{p-2} \frac{[\phi]}{[\eta^\lambda]} \:d\lambda ds,\\
[\bar v] K_a'(\eta|\bar\eta)  &= +(q-1)(q-2)[\eta-\bar\eta]^2\int_0^1 \int_0^s \Big|[\eta^\lambda]\Big|^{q-2} \frac{[\phi]}{[\eta^\lambda]} \:d\lambda ds,
\end{aligned}
\end{equation}
and 
$$\frac{[\bar{v}]}{[\lambda \eta + (1-\lambda)\bar\eta]} \ge \frac{\ell(t)}{\Lip(\eta(t,\cdot)) + \Lip(\bar\eta(t,\cdot))}.$$
Since $\gamma>1$, $p\in(-1,1)$, and $q\in(1,2)$, $c_0:= \min\big\{\gamma(\gamma^2-1), (p-1)(p-2), -(q-1)(q-2)\big\}$ is a positive constant. Let $L(t)$ be $\frac{c_0\ell(t)}{\Lip(\eta(t,\cdot)) + \Lip(\bar\eta(t,\cdot))}$. By \eqref{relineq}, we have the decay estimate
\begin{align} 
 \H[\eta,m|\bar\eta,\bar{m}](t) + \int_{0}^t L(s)\E(\eta|\bar\eta)(s) \:ds \le \H[\eta,m|\bar\eta,\bar{m}](0).  
\end{align}
% Let $\varphi(t): = \int_0^t \kk(\tau)E(\tau)$, then we have for all $t\in[0,T]$
% \begin{align*}
%  \varphi'(t) + \kk(t)K(t) + \kk(t)\varphi(t) - \kk(t)H(0) \le 0.
% \end{align*}
% Multiplying both sides by $e^{\int_0^t \kk(s)\:ds}$ and integrating it, we have for all $t\in [0,T]$,
% \begin{align*}
%  \int_0^t \kk(\tau)E(\tau) \:d\tau + e^{-\int_0^t \kk(s)\:ds} \int_0^t e^{\int_0^\tau a(s)\:ds} \kk(\tau) K(\tau) \:d\tau + H(0)\left( 1 - e^{-\int_0^t \kk(s)\:ds}\right) \le 0.
% \end{align*}
% Therefore for all $t \in [0,T]$, we have that
% $$\kk(t) \left( E(t)  + K(t) - e^{-\int_0^t \kk(s)\:ds}\left(H(0) + \int_0^t e^{\int_0^\tau \kk(s)\:ds} \kk(\tau)K(\tau)\: d\tau\right)\right)\le 0$$
% {
% now},

% 
%     From \eqref{relineq} and the proof of Lemma \ref{plaw},
%     \begin{equation} \label{ref:relineq}
%         \begin{aligned}
%             &\left.H[\eta, {m} | \bar{\eta}, \bar{{m}}]\right|_{\tau=t} - \left.H[\eta, {m} | \bar{\eta}, \bar{{m}}]\right|_{\tau=0} \\
%             &\leq\int_0^t \iint_{X \times X}\big(\bar{\mathbf{v}}(x) - \bar{\mathbf{v}}(x')\big)\Big(\mathcal{K}'\big(\eta(x) - \eta(x')\big) - \mathcal{K}'\big( \bar{\eta}(x) - \bar{\eta}(x')\big) \\
%             &\quad  - \mathcal{K}''\big(\bar{\eta}(x) - \bar{\eta}(x')\big)\big(\eta(x) - \eta(x') - \bar{\eta}(x) + \bar{\eta}(x')\big)  \Big)\rho_0(x)\rho_0(x')dxdx'd\tau \\
%             & =\int_0^t\iint_{X \times X} \big(\eta(x, x') - \bar\eta(x, x')\big)^2 \\
%             & \quad \quad  \bigg( \int_0^1 \int_0^s \bar{\mathbf{v}}(x, x') \mathcal{K}'''\Big(\tau\big(\eta(x, x')\big)+ (1-\tau)\big(\bar\eta(x, x')\big)\Big)d\tau ds \bigg)\rho_0(x)\rho_0(x')dxdx'd\tau
%         \end{aligned}
%     \end{equation}
%     As in the proof of Lemma \ref{plaw}, we compute for $\mathcal{K}_r$
%     \begin{equation} \label{uest:rep}
%         \begin{aligned}
%         & \int_0^1 \int_0^s \bar{\mathbf{v}}(x, x') \big(\eta(x, x') - \bar\eta(x, x')\big)^2 \mathcal{K}_r'''\Big(\tau\eta(x, x')+ (1-\tau)\bar\eta(x, x')\Big)d\tau ds  \\
%         &= -(p-1)(p-2)\big(\eta(x, x') - \bar\eta(x, x')\big)^2  \\ 
%         & \quad \quad \int_0^1 \int_0^s \frac{\bar{\mathbf{v}}(x, x')}{x-x'}(x-x') \operatorname{sgn}(x-x') |\tau\eta(x,x') + (1-\tau)\bar\eta(x, x')|^{p-2}d\tau ds \\
%         & \leq 0.
%         \end{aligned}
%     \end{equation}
%     On the other hand, the similar computation for $\mathcal{K}_a$ gives
%     \begin{equation} \label{uest:atr}
%         \begin{aligned}
%         & \int_0^1 \int_0^s \bar{\mathbf{v}}(x, x') \big(\eta(x, x') - \bar\eta(x, x')\big)^2 \mathcal{K}_a'''\Big(\tau\eta(x, x')+ (1-\tau)\bar\eta(x, x')\Big)d\tau ds  \\
%         &= (q-1)(q-2)\big(\eta(x, x') - \bar\eta(x, x')\big)^2  \\ 
%         & \quad \quad \int_0^1 \int_0^s \frac{\bar{\mathbf{v}}(x, x')}{x-x'}(x-x') \operatorname{sgn}(x-x')|\tau\eta(x,x') + (1-\tau)\bar\eta(x, x')|^{q-3}d\tau ds \\
%         & \leq 0.
%         \end{aligned}
%     \end{equation}
%     Combining \eqref{ref:relineq}, \eqref{uest:rep} and \eqref{uest:atr}, we get the result.
\end{proof}

\section{Pressureless Euler-Poisson system in $1$ space dimension with $p=1$ and $q=2$}
In this section, the stability of pressureless system of equations is studied. For this pressureless system, $p=1$ and $q=2$ are chosen, and normalization is 
\begin{align*}
 &\E = \E_r + \E_a \quad \text{in \eqref{Era}}, \quad K_r(y) = - \frac{|y|}{2}, \quad K_a(y) = \frac{y^2}{2},\end{align*}
so that we have 
$$-\partial^2_{yy} K_r(y) = \delta_0, \quad -\partial^2_{yy} K_a(y) \equiv -1.$$ 
For the formulation in Eulerian coordinate therefore 
\begin{align*}
 c(y) = \int_\mathbb{R} K(y-z) \rho(z)\: dz, \quad K(y) = K_r(y) + K_a(y)
\end{align*}
solves the Poisson equation
\begin{align*}
 -\partial^2_{yy} c(y) = \rho(y) - (\rho), \quad (\rho):=\int_\mathbb{R} \rho(y) \:dy.
\end{align*}

To author's knowledge, the questions of whether the presureless system exhibits singularities are not completely understood. Specifically, we do not have knowledge whether delta shock, or vaccuum form. {\blue some survey}. Still, in the absense of pressure, it would be reasonble not to expect to have $\eta_x$ bounded above and below away from $0$. The framework in this section is thus formulated differently from those in Section \ref{sec:EP}, so that delta shock and vaccuum are allowed to form for a weak solution.

Advantages of having relative energy with the position map $\eta(t,\cdot)$ as state variable is best illustrated for this presureless Euler-Poisson system in $1D$ with $p=1$, $q=2$:
\begin{enumerate}
 \item The relative hamiltonian energy measures $L^2$ distance of states:
 $$ \H[\eta,m|\bar\eta,\bar{m}] = \frac{1}{2} \|v - \bar{v}\|_{L^2_{\rho_0}}^2 + \|\eta - \bar\eta_{c(t)}\|_{L^2_{\rho_0}}^2,$$
 where $\bar\eta_{c(t)} = \bar\eta - c(t)$ for each $t$, where the shift factor $c(t)$ is so that the center of masses of $\eta$ and $\bar\eta$ coincide. In particular, we will observe that the relative energy contribution from the repulsive kernel $K_r(y) = -\frac{|y|}{2}$ simply vanish.
 \item In the assumption of $\eta$ and $\bar\eta$ are both strictly monotone, i.e., no  delta shock appears, we will observe that the $L^2$ distance only decreases in time, or we have uniform stability
 $$\int_X \frac{|v-\bar v|^2}{2} + |\eta - \bar\eta_{c(t)}|^2 \:\dr\bigg|_{t} \le \int_X \frac{|v-\bar v|^2}{2} + |\eta - \bar\eta_{c(0)}|^2 \:\bigg|_{t=0}.$$
 Even for $\eta$ that is merely monotone, we only have linear growth of $L^2$ distance squared. This is much stronger stability estimates than the one with Gronwall type inequality, where the constant exponentially grows.
\end{enumerate}


\subsection{Formulation of the problem}
Formulation of the problem will mostly go parallel to that in Section \ref{sec:EP}. Not to make the formulation too long, we collect the necessary material all in this section. Our working assumptions are summarized in the followings. Let $I$ be an open interval.
\begin{equation} \label{assumption}\tag{A}
 \begin{aligned}
  1.& \quad \rho_0 \in L^1(X).\\
  2.& \quad \text{For each $t\in I$} \quad \eta(t,\cdot) \in L^1_{\rho_0}\cap L^2_{\rho_0}(X).\\
  3.& \quad \text{For each $t\in I$} \quad \dot{\eta}(t,\cdot) \in L^1_{\rho_0}\cap L^2_{\rho_0}(X). 
 \end{aligned}
\end{equation}
For $\eta$ satisfying \eqref{assumption}, we speak of center of mass defined by
$$ c[\eta](t) = \frac{\int_X \eta(t,x) \:\dr}{\int_X \dr}.$$
Energy is given by 
\begin{align*}
 &\E = \E_r + \E_a \quad \text{in \eqref{Era}}, \quad K_r(y) = - \frac{|y|}{2}, \quad K_a(y) = \frac{y^2}{2},\end{align*}
and the hamiltonian is $\H[\eta,m] = \K[m] + \E[\eta]$, where $m= \rho_0v$, $v=\dot{\eta}$. Hamiltonian is finite by \eqref{assumption}. We also define the {\it virtual work} functional $\ell[\eta]$ 
$$ \phi \mapsto -\frac{d}{ds} \E(\eta + s\phi) \Big|_{s=0}$$
on $\phi\in L^1_{\rho_0}(X) \cap L^2_{\rho_0}(X)$. Continuity of $\ell[\eta]$ is justified in Proposition \ref{vw2}. 

Having defined $\ell[\eta]$, we define the weak and strong solutions. The map $\eta$ defined on $I\times X$ satisfying \eqref{assumption}, with $m=\rho_0 v$, $v= \dot{\eta}$,  is said to be a dissipative weak solution if 
        \begin{align}
         -\int_I\int_{X}^{}{m}\dot{\phi}(t,x) \: dxdt &= \int_I \ell{[\eta(t,\cdot)]}\big(\phi(t,\cdot)\big) \: dt, \label{wksol2}\\
            -\int_I \H[\eta, {m}]\dot{\theta}(t)dt &\leq 0, \label{dissip2}
\end{align}
holds for all $\phi \in C_c(I \times X)$, and for any non-negative $\theta\in C^{0,1}_c(I)$. A weak solution is said to be a strong solution if it additionally satisfies that
\begin{enumerate}
 \item $\bar\eta$ is strictly increasing in $x$.
 \item $\ddot{\bar\eta}$ is bounded $\rho_0$-a.e.,
 \item 
 \begin{equation}
  \int_X \dot{\bar{{m}}} \phi(x) \: dx = \ell[\bar\eta](\phi) \quad \text{for all $\phi\in C_c(X)$, and for a.e. $t\in I$.}
 \end{equation}
 \item and \eqref{dissip2} holds with equality.
\end{enumerate}
Now we prove the continuity of the virtual work functional.
\begin{proposition} \label{vw2}
 Suppose $\E$ is given by the formula \eqref{H}. Then the functional $\ell[\eta] : \phi \mapsto -\frac{d}{ds} \E(\eta + s\phi) \Big|_{s=0}$ is continuous on
  $\phi \in L^1_{\rho_0}(X) \cap L^2_{\rho_0}(X)$
%  $$\displaystyle\mathcal{C}_\eta=\left\{\phi  ~|~ \iint_{X\times X} (1+ |[\eta]| + |[\phi]|) [\phi] \:\drr < \infty\right\}$$ 
 and has the representation
 \begin{equation} 
%  \begin{aligned}
%   \ell[\eta](\phi)=&-\iint_{X\times X} \sgn([\eta])\Big(\phi(x)-\phi(x')\Big)\: d\rho_0(x)d\rho_0(x')\\
%   &-\iint_{X\times X} \Big(\eta(x) - \eta(x')\Big)\Big(\phi(x)-\phi(x')\Big)\: d\rho_0(x)d\rho_0(x').
%  \end{aligned}
 \begin{aligned}
  \ell[\eta](\phi)=&-\iint_{X\times X} \Big(\frac{\sgn([\eta])}{2} - [\eta]\Big)[\phi]\: d\rho_0(x)d\rho_0(x').
 \end{aligned}
 \end{equation}
\end{proposition}
\begin{proof}
\begin{align*}
  &\frac{1}{2s}\iint_{X\times X} \Big| (\eta + s\phi)(x) - (\eta + s\phi)(x')\Big|- \Big|\eta(x) - \eta(x')\Big| \: d\rho_0(x)d\rho_0(x') \\
  &\le  \frac{1}{2}\iint_{X\times X}\Big|\phi(x) - \phi(x')\Big| \: \drr% < \infty
\end{align*}
is bounded uniformly of $s$. Taking limit $s \rightarrow 0+$ gives the result. Also,
\begin{align*}
 &\frac{1}{2s}\iint_{X\times X} \Big( (\eta + s\phi)(x) - (\eta + s\phi)(x')\Big)^2- \Big(\eta(x) - \eta(x')\Big)^2 \: d\rho_0(x)d\rho_0(x') \\
  &= \frac{1}{2} \iint_{X\times X}\Big( (2\eta + s\phi)(x) - (2\eta + s\phi)(x')\Big)\Big(\phi(x)-\phi(x')\Big)\: d\rho_0(x)d\rho_0(x')\\
  &= \iint_{X\times X} \Big( \eta(x) - \eta(x')\Big)\Big(\phi(x)-\phi(x')\Big) + \frac{s}{2} \Big(\phi(x)-\phi(x')\Big)^2  \: d\rho_0(x)d\rho_0(x')% < E_1 \quad \text{in \eqref{suff}}
\end{align*}
is bounded uniformly of $s$. Taking limit $s \rightarrow 0+$ gives the result. Continuity follows easily from the representation. % from the linearity.
\end{proof}


\subsection{Relative energies}
Relative energies are defined in the same way as in \eqref{expK}-\eqref{exp3}. Calculation shows that
\begin{align*}
 \E_a[\eta|\bar\eta] &= \frac{1}{2}\iint [\eta-\bar\eta]^2\:\drr,\\
 \E_r[\eta|\bar\eta] &= \frac{1}{2}\iint \Big[ -|[\eta]| + |[\bar\eta]| +\sgn([\bar\eta])[\eta-\bar\eta] \:\drr\\
  &= \frac{1}{2}\iint [\eta]\Big(\sgn([\bar\eta]) - \sgn([\eta])\Big)\:\drr = 0.
\end{align*}
In the last equality, we used that $[\eta]\ne 0 \Longrightarrow \sgn([\eta])=\sgn([\bar\eta]) = \sgn(x-x')$. Relative kinetic energy is $\displaystyle\K[m|\bar m]  = \int_X \frac{\rho_0(v-\bar v)^2}{2} \: dx$. Notably, the relative energy for the repulsive kernel $K_r(y) = -\frac{|y|}{2}$ vanishes.

The relative energy for the attractive kernel can be written as follows. Let $\bar\eta_{c(t)}(t,\cdot): = \bar\eta(t,\cdot) - c(t)$, $c(t) = c[\bar\eta] + c[\eta]$. Then the center of mass $c[\bar\eta_{c(t)}]$ is matched to that of $\eta$, or
$$ \int_X \big(\eta - \bar\eta_{c(t)}\big) \: \dr =0.$$
Consequently, 
\begin{align*}
 &\E[\eta|\bar\eta] = \frac{1}{2}\iint [\eta-\bar\eta]^2\:\drr = \frac{1}{2}\iint [\eta-\bar\eta_{c(t)}]^2\:\drr\\
 &= \frac{1}{2} \iint_{X\times X} \big(\eta(x)-\bar\eta_{c(t)}(x)\big)^2 + \big(\eta(x')-\bar\eta_{c(t)}(x')\big)^2 \\
 & \quad \quad \quad - 2\big(\eta(x)-\bar\eta_{c(t)}(x)\big)\big(\eta(x')-\bar\eta_{c(t)}(x')\big)\:\drr\\
 &= \int_X \big(\eta(x)-\bar\eta_{c(t)}\big)^2 \:\dr.
\end{align*}
In conclusion, the relative energy $\H[\eta,m|\bar\eta,\bar{m}] = \K[m|\bar{m}] + \E[\eta|\bar\eta]$ measures $L^2$ differences of weak and strong states after the latter being shifted to match the center of masses. More specifically,
$$ \H[\eta,m|\bar\eta,\bar{m}] = \frac{1}{2} \|v - \bar{v}\|_{L^2_{\rho_0}}^2 + \|\eta - \bar\eta_{c(t)}\|_{L^2_{\rho_0}}^2.$$

Now, we calculate the relative energy inequality.
\begin{theorem} \label{ineqthm2}
 Let $\eta$ and $\bar\eta$ be respectively a weak and strong solution on $I=(0,T)$. Then $\H[\eta,m|\bar\eta, \bar{m}](t)$ continuously extends to the endpoints and 
 \begin{enumerate}
  \item if $\eta(t,\cdot)$ for each $t\in I$ is strictly monotone,
    \begin{equation} \label{relineq2}
            \H[\eta, m|\bar{\eta}, \bar{m}](T) - \H[\eta, m| \bar{\eta}, \bar{m}](0) \le 0,
    \end{equation}
  \item otherwise, 
    \begin{equation} \label{relineq3}
    \begin{aligned}
            &\H[\eta, m|\bar{\eta}, \bar{m}](T) - \H[\eta, m| \bar{\eta}, \bar{m}](0) \\
            & \leq  2\int_0^T\int_{\{(x,x')~|~\eta(t,x)=\eta(t,x'), ~x>x'\}} \big(\bar{v}(t, x) - \bar{v}(t, x')\big)\: \drr dt.
    \end{aligned}
    \end{equation}
  \end{enumerate}
\end{theorem}

\begin{proof}
Similarly to \eqref{firsteqn},
\begin{align*}
        -\int_I &\H[\eta, {m} | \bar{\eta}, {\bar{m}}] \dot\theta(t)\:dt \\
        & \le \int_I\iint_{X \times X} D{K}\big([\bar{\eta}](x, x')\big)\big([\eta](x, x') - [\bar{\eta}](x, x')\big)\dot\theta \: \drr dt\\
        &  + \int_I\int_X ({m}-\bar{{m}})\bar{v}\dot\theta dxdt.
\end{align*}
Since $\bar\eta$ is a strong solution, we can write
\begin{align*}
        \int_I \iint_{X\times X} D{K}_r\big([\bar{\eta}]\big)[\eta-\bar{\eta}]\dot\theta \: \drr dt &= \int_I \iint_{X\times X} \sgn(x-x')\dot{\theta}[\eta - \bar\eta] \:\drr dt,\\
        \int_I \iint_{X\times X} D{K}_a\big([\bar{\eta}]\big)[\eta-\bar{\eta}]\dot\theta \: \drr dt &= \int_I \iint_{X\times X} \Big(\big(\theta[\bar\eta]\big)^\bdot-\theta[\bar v]\Big)[\eta - \bar\eta] \:\drr dt,\\
        \int_I\int_X ({m}-\bar{{m}})\bar{v}\dot\theta dxdt &= \int_I\int_X \Big(\big(\theta \bar{v}\big)^\bdot - \theta \dot{\bar{v}}\Big)({m}-\bar{{m}}) \: dxdt.
\end{align*}
Similarly to \eqref{secondeqn}, we have cancelation
\begin{align*}
 \int_I\int_X -\theta \dot{\bar{v}}(m-\bar m) \:dx dt +\int_I\iint_{X\times X} \big(\theta \sgn(x-x') + \theta[\bar\eta] \big)^\bdot [\eta-\bar\eta] \:\drr dt = 0.
\end{align*}
% \begin{align*}
%  \int_I\iint_{X\times X} \big(\theta DK([\bar\eta])\big)^\bdot [\eta-\bar\eta] \:\drr dt &= \int_I\iint_{X\times X} -\big(\theta DK([\bar\eta])\big) [\dot\eta-\dot{\bar\eta}] \:\drr dt.
% \end{align*}
Remaining terms are
\begin{align*}
 &\int_I\int_X \big(\theta \bar{v}\big)^\bdot ({m}-\bar{{m}}) \:dxdt -\int_I\iint_{X\times X}\theta [\bar v][\eta-\bar\eta] \:\drr dt\\
 &= \int_I \theta \int_{X\times X} [\bar v] \Big( DK([\eta]) - DK([\bar\eta]) - [\eta-\bar\eta]\Big) \: \drr dt\\
 &=\int_I \theta \int_{X\times X} [\bar v] \big( -\sgn([\eta]) + \sgn([\bar\eta])\big)\: \drr dt\\
%  &=\int_I \theta \int_{\{\eta(x)=\eta(x')\}} \big(\bar{v}(x) - \bar{v}(x')\big)\sgn(x-x')\: \drr dt. %\\
  &=2\int_I \theta \int_{\{\eta(x)=\eta(x'), x>x'\}} \big(\bar{v}(x) - \bar{v}(x')\big)\: \drr dt.
\end{align*}
Simlarly as in the proof of \eqref{ineqthm}, we obtain the result.
\end{proof}

\subsection{Applications of relative energy inequality}
\subsubsection{Uniform stability before delta shock}
\begin{theorem}
 Suppose $\eta$ and $\bar\eta$ are weak and strong solutions and $\bar\eta_{c(t)}$ be the translation $\bar\eta(t,\cdot) -c[\bar\eta(t,\cdot)] + c[\eta(t,\cdot)]$ for $t\in I$. If  for $t\in (0,T)$ $\eta(t,\cdot)$ is strictly monotone,
 $$\int_X \frac{|v-\bar v|^2}{2} + |\eta - \bar\eta_{c(t)}|^2 \:\dr\bigg|_{t} \le \int_X \frac{|v-\bar v|^2}{2} + |\eta - \bar\eta_{c(0)}|^2 \:\bigg|_{t=0}.$$
\end{theorem}
\subsubsection{Linear growth of $L^2$ distance after delta shock}
\begin{theorem}
 Suppose $\eta$ and $\bar\eta$ are weak and strong solutions and $\bar\eta_{c(t)}$ be the translation $\bar\eta(t,\cdot) -c[\bar\eta(t,\cdot)] + c[\eta(t,\cdot)]$ for $t\in I$. 
 $$\int_X \frac{|v-\bar v|^2}{2} + |\eta - \bar\eta_{c(t)}|^2 \:\dr\bigg|_{t} \le \int_X \frac{|v-\bar v|^2}{2} + |\eta - \bar\eta_{c(0)}|^2 \:\bigg|_{t=0} + C_0t.$$ 
\end{theorem}
\subsubsection{Large friction limit}
In this section we study the limit for $\eps \rightarrow 0$ of solutions $(x^\eps,m^\eps)$ of the damped Euler-Poisson system:
\begin{equation}\label{eq:hamsca}
 \begin{split}
  \dot{x}^\eps &= \frac{1}{\eps} \frac{\delta H}{\delta m}(x^\eps,m^\eps), \\
  {\dot{ m}}^\eps &= -\frac{1}{\eps}\frac{\delta H}{\delta x}(x^\eps,m^\eps) - \frac{1}{\eps^2} m^\eps
 \end{split}
\end{equation}
with Hamiltonian given by 
\begin{equation}
\begin{split}
 H[x, \mathbf{m}] &=  \int \frac{| \mathbf{m} |^2}{2\rho_0(X)} \; dX  \;+\; \E[x],\\
\E[x]&= \frac{1}{2} \iint \rho_0(X)\mathcal{K}(x(X) - x(Y)) \rho_0(Y) \; dX\,dY.
 \end{split}
\end{equation}
In view of $\frac{\delta H}{\delta m}(x^\eps,m^\eps)=v^\eps$ a formal asymptotic analysis indicates that $x^\eps$ converges to
a motion $\bx$ satisfying the gradient flow equation
\begin{equation}\label{eq:gf}
 \dot {\bar x} = -\frac{1}{\rho_0}\frac{\delta \E}{\delta x}(\bar x)
\end{equation}
and $v^\eps$ converges to $0$.
The system \eqref{eq:gf} is already mentioned in \cite[Rem. 4.3]{CCZ}. Well-posedness and long time behavior of this system were investiagted in \cite{BLL12}.\footnotemark[1]
\footnotetext[1]{Note that in this model the change of sign of $\rho$ corresponds to a change of sign in $W$. Thus, the case at hand is similar to the case of NEGATIVE densities in \cite{BLL12}.}
Note that the limit system \eqref{eq:gf} is independent of $\eps$ such that its solution $\bx$ is also independent of $\eps.$

The purpose of the rest of this section is to rigorously prove this convergence (under the assumption that a smooth solution to \eqref{eq:gf} exists).
We will also assume that $\rho_0$ is bounded from below by a positive constant.
To this end we follow the methodology exhibited in \cite{LTforth}.
We consider the pair $(\bar x, \bar m)$ with 
\begin{equation}
 \bar m:= - \eps  \frac{\delta \E}{\delta x}(\bar x),
\end{equation}
 i.e. $\bar v:= - \frac{\eps}{\rho_0}  \frac{\delta \E}{\delta x}(\bar x).$
Note that $\bar m$ does, in fact, depend on $\eps$ and satisfies $\bar m = \mathcal{O}(\eps).$
Let $L^2_{\rho_0}(\Omega_0)$ denote the space with scalar product
\[ f,g \mapsto \int_{\Omega_0} \rho_0(X) f(X) g(X) \, dX.\]

\begin{theorem}
 Let $\bx$ be a smooth solution to \eqref{eq:gf} and let $(x^\eps,m^\eps)_{\eps>0}$ be a family of (dissipative) solutions to \eqref{eq:hamsca}.
 Let the initial data satisfy 
 \begin{equation}
 \label{eq:condini}
 \begin{split}
   \| \bx_0 - x_0^\eps\|_{L^2_{\rho_0}(\Omega_0)} = \mathcal{O}(\eps^2), \quad | c[\bx_0] - c[x_0^\eps] | = \mathcal{O}(\eps^2),\\
  \Big\|v_0^\eps + \frac{\eps}{\rho_0} \frac{\delta \E}{\delta x}(\bx_0)\Big\|_{L^2_{\rho_0}(\Omega_0)} =\mathcal{O}(\eps^2), %\quad \int_{\Omega_0} \rho_0(X) v_0^\eps(X) \, dX =\mathcal{O}(\eps^2)
   \end{split}
 \end{equation}
 then for any $T>0$ it holds
 \[ \| \bx(\cdot,t) - x^\eps(\cdot,t) \|_{L^2_{\rho_0}(\Omega_0)}=\mathcal{O}(\eps^2) \quad \text{and} \quad  \| v^\eps(\cdot,t) \|_{L^2_{\rho_0}(\Omega_0)}=\mathcal{O}(\eps)\]
 uniformly for $0 \leq t \leq T.$
\end{theorem}

\begin{proof}

We insert $(\bar x, \bar m)$ into the scaled Hamiltonian system \eqref{eq:hamsca} and observe that it satisfies
\begin{equation}\label{eq:hamscap}
 \begin{split}
  \dot{\bar x} =-\frac{1}{\rho_0}\frac{\delta \E}{\delta x}(\bar x) = \frac{1}{\eps}\bar v=  \frac{1}{\eps} \frac{\delta H}{\delta m}(\bx,\bar m), \\
  {\dot{ \bar m}} =  R =  -\frac{1}{\eps}\frac{\delta H}{\delta x}(\bar x,\bar m) - \frac{1}{\eps^2} \bar m + R
 \end{split}
\end{equation}
with $R:= \dot{ \bar m}= -\eps \frac{\delta^2 \mathcal{E}}{\delta x^2}(\bar{x}) \dot{\bar{x}}=\mathcal{O}(\eps)$, i.e. $(\bar x, \bar m)$ satisfies a perturbed version of \eqref{eq:hamsca} with residual proportional to $\eps.$


Now we compute the time evolution of the relative entropy between $(x^\eps,m^\eps)$ and $(\bar x, \bar m)$ using $\frac{\delta^2 H}{\delta x \delta m}=0$:
\begin{align*}
 \frac{d}{dt} H(x^\eps, m^\eps| \bar{x}, \bar{m }) &= \frac{d}{dt} H^\eps - \frac{d}{dt} \bar{H} - \frac{d}{dt} \langle \frac{\delta \bar{H}}{\delta x}, x^\eps -\bar{x}\rangle 
 - \frac{d}{dt} \langle \frac{\delta \bar{H}}{\delta m}, m^\eps-\bar{ m }\rangle \\
 &= \frac{\delta {H}^\eps}{\delta x}\dot x^\eps + \frac{\delta H^\eps}{\delta m}\dot m^\eps- \frac{\delta \bar{H}}{\delta x}\dot{\bar{x}} - \frac{\delta \bar{H}}{\delta {m}}\dot{\bar{ {m}}} \\
 &- \frac{\delta \bar{H}}{\delta x}\dot{{x}^\eps} + \frac{\delta \bar{H}}{\delta x}\dot{\bar{x}} - \frac{\delta^2 \bar{H}}{\delta x^2}\dot{\bar{x}} (x^\eps-\bar{x})  \\
 &-\frac{\delta \bar{H}}{\delta {m}}\dot{{ {m}^\eps}} + \frac{\delta \bar{H}}{\delta {m}}\dot{\bar{ {m}}}
 - \frac{\delta^2 \bar{H}}{\delta {m}^2}\dot{\bar{ {m}}} ( {m}^\eps-\bar{ {m}}).
 \end{align*}
 Inserting the evolution equations we get
 \begin{align*}
 \frac{d}{dt} H(x^\eps, m^\eps| \bar{x}, \bar{m })  
 &=\frac{1}{\eps}\frac{\delta {H}^\eps}{\delta x} v^\eps + v^\eps \big( -\frac{1}{\eps}\frac{\delta H^\eps}{\delta x} - \frac{1}{\eps^2} m^\eps\big) 
 -\frac{1}{\eps} \frac{\delta \bar{H}}{\delta x}\bar v - \bar v \big( -\frac{1}{\eps}\frac{\delta \bar H}{\delta x} - \frac{1}{\eps^2} \bar m + R\big) \\
 &- \frac{1}{\eps}\frac{\delta \bar{H}}{\delta x}v^\eps + \frac{\delta \bar{H}}{\delta x}\frac{1}{\eps} \bar v - \frac{\delta^2 \bar{H}}{\delta x^2}\frac{1}{\eps} \bar v (x^\eps-\bar{x}) 
 -\bar v\big( -\frac{1}{\eps}\frac{\delta H^\eps}{\delta x} - \frac{1}{\eps^2} m^\eps\big)\\
 &  + \bar v\big( -\frac{1}{\eps}\frac{\delta \bar H}{\delta x} - \frac{1}{\eps^2} \bar m + R\big)
 - \frac{1}{\rho_0}\big( -\frac{1}{\eps}\frac{\delta \bar H}{\delta x} - \frac{1}{\eps^2} \bar m + R\big)( {m}^\eps-\bar{ {m}})\\
 %
 &= \int - \frac{\rho_0}{\eps^2} (v^\eps - \bar v)^2
+ \frac{1}{\eps} \bar v\big( \frac{\delta H^\eps}{\delta x}  - \frac{\delta  \bar H}{\delta x}- \frac{\delta^2 \bar{H}}{\delta x^2} (x^\eps-\bar{x})  \big)  
- R( {v}^\eps-\bar{ {v}}) .
\end{align*}

Using \eqref{d3h} we infer
 \begin{equation}\label{eq:reg}
  \frac{d}{dt} H(x^\eps, m^\eps| \bar{x}, \bar{m })  =  - \int \frac{\rho_0}{\eps^2} (v^\eps - \bar v)^2 + R( {v}^\eps-\bar{ {v}})\; dX \\
  \leq \int \frac{\eps^2}{\rho_0} R^2 \; dX = \mathcal{O}(\eps^4).
 \end{equation}
 
 Assumption \eqref{eq:condini} ensures that $H(x^\eps, m^\eps| \bar{x}, \bar{m })|_{t=0} =\mathcal{O}(\eps^4),$
 such that \eqref{eq:reg} implies $H(x^\eps, m^\eps| \bar{x}, \bar{m })= \mathcal{O}(\eps^4)$ for all finite times.
  Since the relative energy is an upper bound for the squared $L^2_{\rho_0}$ norm of the difference $v^\eps - \bar v$ we have proven  the estimate on $v^\eps$ asserted by the theorem.
 Since the relative energy is an upper bound for the squared $L^2_{\rho_0}$ norm of the difference $x^\eps - \bx$ up to a shift, the estimate on $x^\eps- \bx$ asserted by the theorem is proven once we
 control the difference in center of mass between $x^\eps$ and $\bar x.$
 
 Condition \eqref{eq:condini} ensures that the initial centers of mass of $x^\eps$ and $\bar x$ are $\mathcal{O}(\eps^2)$ close.
 According to \cite[eq. (2.3)]{CCZ} the average velocity of the damped Hamiltonian solution decreases exponentially with decay rate $\eps^{-2}$. Thus,  the center of mass of $x^\eps$
 moves by, at most, $\mathcal{O}(\eps^2)$ for any finite initial average velocity.
 Thus, it remains to show that the center of mass of $\bar x$ does not move. 
 This follows from
 \begin{multline}
  \frac{d}{dt} \int_{\Omega_0} \rho_0(X) \bx(X,t) \, dX = \int_{\Omega_0} \rho_0(X) \dot{\bx}(X,t) \, dX = \int_{\Omega_0} \frac{\delta \E}{\delta x}(\bar x) \, dX
  \\
  = \int_{\Omega_0} \rho_0(X) \nabla_x (W * \rho) (X,t) \, dX
  =  \int_{\Omega(t)} \rho(x,t) W'(x-y) \rho(y,t)\, dx\; dy =0,
 \end{multline}
where we have used that $W'$ is odd.
 

 \end{proof}



\end{document}
\subsubsection{Uniform stability of pressureless case with $p=1$ and $q=2$}
In this section, pressureless system with $p=1$, $q=2$ is investigated, to establish the uniform stability for any pair of weak and strong solutions. Throughout this section, $\E[\eta] = \E_r[\eta] + \E_a[\eta]$.


\begin{theorem}[Uniform stability]
    If $\E = E_r + E_a$, $p=1$ and $q=2$, then 
\begin{align} \label{decay2}
 \H[\eta,m|\bar\eta,\bar{m}](t) \le \H[\eta,m|\bar\eta,\bar{m}](0).  
\end{align}
%     \begin{equation}
%         \left.H[\eta, {m}| \bar{\eta}, \bar{{m}}]\right|_{\tau = t} \leq \left.H[\eta, {m}| \bar{\eta}, \bar{{m}}]\right|_{\tau=0}.
%     \end{equation}    
\end{theorem}
\begin{proof}
 \eqref{relK} shows that if $p=1$ and $q=2$, $[\bar v]K'_r(\eta|\bar\eta) = [\bar v]K'_r(\eta|\bar\eta) = 0$.
\end{proof}




{\blue

In case $p=1$, we have $K_r(y) = -|y| = \frac{1}{2}\Phi(y)$ with $-\partial^2_{yy} \Phi(y) = \delta_0$. We also have if $q=2$, $-\partial^2_{yy} K_a(y) \equiv -1$. Therefore, for $K(y) = K_r(y) + K_a(y)$
\begin{align*}
 c(y) = \int_\mathbb{R} K(y-z) \rho(z)\: dz
\end{align*}
solves the equation
\begin{align*}
 -\partial^2_{yy} c(y) = \rho(y) - (\rho), \quad (\rho):=\int_\mathbb{R} \rho(y) \:dy.
\end{align*}



0. regularity issues on $\eta$.

1. $\E_r[\eta|\bar\eta]$ is identically zero.

2. $L^2$ difference with center of mass shift.

Let us divide $\K(x) = \K_-(x) + \K_+(x)$, where $\K_-(x) = -|x|$ and $\K_+(x) = \frac{|x|^2}{2}$. The contribution from the first part is
\begin{align*}
  &\frac{1}{2}\iint \rho_0(X)\Big[ -|x(X)-x(Y)| + |\bx(X) - \bx(Y)| \\
  &\qquad \qquad \qquad+\sgn(\bx(X)-\bx(Y))\Big(x(X)-x(Y)-\bx(X) + \bx(Y)\Big)\Big]\rho_0(Y)  \; dY \; dX\\
  &= \frac{1}{2}\iint \rho_0(X) \big(x(X)-x(Y))\Big( \sgn(\bx(X) - \bx(Y)) - \sgn(x(X)- x(Y) \Big)\rho_0(Y) \; dY \; dX \\
  &=0
\end{align*}
\medskip\noindent{\bf Relative (hamiltonian) energy}
\begin{align}
H (x, \mathbf{m}| \bar{x}, \bar{ \mathbf{m} }) = \int \frac{| \mathbf{m} -\bm|^2}{2\rho_0(X)} \, dX + \frac{1}{4}\iint \rho_0(X)\Big(x(X)-x(Y)-\bx(X) + \bx(Y)\Big)^2 \rho_0(Y) \; dY\; dX. \label{eq:relH}
\end{align}


2. 
  Let us start by explaining how the relative (potential) energy between two maps $x, \bx: \Omega \rightarrow \mathbb{R}$
 \begin{equation}
 \E [x,\bx]= \frac{1}{4}\iint \rho_0(X)\Big(x(X)-x(Y)-\bx(X) + \bx(Y)\Big)^2 \rho_0(Y) \; dY\; dX
 \end{equation}
 controls the difference between these maps.
 The first observation is that $\E[x,\bx]=0$ allows for $x$ and $\bx$ differing by an additive constant.
 Let us now define the center of mass of $x$ as 
 \[ c[x]:= \frac{\int_{\Omega_0} x(X) \rho_0(X) \, dX}{\int_{\Omega_0}  \rho_0(X) \, dX} \]
 and a shifted version of $\bx$ by
 \[ \tx(X) = \bx (X) -c[\bx] + c[x]\] which has the same center of mass as $x$, i.e.,
 \begin{equation}\label{eq:eqcent} \int_\Omega (\tx (X) - x(X)) \rho_0(X) \, dX =0.\end{equation}
 Note that
 \begin{multline}
  \E[x,\bx]=\E[x,\tx]\\= \frac{1}{4}\iint \rho_0(X)\Big[ \big(x(X)- \tx(X)\big)^2 + \big(x(Y)- \tx(Y)\big)^2 - 2 \big(x(Y)- \tx(Y)\big)\big(x(X)- \tx(X)\big) \Big] \rho_0(Y) \; dY\; dX\\
  \stackrel{\eqref{eq:eqcent}}{=}\frac{1}{2}\int \rho_0(X)\big(x(X)- \tx(X)\big)^2 \; dX \int \rho_0(Y) \; dY.
 \end{multline}
Thus, $\E[x,\bx]$ controls a (squared) weighted  $L^2$-distance between the maps up to an additive shift.

% For motions $x: \Omega \times [0,T) \rightarrow \mathbb{R}$ it holds
% \begin{equation}
% \frac{d}{dt} c [x(\cdot,t)] = \frac{\int_{\Omega_0} m(X,t) \, dX}{\int_{\Omega_0}  \rho_0(X) \, dX}=  \frac{\int_{\Omega_0} m_0(X) \, dX}{\int_{\Omega_0}  \rho_0(X) \, dX}.
% \end{equation}
% This means the center of mass is determined for all times by the initial data, i.e. by the initial center of mass and initial average velocity.
% Thus, for two motions with identical initial center of mass and initial average velocity vanishing of the relative entropy between them for all times implies that
% the motions coincide.



% {\blue
% \medskip\noindent{\bf Two different initial conditions}\footnotemark[1]\medskip
% }
% 
% We are now treating the problem as the one on the motion $x(X,t)$ and the momentum $\mathbf{m}(X,t)$. Of course, one can derive the continuity equation for $\rho$ from $\dot{x} = \mathbf{v}$.
% 
% If we compare two solutions and if two initial motions $x(X,0)$ and $\bx(X,0)$ are at our disposal, which are not necessarily identified with the coordinate $X$ in this case,
% we can compute the $\rho(X,0)$ and $\bar\rho(X,0)$ by taking the derivative in $X$ and the determinant in turn.\footnotemark[2]
% 
% Now we consider the opposite direction. Suppose we are provided with two initial densities $\rho(X,0)$ and $\bar\rho(X,0)$.
% We don't complicate on $X$, the referential coordinate is always shared, otherwise we lose the point using the referential description.
% However, can we always find the referential coordinate $X$ such that the referential density $\rho_0(X)$ is also shared, i.e.,
% $$ \rho_0(X) = \frac{\rho(X,0)}{\det F(X,0)} = \frac{\bar\rho(X,0)}{\det \bar{F}(X,0)} \,?$$
% In general, for the pre-assigned metric we can't find a coordinate chart that matches to the given metric matrix, but here it is only required to match the determinant, so this is for sure an underdetermined problem.
% So I would think it is in general possible even in multi-D. But it needs to be clarified before we advance our relative energy techniques.
% 
% \footnotetext[1]{I do not really understand the point you are making here. From my perspective $\rho_0$ is not really a  property of the initial motion $x(\cdot,0)$ but of the mass distribution of the undeformed body in reference coordinates.
% Thus, I would always assume $\rho_0=\bar \rho_0.$ }
% 
% 
% \footnotetext[2]{ I think we can only compute $\rho(X,0)$ from $x(X,0)$ provided we know $\rho_0(X).$}
% \medskip\noindent{\bf Time evolution of the relative energy}
\medskip
\medskip\noindent{\bf Relative energy identity}\footnotemark[1]
\footnotetext[1]{HAS TO BE REVISITED ALONG WITH THE TODOLIST.}




\subsubsection{Large friction limit of pressureless case with $p=1$ and $q=2$}

Here, we are standing on the previous work of Carrillo, Choi and Zatorska \cite{CCZ}. As was explained in their paper, this problem goes well with the Lagrangian description in the following sense:
\begin{itemize}
 \item The initial data is compactly supported in $\Omega(t=0)$ and is so (or turns out to be so) at all $t>0$. In Eulerian coordinate system we may treat this problem either 1) as a free boundary problem with evolving domain $\Omega(t)$, or 2) as a problem in whole domain but with the vacuum region where $\rho=0$, either of which is not easy to handle.
 \item Due to pressureless nature we use the relative energy formula \eqref{eq:relHx} in Lagrangian description, and \eqref{eq:relHx} has some advantages in manipulating terms.
\end{itemize}

%Now, we go for the calculation.

\subsection{Relative energy for the pressureless Euler-Poisson in 1D}
%In $\mathbb{R}^d$ for any $d\ge 1$, the power functions $|x|^p$ with the negative exponent $p<0$ is not convex in $x$ and is only convex in the radial direction. 
As was in the free energy density $h(\rho)$, the convexity of the kernel $\mathcal{K}(x)$ in the energy formula is crucial in developing the relative energy theory.


\begin{remark}[Properties of the kernel]
\begin{enumerate}
\item
We divide the kernel into two parts, $\mathcal{K}(x)  = \gamma\Phi(x) + \frac{|x|^2}{2} = \gamma\Phi(x) + \mathcal{K}_+(x)$, $\gamma>0$. For instance, we set $\gamma\Phi(x) = -|x|$ for 1D, $-\log(|x|)$ for 2D, $|x|^{2-d}$ for higher dimensions. 

Note that $K_+(x)$ is convex whereas $\gamma\Phi(x)$ is neither convex nor concave, except for the 1D case, where $-|x|$ is a concave function. In any case, the presence of $\gamma\Phi(x)$ makes the whole kernel be neither convex nor concave.

\item The situation is similar for kernels of type $-\frac{|x|^p}{p} + \frac{|x|^q}{q} = K_-(x) + K_+(x)$, $p,q>0$, where one studies the competition between the attracting mechanism out of the positive contribution and the repelling mechanism out of the negative contribution. %Again, the kernel is neither convex nor concave.

\item This fact indicates that the relative energy would not have a definite sign in both of examples.\\
We have no idea about how to incorporate those non-convex contributions in the relative energy framework. 

\item However, the special case of the linear non-convex contribution $-|x|$ in 1D is incorporated into the framework, and we detail it in the below.
\end{enumerate}
\end{remark}

A crucial observation is 
\begin{equation} \det F = F = \frac{\partial x}{\partial X}>0 \Longrightarrow \sgn(X-Y)= \sgn\big(x(X,t) - x(Y,t)\big), \quad \forall t>0,\forall (X,Y) \label{eq:obser}
\end{equation}
such that the part $-|x|$ of the kernel does not contribute any in the relative energy expression. Thus the relative energy happens to have the definite sign. We proceed to calculate this in the below. Let $K$ be the kinetic energy and $\E$ be the rest of the energy so that $H = K + \E$.

% There are properties that only apply in 1D that turn out to play the crucial roles here, enabling us to study the relative energy even with
% \begin{enumerate}
%  \item $|x|^p$, \quad $p<0$
%  \item $-|x|$, the concave linear kernel.
% \end{enumerate}
% The latter comes with importance here because it corresponds to the the fundamental solution contribution in the kernel in 1D. All of these are due to the observation that for any two motions $x$ and $\bar{x}$,
% % \begin{enumerate}
% %  \item
% due to $\det F = F = \frac{\partial x}{\partial X}$, and the assumption \eqref{eq:A1}, the order of two points $X$ and $Y$ is preserved, i.e.
%  $$ \sgn(X-Y) = \sgn\big(x(X,t) - x(Y,t)\big) = \sgn\big(\bar{x}(X,t) - \bar{x}(Y,t)\big). $$
% % \end{enumerate}
% Anyhow, we proceed to calculate the relative energy. 


In conclusion, if $(x,\mathbf{m})$ and $(\bx, \bm)$ are two strong solutions, then the relative energy is conserved.
If $(x,\mathbf{m})$ is a dissipative weak solution with decreasing energy and $(\bx, \bm)$ is a smooth solution, then the relative energy decreases as time proceeds.



\subsection{Consequences of the conservation (decay) of relative energy}
The conservation (decay) of the relative energy leads to the following conclusions.


\begin{lemma}[Weak-Strong uniqueness] \label{lem:WS-Uniq}If initial motions and momentums coincide, i.e., $x(X,0)=\bx(X,0)$ and $\mathbf{m}(X,0)=\bm(X,0)$, then $x=\bx$ and $\mathbf{m}=\bm$ for $t>0$.

Furthermore, if initial motions and momentums are such that, $x(X,0)=\bx(X,0)+E_0$ and $\mathbf{v}(X,0)=\mathbf{\bar{v}}(X,0)+F_0$, then $x(X,t)=\bx(X,t) + E_0 + F_0t$ and $\mathbf{v}(X,t)=\mathbf{\bar{v}}(X,t) + F_0$ for $t>0$.
 \end{lemma}

\begin{proof}
  Let us start by explaining how the relative (potential) energy between two maps $x, \bx: \Omega \rightarrow \mathbb{R}$
 \begin{equation}
 \E [x,\bx]= \frac{1}{4}\iint \rho_0(X)\Big(x(X)-x(Y)-\bx(X) + \bx(Y)\Big)^2 \rho_0(Y) \; dY\; dX
 \end{equation}
 controls the difference between these maps.
 The first observation is that $\E[x,\bx]=0$ allows for $x$ and $\bx$ differing by an additive constant.
 Let us now define the center of mass of $x$ as 
 \[ c[x]:= \frac{\int_{\Omega_0} x(X) \rho_0(X) \, dX}{\int_{\Omega_0}  \rho_0(X) \, dX} \]
 and a shifted version of $\bx$ by
 \[ \tx(X) = \bx (X) -c[\bx] + c[x]\] which has the same center of mass as $x$, i.e.,
 \begin{equation}\label{eq:eqcent} \int_\Omega (\tx (X) - x(X)) \rho_0(X) \, dX =0.\end{equation}
 Note that
 \begin{multline}
  \E[x,\bx]=\E[x,\tx]\\= \frac{1}{4}\iint \rho_0(X)\Big[ \big(x(X)- \tx(X)\big)^2 + \big(x(Y)- \tx(Y)\big)^2 - 2 \big(x(Y)- \tx(Y)\big)\big(x(X)- \tx(X)\big) \Big] \rho_0(Y) \; dY\; dX\\
  \stackrel{\eqref{eq:eqcent}}{=}\frac{1}{2}\int \rho_0(X)\big(x(X)- \tx(X)\big)^2 \; dX \int \rho_0(Y) \; dY.
 \end{multline}
Thus, $\E[x,\bx]$ controls a (squared) weighted  $L^2$-distance between the maps up to an additive shift.

For motions $x: \Omega \times [0,T) \rightarrow \mathbb{R}$ it holds
\begin{equation}
\frac{d}{dt} c [x(\cdot,t)] = \frac{\int_{\Omega_0} m(X,t) \, dX}{\int_{\Omega_0}  \rho_0(X) \, dX}=  \frac{\int_{\Omega_0} m_0(X) \, dX}{\int_{\Omega_0}  \rho_0(X) \, dX}.
\end{equation}
This means the center of mass is determined for all times by the initial data, i.e. by the initial center of mass and initial average velocity.
Thus, for two motions with identical initial center of mass and initial average velocity vanishing of the relative entropy between them for all times implies that
the motions coincide.

To prove the second assertion, set the new strong pair $(\tilde{x}, \mathbf{\tilde{v}})$, 
 $$ \tilde{x}(X,t) = \bx(X,t) + E_0 + F_0t, \quad \mathbf{\tilde{v}}(X,t) = \dot{\bx} + F_0 = \mathbf{\bar{v}}+F_0.$$
 Note that initially $x=\tilde{x}$, $\mathbf{v}=\mathbf{\tilde{v}}$ and that ($\tx,  \mathbf{\tilde{v}})$ is again a solution, since
 \begin{equation}
 \begin{split}
  \dot{\tx}(X,t) &= \dot{\bx}(X,t) + F_0 = \mathbf{\tilde{v}}(X,t)\\
  \rho_0 \mathbf{\dot{\tilde{v}}} &=\rho_0  \mathbf{\dot{\bar{v}}} = - \frac{\delta H}{\delta x} (\bx) = - \frac{\delta H}{\delta x}(\tx).
  \end{split}
 \end{equation}
The proof follows from the first assertion.
\end{proof}







    where $C$ is a positive constant.

For the proof of Theorem \ref{ineqthm}, we need the following proposition.
\begin{proposition}\label{plaw}
    Let $\mathcal{K}:=\mathcal{K}_r + \mathcal{K}_a$.
    For each $x , x' \in X$
    \begin{equation}
    \begin{aligned}
        \bar{v}_x \psi'(\eta | \bar{\eta}) \le C \psi(\eta | \bar\eta), \quad
        [\bar{{v}}]{K}'(\eta | \bar{\eta}) \leq C {K}(\eta | \bar{\eta})
    \end{aligned}
    \end{equation}
    for some constant $C>0$ depending on $\Lip(\bar{v})$, $\Lip(\bar\eta^{-1})$, $\Lip(\eta^{-1})$, $p$, $q$, and $\gamma$.
\end{proposition}
\begin{proof}
    \begin{equation*}
        \begin{aligned}
        &[\bar{{v}}](x, x')\Big(\mathcal{K'}\big([\eta](x,x')\big) - \mathcal{K}'\big([\bar{\eta}](x, x')\big) - \mathcal{K}''\big([\bar{\eta}](x, x')\big)\big([\eta](x, x') - [\bar{\eta}](x, x')\big)\Big) \\
%         &=\bar{{v}}(x, x') \bigg( \int_0^1 \frac{d}{ds}\mathcal{K}'\Big(s\big(\eta(x, x')\big)+ (1-s)\big(\bar\eta(x, x')\big)\Big)ds  - \mathcal{K}''\big(\bar\eta(x, x')\big)\big(\eta(x, x')  - \bar\eta(x, x')\big)\bigg) \\
%         &=\bar{{v}}(x, x') \bigg( \int_0^1 \mathcal{K}''\Big(s\big(\eta(x, x')\big) + (1-s)\big(\bar\eta(x, x')\big)\Big)\big(\eta(x, x') - \bar{\eta}(x, x')\big)ds \\
%         & \quad - \mathcal{K}''\big(\bar\eta(x, x')\big)\big(\eta(x, x') - \bar\eta(x, x')\big)\bigg)\\ 
%         &= \bar{{v}}(x, x')\big(\eta(x, x') - \bar\eta(x, x')\big) \bigg( \int_0^1 \int_0^s \frac{d}{d\tau}\mathcal{K}''\Big(\tau\big(\eta(x, x')\big)+ (1-\tau)\big(\bar\eta(x, x')\big)\Big)d\tau ds  \bigg) \\
%         &=\bar{{v}}(x, x') \big(\eta(x, x') - \bar\eta(x, x')\big)^2 \bigg( \int_0^1 \int_0^s \mathcal{K}'''\Big(\tau\big(\eta(x, x')\big)+ (1-\tau)\big(\bar\eta(x, x')\big)\Big)d\tau ds \bigg) \\
        &= [\eta - \bar\eta]^2 \int_0^1 \int_0^s [\bar{v}] {K}'''\Big([\lambda \eta+ (1-\lambda)\bar\eta]\Big)d\tau ds, \quad \text{and let $\eta_\lambda = \lambda \eta + (1-\lambda)\bar\eta$}, \\
        &= [\eta - \bar\eta]^2 \int_0^1 \int_0^s \frac{[\bar{{v}}]}{[\eta_\lambda]} \Big(-(p-1)(p-2)|[\eta_\lambda]|^{p-2} + (q-1)(q-2)|[\eta_\lambda]|^{q-2} \Big) d\lambda ds.
        \end{aligned}
    \end{equation*}
    We have that 
    \begin{align*}
     \left|\frac{\bar{v}(x) - \bar{v}(x')}{\lambda\big(\eta(x) - \eta(x')\big) + (1-\lambda)\big( \bar\eta(x) - \bar\eta(x')\big)}\right| \le \Lip(\bar v) \big(\Lip(\eta^{-1}) + \Lip(\bar\eta^{-1})\big).
    \end{align*}
% 
%     For $\mathcal{K}_r$,
%     \begin{equation}
%         \begin{aligned}
%         & \int_0^1 \int_0^s \bar{{v}}(x, x') \big(\eta(x, x') - \bar\eta(x, x')\big)^2 \mathcal{K}_r'''\Big(\tau\eta(x, x')+ (1-\tau)\bar\eta(x, x')\Big)d\tau ds  \\
%         &= -(p-1)(p-2)\big(\eta(x, x') - \bar\eta(x, x')\big)^2  \\ 
%         & \quad \quad \int_0^1 \int_0^s \frac{\bar{{v}}(x, x')}{\tau\eta(x, x') + (1-\tau)\bar\eta(x, x')} |\tau\eta(x,x') + (1-\tau)\bar\eta(x, x')|^{p-2}d\tau ds \\
%         &\leq -(p-1)(p-2)\big(\eta(x, x') - \bar\eta(x, x')\big)^2  \\ 
%         & \quad \quad \int_0^1 \int_0^s \left| \frac{\bar{{v}}(x, x')}{\tau\eta(x, x') + (1-\tau)\bar\eta(x, x')} |\tau\eta(x,x') + (1-\tau)\bar\eta(x, x')|^{p-2}\right| d\tau ds \\
%         &\leq -C_{Lip}(p-1)(p-2)\big(\eta(x, x') - \bar\eta(x, x')\big)^2 \int_0^1 \int_0^s  |\tau\eta(x,x') + (1-\tau)\bar\eta(x, x')|^{p-2} d\tau ds. \\
%         \end{aligned}
%     \end{equation}
%     The last inequality holds because $\eta, \bar\eta \in \mathcal{A}$ so that
%     \begin{equation}
%         C_1 \leq \frac{\eta(x, x')}{x-x'} \leq C_2
%     \end{equation}
%     for some constants $C_1>0, C_2>0$, and $\bar{{v}}$ is Lipschitz.
%     Since
%     \begin{equation}
%         \mathcal{K}_r(\eta | \bar\eta) = -(p-1)\big(\eta(x, x') - \bar\eta(x, x')\big)^2 \int_0^1\int_0^s |\tau\eta(x,x') + (1-\tau)\bar\eta(x, x')|^{p-2} d\tau ds, 
%     \end{equation}
%     we get 
%     \begin{equation}
%         \bar{{v}}(x, x')\mathcal{K}'_r(\eta | \bar\eta) \leq C\mathcal{K}_r(\eta | \bar\eta)
%     \end{equation}
%     for some constant $C>0$. The case of $\mathcal{K}_a$ is similar.
%     Hence we get the desired result.
\end{proof}




\begin{proof}

Finally, if $I=(0,T)$, we choose 
\begin{equation} \label{theta}
    \theta(t) = 
    \begin{cases}
        \frac{t}{\epsilon} & 0 \leq t < \epsilon, \\
        1 & \epsilon \le t \le T-\epsilon,\\
%         -\frac{1}{\epsilon}(\tau-t-\epsilon) & t \leq \tau < t+\epsilon \\
        -\frac{1}{\epsilon}(T-t) & T-\epsilon \le t \le T,
    \end{cases}
\end{equation}

From \eqref{relineq},
\begin{equation}
    \begin{aligned}
        &\left.H[\eta, {m} | \bar{\eta}, \bar{{m}}]\right|_{\tau=t} - \left.H[\eta, {m} | \bar{\eta}, \bar{{m}}]\right|_{\tau=0} \\
        &\leq\int_0^t \iint_{X \times X}\big(\bar{{v}}(x) - \bar{{v}}(x')\big)\Big(\mathcal{K}'\big(\eta(x) - \eta(x')\big) - \mathcal{K}'\big( \bar{\eta}(x) - \bar{\eta}(x')\big) \\
        &\quad  - \mathcal{K}''\big(\bar{\eta}(x) - \bar{\eta}(x')\big)\big(\eta(x) - \eta(x') - \bar{\eta}(x) + \bar{\eta}(x')\big)  \Big)\rho_0(x)\rho_0(x')dxdx'd\tau.
    \end{aligned}
\end{equation}
By the previous lemma,
\begin{equation}
    \bar{{v}}(x, x')(D\mathcal{K})(\eta | \bar{\eta}) \leq C \mathcal{K}(\eta | \bar{\eta})
\end{equation}
for some constant $C>0$. Thus we get the result.
\end{proof}
% \begin{remark} 
%     Roughly speaking, this result may be interpreted as that the time evolution of relative Hamiltonian can be controlled by only potential energies.
% \end{remark}
% 
% \begin{corollary} 
%     \begin{equation}
%         H[\eta,{m}|\bar{\eta}, \bar{{m}}]\Big|_{\tau=t} \leq C H[\eta,{m}|\bar{\eta}, \bar{{m}}]\Big|_{\tau=0}.
%     \end{equation}
%     where $C$ is a positive constant depending on \textcolor{red}{what}.
% \end{corollary}
% \begin{proof}
%     By the previous result, the Gronwall lemma gives the result.
% \end{proof}








\begin{align*}
        & = H[\eta, {m}]\theta|_{\tau=0} - H[\bar{\eta}, \bar{m}]\theta|_{\tau=0} \\
        & \quad -\int_0^\infty\int_X \psi''(\bar{\eta}_x)\bar{\mathbf{v}}_x(\eta_x - \bar{\eta}_x)\theta\rho_0 dx d\tau - \int_0^\infty\int_X \psi'(\bar{\eta}_x)(\mathbf{v}_x - \bar{\mathbf{v}}_x)\theta \rho_0 dxd\tau \\
        & \quad - \left.\int_X\psi'(\bar{\eta}_x)(\eta_x - \bar{\eta}_x)\theta\rho_0(x)dx\right|_{\tau = 0} \\
        & \quad - \int_0^\infty\iint_{X \times X} D^2\mathcal{K}\big(\bar\eta(x, x')\big)\big(\bar{\mathbf{v}}(x, x')\big)\big( \eta(x, x') - \bar\eta(x, x') \big)\theta \rho_0(x)\rho_0(x')dxdx'd\tau \\
        & \quad - \int_0^\infty \iint_{X \times X} D\mathcal{K}(\bar\eta(x, x'))\big(\mathbf{v}(x, x') - \bar{\mathbf{v}}(x, x')\big)\theta \rho_0(x)\rho_0(x')dxdx'd\tau \\
        & \quad - \left.\iint_{X \times X} D\mathcal{K}\big(\bar\eta(x, x')\big)\big(\eta(x, x') - \bar\eta(x, x')\big)\theta \rho_0(x)\rho_0(x')dxdx'd\tau\right|_{\tau = 0} \\
        & \quad + \int_0^\infty\int_X \big(\psi'(\eta_x) - \psi'(\bar{\eta}_x)\big)\bar{\mathbf{v}}\theta dxd\tau \\
        & \quad + \int_0^\infty \iint_{X \times X} \big(D\mathcal{K}(\eta) - D\mathcal{K}(\bar\eta)\big)\big(\bar{\mathbf{v}}(x, x')\big)\theta\rho_0(x)\rho_0(x')dxdx'd\tau \\
        & \quad - \left.\int_X ({m} - \bar{{m}})(\bar{\mathbf{v}}\theta)dx \right|_{\tau = 0} - \int_0^\infty\int_X ({m} - \bar{{m}}) \dot{\bar{\mathbf{v}}}\theta dxd\tau \\
        & =  \left.H[\eta, {m}|\bar\eta, \bar{{m}}]\theta\right|_{\tau = 0} \\
        & \quad + \int_0^\infty \int_X \bar{\mathbf{v}}(D\psi)(\eta_x | \bar{\eta}_x)\theta\rho_0dxd\tau \\
        & \quad + \int_{0}^{\infty} \iint_{X \times X} \bar{\mathbf{v}}(x, x')(D\mathcal{K})(\eta | \bar\eta)\theta\rho_0(x)\rho_0(x')dxdx'd\tau.
\end{align*}


\begin{equation*}
    \begin{aligned}
        &-\int_0^\infty H[\eta, {m} | \bar{\eta}, \bar{m}] \dot\theta(\tau)d\tau \leq H[\eta, {m}]\theta|_{\tau=0} - H[\bar{\eta}, \bar{m}]\theta|_{\tau=0} \\ 
        & \quad + \int_0^\infty\left\langle \frac{\delta H}{\delta \eta}(\bar{\eta}), \eta - \bar\eta\right\rangle \dot{\theta}d\tau + \int_0^\infty \left\langle \frac{\delta H}{\delta {m}}(\bar{m}),{m} - \bar{m}\right\rangle \dot{\theta}d\tau \\
        & = H[\eta, {m}]\theta|_{\tau=0} - H[\bar{\eta}, \bar{m}]\theta|_{\tau=0} \\
        & \quad + \int_0^\infty\int_X \psi'(\bar{\eta}_x)(\eta_x - \bar{\eta}_x)\dot\theta\rho_0dxd\tau \\
        & \quad + \int_0^\infty\iint_{X \times X} D\mathcal{K}\big(\bar{\eta}(x, x')\big)\big(\eta(x, x') - \bar{\eta}(x, x')\big)\dot\theta \rho_0(x)\rho_0(x')dxdx'd\tau \\
        & \quad + \int_0^\infty\int_X \frac{\bar{{m}}}{\rho_0}({m}-\bar{{m}})\dot\theta dxd\tau \\
        & = H[\eta, {m}]\theta|_{\tau=0} - H[\bar{\eta}, \bar{m}]\theta|_{\tau=0} \\
        & \quad -\int_0^\infty\int_X \psi''(\bar{\eta}_x)\bar{\mathbf{v}}_x(\eta_x - \bar{\eta}_x)\theta\rho_0 dx d\tau - \int_0^\infty\int_X \psi'(\bar{\eta}_x)(\mathbf{v}_x - \bar{\mathbf{v}}_x)\theta \rho_0 dxd\tau \\
        & \quad - \left.\int_X\psi'(\bar{\eta}_x)(\eta_x - \bar{\eta}_x)\theta\rho_0(x)dx\right|_{\tau = 0} \\
        & \quad - \int_0^\infty\iint_{X \times X} D^2\mathcal{K}\big(\bar\eta(x, x')\big)\big(\bar{\mathbf{v}}(x, x')\big)\big( \eta(x, x') - \bar\eta(x, x') \big)\theta \rho_0(x)\rho_0(x')dxdx'd\tau \\
        & \quad - \int_0^\infty \iint_{X \times X} D\mathcal{K}(\bar\eta(x, x'))\big(\mathbf{v}(x, x') - \bar{\mathbf{v}}(x, x')\big)\theta \rho_0(x)\rho_0(x')dxdx'd\tau \\
        & \quad - \left.\iint_{X \times X} D\mathcal{K}\big(\bar\eta(x, x')\big)\big(\eta(x, x') - \bar\eta(x, x')\big)\theta \rho_0(x)\rho_0(x')dxdx'd\tau\right|_{\tau = 0} \\
        & \quad + \int_0^\infty\int_X \big(\psi'(\eta_x) - \psi'(\bar{\eta}_x)\big)\bar{\mathbf{v}}\theta dxd\tau \\
        & \quad + \int_0^\infty \iint_{X \times X} \big(D\mathcal{K}(\eta) - D\mathcal{K}(\bar\eta)\big)\big(\bar{\mathbf{v}}(x, x')\big)\theta\rho_0(x)\rho_0(x')dxdx'd\tau \\
        & \quad - \left.\int_X ({m} - \bar{{m}})(\bar{\mathbf{v}}\theta)dx \right|_{\tau = 0} - \int_0^\infty\int_X ({m} - \bar{{m}}) \dot{\bar{\mathbf{v}}}\theta dxd\tau \\
        & =  \left.H[\eta, {m}|\bar\eta, \bar{{m}}]\theta\right|_{\tau = 0} \\
        & \quad + \int_0^\infty \int_X \bar{\mathbf{v}}(D\psi)(\eta_x | \bar{\eta}_x)\theta\rho_0dxd\tau \\
        & \quad + \int_{0}^{\infty} \iint_{X \times X} \bar{\mathbf{v}}(x, x')(D\mathcal{K})(\eta | \bar\eta)\theta\rho_0(x)\rho_0(x')dxdx'd\tau.
    \end{aligned}
\end{equation*}
Choose a test function
\begin{equation} \label{theta}
    \theta(\tau) = 
    \begin{cases}
        1 & 0 \leq \tau < t \\
        -\frac{1}{\epsilon}(\tau-t-\epsilon) & t \leq \tau < t+\epsilon \\
        0 & \tau \geq t+\epsilon.
    \end{cases}
\end{equation}
Letting $\epsilon \rightarrow 0$, we arrive at the desired result.



\subsection{Relative energy estimate for weak solutions}
We proceed to compare a weak solution $(\eta, {m})$ with a strong solution $(\bar{\eta}, \bar{{m}})$ via a relative energy computation. Assume, for almost everywhere $t\in[0,\infty)$, monotonically increasing bi-Lipschitz maps $\eta(t,\cdot)$ such that
\begin{equation}
    \iint_{X \times X}|x-x'|^{p-1}(\eta(x)-\eta(x') - x+x')d\rho_0(x)d\rho_0(x') < \infty
\end{equation}
where $p>0$. We will write $\mathcal{C}$ to denote this class of maps $\eta$. Note that $\mathcal{C}$ is a cone, that is $\eta + \bar\eta \in \mathcal{C}$ and $a\eta \in \mathcal{C}$ whenever $\eta, \bar\eta \in \mathcal{C}$ and $a > 0$.

\begin{definition} 
    Let $\eta \in L_{loc}^1([0, \infty) \times X)$ and $\eta(t, \cdot) \in \mathcal{C}$ for almost everywhere $t \in [0, \infty)$. A function $\mathbf{v} \in L_{loc}^1([0,\infty) \times X)$ is the \textit{velocity} of the motion, written
    \begin{equation}
        \dot{\eta} = \mathbf{v},
    \end{equation}
    if
    \begin{equation}
        -\int_0^\infty\int_X \eta\dot{\phi}dxdt - \int_X\eta(0, x)\phi(0, x)dx = \int_0^\infty\int_X\mathbf{v}\phi dxdt
    \end{equation}
    for all $\phi\in C_c^1([0, \infty)\times X)$. Moreover, given the referential mass density $\rho_0$, the \textit{conjugate momenta} is ${m}:=\rho_0\mathbf{v}$.
\end{definition}
Here we consider only a motion map $\eta$ such that
\begin{equation}
        \eta(t, \cdot) \in \mathcal{C} \quad \text{for all} \ t \in [0, \infty)
\end{equation}
with the velocity field $\mathbf{v}=\dot{\eta}$ (in the sense of distribution).
We will write $\mathcal{A}$ to denote the class of this admissible motion maps $\eta$.

\begin{definition}
    Let $\eta \in \mathcal{A}$ and $\rho_0$ is the referential mass density. Suppose $\mathbf{v} \in \textcolor{red}{L^\infty([0, \infty);W^{1,1}(\rho_0, X))}$. A pair $\left( \eta, {m} \right)$ is called a \textit{weak solution} of \eqref{hamflow}, if it satisfies, for any \textcolor{red}{$\phi\in C^1_c([0,\infty) \times X)$}
    \begin{equation} \label{wksol}
        \begin{aligned}
        & -\int_{0}^{\infty}\int_{X}^{}{m}\dot{\phi}(t,x)dxdt - \int_{X}^{} {m}(0, x)\phi(0, x) dx \\
        &\quad = -\int_0^\infty \int_X (\gamma-1)(\eta_x)^{\gamma-2}\phi_x\rho_0(x)^r dx \\ & \quad\quad-  \int_{0}^{\infty}\iint_{X \times X}\operatorname{D}\mathcal{K}_r(\eta(x)-\eta(x'))(\phi(x)-\phi(x')) d\rho_0(x)d\rho_0(x')dt \\
        &\quad\quad-  \int_{0}^{\infty}\iint_{X \times X}\operatorname{D}\mathcal{K}_a(\eta(x)-\eta(x'))(\phi(x)-\phi(x')) d\rho_0(x)d\rho_0(x')dt.
        \end{aligned}
    \end{equation}
    and it satisfies
    \begin{equation} \label{dissip}
        \begin{aligned}
        -\int_{0}^{\infty}H[\eta, {m}]\dot{\theta}(t)dt \leq \left.H[{\eta}, {m}] \theta\right|_{t=0},
        \end{aligned}
    \end{equation}
    for any non-negative $\theta(t)\in W^{1,\infty}[0,\infty)$ supported compactly in $[0, \infty)$.
\end{definition}

We assume that $(\bar{\eta}, \bar{{m}})$ is a \textit{strong solution} of \eqref{hamflow} with further conditions as follows:

\begin{itemize}
    \item The Hamiltonian for $(\bar{\eta}, \bar{{m}})$ is conserved, that is
    \begin{equation}
        \left.-\int_{0}^{\infty}H[\bar{\eta}, \bar{m}]\dot{\theta} = H[\bar{\eta}, \bar{m}] \theta\right|_{t=0}.
    \end{equation}
    \item The velocity $\bar{\mathbf{v}}$ is Lipschitz.
    \item The derivative $\dot{\bar{\mathbf{v}}}$ exists and is in $L^\infty([0, \infty)\times X)$.
    \item The gradient of kernel $D\mathcal{K}\big(\bar{\eta}(x, x')\big) \in L^\infty(X \times X)$.
    \item \textcolor{red}{The derivative of pressure} $\psi'(\bar\eta_x)$ is in $L^\infty([0, \infty)\times X)$.
    \item \textcolor{red}{Add later.} 
\end{itemize}
With these assumptions, \eqref{wksol} can be rephrased as follows. 
\begin{equation} \label{strsol}
    \begin{aligned}
    & \int_{0}^{\infty}\int_{X}^{}\dot{\bar{{m}}}\phi(t,x)dxdt \\
    &\quad = -\int_0^\infty \int_X (\gamma-1)(\bar{\eta}_x)^{\gamma-2}\phi_x\rho_0(x)^r dx \\ 
    & \quad\quad-  \int_{0}^{\infty}\iint_{X \times X}\operatorname{D}\mathcal{K}_r(\bar{\eta}(x)-\bar{\eta}(x'))(\phi(x)-\phi(x')) d\rho_0(x)d\rho_0(x')dt \\
    &\quad\quad-  \int_{0}^{\infty}\iint_{X \times X}\operatorname{D}\mathcal{K}_a(\bar{\eta}(x)-\bar{\eta}(x'))(\phi(x)-\phi(x')) d\rho_0(x)d\rho_0(x')dt.
    \end{aligned}
\end{equation}
for all \textcolor{red}{$\phi \in C_c([0, \infty);W^{1,1}(X))$}.




We define the relative kinetic and potential energies as follow:
\begin{align} \small
    K[{m} | \bar{m}] &:= K[{m}] - K[\bar{m}]-\left\langle \frac{\delta K}{\delta{m}}(\bar{m}), {m}-\bar{m} \right\rangle \nonumber \\
     &\ =\int_X \frac{|{m}|^2}{2\rho_0}- \frac{|\bar{m}|^2}{2\rho_0} - \frac{\bar{m}}{\rho_0}({m}-\bar{m})dx \nonumber\\ 
     &\  = \int_X \frac{|{m} - \mathbf{\bar{m}|}^2}{2\rho_0}dx \\
    \mathcal{E}_p[\eta | \bar{\eta}] &:= \mathcal{E}_p[\eta] - \mathcal{E}_p[\bar{\eta}] - \left\langle \frac{\delta \mathcal{E}_p}{\delta \eta}(\bar{\eta}), \eta - \bar{\eta} \right\rangle   \nonumber  \\
     & \  =\int_{X}^{}\psi(\eta_x) - \psi(\bar{\eta}_x) - \psi'(\bar{\eta}_x)\left( \eta_x - \bar{\eta}_x \right)d\rho_0(x) \\
     & \ =: \int_X\psi(\eta_x |\bar{\eta}_x ) d\rho_0(x) \\
    \mathcal{E}_r[\eta | \bar{\eta}] &:=  \mathcal{E}_r[\eta] - \mathcal{E}_r[\bar{\eta}] - \left\langle \frac{\delta \mathcal{E}_r}{\delta \eta}, \eta - \bar{\eta}\right\rangle \nonumber  \\ 
    & \  =\iint_{X \times X}\mathcal{K}_r(\eta(x)-\eta(x')) - \mathcal{K}_r(\bar{\eta}(x)-\bar{\eta}(x')) \nonumber\\ 
    & \ \quad - \mathcal{K}_r'(\bar{\eta}(x) - \bar{\eta}(x'))\left( \eta(x) - \eta(x') - \bar{\eta}(x) + \bar{\eta}(x') \right)d\rho_0(x)d\rho_0(x') \\
    & \ =: \iint_{X \times X} \mathcal{K}_r(\eta | \bar{\eta})d\rho_0(x)d\rho_0(x')\\
    \mathcal{E}_a[\eta | \bar{\eta}] &:=  \mathcal{E}_a[\eta] - \mathcal{E}_a[\bar{\eta}] - \left\langle \frac{\delta \mathcal{E}_a}{\delta \eta}, \eta - \bar{\eta}\right\rangle  \nonumber \\ 
    & \ =\iint_{X \times X}\mathcal{K}_a(\eta(x)-\eta(x')) - \mathcal{K}_a(\bar{\eta}(x)-\bar{\eta}(x')) \nonumber \\ 
    & \ \quad - \mathcal{K}_a'(\bar{\eta}(x) - \bar{\eta}(x'))\left( \eta(x) - \eta(x') - \bar{\eta}(x) + \bar{\eta}(x') \right)d\rho_0(x)d\rho_0(x') \\
    & \ =: \iint_{X \times X}\mathcal{K}_a(\eta | \bar{\eta})d\rho_0(x)d\rho_0(x')
\end{align}
Thus the relative Hamiltonian is
\begin{equation} \label{relHam}
    \begin{aligned}
    H[\eta, {m}| \bar\eta, \bar{m}]:&= H[\eta, {m}] - H[\bar{\eta}, \bar{m}] - \left\langle \frac{\delta H}{\delta \eta}(\bar{\eta}), \eta - \bar\eta\right\rangle - \left\langle \frac{\delta H}{\delta {m}}(\bar{m}),{m} - \bar{m}\right\rangle
    \\ &=K[{m} | \bar{m}] + \mathcal{E}_p[\eta | \bar{\eta}] + \mathcal{E}_r[\eta | \bar{\eta}] + \mathcal{E}_a[\eta | \bar{\eta}].
    \end{aligned}
\end{equation}

\begin{theorem}[Relative entropy inequality]
    Let $(\eta, {m})$ be a weak solution and $(\bar{\eta}, \bar{m})$ a strong solution for \eqref{hamflow}. Then
    \begin{equation} \label{relineq}
        \begin{aligned}
            \left.H[\eta, m|\bar{\eta}, \bar{m}]\right|_{\tau=t} - \left.H[\eta, m| \bar{\eta}, \bar{m}]\right|_{\tau=0} & \leq  \int_0^t \int_X \bar{\mathbf{v}}(D\psi)(\eta_x | \bar{\eta}_x)\rho_0dxdt   \\
            & \quad + \int_{0}^{t} \iint_{X \times X} \bar{\mathbf{v}}(x, x')(D\mathcal{K}_a)(\eta | \bar\eta)d\rho_0(x)d\rho_0(x') \\
            & \quad + \int_{0}^{t} \iint_{X \times X} \bar{\mathbf{v}}(x, x')(D\mathcal{K}_r)(\eta | \bar\eta)d\rho_0(x)d\rho_0(x') \\
        \end{aligned}
    \end{equation}
    where $\bar{\mathbf{v}}(x, x'):= \bar{\mathbf{v}}(x) - \bar{\mathbf{v}}(x')$.
\end{theorem}

\begin{proof}
From \eqref{relHam},
\begin{equation*}
    \begin{aligned}
        &-\int_0^\infty H[\eta, {m} | \bar{\eta}, \bar{m}]\theta(\tau)d\tau \leq H[\eta, {m}]\theta|_{\tau=0} - H[\bar{\eta}, \bar{m}]\theta|_{\tau=0} \\ 
        & \quad + \int_0^\infty\left\langle \frac{\delta H}{\delta \eta}(\bar{\eta}), \eta - \bar\eta\right\rangle \dot{\theta}d\tau + \int_0^\infty \left\langle \frac{\delta H}{\delta {m}}(\bar{m}),{m} - \bar{m}\right\rangle \dot{\theta}d\tau \\
        & = H[\eta, {m}]\theta|_{\tau=0} - H[\bar{\eta}, \bar{m}]\theta|_{\tau=0} \\
        & \quad + \int_0^\infty\int_X \psi'(\bar{\eta}_x)(\eta_x - \bar{\eta}_x)\dot\theta\rho_0dxd\tau \\
        & \quad + \int_0^\infty\iint_{X \times X} D\mathcal{K}\big(\bar{\eta}(x, x')\big)\big(\eta(x, x') - \bar{\eta}(x, x')\big)\dot\theta \rho_0(x)\rho_0(x')dxdx'd\tau \\
        & \quad + \int_0^\infty\int_X \frac{\bar{{m}}}{\rho_0}({m}-\bar{{m}})\dot\theta dxd\tau \\
        & = H[\eta, {m}]\theta|_{\tau=0} - H[\bar{\eta}, \bar{m}]\theta|_{\tau=0} \\
        & \quad -\int_0^\infty\int_X \psi''(\bar{\eta}_x)\bar{\mathbf{v}}_x(\eta_x - \bar{\eta}_x)\theta\rho_0 dx d\tau - \int_0^\infty\int_X \psi'(\bar{\eta}_x)(\mathbf{v}_x - \bar{\mathbf{v}}_x)\theta \rho_0 dxd\tau \\
        & \quad - \left.\int_X\psi'(\bar{\eta}_x)(\eta_x - \bar{\eta}_x)\theta\rho_0(x)dx\right|_{\tau = 0} \\
        & \quad - \int_0^\infty\iint_{X \times X} D^2\mathcal{K}\big(\bar\eta(x, x')\big)\big(\bar{\mathbf{v}}(x, x')\big)\big( \eta(x, x') - \bar\eta(x, x') \big)\theta \rho_0(x)\rho_0(x')dxdx'd\tau \\
        & \quad - \int_0^\infty \iint_{X \times X} D\mathcal{K}(\bar\eta(x, x'))\big(\mathbf{v}(x, x') - \bar{\mathbf{v}}(x, x')\big)\theta \rho_0(x)\rho_0(x')dxdx'd\tau \\
        & \quad - \left.\iint_{X \times X} D\mathcal{K}\big(\bar\eta(x, x')\big)\big(\eta(x, x') - \bar\eta(x, x')\big)\theta \rho_0(x)\rho_0(x')dxdx'd\tau\right|_{\tau = 0} \\
        & \quad + \int_0^\infty\int_X \big(\psi'(\eta_x) - \psi'(\bar{\eta}_x)\big)\bar{\mathbf{v}}\theta dxd\tau \\
        & \quad + \int_0^\infty \iint_{X \times X} \big(D\mathcal{K}(\eta) - D\mathcal{K}(\bar\eta)\big)\big(\bar{\mathbf{v}}(x, x')\big)\theta\rho_0(x)\rho_0(x')dxdx'd\tau \\
        & \quad - \left.\int_X ({m} - \bar{{m}})(\bar{\mathbf{v}}\theta)dx \right|_{\tau = 0} - \int_0^\infty\int_X ({m} - \bar{{m}}) \dot{\bar{\mathbf{v}}}\theta dxd\tau \\
        & =  \left.H[\eta, {m}|\bar\eta, \bar{{m}}]\theta\right|_{\tau = 0} \\
        & \quad + \int_0^\infty \int_X \bar{\mathbf{v}}(D\psi)(\eta_x | \bar{\eta}_x)\theta\rho_0dxd\tau \\
        & \quad + \int_{0}^{\infty} \iint_{X \times X} \bar{\mathbf{v}}(x, x')(D\mathcal{K})(\eta | \bar\eta)\theta\rho_0(x)\rho_0(x')dxdx'd\tau.
    \end{aligned}
\end{equation*}
Choose a test function
\begin{equation} \label{theta}
    \theta(\tau) = 
    \begin{cases}
        1 & 0 \leq \tau < t \\
        -\frac{1}{\epsilon}(\tau-t-\epsilon) & t \leq \tau < t+\epsilon \\
        0 & \tau \geq t+\epsilon.
    \end{cases}
\end{equation}
Letting $\epsilon \rightarrow 0$, we arrive at the desired result.


\end{proof}

\subsubsection*{Pressureless case}
Here, we only consider the interacting energy, that is, $\mathcal{E} := \mathcal{E}_r + \mathcal{E}_a$. Recall that
\begin{equation}
    \begin{aligned}
        \mathcal{E}_r & =\frac{1}{2} \iint \rho_{0}(x) \Big(\mathcal{K}_r(\eta(x)-\eta(x')) -  \textcolor{red}{\mathcal{K}_r(x-x')}\Big) \rho_{0}(x') d x d x', \\ 
        \mathcal{E}_a &=  \frac{1}{2} \iint \rho_{0}(x) \Big(\mathcal{K}_a(\eta(x)-\eta(x')) - \mathcal{K}_a(x-x')\Big) \rho_{0}(x') d x d x',
        \end{aligned}
\end{equation}
where
\begin{equation}
    \mathcal{K}_r(y) = -\frac{|y|^p}{p} \quad\text{ and }\quad \mathcal{K}_a(y) = \frac{|y|^q}{q} \quad \text{for} \ \  p, q>0.
\end{equation}
For technical reasons, let us assume that $0<p<1$ and $q>1$. In fact, the relative energy technique doesn't work in any other cases due to the non-convexity of kernels.

\begin{lemma} \label{plaw}
    Let $\mathcal{K}:=\mathcal{K}_r + \mathcal{K}_a$.
    For each $x , x' \in X$
    \begin{equation}
        \bar{\mathbf{v}}(x, x')(D\mathcal{K})(\eta | \bar{\eta}) \leq C \mathcal{K}(\eta | \bar{\eta})
    \end{equation}
    for some constant $C>0$.
\end{lemma}
\begin{proof}
    \begin{equation}
        \begin{aligned}
        &\bar{\mathbf{v}}(x, x')\Big(\mathcal{K'}\big(\eta(x,x')\big) - \mathcal{K}'\big(\bar{\eta}(x, x')\big) - \mathcal{K}''\big(\bar{\eta}(x, x')\big)\big(\eta(x, x') - \bar{\eta}(x, x')\big)\Big) \\
        &=\bar{\mathbf{v}}(x, x') \bigg( \int_0^1 \frac{d}{ds}\mathcal{K}'\Big(s\big(\eta(x, x')\big)+ (1-s)\big(\bar\eta(x, x')\big)\Big)ds  - \mathcal{K}''\big(\bar\eta(x, x')\big)\big(\eta(x, x')  - \bar\eta(x, x')\big)\bigg) \\
        &=\bar{\mathbf{v}}(x, x') \bigg( \int_0^1 \mathcal{K}''\Big(s\big(\eta(x, x')\big) + (1-s)\big(\bar\eta(x, x')\big)\Big)\big(\eta(x, x') - \bar{\eta}(x, x')\big)ds \\
        & \quad - \mathcal{K}''\big(\bar\eta(x, x')\big)\big(\eta(x, x') - \bar\eta(x, x')\big)\bigg)\\ 
        &= \bar{\mathbf{v}}(x, x')\big(\eta(x, x') - \bar\eta(x, x')\big) \bigg( \int_0^1 \int_0^s \frac{d}{d\tau}\mathcal{K}''\Big(\tau\big(\eta(x, x')\big)+ (1-\tau)\big(\bar\eta(x, x')\big)\Big)d\tau ds  \bigg) \\
        &=\bar{\mathbf{v}}(x, x') \big(\eta(x, x') - \bar\eta(x, x')\big)^2 \bigg( \int_0^1 \int_0^s \mathcal{K}'''\Big(\tau\big(\eta(x, x')\big)+ (1-\tau)\big(\bar\eta(x, x')\big)\Big)d\tau ds \bigg) \\
        &= \big(\eta(x, x') - \bar\eta(x, x')\big)^2 \bigg( \int_0^1 \int_0^s \bar{\mathbf{v}}(x, x') \mathcal{K}'''\Big(\tau\big(\eta(x, x')\big)+ (1-\tau)\big(\bar\eta(x, x')\big)\Big)d\tau ds \bigg) .
        \end{aligned}
    \end{equation}
    For $\mathcal{K}_r$,
    \begin{equation}
        \begin{aligned}
        & \int_0^1 \int_0^s \bar{\mathbf{v}}(x, x') \big(\eta(x, x') - \bar\eta(x, x')\big)^2 \mathcal{K}_r'''\Big(\tau\eta(x, x')+ (1-\tau)\bar\eta(x, x')\Big)d\tau ds  \\
        &= -(p-1)(p-2)\big(\eta(x, x') - \bar\eta(x, x')\big)^2  \\ 
        & \quad \quad \int_0^1 \int_0^s \frac{\bar{\mathbf{v}}(x, x')}{\tau\eta(x, x') + (1-\tau)\bar\eta(x, x')} |\tau\eta(x,x') + (1-\tau)\bar\eta(x, x')|^{p-2}d\tau ds \\
        &\leq -(p-1)(p-2)\big(\eta(x, x') - \bar\eta(x, x')\big)^2  \\ 
        & \quad \quad \int_0^1 \int_0^s \left| \frac{\bar{\mathbf{v}}(x, x')}{\tau\eta(x, x') + (1-\tau)\bar\eta(x, x')} |\tau\eta(x,x') + (1-\tau)\bar\eta(x, x')|^{p-2}\right| d\tau ds \\
        &\leq -C_{Lip}(p-1)(p-2)\big(\eta(x, x') - \bar\eta(x, x')\big)^2 \int_0^1 \int_0^s  |\tau\eta(x,x') + (1-\tau)\bar\eta(x, x')|^{p-2} d\tau ds. \\
        \end{aligned}
    \end{equation}
    The last inequality holds because $\eta, \bar\eta \in \mathcal{A}$ so that
    \begin{equation}
        C_1 \leq \frac{\eta(x, x')}{x-x'} \leq C_2
    \end{equation}
    for some constants $C_1>0, C_2>0$, and $\bar{\mathbf{v}}$ is Lipschitz.
    Since
    \begin{equation}
        \mathcal{K}_r(\eta | \bar\eta) = -(p-1)\big(\eta(x, x') - \bar\eta(x, x')\big)^2 \int_0^1\int_0^s |\tau\eta(x,x') + (1-\tau)\bar\eta(x, x')|^{p-2} d\tau ds, 
    \end{equation}
    we get 
    \begin{equation}
        \bar{\mathbf{v}}(x, x')\mathcal{K}'_r(\eta | \bar\eta) \leq C\mathcal{K}_r(\eta | \bar\eta)
    \end{equation}
    for some constant $C>0$. The case of $\mathcal{K}_a$ is similar.
    Hence we get the desired result.
\end{proof}




An additional restriction on the exponent $q$ of the attracting kernel $\mathcal{K}_a$ gives sharper estimates for the relative Hamiltonian.

\bibliographystyle{abbrv}
\bibliography{./references.bib}
\end{document}
