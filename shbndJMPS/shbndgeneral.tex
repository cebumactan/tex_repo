%%%%%%%%%%%%%%%%%%%%%%%%%%%%%%%%%%%%%%%%%%%%%%%
%
%    thermalelectric equation
%
%                                                      
%
%                                       
%
%                                          version Aug 2017
%
%
%%%%%%%%%%%%%%%%%%%%%%%%%%%%%%%%%%%%%%%%%%%%%%%
\documentclass[a4paper,11pt]{article}

\usepackage[margin=3cm]{geometry}
\usepackage{setspace}
\onehalfspacing
%\doublespacing
%\usepackage{authblk}
\usepackage{amsmath}
\usepackage{amssymb}
\usepackage{amsthm}
\usepackage{calrsfs}
%\usepackage[notcite,notref]{showkeys}

\usepackage{psfrag}
\usepackage{graphicx,subfigure}
\usepackage{color}
\def\red{\color{red}}
\def\blue{\color{blue}}
%\usepackage{verbatim}
% \usepackage{alltt}
%\usepackage{kotex}



\usepackage{enumerate}

%%%%%%%%%%%%%% MY DEFINITIONS %%%%%%%%%%%%%%%%%%%%%%%%%%%

\def\tr{\,\textrm{tr}\,}
\def\div{\,\textrm{div}\,}
\def\sgn{\,\textrm{sgn}\,}

\newcommand{\tcr}{\textcolor{red}}
\newcommand{\tcb}{\textcolor{blue}}
\newcommand{\ubar}[1]{\text{\b{$#1$}}}
\newtheorem{theorem}{Theorem}
\newtheorem{lemma}{Lemma}[section]
\newtheorem{proposition}{Proposition}[section]
\newtheorem{corollary}{Corollary}[section]
\newtheorem{definition}{Definition}[section]
\newtheorem{claim}{Claim}

\theoremstyle{remark}
\newtheorem{remark}{Remark}[section]


%%%%%%%%%%%%%%%%%%%%%%%%%%%%%%%%%%%%%%%%%%%%%%%%%%%%%%%%%%
\begin{document}
\title{Localization in thermo-plasticity}
\author{Theodoros Katsaounis\footnotemark[1]\ \footnotemark[2]
\and Min-Gi Lee\footnotemark[1]
\and Athanasios Tzavaras\footnotemark[1]\  \footnotemark[2]  \footnotemark[3]}
\date{}

\maketitle
\renewcommand{\thefootnote}{\fnsymbol{footnote}}
\footnotetext[1]{Computer, Electrical and Mathematical Sciences \& Engineering Division, King Abdullah University of Science and Technology (KAUST), Thuwal, Saudi Arabia}
\footnotetext[2]{Institute of Applied and Computational Mathematics, FORTH, Heraklion, Greece}
\footnotetext[3]{Corresponding author : \texttt{athanasios.tzavaras@kaust.edu.sa}}
%\footnotetext[4]{Research supported by the King Abdullah University of Science and Technology (KAUST) }
\renewcommand{\thefootnote}{\arabic{footnote}}
\maketitle

% \begin{abstract}
% \end{abstract}
% 
% \tableofcontents
% \pagebreak 

\section{Linear stability of uniform shear}
In the below we assume 
$$ \alpha>0, \quad n\ge0, \quad 1+m>0.$$
We consider
\begin{equation} \label{eq:system}
 \begin{aligned}
  \partial_\tau U &= \Sigma_{xx},\\
  \partial_\tau\Gamma &= \frac{1}{\tau}\frac{1+\alpha}{1+m}(U-\Gamma),\\
  \partial_\tau\Theta &= \frac{1}{\tau}\Big(\Sigma U -\Theta\Big),\\
  \Sigma &= \Theta^{-\alpha}\Gamma^m U^n
 \end{aligned}
\end{equation}
in $(\tau,x)\in \mathbb{R}^+\times [0,1]$ with boundary conditions
\begin{equation}
 \Sigma_x(\tau,0)=\Sigma_x(\tau,1)=0.
\end{equation}
The boundary conditions imply that 
\begin{equation}
 \frac{d}{d\tau}\int_0^1 U(\tau,x) \: dx = 0, \quad \text{and we set } \int_0^1 U(\tau,x) \: dx = 1.
\end{equation}
$ U=\Gamma=\Theta=\Sigma=1 $ corresponds to the uniform shearing solution.
Let 
\begin{align*}
 U &= 1 + \delta U_1 + \mathcal{O}(\delta^2), & \Gamma &= 1 + \delta \Gamma_1 + \mathcal{O}(\delta^2), &
 \Sigma &= 1 + \delta \Sigma_1 + \mathcal{O}(\delta^2), & \Theta &= 1 + \delta \Theta_1 + \mathcal{O}(\delta^2).
\end{align*}
By collecting the leading order terms and neglecting terms higher than the  second order,
\begin{equation} \label{eq:linsystem}
 \begin{aligned}
  \partial_\tau U_1 &= \Sigma_{1xx},\\
  \partial_\tau\Gamma_1 &= \frac{1}{\tau}\frac{1+\alpha}{1+m}(U_1-\Gamma_1),\\
  \partial_\tau\Theta_1 &= \frac{1}{\tau}\Big(\Sigma_1+ U_1 -\Theta_1\Big),\\
  0&=\Sigma_1 + \alpha\Theta_1 -m\Gamma_1 - nU_1 .
 \end{aligned}
\end{equation}
The boundary conditions and the constraint equation read
\begin{equation} \label{eq:linbdry}
 \Sigma_{1x}(\tau,0)=\Sigma_{1x}(\tau,1)=0, \quad \int_0^1 U_1(t,x) \: dx = 0.
\end{equation}
In the sequel, we drop the subscript $1$ in studying the linear system \eqref{eq:linsystem}. From the fourth equation in \eqref{eq:linsystem}, we remove one of the variable. We remove $\Theta$ and get three linear equations
\begin{equation}
 \begin{aligned}
  \partial_\tau U &= \Sigma_{xx},\\
  \partial_\tau\Gamma &= \frac{1}{\tau}\frac{1+\alpha}{1+m}(U-\Gamma),\\
  \partial_\tau\Sigma 
%   &= -\alpha \partial_\tau\Sigma + m \partial_\tau\Gamma +n \partial_\tau U\\
%   &=\frac{1}{\tau}\Big(-\alpha\Big(\Sigma + U - \Big(\Big(\frac{-\Sigma+m\Gamma+nU}{\alpha}\Big)\Big) +\frac{m(1+\alpha)}{1+m}(U-\Gamma)\Big) + n\Sigma_{xx}\\
  &= n\Sigma_{xx} +\frac{1}{\tau}\Big( -(1+\alpha)\Sigma + \big(n-\alpha + \frac{m(1+\alpha)}{1+m}\big)U + \big(m - \frac{m(1+\alpha)}{1+m}\big)\Gamma\Big).
 \end{aligned}
\end{equation}
We consider the even extensions of $U,\Gamma,\Theta,\Sigma$ to the domain $[-1,1]$ which is compatible with the system and the boundary conditions in \eqref{eq:linbdry}. We consider the cosine series of the variables,
\begin{align*}
 U(\tau,x) = U_0(\tau) + \sum_{k=1}^\infty U_k(\tau)\cos(2\pi kx),\\
 \Gamma(\tau,x) = \Gamma_0(\tau) + \sum_{k=1}^\infty \Gamma_k(\tau)\cos(2\pi kx),\\
 \Theta(\tau,x) = \Theta_0(\tau) + \sum_{k=1}^\infty \Theta_k(\tau)\cos(2\pi kx),\\
 \Sigma(\tau,x) = \Sigma_0(\tau) + \sum_{k=1}^\infty \Sigma_k(\tau)\cos(2\pi kx).
\end{align*}
In particular, the constraint equation implies $U_0(\tau)\equiv0$. For each $k$-th mode, we collect terms in the matrix form $\dot{X}_k = A_k(\tau)X_k$. At $0$-th mode,
\begin{align*}
   \partial_\tau U_0 &= 0,\\
  \partial_\tau\Gamma_0 &= \frac{1}{\tau}\frac{1+\alpha}{1+m}(U_0-\Gamma_0),\\
  \partial_\tau\Sigma_0 &= \frac{1}{\tau}\Big( -(1+\alpha)\Sigma_0 + \big(n-\alpha + \frac{m(1+\alpha)}{1+m}\big)U_0 + \big(m - \frac{m(1+\alpha)}{1+m}\big)\Gamma_0\Big).
\end{align*}
Since $U_0=0$, $\Gamma_0(\tau)$ decays to $0$, and $\Sigma_0(\tau)$ decays to $0$.
At $k$-th mode $k\ge1$,
\begin{align}
  \frac{d}{d\tau}
  \begin{pmatrix}
   U_k \\ \Gamma_k \\ \Sigma_k
  \end{pmatrix} &= 
  \begin{pmatrix}
   0 & 0 & -(2\pi k)^2\\
   \frac{1}{\tau}\frac{1+\alpha}{1+m} & -\frac{1}{\tau}\frac{1+\alpha}{1+m} & 0\\
   \frac{1}{\tau}\Big( n-\alpha + \frac{m(1+\alpha)}{1+m}\Big) &
   \frac{1}{\tau}\Big(m - \frac{m(1+\alpha)}{1+m}\Big) &
   -\frac{1}{\tau}(1+\alpha) -n(2\pi k)^2
  \end{pmatrix}
  \begin{pmatrix}
   U_k \\ \Gamma_k \\ \Sigma_k
  \end{pmatrix} \nonumber\\
  &=A_k(\tau)\begin{pmatrix}
   U_k \\ \Gamma_k \\ \Sigma_k
  \end{pmatrix}.  \label{eq:k-system}
\end{align}

\subsection{Characteristic polynomial of $A_k(\tau)$}
Let $\lambda_i(\tau)$, $i=1,2,3$ be the three eigenvalues of $A_k(\tau)$. %We have
% \begin{equation} \label{eq:eig_sum_mult}
%  \begin{aligned}
%  \lambda_1 + \lambda_2 + \lambda_3 &< 0,\\
%  \lambda_1\lambda_2\lambda_3 &= -(2\pi k)^2(-\alpha+m+n)\frac{1+\alpha}{\tau^2(1+m)}.
%  \end{aligned}
% \end{equation}
The characteristic polynomial of $A_k(\tau)$ reads
% $$-\frac{m(1+\alpha)}{\tau(1+m)}\Big(\lambda + \frac{1}{\tau}\Big) = \frac{1}{(2\pi k)^2}\Big(\lambda + \frac{1+\alpha}{\tau(1+m)}\Big)\Big(\lambda^2 + \lambda\big(\frac{1+\alpha}{\tau} + n(2\pi k)^2\big) + \frac{n-\alpha}{\tau}(2\pi k)^2\Big).$$
% 
% $$\frac{\alpha-m-n-mn}{\tau(1+m)}\lambda - \frac{(1+\alpha)(-\alpha+m+n)}{\tau^2(1+m)}=\frac{1}{(2\pi k)^2}\lambda\Big(\lambda + \frac{1+\alpha}{\tau(1+m)}\Big)\Big(\lambda + \frac{1+\alpha}{\tau} + n(2\pi k)^2\Big).$$
% 
% 
% $$\frac{\alpha-m-n}{\tau(1+m)}\Big(\lambda + \frac{1+\alpha}{\tau}\Big) = \frac{1}{(2\pi k)^2}\lambda\Big(\lambda^2 + \lambda\Big(\frac{1+\alpha}{\tau(1+m)} + \frac{1+\alpha}{\tau} + n(2\pi k)^2\Big) + \frac{n(1+m+\alpha)}{\tau(1+m)}(2\pi k)^2\Big).$$
% 
% \begin{equation}
% -(2\pi k)^2\frac{-\alpha+m+n}{\tau(1+m)}\Big(\lambda + \frac{1+\alpha}{\tau}\Big) = \lambda\Big[\Big(\lambda + \frac{1+\alpha}{\tau(1+m)}\Big)\Big(\lambda + \frac{1+\alpha}{\tau}\Big) + n(2\pi k)^2\Big(\lambda + \frac{1+m+\alpha}{\tau(1+m)}\Big)\Big]. \label{eq:charpoly}
% \end{equation}
\begin{align}
0&= \lambda\Big(\lambda + \frac{1+\alpha}{\tau(1+m)}\Big)\Big(\lambda + \frac{1+\alpha}{\tau}\Big) + n(2\pi k)^2\lambda\Big(\lambda + \frac{1+m+\alpha}{\tau(1+m)}\Big)\nonumber\\
&+ (2\pi k)^2\frac{-\alpha+m+n}{\tau(1+m)}\Big(\lambda + \frac{1+\alpha}{\tau}\Big)\triangleq f(\lambda). \label{eq:poly}
\end{align}
Let $\epsilon = \frac{1}{\tau}$. In general, the polynomial $f=f(\lambda,m,n,q,\epsilon)$, where $q=-\alpha+m+n$. Its behavior could be qualitatively altered passing parameter values of $m=-1$, $n=0$, $q=0$, and $\epsilon=0$. We  let $m>-1$ but let other critical parameters reachable; we fix $n$ and $q$ in a certain regime and study the perturbative behaviors on $\epsilon$ according to the regime. We write $f= f^{\alpha,m,n}(\lambda,\epsilon)$ and suppress the superscript when the context of the regime is clear.

\noindent{\bf Summary:}
\begin{align}
f^{\alpha,m,n}(\lambda,\epsilon)&= \lambda\Big(\lambda + \epsilon\frac{1+\alpha}{(1+m)}\Big)\Big(\lambda + \epsilon(1+\alpha)\Big) + n(2\pi k)^2\lambda\Big(\lambda + \epsilon\frac{1+m+\alpha}{1+m}\Big) \nonumber \\
&+ \epsilon(2\pi k)^2\frac{-\alpha+m+n}{(1+m)}\Big(\lambda + \epsilon(1+\alpha)\Big).
\end{align}
The limiting problem $f^{\alpha,m,n}(\lambda,0)=0$ gives 
$$ \lambda_1=\lambda_2=0, \quad \lambda_3 = -n(2\pi k)^2,$$
where we see the bifurcation at $n=0$. To analyze the roots $\lambda_1=\lambda_2=0$  of $f(\lambda,0)$, let $z = \frac{\lambda}{\epsilon}$, and $\epsilon^2\tilde{f}(z,\epsilon) =  f(\lambda, \epsilon)$. Then
\begin{align}
 \tilde{f}(z,\epsilon) &= n(2\pi k)^2 z\Big(z + \frac{1+m+\alpha}{1+m}\Big)+ (2\pi k)^2\left(\frac{-\alpha+m+n}{1+m}\right)\Big(z + (1+\alpha)\Big) \nonumber\\
 &+ \epsilon z\Big(z+\frac{1+\alpha}{1+m}\Big)\Big(z+(1+\alpha)\Big). \label{eq:reduced_poly}
\end{align}
Provided $n>0$, the limiting problem $\tilde{f}(z,0)=0$ has the two roots $z_1,z_2$ that are the roots of the quadratic polynomial
\begin{equation} 
nz^2 + z\Big( \frac{n(1+m+\alpha) + (-\alpha+m+n)}{1+m}\Big) + \frac{(1+\alpha)(-\alpha+m+n)}{1+m} =0. \label{eq:auxquad} 
\end{equation}
From that we can see if $q=-\alpha+m+n>0$ then the real parts of two roots of \eqref{eq:auxquad} must be negative because $z_2+z_3 = -\frac{n(1+m+\alpha) + (-\alpha+m+n)}{n(1+m)} < 0$ and $z_2z_3=\frac{(1+\alpha)(-\alpha+m+n)}{n(1+m)}>0$ whereas if $q=-\alpha+m+n<0$, there must be one positive and one negative real roots.
In each of the regime, the perturbation on $\epsilon$ is regular as exposed in the below.

\subsection{Inviscid case $n=0$:}
When $n=0$, the polynomial \eqref{eq:poly} factorizes to get,
$$ f(\lambda,\epsilon) = (\lambda +\epsilon(1+\alpha))\Big( \lambda^2 + \epsilon\lambda \frac{1+\alpha}{1+m} + \epsilon (2\pi k)^2 \frac{-\alpha+m}{1+m}\Big)=0.$$

\noindent{\bf subcase 1 $q=-\alpha+m>0$ for $\epsilon$ small:}
\begin{align*}
 \lambda_1(\epsilon) &= - \epsilon\frac{1+\alpha}{2(1+m)} + i\sqrt{\epsilon}(2\pi k)\sqrt{\frac{-\alpha+m}{1+m}}\sqrt{\Big(1-\epsilon \frac{(1+\alpha)^2}{4(2\pi k)^2(-\alpha+m)(1+m)}\Big)},\\ 
 \lambda_2(\epsilon) &= - \epsilon\frac{1+\alpha}{2(1+m)} - i\sqrt{\epsilon}(2\pi k)\sqrt{\frac{-\alpha+m}{1+m}}\sqrt{\Big(1-\epsilon \frac{(1+\alpha)^2}{4(2\pi k)^2(-\alpha+m)(1+m)}\Big)},\\ 
 \lambda_3(\epsilon) &= -\epsilon(1+\alpha)<0. 
\end{align*}
% Note that for sufficiently small $\epsilon$, we have an order 
% \begin{align*}
%  Re(\lambda_0)< Re(\lambda)=Re(\lambda^*) < 0, \quad \text{if $m>0$},\\
%  Re(\lambda)=Re(\lambda^*) < Re(\lambda_0) < 0, \quad \text{if $m<0$}
% \end{align*}
% and the order is preserved as $\epsilon \rightarrow 0^+$.
\noindent{\bf subcase 2 $q=-\alpha+m=0$:}
\begin{align*}
 \lambda_1(\epsilon)\equiv 0, \quad \lambda_2(\epsilon)= -\epsilon, \quad \lambda_3(\epsilon) = -\epsilon(1+\alpha).
\end{align*}

\noindent{\bf subcase 3 $q=-\alpha+m<0$ for $\epsilon$ small:}
\begin{align*}
 \lambda_1(\epsilon) &= - \epsilon\frac{1+\alpha}{2(1+m)} + \sqrt{\epsilon}(2\pi k)\sqrt{\frac{\alpha-m}{1+m}}\sqrt{\Big(1-\epsilon \frac{(1+\alpha)^2}{4(2\pi k)^2(-\alpha+m)(1+m)}\Big)},\\ 
 \lambda_2(\epsilon) &= - \epsilon\frac{1+\alpha}{2(1+m)} - \sqrt{\epsilon}(2\pi k)\sqrt{\frac{\alpha-m}{1+m}}\sqrt{\Big(1-\epsilon \frac{(1+\alpha)^2}{4(2\pi k)^2(-\alpha+m)(1+m)}\Big)},\\ 
 \lambda_3(\epsilon) &= -\epsilon(1+\alpha)<0. 
\end{align*}

\begin{align*}
 \lambda_1(\epsilon) &= +\sqrt{\epsilon} (2\pi k)\sqrt{\frac{\alpha-m}{1+m}} - \epsilon\frac{1+\alpha}{2(1+m)} + \mathcal{O}(\epsilon^{\frac{3}{2}}) > 0,\\
 \lambda_2(\epsilon) &= -\sqrt{\epsilon} (2\pi k)\sqrt{\frac{\alpha-m}{1+m}} - \epsilon\frac{1+\alpha}{2(1+m)} + \mathcal{O}(\epsilon^{\frac{3}{2}}) < 0,\\
 \lambda_3(\epsilon) &= -\epsilon(1+\alpha)<0. 
\end{align*}
% Note that for sufficiently small $\epsilon$, we have an order
% \begin{align*}
%  Re(\lambda^-)<Re(\lambda_0) < 0 < Re(\lambda^+) \quad \text{with $Re(\lambda^+) = \mathcal{O}(\sqrt{\epsilon})$}
% \end{align*}
% and the order is preserved as $\epsilon \rightarrow 0^+$.
% 
% \noindent{\bf Viscous case $n>0$:}
% 
% \begin{proposition}
%  $\exists \epsilon_0$ such that if $|\epsilon| \le \epsilon_0$, then the three roots of the characteristic polynomial is analytic in $\epsilon$.
% \end{proposition}
% \begin{proof}
%  Implicit function theorem by Lars Hormander.
% \end{proof}

\subsection{Viscous case $n>0$}
For a technical reason we assume the following, which holds if and only if \eqref{eq:auxquad} does not have a double root.
 \begin{equation} 
  \alpha\ne m \quad \text{and} \quad (-\alpha+m)(1+n)^2 \ne 4nm(1+\alpha). \tag{$A0$} \label{eq:nodoubleroot} 
\end{equation}
To apply the theorem of Coddington-Levinson, that the coefficient matrix is continuously diagonalizable up to the infinity is necessary.
\begin{proposition}[Absolutely Continuous Diagonalization]
 Assume that $(\alpha,m,n)$ is such that \eqref{eq:nodoubleroot} holds. Then for some $\tau_0$, $\exists$ $\lambda_1(\tau)$, $\lambda_2(\tau)$, $\lambda_3(\tau)$, and $S(\tau)$ the matrix of right eigenvectors that is invertible, for $\tau\ge\tau_0$  with the following properties.
 \begin{enumerate}
  \item $A(\tau) = S(\tau) \begin{pmatrix} \lambda_1(\tau) & 0 & 0\\ 0 & \lambda_2(\tau) & 0 \\ 0 & 0 & \lambda_3(\tau) \end{pmatrix} S^{-1}(\tau)$.%= S(\tau) \Lambda(\tau) S^{-1}(\tau)$.
  \item $|S_{ij}(\tau)|, |S^{-1}_{ij}(\tau)|$ are uniformly bounded and $S(\tau)$ is absolutely continuous, i.e., $\int_{\tau_0}^\infty |\dot{S}_{ij}(s)|\; ds <\infty$, $\forall i,j=1,2,3$.
  \item If $\epsilon\triangleq \frac{1}{\tau}$ and $(\cdot)'=\frac{d}{d\epsilon}(\cdot)$ then $\lambda_1(\epsilon)$, $\lambda_2(\epsilon)$, $\lambda_3(\epsilon)$, and $S(\epsilon)$ are analytic in the radius $\frac{1}{\tau_0}$ ball of $0$ and their leading order expansions are
  \begin{align}
   \lambda_1(\epsilon) &\triangleq \epsilon z_1(\epsilon) = \epsilon z_1(0) + \epsilon^2 z_1'(0) + \cdots,\nonumber\\
   \lambda_2(\epsilon) &\triangleq \epsilon z_2(\epsilon) =\epsilon z_2(0) + \epsilon^2 z_1'(0) + \cdots,\nonumber\\ &\text{where $z_1(0)$ and $z_2(0)$ are from the two roots of \eqref{eq:auxquad},}\nonumber \\
   \lambda_3(\epsilon) &= -n(2\pi k)^2 + \epsilon\lambda_3'(0) + \cdots,\\
   S(\epsilon) &= 
   \begin{pmatrix}
    1 + z_1(0)\frac{1+m}{1+\alpha} & 1 + z_2(0)\frac{1+m}{1+\alpha} & 1\\
    1 & 1 & 0\\
    0 & 0 & n\\
   \end{pmatrix} \\&+ \epsilon
   \begin{pmatrix}
    z_1'(0)\frac{1+m}{1+\alpha} & z_2'(0)\frac{1+m}{1+\alpha} & \frac{\lambda_3'(0)}{n(2\pi k)^2}\\
    0 & 0 & -\frac{\frac{1+\alpha}{1+m}}{n(2\pi k)^2}\\
    -\frac{z_1(0)\big(1+z_1(0)\frac{1+m}{1+\alpha}\big)}{(2\pi k)^2} & -\frac{z_2(0)\big(1+z_2(0)\frac{1+m}{1+\alpha}\big)}{(2\pi k)^2} & 0\\
   \end{pmatrix} + \cdots. \label{eq:S}
  \end{align}
 \end{enumerate}
\end{proposition}
\begin{proof}
 First we prove the assertion 3. That the eigenvalues are analytic in $\epsilon$ in some neighborhood of $0$ follows from that $-n(2\pi k)^2$ is a simple root of \eqref{eq:poly} and $z_1(0)$ and $z_2(0)$ are the simple roots of \eqref{eq:auxquad}: For a fixed $n>0$, $\lambda_3(0)= -n(2\pi k)^2$ is a simple root of \eqref{eq:poly} which leads to that $\frac{\partial f}{\partial \lambda}(-n(2\pi k)^2,0)\ne 0$ and the existence of an analytic $\lambda_3(\epsilon)$ follows from the Theorem \ref{thm:anal}. By the same reasoning there are analytic functions $z_1(\epsilon)$ and $z_2(\epsilon)$ in some neighborhood of $0$. Note that $f(\epsilon z(\epsilon),\epsilon)=0$ if and only if $\tilde{f}(z(\epsilon),\epsilon)=0$. For a fixed $n>0$, we take $\epsilon_0$ sufficiently small so that $\epsilon z_i(\epsilon)\ne \lambda_3(\epsilon)$, $i=1,2$ are assured in $\epsilon_0$-ball and all of the eigenvalues are analytic. We set $\epsilon_0$ be this radius of analyticity and let $\tau_0 = \frac{1}{\epsilon_0}$.
 
 The right eigenvector matrix is directly computable in terms of $\lambda_i(\epsilon)$ and $A_{ij}(\epsilon)$. We prove its non-degeneracy. Note that we chose the normalization so that expansion of $S(\epsilon)$ appears as in \eqref{eq:S}. Since $z_1(0)\ne z_2(0)$, the constant part $S(0)$ is invertible. Thus, by taking $\epsilon_0$ smaller if necessary, $S(\epsilon)$ is invertible $\epsilon\in[0,\epsilon_0]$. Thus the assertion 1 follows.
 
 That all entries of $S_{ij}(\epsilon)$ and $S^{-1}_{ij}(\epsilon)$ are uniformly bounded follows from that the determinant is uniformly bounded. The latter assertion is true because $A(\epsilon)$ is uniformly bounded and the determinant is a continuous function of entries. Further, because $S_{ij}(\epsilon)$ is analytic, $S_{ij}'(\epsilon)$ is bounded in the $\epsilon_0$-ball and thus $\frac{d}{d\tau} S_{ij}(\tau) = -\frac{1}{\tau^2}S_{ij}'(\epsilon)$, which is integrable from $\tau_0$ to infinity. This completes the proof.
\end{proof}

Now, we let $Y_k \triangleq S^{-1}(\tau)X_k$ then the simple calculation shows
$$ \dot{Y}_k(\tau) = \Big(\Lambda(\tau;k) - S^{-1}\dot{S}\Big) Y_k,$$
where $\Lambda(\tau;k)$ is the diagonal matrix of eigenvalues given by the proposition. By the proposition the latter term $- S^{-1}\dot{S}$ is an integrable remainder term. Further, we write $\Lambda(\tau;k) = \Lambda_0 + \frac{1}{\tau}\Lambda_1 + \frac{1}{\tau^2}\tilde{\Lambda}(\tau;k)$ which is the expansion of $\Lambda(\tau;k)$ in $\frac{1}{\tau}$. We see that $\frac{1}{\tau^2}\tilde{\Lambda}(\tau;k)$ and $- S^{-1}\dot{S}$ comprise the integrable remainder term. We write the equation
\begin{align} 
\dot{Y}_k(\tau) &= \Big(\Lambda_0 + \frac{1}{\tau}\Lambda_1\Big) Y_k + R(\tau) Y_k, \quad R(\tau) = \frac{1}{\tau^2}\Big(\tilde{\Lambda}(\tau;k)- S^{-1}\frac{d}{d\epsilon}{S}\Big), \label{eq:diagonalsystem} \\
 \Lambda_0 + \frac{1}{\tau}\Lambda_1 &= 
 \begin{pmatrix} 
  \frac{z_1(0)}{\tau} & 0 & 0\\
  0 & \frac{z_2(0)}{\tau} & 0\\
  0 & 0 & -n(2\pi k)^2 + \frac{\lambda_3'(0)}{\tau}
 \end{pmatrix}
\end{align}




\noindent{\bf Spectral gaps:}
% It turns out that \eqref{eq:auxquad} has a double root if and only if
% \begin{equation} 
%   \alpha=m, \quad \text{or} \quad (-\alpha+m)(1+n)^2 = 4nm(1+\alpha) \label{eq:double}
% \end{equation}

If $-\alpha+m+n>0$ and the polynomial has two complex roots, then their real parts are given as
$$ Re(z_1) = Re(z_2) = -\frac{n(1+m+\alpha) + (-\alpha+m+n)}{1+m} < 0.$$

Regardless of the sign of $-\alpha+m+n$, the quadratic polynomial has two real roots if $n$ is small enough ($0<\epsilon\ll n\ll1$). The asymptotic expansion in $n$ can be computed.  Provided they are not the double root, one root is $$-(1+\alpha) + \mathcal{O}(n) \quad \text{as $n \rightarrow 0$.} $$ and the other root is $$- \frac{-\alpha+m}{n(1+m)} - \frac{1-\alpha m}{1+m} + \mathcal{O}(n) \quad \text{as $n \rightarrow 0$.}$$
% \begin{equation} \label{eq:nexpand}
% \begin{aligned}
%  &\text{} -(1+\alpha) + \mathcal{O}(n),\\
%  &\text{the other root is} - \frac{-\alpha+m}{n(1+m)} - \frac{1-\alpha m}{1+m} + \mathcal{O}(n), \quad \text{as $n \rightarrow 0$.}
% \end{aligned}
% \end{equation}


\noindent{\bf Index conditions of Coddington-Levinson:}

In the below we check the index condition of the Coddington-Levinson. 

\noindent{\bf subcase 1 $q=-\alpha+m+n>0$:} The real parts of eigenvalues are all negative for large $\tau$. Suppose \eqref{eq:auxquad} has two negative real roots so that  $-n(2\pi k)^2 + \frac{1}{\tau}\lambda_3'(0)<\frac{1}{\tau}z_2(0)<\frac{1}{\tau}z_1(0)<0$, for large $\tau<\infty$.  Then we can check for instance
$\displaystyle\int_{\tau_1}^{\tau_2} (z_1(0)-z_1(0))\frac{1}{s} \; ds = 0< 1$,
or $1\in I_2$ for the fixed index $1$. $K=1$ is used in the computation. On the other hand,
$$ \lim_{\tau \rightarrow \infty}\int_{\tau_0}^\tau (z_1(0)-z_2(0))\frac{1}{s} \; ds = \infty, \quad  \int_{\tau_1}^{\tau_2} Re (z_1(0)-z_2(0))\frac{1}{s} \; ds > -1,$$
or $2\in I_1$ for the fixed index $1$ and
$$ \lim_{\tau \rightarrow \infty}\int_{\tau_0}^\tau (z_1(0)-\lambda_3'(0))\frac{1}{s} + n(2\pi k)^2 \; ds =\infty, \quad  \int_{\tau_1}^{\tau_2} (z_1(0)-\lambda_3'(0))\frac{1}{s} + n(2\pi k)^2 \; ds > -1,$$
or $3\in I_1$ for the fixed index $1$.

 Similarly we can proceed to check the index condition for all other cases, with $K=1$.
%For $2$ fixed, on the other hand, $\displaystyle\int_{t_1}^{t_2} Re (z_2(0)-z_1(0))\frac{1}{s} \; ds < 1$, or $1\in I_2$. 
We summarized the result in the below. In all cases in the below $K=1$ is enough as were in the examples.
\begin{align*}
  \text{Fix $\ell=1$:}&& 2,3&\in I_1, & 1&\in I_2\\
  \text{Fix $\ell=2$:}&& 3&\in I_1, & 1,2&\in I_2\\
  \text{Fix $\ell=3$:}&& & & 1,2,3&\in I_2.
\end{align*}
Now, when \eqref{eq:auxquad} has two complex roots with negative real parts so that $-n(2\pi k)^2 + \frac{1}{\tau}\lambda_3'(0)<Re\Big(\frac{1}{\tau}z_2(0)\Big)=Re\Big(\frac{1}{\tau}z_1(0)\Big)<0$ for large $\tau<\infty$, the results are
\begin{align*}
  \text{Fix $\ell=1$:}&& 3&\in I_1, & 1,2&\in I_2,\\
  \text{Fix $\ell=2$:}&& 3&\in I_1, & 1,2&\in I_2,\\
  \text{Fix $\ell=3$:}&& & & 1,2,3&\in I_2.
\end{align*}


\noindent{\bf subcase 2 $q=-\alpha+m+n=0$:} In this case, the roots of \eqref{eq:auxquad} are 
$$ z_1(0)=0, \quad z_2(0) = -\frac{1+m+\alpha}{1+m}$$
and the sign of $z_1(\epsilon)$ is determined by the next order term. Nevertheless we have 
$-n(2\pi k)^2 + \frac{1}{\tau}\lambda_3'(0)<\frac{1}{\tau}z_2(0)<0=\frac{1}{\tau}z_1(0)$ for large $\tau<\infty$ and
\begin{align*}
  \text{Fix $\ell=1$:}&& 2,3&\in I_1, & 1&\in I_2\\
  \text{Fix $\ell=2$:}&& 3&\in I_1, & 1,2&\in I_2\\
  \text{Fix $\ell=3$:}&& & & 1,2,3&\in I_2.
\end{align*}

\noindent{\bf subcase 3 $q=-\alpha+m+n<0$:}
The characteristic polynomial has precisely one positive real root and two negative real roots. If we label them so that
$-n(2\pi k)^2 + \frac{1}{\tau}\lambda_3'(0)<\frac{1}{\tau}z_2(0)<0<\frac{1}{\tau}z_1(0)$, for large $\tau<\infty$ then
\begin{align*}
  \text{Fix $\ell=1$:}&& 2,3&\in I_1, & 1&\in I_2\\
  \text{Fix $\ell=2$:}&& 3&\in I_1, & 1,2&\in I_2\\
  \text{Fix $\ell=3$:}&& & & 1,2,3&\in I_2.
\end{align*}

\begin{theorem}[Unstable regime $n>0$] Suppose $n>0$ and $q=-\alpha+m+n<0$. Then, for each $k\ge1$ there is a growing mode solution $\varphi^*(\tau)$ of \eqref{eq:system} such that
$$\lim_{\tau \rightarrow \infty} \varphi^*(\tau)\left(\frac{\tau_0}{\tau}\right)^{z_1(0)} = \begin{pmatrix} 1+z_1(0)\frac{1+m}{1+\alpha}\\1\\0\end{pmatrix}.$$
Furthermore, for any solution $x(\tau)$
$$\lim_{\tau \rightarrow \infty} \left|\frac{x(\tau)}{x(\tau_0)}\left(\frac{\tau_0}{\tau}\right)^{z_1(0)}\right| \le \left|\begin{pmatrix} 1+z_1(0)\frac{1+m}{1+\alpha}\\1\\0\end{pmatrix}\right|.$$
\end{theorem}
\begin{remark}
$z_1(0)$ and $\begin{pmatrix} 1+z_1(0)\frac{1+m}{1+\alpha}\\1\\0\end{pmatrix}$ are independent of $k$. 
\end{remark}
\begin{proof}
We let $\ell=1$ and apply the Theorem \ref{thm:CL} on the diagonalized system \eqref{eq:diagonalsystem} to find
$$ \lim_{\tau \rightarrow \infty} \phi^*(\tau)\exp\left(-\int_{\tau_0}^\tau z_1(0)\frac{1}{s}\; ds\right) = \lim_{\tau \rightarrow \infty} \phi^*(\tau)\left(\frac{\tau_0}{\tau}\right)^{z_1(0)} = \hat{\ell}.$$
Now, $\displaystyle\lim_{\tau \rightarrow \infty} S(\tau)$ exists and let $\varphi^*(\tau) = S(\tau)\phi^*(\tau)$. We conclude
$$ \lim_{\tau \rightarrow \infty} S(\tau)\phi^*(\tau)\left(\frac{\tau_0}{\tau}\right)^{z_1(0)} = \lim_{\tau \rightarrow \infty}S(\tau)\hat{\ell} = \begin{pmatrix} 1+z_1(0)\frac{1+m}{1+\alpha}\\1\\0\end{pmatrix}.$$
Since the index condition holds for all $\ell=1,2,3$, $\varphi_\ell(\tau)$ $\ell=1,2,3$ comprise a fundamental matrix for $\tau\ge\tau_0$. However, because $\frac{1}{\tau}z_1(0)$ is the biggest among $\frac{1}{\tau}z_1(0)$, $\frac{1}{\tau}z_2(0)$, and $-n(2\pi k)^2 + \frac{1}{\tau}\lambda_3'(0)$, in the applying the Theorem \ref{thm:CL} we find 
$$ \lim_{\tau \rightarrow \infty} \varphi_\ell(\tau)\left(\frac{\tau_0}{\tau}\right)^{z_1(0)} = 0, \quad \ell=2,3.$$
Thus the second assertion follows.

\end{proof}



\section{Nonlinear stability of uniform shear}
{\blue Can I use the method of invariant region?}
% 
% \noindent{\bf Case $-\alpha+m+<0$, $n>0$:}
% 
% \noindent{\bf Case $-\alpha+m+<0$, $n=0$:}
% The positive real root is unbounded?

\section{Numerical computation of the heteroclinic orbit}
 
\appendix
\section*{Appendix}
\renewcommand\thetheorem{\Alph{theorem}}
\newcounter{tmp}
\setcounter{theorem}{\thetmp}
% \section{Analyticity of a simple root of a polynomial}

\begin{theorem}{\cite[Diagonal Version]{CL1955}}\label{thm:CL} Let $X(t)\in \mathbb{R}^d$ and $X'(t) = \big(\Lambda(t) + R(t)\big)X$, where $\Lambda(t)$ is a diagonal matrix with diagonal entries $\lambda_j(t)$, $j=1,\cdots,d$ continuous and $R(t)$ is a matrix with entries $R_{ij}$ integrable, i.e., $\int_{a}^\infty |R_{ij}(s)|\; ds < \infty$ $\forall i,j=1,\cdots,d$. 

For fixed $\ell$ and for some constant $K$, we say $j$ belongs to $I_1$ if 

$(i)$ $\int_{a}^\infty Re(\lambda_k(s) -\lambda_j(s))\; ds \rightarrow \infty$ as $t \rightarrow \infty$ and $(ii)$ $\forall (t_2, t_1), \:t_2\ge t_1 \ge a$, $\int_{t_1}^{t_2} Re(\lambda_k(s) -\lambda_j(s))\; ds > -K$.

We say $j$ belongs to $I_2$ if $\forall (t_2, t_1), \: t_2\ge t_1 \ge a$, $\int_{t_1}^{t_2} Re(\lambda_k(s) -\lambda_j(s))\; ds < K$.

Suppose that we are in situation such that for some $K$ $\forall j, j=1,\cdots,d$ belongs to either $I_1$ or $I_2$. Then there is an orbit $\varphi_\ell(t)$ $t\ge a$ and $t_0$ such that
\begin{equation}
 \lim_{t \rightarrow \infty} \varphi_\ell(t) \exp\left(-\int_{t_0}^t \lambda_\ell(s)\; ds\right) = \hat{\ell}, \quad \text{where $\hat{\ell}$ is the $\ell$-th coordinate basis of $\mathbb{R}^d$.}
\end{equation}
\end{theorem}
\begin{proof}
{\blue proof}
\end{proof}

\begin{theorem}{\cite[p. 24]{L1966}} \label{thm:anal} Let $f_j(w,z)$, $j=1,\cdots,m$, be analytic functions of $(w,z)=(w_1,\cdots,w_m,z_1,\cdots,z_n)$ in a neighborhood of a point $(w^0,z^0)$ in $\mathbb{C}^m\times \mathbb{C}^n$, and assume that $f_j(w^0,z^0)=0$, $j=1,\cdots,m$ and that
$$ \det\Big( \frac{\partial f_j}{\partial w_k} \Big)_{j,k=1}^m \ne 0 \quad \text{at $(w^0,z^0)$}.$$
Then the equations $f_j(w,z)=0$, $j=1,\cdots,m$ have a uniquely determined analytic solution $w(z)$ in a neighborhood of $z_0$, such that $w(z^0)=w^0$.
\end{theorem}







\begin{thebibliography}{10}

\bibitem{L1966}
{\sc H\"ormander, Lars},
{\it An introduction to complex analysis in several variables}, 
(D. Van Nostrand Co., Inc., Princeton, N.J.-Toronto, Ont.-London (1966).

\bibitem{CL1955}
{\sc Coddington, Earl A. and Levinson, Norman},
{\it Theory of ordinary differential equations},
(McGraw-Hill Book Company, Inc., New York-Toronto-London 1955).

% \bibitem{bertsch_effect_1991}
% {\sc M.~Bertsch, L.~Peletier, and S.~Verduyn~Lunel},
% The effect of temperature dependent viscosity on shear flow of  incompressible fluids,
% {\it SIAM J. Math. Anal.} {\bf 22 } (1991), 328--343.
% 
% \bibitem{clifton_rev_1990}
% {\sc R.J. Clifton},  High strain rate behaviour of metals,
% % {\it Applied Mechanics Review}
% {\it Appl. Mech. Rev.}
% {\bf 43} (1990), S9-S22.
% 
% \bibitem{DH_1983}
% {\sc C.M. Dafermos and L.~Hsiao},
% Adiabatic shearing of incompressible fluids with temperature-dependent viscosity.
% {\it Quart.  Applied Math.} {\bf 41} (1983), 45--58.
% 
% % \bibitem{fenichel_persistence_1972}
% % {\sc N.~Fenichel},
% % Persistence and smoothness of invariant manifolds for  flows,
% % {\it Indiana Univ. Math. J.} {\bf 21} (1972) 193--226.
% 
% \bibitem{fenichel_asymptotic_1974}
% {\sc N.~Fenichel},
% Asymptotic stability with rate conditions,
% {\it Indiana Univ. Math. J.} {\bf 23} (1974) 1109--1137.
% 
% \bibitem{fenichel_asymptotic_1977}
% {\sc N.~Fenichel},
% Asymptotic stability with rate conditions \textrm{II},
% {\it Indiana Univ. Math. J.} {\bf 26} (1977) 81--93.
% 
% \bibitem{fenichel_geometric_1979}
% {\sc N.~Fenichel},
% Geometric singular perturbation theory for ordinary differential equations,
% {\it J. Differ. Equations} {\bf 31} (1979), 53--98.
% 
% %\bibitem{FM87}
% %{\sc C.~Fressengeas and A.~Molinari},
% %Instability and localization of plastic flow in shear at high strain rates,
% %{\it J.  Mech. Physics of Solids} {\bf 35} (1987), 185--211.
% 
% \bibitem{HPS_1977}
% {\sc M.W. Hirsch, C.C. Pugh, and M. Shub},
% {\it Invariant Manifolds}, LNM {\bf 583}, (Springer-Verlag, New York/Heidelberg/Berlin 1977)
% 
% \bibitem{HN77}
% {\sc J.W.~Hutchinson and K.W.~Neale},
% Influence of strain-rate sensitivity on necking under uniaxial tension,
% {\it  Acta Metallurgica} {\bf 25} (1977), 839-846.
% 
% \bibitem{KOT14}
% {\sc Th.~Katsaounis, J.~Olivier, and A.E.~Tzavaras},
% Emergence of coherent localized structures in shear deformations of temperature dependent fluids,
% {\it Archive for Rational Mechanics and Analysis} {\bf 224} (2017), 173--208.
% 
% \bibitem{KT09}
% {\sc Th. Katsaounis and A.E.~Tzavaras},
% Effective equations for localization and shear band formation,
% {\it SIAM J. Appl. Math.}  {\bf 69} (2009), 1618--1643.
% 
% \bibitem{KLT_2016}
% {\sc Th. Katsaounis, M.-G. Lee, and A.E. Tzavaras},
% Localization in inelastic rate dependent shearing deformations,
% {\it J. Mech. Phys. of Solids} {\bf 98} (2017), 106--125.
% 
% \bibitem{KUEHN_2015}
% {\sc C.~ Kuehn}, 
% {\it Multiple time scale dynamics}, Applied Mathematical Sciences, Vol. {\bf 191} (Springer Basel 2015).
% 
% \bibitem{LT16}
% {\sc M.-G.~Lee and A.E.~Tzavaras},
% Existence of localizing solutions in plasticity via the geometric singular perturbation theory,
% {\it Siam J. Appl. Dyn. Systems} {\bf 16} (2017), 337--360.
% 
% \bibitem{KLT_HYP2016}
% {\sc M.-G. Lee, Th. Katsaounis, and A.E. Tzavaras},
% Localization of Adiabatic Deformations in Thermoviscoplastic Materials, In Proceedings of the 16th International Conference on Hyperbolic Problems: Theory, Numerics, Applications (HYP2016), to appear.
% 
% %
% % \bibitem{jones_geometric_1995}
% % {\sc C.~K. R.~T. Jones},
% % Geometric singular perturbation theory, in {\it Dynamical systems}, LNM {\bf 1609} (Springer Berlin Heidelberg 1995) 44--118.
% %
% %
% 
% %
% % % \bibitem{perko_differential_2001}
% % % {\sc L.~Perko},
% % % {\it Differential equations and dynamical systems 3rd. ed.}, TAM {\bf 7} (Springer-Verlag New York 2001).
% %
% 
% \bibitem{shawki_shear_1989}
% {\sc T.G. Shawki and R.J. Clifton},
% Shear band formation in thermal viscoplastic materials,
% % {\it Mechanics of Materials}
% {\it Mech. Mater.}
% {\bf 8 } (1989), 13--43.
% 
% \bibitem{Sz1991}
% {\sc P.~Szmolyan},
% Transversal heteroclinic and homoclinic orbits in singular perturbation problems,
% {\it J. Differ. Equations}
% {\bf 92} (1991), 252--281.
% 
% \bibitem{Tz_1986}
% {\sc A.E. Tzavaras},
% Shearing of materials exhibiting thermal softening or temperature dependent viscosity,
% {\em Quart.  Applied Math.} {\bf 44} (1986), 1--12.
% 
% \bibitem{Tz_1987}
% {\sc A.E. Tzavaras},
% Effect of thermal softening in shearing of strain-rate dependent materials.
% {\em Archive for Rational Mechanics and Analysis}, {\bf 99} (1987), 349--374.
% 
% \bibitem{tzavaras_plastic_1986}
% {\sc A.E. Tzavaras},
% Plastic shearing of materials exhibiting strain hardening or strain softening,
% % {\it Archive for Rational Mechanics and  Analysis}
% {\it Arch. Ration. Mech. Anal.}
% {\bf 94} (1986), 39--58.
% 
% %\bibitem{tzavaras_strain_1991}
% %{\sc A.E. Tzavaras},
% %Strain softening in viscoelasticity of the rate type.
% %{\it J. Integral Equations Appl.} {\bf  3}  (1991), 195--238.
% 
% \bibitem{tzavaras_nonlinear_1992}
% %\leavevmode\vrule height 2pt depth -1.6pt width 23pt,
% {\sc A.E. Tzavaras},
% Nonlinear analysis techniques for shear band formation at high strain-rates,
% % {\it Applied Mechanics Reviews}
% {\it Appl. Mech. Rev.}
% {\bf  45} (1992), S82--S94.
% 
% 
% 
% %
% % \bibitem{clifton_critical_1984}
% % {\sc R.~J. Clifton, J.~Duffy, K.~A. Hartley, and T.~G. Shawki},
% % On critical conditions for shear band formation at high strain rates.
% % % {\it Scripta Metallurgica}
% % {\it Scripta. Metall. Mater.}
% % {\bf 18} (1984), 443--448.
% %
% 
% 
% %
% % \bibitem{freistuhler_spectral_2002}
% % {\sc H.~Freistühler and P.~Szmolyan},
% % {Spectral stability of small shock waves},
% % {\it Arch. Ration. Mech. Anal.}
% % {\bf 164} (2002), 287--309.
% % %   \href{http://dx.doi.org/10.1007/s00205-002-0215-8}{doi:\nolinkurl{10.1007/s00205-002-0215-8}},
% % %   \url{http://dx.doi.org/10.1007/s00205-002-0215-8}.
% % \bibitem{fressengeas_instability_1987}
% % {\sc C.~Fressengeas, A.~Molinari},
% % {Instability and localization of plastic flow in shear at high strain rates},
% % {\it J. Mech. Phys. of Solids}
% % {\bf 35} (1987), 185--211.
% %
% % \bibitem{gasser_geometric_1993}
% % {\sc I.~Gasser and P.~Szmolyan},
% % {A geometric singular perturbation analysis of detonation and deflagration waves},
% % {\it {SIAM} J. Math. Anal.}
% % {\bf 24} (1993), 968--986.
% % %   \href{http://dx.doi.org/10.1137/0524058}{doi:\nolinkurl{10.1137/0524058}},
% % %   \url{http://dx.doi.org/10.1137/0524058}.
% % \bibitem{ghazaryan_traveling_2007}
% % {\sc A.~Ghazaryan, P.~Gordon, and C.~K. R.~T. Jones},
% % {Traveling waves in porous media combustion: uniqueness of waves for small thermal diffusivity},
% % {\it J. Dyn. Differ. Equ.}
% % {\bf 19} (2007), 951--966.
% % %   \href{http://dx.doi.org/10.1007/s10884-007-9079-9}{doi:\nolinkurl{10.1007/s10884-007-9079-9}},
% % %   \url{http://dx.doi.org/10.1007/s10884-007-9079-9}.
% %
% %
% 
% % \bibitem{MC_1987}
% % {\sc A.~Molinari and R.~J. Clifton},
% % Analytical characterization of shear localization in thermoviscoplastic materials,
% % {\it Journal of Applied Mechanics}
% % {\it J. Appl. Mech.}
% % {\bf 54} (1987), 806--812.
% %
% %
% % \bibitem{jones_geometric_1995}
% % {\sc C.~K. R.~T. Jones},
% % Geometric singular perturbation theory, in {\it Dynamical systems}, LNM {\bf 1609} (Springer Berlin Heidelberg 1995) 44--118.
% %
% %
% %
% %
% %
% %
% 
% %
% % \bibitem{KUEHN_2015}
% % {\sc C.~ Kuehn},
% % {\it Multiple time scale dynamics}, Applied Mathematical Sciences, Vol. {\bf 191} (Springer Basel 2015).
% %
% 
% %
% % \bibitem{perko_differential_2001}
% % {\sc L.~Perko},
% % {\it Differential equations and dynamical systems 3rd. ed.}, TAM {\bf 7} (Springer-Verlag New York 2001).
% %
% %
% 
% 
% 
% %
% % \bibitem{shawki_energy_1994}
% % {\sc T.~G. Shawki},
% % {An Energy Criterion for the Onset of Shear Localization in Thermal Viscoplastic Materials, Part II: Applications and Implications}, {\it ASME. J. Appl. Mech.}
% % {\bf 61} (1994), 538--547.
% %
% %
% % \bibitem{SS_2004}
% % {\sc S. Schecter and P. Szmolyan}
% % Composite waves in the Dafermos regularization.
% % {\it J. Dynamics Diff. Equations} {\bf 16} (2004), 847-867.
% %
% 
% %
% % \bibitem{wiggins_normally_1994}
% % {\sc S.~Wiggins},
% % {\it Normally hyperbolic invariant manifolds in dynamical  systems}, AMS {\bf 105} (Springer-Verlag New York 1994).
% %
% \bibitem{wiggins_normally_1994}
% {\sc S.~Wiggins}, 
% {\it Normally hyperbolic invariant manifolds in dynamical  systems}, AMS {\bf 105} (Springer-Verlag New York 1994).
% 
% \bibitem{wright_survey_2002}
% {\sc T.W. Wright},
% {\it The Physics and Mathematics of Shear Bands.} (Cambridge Univ. Press 2002).
% %
% % \bibitem{xiao_stability_2003}
% % {\sc L.~Xiao-Biao and S.~ Schecter},
% % {Stability of self-similar solutions of the {D}afermos regularization of a system of conservation laws},
% % {SIAM J. Math. Anal.}
% % {\bf 35} (2003), 884--921.
% 
% %\bibitem{WF83}
% %{\sc F.H. Wu and L.B. Freund},
% %Deformation trapping due to thermoplastic instability in one-dimensional wave propagation,
% %{\it J. Mech. Phys. of Solids} {\bf  32} (1984), 119-132.
% 
% \bibitem{zener_effect_1944}
% {\sc C.~Zener and J.~H. Hollomon},
% Effect of strain rate upon plastic flow of steel,
% % {\it  Journal of Applied Physics}
% {\it J. Appl. Phys.}
% {\bf 15} (1944), 22--32.

\end{thebibliography}
\end{document}

% \hrulefill
% \bibitem{dafermos_adiabatic_1983}
% {\sc C.~M. Dafermos and L.~Hsiao},
% Adiabatic shearing of incompressible fluids with temperature-dependent viscosity.
% {\it Quart.  Applied Math.} {\bf 41} (1983), 45--58.

% \bibitem{katsaounis_effective_2009}
% {\sc Th. Katsaounis and A.E.~Tzavaras},
%  Effective equations for localization and shear band formation,
%  {\it SIAM J. Appl. Math.}  {\bf 69} (2009), 1618--1643.

% \bibitem{schecter_undercomp_2002}
% {\sc S.~Schecter},
% {Undercompressive shock waves and the {D}afermos regularization},
% {\it Nonlinearity}
% {\bf 15} (2002), 1361--1377.
% \bibitem{deng_homoclinic_1990}
% {\sc B.~Deng},
% {Homoclinic bifurcations with nonhyperbolic equilibria},
% {\it {SIAM} J. Math. Anal.}
% {\bf 21} (1990),  693--720.
% %   \href{http://dx.doi.org/10.1137/0521037}{doi:\nolinkurl{10.1137/0521037}},
% %   \url{http://dx.doi.org/10.1137/0521037}.

%
% \bibitem{ghazaryan_travelling_2015}
% {\sc A.~Ghazaryan, V.~Manukian, and S.~Schecter},
% {Travelling waves in the holling-tanner model with weak diffusion},
% {\it Proc. R. Soc. A}
% {\bf 471} (2015), 20150045, 16.
% %   \href{http://dx.doi.org/10.1098/rspa.2015.0045}{doi:\nolinkurl{10.1098/rspa.2015.0045}},
% %   \url{http://dx.doi.org/10.1098/rspa.2015.0045}.
%
% \bibitem{gucwa_geometric_2009}
% {\sc I.~Gucwa and P.~Szmolyan},
% {Geometric singular perturbation analysis of an autocatalator model},
% {\it Discrete Contin. Dyn. Syst. Ser. S}
% {\bf 2} (2009), 783--806.
% %   \href{http://dx.doi.org/10.3934/dcdss.2009.2.783}{doi:\nolinkurl{10.3934/dcdss.2009.2.783}},
% %   \url{http://dx.doi.org/10.3934/dcdss.2009.2.783}.
%
% \bibitem{huber_geometric_2005}
% {\sc A.~Huber and P.~Szmolyan},
% {Geometric singular perturbation analysis of the yamada model},
% {\it {SIAM} J. Appl. Dyn. Syst.}
% {\bf 4} (2005), 607--648.
% %   \href{http://dx.doi.org/10.1137/040604820}{doi:\nolinkurl{10.1137/040604820}},
% %   \url{http://dx.doi.org/10.1137/040604820}.
% \bibitem{jones_construction_1991}
% {\sc C.~K. R.~T. Jones, N.~Kopell, and R.~Langer},
% {Construction of the {FitzHugh}-nagumo pulse using differential forms.} In: {\it Patterns and dynamics
%   in reactive media},
% IMA Volumes Math Appl 37, Springer, 1989, 101--115.
%
% \bibitem{jones_tracking_1994}
% {\sc C.~K. R.~T. Jones and N.~Kopell},
% {Tracking invariant manifolds with differential forms in singularly perturbed systems},
% {\it J. Differ. Equations}
% {\bf 108} (1994), 64--88.
% %   \href{http://dx.doi.org/10.1006/jdeq.1994.1025}{doi:\nolinkurl{10.1006/jdeq.1994.1025}},
% %   \url{http://dx.doi.org/10.1006/jdeq.1994.1025}.
% \bibitem{kaper_primer_2001}
% {\sc T.~J. Kaper and C.~K. R.~T. Jones},
% {A primer on the exchange lemma for fast-slow systems.} In: {\it Multiple-time-scale dynamical systems},
% IMA Volumes Math Appl 122, Springer, 1997, 65--87.
% \bibitem{katsaounis_localization_2011}
% {\sc Th. Katsaounis and A.E.~Tzavaras},
% Localization and shear bands in high strain-rate plasticity.
% In: {\it Nonlinear conservation laws and  applications}, A.~Bressan, G.-Q.~Chen, M.~Lewicka, D.~Wang, eds;
% IMA Volumes Math Appl 153, Springer, 2011, 365--377.
% \bibitem{popovic_geometric_2004}
% {\sc N.~Popović and P.~Szmolyan},
% {A geometric analysis of the lagerstrom model problem},
% {\it J. Differ. Equations}
% {\bf 199} (2004), 290--325.
% %   \href{http://dx.doi.org/10.1016/j.jde.2003.08.004}{doi:\nolinkurl{10.1016/j.jde.2003.08.004}},
% %   \url{http://dx.doi.org/10.1016/j.jde.2003.08.004}.
% \bibitem{bates_existence_1997}
% {\sc P.~W. Bates, P.~C. Fife, R.~A. Gardner, and C.~K. R.~T. Jones},
% {The existence of travelling wave solutions of a generalized phase-field model},
% {\it {SIAM} J. Math. Anal.}
% {\bf 28} (1997), 60--93.
% %   \href{http://dx.doi.org/10.1137/S0036141095283820}{doi:\nolinkurl{10.1137/S0036141095283820}},
% %   \url{http://dx.doi.org/10.1137/S0036141095283820}.
%
% \bibitem{beck_electrical_2008}
% {\sc M.~Beck, C.~K. R.~T. Jones, D.~Schaeffer, and M.~Wechselberger},
% {Electrical waves in a one-dimensional model of cardiac tissue},
% {\it {SIAM} J. Appl. Dyn. Syst.}
% {\bf 7} (2008),  1558--1581.
% %   \href{http://dx.doi.org/10.1137/070709980}{doi:\nolinkurl{10.1137/070709980}},
% %   \url{http://dx.doi.org/10.1137/070709980}.

% \bibitem{schecter_exchange_2008}
% {\sc S.~Schecter},
% {Exchange lemmas. i. deng's lemma},
% {\it J. Differ. Equations}
% {\bf 245} (2008), 392--410.
% %   \href{http://dx.doi.org/10.1016/j.jde.2007.08.011}{doi:\nolinkurl{10.1016/j.jde.2007.08.011}},
% %   \url{http://dx.doi.org/10.1016/j.jde.2007.08.011}.
%
% \bibitem{schecter_exchange_2008-1}
% {\sc S.~Schecter}, {Exchange lemmas. {ii}. general exchange lemma},
% {\it J. Differ. Equations}
% {\bf 245} (2008), 411--441.
% %   \href{http://dx.doi.org/10.1016/j.jde.2007.10.021}{doi:\nolinkurl{10.1016/j.jde.2007.10.021}},
% %   \url{http://dx.doi.org/10.1016/j.jde.2007.10.021}.

% \bibitem{wright_stress_1987}
% {\sc Thomas~W. Wright and John~W. Walter},
% On stress collapse in adiabatic shear bands,
% {\it J. Mech. Phys. of Solids} {\bf 35} (1987),
%  701--720.
%
% \bibitem{WF83}
% {\sc F.H. Wu and L.B. Freund},
% Deformation trapping due to thermoplastic instability in one-dimensional wave propagation,
% {\it J. Mech. Phys. of Solids} {\bf  32} (1984), 119-132..
%
% \bibitem{AKS87}
% L.~Anand, K.H.~Kim and T.G.~Shawki,
% Onset of shear localization in viscoplastic solids,
% {\it J. Mech. Phys. Solids}
% {\bf 35} (1987), 407-429.
%
% \bibitem{BC04}
% {\sc Th.~ Baxevanis and N.~Charalambakis},
% The role of material non-homogeneities on the formation and evolution of strain non-uniformities in thermoviscoplastic shearing,
% {\it Quart. Appl. Math.} {\bf 62} (2004), . 97-116.
%
% \bibitem{baxevanis_adaptive_2010}
% {\sc Th~H. Baxevanis, Th~Katsaounis, and A.~E. Tzavaras},
% Adaptive finite element computations of shear band formation,
%   {\it Math. Models  Methods Appl. Sci.} {\bf 20}  (2010),  423--448.
%
% \bibitem{BD02}
% {\sc T.J.~Burns and M.A.~Davies},
% On repeated adiabatic shear band formation during high speed machining,
% {\it International Journal of Plasticity} {\bf 18 } (2002),  507-530.
%
% \bibitem{CB99}
% {\sc L.~ Chen and R.C.~Batra },
% The asymptotic structure of a shear band in mode-II deformations.
% {\it International Journal of Engineering Science} {\bf 37} (1999),  895-919.
%
% \bibitem{estep_2001}
% {\sc Donald~J Estep, Sjoerd M~Verduyn Lunel, and Roy~D Williams},
% {Analysis of Shear Layers in a Fluid with Temperature-Dependent Viscosity},
%  {\it  J. Comp. Physics}  {\bf 173} (2001), 17--60.
%
% \bibitem{KCS85}
% {\sc R.W. Klopp. R.J. Clifton, and T.G. Shawki},
% Pressure-shear impact and the dynamic viscoplastic response of metals,
% {\it Mechanics of Materials} {\bf 4} (1985), 375-385.

% \bibliography{dynamical}
\end{thebibliography}
\end{document}


\end{thebibliography}

\end{document}	

\pagebreak