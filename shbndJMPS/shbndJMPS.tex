%%%%%%%%%%%%%%%%%%%%%%%%%
%
%
%
%     Dynamic shear band formation in high strain-rate plasticity of metals
%
%         contents:  Linearized stability, self-similar localizing solutions,  survey for JMPS 
%
%          Th. Katsaounis, M-G Lee and A.Tzavaras
%
%
%
%
%
%%%%%%%%%%%%%%%%%%%%%%%%%


\documentclass[a4paper,11pt]{article}

\usepackage[margin=3cm]{geometry}
\usepackage{setspace}
\onehalfspacing
%\doublespacing
%\usepackage{authblk}
\usepackage{amsmath}
\usepackage{amssymb}
\usepackage{amsthm}
% \usepackage{calrsfs}
\usepackage[notcite,notref]{showkeys}

\usepackage{psfrag}
\usepackage{graphicx,subfigure}
\usepackage{color}
\def\red{\color{red}}
\def\blue{\color{blue}}
%\usepackage{verbatim}
% \usepackage{alltt}
%\usepackage{kotex}



\usepackage{enumerate}


%%%%%%%%%%%%%%%%%%

\newcommand{\R}{\mathbb{R}}
\newcommand{\del}{\partial}
\newcommand{\sg}{\sigma}
\newcommand{\Sg}{\Sigma}
\newcommand{\tht}{\theta}
\newcommand{\Th}{\Theta}
\newcommand{\gm}{\gamma}
\newcommand{\Gm}{\Gamma}
\newcommand{\gk}{\kappa}
\newcommand{\ga}{\alpha}
\newcommand{\gb}{\beta}
\newcommand{\gd}{\delta}
\newcommand{\gee}{\epsilon}
\newcommand{\eps}{\varepsilon}
\newcommand{\gl}{\lambda}
\newcommand{\gz}{\zeta}
\newcommand{\oot}{\frac{1}{T}}
\newcommand{\oott}{\frac{1}{T^2}}
\newcommand{\sqt}{\sqrt{T}}
\newcommand{\dxx}{\partial_{xx}}
\newcommand{\dyy}{\partial_{yy}}
\newcommand{\cL}{{\cal L}}


%%%%%%%%%%%%%%%%


%%%%%%%%%%%%%% MY DEFINITIONS %%%%%%%%%%%%%%%%%%%%%%%%%%%

\def\tr{\,\textrm{tr}\,}
\def\div{\,\textrm{div}\,}
\def\sgn{\,\textrm{sgn}\,}
\def\l{{\ell}}



\def\bG{{\bar{\Gamma}}}
\def\bV{{\bar{V}}}
\def\bTh{{\bar{\Theta}}}
\def\bS{{\bar{\Sigma}}}
\def\bU{{\bar{U}}}

\def\bg{{\bar{\gamma}}}
\def\bv{{\bar{v}}}
\def\bth{{\bar{\theta}}}
\def\bs{{\bar{\sigma}}}
\def\bu{{\bar{u}}}
\def\bph{{\bar{\varphi}}}


\def\tg{{\tilde{\gamma}}}
\def\tv{{\tilde{v}}}
\def\tth{{\tilde{\theta}}}
\def\ts{{\tilde{\sigma}}}
\def\tu{{\tilde{u}}}
\def\tph{{\tilde{\varphi}}}

\def\hg{{\hat{\gamma}}}
\def\hv{{\hat{v}}}
\def\hth{{\hat{\theta}}}
\def\hs{{\hat{\sigma}}}
\def\hu{{\hat{u}}}
\def\hph{{\hat{\varphi}}}



\def\dtg{{\dot{\tilde{\gamma}}}}
\def\dtv{{\dot{\tilde{v}}}}
\def\dtth{{\dot{\tilde{\theta}}}}
\def\dts{{\dot{\tilde{\sigma}}}}
\def\dtu{{\dot{\tilde{u}}}}
\def\dtph{{\dot{\tilde{\varphi}}}}

\def\dpp{\dot{p}}
\def\dqq{\dot{q}}
\def\drr{\dot{r}}
\def\dss{\dot{s}}

\def\BO{{{O}}}
\def\lio{{{o}}}

\newcommand{\tcr}{\textcolor{red}}
\newcommand{\tcb}{\textcolor{blue}}
\newcommand{\ubar}[1]{\text{\b{$#1$}}}
\newtheorem{theorem}{Theorem}
\newtheorem{lemma}{Lemma}[section]
\newtheorem{proposition}{Proposition}[section]
\newtheorem{corollary}{Corollary}[section]
\newtheorem{definition}{Definition}[section]
\newtheorem{claim}{Claim}

\newcounter{mycounter}
\newtheorem{step}{Step}[mycounter]

\theoremstyle{remark}
\newtheorem{remark}{Remark}[section]


%%%%%%%%%%%%%%%%%%%%%%%%%%%%%%%%%%%%%%%%%%%%%%%%%%%%%%%%%%
\begin{document}
\title{Dynamic shear band formation in high strain-rate plasticity of metals}
\author{Theodoros Katsaounis\footnotemark[1]\ \footnotemark[2]\ \footnotemark[3]
\and Min-Gi Lee\footnotemark[1]
\and Athanasios E. Tzavaras\footnotemark[1]\  \footnotemark[4]}
\date{}

\maketitle
\renewcommand{\thefootnote}{\fnsymbol{footnote}}
\footnotetext[1]{Computer, Electrical and Mathematical Sciences \& Engineering Division, King Abdullah University of Science and Technology (KAUST), Thuwal, Saudi Arabia}
\footnotetext[2]{Institute of Applied and Computational Mathematics, FORTH, Heraklion, Greece}
\footnotetext[3]{Department of Mathematics and Applied Mathematics, University of Crete, Heraklion, Greece}
\footnotetext[4]{Corresponding author : \texttt{athanasios.tzavaras@kaust.edu.sa}}
%\footnotetext[4]{Research supported by the King Abdullah University of Science and Technology (KAUST) }
\renewcommand{\thefootnote}{\arabic{footnote}}

 \begin{abstract}
 This article is devoted to the explanation of the onset of localization and the dynamic formation of shear band in high strain-rate plasticity of metals. 
 We consider a model from thermoviscoplasticity encompassing the basic mechanisms of thermal softening, strain hardening and strain-rate hardening 
 that contribute to the formation of shear bands. We first investigate the stability of the uniform shearing solutions.
 Since these are time-dependent, the linearized theory has to account for the behavior of  non-autonomous systems.
 We present a rigorous theory that illustrates the behavior of  linearized modes around the time dependent base solutions. 
 The analysis provides a criterion for linearized stability or instability of uniform shear. Next we focus on the parameter range of instability,
 and investigate the development of shear bands in the nonlinear regime. We establish a class of localizing self-similar solutions 
 capturing  how shear bands develop beyond the initial stage. They are obtained by transforming the problem to the construction
 of a heteroclinic orbit and using the geometric singular perturbation theory from dynamical systems. 
 The heteroclinic orbits are constructed numerically, and provide instances of  localizing solutions.
 \end{abstract}

 \tableofcontents
% \pagebreak


\vfil\eject


\section{Introduction}

Shear bands are narrow zones of intensely localized shear that are formed during the high speed plastic deformations of metals \cite{ZH44, Clifton90,Wright02}. 
They often precede rupture and are one of the striking instances of material instability leading to failure.  Considerable attention has been devoted to the
problem of shear band formation in both the mechanics and the applied mathematics literature, and section \ref{mathmodel} is devoted to outlining
a basic model from thermoviscoplasticity
\begin{equation}
  \label{sbeq}
  \begin{aligned}
    & v_{t} =  \kappa \theta_{ x x} +  \sigma_{x},\\
    & \theta_{t} =  \sigma \gamma_{t}, \\
    & \gamma_{t} = v_{x},
  \end{aligned}
\end{equation}
\begin{align}
&  \sigma =  f(\theta, \gamma, u) = \theta^{-\alpha} \gamma^{m} u^{n} \, , \quad \quad \qquad \text{$u = \gamma_t$}. \label{PL0}
\end{align}
that is extensively studied in the present work. The model describes the adiabatic plastic shearing of an plate for
a thermoviscoplastic material obeying the constitutive law \eqref{PL0}. The latter can be viewed as a yield surface or a plastic flow rule, and encompasses the basic
mechanisms entering in theories for the explanation of shear bands \cite{ZH44,Clifton90}.
The parameters $\alpha>0$, $m>0$ and $n>0$ measure respectively the degrees of thermal softening, strain hardening and  strain-rate hardening and $n \ll 1$
is typically small, \cite{Clifton90}. 
Throughout, the material parameters $(\alpha,m,n)$ are restricted to the range
\begin{equation}
 \begin{aligned}
  \alpha>0\quad&\text{(thermal softening)},\\
  m>-1 \quad&\text{(strain softening/hardening)}, \\%\label{eq:a1}\\
  \red-\alpha+m<0 \quad&\text{(loss of hyperbolicity)},\\% \\%\label{eq:a3}\\
  n\ge0 \quad&\text{(strain rate sensitivity)}. %\label{eq:a2}\\
%   0< \lambda < \frac{2(\alpha-m-n)}{1+m+n}\left(\frac{1+m}{1+m+n}\right) \quad&\text{(localizing rate bound)}. %\label{eq:a4}
\end{aligned}\label{eq:paramrange}
\end{equation}
We complement the model by providing boundary conditions of prescribed shearing velocities,
\begin{equation} \label{BCOND1}
 v(t,0)=0, \quad v(t,1)=1, 
\end{equation}
and thermal insulation of the end plates,
\begin{equation} \label{BCOND2}
 \theta_x(t,0)=\theta_x(t,1)=0, \quad \text{in case of $\kappa\ne0$.}
\end{equation}



In the sequel, we will mostly focus on adiabatic deformations $\kappa = 0$. This hypothesis is quite reasonable for the initial development of shear bands,
which is the main focus here, and not very realistic once a shear band fully forms, due to the high temperature differences involved.
The adiabatic hypothesis leads to the system of partial differential equations
\begin{equation}
  \label{pls}
  \begin{aligned}
    & v_{t} = \del_x \Big ( \theta^{-\alpha}\gamma^{m}  v_{x}^n \Big )  ,\\
    & \theta_{t} =  \theta^{-\alpha}\gamma^{m} v_{x}^{n+1}   \\
    & \gamma_{t} = v_{x},  \\
%    & \sigma  = \theta^{-\alpha}\gamma^{m}\gamma_{t}^n \,
  \end{aligned}
\end{equation}
The variables $v$ is the shear velocity, $\theta$ the temperature and $\sigma$ the stress.
In the models \eqref{sbeq} or \eqref{pls} the elastic effects are ignored and $\gamma$ stands for the plastic strain. The reader is refered to
\cite{SC89} for models including the effects of elasticity and unloading.


\tcr{To Do -  Introduction should be brief - present the basic mechanism of shear band formation - present the issues that 
will be studied with a very general outline - no spoilers -defer the presentation for the sections }


 
\vfil\eject

 
 \section{Description of the shear band formation problem}
\label{mathmodel}

Shear bands occur during
high strain-rate plastic deformations of certain steels and other metal alloys. Instead of the shear strain distributing evenly across the loading region, 
it concentrates in a narrow band with a concurrent elevation of the temperature in the interior of the band, \cite{ZH44,HDH87}.
Shear bands are often precursors to rupture and their study has attracted considerable
attention including  experimental works  \cite{CCHD79,HDH87}, mechanical modeling 
and  linearized analysis studies  ({\it e.g.} \cite{CDHS84,FM87,MC87,SC89,Wright02} and references therein) and
nonlinear analysis investigations  \cite{DH83,Tzavaras87,BPV91}.

\subsection{Modeling shear bands}
Shear bands appear and propagate as one dimensional structures (up to interaction times), and
many investigations focus on the study of one-dimensional, simple shear. 
A specimen located in the $xy$-plain undergoes shear motion in the $y$-direction. The motion is described by the (plastic) shear strain
$\gamma(t,x)$, the strain rate $u(t,x) = \gamma_t (t,x)$, the velocity $v(t,x)$ in the shear direction, the temperature $\theta(t,x)$ and the shear stress
$\sigma (t,x)$ all defined in $(t,x)\in \mathbb{R}^+ \times \mathbb{R}$. It is described by the equations
\begin{equation} \label{mechmodel}
\begin{aligned}
 \gamma_t &= v_x  \\	
  v_t &= \sigma_x  
 \\
 \theta_t &= \kappa \theta_{xx} + \sigma v_x, 	\\
\end{aligned}
\end{equation}
which stand respectively for the kinematic compatibility equation, the balance of momentum and
the balance of energy equation. Here, the elastic effects are neglected and all strain is considered to be plastic, and 
a Fourier heat conduction is considered with $\kappa$ the thermal diffusivity.


Under shearing most materials deform in a uniform fashion until they break. By contrast, in high strain-rate deformations of
certain steels, it is observed that nonuniformities develop in the  plastic strain and  localize in a narrow region, called shear band; see 
 Fig. \ref{ShearFlow} for a caricature of shear band forming.
\begin{figure}
\centering
\vspace{-0.1cm}
\includegraphics[scale=0.5]{Shflow-w-sb}
\vspace{0.1cm}
\caption{Uniform shear versus shear band}.
\label{ShearFlow}
\end{figure}
It was recognized by Zener and Hollomon \cite{ZH44} that the  high deformation speed has two effects: 
First, an increase in the deformation speed changes the deformation conditions from isothermal 
to nearly adiabatic. Under such conditions the combined effect of thermal softening and strain hardening 
tends to produce net softening response. (Indeed, experimental observations of shear bands 
are  typically associated  with strain softening response -- past a critical strain -- 
of the measured stress-strain curve \cite{CDHS84}.)
Second, strain rate has an effect {\it per se}, and needs to be included in the constitutive modeling.

Both effects are captured by modeling shear band formation via constitutive models
within the framework  of thermoviscoplasticity: 
\begin{equation}
\label{thermovpcl}
\sigma = f(\theta, \gamma, \gamma_t) \quad \mbox{where} \quad f_p (\theta, \gamma, p)  > 0 \, .
\end{equation}
Equation \eqref{thermovpcl} may be viewed as a yield surface or,  upon inverting it, as a plastic flow rule. This suggests the terminology: 
the material exhibits thermal softening at state variables $(\theta, \gamma, p)$ 
where $f_\theta(\theta, \gamma, p) < 0$, strain hardening at state variables where $f_\gamma(\theta, \gamma, p) > 0$ , and strain softening when $f_\gamma(\theta, \gamma, p) < 0$. 
The slopes of $f$ measure the degree of thermal softening, strain hardening (or softening) 
and strain-rate sensitivity, respectively. 
The difficulty of performing high strain-rate experiments causes uncertainty as to the specific form of the constitutive form of the stress.  The power law
\begin{align}
%&  \sigma =  \theta^{-\alpha} \gamma^{m} \gamma_{t}^{n}, \quad & &  \text{ Power Law }, \label{PL0}\\
%& \sigma = e^{-\ga\tht} v_x^n, \quad & & \text{ Exponential Law} \label{ARL0}  \, .
&  \sigma =  \theta^{-\alpha} \gamma^{m} \gamma_{t}^{n}, 
 \label{PL0}
\end{align}
is a widely used relation, obtained by fitting experimental data, where the parameter $\alpha>0$ 
measures the degree of thermal softening, $m>0$ measures the degree of strain hardening (or $m<0$ in case of softening plastic flow), while $n>0$ measures strain-rate hardening and is typically small, $n \ll 1$, \cite{Clifton90}. 



 We summarize the equations describing the model. For the power law
the resulting system reads
\begin{equation}
  \label{PLS}
  \begin{aligned}
    & v_{t} =  \sigma_{x},\\
    & \theta_{t} = \kappa \theta_{ x x}  +  \sigma \gamma_{t}, \\
    & \gamma_{t} = v_{x},  \\
    & \sigma  = \theta^{-\alpha}\gamma^{m}\gamma_{t}^n \, .
  \end{aligned}
\end{equation}
The system \eqref{PLS} captures the simplest mechanism proposed for shear localization in
high-speed deformations of metals \cite{ZH44, Clifton90}, and an (isothermal) variant appears in early studies of necking \cite{HN77}. 
Very often attention is restricted to the adiabatic model $\kappa = 0$ which is appropriate for the initial
development of shear bands under very fast deformations. 


An alternative model is offered by the constitutive law
\begin{align}
& \sigma = e^{-\ga\tht} v_x^n, \quad & & \text{ exponential law} \label{ARL0}  \, .
\end{align}
which may be interpreted as describing a temperature dependent non-Newtonian fluid. 
The exponential law does not exhibit any strain hardening and \eqref{mechmodel} decouples and
leads to the simplified system
\begin{equation}
  \label{ARS}
  \begin{aligned}
    & v_{t} = \sigma_{x},\\
    & \theta_{t} = \kappa \theta_{ x x}  +  \sigma v_x \\
    & \sigma  = e^{-\ga\theta} v_x^n \, .
  \end{aligned}
\end{equation}



\subsection{Uniform shearing solutions} \label{sec:uss}

In the study of shear bands,  a special problem is often considered  where an infinite slab of material
is  sheared by prescribed constant velocity $V=1$ at the upper plate while the lower plate is held fixed.
This is described by setting the plates at $x=0, 1$ and imposing prescribed (normalized) velocities $v(t,0) = 0$,  $v(t,1) = 1$, respectively.
The plates are thermally insulated:  $ \theta_{x}(t,0) = \theta_{x}(t,1) = 0$.
For the heat flux $Q$ one either uses the adiabatic assumption $Q = 0$ (equivalently $\kappa = 0$)
or alternatively a Fourier  law $Q = \kappa \theta_{x}$ with thermal diffusivity
parameter $\kappa$.
Imposing adiabatic conditions projects the belief that, at high strain rates,
heat diffusion operates at a slower time scale than the time-scale of the development 
of a shear band. It appears a plausible assumption for the shear band initiation process, 
but not necessarily for the evolution of a developed band,
 due to the high temperature differences involved.
 

The model \eqref{PLS} admits  a special class of solutions describing uniform shearing: They emanate
from spatially uniform (constant) initial data $\gamma_0$ and $\theta_0$, and are obtained by the {\it ansatz} $\gamma_s (t) = t + \gamma_0 $ and $v_s (x) = x$ 
for the strain and velocity respectively. If so, the rest of variables are obtained 
upon solving the ordinary differential equation
\begin{equation}
\label{uss2}
\frac{d \theta_s }{dt} = \sigma_s = \theta_s^{-\alpha} (t + \gamma_0)^m \, , \quad \theta_s(0) = \theta_0 \, ,
\end{equation}
and read
\begin{equation} \label{uss}
\begin{aligned}
v_s (x)  &=x \, ,   \quad  \gamma_s(t) = t+\gamma_0,  \quad 
\\
\theta_s(t) &=  \left( \tfrac{1+\alpha}{1+m }\right )^{\frac{1}{1+\alpha}}  (t+\gamma_0)^{\frac{1 + m}{1+\alpha}} 
 \left( 1 +  \tfrac{1+m}{1+\alpha} \big (  \theta_0^{1+\alpha}  - \tfrac{1+\alpha}{1+m} \gamma_0^{1+m} \big ) \tfrac{1}{(t+\gamma_0)^{m+1}}  \right)^{\frac{1}{1+\alpha}}
 \\
%&=
%\left(\tfrac{1+\alpha}{1+m} (t+\gamma_0)^{1+m}  +
% \left (  \theta_0^{1+\alpha} - \tfrac{1+\alpha}{1+m} \gamma_0^{1+m}  \right )   \right)^{\frac{1}{1+\alpha}}\\
  \sigma_s(t)&=
   \left( \tfrac{1+\alpha}{1+m}\right )^{-\frac{\alpha}{1+\alpha}}  (t+\gamma_0)^{\frac{-\alpha + m}{1+\alpha}}  
   \left( 1 +  \tfrac{1+m}{1+\alpha} \big (  \theta_0^{1+\alpha}  - \tfrac{1+\alpha}{1+m} \gamma_0^{1+m} \big ) \tfrac{1}{(t+\gamma_0)^{m+1}}  \right)^{\frac{-\alpha}{1+\alpha}} .
\end{aligned}
\end{equation}
Expression $\left( 1 +  \tfrac{1+m}{1+\alpha} \big (  \theta_0^{1+\alpha}  - \tfrac{1+\alpha}{1+m} \gamma_0^{1+m} \big ) \tfrac{1}{(t+\gamma_0)^{m+1}}  \right) \rightarrow 1$ as $t \rightarrow\infty$.
Equation \eqref{uss}$_3$ describes the stress-strain curve $\sigma_s$ versus $\gamma_s$ for uniform shear.
The stress-strain curve is increasing when $\alpha < m$ but it is decreasing for large times when $\alpha > m$.
Here, we are interested in the regime $\alpha > m$ where thermal softening dominates strain hardening and produces net softening.



\subsection{Challenges posed by the localization problem}
\label{sec:challenges}

The system \eqref{pls} for $n=0$ is a first-order system. When $\alpha > m$, the initial value problem  has two purely imaginary eigenvalues in a regime 
of strain beyond the maximum of the stress-strain curve (see Appendix \ref{append:hadamard}). 
Accordingly, the linearized system (for $n=0$) around the uniform shearing solution \eqref{uss} exhibits {\it Hadamard instability}, see Appendix \ref{append:hadamard}.
\tcr{review and correct}


\tcr{To-Do  edit}

\bigskip

In this review work we focus on the initiation stage of shear bands and seek to understand the
mechanism of formation.
There are the following tasks to be carried out:
\begin{itemize}
\item To define a notion of stability (or instability) for the uniform shear; this task is complicated
by the time dependent nature of the base solutions.
\item To derive quantitative criteria for stability (or instability) and a mathematical mechanism pinpointing the onset of localization.
\item To describe the localization process and the formation of the shear band.
\end{itemize}

Naturally, one distinguishes two stages in the process: The initial stage of stability or instability 
of the uniform shearing solutions leads to questions in the realm of linearized analysis, but with
certain special features. Because the uniform shearing solution
is time dependent -- shear strain grows linearly -- the notion of stability has to be specified. An efficient definition due to
Molinari and Clifton \cite{MC87,FM87} espouses the idea of relative perturbation. That is, the uniform shear
is stable if perturbations grow slower than the base solutions, and unstable if perturbations grow
faster than the base solution. Following that framework leads to questions of linearized stability analysis 
for non-autonomous linear systems.
A rigorous analysis for a simplified model illustrating the difficulties can be found in \cite{Tzavaras92}.
In section \ref{STABAR}  we carry out this program for the Arrhenius model \eqref{ARS}.


The second stage of localization lies within the realm of nonlinear analysis, and the focal issue is how the catastrophic growth of high frequency oscillations resulting from Hadamard instability interacts with
the nonlinearity to form a coherent structure.
A quantitative criterion accounting for the nonlinear aspects of localization was recently derived in \cite{KT09},
based on ideas from the theory of relaxation system and the Chapman Enskog expansion. In later work
\cite{LKT17},  this analysis is supplemented with a study of self-similar localizing 
solutions that capture the nonlinear stage of localization. An account of this result is presented in Section \ref{sec:selfsimilar} using a new way to view the problem that simplifies matters a lot; the core results are cited from \cite{LKT17}, where the numerical computations and visualizations are available. % as well so that obtained solution is visualizable. to visualize the form of the obtained solutions.
% are also presented that indiicate the form of the obtained solutions.
 

\tcr{end edit}


\vfil\eject

\section{Relative perturbations and linearized stability analysis}
\label{sec:relative}

Since the uniform shearing solutions are time dependent their stability analysis immediately leads to issues with non-autonomous problems.
This is not surprising. Most notions of stability that are used in partial differential equations, but even notions of well--posedness 
(for instance hyperbolicity) are induced by stability analysis of steady in time structures like equilibria or periodic orbits.
The shear-band formation problem by its nature has to deal with stability of time-dependent solutions, and early nonlinear
analyses \cite{DH83,Tzavaras86a} indicated that the rate of growth (or decay) of perturbations have to be compared with the
rate of growth of the base solution. This was conceptualized  by Molinari-Clifton \cite{MC87} and Fressengeas-Molinari \cite{FM87} who
introduced the idea of relative perturbations; an even earlier variant of this idea appears in the context of necking
problems in Hutchinson-Neale \cite{HN77}.




\subsection{Formulation via relative perturbations}

Consider the  functions
\begin{equation}
\label{timescale}
\begin{aligned}
\gamma^*_s (t)  = \gamma_s (t)  = t + \gamma_0 , \quad 
\theta^*_s (t)  =\left( \frac{1+\alpha}{1+m}\right)^{\frac{1}{1+\alpha}} (t + \gamma_0)^{\frac{1+m}{1+\alpha}}
\\
\sigma^*_s (t) = \left( \frac{1+\alpha}{1+m}\right)^{\frac{-\alpha}{1+\alpha}} (t + \gamma_0)^{\frac{-\alpha+m}{1+\alpha}} 
= (\theta^*_s)^{-\alpha} (\gamma^*_s)^m \, ,
\end{aligned}
\end{equation}
describing the leading order in time of the uniform shearing solution \eqref{uss}, and introduce new dependent variables $u(t,x)$,
 $\Gamma (t, x)$, $\Theta (t,x)$ and $\Sigma(t,x)$ defined by 
\begin{equation}
\label{rescale}
 \begin{aligned}
 u(t,x) = v_x (t,x) \, , \quad
  \hat \Gamma (t,x) &= \frac{\gamma(t,x)}{\gamma^*_s(t)}, \quad \hat\Theta\big(t,x\big)=\frac{\theta(t,x)}{\theta^*_s (t)}, \quad 
  \hat\Sigma(t,x) &= \frac{\sigma(t,x)}{\sigma^*_s(t)} \, . \
   \end{aligned}
\end{equation}
These new variables are called in \cite{HN77,MC87,FM87}  {\it relative perturbations} and can be used to give a 
precise the definition for the stability for the uniform shearing solutions:
\begin{itemize}
\item
The uniform shear solution is {\it asymptotically stable} when the solution emanating from small perturbations
of \eqref{uss} satisfies that $(u, \hat\Gamma, \hat\Theta) \to (1,1,1)$ as time goes to infinity.
\item
The uniform shear solution is {\it unstable} if for small perturbations of \eqref{uss} the relative perturbations 
$(u, \hat\Gamma, \hat\Theta)$ drift away from $(1,1,1)$ as time increases.
\end{itemize}
As already mentioned, it is precisely this notion of stability that emerges from the nonlinear stability studies of shear bands \cite{DH83,Tzavaras86a} as well as
in linearized stability analyses by  Molinari and Clifton \cite{MC87,FM87} 
who coined the name {\it stability analysis of relative perturbations}. 

The problem of stability is presently resolved only for the special cases 
$m=0$ or $\alpha = 0$ for \eqref{pls};  these are cases that the system decouples and reduces:% to simpler models:
\begin{itemize}
\item[(i)] Case $m=0$: The uniform shear is linearly stable for $-\alpha + n > 0$ and linearly unstable for $-\alpha + n < 0$, \cite{FM87};
it is nonlinearly stable when $-\alpha + n > 0$, \cite{Tzavaras86a}.
\item[(ii)] Case $\alpha =0$, $m > -1$: The uniform shear is linearly stable when $m + n > 0$ and linearly unstable when $m+ n < 0$, \cite{FM87,Tzavaras92};
it is nonlinearly stable in the region $m + n > 0$, \cite{Tzavaras91}.
\end{itemize}
The general case for linearized stability will be considered here.

\tcr{To-Do:  Phrase better the literature review, also motivate the problem}

\subsection{Time rescaling and an equivalent system for relative perturbations}
\label{sec:time}

Next, we introduce a time rescaling. Observe that the functions in \eqref{timescale} satisfy
\begin{equation}
\label{prope}
\frac{d \theta^*_s}{dt} (t) = \sigma^*_s (t)  \, , \quad \frac{\sigma^*_s (t) \gamma^*_s (t)}{\theta^*_s (t)} = \frac{m+1}{1+\alpha} \, .
\end{equation}
We set 
\begin{equation}
\label{timeresc}
\tau (t) := \theta^*_s(t) \, , \quad \dot \tau (t) = \frac{d \theta^*_s}{dt} (t) = \sigma^*_s (t)
\end{equation}
and introduce a rescaled time $\tau$ with new dependent variables 
$\Gamma\big(\tau,x\big)$, $\Theta\big(\tau,x\big)$, $U\big(\tau,x\big)$, $\Sigma\big(\tau,x\big)$ defined by 
\begin{equation}
\label{defvar}
 \begin{aligned}
 u (t, x) &= v_x (t, x) = U(\tau(t), x) \, , \qquad \gamma(t,x) = \gamma^*_s(t) \, \Gamma (\tau (t), x)\, ,
 \\
\theta(t,x) &= \theta^*_s (t)  \; \Theta (\tau(t), x) \, , \qquad \qquad \sigma(t,x) = \sigma^*_s (t) \, \Sigma (\tau (t) , x)\,.
\end{aligned}
\end{equation}


Using \eqref{timescale}, \eqref{prope}, \eqref{timeresc} and \eqref{defvar}, the system \eqref{PLS} for  $\kappa = 0$ is
expressed in the form
\begin{equation} \label{plsrel}
 \begin{aligned}
  \partial_\tau U &= \Sigma_{xx},\\
  \partial_\tau \Gamma &= \frac{1}{\tau}\frac{1+\alpha}{1+m} \Big(U-\Gamma\Big),\\
  \partial_\tau \Theta &= \frac{1}{\tau}\Big(\Sigma U - \Theta\Big), \\
  \Sigma&=\Theta^{-\alpha}\Gamma^m U^n,
 \end{aligned}
\end{equation}
in terms of the rescaled solution $(U, \Gamma, \Theta, \Sigma)(\tau, x)$ defined via \eqref{defvar}. The new time-variable has the
behavior of the temperature in the uniform shear solution \eqref{uss}.
We note that \eqref{plsrel} is non-autonomous and may be thought of as a reaction-diffusin system but with non-diagonal diffusion.
In this formulation, the uniform shearing solutions are mapped  (asymptotically in time) to the constant state 
$\big(U, \Gamma, \Theta,\Sigma \big)=(1,1,1,1)$. 


\subsection{Linearization of \eqref{plsrel} around the uniform shear}

We consider \eqref{plsrel}
in $(\tau,x)\in \mathbb{R}^+\times [0,\pi]$ with boundary conditions
\begin{equation} \label{BCOND3}
 \Sigma_x(\tau,0)=\Sigma_x(\tau,\pi)=0.
\end{equation}
The domain $[0,\pi]$ of \eqref{plsrel} has been chosen for convenience and may be achieved via space rescaling.
The boundary condition \eqref{BCOND3} implies that
\begin{equation*}
 \frac{d}{d\tau}\int_0^\pi U(\tau,x) \: dx = 0, \quad \text{and we normalize } \quad  \frac{1}{\pi} \int_0^\pi U(\tau,x) \: dx = 1 \, .
\end{equation*}
Now we let
\begin{align*}
 U &= 1 + \delta \bar{u} + \mathcal{O}(\delta^2), & \Gamma &= 1 + \delta \bar\gamma + \mathcal{O}(\delta^2), &
 \Sigma &= 1 + \delta \bar\sigma + \mathcal{O}(\delta^2), & \Theta &= 1 + \delta \bar\theta + \mathcal{O}(\delta^2).
\end{align*}
By collecting the leading order terms we arrive at the linearized system around the base state $(1,1,1,1)$ (which coincides as the
reader recalls with the uniform shearing solution),
\begin{equation} \label{eq:linsystem}
 \begin{aligned}
  \partial_\tau \bar u &= \bar\sigma_{xx},\\
  \partial_\tau \bar\gamma &= \frac{1}{\tau}\frac{1+\alpha}{1+m}(\bar u-\bar\gamma),\\
  \partial_\tau \bar\theta &= \frac{1}{\tau}\Big(\bar\sigma+ \bar u -\bar\theta\Big),\\ % + \kappa(\tau)\bar\Theta_{xx},\\
  0&=\bar\sigma + \alpha\bar\theta -m\bar\gamma - n\bar u .
 \end{aligned}
\end{equation}
The boundary condition and the constraint equation yield
\begin{equation} \label{eq:linbdry}
 \bar\sigma_{x}(\tau,0)=\bar\sigma_{x}(\tau,\pi)=0, \quad \int_0^\pi \bar u(t,x) \: dx = 0.
\end{equation}
% In the sequel, we drop the bar over the variables and continue to study the linear system \eqref{eq:linsystem}-\eqref{eq:linbdry}.


\tcr{Edited Up to here}
% \subsection{subsection title?}

Next, the perturbations $(\bu, \bg, \bth, \bs)$ are expanded into series of periodic functions. For $k\ge0$ integer, we consider
\begin{align*}
 \bar u(\tau,x) &= u_0(\tau) + \sum_{k=1}^\infty  u_k(\tau)\cos(k x), &
\bar \gamma(\tau,x) &= \gamma_0(\tau) + \sum_{k=1}^\infty \gamma_k(\tau)\cos(k x),\\
\bar \theta(\tau,x) &= \theta_0(\tau) + \sum_{k=1}^\infty \theta_k(\tau)\cos(k x), &
 \bar\sigma(\tau,x) &= \sigma_0(\tau) + \sum_{k=1}^\infty \sigma_k(\tau)\cos(k x).
\end{align*}
Specifically, this is to consider cosine series for even function over $[-\pi,\pi]$. Justification is that any solution of \eqref{eq:linsystem}-\eqref{eq:linbdry} over $[0,\pi]$ is a restriction of the solution over $[-\pi,\pi]$ that is an even extension of itself. This is due to the reflexive symmetry of \eqref{eq:linsystem}-\eqref{eq:linbdry} that $(\bu_1(x)$, $\bg_1(x)$, $\bth_1(x)$, $\bs_1(x)):=(\bu(-x)$, $\bg(-x)$, $\bth(-x)$, $\bs(-x))$ is a solution if the $(\bu, \bg, \bth, \bs)$ is.  Hence, the form in the above gives an expedient but explicit choice without using imaginary numbers. %is expediently taken for even functions over $[-\pi,\pi]$. %as the cosine series with the period $2\pi$. 

Then, for each $k\ge0$ we obtain ordinary differential equations
\begin{equation} \label{eq:l-system}
 \begin{aligned}
  \partial_\tau u_k &= -k^2 \sigma_k,\\
  \partial_\tau\gamma_k &= \frac{1}{\tau}\frac{1+\alpha}{1+m}(u_k-\gamma_k),\\
  \partial_\tau\theta_k &= \frac{1}{\tau}\Big(\sigma_k+ u_k -\theta_k\Big),\\% - \kappa(\tau)k^2\Theta_k,\\
  0&=\sigma_k + \alpha\theta_k -m\gamma_k - n \, u_k .
 \end{aligned}
\end{equation}
We proceed to remove one of the variable by the fourth equation. Removing $\theta_k$ was chosen in our context. We collect the linear systems in the form 
\begin{equation}\label{eq:l-system2} \tag{$L_k$}
 \frac{d}{d\tau}\begin{pmatrix} u_k\\ \gamma_k \\ \sigma_k \end{pmatrix}
 = A_k(\tau) \begin{pmatrix} u_k\\ \gamma_k \\ \sigma_k \end{pmatrix}
\end{equation}
for each $k\ge0$, where
\begin{equation} 
\begin{aligned}
%  \begin{pmatrix} u_k\\ \gamma_k \\ \sigma_k \end{pmatrix}'
%  &= A_k(\tau) \begin{pmatrix} u_k\\ \gamma_k \\ \sigma_k \end{pmatrix}, \\
 A_k(\tau)&:=\begin{pmatrix}
   0 & 0 & -k^2\\
   0 & 0 & 0\\
   0 & 0 & -nk^2
  \end{pmatrix} 
  + \frac{1}{\tau}
  \begin{pmatrix}
   0 & 0 & 0\\
   \tfrac{1+\alpha}{1+m} & -\tfrac{1+\alpha}{1+m} & 0\\
   n+ \tfrac{-\alpha+m}{1+m} & m\tfrac{-\alpha+m}{1+m}& -(1+\alpha)
  \end{pmatrix}.  
%  A_k(\tau)&:=\overbrace{\begin{pmatrix}
%    0 & 0 & -k^2\\
%    0 & 0 & 0\\
%    0 & 0 & -nk^2
%   \end{pmatrix} }^{\triangleq A_{k,\infty}}
%   + \frac{1}{\tau}
%   \overbrace{\begin{pmatrix}
%    0 & 0 & 0\\
%    \tfrac{1+\alpha}{1+m} & -\tfrac{1+\alpha}{1+m} & 0\\
%    n+ \tfrac{-\alpha+m}{1+m} & m\tfrac{-\alpha+m}{1+m}& -(1+\alpha)
%   \end{pmatrix} }^{\triangleq A_{k,1}} 
\end{aligned}
\end{equation}
% for each $k\ge0$.
 %The vector $(U_k,\Gamma_k, \Sigma_k)^T$ will be denoted by $x_k(\tau)$. %Note we also defined the coefficient matrix $A_k(\tau)$, and the two constant matrices $A_{k,\infty}$ and $A_{k,1}$ as in the above.
% For $k=0$, 
% The $0$-th mode is treated separately. % at once and is not visited again in the sequel: We have
% \begin{align*}
%    \partial_\tau u_0 &= 0,\\
%   \partial_\tau\gamma_0 &= \frac{1}{\tau}\frac{1+\alpha}{1+m}(u_0-\gamma_0),\\
%   \partial_\tau\sigma_0 &= \frac{1}{\tau}\Big( -(1+\alpha)\sigma_0 + \big(n\frac{-\alpha+m}{1+m}\big)u_0 + \big(\frac{m(-\alpha+m))}{1+m}\big)\gamma_0\Big).
%   %\partial_\tau\Sigma_0 &= \frac{1}{\tau}\Big( -(1+\alpha)\Sigma_0 + \big(n-\alpha + \frac{m(1+\alpha)}{1+m}\big)U_0 + \big(m - \frac{m(1+\alpha)}{1+m}\big)\Gamma_0\Big).
% \end{align*}
% Since $u_0\equiv0$ from \eqref{eq:linbdry}, $\gamma_0(\tau)$ decays to $0$, and so do $\sigma_0(\tau)$ and $\theta_0(\tau)$.

\subsection{Objectives and challenges of the study}

Via following two consecutive sections, we derive the criterion for the linear stability, in the sense of relative perturbations formulated in the above, of the uniform shear. This is to complete the program of linear stability in case of the power law materials that has partly done by Fressengeas-Molinari \cite{FM87} and Tzavaras \cite{Tzavaras91} . For the material parameters in the valid range of \eqref{eq:paramrange}, we will conclude that
\begin{enumerate}
 \item uniform shear is linearly stable if $-\alpha+m+n>0$ and that
 \item uniform shear is linearly unstable if $-\alpha+m+n<0$.
\end{enumerate}
This picture is further consolidated by the discussions in Section  \ref{sec:selfsimilar}. When parameters falls into the unstable regime, we construct the localizing (in self-similar manner) solution of the nonlinear problem \eqref{plsrel}. 
\subsubsection{Effect of nonlinear viscosity}
Accompanied by the criterion is the precise asymptotic behaviors of the linear system parametrized by the given material parameters. The significance of monitoring the asymptotic behaviors is that they clearly show the contrasting behaviors from when viscosity effects ($n>0$) to when it does not ($n=0$). This is the focal issue of the following sections.
{\blue 
We have seen the loss of hyperbolicity of the system \eqref{PLS} passing the parameter value $-\alpha+m$ from positive to negative, along the uniform shear. The Hadamard Instability is often associated to the loss of hyperbolicity in the system with respect to the {\it constant mean states}. It can be characterized by the exponential growth of the amplitude in the linearized system around the constant state for given an oscillatory mode in the initial data. On the other hand, the instability exhibiting here, is in fact not exactly same, due to the time-dependent nature and has not been clarified. The purpose of this section is to clarify the instability exhibited in the linearized problem around the {\it growing mean state} with respect to the system hyperbolicity is lost.


As seen, ..., when $n=0$, system as a first order system changes type from hyperbolic to elliptic passing the value of $-\alpha+m$ from positive to negative. The ill-posedness of the elliptic problem as an initial value problem has coined the term Hadamard Instability, where, linearized around the constant mean state, the problem exhibits the catastrophic exponential growth of high oscillations. Since our base solution is not a constant mean state, the notion of stability and the ill-posedness behavior of our problem had to be re-examined. Nevertheless, the first result we will see is again the exponential behavior for the cases $-\alpha+m<0$ and $n=0$.}

This exponential behavior is drastically altered into the power behavior once the viscosity effects $(n>0)$. To contrast the two, let us first list the characterising properties of instability exhibited when $n=0$. %The exhibited is the Hadamard Instability.
\begin{enumerate}
 \item There is a growing solution $x_k(\tau)$ for each $k\ge1$, finding infinitely many growing modes.
 \item Each of those $x_k(\tau)$ grows exponentially.
 \item The rate of exponential growth diverges as $k \rightarrow \infty.$
\end{enumerate}
By contrast, the characterising properties of instability exhibited when $n>0$ is as follows. We say it is of Turing type instability.
\begin{enumerate}
 \item There is still a growing solution $x_k(\tau)$ for each $k\ge1$, finding infinitely many growing modes. However,
 \item Each of those $x_k(\tau)$ grows in polynomial order.
 \item The rate of polynomial growth is bounded from above. More precisely, the upper bound of the rate is dependent on $(\alpha,m,n)$ which is not surprising but is uniform in $k$.
\end{enumerate}
One may find all three properties above interesting as to the localization because one may say that the concentration phenomenon is, frequency-wisely speaking, a superposition of all oscillatory modes with equal strength. But the shear band formation is undoubtedly dynamic and nonlinear phenomenon. % and the picture provided by the linear stability analysis only accounts for the very initial onset of instability.

To draw attentions on what would be challenges in the linear stability of \eqref{eq:l-system2}, we close this section by presenting two counter examples in the below, where the time dependent linear system with negative eigenvalues gives rise to instability as $t \rightarrow \infty$. In the first example, 
$$ 
x'=\left(\begin{array}{cc} -1 + \frac{3}{2}\cos^2 t & 1-\frac{3}{2}\cos t \sin t \\ -1 -\frac{3}{2}\cos t \sin t & -1 + \frac{3}{2}\sin^2 t \end{array}\right)x, \quad x(t) = \left(\begin{array}{c} -\cos t\\ \sin t \end{array}\right) e^{\tfrac{t}{2}}.$$
In this example, $\lambda_\pm = \tfrac{-1 + i\sqrt{7}}{4}$, or $\textrm{Re}\, \lambda_\pm(t)<0$ but $x(t)$ grows exponentially. Next example is where we can specialize the fact that the coefficient matrix $A(t)$ is {\it frozen} in the limit, i.e. $A(t) \rightarrow A_\infty$ as $t\rightarrow \infty$. This is the case of our system \eqref{eq:l-system2} and those in \cite{FM87}. The problematic case is when $A_\infty$ has repeated eigenvalue and further the geometric and algebraic multiplicity discern. For two positive real numbers $r_1, r_2>0$ such that $r_1-r_2+1\ne0$,
$$ 
x'=\left(\begin{array}{cc} -\frac{r_1}{t} & 1 \\ 0 & -\frac{r_2}{t}\end{array}\right)x, \quad x(t) = \left(\begin{array}{c} C_1 t^{-r_1} + \frac{C_2}{r_1-r_2+1} t^{-r_2+1}\\ C_2t^{-r_2}\end{array}\right).$$
If $0<r_2<1$, the resonance leads to the the sub-linearly growing instability. %Later, we will see that this is indeed a case our problem falls into.

% 
% The study in \cite{FM87} and ours as well specializes the fact that the coefficient matrix $A(t)$ is {\it frozen} in the limit, i.e. $A(t) \rightarrow A_\infty$ as $\tau \rightarrow \infty$. We probe to apply the classical theorem of Coddington-Levinson, namely, when $A_\infty$ has distinct eigenvalues. Under suitable assumptions on the spectrum $\lambda_i(t)$, the asymptotic behaviors can be completely characterized. However, if some of the eigenvalues of $A_\infty$ are repeated, then it becomes subtle and one may suffer from the resonant. Later, we will see that this is the case for our linear system, where the eigenvalue $0$ is repeated. The following example shows this indeed is problematic. For two positive numbers $r_1, r_2>0$,
% $$ 
% x'=\left(\begin{array}{cc} -\frac{r_1}{t} & 1 \\ 0 & -\frac{r_2}{t}\end{array}\right)x \quad \text{then} \quad x(t) = \left(\begin{array}{c} C_1 t^{-r_1} + \frac{C_2}{r_1-r_2+1} t^{-r_2+1}\\ C_2t^{-r_2}\end{array}\right)$$
% 
% 
%  our perspective to come to the conclusion, a few points as to the stability problem of a non-autonomous system needs to be illustrated. First, we specialize the fact that the coefficient in \eqref{eq:l-system2} is {\it frozen} in the limit $\tau \rightarrow \infty$. An example of instability in the complementary case where
% $$ 
% A(t) = \left(\begin{array}{cc} -1 + \frac{3}{2}\cos^2 t & 1-\frac{3}{2}\cos t \sin t \\ -1 -\frac{3}{2}\cos t \sin t & -1 + \frac{3}{2}\sin^2 t \end{array}\right), \;\; \lambda_\pm = \frac{-1 + i\sqrt{7}}{4}, 
% \; \; x(t) = \left(\begin{array}{c} -\cos t\\ \sin t \end{array}\right) e^{\tfrac{t}{2}}$$
% shows that the spectral information  $\textrm{Re}\, \lambda_\pm(t)<0$ does not help much in determining the asymptotic stability.
% 
% In such a case where the coefficient is frozen in the limit however, the classical analysis, for example as in \cite{CL55}, shows that the spectral information of the limit coefficient $\displaystyle A_\infty=\lim_{t \rightarrow\infty} A(t)$ coveys an important piece of information but does not fully account for the asymptotic behavior. The part contributed by an order of $O(\tfrac{1}{\tau})$ in the coefficient accumulates effect in the asymptotic behavior, which can be enough to alter the nature of stability.
% 
% This, in turn, calls for a simultaneous diagonalization of the whole one-parameter family of matrices $A(t)$ to sort out all the contribution. It is well-known that the simultaneous diagonalization in general fails. One sufficient condition for this to be done is to have $A_\infty$ have eigenvalues all distinct. This is the case of the classical Coddington-Levsinson Theorem \cite{CL55} that states the precise stability by inspecting the time dependent spectrum.

%%%%
%


\section{Linearized stability for the inviscid model ($n=0$)} \label{sec:Hadamard}

\tcr{To-Do  Split the section in 
\begin{itemize}
\item Case $\alpha > m$ eventually decreasing uniform shear solution
\item Case $\alpha < m$ increasing uniform shear.
In this case $Re \lambda_i (\tau ) < 0 \; \forall i$ and thus solution is stable by Thm in Appendix
\end{itemize}
}




\subsection{Case that thermal softening outweighs strain hardening: $\alpha > m$}


\subsubsection{Spectrum of $A_k(\tau)$}

We are to study a non-autonomous system with a coefficient
\begin{equation} \label{eq:n0A}
  A_k(\tau)=\begin{pmatrix}
   0 & 0 & -k^2\\
   0 & 0 & 0\\
   0 & 0 & 0
  \end{pmatrix}
  + \frac{1}{\tau}   \begin{pmatrix}
   0 & 0 & 0\\
   \frac{1+\alpha}{1+m} & -\frac{1+\alpha}{1+m} & 0\\
   \frac{-\alpha+m}{1+m} & m\frac{-\alpha+m}{1+m}& -(1+\alpha)
  \end{pmatrix} = A_{k,\infty} + \frac{1}{\tau}A_{k,1}.
\end{equation}
The characteristic polynomial $\det\big(\lambda \textrm{I} - A_k(\tau)\big)=0$ reads
$$ \Big(\lambda +\frac{1}{\tau}(1+\alpha)\Big)\Big( \lambda^2 + \frac{1}{\tau}\lambda \frac{1+\alpha}{1+m} + \frac{1}{\tau} k^2 \frac{-\alpha+m}{1+m}\Big)=0.$$
For convenience of computations later, we define $\beta(\tau) = \frac{1+\alpha}{2\tau(1+m)}>0$, and $\gamma(\tau)= k^2\frac{-\alpha+m}{\tau(1+m)}<0$ and denote the three eigenvalues by
\begin{equation} \label{eq:inviscid_roots}
\begin{aligned}
 &\lambda_1(\tau) = -\beta + \sqrt{\beta^2-\gamma}, \quad \lambda_2(\tau) = -\beta - \sqrt{\beta^2-\gamma}, \quad \lambda_3(\tau) = -\frac{1+\alpha}{\tau}.
\end{aligned}
\end{equation}
Note that all the eigenvalues vanish in the limit. In particular, 
$$ \beta(\tau)=\mathcal{O}\Big(\frac{1}{\tau}\Big), \quad \gamma(\tau)=\mathcal{O}\Big(\frac{1}{\tau}\Big), \quad \lambda_1(\tau)=\mathcal{O}\Big(\frac{1}{\sqrt{\tau}}\Big), \quad \lambda_2(\tau)=\mathcal{O}\Big(\frac{1}{\sqrt{\tau}}\Big), \quad\lambda_3(\tau)=\mathcal{O}\Big(\frac{1}{\tau}\Big)$$
as $\tau \rightarrow \infty$. Since $-\alpha+m<0$, all roots are real and 
$$\lambda_1(\tau) > 0 > \lambda_3(\tau) \gg \lambda_2(\tau) \quad \text{for $\tau$ large enough.}$$

% 
% $\beta(\tau)$ and $\gamma(\tau)$ are both $\mathcal{O}\big(\frac{1}{\tau}\big)$ as $\tau \rightarrow \infty$ and thus $\lambda_1(\tau)$ and $\lambda_2(\tau)$  are $\mathcal{O}\big(\frac{1}{\sqrt{\tau}}\big)$ while $\lambda_3(\tau)=\mathcal{O}(\frac{1}{\tau})$. For $\tau$ large (since $-\alpha+m<0$), $(i)$ all roots are real; $(ii)$ $\lambda_1(\tau) > 0 > \lambda_3(\tau) \gg \lambda_2(\tau)$. %  Hence, roots are complex or pure real according to the sign of $-\alpha+m$. The square root in the above formulas are understood as taking the branch of nonnegative imaginary part.

\subsubsection{Smooth in time factorization of $A_k(\tau)$}
The three eigenvalues all merge into the single value $0$ in the limit $\tau \rightarrow \infty$ and it is readily checked that this eigenvalue $0$ of $A_{k,\infty}$ has a geometric multiplicity $2$. Therefore, despite all $\lambda_i(\tau)$, $i=1,2,3$ at finite $\tau$ are generically distinct, the diagonalization will never be continuous in the limit.
The precise asymptotic behaviors of the eigenspaces are further available. %As the order $\frac{1}{\tau}$ perturbation of $A(\tau)$ is explicitly given,
If we define
$$ E(\tau) = \frac{1}{\sqrt{\tau}} \begin{pmatrix} \tau\beta \\ \frac{1+\alpha}{1+m} \\ \frac{-\alpha+m}{1+m} \end{pmatrix}, \quad  F(\tau)= \begin{pmatrix}  \sqrt{\tau}\sqrt{\beta(\tau)^2-\gamma(\tau)}\\0\\0 \end{pmatrix},$$
the eigenvectors $s_i(\tau)$, $i=1,2,3$ are by straightfoward calculations
\begin{align*}
 s_1(\tau)&=E(\tau) + F(\tau), \quad s_2(\tau)=E(\tau)-F(\tau), \quad s_3(\tau) = \begin{pmatrix} -m\\1\\ -\frac{m(1+\alpha)}{\tau\l^2} \end{pmatrix}.
\end{align*}
Since $E(\tau) \rightarrow \mathbf{0}$ as $\tau \rightarrow \infty$, and $s_3$ is linearly independent from $F(\tau)$, it is clear that the null spaces $\mathbf{N}\Big(A(\tau) - \lambda_1(\tau)\Big)$ and $\mathbf{N}\Big(A(\tau) - \lambda_2(\tau)\Big)$ merge up in the limit.

% Recalling that both of $\beta(\tau)$ and $\gamma(\tau)$ are of $\mathcal{O}(\frac{1}{\tau})$, $E(\tau)=\mathcal{O}(\frac{1}{\sqrt{\tau}})$ and $F(\tau)=\mathcal{O}(1)$. Thus in the limit $s_1(\tau)$ and $s_2(\tau)$ become parallel, or 

Having above done, although at finite $\tau$ the matrix $A_k(\tau)$ is diagonalizable, we rather consider the following smooth triangularization motivated from the Jordan form factorization. Speaking of the smooth factorization of one-parameter family of matrices, neither the diagonalization nor the Jordan form factorization in general is continuous for given smooth family. Thus, we are making use of the special feature of $A_k(\tau)$ that it admits such a factorization.


We pick $s_1$, renormalized by a factor of $\Big(\sqrt{\tau}(-\beta+\sqrt{\beta^2-\gamma})\Big)$ at our convenience. Then a pre-image $\tilde s_2$ is found so that
$$ (A-\lambda_2) \tilde s_2 = \Big(\sqrt{\tau}(-\beta+\sqrt{\beta^2-\gamma})\Big)s_1. \quad \text{We can pick $\tilde s_2 = \begin{pmatrix}  0 \\ \tfrac{1+\alpha}{1+m} \\ \tfrac{-\alpha+m}{1+m} \end{pmatrix}$}.$$
With this new basis $\left\{\Big(\sqrt{\tau}(-\beta+\sqrt{\beta^2-\gamma})\Big)s_1,\tilde s_2, s_3 \right\}$, we factorize \\$A_k(\tau) = P_k(\tau)B_k(\tau)P_k^{-1}(\tau)$ and define the linear system:
\begin{equation}\label{eq:factor}
\begin{aligned}
 y &= P_k(\tau)^{-1} x, \quad  y' = \underbrace{\begin{pmatrix} \lambda_1(\tau) & 1 & 0\\0 & \lambda_2(\tau) & 0\\0 & 0 & \lambda_3(\tau)\end{pmatrix}}_{\triangleq B(\tau)} y - P_k(\tau)^{-1}P_k'(\tau) y,\\
 P_k(\tau) &= \begin{pmatrix} -\tau\gamma& 0 & -m\\ \Big(-\beta+\sqrt{\beta^2-\gamma}\Big)\frac{1+\alpha}{1+m} & \frac{1+\alpha}{1+m} & 1\\ \Big(-\beta+\sqrt{\beta^2-\gamma}\Big)\frac{-\alpha+m}{1+m} & \frac{-\alpha+m}{1+m} & -\frac{m(1+\alpha)}{\tau\l^2}\end{pmatrix}.
\end{aligned}
\end{equation}

\subsubsection{Proof of Hadamard Instability}

We view \eqref{eq:factor} as a system perturbed from
\begin{equation} \label{eq:trisystem}
  \hat{y}' = B(\tau)\hat{y} \quad \text{with $B(\tau)$ upper triangular}
\end{equation}
by the error of $o(\frac{1}{\tau})$ (more precisely $O(\frac{1}{\tau^{3/2}})$) that is $- P(\tau)^{-1}P'(\tau)$. The latter size estimate is in particular a result of the Proposition \ref{prop:factor}. We merely inspect the asymptotic behavior of \eqref{eq:trisystem} and the Theorem \ref{prop:stab} in the Appendix furnishes the proof by showing that the maximum growths of \eqref{eq:trisystem} and \eqref{eq:factor} are same (by a factor of a constant). %sThe maximum growth of a system with an upper triangular coefficient is computable in an abstract setting, but we rather use an explicit formula for our specific case.
negligible

\begin{proposition} \label{prop:factor}$\exists \tau_0$ such that for $\tau\ge\tau_0$
 \begin{enumerate}
  \item  $\big|P_{ij}(\tau)\big|$ and $\big|P^{-1}_{ij}(\tau)\big|$ are uniformly bounded for all $i,j=1,2,3$.
  \item $\displaystyle \int_{\tau_0}^\infty \big|P_{ij}'(s)\big|\; ds < \infty$ for all $i,j=1,2,3$.
 \end{enumerate}
\end{proposition}
\begin{proof}
$$ \lim_{\tau \rightarrow \infty} P(\tau) = \begin{pmatrix} -\l^2\frac{-\alpha+m}{1+m}& 0 & -m\\ 0 & \frac{1+\alpha}{1+m} & 1\\ 0 & \frac{-\alpha+m}{1+m} & 0\end{pmatrix}$$
whose determinant is $\l^2 \left(\frac{-\alpha+m}{1+m}\right)^2$. Since $P(\tau)$ is a smooth function of $\tau$, $\exists \tau_0$ such that $\big|P_{ij}(\tau)\big|$ for $\tau\ge\tau_0$ is uniformly bounded, and also that $P(\tau)$, $\tau\ge\tau_0$ is invertible with the determinant uniformly bounded from above, and below away from $0$ since the determinant is a continuous function of entries. With the same reason, any minor of $P(\tau)$ is also uniformly bounded. Hence $\big|P^{-1}_{ij}(\tau)\big|$ is also uniformly bounded for $\tau\ge \tau_0$. The second assertion follows from straightforward calculation of $P_{ij}'(\tau)$ whose leading order is $\mathcal{O}\big(\frac{1}{\tau^{3/2}}\big)$.
\end{proof}




\begin{theorem} \label{thm:Hadamard} Suppose $n=0$ among material parameters $(\alpha,m,n)$ in ranges \eqref{eq:paramrange}. Then
  \begin{enumerate}
    \item There is a growing mode $x_k^+(\tau)$ for each $k\ge1$, thus there are infinitely many growing modes.
    \item $x_k^+(\tau)$ grows exponentially as $\tau \rightarrow \infty$, more precisely, there are constants $\chi>0$, $\tau_0>0$ and a constant vector $x_{k,\infty}^+$ such that
    \begin{equation}
      \lim_{\tau \rightarrow \infty} e^{-\chi \big(\sqrt{\tau}-\sqrt{\tau_0}\big)} x_k^+(\tau) = x_{k,\infty}^+.
    \end{equation}
    \item The exponent $\chi=\chi(k,\alpha,m)=2\l\sqrt{\frac{\alpha-m}{(1+m)}}$ and is unbounded as $k \rightarrow \infty$.
  \end{enumerate}
\end{theorem}

\begin{remark} By Theorem \ref{thm:Hadamard} we come to a conclusion that the nature of instability exhibited in our occasion is qualitatively not a different kind, compared to the Hadamard Instability exhibited around the constant state: $(i)$ the growth is exponential; $(ii)$ the growth rate is unbounded as $k \rightarrow \infty$.
\end{remark}

\subsubsection{Stability of a non-autonomous linear system}
To quickly summarize our perspective to come to the conclusion, a few points as to the stability problem of a non-autonomous system needs to be illustrated. First, we specialize the fact that the coefficient in \eqref{eq:l-system2} is {\it frozen} in the limit $\tau \rightarrow \infty$. An example of instability in the complementary case where
$$ 
A(t) = \left(\begin{array}{cc} -1 + \frac{3}{2}\cos^2 t & 1-\frac{3}{2}\cos t \sin t \\ -1 -\frac{3}{2}\cos t \sin t & -1 + \frac{3}{2}\sin^2 t \end{array}\right), \;\; \lambda_\pm = \frac{-1 + i\sqrt{7}}{4}, 
\; \; x(t) = \left(\begin{array}{c} -\cos t\\ \sin t \end{array}\right) e^{\tfrac{t}{2}}$$
shows that the spectral information  $\textrm{Re}\, \lambda_\pm(t)<0$ does not help much in determining the asymptotic stability.

In such a case where the coefficient is frozen in the limit however, the classical analysis, for example as in \cite{CL55}, shows that the spectral information of the limit coefficient $\displaystyle A_\infty=\lim_{t \rightarrow\infty} A(t)$ coveys an important piece of information but does not fully account for the asymptotic behavior. The part contributed by an order of $O(\tfrac{1}{\tau})$ in the coefficient accumulates effect in the asymptotic behavior, which can be enough to alter the nature of stability.

This, in turn, calls for a simultaneous diagonalization of the whole one-parameter family of matrices $A(t)$ to sort out all the contribution. It is well-known that the simultaneous diagonalization in general fails. One sufficient condition for this to be done is to have $A_\infty$ have eigenvalues all distinct. This is the case of the classical Coddington-Levsinson Theorem \cite{CL55} that states the precise stability by inspecting the time dependent spectrum.

It is this point to which our linear system is contrasted; the limit coefficient $A_{k,\infty}$ in \eqref{eq:l-system2} has the repeated eigenvalue $0$. Its consequences are varied in the inviscid ($n=0$) and the viscous ($n>0$) case. For the inviscid case, the simultaneous diagonalization is unachievable and we will have to rely on mere the upper triangularization of the one-parameter family of matrices, whereas for the viscous case the simultaneous diagonalization is achievable despite of the repeated eigenvalues. In the former case, the third effect of resonance also has to be taken account in determining the asymptotic stability. To summarize, our study boils down to studying the stability of a system where the coefficient is upper triangular or diagonal for all $t\ge0$.




\subsubsection{Proof of Hadamard Instability}


\begin{proof}[proof of Theorem \ref{thm:Hadamard}]
Let $\tau_0$ be large enough so that $\lambda_2(\tau) < \lambda_3(\tau) < 0 < \lambda_1(\tau)$ and all eigenvalues are real for $\tau\ge \tau_0$. Explicit formulas for the general solutions of \eqref{eq:trisystem} are available. The maximum growth  takes place when
$$
\begin{aligned}
 \hat{y}(\tau) &=\left(\exp{\int_{\tau_0}^\tau \lambda_1(s)\, ds} \right)\hat y_1(\tau_0) 
 \\
 &\quad +  \left(\exp{\int_{\tau_0}^\tau \lambda_1(s)\, ds}\right) \underbrace{\int_{\tau_0}^\tau \exp{\left(-\int_{\tau_0}^z \lambda_1(s)-\lambda_2(s)\,ds\right)}\hat y_2(\tau_0) \, dz}_{\triangleq \varphi(\tau)}
\end{aligned}
$$
for some constants $\hat{y}_1(\tau_0)$ and $\hat{y}_2(\tau_0)$. Inspired from the proof of the Coddington-Levinson Theorem \cite{CL55}, we examine the accumulated spectral gap $\int_{\tau_0}^\tau \lambda_1(s)-\lambda_2(s)\,ds$. In particular, we use the special fact that the accumulated spectral gap is ${O}(\sqrt{\tau})$ as $\tau \rightarrow 0$. This implies that the integrand in $\varphi(\tau)$ is integrable and that $\varphi(\tau)$ is bounded. Hence
$$|\hat{y}(\tau)|\le C \left| \exp{\int_{\tau_0}^\tau \lambda_1(s)\, ds}\right| \quad \text{for }\tau\ge\tau_0$$
and there is a $y^+(\tau)$ that attains the maximum growth. By Proposition \ref{prop:stab}, there is a $x^+(\tau)$ and a constant vector $x^+_\infty$ such that
\begin{equation*}
  \lim_{\tau \rightarrow \infty}\exp{\left(-\int_{\tau_0}^\tau \lambda_1(s)\, ds\right)} x^+(\tau) = x^+_\infty.
\end{equation*}
Recalling that $\beta(\tau) = \frac{1+\alpha}{2\tau(1+m)}$, $\gamma(\tau)= \l^2\frac{-\alpha+m}{\tau(1+m)}$ and $\lambda_1(\tau) = -\beta + \sqrt{\beta^2-\gamma}$, the above gives the estimate
% \begin{equation}
%   \lim_{\tau \rightarrow \infty}\exp{\left(-\int_{\tau_0}^\tau \l\sqrt{\frac{-\alpha+m}{s(1+m)}}\, ds\right)} x^+(\tau) = x^+_\infty.
% \end{equation}
\begin{equation}
  \lim_{\tau \rightarrow \infty}\exp{\left(-2\l\sqrt{\frac{\alpha-m}{(1+m)}}\big(\sqrt{\tau}-\sqrt{\tau_0}\big)\right)} x^+(\tau) = x^+_\infty.
\end{equation}
\end{proof}

\subsection{Case that  strain hardening outweighs   thermal softening : $\alpha < m$}



\vfil\eject


\section{Linearized stability for the thermoviscoplastic model, effect of strain-rate hardening}

\tcr{ To-Do  Develop this section separately. Develop both cases
\begin{itemize}
\item Cases $-\alpha + m + n  > 0$ linearized stability
\item Case $-\alpha + m + n  <  0$ linearized instability
\end{itemize}
}

\subsection{Turing Instability of viscous model}
While the purpose of the previous section was to clarify a type of Hadamard Instability for the inviscid case ($n=0$), the purpose of this section is to reveal a type of Turing Instability for the viscous case ($n>0$). Again, the term Turing Instability is to be understood properly. To brief, the Turing instability in the {morphogenesis} and in many other interesting pattern forming phenomena, it refers to an instability due to the interaction of two or a few different  mechanisms each of that would stabilize by acting itself alone. In the pattern formation, it is important to have a few but finite unstable modes that give rise to interesting patterns and these patterns are quite often steady states.

By contrast, for the localizing instability exhibited in the shear band problem, we view it as a consequence attributed to a different kind of mechanism. The results in Theorem \ref{thm:Turing} in this section can be summarized by the appearance of infinitely many unstable modes for each frequency and the saturation of the growth rates, meaning that the growths of modes are all comparable. These two phenomena was observed in the simpler model that corresponds to $\alpha=0$ and $m=-1$ in \cite{KLT16}, where the linearized system happened to be an autonomous one.

In our point of view, the appearance of infinitely many unstable modes for each frequency and their growth rates being saturated are important for the formation of initial peak-like coherent profile. In the later dynamics, the nonlinear effect takes over and this is a subject of our second paper of the series.

\begin{theorem} \label{thm:Turing} Suppose $n>0$ among material parameters $(\alpha,m,n)$ in ranges \eqref{eq:paramrange}. Then
  \begin{enumerate}
    \item There is a growing mode $x_k^+(\tau)$ for each $k\ge1$, thus there are  infinitely many growing modes.
    \item $x_k^+(\tau)$ grows in a polynomial order as $\tau \rightarrow \infty$, more specifically, there are constants $\chi>0$ and $\tau_0$ and a constant vector $x_{k,\infty}^+$ such that
    \begin{equation}
      \lim_{\tau \rightarrow \infty} \left(\frac{\tau}{\tau_0}\right)^{\chi} x_k^+(\tau) = x_{k,\infty}^+.
    \end{equation}
    \item The exponent $\chi=\chi(\alpha,m,n)$, is bounded and independent of $k$.
  \end{enumerate}
\end{theorem}
\begin{remark}
 The asymptotic expansion of $\chi$ as $n \rightarrow 0$ is given by
\begin{align*}
 \chi = -\frac{-\alpha+m+n}{n(1+m)} + \frac{\alpha m}{1+m} + \mathcal{O}(n) \quad \text{as } n \rightarrow 0. \quad \text{(See proof of Theorem \ref{thm:Turing} and \eqref{eq:rootformulas})}.%, \quad
 % z_2(0) = -(1+m)  + \mathcal{O}(n).
\end{align*}
\end{remark}
\subsubsection{Spectrum of $A_k(\tau)$}

Let $\lambda_i(\tau)$, $i=1,2,3$ be the three eigenvalues of $A_\ell(\tau)$. %We have
The characteristic polynomial $\det\big(\lambda \textrm{I} - A(\tau)\big)$ reads
\begin{equation}
\begin{aligned}
 \lambda^3 &+ \lambda^2\left( \tfrac{1+\alpha}{\tau} + \tfrac{1+\alpha}{\tau(1+m)} + n\l^2\right)
 + \lambda\left( \tfrac{(1+\alpha)^2}{\tau^2(1+m)}
 + \l^2\left( n\tfrac{1+m+\alpha}{\tau(1+m)} + \tfrac{-\alpha+m+n}{\tau(1+m)}\right)\right) \\
 &+ \l^2\tfrac{(-\alpha+m+n)(1+\alpha)}{\tau^2(1+m)} \: \triangleq \: f(\lambda,\tau,\alpha,m,n). \label{eq:poly}
\end{aligned}
\end{equation}
In the below, we let $\epsilon\triangleq \frac{1}{\tau}$ and write $f = f^{\alpha,m,n}(\lambda,\epsilon)$. If context is clear, this abuse of notation keeping same symbol for functions with the different aruments $\epsilon$ and $\tau$ will be applied in other occasions as well.

While explicit formulas for eigenvalues are not available anymore, we retrieve enough information from the asymptotic expansion of $f(\lambda,\epsilon)$ with respect to $\epsilon$. The three roots will be denoted as functions of $\epsilon$, i.e., $\lambda_i(\epsilon)$, $i=1,2,3$. The limiting problem $f^{\alpha,m,n}(\lambda,0)=0$ gives
$$ \lambda_1(0)=\lambda_2(0)=0, \quad \lambda_3(0) = -n\l^2.$$
Hence if $n>0$ $\lambda_3(0)$ is a simple root and we see the bifurcation by $n=0$ for that.

To discern the behaviors of $\lambda_1(\epsilon)$ and $\lambda_2(\epsilon)$, we consider another limiting problem by following procedure.
We let $z = \frac{\lambda}{\epsilon}$ and $\epsilon^2 g(z,\epsilon) =  f(\lambda, \epsilon)$. For $\epsilon\ne0$, $g(z,\epsilon)=0$ if and only if $f(\lambda,\epsilon)=0$. Now,
\begin{align}
 g(z,\epsilon) &= \epsilon\left( z^3 + z^2\left(1+\alpha + \tfrac{1+\alpha}{1+m}\right) + z\left(\tfrac{(1+\alpha)^2}{1+m}\right)\right) \nonumber\\
 +& \l^2\left(nz^2 + z\Big( \tfrac{n(1+m+\alpha) + (-\alpha+m+n)}{1+m}\Big) + \tfrac{(1+\alpha)(-\alpha+m+n)}{1+m}  \right). \label{eq:reduced_poly}
\end{align}
Provided $n>0$, the limiting problem $g(z,0)=0$ is a quadratic polynomial
\begin{equation}
nz^2 + z\Big( \frac{n(1+m+\alpha) + (-\alpha+m+n)}{1+m}\Big) + \frac{(1+\alpha)(-\alpha+m+n)}{1+m} =0. \label{eq:auxquad}
\end{equation}
Two roots are  denote by $z_1(0)$ and $z_2(0)$ respectively.  The explicit formulas for $z_1(0)$ and $z_2(0)$ are available,
\begin{equation}\label{eq:rootformulas}
\begin{aligned}
 z_1(0) &= -\tfrac{n(1+m+\alpha) + (-\alpha+m+n)}{2n(1+m)} + \sqrt{\left(\tfrac{n(1+m+\alpha) + (-\alpha+m+n)}{1+m}\right)^2 -4 \tfrac{n(1+\alpha)(-\alpha+m+n)}{1+m}},\\
z_2(0) &= -\tfrac{n(1+m+\alpha) + (-\alpha+m+n)}{2n(1+m)}  - \sqrt{\left(\tfrac{n(1+m+\alpha) + (-\alpha+m+n)}{1+m}\right)^2 -4 \tfrac{n(1+\alpha)(-\alpha+m+n)}{1+m}}.
\end{aligned}
\end{equation}
Note that $z_1(0)$ and $z_2(0)$ are independent of $\ell$. Provided the two roots  \eqref{eq:rootformulas} are not repeated, we apply the well-known theorem of analyticity of a simple root to conclude the two roots $z_1(\epsilon)$ and $z_2(\epsilon)$ are near to $z_1(0)$ and $z_2(0)$ respectively. Precise statements are provided in the following section.

These two roots obtained in the case $n>0$ show the bifurcation by $q=-\alpha+m+n$. If $q=-\alpha+m+n<0$, there must be one positive and one negative real roots whereas if $q=-\alpha+m+n>0$ then $z_1(0)+z_2(0) = -\frac{n(1+m+\alpha) + (-\alpha+m+n)}{n(1+m)} < 0$ and $z_1(0)z_2(0)=\frac{(1+\alpha)(-\alpha+m+n)}{n(1+m)}>0$ and thus the real parts of two roots must be negative.




\subsubsection{Diagonalization of $A(\tau)$}
Since $\lambda_1(\tau) \rightarrow 0$ and $\lambda_2(\tau) \rightarrow 0$ as $\tau \rightarrow \infty$, the limit coefficient has a reapeted eigenvalues $0$. The classical Coddington-Levinson Thoerem assumes a limit coefficient who has distinct eigenvalues, and it cannot be applied to our system. We take the first ingredient of proof of the Coddington-Levinson Theorem, namely the stability of a system with diagonal coefficient matrices. Without using the assumption of distinct eigenvalues, we directly provide the simultaneous diagonalization of $A_\ell(\tau)$ from the fact that the repeated eigenvalue $0$ has a full geometric multiplicity, or in other words that the subspaces $\mathbf{N}\Big(A(\tau)-\lambda_1(\tau)\Big)$ and  $\mathbf{N}\Big(A(\tau)-\lambda_2(\tau)\Big)$ survive to be linearly independent up to infinity. The proof of the first ingredient, for completeness, is provided in the Appendix.

\begin{proposition}\label{prop:ctsdiag} $\exists \tau_0$ such that for $\tau\ge \tau_0$% $\lambda_1(\tau)$, $\lambda_2(\tau)$, $\lambda_3(\tau)$, and $S(\tau)$ the matrix of right eigenvectors that is invertible exist with the following properties.
 \begin{enumerate}
  %\item $A(\tau) = S(\tau) \begin{pmatrix} \lambda_1(\tau) & 0 & 0\\ 0 & \lambda_2(\tau) & 0 \\ 0 & 0 & \lambda_3(\tau) \end{pmatrix} S^{-1}(\tau)$.%= S(\tau) \Lambda(\tau) S^{-1}(\tau)$.
  \item $|S_{ij}(\tau)|, |S^{-1}_{ij}(\tau)|$ are uniformly bounded and % $S(\tau)$ is absolutely continuous, i.e.,
  $\int_{\tau_0}^\infty |{S}'_{ij}(s)|\; ds <\infty$, $\forall i,j=1,2,3$.
  \item If $\epsilon\triangleq \frac{1}{\tau}$ then $\lambda_1(\epsilon)$, $\lambda_2(\epsilon)$, $\lambda_3(\epsilon)$, and $S(\epsilon)$ are analytic in the radius $\frac{1}{\tau_0}$ ball of $0$ and their leading order expansions are
  \begin{align}
   \lambda_1(\epsilon) & = \epsilon z_1(0) + \mathcal{O}(\epsilon^2),\quad %\epsilon^2 \dot z_1(0) + \cdots,\nonumber\\
   \lambda_2(\epsilon) = \epsilon z_2(0) + \mathcal{O}(\epsilon^2),\quad %\epsilon^2 \dot z_1(0) + \cdots,\nonumber\\ &\text{where $z_1(\epsilon)$ and $z_2(\epsilon)$ are from the two roots of \eqref{eq:reduced_poly},}\nonumber \\
   \lambda_3(\epsilon) = -n\l^2 + \mathcal{O}(\epsilon),\\%\epsilon\dot\lambda_3(0) + \cdots, \nonumber\\
   S(\epsilon)
   &=\begin{pmatrix}
    1 + z_1(0)\frac{1+m}{1+\alpha} & 1 + z_2(0)\frac{1+m}{1+\alpha} & 1 \label{eq:S(0)}\\
    1 & 1 & 0\\
    0 & 0 & n
   \end{pmatrix} + \mathcal{O}(\epsilon).%\\&+ \epsilon
  \end{align}
 \end{enumerate}
\end{proposition}
\begin{proof}
 First we prove the assertion 2. That the eigenvalues are analytic in $\epsilon$ in some neighborhood of $0$ follows from that $-n\l^2$ is a simple root of \eqref{eq:poly} and $z_1(0)$ and $z_2(0)$ are the simple roots of \eqref{eq:auxquad}: For a fixed $n>0$, $\lambda_3(0)= -n\l^2$ is a simple root of \eqref{eq:poly} which leads to that $\frac{\partial f}{\partial \lambda}(-n\l^2,0)\ne 0$ and the existence of an analytic $\lambda_3(\epsilon)$ follows from the Theorem \ref{thm:anal}. By the same reasoning there are analytic functions $z_1(\epsilon)$ and $z_2(\epsilon)$ in some neighborhood of $0$. Note that $f(\epsilon z(\epsilon),\epsilon)=0$ if and only if $g(z(\epsilon),\epsilon)=0$ for $\epsilon\ne0$. For a fixed $n>0$, we take $\epsilon_0$ sufficiently small so that $\epsilon z_i(\epsilon)\ne \lambda_3(\epsilon)$, $i=1,2$ are assured in $\epsilon_0$-ball. We set $\epsilon_0$ be this radius of analyticity and let $\tau_0 = \frac{1}{\epsilon_0}$.

 The right eigenvector matrix is computable, in terms of $\lambda_i(\epsilon)$ and entries $A_{ij}(\epsilon)$.
 \begin{equation} \label{eq:S}
  \begin{aligned}
 S(\epsilon) &=   \begin{pmatrix}
     1 + z_1(\epsilon)\frac{1+m}{1+\alpha} & 1 + z_2(\epsilon)\frac{1+m}{1+\alpha} & 1\\
     1 & 1 & \frac{\epsilon \tfrac{1+\alpha}{1+m}}{\lambda_3(\epsilon) + \epsilon\tfrac{1+\alpha}{1+m}}\\
     -\frac{\epsilon z_1(\epsilon)}{\l^2}\Big(1 + z_1(\epsilon)\frac{1+m}{1+\alpha}\Big) & -\frac{\epsilon z_2(\epsilon)}{\l^2}\Big(1 + z_2(\epsilon)\frac{1+m}{1+\alpha}\Big) & -\frac{\lambda_3(\epsilon)}{\l^2}
    \end{pmatrix}\\
    &=\begin{pmatrix}
     1 + z_1(0)\frac{1+m}{1+\alpha} & 1 + z_2(0)\frac{1+m}{1+\alpha} & 1 \\
     1 & 1 & 0\\
     0 & 0 & n
    \end{pmatrix}\\&+ \epsilon
    \begin{pmatrix}
     \dot z_1(0)\frac{1+m}{1+\alpha} & \dot z_2(0)\frac{1+m}{1+\alpha} & 0 \nonumber\\
     0 & 0 & -\frac{\frac{1+\alpha}{1+m}}{n\l^2}\\
     -\frac{z_1(0)\big(1+z_1(0)\frac{1+m}{1+\alpha}\big)}{\l^2} & -\frac{z_2(0)\big(1+z_2(0)\frac{1+m}{1+\alpha}\big)}{\l^2} & -\frac{\dot\lambda_3(0)}{n\l^2} \nonumber\\
    \end{pmatrix} + \cdots,
  \end{aligned}
 \end{equation}
 where $\dot{(\cdot)}=\frac{d}{d\epsilon}(\cdot)$.

Now we prove the first assertion. Since $z_1(0)\ne z_2(0)$, the constant part $S(0)$ is invertible. Since $S(\epsilon)$ is analytic, by taking $\epsilon_0$ smaller if necessary, $S(\epsilon)$ is invertible $\epsilon\in[0,\epsilon_0]$. With the similar arguments in the proof of Theorem \ref{thm:Hadamard}, $\big|S_{ij}(\epsilon)\big|$ and $\big|S^{-1}_{ij}(\epsilon)\big|$ are uniformly bounded.
Now, because $S_{ij}(\epsilon)$ is analytic, $\frac{d}{d\epsilon}S_{ij}(\epsilon)$ is bounded in the $\epsilon_0$-ball and $\frac{d}{d\tau} S_{ij}(\tau) = -\frac{1}{\tau^2}\frac{d}{d\epsilon} S_{ij}(\epsilon)$, which is integrable from $\tau_0$ to infinity. %Thus the assertion 1 follows.
\end{proof}

Now, we similarly as in Section \ref{sec:Hadamard} consider the diagonal system:
\begin{equation*}
\begin{aligned}
 y &= S(\tau)^{-1} x, \quad  y' = \underbrace{\begin{pmatrix} \lambda_1(\tau) & 0 & 0\\0 & \lambda_2(\tau) & 0\\0 & 0 & \lambda_3(\tau)\end{pmatrix}}_{\triangleq \Lambda(\tau)} y - S(\tau)^{-1}S'(\tau) y.
\end{aligned}
\end{equation*}
Since $\Lambda(\tau)$ is analytic in $\frac{1}{\tau}$, we can further take out the high order contribution in $\frac{1}{\tau}$ from $\Lambda(\tau)$. Let $\Lambda(\tau) = \Lambda_0 + \frac{1}{\tau}\Lambda_1 + \frac{1}{\tau^2}\tilde{\Lambda}(\tau)$.

\begin{equation}\label{eq:diagonalsystem}
\begin{aligned}
y'(\tau) &= \Big(\Lambda_0 + \frac{1}{\tau}\Lambda_1\Big) y + R(\tau) y, \quad R(\tau) = \frac{1}{\tau^2}\Big(\tilde{\Lambda}(\tau)+ S^{-1}\frac{d}{d\epsilon}{S}\Big), \\
 \Lambda_0 + \frac{1}{\tau}\Lambda_1 &=
 \begin{pmatrix}
  \frac{z_1(0)}{\tau} & 0 & 0\\
  0 & \frac{z_2(0)}{\tau} & 0\\
  0 & 0 & -n\l^2 + \frac{\lambda_3'(0)}{\tau}
 \end{pmatrix}.
\end{aligned}
\end{equation}
Note the error $R(\tau)$ is $o(\frac{1}{\tau})$. %This size estimate is in particular a result of the Proposition \ref{prop:ctsdiag}.



\subsubsection{Proof of Turing Instability}
Now we are ready to prove the Theorem \ref{thm:Turing}, applying Theorem \ref{thm:CL}.
\begin{proof}[proof of Theorem \ref{thm:Turing}]
We fix parameter $(\alpha,m,n)$ and apply the Theorem \ref{thm:CL} on the system \eqref{eq:diagonalsystem}. That for every parameter $(\alpha,m,n)$ and for every row index $k$, the spectral gap condition in Theorem \ref{thm:CL} is fulfilled is to be checked:
For large $\tau<\infty$,
\begin{align*}
  \text{Fix $k=1$:}&& 2,3&\in I_1, & 1&\in I_2\\
  \text{Fix $k=2$:}&& 3&\in I_1, & 1,2&\in I_2\\
  \text{Fix $k=3$:}&& & & 1,2,3&\in I_2.
\end{align*}

Having that checked, by Theorem \ref{thm:CL}, we find $y_i(\tau;\ell)$, $i=1,2,3$ the solutions of \eqref{eq:diagonalsystem} such that
 \begin{align*}
 &\lim_{\tau \rightarrow \infty} y_1(\tau) \exp\left(-\int_{\tau_0}^\tau z_1(0)\frac{1}{s}\; ds\right) = \lim_{\tau \rightarrow \infty} y_1(\tau)\left(\frac{\tau}{\tau_0}\right)^{-z_1(0)} = \hat{1}, \\
 &\lim_{\tau \rightarrow \infty} y_2(\tau)\left(\frac{\tau}{\tau_0}\right)^{-z_2(0)} = \hat{2}, \\
 &\lim_{\tau \rightarrow \infty} y_2(\tau)\left(\frac{\tau}{\tau_0}\right)^{-\lambda_3'(0)}\exp\Big(n\l^2(\tau-\tau_0)\Big) = \hat{3},
 %\begin{pmatrix} 1+z_1(0)\frac{1+m}{1+\alpha}\\1\\0\end{pmatrix}.
 \end{align*}
 where $\hat{i}$, $i=1,2,3$ denotes the standard coordinate basis vectors. We define $x_i(\tau)\triangleq S(\tau)y_i(\tau)$. We check that the eigenvector matrix $S(\tau)$ is continuous and $S(\tau) \rightarrow S_\infty$ as $\tau \rightarrow \infty$ and that multiplication of $S(\tau)$ in the limits on the left hand side converge to $S_\infty \hat{i}$, $i=1,2,3$ the eigenvector of $A_{\ell,\infty}$.

 Recalling that $z_1(0)>0$ and $z_2(0)<0$, $x_1(\tau)$ is the only growing mode. The exponent $\chi=z_1(0)$ is bounded and determined solely by $(\alpha,m,n)$.
%
%  The exponents $z_1(0)$ and $z_2(0)$ are two roots of \eqref{eq:auxquad} and are independent of $\ell$. As discussed earlier,
%  \begin{enumerate}
%   \item If $-\alpha+m+n=0$, then $z_1(0)=0$ and $z_2(0)= -\frac{1+m+\alpha}{1+m}$.
%   \item If $-\alpha+m+n>0$, then $Re ~z_1(0)<0$ and $Re ~z_2(0)<0$.
%   \item If $-\alpha+m+n<0$, then $z_1(0)>0$ and $z_2(0)<0$.
%  \end{enumerate}
% To compute the $\chi_1$ and $\chi_2$, we go back to the independent variable $\gamma_s$; we recall $\tau(\gamma_s)= \theta_s^*=\left( \frac{1+\alpha}{1+m}\right)^{\frac{1}{1+\alpha}} \gamma_s^{\frac{1+m}{1+\alpha}}$.
\end{proof}

%
%
% \begin{theorem}[stability criteria]
%   criteria result
% \end{theorem}

\section{Heat conducting model}

\subsection{Formulation of Relative perturbations}
% \begin{equation}
%  \begin{aligned}
%   \partial_\tau U &= \Sigma_{xx},\\
%   \partial_\tau \Gamma &= \frac{1}{\tau}\frac{1+\alpha}{1+m} \Big(U-\Gamma\Big),\\
%   \partial_\tau \Theta &= \frac{1}{\tau}\Big(\Sigma U - \Theta\Big) + \kappa\left(\frac{1+\alpha}{1+m}\right)^{\frac{m}{1+m}}\tau^{\frac{\alpha-m}{1+m}}\Theta_{xx}, \\
%   \Sigma&=\Theta^{-\alpha}\Gamma^m U^n.
%  \end{aligned}
% \end{equation}
% When $\kappa=0$, it retains simpler form,
The new independent variable $\zeta$ will also be probed. Given that, $\Gamma\big(\zeta,x\big)$, $\Theta\big(\zeta,x\big)$, $U\big(\zeta,x\big)$, $\Sigma\big(\zeta,x\big)$ are defined by \begin{equation*}
 \begin{aligned}
  \Gamma\big(\zeta,x\big)&=\frac{\gamma(\zeta,x)}{\gamma^*_s(\zeta)}, \quad \Theta\big(\zeta,x\big)=\frac{\theta(\zeta,x)}{\theta^*_s(\zeta)}, \quad U\big(\zeta,x\big)=\frac{u(\zeta,x)}{u^*}=u(\zeta,x),\\
  \Sigma\big(\zeta,x\big)&=\frac{\sigma(\zeta,x)}{\sigma^*_s(\zeta)} \quad \Longrightarrow \quad \Sigma = \Theta^{-\alpha}\Gamma^m U^n.
 \end{aligned}
\end{equation*}
\begin{equation}
 \begin{aligned}
  \partial_\zeta U &= \left( \frac{1+\alpha}{1+m}\right)^{\frac{-\alpha}{1+\alpha}}\zeta^{\frac{-\alpha+m}{1+\alpha}} \Sigma_{xx},\\
  \partial_\zeta \Gamma &= \frac{1}{\zeta} \Big(U-\Gamma\Big),\\
  \partial_\zeta \Theta &= \frac{1}{\zeta}\Big(\Sigma U - \Theta\Big) + \kappa\Theta_{xx}, \\
  \Sigma&=\Theta^{-\alpha}\Gamma^m U^n.
 \end{aligned}
\end{equation}

For each $\ell\ge0$ we obtain
\begin{equation} \label{eq:l-system_heat}
 \begin{aligned}
  \partial_\zeta U_\ell &= -\left(\frac{1+\alpha}{1+m}\right)^{\frac{-\alpha}{1+\alpha}}\zeta^{\frac{-\alpha+m}{1+\alpha}}\l^2 \Sigma_\ell,\\
  \partial_\zeta\Gamma_\ell &= \frac{1}{\zeta}(U_\ell-\Gamma_\ell),\\
  \partial_\zeta\Theta_\ell &= \frac{1}{\zeta}\frac{1+m}{1+\alpha}\Big(\Sigma_\ell+ U_\ell -\Theta_\ell\Big) - \kappa\l^2\Theta_\ell,\\
  0&=\Sigma_\ell + \alpha\Theta_\ell -m\Gamma_\ell - nU_\ell .
 \end{aligned}
\end{equation}


With the fourth equation we remove one of the variables. The choice of removing $\Theta_\ell$ gives the linear system
\begin{equation} \label{eq:l-system_heat2}
\begin{aligned}
  \begin{pmatrix} U_\ell\\ \Gamma_\ell \\ \Sigma_\ell \end{pmatrix}'
  &= \left[\overbrace{\kappa\begin{pmatrix}
   0 & 0 & 0\\
   0 & 0 & 0\\
   n\l^2 & m\l^2 & -\l^2
  \end{pmatrix} }^{\triangleq A_{\ell,\infty}}
  + \zeta^{\tfrac{-\alpha+m}{1+\alpha}}
  \overbrace{\left(\frac{1+\alpha}{1+m}\right)^{\frac{-\alpha}{1+\alpha}}\begin{pmatrix}
   0 & 0 & -\l^2\\
   0 & 0 & 0\\
   0 & 0 & -n\l^2
 \end{pmatrix} }^{\triangleq A_{\ell,\frac{-\alpha+m}{1+\alpha}}}\right.\\
  &+  \left.  \frac{1}{\zeta}\overbrace{\frac{1+m}{1+\alpha}
  \begin{pmatrix}
   0 & 0 & 0\\
   \frac{1+\alpha}{1+m} & -\frac{1+\alpha}{1+m} & 0\\
   n+ \frac{-\alpha+m}{1+m} & m\frac{-\alpha+m}{1+m}& -(1+\alpha)
  \end{pmatrix} }^{\triangleq A_{\ell,1}}\right] \begin{pmatrix} U_\ell\\ \Gamma_\ell \\ \Sigma_\ell \end{pmatrix}
\end{aligned}
\end{equation}




\pagebreak
\section{Localizing self-similar solutions} \label{sec:selfsimilar}
Next, we focus in the region $\alpha - m - n >0$ where linearized analysis predicts instability. As usual, linear
analysis predicts the very initiation of instability. Beyond the initial stage, nonlinear effects become important, and, in particular,
they are expected to kill the oscillations of high-frequency modes and lead to localization.
The behavior in this regime is captured by a class of self-similar solutions for \eqref{pls} recently developed  in \cite{LKT17}. 
They are constructed using similarity methods and techniques from dynamical systems and
turn out to exhibit dynamic localization.
We review here this result, but with a better exposition that reveals the intimate
connection to the scaling properties of the system of relative perturbations \eqref{plsrel}.
The result is stated as follows:


\begin{theorem}[\cite{LKT17}] \label{mainthm1}
Suppose the material parameters $(\alpha, m, n)$ take values in the range \eqref{eq:paramrange} and satisfy $-\alpha+m+n<0$. If $n>0$ is sufficiently small, then
for each $U_0>0$, $\Gamma_0>0$ with ratio in the interval
\begin{equation} \label{eq:restriction}
 \frac{2(1+\alpha) -n}{D} < \frac{U _0}{\Gamma _0} < \frac{2(1+\alpha) -n}{D} + \frac{4(1+\alpha)(\alpha-m-n)(1+m)}{D(1+m+n)^2},
\end{equation}
there is a focusing self-similar solution to the system \eqref{pls} of the form
\begin{equation}
\label{ssol}
\begin{aligned}
 \Gamma (\tau,x) &= \left(\tau^{1+ 2 \lambda}\right) ^{ \frac{1+ \alpha}{D}} \bG(\tau ^\lambda x), & U (\tau,x) &=\left(\tau^{1+ 2 \lambda}\right) ^{ \frac{1+ \alpha}{D}} \bU( \tau ^\lambda x),\\
 \Theta (\tau,x) &= \left(\tau^{1+ 2 \lambda}\right) ^{ \frac{1+ m +n}{D}} \bTh( \tau ^\lambda x), & \Sigma (\tau,x) &= \left(\tau^{1+ 2 \lambda}\right) ^{ \frac{- \alpha +m+n}{D}} \bS( \tau ^\lambda x),\\
 V (\tau,x) &= \tau ^{-\lambda} \left(\tau^{1+ 2 \lambda}\right) ^{ \frac{1+ \alpha}{D}} \bV( \tau ^\lambda x),
\end{aligned}
\end{equation}
where the focusing rate $\lambda$ is determined by 
\begin{equation} \label{eq:lambda}
 \lambda = \Big(\frac{U _0}{\Gamma _0} - \frac{2(1+\alpha)-n}{D}\Big)\frac{D}{2(1+m)}  > 0 ,
\end{equation}
and the profiles $\big(\bG(\xi), \bU(\xi), \bTh(\xi),\bS(\xi),\bV(\xi)\big)$ as  functions of $\xi = \tau^\lambda x$ have the properties
that $\bV(\xi)$ odd while  $\bU (\xi)$, $\bG(\xi)$, $\bTh(\xi)$, $\bS(\xi)$ are even functions.
\end{theorem}


The reader should recall that  usually parabolic systems (such as \eqref{pls}) admit self simlar solutions that are constant on lines $\xi = \frac{x}{t^\rho}$, $\rho > 0$, and capture
the effect of diffusion. By contrast, the self similar solutions  \eqref{ssol} propagate information on lines $x t^\lambda = const$ that focus around the origin. 
We further demand that these profiles are localizing. We will call a self-similar function of the form
$$
F(\tau,x) = t^b \bar F(x \tau^\lambda) \, , \quad \mbox{with $\bar F(-\xi) = \bar F(\xi)$ and $\lambda > 0$}
$$
{\it localizing} if it has the asymptotic behavior
$$
\bar F(\xi) = \BO (\xi^p)    \quad \mbox{ as $\xi \to \infty$ }
$$
and satisfies that $p < 0$ when $b > 0$ while $p > 0$ when $b < 0$. Under this definition, if $F(t,0)$ grows then $F(t,x)$ grows at a slower
rate when $x \ne 0$, while when $F(t,0)$ decays then $F(t,x)$ decays at a slower rate at $x \ne 0$. We will call a self-similar function with an odd-profile 
$\bar F(-\xi) = - \bar F(\xi)$ localizing when its derivative $F_x( t,x)$  behaves as above.







\begin{proposition}[\cite{LKT17}] \label{mainthm2}
If $-\frac{1+m+n}{\alpha-m-n}\ne-1$ then 
$\big(\bG(\xi), \bU(\xi), \bTh(\xi),\bS(\xi),\bV(\xi)\big)$ has the  properties:

 \begin{enumerate}
  \item[(i)] $\bU (\xi)$, $\bG(\xi)$, $\bTh(\xi)$, $\bS(\xi)$ are even functions satisfying
    \begin{equation*}
    \Gamma_\xi(0) = \Theta_\xi(0)=\Sigma_\xi(0) = {U}_\xi(0)=0 \, ,
  \end{equation*}
and have the asymptotic behavior as $\xi \rightarrow 0$ 
  \begin{equation} \label{eq:ss_asymp0}
  \begin{aligned}
    \Gamma(\xi) -\Gamma_0 &= \Gamma^{''}(0)\frac{\xi^2}{2} + o(\xi^2), & \Gamma^{''}(0)&<0,\\
    \red \Theta(\xi) - \Theta_0 &= \Theta^{''}(0)\frac{\xi^2}{2} + o(\xi^2), & \Theta^{''}(0)&<0,\\
    \red \Sigma(\xi) - \Sigma_0  &= \Sigma^{''}(0)\frac{\xi^2}{2} + o(\xi^2), & \Sigma^{''}(0)&>0, \\
    U(\xi) - U_0 &= U^{''}(0)\frac{\xi^2}{2} + o(\xi^2), & U^{''}(0)&<0,
   \end{aligned}
  \end{equation}
where 
  $$
  \Theta_0  =  c^{-\frac{1}{1+\alpha}}\Gamma_0^{\frac{m}{1+\alpha}} U_0^{\frac{1+n}{1+\alpha}} \, , \quad 
  \Sigma_0 = c^{\frac{\alpha}{1+\alpha}}\Gamma_0^{\frac{m}{1+\alpha}} U_0^{-\frac{\alpha-n}{1+\alpha}} \, .
  $$


\item[(ii)] $\bV(\xi)$ is an odd function satisfying ${V}(0) = 0$ and 
 \begin{equation} \label{eq:ss_asymp1}
  \begin{aligned}
  V(\xi) - U_0\xi &= U^{''}(0)\frac{\xi^3}{6} + o(\xi^3)  \, \quad \mbox{as $\xi \to 0$ }\, .
  \end{aligned}
  \end{equation}


 \item[(iii)] The asymptotic behavior as $\xi \rightarrow \infty$ is given by
  \begin{equation} \label{eq:ss_asymp1}
  \begin{aligned}
    \Gamma(\xi) &= \BO\big(\xi^{-\frac{1+\alpha}{\alpha-m-n}}), & V(\xi) &= \BO\big(1), &    \Theta(\xi) &= \BO\big(\xi^{-\frac{1+m+n}{\alpha-m-n}}),\\
   \Sigma(\xi) &= \BO\big(\xi), &   U(\xi) &= \BO\big(\xi^{-\frac{1+\alpha}{\alpha-m-n}})
  \end{aligned}
  \end{equation}
 \end{enumerate}
\end{proposition}
% The exposition will be of two parts: One is to investigate the scale invariance property of \eqref{PLS} and the other is, out of those investigations, to construct a family of localizing self-similar solutions.  We address that the numerical computation in Section \ref{sec:num_cont} conducted by the method of numerical continuation is freshly added.

Note that due to the relations \eqref{eq:ss_asymp1} the self-similar profiles $U(\xi)$, $\Gamma (\xi)$, $\Theta (\xi)$, $\Sigma(\xi)$ and $V(\xi)$ are all localizing.
In addition we can easily compute the asymptotic behavior o fhe resulting solution. For instance , ...

...

Also, numerical computation of the associated heteroclinic orbits is performed in ... using ... and are presented here.


\bigskip

\subsection{Outline of the construction of self-similar profiles}

A special feature of the presentation below, which differs than the original approach in \cite{LKT17}, is that the similarity structure of]the system of relative perturbations is exploited.
(The original approach was based on the similarity of the original system.) This has the advantage of capturing well the growth
rates of the profiles.


First we investigate the scale invariance property of \eqref{PLS}. For prescribed parameters $( \alpha, m,n)$, the system \eqref{PLS} has the two-parameters symmetry group as to the scale invariance. More precisely, if the tuple $W=\big( \Gamma(\tau,x)$, $U(\tau,x)$, $\Theta(\tau,x)$, $\Sigma(\tau,x)$, $V(\tau,x)\big)$ is a solution of the system \eqref{PLS}, then the rescaled tuple $W _R=\big( \Gamma _R(\tau,x)$, $U _R(\tau,x)$, $\Theta _R(\tau,x)$, $\Sigma _R(\tau,x)$, $V _R(\tau,x)\big)$ defined by
\begin{align*}
 \Gamma _R(\tau,x) &= \left(\frac{\nu^2}{\rho}\right) ^{ \frac{1+ \alpha}{D}} \Gamma( \rho \tau, \nu x), & U _R(\tau,x) &=\left(\frac{\nu^2}{\rho}\right) ^{ \frac{1+ \alpha}{D}} U( \rho \tau, \nu x),\\
 \Theta _R(\tau,x) &= \left(\frac{\nu^2}{\rho}\right) ^{ \frac{1+ m +n}{D}} \Theta( \rho \tau, \nu x), & \Sigma _R(\tau,x) &= \left(\frac{\nu^2}{\rho}\right) ^{ \frac{- \alpha +m+n}{D}} \Sigma( \rho \tau, \nu x),\\
 V _R(\tau,x) &= \nu ^{-1} \left(\frac{\nu^2}{\rho}\right) ^{ \frac{1+ \alpha}{D}} V( \rho \tau, \nu x), & \text{with } D&=1+ 2\alpha -m-n
\end{align*}
again is a solution of \eqref{PLS}. Of course, this observation is suggestive in an attempt constructing a special solution that keeps a symmetry subgroup. Of particularly interesting cases are the self-similar ones where we require
$$ W(\tau,x) = W _R(1,\tau ^\lambda x)= \bar{W}(\tau ^\lambda x) \quad \text{by setting} \quad \rho = \tau ^{-1}, \quad \nu=\tau^ \lambda.$$
%We will see the self-similar profiles with $\lambda>0$ exhibit the localization.
This is to consider ansatz
\begin{align*}
 \Gamma (\tau,x) &= \left(\tau^{1+ 2 \lambda}\right) ^{ \frac{1+ \alpha}{D}} \bG(\tau ^\lambda x), & \bU (\tau,x) &=\left(\tau^{1+ 2 \lambda}\right) ^{ \frac{1+ \alpha}{D}} U( \tau ^\lambda x),\\
 \Theta (\tau,x) &= \left(\tau^{1+ 2 \lambda}\right) ^{ \frac{1+ m +n}{D}} \bTh( \tau ^\lambda x), & \bS (\tau,x) &= \left(\tau^{1+ 2 \lambda}\right) ^{ \frac{- \alpha +m+n}{D}} \Sigma( \tau ^\lambda x),\\
 V (\tau,x) &= \tau ^{-\lambda} \left(\tau^{1+ 2 \lambda}\right) ^{ \frac{1+ \alpha}{D}} \bV( \tau ^\lambda x), \quad \big(\bV _\xi(\xi) = \bU(\xi)\big),
\end{align*}
where we set $\xi\triangleq\tau ^\lambda x$. By plugging into the system, we obtain a system of ordinary differential equations
\begin{equation}\label{eq:sseq}
\begin{aligned}
 (1+2 \lambda) \left(\frac{1 + \alpha}{D}\right) \bU + \lambda \xi \bU_\xi &= \bS_{\xi\xi},\\
 (1+2 \lambda) \left(\frac{1 + \alpha}{D}\right) \bG + \lambda \xi \bG_\xi&=\frac{1+ \alpha}{1+m}(\bU - \bG),\\
 (1+2 \lambda) \left(\frac{1 + m+n}{D}\right) \bTh + \lambda \xi \bTh_\xi&=\bS\bU - \bTh,\\
 \bS &= \bTh^ {- \alpha} \bG ^m \bU ^n,\\
\end{aligned}
\end{equation}
We address that the formulation with $\bV(\xi)$ instead of $\bU(=\bV_\xi)$ is possible and will be with the equation
% with $\eqref{eq:sseq}_1$ equivalently replaceble (among smooth solutions) by
$$ (1+2 \lambda) \left(\frac{1 + \alpha}{D}\right) \bV - \lambda \bV + \lambda \xi \bV_\xi=\bS_\xi,$$
instead of $\eqref{eq:sseq}_1$.

It turns out that the classical o.d.e. theory to solve this system does not apply because of the singularity at $\xi=0$. Note that the coefficient $\xi$ not only makes the system non-autonomous but also singular because it is at terms of highest derivatives. The major ingredient advanced in \cite{KOT14,LKT17} is to turn this singular non-autonomous system into an autonomous dynamical system and formulate the problem to find the suitable heteroclinic orbit of the derived autonomous system.

To this end, we introduce nonlinear transformations and the new independent variable $\eta = \log\xi$. First we notice is the reflexive symmetry that if $\big(\bG(\xi)$, $\bU(\xi)$, $\bTh(\xi)$, $\bS(\xi)$, $\bV(\xi)\big)$ is a solution of \eqref{eq:sseq} then $\big(\bG(-\xi)$, $\bU(-\xi)$, $\bTh(-\xi)$, $\bS(-\xi)$, $-\bV(-\xi)\big)$ again is a solution. We thus look for $\bG(\xi)$, $\bU(\xi)$, $\bTh(\xi)$, $\bS(\xi)$ that are even and $\bV(\xi)$ that is odd function of $\xi$. This symmetry leads to imposing conditions
\begin{equation} \label{eq:extension_cond}
 0=\bG'(0)=\bU'(0)=\bTh'(0)=\bS'(0)=\bV(0).
\end{equation}
and formulate the problem to solve \eqref{eq:sseq} on the right half space $[0,\infty)$ with the boundary values \eqref{eq:extension_cond}.

Second we observe is its own scale invariance of the system for self-similar variables. If $\bar{W}(\xi)=\big(\bG(\xi)$, $\bU(\xi)$, $\bTh(\xi)$, $\bS(\xi)$, $\bV(\xi)\big)$ is a solution of \eqref{eq:sseq} then $\bar{W}_R(\xi)=\big(\bG _R(\xi)$, $\bU_R(\xi)$, $\bTh_R(\xi)$, $\bS_R(\xi)$, $\bV_R(\xi)\big)$ again is a solution if
{\blue
\begin{align*}
\bG _R(\xi)&=\rho^ \frac{2(1+ \alpha)}{D}\bG(\rho\xi), & \bU _R(\xi) &= \rho^ \frac{2(1+ m+n)}{D}\bU(\rho\xi), &
\bTh _R(\xi)&=\rho^ \frac{2(1+ \alpha)}{D}\bTh(\rho\xi), \\
\bS _R(\xi)&=\rho^ \frac{2(-\alpha+m+n)}{D}\bS(\rho\xi), & \bV _R(\xi) &=\rho^ {\frac{2(1+ \alpha)}{D}-1}\bV(\rho\xi)
\end{align*}
}
One could consider the monomial solutions that are self-similar to \eqref{eq:sseq}, that is
\begin{align*}
\bG_M(\xi) &= \xi^ {-\frac{2(1+ \alpha)}{D}}, & \bU_M(\xi) &= \xi^ {-\frac{2(1+ m+n)}{D}}, &
\bTh_M(\xi)&=\xi^ {-\frac{2(1+ \alpha)}{D}}, \\
\bS_M(\xi)&=\xi^ {-\frac{2(-\alpha+m+n)}{D}}, & \bV_M(\xi) &=\xi^ {-\frac{2(1+ \alpha)}{D}+1}.
\end{align*}
This is not compatible to \eqref{eq:extension_cond} but this symmetry allows us to consider a sort of relatively perturbed variables which entail the nice scaling property. We introduce residuals
\begin{align*}
\bg(\xi) &= \xi^ \frac{2(1+ \alpha)}{D}\bG(\xi), & \bu(\xi) &= \xi^ \frac{2(1+ m+n)}{D}\bU(\xi), &
\bth(\xi)&=\xi^ \frac{2(1+ \alpha)}{D}\bTh(\xi), \\
\bs(\xi)&=\xi^ \frac{2(-\alpha+m+n)}{D}\bS(\xi), & \bv(\xi) &=\xi^ {\frac{2(1+ \alpha)}{D}-1}\bV(\xi).
\end{align*}
Along defining new variables, we also introduce the independent variable $\eta = \log\xi$ which runs from $-\infty$ to $\infty$. To denote variables in the new independent variable, we annotate the hat for example $\hg(\eta)|_{\eta=\log\xi}=\bg(\xi)$ and we denote $\frac{d}{d\eta} \hg = \dot\hg$, then the system new variables satisfy reads
\begin{equation} \label{eq:tildesys}
\begin{aligned}
%\frac{1+ \alpha}{D} \bu + \lambda \dot\bu &= \left(\frac{-2(- \alpha +m+n)}{D}\right)\left(\frac{-2(- \alpha +m+n)}{D}-1\right)\bs + \left(\frac{-2(- \alpha +m+n)}{D} + \frac{-2(- \alpha +m+n)}{D}-1\right)\dot\bs + \ddots\bs,\\
\frac{1+ \alpha}{D} \hu + \lambda \dot\hu &= d_1(d_1-1)\hs + (d_1+d_1-1)\dot\hs + \ddot\hs,\\
\frac{1+ \alpha}{D} \hg + \lambda \dot\hg &= \frac{1 + \alpha}{1+m}( \hu - \hg ),\\
\frac{1+ m+n}{D} \hth + \lambda \dot\tth &= \hs\hu-\hth,\\
\hs &=\hth^{- \alpha} \hg ^m \hu ^n, \quad \text{where $d_1=\frac{-2(- \alpha +m+n)}{D}$},
\end{aligned}
\end{equation}
Again, $\eqref{eq:tildesys}_1$ is replaceble by
$$\frac{1+ \alpha}{D} \hv + \lambda \dot\hv = d_1 \hs -\dot\hs, \quad \text{with relation} \quad \hu = -\frac{1+m+n}{D}\hv + \dot\hv.$$
Note that this is an autonomous system. Before we complete the formulation, one technical point needs to be clarified. The existence theorem in \cite{LKT17} on the system (18) in the paper will be quoted in our presentation, but \eqref{eq:tildesys} is not identical to this system. To arrive at the system (18) in \cite{LKT17}, the constant factors are to be multiplied: 
$$(\tu,\tg,\tth,\tv):=\left(\frac{1+m}{1+\alpha}\right)^{\frac{1}{-\alpha+m+n}}(\hu,\hg,\hth,\hv), \quad \ts:=\frac{1+m}{1+\alpha}\hs, \quad \tilde\lambda :=\frac{1+m}{1+\alpha}\lambda.$$
This difference stems from that while \eqref{eq:tildesys} has been derived from \eqref{plsrel}, the system (18) in \cite{LKT17} has been derived from \eqref{PLS} of original system, and in the course of transformations the different constant factor has been multiplied. While we find the current exposition more systematic, not to reproduce the proof over again, we proceed with the system (18) in \cite{LKT17} that are
\begin{equation} \label{eq:tildesys2}
 \begin{aligned}
  a_0\tg + \tilde\lambda\dtg &=\tu, & a_0&=\frac{2+2\alpha-n}{D},\\
  b_0\tv + \tilde\lambda\dtv &=-d_1 \ts + \dts,& b_0&=\frac{1+m}{D},\\
  c_0\tth+ \tilde\lambda\dtth&=\ts\tu, & c_0&=\frac{2(1+m)}{D},\\
  \ts &=\tth^{-\alpha}\tg^m\tu^n, \\
  -b_1\tv+\dtv &= \tu, & b_1&=\frac{1+m+n}{D}.
 \end{aligned}
\end{equation}
% when we derive \eqref{eqref:tildesys}, 
% 
% 
% 
% the system (18) in \cite{LKT17} has been derived from \eqref{PLS} of the original formulation while we derived \eqref{eq:tildesys} from \eqref{plsrel} of relative perturbations. The constant multiple of $(\tu_1,\tg_1,\tth_1,\tv_1)=\frac{1+m}{1+\alpha}^{\frac{1}{-\alpha+m+n}}(\tu,\tg,\tth,\tv)$ with the $\lambda_1 =\frac{1+m}{1+\alpha}\lambda$ give the system (18), saying there are no essential differences between the two. While we find the current formulation more systematic, not to reproduce the proof over again, we proceed with the system (18) in \cite{LKT17} without annotating any subscripts, which is
% \begin{equation} \label{eq:tildesys2}
%  \begin{aligned}
%   a_0\tg + \lambda\dtg &=\tu, & a_0&=\frac{2+2\alpha-n}{D},\\
%   b_0\tv + \lambda\dtv &=-d_1 \ts + \dts,& b_0&=\frac{1+m}{D},\\
%   c_0\tth+ \lambda\dtth&=\ts\tu, & c_0&=\frac{2(1+m)}{D},\\
%   \ts &=\tth^{-\alpha}\tg^m\tu^n, \\
%   -b_1\tv+\dtv &= \tu, & b_1&=\frac{1+m+n}{D},
%  \end{aligned}
% \end{equation}

\noindent
\tcr{MGL To-Do:    A better writing is needed here, jplease write the precise transformation that carries you
 from the system \eqref{eq:tildesys} to the system \eqref{eq:tildesys2}; use different notations for the two systems, 
 also give some details about which
 numerical relations are used to simplify then follow \cite{LKT17}. }

Having that intermediate transformation taken, lastly, we come up with a new choice of variables that tend to equilibria as $\eta \rightarrow \pm \infty$ and accommodate
orbits with power growth or decay at infinities. Reasoning for the choice is following. \eqref{eq:tildesys2} can be rewritten in the form
\begin{equation}
 \label{eq:tildesys3}
\begin{aligned}
\frac{d}{d\eta}{(\ln{\tg})}  &=  \tfrac{1}{\tilde\lambda} \big (- a_0 +  \frac{\tu}{\tg} \big ),
\\
\frac{d}{d\eta}{(\ln{\tv})}  &=  - b_1 + \frac{\tu}{\tv} ,
\\
\frac{d}{d\eta}{(\ln{\tth})} &=   \tfrac{1}{\tilde\lambda} \big (- c_0 +  \frac{\ts \tu}{\tth} \big ),
\\
\frac{d}{d\eta}{(\ln{\ts})} &= d_1 + b \frac{\tv}{\ts} + \tilde\lambda + \frac{\tu}{\ts}, \quad b = \frac{1+m}{D} + \frac{1+m+n}{D}\tilde\lambda
\end{aligned}
\end{equation}
% \begin{equation}
%  \label{eq:tildesys2}
% \begin{aligned}
% \frac{d}{d\eta}{(\ln{\tg})}  &=  \tfrac{1}{\tilde\lambda} \left (- \frac{2+2\alpha-n}{D} +  \frac{\tu}{\tg} \right ),
% \\
% \frac{d}{d\eta}{(\ln{\tv})}  &=  - \frac{1+m+n}{D} + \frac{\tu}{\tv} ,
% \\
% \frac{d}{d\eta}{(\ln{\tth})} &=   \tfrac{1}{\tilde\lambda} \left (- \frac{2(1+m)}{D} +  \frac{\ts \tu}{\tth} \right ),
% \\
% \frac{d}{d\eta}{(\ln{\tth})} &= \frac{2(-\alpha+m+n)}{D} + \left( \frac{1+m}{D} + \frac{1+m+n}{D}\tilde\lambda\right) \frac{\tv}{\ts} + \tilde\lambda + \frac{\tu}{\ts}
% \end{aligned}
% \end{equation}
and we project our expectation that the physical solution can at most exponentially grow (power growth in $\xi$). That is to say the right-hand-sides are presumed to be bounded. 

This leads us to define
\begin{equation}\label{eq:pqrdef}
 \begin{aligned}
  p :=\frac{\tg}{\ts}, \quad q :=b \frac{\tv}{\ts},  \quad r = \frac{\tu}{\tg} = \left ( \frac{\ts}{ \tth^{-\alpha} \tg^{m+n}} \right )^\frac{1}{n}  , \quad s := \frac{\ts\tg}{\tth} \, .
 \end{aligned}
\end{equation}
Here we view it as describing the evolution of $(\tg,\tv,\tth,\ts)$ with $\tu$ determined by $\tu = \left ( \frac{\ts}{ \tth^{-\alpha} \tg^m} \right )^\frac{1}{n}$. The transformation $(p,q,r,s) \leftrightarrow (\tg,\tv, \tth,\ts)$ is a bijection in the positive sector with the inverse determined by
$$
\tg = p^\frac{1+\alpha}{D} s^\frac{\alpha}{D} r^\frac{n}{D} \, \quad \tth = p^\frac{1+m+n}{D} s^\frac{m+n-1}{D} r^\frac{2n}{D}
$$
and then
$$
\sigma = \frac{1}{\tg} p \, , \quad v = \frac{1}{b} \sigma \, q
$$
After a cumbersome but straightforward calculation, we derive the $(p,q,r,s)$-system:
\begin{equation}\label{eq:slow} \tag{S}
 \begin{aligned}
 \dot{p} &=p\Big(\frac{1}{\tilde\lambda}(r-a) + 2- \tilde\lambda p r -q\Big), \quad a=\frac{2+2\alpha-n}{D} + \frac{2(1 + \alpha)}{D}\tilde\lambda,\\
 \dot{q} &=q\Big(1 -\tilde\lambda p r -q\Big) + b p r, \quad \quad \quad \quad b=\frac{1+m}{D} + \frac{1+m+n}{D}\tilde\lambda,\\
 n\dot{r} &=r\Big(\frac{\alpha-m-n}{\tilde\lambda(1+\alpha)}(r-a) + \tilde\lambda pr + q +\frac{\alpha}{\tilde\lambda}r\big(s- \frac{1+m+n}{1+\alpha}\big) + \frac{n\alpha}{\tilde\lambda(1+\alpha)}\Big),\\
 \dot{s} &=s\Big(\frac{\alpha-m-n}{\tilde\lambda(1+\alpha)}(r-a) + \tilde\lambda pr + q - \frac{1}{\tilde\lambda}r\big(s- \frac{1+m+n}{1+\alpha}\big) - \frac{n}{\tilde\lambda(1+\alpha)}\Big).
 \end{aligned}
\end{equation}
It is this system \eqref{eq:slow} we conduct analysis. While the connections to the physical variables are hidden, the behavior as an autonomous dynamical system is quite nicer than the original one, later we will see that we can take a compact set in which the meaningful orbits are confined.

Quoting the proof of Theorem \ref{mainthm1} from \cite{LKT17}, we provide instead a few logical steps to take. Among all the orbits of \eqref{eq:slow}, the orbits we look for must have the specific asymptotic behavior as $\eta \rightarrow -\infty$, since the boundary condition \eqref{eq:extension_cond} were imposed. This characterises three dimensional invariant surface. Again we project our expectation of the power behavior as $\eta \rightarrow \infty$, which identifies the one dimensional curve on the hypersurface, that is the heteroclinic we look for.

The main ingredient of the actual construction of the heteroclinic is the {\it geometric singular perturbation theory}. Since its developments in '70s, this theory has been applied  for many different problems. The interested readers are referred to \cite{Fenichel79}. As to applying the theory, we make used of the so called {\it fast-slow} structure the system possesses: Recall that $n$ is assumed to be very small positive number, and $\eqref{eq:slow}_3$ indicates the system has this structure.
% 
% 
% Having arrived at \eqref{eq:slow}, it remains to clarify what orbit we are aiming to capture and it needs to perform a phase space analysis of it revealing the nonlinear behavior of the dynamical system; regarding the former, the far field conditions at $\pm\infty$ is imposed accordingly to \eqref{eq:extension_cond}. This is to go beyond introducing the existence result of the localizing self-similar solutions and we refer the exposition to \cite{LKT17}. One most important aspect is that \eqref{eq:slow} possesses the {\it fast-slow} structure due to the presence of the small parameter $n$ in the left-hand-side of $\eqref{eq:slow}_3$. We exploit this aspect to prove the existence of the heteroclinic orbit of \eqref{eq:slow},  applying the {\it geometric singular perturbation theory} \cite{Fenichel79}. Further exposition of this discussion is available in \cite{LKT17}. These results lead to the Theorem \ref{mainthm1} and \ref{mainthm2}.
% 
% 


\pagebreak

\appendix
% \section*{Appendix}
\renewcommand\thetheorem{\Alph{theorem}}
\newcounter{tmp}
\setcounter{theorem}{\thetmp}
\section{Analyticity of a simple root of a polynomial}



\begin{theorem}{\cite[p. 24]{Hormander66}} \label{thm:anal} Let $f_j(w,z)$, $j=1,\cdots,m$, be analytic functions of $(w,z)=(w_1,\cdots,w_m,z_1,\cdots,z_n)$ in a neighborhood of a point $(w^0,z^0)$ in $\mathbb{C}^m\times \mathbb{C}^n$, and assume that $f_j(w^0,z^0)=0$, $j=1,\cdots,m$ and that 
$$ \det\Big( \frac{\partial f}{\partial w} \Big) \ne 0 \quad \text{at $(w^0,z^0)$, where $\frac{\partial f}{\partial w}$ is the $m\times m$ Jacobian matrix.} $$
Then the equations $f_j(w,z)=0$, $j=1,\cdots,m$ have a uniquely determined analytic solution $w(z)$ in a neighborhood of $z_0$, such that $w(z^0)=w^0$.
\end{theorem}



\section{Stability analysis of a non-autonomous system}
\begin{theorem} \label{prop:stab}
 Let $F(t)$ be the solution map of the dynamical system
 $$ y' = U(t)y, \quad \text{with} \quad |U_{ij}(t)|\le C<\infty \quad \forall i,j=1,\cdots,d$$
 and assume its operator norm 
 $$\big\|F(t-\tau)\big\| \le \exp\left(\int_\tau^t \theta(s)\; ds\right) \quad \text{for some $\theta$.}$$
 Suppose
 \begin{equation}
x' = U(t)x + \mathcal{E}(t)x, \label{eq:x}
 \end{equation}
and $\mathcal{E}_{ij}(t)$ decays as $t \rightarrow \infty$ so that they are integrable, i.e.,
 $$ \int_{a_0}^\infty |\mathcal{E}_{ij}(s)| \; ds < \infty, \quad \forall i,j=1,\cdots,d \quad \text{for some $a_0$.}$$
 Then, $\exists a$ such that for any $t\ge a$
 $$ |x(t)| \le 2\exp\left(\int_a^t \theta(s)\; ds\right) |x(a)|.$$
 Furthermore, if $y(t)=F(t-a)x(a)$, it holds for any $t\ge a$ that
 $$\frac{2}{3}\left\|\exp\left(-\int_a^t \theta(s)\; ds\right)y(t)\right\|_{L^\infty[a,t]} \le  \left\|\exp\left(-\int_a^t \theta(s)\; ds\right)x(t)\right\|_{L^\infty[a,t]}.$$
%  there is an $x(t)$ such that $$\displaystyle \frac{2}{3}\le\frac{\|x\|_{L^\infty[a,t]}}{\big|F(t)\big|} \le 2.$$ 
%  
%  
%  $\displaystyle \frac{2}{3}\frac{\|y(t)\|_{L^\infty[a,t]}}{\|F\|(t)}\le\frac{\|x(t)\|_{L^\infty[a,t]}}{\|F\|(t)} \le 2\frac{\|y(t)\|_{L^\infty[a,t]}}{\|F\|(t)},$ where $y(t)$ is the solution that attains the operator norm $y(t)=F(t)y(0)$ is the one attains the operator norm, i.e.,
%  $$ \lim_{t \rightarrow \infty}\frac{y(t)}{\|F\|(t)} = y_\infty \quad \text{nontrivial,}$$
%  there is a solution $x(t)$ of \eqref{eq:x} such that 
\end{theorem}
\begin{proof}
%  Without loss, we may set $h(t) = \exp\left(\int_0^t \theta(s)\; ds\right)$ by setting $\theta(t) = \log(h(t))'$.
% One can formally consider \eqref{eq:x} in the form
%  \begin{align*}
%   x' - \theta(t)x &= \big(U(t)-\theta(t)\big)x + \mathcal{E}x \quad \text{or}\\
%   \xi' &=\big(U(t)-\theta(t)\big)\xi + \mathcal{E}\xi, \quad \text{where $\xi\triangleq \exp\left(-\int^t \theta(s)\; ds\right) x$}.
%  \end{align*}
 Let $F(t-\tau)\exp\left(-\int_\tau^t \theta(s)\; ds\right)=:G(t-\tau)$ noticing that $\|G(t-\tau)\|\le 1$.
Then the solution $x(t)$ of \eqref{eq:x}, by letting $\xi(t):=x(t)\exp\left(-\int_a^t \theta(s)\; ds\right)$, defines the solution of the following integral equation  $$ \xi(t) = G(t-a)\xi(a) + \int_a^t G(t-\tau)\mathcal{E}(\tau)\xi(\tau) \; d\tau \quad \text{for some $a$ we choose later.}$$ 

%  be the solution map associated to $U(t)-\theta(t)$. Since $F(t)x(0)=x(t)$ \\$= \exp\left(\int_0^t \theta(s)\; ds\right)\xi(t)= \exp\left(\int_0^t \theta(s)\; ds\right)G(t)\xi(0)=\exp\left(\int_0^t \theta(s)\; ds\right)G(t)x(0)$ for all $x(0)\in \mathbb{R}^d$, $G(t) = F(t)\exp\left(-\int_0^t \theta(s)\; ds\right)$ and $\|G\|(t)\le 1$. Using $G(t)$ we further write

 Now, if $\bar\xi$ is a bounded function then the the expression right-hand-side well-defines an operator $S:L^\infty([a,\infty)) \mapsto L^\infty([a,\infty))$ because $\|G\|$ is bounded and $\mathcal{E}$ is integrable. If $a$ is chosen so large that $\int_a^\infty \|G\| \; |\mathcal{E}_{ij}|\; d\tau < \frac{1}{2}$ then this is a contraction mapping. Therefore, $\xi(t):=x(t)\exp\left(-\int_a^t \theta(s)\; ds\right)$, $t\ge a$ is the unique fixed point of this mapping.
%  
%  
%  Since $C([a,\infty))$ is a closed subspace of $L^\infty([a,\infty))$, the unique continuous function that has the initial value $\xi(a)$ is attained for each $\xi(a)$. Of course, this is the unique solution that is defined from $x(t)$.%, this defines the unique solution of Since $\xi(a)$ was arbitrary, every independent solution is attained.
 
 From the integral representation of $\xi(t)$, for any $t\ge a$ we have
 \begin{align*}
 \|\xi\|_{L^\infty([a,t])} &\le \|G(t-a)\xi(a)\|_{L^\infty([a,t])} + \left\|\int_a^t G(t-\tau)\mathcal{E}(\tau)\xi(\tau) \; d\tau\right\|_{L^\infty([a,t])} \\
 &\le \|G(t-a)\xi(a)\|_{L^\infty([a,t])} + \frac{1}{2} \|\xi\|_{L^\infty([a,t])},\\
 \text{or} \quad \|\xi\|_{L^\infty([a,t])} &\le 2 \|G(t-a)\xi(a)\|_{L^\infty([a,t])} \le 2|\xi(a)|.
 \end{align*}
 This gives the proof of the first assertion. The second assertion again follows from the triangular inequality
 \begin{align*}
  \big\|G(t-a)\xi(a)\big\|_{L^\infty([a,t])} \le\frac{3}{2} \|\xi(t)\|_{L^\infty([a,t])}.
 \end{align*}
% 
%  
%  Now, pick $y(t)$ that attains the operator norm  let $\xi(t)$ and $\xi(a)$ be that its semi-group part is the $y(t)$ grows as same as the operator norm does. 
 
%  \begin{align*}
% \big\|\xi(t) - G(t-a)\xi(a)\big\|_{L^\infty([a,t])} &= \left\|\int_a^t G(t-\tau)\mathcal{E}(\tau)\xi(\tau) \; d\tau\right\|_{L^\infty([a,t])} \le \frac{1}{2} \|\xi(t)\|_{L^\infty([a,t])},
% \\
% \big\|G(t-a)\xi(a)\big\|_{L^\infty([a,t])} &\le \big\|\xi(t) - G(t-a)\xi(a)\big\|_{L^\infty([a,t])} + \big\|\xi(t)\big\|_{L^\infty([a,t])} 
% \\
% &\le \frac{3}{2} \|\xi(t)\|_{L^\infty([a,t])}.
%  \end{align*}
\end{proof}
%
% \begin{proposition}[stability of triangular matrix] \label{prop:tri-stab}
% Suppose $y' = U(t) y$ and $U(t)$ be an upper triangular matrix with bounded entries. Suppose that there is a function $\theta(t)$ and a constant $A$ such that for all diagonal entries $\lambda_i(t)$,
% \begin{align} \label{eq:stabcond}
% %   &\lim_{t \rightarrow \infty} t^{-d} \int_0^t \theta(s)- Re\lambda_i(s)\; ds \rightarrow \infty,\\
%  &\int_{t_1}^{t_2} \theta(s)-Re\lambda_i(s)\; ds > -A \quad \text{whenever $0\le t_1 \le t_2$.}
% \end{align}
% Then for $i=1,\cdots,d$
% \begin{equation} \label{eq:triestim}
% |y_{i}(t)| \le C\big( 1 + t + \cdots + t^{d-i}\big) \exp\left( \int_0^t \theta(s)\;ds\right)% \rightarrow 0 \quad \text{as $t \rightarrow \infty$.}
% \end{equation}
% \end{proposition}
% \begin{proof}
% We prove the assertion by induction in the descending order. For $i=d$, $y_d(t)=y_d(0)\exp\left( \int_0^t \lambda_d(s)\;ds\right)=y_d(0)\exp\left( \int_0^t \theta(s)-\theta(s)+\lambda_d(s)\;ds\right)$ and thus $|y_d(t)| \le Ce^A\exp\left( \int_0^t \theta(s)\;ds\right)$ by \eqref{eq:stabcond}. Now, if the statement holds for $i$, we see that
% \begin{align*}
%  y_{i-1}' -\lambda_{i-1}y_{i-1} = \sum_{j\ge i} U_{i-1,j}(t)y_j(t) = g(t)
% \end{align*}
% and $|g(t)| \le C\big( 1 + t + \cdots + t^{d-i}\big) \exp\left( \int_0^t \theta(s)\;ds\right)$ for some another constant $C$ because $U_{ij}(t)$ are bounded. Therefore
% \begin{align*}
%  y_{i-1}(t) &= \exp\left( \int_0^t \lambda_{i-1}(s)\;ds\right) \left(y_{i-1}(0) + \int_0^t \exp\left( \int_0^\tau -\lambda_{i-1}(s)\;ds\right)g(\tau) \; d\tau\right)\\
%  &=\exp\left( \int_0^t \theta(s)-\theta(s)+\lambda_{i-1}(s)\;ds\right) y_{i-1}(0) \\
%  &+ \exp\left( \int_0^t \theta(s)\;ds\right)\underbrace{\int_0^t \exp\left( \int_\tau^t \lambda_{i-1}(s)-\theta(s)\;ds\right)g(\tau)\exp\left( -\int_0^\tau \theta(s)\;ds\right) \; d\tau}_{\triangleq \varphi(t)}
%  \end{align*}
% Note that $\varphi(0)=0$ and $|\varphi'(t)| = \left|g(t)\exp\left( \int_0^t -\theta(s)\;ds\right)\right|\le C\big( 1 + t + \cdots + t^{d-i}\big)$ by the assumptions. Thus $|\varphi(t)|\le C\big( 1 + t + \cdots + t^{d-i+1}\big)$ for some another constant $C$. Therefore
% $$|y_{i-1}(t)| \le C\big( 1 + t + \cdots + t^{d-i+1}\big)\exp\left( \int_0^t \theta(s)\;ds\right).$$
% \end{proof}


\begin{theorem}{\cite[Diagonal Version]{CL55}}\label{thm:CL} Let $x(t)\in \mathbb{R}^d$ and $x'(t) = \big(\Lambda(t) + \mathcal{E}(t)\big)x$, where $\Lambda(t)$ is a diagonal matrix with diagonal entries $\lambda_j(t)$, $j=1,\cdots,d$ bounded and $\mathcal{E}(t)$ is a matrix with entries $\mathcal{E}_{ij}$ integrable, i.e., $\int_{a_0}^\infty |\mathcal{E}_{ij}(s)|\; ds < \infty$ $\forall i,j=1,\cdots,d$ for some $a_0$.
Fix an index $k$. Suppose we can find the constant $A$ so that either of the following two membership conditions holds for every $i$.

$i \in I_1$ if
\begin{align}
 &\int_{a_0}^\infty Re(\lambda_k(s) -\lambda_i(s))\; ds \rightarrow \infty \quad \text{as $t \rightarrow \infty$ for some $a_0$},\label{eq:I1cond1}\\
 &\int_{t_1}^{t_2} Re(\lambda_k(s) -\lambda_i(s))\; ds > -A, \quad \text{whenever $t_2\ge t_1\ge 0$} \label{eq:I1cond2}
\end{align}
and $i \in I_2$ if
\begin{align}
 &\int_{t_1}^{t_2} Re(\lambda_k(s) -\lambda_i(s))\; ds < A, \quad \text{whenever $t_2\ge t_1\ge 0$}. \label{eq:I2cond}
\end{align}
Then there is an orbit $\varphi_k(t)$ $t\ge a$ for some $a$ such that,
\begin{equation}
 \lim_{t \rightarrow \infty} \varphi_k(t) \exp\left(-\int_{a}^t \lambda_k(s)\; ds\right) = \hat{k}, \quad \text{where $\hat{k}$ is the $k$-th coordinate basis of $\mathbb{R}^d$.}
\end{equation}
\end{theorem}
\begin{proof}
Component-wisely, we can write
\begin{align*}
 \xi_i' & = (\lambda_i-\lambda_k)\xi_i + \mathcal{E}_{ij}\xi_j, \quad \text{where $\xi = \exp\left(-\int_a^t \lambda_k(s) \; ds\right)x$.}
\end{align*}
We look for a solution of the integral representation
\begin{align*}
 \xi_i(t) &= \hat{k}_i + \int_a^t \exp\left(\int_\tau^t \lambda_i(s)-\lambda_k(s) \; ds\right)\mathcal{E}_{ij}(\tau)\xi(\tau) \; d\tau && \text{if $i\in I_1$,}\\
% \end{align*}
% if $i\in I_1$, where $\hat{k}_i = \delta_{k i}$ of Kronecker delta and
% \begin{align*}
 \xi_i(t) &= \hat{k}_i -\int_t^\infty \exp\left(\int_t^\tau -\lambda_i(s)+\lambda_k(s) \; ds\right)\mathcal{E}_{ij}(\tau)\xi(\tau) \; d\tau && \text{if $i\in I_2$,}
\end{align*}
where $\hat{k}_i = \delta_{k i}$ of Kronecker delta. Let $t\ge a$ so large that $e^A\int_a^\infty |\mathcal{E}_{ij}(\tau)|\; d\tau < \frac{1}{2}$. Then by \eqref{eq:I1cond2} and \eqref{eq:I2cond}, for given $\bar\xi(t)$ bounded $t\ge a$, the expression right-hand-side defines an operator $S$ that maps $\bar\xi$ to another bounded function that is defined by the expression. In particular, $\xi=S\bar\xi$ has same initial data $\xi_i(a) = \hat{k}_i$ if $i\in I_1$.
% and has same finial data $\displaystyle\lim_{t \rightarrow \infty} \xi_i(t) = \hat{k}_i$ if $i\in I_2$.
By the choice of $a$, $\|S\xi - S\bar\xi\|_\infty \le \frac{1}{2}\|\xi-\bar\xi\|_\infty$, $t\ge a$ and thus $S$ is a contraction mapping. The integral equation has the unique solution and $\|\xi_i(t)\|_\infty \le 2$ because the integral is bounded above by $ \frac{1}{2} \|\xi\|_\infty$ and $|\hat{k}|=1$.

Now we show that $|\xi(t)-\hat{k}| \rightarrow 0$ as $t \rightarrow \infty$. For given $\epsilon>0$, we show we can choose $t_0$ so large that for $t\ge t_0$, $\big|\xi_i(t)-\hat{k}_i\big| \le \epsilon$. If $i\in I_1$, we divide the integral into two parts
\begin{align*}
 &\big|\xi_i(t)-\hat{k}_i\big| \le \left|\left\{ \int_a^{t_1} + \int_{t_1}^t \right\} \exp\left(\int_\tau^t \lambda_i(s)-\lambda_k(s) \; ds\right)\mathcal{E}_{ij}(\tau)\xi(\tau) \; d\tau \right|.
\end{align*}
By choosing $t_1$ so large the second integral can be made smaller than $ \frac{\epsilon}{2}$ for all $t\ge t_1$. The first integral $$\left|\exp\left(\int_a^t \lambda_i(s)-\lambda_k(s) \; ds\right)\int_a^{t_1} \exp\left(\int_a^\tau -\lambda_i(s)+\lambda_k(s) \; ds\right)\mathcal{E}_{ij}(\tau)\xi(\tau) \; d\tau \right|$$
can be made smaller than $ \frac{\epsilon}{2}$ because the latter integral in the compact interval $[a, t_1]$ is finite and $\exp\left(\int_a^t \lambda_i(s)-\lambda_k(s) \; ds\right) \rightarrow 0$ as $t \rightarrow \infty$ by \eqref{eq:I1cond1}.

If $i\in I_2$, then $\left|\int_t^\infty \exp\left(\int_t^\tau -\lambda_i(s)+\lambda_k(s) \; ds\right)\mathcal{E}_{ij}(\tau)\xi(\tau) \; d\tau\right| \rightarrow 0$ as $t \rightarrow \infty$.
\end{proof}

%
% \section{Stability analysis of a non-autonomous system}
% We study the stability of a linear non-autonomous system
% $$ x'=A(t) x.$$%, \quad A(t) \rightarrow A_\infty \quad \text{as $t \rightarrow \infty$}.$$
% The entries $A_{ij}(t)$, $i,j=1,\cdots,d$ are assumed to be bounded. As the vector field is Lipschitz in $x$, it admits the semi-group $F(t)$ (in fact a group for all $t$) characterized by the property $y(t)=F(t)y(0)$. When it is a system of non-autonomous equations, due to the coupling, the maximum of the real parts of the eigenvalues does not necessarily accounts for the operator norm of $F(t)$; $A(t)$ with negative eigenvalues can results in having an exponentially growing solution, showing $\|F(t)\|$ exponentially grows.
%
% On the contrary, like for diagonal matrices, $\|F(t)\|$ can be read off from $A(t)$ when the coupling is in an appropriate sense weaker. $A(t)$ can be by similarity transform turned into the form that has less coupling such as diagonal or Jordan normal form. The price is an additional term due to that the decomposition is also time dependent. More precisely, If $A(t) = P(t)U(t)P(t)^{-1}$, and if $y\triangleq P(t)^{-1}x$ then $y$ satisfies the system
% \begin{equation} y' = U(t)y - P(t)^{-1}P'(t)y. \label{eq:after_fact} \end{equation}
%
% Assuming the operator norm of the solution map $F(t)$ associated to $U(t)$ is computable, we look if the coupling by the remainder term is negligible. Following proposition sheds light on which cases are feasible ones.
% \begin{proposition} \label{prop:stab}
%  Suppose
%  \begin{equation}
% x' = U(t)x + \mathcal{E}(t)x, \label{eq:x}
%  \end{equation}
% where $U_{ij}(t)$ are bounded and the associated solution map $F(t)$ has the norm $\|F\|(t) \le h(t)=\exp\left(\int_0^t \theta(s)\; ds\right)$ for some $\theta(t)$. Assume that $\mathcal{E}_{ij}(t)$ decay as $t \rightarrow \infty$ so that they are integrable, i.e.,
%  $$ \int_{a_0}^\infty |\mathcal{E}_{ij}(s)| \; ds < \infty, \quad \forall i,j=1,\cdots,d \quad \text{for some $a_0$.}$$
%  Then, for some $a$ every solution $\|x\|_{L^\infty[a,t]} \le 2h(t)$. Furthermore, if $y(t)=F(t)y(0)$ is the one attains the operator norm, i.e.,
%  $$ \lim_{t \rightarrow \infty}\frac{y(t)}{\|F\|(t)} = y_\infty \quad \text{nontrivial,}$$
%  there is a solution $x(t)$ of \eqref{eq:x} such that $\displaystyle \frac{2}{3}\frac{\|y(t)\|_{L^\infty[a,t]}}{\|F\|(t)}\le\frac{\|x(t)\|_{L^\infty[a,t]}}{\|F\|(t)} \le 2\frac{\|y(t)\|_{L^\infty[a,t]}}{\|F\|(t)}.$
% \end{proposition}
% \begin{proof}
% %  Without loss, we may set $h(t) = \exp\left(\int_0^t \theta(s)\; ds\right)$ by setting $\theta(t) = \log(h(t))'$.
% We rewrite \eqref{eq:x} in the form
%  \begin{align*}
%   x' - \theta(t)x &= \big(U(t)-\theta(t)\big)x + \mathcal{E}x \quad \text{or}\\
%   \xi' &=\big(U(t)-\theta(t)\big)\xi + \mathcal{E}\xi, \quad \text{where $\xi\triangleq \exp\left(-\int_0^t \theta(s)\; ds\right) x$}.
%  \end{align*}
%  Let $G(t)$ be the solution map associated to $U(t)-\theta(t)$. Since $F(t)x(0)=x(t)$ \\$= \exp\left(\int_0^t \theta(s)\; ds\right)\xi(t)= \exp\left(\int_0^t \theta(s)\; ds\right)G(t)\xi(0)=\exp\left(\int_0^t \theta(s)\; ds\right)G(t)x(0)$, for all $x(0)\in \mathbb{R}^d$, $G(t) = F(t)\exp\left(-\int_0^t \theta(s)\; ds\right)$ and $\|G\|(t)\le 1$. Using $G(t)$ we further write
%  $$ \xi(t) = G(t-a)\xi(a) + \int_a^t G(t-\tau)\mathcal{E}(\tau)\xi(\tau) \; d\tau \quad \text{for some $a$ we choose later.}$$
%  Note that if $\bar\xi$ is a bounded function then the the expression right-hand-side well-defines an operator $S:L^\infty([a,\infty)) \mapsto L^\infty([a,\infty))$ because $t\ge \tau\ge a$, $G$ is bounded and $\mathcal{E}$ is integrable. $S\bar\xi$ has the same initial value $\xi(a)$. If $a$ is chosen so large that $\int_a^\infty \|G\| \; |\mathcal{E}_{ij}|\; d\tau < \frac{1}{2}$ then this is a contraction mapping. Therefore solution exists with the initial value $\xi(a)$. Since $\xi(a)$ was arbitrary, every independent solution is attained.
%  From the integral representation of $\xi(t)$, we have
%  \begin{align*}
%  \|\xi(t)\|_{L^\infty([a,t])} &\le \|G(t-a)\xi(a)\|_{L^\infty([a,t])} + \left\|\int_a^t G(t-\tau)\mathcal{E}(\tau)\xi(\tau) \; d\tau\right\|_{L^\infty([a,t])} \\
%  &\le \|G(t-a)\xi(a)\|_{L^\infty([a,t])} + \frac{1}{2} \|\xi(t)\|_{L^\infty([a,t])},\\
%  \|\xi(t)\|_{L^\infty([a,t])} &\le 2 \|G(t-a)\xi(a)\|_{L^\infty([a,t])} \le 2\sup_t \|G\|(t) \le 2.
%  \end{align*}
%  This gives the boundedness of the solution. Now, let $\xi(t)$ and $\xi(a)$ be that its semi-group part is the $y(t)$ grows as same as the operator norm does. The second assertion follows from
%  \begin{align*}
% &\big\|\xi(t) - G(t-a)\xi(a)\big\|_{L^\infty([a,t])} = \left\|\int_a^t G(t-\tau)\mathcal{E}(\tau)\xi(\tau) \; d\tau\right\|_{L^\infty([a,t])} \le \frac{1}{2} \|\xi(t)\|_{L^\infty([a,t])},\\
% &\big\|G(t-a)\xi(a)\big\|_{L^\infty([a,t])} \le \big\|\xi(t) - G(t-a)\xi(a)\big\|_{L^\infty([a,t])} + \big\|\xi(t)\big\|_{L^\infty([a,t])} \le \frac{3}{2} \|\xi(t)\|_{L^\infty([a,t])}.
%  \end{align*}
% \end{proof}
% \begin{remark}
%  \begin{enumerate}
%   \item Proposition \ref{prop:stab} does not assume a special structure of $U(t)$ but assumes its associated $\|F\|(t)$ has estimated, no matter what couplings take place. Given the estimate, it persists under the perturbation by $\mathcal{E}$, whose entries are integrable. Moreover, the largest growth of the perturbed system is precisely as same as unperturbed one.
%   \item The condition that the $\mathcal{E}_{ij}(t)$ is integrable cannot be relaxed. $x=t$ of  $x'=\frac{x}{t}$, is an example in which the asymptotic stability fails by the contribution of the non-integrable coefficient.
%  \end{enumerate}
% \end{remark}
%
% It remains how we can estimate $\|F\|(t)$ that is associated with $U(t)$. For completeness, we next provide an estimate of $\|F\|(t)$ associated to an upper triangular matrix. This is extended to a $U(t)$ that is block diagonal with upper triangular blocks because the solution map $F(t)$ is then decoupled as well block-wisely.
%
%  \begin{proposition}[stability of triangular matrix] \label{prop:tri-stab}
%  Suppose $y' = U(t) y$ and $U(t)$ be an upper triangular matrix with bounded entries. Suppose that there is a function $\theta(t)$ and a constant $A$ such that for all diagonal entries $\lambda_i(t)$,
%  \begin{align} \label{eq:stabcond}
% %   &\lim_{t \rightarrow \infty} t^{-d} \int_0^t \theta(s)- Re\lambda_i(s)\; ds \rightarrow \infty,\\
%   &\int_{t_1}^{t_2} \theta(s)-Re\lambda_i(s)\; ds > -A \quad \text{whenever $0\le t_1 \le t_2$.}
%  \end{align}
%  Then for $i=1,\cdots,d$
%  \begin{equation} \label{eq:triestim}
% |y_{i}(t)| \le C\big( 1 + t + \cdots + t^{d-i}\big) \exp\left( \int_0^t \theta(s)\;ds\right)% \rightarrow 0 \quad \text{as $t \rightarrow \infty$.}
%  \end{equation}
% \end{proposition}
% \begin{proof}
%  We prove the assertion by induction in the descending order. For $i=d$, $y_d(t)=y_d(0)\exp\left( \int_0^t \lambda_d(s)\;ds\right)=y_d(0)\exp\left( \int_0^t \theta(s)-\theta(s)+\lambda_d(s)\;ds\right)$ and thus $|y_d(t)| \le Ce^A\exp\left( \int_0^t \theta(s)\;ds\right)$ by \eqref{eq:stabcond}. Now, if the statement holds for $i$, we see that
%  \begin{align*}
%   y_{i-1}' -\lambda_{i-1}y_{i-1} = \sum_{j\ge i} U_{i-1,j}(t)y_j(t) = g(t)
%  \end{align*}
%  and $|g(t)| \le C\big( 1 + t + \cdots + t^{d-i}\big) \exp\left( \int_0^t \theta(s)\;ds\right)$ for some another constant $C$ because $U_{ij}(t)$ are bounded. Therefore
%  \begin{align*}
%   y_{i-1}(t) &= \exp\left( \int_0^t \lambda_{i-1}(s)\;ds\right) \left(y_{i-1}(0) + \int_0^t \exp\left( \int_0^\tau -\lambda_{i-1}(s)\;ds\right)g(\tau) \; d\tau\right)\\
%   &=\exp\left( \int_0^t \theta(s)-\theta(s)+\lambda_{i-1}(s)\;ds\right) y_{i-1}(0) \\
%   &+ \exp\left( \int_0^t \theta(s)\;ds\right)\underbrace{\int_0^t \exp\left( \int_\tau^t \lambda_{i-1}(s)-\theta(s)\;ds\right)g(\tau)\exp\left( -\int_0^\tau \theta(s)\;ds\right) \; d\tau}_{\triangleq \varphi(t)}
%   \end{align*}
%  Note that $\varphi(0)=0$ and $|\varphi'(t)| = \left|g(t)\exp\left( \int_0^t -\theta(s)\;ds\right)\right|\le C\big( 1 + t + \cdots + t^{d-i}\big)$ by the assumptions. Thus $|\varphi(t)|\le C\big( 1 + t + \cdots + t^{d-i+1}\big)$ for some another constant $C$. Therefore
%  $$|y_{i-1}(t)| \le C\big( 1 + t + \cdots + t^{d-i+1}\big)\exp\left( \int_0^t \theta(s)\;ds\right).$$
% \end{proof}
%
%
% \begin{remark}
%   This estimate shows that the couplings through off-diagonal terms does spoil the asymptotic stability when the eigenvalues decay to $0$ as $t \rightarrow \infty$ but not so much to be integrable. Suppose that every eigenvalue has negative real parts bounded by $\theta(t) = -\frac{k}{t}$ for some $k$. Then the estimate \eqref{eq:triestim} fails to guarantee the asymptotic stability unless otherwise $k\ge d$ because $\exp\left( \int_1^t \theta(s)\;ds\right) =\exp\left( -\int_1^t \frac{k}{s}\;ds\right) =t^{-k}.$ This is the typical behavior of the Jordan block.
% \end{remark}
%
% Lastly, the case where $U(t)$ is diagonal is considered. In this case, not only the explicit formula of the solution map $F(t) = \textrm{diag}\left[ \exp\left(\int_0^t \lambda_i(s)\; ds\right) \right]$ is available but also it is possible to spot every modes. Indeed, the unperturbed system is totally decoupled so that every mode remains unmixed and survives. This property is in fact shared by the upper triangular matrix; this can be seen by just taking $\hat{\ell}$ the coordinate basis as an initial data. The next Coddington-Levinson theorem shows that for a diagonal matrix, under the suitable spectral gap conditions, the ability to discern all modes persists after adding a weak coupling $\mathcal{E}(t)$. The spectral condition is notably precise where the accumulated contribution by non-integrable real parts of eigenvalues matters.
%
% \begin{theorem}{\cite[Diagonal Version]{CL55}}\label{thm:CL} Let $x(t)\in \mathbb{R}^d$ and $x'(t) = \big(\Lambda(t) + \mathcal{E}(t)\big)x$, where $\Lambda(t)$ is a diagonal matrix with diagonal entries $\lambda_j(t)$, $j=1,\cdots,d$ bounded and $\mathcal{E}(t)$ is a matrix with entries $\mathcal{E}_{ij}$ integrable, i.e., $\int_{a_0}^\infty |\mathcal{E}_{ij}(s)|\; ds < \infty$ $\forall i,j=1,\cdots,d$ for some $a_0$.
% Fix an index $\ell$. Suppose we can find the constant $A$ so that either of the following two membership conditions holds for every $i$.
%
% $i \in I_1$ if
% \begin{align}
%  &\int_{a_0}^\infty Re(\lambda_\ell(s) -\lambda_i(s))\; ds \rightarrow \infty \quad \text{as $t \rightarrow \infty$ for some $a_0$},\label{eq:I1cond1}\\
%  &\int_{t_1}^{t_2} Re(\lambda_\ell(s) -\lambda_i(s))\; ds > -A, \quad \text{whenever $t_2\ge t_1\ge 0$} \label{eq:I1cond2}
% \end{align}
% and $i \in I_2$ if
% \begin{align}
%  &\int_{t_1}^{t_2} Re(\lambda_\ell(s) -\lambda_i(s))\; ds < A, \quad \text{whenever $t_2\ge t_1\ge 0$}. \label{eq:I2cond}
% \end{align}
% Then there is an orbit $\varphi_\ell(t)$ $t\ge a$ for some $a$ such that,
% \begin{equation}
%  \lim_{t \rightarrow \infty} \varphi_\ell(t) \exp\left(-\int_{a}^t \lambda_\ell(s)\; ds\right) = \hat{\ell}, \quad \text{where $\hat{\ell}$ is the $\ell$-th coordinate basis of $\mathbb{R}^d$.}
% \end{equation}
% \end{theorem}
% \begin{proof}
% Component-wisely, we can write
% \begin{align*}
%  \xi_i' & = (\lambda_i-\lambda_\ell)\xi_i + \mathcal{E}_{ij}\xi_j, \quad \text{where $\xi = \exp\left(-\int_a^t \lambda_\ell(s) \; ds\right)x$.}
% \end{align*}
% We look for a solution of the integral representation
% \begin{align*}
%  \xi_i(t) &= \hat\ell_i + \int_a^t \exp\left(\int_\tau^t \lambda_i(s)-\lambda_\ell(s) \; ds\right)\mathcal{E}_{ij}(\tau)\xi(\tau) \; d\tau && \text{if $i\in I_1$,}\\
% % \end{align*}
% % if $i\in I_1$, where $\hat{\ell}_i = \delta_{\ell i}$ of Kronecker delta and
% % \begin{align*}
%  \xi_i(t) &= \hat\ell_i -\int_t^\infty \exp\left(\int_t^\tau -\lambda_i(s)+\lambda_\ell(s) \; ds\right)\mathcal{E}_{ij}(\tau)\xi(\tau) \; d\tau && \text{if $i\in I_2$,}
% \end{align*}
% where $\hat{\ell}_i = \delta_{\ell i}$ of Kronecker delta. Let $t\ge a$ so large that $e^A\int_a^\infty |\mathcal{E}_{ij}(\tau)|\; d\tau < \frac{1}{2}$. Then by \eqref{eq:I1cond2} and \eqref{eq:I2cond}, for given $\bar\xi(t)$ bounded $t\ge a$, the expression right-hand-side defines an operator $S$ that maps $\bar\xi$ to another bounded function that is defined by the expression. In particular, $\xi=S\bar\xi$ has same initial data $\xi_i(a) = \hat{\ell}_i$ if $i\in I_1$.
% % and has same finial data $\displaystyle\lim_{t \rightarrow \infty} \xi_i(t) = \hat{\ell}_i$ if $i\in I_2$.
% By the choice of $a$, $\|S\xi - S\bar\xi\|_\infty \le \frac{1}{2}\|\xi-\bar\xi\|_\infty$, $t\ge a$ and thus $S$ is a contraction mapping. The integral equation has the unique solution and $\|\xi_i(t)\|_\infty \le 2$ because the integral is bounded above by $ \frac{1}{2} \|\xi\|_\infty$ and $|\hat{\ell}|=1$.
%
% Now we show that $|\xi(t)-\hat\ell| \rightarrow 0$ as $t \rightarrow \infty$. For given $\epsilon>0$, we show we can choose $t_0$ so large that for $t\ge t_0$, $\big|\xi_i(t)-\hat{\ell}_i\big| \le \epsilon$. If $i\in I_1$, we divide the integral into two parts
% \begin{align*}
%  &\big|\xi_i(t)-\hat{\ell}_i\big| \le \left|\left\{ \int_a^{t_1} + \int_{t_1}^t \right\} \exp\left(\int_\tau^t \lambda_i(s)-\lambda_\ell(s) \; ds\right)\mathcal{E}_{ij}(\tau)\xi(\tau) \; d\tau \right|.
% \end{align*}
% By choosing $t_1$ so large the second integral can be made smaller than $ \frac{\epsilon}{2}$ for all $t\ge t_1$. The first integral $$\left|\exp\left(\int_a^t \lambda_i(s)-\lambda_\ell(s) \; ds\right)\int_a^{t_1} \exp\left(\int_a^\tau -\lambda_i(s)+\lambda_\ell(s) \; ds\right)\mathcal{E}_{ij}(\tau)\xi(\tau) \; d\tau \right|$$
% can be made smaller than $ \frac{\epsilon}{2}$ because the latter integral in the compact interval $[a, t_1]$ is finite and $\exp\left(\int_a^t \lambda_i(s)-\lambda_\ell(s) \; ds\right) \rightarrow 0$ as $t \rightarrow \infty$ by \eqref{eq:I1cond1}.
%
% If $i\in I_2$, then $\left|\int_t^\infty \exp\left(\int_t^\tau -\lambda_i(s)+\lambda_\ell(s) \; ds\right)\mathcal{E}_{ij}(\tau)\xi(\tau) \; d\tau\right| \rightarrow 0$ as $t \rightarrow \infty$.
% \end{proof}
%
% \subsection{Possibility of having absolutely continuous factorization}
% As indicated, $U(t)$ is presumed to be in the form we can compute its associated semi-group $F(t)$ while $A(t)$ may not. We have so far discussed the persistence properties of system defined by coefficient $U(t)$, the semi-group part, under the presence of weak coupling $\mathcal{E}(t)$.  Now, we turn our attentions to the continuous factorization of $A(t)$ along with the system \eqref{eq:after_fact}.
%
% All earlier discussions assume the integrability of $|\mathcal{E}_{ij}(t)|$ and this was sharp. In order for the factored system \eqref{eq:after_fact} to be analyzed in this context, the term $-P(t)^{-1}P'(t)$ appeared in \eqref{eq:after_fact} has to decay so fast that the entries are integrable. We look for a factorization where $P'(t)$ decays suitably and $P(t)$ and $P(t)^{-1}$ are absolutely continuous up to infinity.
%
% At best possibility is the diagonalization. If so, then we apply the Theorem by Coddington-Levinson under the spectral gap conditions there. However, not every matrix is diagonalizable and more importantly, this factorization is prone to a perturbation, or we do not in general expect the continuity of the eigenvectors.
%
% At worst possibility is the block Schur factorization, $A(t)=U(t)B(t)U^*(t)$, where $U(t)$ is unitary and $B(t)$ is block upper triangular. This factorization is in general abound and not unique while continuous one can be picked. See [Delci] for the smooth factorization: Under the assumption that the spectrum is partitioned into disjoint sets for all time, the method admits a continuous factorization. What can be said to lie in the middle is a Jordan canonical form. It always exists and is unique up to the permutation of blocks but this factorization lacks the continuity as well. Roughly, better the chance we have for the factorization to be continuous, worse the coupling properties we obtain.



\vfil\eject

\begin{thebibliography}{10}

\bibitem{BPV91}
{\sc M.Bertsch, L.A.Peletier and S.M.Verduyn Lunel},
{\sl The effect of temperature dependent viscosity on shear flow of incompressible fluids}
\newblock SIAM J. Math. Anal., {\bf 22} (1991), 328-343.

\bibitem{CL55}
{\sc E.A. Coddington and N. Levinson},
{\it Theory of ordinary differential equations},
McGraw-Hill Inc., New York, 1955.

\bibitem{CDHS84}
{\sc R.J. Clifton, J. Duffy, K.A. Hartley and T.G. Shawki},
{ On critical conditions for shear band formation at high strain rates}, {\it Scripta Met.,} {\bf 18} (1984), 443-448.

\bibitem{Clifton90}
{\sc R.J. Clifton},  High strain rate behaviour of metals,
{\it Applied Mechanics Reviews}
{\bf 43} (1990), S9-S22.

\bibitem{CCHD79} 
{\sc L.S.Costin, E.E.Crisman, R.H.Hawley, and J.Duffy},
{\sl On the localization of plastic flow in mild steel tubes under dynamic torsional loading}
 In "Proc. 2nd Conf. on the Mechanical Properties of Materials at high rates of strain",  
 Inst. Phys. Conf. Ser. no 47, Oxford, 90, 1979.

\bibitem{DH83}
{\sc C.M.Dafermos and L.Hsiao},
{\sl Adiabatic shearing of incompressible fluids with temperature dependent viscosity}
Quart. Appl. Math., {\bf 41} (1983), 45 - 58.

\bibitem{FM87}
{\sc C. Fressengeas and A. Molinari }
{ Instability and localization of plastic flow in shear at high strain rates}, {\it J Mech Phys Solids} {\bf 35} (1987), 185-211.

\bibitem{HDH87}
{\sc K.A.Hartley, J.Duffy, and R.J.Hawley},
{\sl Measurement of the temperature profile during shear band formation in steels deforming at high-strain rates}
 J. Mech. Physics Solids, {\bf 35} (1987), 283-301.

\bibitem{Hormander66}
{\sc L.~H\"ormander},
{\it An introduction to complex analysis in several variables},
D. Van Nostrand Inc., Princeton, (1966).


\bibitem{HN77}
{\sc J.W.~Hutchinson and K.W.~Neale},
Influence of strain-rate sensitivity on necking under uniaxial tension,
{\it  Acta Metallurgica} {\bf 25} (1977), 839-846.

\bibitem{KT09}
{\sc Th. Katsaounis and A.E.~Tzavaras},
Effective equations for localization and shear band formation,
{\it SIAM J. Appl. Math.}  {\bf 69} (2009), 1618--1643.

\bibitem{LKT17}
{\sc M.-G.~Lee, Th. Katsaounis and A.E.~Tzavaras},
Localization in adiabatic shear flow via geometric theory of singular perturbations,
arXiv preprint arXiv:1707.05283 (2017).


\bibitem{MC87}
{\sc A. Molinari and R.J. Clifton},
{ Analytical characterization of shear localization in thermoviscoplastic materials}
 {\it J. Appl. Mech.},  {\bf 54} (1987), 806-812.
 
\bibitem{SC89}
{\sc T.G. Shawki and R.J. Clifton},
Shear band formation in thermal viscoplastic materials,
{\it Mechanics of Materials}
{\bf 8 } (1989), 13--43.

\bibitem{Tzavaras87}
{\sc A.E. Tzavaras},
{Effect of thermal softening in shearing of strain-rate dependent materials}
{\it Arch. Rational Mech. Analysis}, {\bf 99} (1987), 349 - 374.

\bibitem{Tzavaras92}
{\sc A.E. Tzavaras},
Nonlinear analysis techniques for shear band formation at high strain-rates,
 {\it Applied Mechanics Reviews}
{\bf  45} (1992), S82--S94.

\bibitem{WW88}
{\sc T.W.Wright and J.W.Walter},
{\sl On stress collapse in adiabatic shear bands}
J. Mech. Phys. Solids, {\bf 35} (1988), 701-720.

\bibitem{Wright02}
{\sc T.W. Wright},
{\sl The Physics and Mathematics of Shear Bands},
Cambridge University Press, 2002.


\bibitem{ZH44}
{\sc  C. Zener and J.H. Hollomon},
{ Effect of strain rate upon plastic flow of steel.}
{\it J. Appl. Physics}, {\bf 15} (1944), 22-32.



-----------------------------------------------------------------------

\tcr{ Further References }



\bibitem{Fenichel74}
{\sc N.~Fenichel},
Asymptotic stability with rate conditions,
{\it Indiana Univ. Math. J.} {\bf 23} (1974) 1109--1137.

\bibitem{Fenichel77}
{\sc N.~Fenichel},
Asymptotic stability with rate conditions \textrm{II},
{\it Indiana Univ. Math. J.} {\bf 26} (1977) 81--93.

\bibitem{Fenichel79}
{\sc N.~Fenichel},
Geometric singular perturbation theory for ordinary differential equations,
{\it J. Differ. Equations} {\bf 31} (1979), 53--98.

 \bibitem{Jones95}
 {\sc C.~K. R.~T. Jones},
 Geometric singular perturbation theory, in {\it Dynamical systems}, LNM {\bf 1609} (Springer Berlin Heidelberg 1995) 44--118.

\bibitem{KOT14}
{\sc Th.~Katsaounis, J.~Olivier, and A.E.~Tzavaras},
Emergence of coherent localized structures in shear deformations of temperature dependent fluids,
{\it Archive for Rational Mechanics and Analysis} {\bf 224} (2017), 173--208.


\bibitem{KLT16}
{\sc Th. Katsaounis, M.-G. Lee, and A.E. Tzavaras},
Localization in inelastic rate dependent shearing deformations,
{\it J. Mech. Phys. of Solids} {\bf 98} (2017), 106--125.

\bibitem{Kuehn15}
{\sc C.~ Kuehn},
{\it Multiple time scale dynamics}, Applied Mathematical Sciences, Vol. {\bf 191} (Springer Basel 2015).

\bibitem{LT17}
{\sc M.-G.~Lee and A.E.~Tzavaras},
Existence of localizing solutions in plasticity via the geometric singular perturbation theory,
{\it Siam J. Appl. Dyn. Systems} {\bf 16} (2017), 337--360.



\bibitem{Szmolyan91}
{\sc P.~Szmolyan},
Transversal heteroclinic and homoclinic orbits in singular perturbation problems,
{\it J. Differ. Equations}
{\bf 92} (1991), 252--281.

\bibitem{Tzavaras86a}
{\sc A.E. Tzavaras},
Shearing of materials exhibiting thermal softening or temperature dependent viscosity,
{\em Quart.  Applied Math.} {\bf 44} (1986), 1--12.


\bibitem{Tzavaras86b}
{\sc A.E. Tzavaras},
Plastic shearing of materials exhibiting strain hardening or strain softening,
{\it Archive for Rational Mechanics and  Analysis}
{\bf 94} (1986), 39--58.


\bibitem{Tzavaras91}
{\sc A.E.Tzavaras},
{\sl Strain softening in viscoelasticity of the rate type}
J. Integral Equations Appl., {\bf 3} (1991), 195-238.





%
% \bibitem{SS_2004}
% {\sc S. Schecter and P. Szmolyan}
% Composite waves in the Dafermos regularization.
% {\it J. Dynamics Diff. Equations} {\bf 16} (2004), 847-867.
%
%
% \bibitem{wiggins_normally_1994}
% {\sc S.~Wiggins},
% {\it Normally hyperbolic invariant manifolds in dynamical  systems}, AMS {\bf 105} (Springer-Verlag New York 1994).
%
%
% \bibitem{xiao_stability_2003}
% {\sc L.~Xiao-Biao and S.~ Schecter},
% {Stability of self-similar solutions of the {D}afermos regularization of a system of conservation laws},
% {SIAM J. Math. Anal.}
% {\bf 35} (2003), 884--921.
%\bibitem{WF83}
%{\sc F.H. Wu and L.B. Freund},
%Deformation trapping due to thermoplastic instability in one-dimensional wave propagation,
%{\it J. Mech. Phys. of Solids} {\bf  32} (1984), 119-132.
%
% \bibitem{HPS_1977}
% {\sc M.W. Hirsch, C.C. Pugh, and M. Shub},
% {\it Invariant Manifolds}, LNM {\bf 583}, (Springer-Verlag, New York/Heidelberg/Berlin 1977)
%
%
%

\end{thebibliography}
\end{document}

