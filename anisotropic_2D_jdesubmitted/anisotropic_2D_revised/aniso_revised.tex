\documentclass[11pt]{amsart}
\usepackage[utf8]{inputenc}
\usepackage{amsmath}
% \usepackage{mathrsfs}
\usepackage[mathscr]{euscript}
\usepackage{psfrag}
\usepackage{graphicx}
\usepackage{caption}
\usepackage{subcaption}
% \usepackage[notref,notcite]{showkeys}
\usepackage{color}
\def\red{\color{red}}
\def\blue{\color{blue}}
% \usepackage[left=30mm,right=30mm,top=35mm,bottom=30mm,a4paper]{geometry}


%%%%%%%%%%%%%% MY  DEFINITIONS %%%%%%%%%%%%%%%%%%%%%%%%%%%


\theoremstyle{plain}
\newtheorem{Thm}{Theorem}
\newtheorem{Assertion}[Thm]{Assertion}
\newtheorem{Prop}[Thm]{Proposition}
\newtheorem{Lem}[Thm]{Lemma}
\newtheorem{Cor}[Thm]{Corollary}
\newtheorem{Main}{Main Theorem}
% \newtheorem{Def}[Thm]{Definition}
\newtheorem{Prob}[Thm]{\textcolor{red}{Problem}}
\newtheorem{Claim}[Thm]{Claim}
\newtheorem{Ques}[Thm]{\textcolor{red}{Qestion}}
\newtheorem{Ex}[Thm]{Example}
\theoremstyle{remark}
\newtheorem{Def}[Thm]{Definition}
\newtheorem{Rem}[Thm]{Remark}

\numberwithin{equation}{section}
\numberwithin{Thm}{section}
\setcounter{tocdepth}{3}


%%%%%%%%%%%%%%%%%%%%%%%%%%
\def\U{\overline{U}}
\def\C{\mathcal{C}}
\def\R{\mbox{\boldmath$R$}}
\def\J{{\bf J}}
\def\N{{\mathbf N}}
\def\M{{\mathbf M}}
\def\F{{\bf F}}
\def\E{{\bf E}}
\def\B{{\bf B}}
\def\H{{\bf H}}
\def\r{{\bf r}}
\def\div{{\nabla\cdot}}
\def\x{{\bf x}}
\def\y{{\bf y}}
\def\z{{\bf z}}
\def\w{{\bf w}}
\def\n{{\bf n}}
\def\eps{\varepsilon}
\def\ds{\displaystyle}
\def\Sigma{\mbox{\boldmath$\sigma$}}
\def\r{\mbox{\boldmath$r$}}
\def\Arg{{\,\textrm{Arg}\,}}
\def\arg{{\,\textrm{arg}\,}}

%%%%%%%%%%%%%%%%%%%%%%%%%%%%%%%%%%%%%
\begin{document}
\title[well-posedness of anisotropic conductivity reconstruction]{Existence and uniqueness in anisotropic conductivity reconstruction with Faraday's law}

\author{Yong-Jung Kim}
\address{\newline National Institute of Mathematical Sciences, 70 Yuseong-daero, Yuseong-gu, Daejeon 305-811,  Korea, \and Department of Mathematical Sciences, KAIST, 291 Daehak-ro, Yuseong-gu, Daejeon 305-701, Korea }
\email{yongkim@kaist.edu}

\author[Min-Gi Lee]{Min-Gi Lee}
\address[Min-Gi Lee]{\newline  Department of Mathematics, Kyungpook National University, 80 Daehak-ro, Buk-gu, Daegu 41566, Republic of Korea}
\email{leem@knu.ac.kr}


\thanks{This research was supported by Kyungpook National University Research Fund, 2018}

%
% \thanks{Min-Gi Lee was supported by the National Research Foundation of Korea(NRF) grant funded by the Korea government(MSIT) (No. 2020R1A4A1018190, 2021R1C1C1011867).}

%
\keywords{}
%\subjclass[2010]{}


%
% \date{\today}

\begin{abstract}
  We show that three sets of internal current densities are the right amount of data that give the existence and the uniqueness at the same time in reconstructing an anisotropic conductivity in two space dimensions. The curl free equation of Faraday's law is taken instead of the usual divergence free equation of the electrical impedance tomography. Boundary conditions related to given current densities are introduced which complete a well determined problem for conductivity reconstruction together with Faraday's law.
\end{abstract}


\maketitle

% \tableofcontents

\section{Introduction}
The human body is composed with muscle fibers and anisotropic conductivity model is required when conductivity distribution is studied as a part of medical imaging technology. This article is a study of anisotropic conductivity reconstruction problem in two dimensions. 

Suppose that $\Omega\subset\R^2$ is a bounded and {simply connected} domain of an electrical conductivity body with a smooth boundary, and $\F_k=\begin{pmatrix} f^1_k & f^2_k \end{pmatrix}:\Omega\to\R^2$, $k=1,2,3$, are three given vector fields in the domain $\Omega$. The purpose of this paper is to show the existence and the uniqueness of the anisotropic resistivity distribution,
\begin{equation}\label{ortho}
\r:=\begin{pmatrix}r^{11}&r^{12} \\ r^{21}& r^{22}\end{pmatrix},\quad r^{12}=r^{21},\quad \x:=(x,y)\in\Omega\subset\R^2,
\end{equation}
that satisfies curl free equations
\begin{equation}\label{Faraday}
\nabla\times(\r\F_k )=0\ \ \mbox{in}\ \ \Omega,\ k=1,2,3.
\end{equation}
and boundary condition specified in \eqref{hBC}. To pose a problem correctly, the right amount of boundary portion and the boundary measurements on it have to be decided dependently on $\F_k$ $k=1,2,3$, and thus introducing \eqref{hBC} needs a development of our theory. This is done in Section \ref{sec:adm} and \ref{sec:hprob}.
% 
% {\blue
%  will be also assigned
% \begin{equation}\label{FaradayBC}
% \begin{array}{l}
% \N_2\cdot(u_1,u_2) = v_{10} \ \mbox{on}\ \ {\Gamma^-_1}\subset\partial U,\\
% \N_1\cdot(u_1,u_2) = v_{20} \ \mbox{on}\ \ {\Gamma^-_2}\subset\partial U,\\
% \end{array}
% \end{equation}
% where $(u_1,u_2)$ are electric fields. The given boundary values are the linear combinations of potentials measured partly on boundary. The vector fields, $\N_1$ and $\N_2$, and the boundaries, $\Gamma_{1}^-$ and $\Gamma_{2}^-$, and the transformed domain $U$ are decided by the three vector fields $\F_1,\F_2$ and $\F_3$. 
% }

For the conductivity reconstruction problem, one of our main objective is to pose a problem that is neither overdetermined nor underdetermined. There are uniqueness results of anisotropic conductivity reconstruction, but they are obtained by using data more than that are required, which in fact is encouraged for better robustness from the practical point of view. There, the existence is rather ignored, or is merely presumed by virtue of that the measurements are from the existing specimen. However, this is hardly true, considering that the electrostatic equation on a body is a model, which is possibly an approximation of larger system that could be nonlinear, and that measurements always accompany noises. Situation has been similar in the theory of electrical impedance tomography; the uniqueness of the conductivity distribution for a given Dirichlet to Neumann map has been well studied (see \cite{nachman_global_1996,sylvester_global_1987}), but any criterion of a map sufficient to have such a conductivity is not known. 

The key is to impose the right number of equations and the right amount of boundary conditions. We take the three equations \eqref{Faraday}. The curl free equation is the Faraday's law for electrostatics. The electrical current density is usually denoted by the letter $\J$ and we will save it for the exact current density. Instead, we use the letter $\F$ for a given vector field which is not necessary a current density field. Also it is not difficult to see that the existence and the uniqueness of the anisotropic resistivity are not obtained from arbitrary vector fields $\F_k$, $k=1,2,3$. In Definition \ref{def:adm3}, the admissibility criteria to be current density fields are given. In addition, we show in Appendix that for a given conductivity there is a relatively simple way to produce such admissible current density fields. This prevents us from having the theory ending in vain, both practically and theoretically. % if there is no such an admissible set of vector fields, or constructing such a one is technically impossible. 

Conductivity reconstruction method using the internal data has been studied by many authors. The magnetic resonance current density imaging (MRCDI) technique is used for the internal measurement of current densities $\F_k$'s (see \cite{Gamba,Joy,Scott}). %The boundary value in \eqref{FaradayBC} is obtained from these vector fields and the resistivity on the boundary. 
The uniqueness of anisotropic conductivity distribution has been obtained in many cases, \cite{bal_inverse_2011, doi:10.1137/140961754, bal_inverse_2014,MR3206987, monard_inverse_2012-1,monard_inverse_2012,doi:10.1080/03605302.2013.787089}. Most of such uniqueness results are based on overdetermined problems. In particular, Monard and Bal \cite{monard_inverse_2012-1} showed the uniqueness of anisotropic conductivity that satisfies $4$ sets of internal power densities, but they also mentioned that they were able to compute anisotropic conductivity numerically only with \emph{three} sets of data. This observation is related to the fact that there are three unknowns $r^{11},r^{12}=r^{21}$ and $r^{22}$ to be recovered. On the other hand, Hoell \emph{et al.} \cite{MR3206987} obtained anisotropic conductivity $\sigma(\x)=c(\x)\sigma_0(\x)$ using a single set of current data when $\sigma_0(\x)$ is a given tensor and the scalar factor $c(\x)$ is unknown. However, the whole conductivity tensor $\sigma(\x)$ is unknown in this paper and two more sets of internal current data are needed to find $\sigma_0(\x)$ in their context.

The remainder of this paper is composed as follows. We briefly review the relations of electrostatics in plane in Section \ref{sec:review}. We discuss the admissibility in Section \ref{sec:adm}. One of the main theory in this paper is to define a hyperbolic problem. This is done in Section \ref{sec:hprob}. Theories are developed in Section \ref{sect.EU}. The existence and the uniqueness are obtained in Theorem \ref{thm:main}. The proof is based on Picard type iteration method that gives the anisotropic resistivity as a fixed point. Conclusions and discussions on related works are given in Section \ref{sect.con}. 

\section{Electrostatics in plane} \label{sec:review}

We first review a few special properties of electrostatics in $\R^2$. This must be elementary in the context of complex analysis. We consider $\Omega\subset \R^2 $ a simply connected bounded open set with a smooth boundary. Any functions appearing in this section is assumed to be smooth in $\Omega$.

If $\mathbf{E}$ is an electric field in $\overline\Omega$ satisfying $\nabla\times \mathbf{E} = 0$ (Faraday's law) in $\Omega$, the potential function $u$ such that
$$ \mathbf{E} = -\nabla u$$
is decided up to an addition of a constant. If $\F$ is an electrical current density field in $\overline\Omega$  satisfying  $\nabla\cdot\F=0$ in $\Omega$, the stream function $\psi$ such that
$$
\F=\nabla^\perp\psi:= \begin{pmatrix} -\partial_y \psi,~ \partial_x \psi \end{pmatrix}
$$
is decided up to an addition of a constant. The stream function can only be considered in two dimensions.

Ohm's law gives a relation between these two vector fields by
\begin{equation}
\F=\Sigma \E \quad \text{or} \quad \r\F=\E, \label{eqn:OhmsLaw0}
\end{equation}
where $\Sigma$ is the conductivity field that takes values in symmetric and positive definite matrices. $\r$ is then its inverse field. $\Sigma$ is assumed to be uniformly bounded, i.e.,
\begin{equation} \label{eigbound}
 \frac{1}{\lambda} I \le \Sigma(x) \le \lambda I
\end{equation}
everywhere for some $0<\lambda <\infty$. The same bound also applies to $\r$.

Considering an electric potential $u$ and then the divergence free equation $\nabla\cdot\F=0$ gives an second order scalar elliptic equation for the potential. If the Neumann boundary condition is imposed, we have
\begin{align}
\nabla\cdot(\Sigma(\nabla u))  &=0, \quad \text{in $\Omega$},\label{ellipticeqn}\\
\Sigma\cdot(\nabla u)\n&=g, \quad \text{on $\partial\Omega$},\label{Neumann}
\end{align}
where the above Neumann boundary condition satisfies $\int_{\partial\Omega}gdx=0$. We can say the second order equation is a manifestation of $\nabla\cdot\F=0$ and $\nabla\times\E=0$, in the connection of the Ohm's law. 

One can consider another way of this manifestation through the stream function $\psi$ which is written as
\begin{equation}\label{curlfreeA}
\nabla\times\big(\r(\nabla^\perp\psi)\big)=0.
\end{equation}
That the underlying space is two dimensional enables us to further rewrite the above equation,
\begin{align*}
-\begin{pmatrix} \partial_x u \\ \partial_y u \end{pmatrix}
=\r \begin{pmatrix} -\partial_y \psi \\ \partial_x \psi \end{pmatrix}
&=\r \begin{pmatrix} 0 & -1 \\ 1 & 0 \end{pmatrix} \begin{pmatrix} \partial_x \psi \\  \partial_y \psi \end{pmatrix}\\
\Longleftrightarrow \quad \begin{pmatrix} -\partial_y u \\ \partial_x u \end{pmatrix} = \begin{pmatrix} 0 & 1 \\ -1 & 0 \end{pmatrix} \begin{pmatrix} -\partial_x u \\ -\partial_y u \end{pmatrix}  &=  \begin{pmatrix} 0 & 1 \\ -1 & 0 \end{pmatrix} \r \begin{pmatrix} 0 & -1 \\ 1 & 0 \end{pmatrix} \begin{pmatrix} \partial_x \psi \\  \partial_y \psi \end{pmatrix}.
\end{align*}
With $S := \begin{pmatrix} 0 & 1 \\ -1 & 0 \end{pmatrix} \r \begin{pmatrix} 0 & -1 \\ 1 & 0 \end{pmatrix}$, there holds $\nabla \cdot \big(S \nabla \psi) = 0$. Note that any one of $\Sigma$, $\r$, $S$ decides all the others and all are valued in symmetric and positive definite matrices. They all share the same bound \eqref{eigbound}. In fact easy calculation shows that $S= \frac{\Sigma}{\det \Sigma}$, or the adjugate matrix of $\r$. 

For the boundary condition, we check that 
$$
g = \Sigma\nabla u \cdot \n= -(-\partial_y \psi, \partial_x \psi) \cdot (n^1,n^2) =(\partial_x \psi,\partial_y \psi) \cdot (-n^2,n^1) = \nabla \psi\cdot T,
$$
where $T:=(-n^2,n^1)$ is the unit tangent vector along the boundary $\partial\Omega$ in the counter-clockwise direction. If $\partial\Omega$ has length $L$ and $\gamma: [0, L] \to \partial\Omega$ is a curve rotating the boundary counter-clockwise with unit speed from $\gamma(0)$, then, 
$$
\int_0^t \frac{d}{d\tau} \psi(\gamma(\tau)) d\tau = \int_0^t g\big(\gamma(\tau)\big) d\tau
=:G(\gamma(t)).
$$
The function $G$ on $\partial\Omega$ is then a Dirichlet boundary condition for $\psi$ that assigns $0$ at $\gamma(0)$. Above calculations are summarized as the dual problem 
\begin{align}
 \nabla\cdot(S\nabla \psi)  &=0,\ \quad \text{in $\Omega$}, \label{eqn:div2}\\
 \psi&=G, \quad \text{on $\partial\Omega$}.                \label{eqn:div2Nbdry}
\end{align}

\section{Admissibility of Data and Inverse Problem} \label{sec:adm}

In this section we define the admissiblity of data. Specifying an admissibility on data reflects the fact that there may be an invisible region of target conductivity map even though the generated data are correctly measured. To put this another way, when $n$ data are to be employed, it is necessary that $n$ data provide independent information everywhere in $\Omega$. This led us to the following admissibility, which at first glance looks restrictive, but are natural properties of $\sigma$-harmonic functions and $\sigma$-harmonic vector fields. 

\begin{Def}[Admissibility] \label{def:adm3} We call a set of three smooth vector fields $\F_k$ on $\overline\Omega$, $k=1,2,3$ satisfying the following four conditions admissible.
\begin{enumerate}
  \item $\nabla\cdot\F_k=0$ for $k=1,2,3$. We let $\psi_k$ the stream function of $\F_k$ such that $\F_k=(-\partial_y\psi_k,\partial_x\psi_k)$, $k=1,2,3$.
   \item The map $\Psi: (x,y) \mapsto \big(\psi_2(x,y), -\psi_1(x,y)\big)$ defines a diffeomorphism between $\overline\Omega$ and its image. A point in $\Psi(\overline\Omega)$ is denoted by $(\xi,\eta)$. %$\Psi(x,y) = \big(\xi(x,y),\eta(x,y)\big):=\big(\psi_2(x,y), -\psi_1(x,y)\big)$.
%    
%    With the first two stream functions, we can define a diffeomorphism , that is 
  \item \label{strict} Let $\tilde{\psi}_3:=\psi_3 \circ \Psi^{-1}$. The scalar curvature
\begin{equation}\label{strictInquality}
\det D^2_{}\tilde{\psi}_3  < 0 \quad \text{on every $(\xi,\eta) \in \Psi(\overline\Omega)$.}
\end{equation}
  \item If $\partial\big( \Psi(\overline\Omega)\big)$ is of length $L$ and $t \mapsto \gamma(t)\in \partial\big( \Psi(\overline\Omega)\big)$ is the arc lenth parametrization of  $\partial\big( \Psi(\overline\Omega)\big)$, then $\big<\gamma'(t), D^2 \tilde\psi_3\big(\gamma(t)\big)\gamma'(t)\big>$ has exactly $4$ simple zeroes in $[0,L)$.
\end{enumerate}
\end{Def}

If $\nabla\cdot\F_k\ne0$, we may take the divergence free part after the Helmholtz decomposition. Hence \ref{def:adm3} (1)  is not a restriction even in the presence of noises. A necessary condition for the \ref{def:adm3} (2) is that $\F_1 \times \F_2 \ne 0$. This sort of condition appears for example in Bauman \emph{et al.} \cite{MR1871388} and cannot be omitted in this inverse problem. Presuming that $\F_1 \times \F_2 \ne 0$ in $\Omega$ and also that the boundary conditions are at our control, the following Lemma justifies that  \ref{def:adm3} (2) also may not be seen as a restriction.

 \begin{Lem}[Meisters and Olech, 1963 \cite{meisters_locally_1963}] \label{lemma:bijection} Let $\y:\overline\Omega\mapsto\R^n$ be differentiable and one-to-one on $\partial\Omega$. If $\det D\y \ne0$ in $\Omega$, $\y$ is one-to-one on $\overline\Omega$.
\end{Lem}

\ref{def:adm3} (3) also is natural. To see the naturality we make an observation on the Ohm's Law under the change of variables.  Let $\Psi$ be a diffeomorphism and let $\xi,\eta$ be a new variables. We set $\tilde{u}:=u\circ\Psi^{-1}$, $\tilde\psi:=\psi\circ\Psi^{-1}$. Then,
\begin{eqnarray*}
  -\begin{pmatrix} \partial_x u & \partial_y u \end{pmatrix} &=&-\begin{pmatrix} \partial_\xi \tilde u & \partial_\eta \tilde u \end{pmatrix} \begin{pmatrix} \partial_x\xi & \partial_y\xi \\ \partial_x\eta & \partial_y\eta\end{pmatrix},\\
  \begin{pmatrix} \partial_x \psi &  \partial_y \psi \end{pmatrix}
  \begin{pmatrix} 0 & 1 \\ -1 & 0 \end{pmatrix}\r
  &=&\begin{pmatrix} \partial_\xi \tilde \psi &  \partial_\eta \tilde \psi \end{pmatrix}
  \begin{pmatrix} \partial_x\xi & \partial_y\xi \\ \partial_x\eta & \partial_y\eta\end{pmatrix}
  \begin{pmatrix} 0 & 1 \\ -1 & 0 \end{pmatrix}\r.
\end{eqnarray*}
Therefore, Ohm's law is written as
$$
  -\begin{pmatrix} \partial_\xi \tilde u & \partial_\eta \tilde u \end{pmatrix}=\begin{pmatrix} \partial_\xi \tilde \psi &  \partial_\eta \tilde \psi \end{pmatrix}
  \begin{pmatrix} \partial_x\xi & \partial_y\xi \\ \partial_x\eta & \partial_y\eta\end{pmatrix}
  \begin{pmatrix} 0 & 1 \\ -1 & 0 \end{pmatrix}\r
  \begin{pmatrix} \partial_x\xi & \partial_y\xi \\ \partial_x\eta & \partial_y\eta\end{pmatrix}^{-1}.
$$
Let
\begin{equation}\label{tilder}
\tilde \r:=\begin{pmatrix} 0 & -1 \\ 1 & 0 \end{pmatrix}\begin{pmatrix} \partial_x\xi & \partial_y\xi \\ \partial_x\eta & \partial_y\eta\end{pmatrix}
  \begin{pmatrix} 0 & 1 \\ -1 & 0 \end{pmatrix}\r
  \begin{pmatrix} \partial_x\xi & \partial_y\xi \\ \partial_x\eta & \partial_y\eta\end{pmatrix}^{-1}.
\end{equation}
We can check $\tilde \r$ is well-defined everywhere since $\begin{pmatrix} \partial_x\xi & \partial_y\xi \\ \partial_x\eta & \partial_y\eta\end{pmatrix}$ is invertible everywhere and also $\tilde \r$ takes values in symmetric positive definite matrices provided $\det D\Psi > 0$. Then we come to the Ohm's Law
\begin{equation}
\begin{pmatrix} -\partial_\eta \tilde\psi & \partial_\xi \tilde\psi \end{pmatrix}\tilde \r=-\begin{pmatrix} \partial_\xi \tilde u &\partial_\eta \tilde u \end{pmatrix} \label{eqn:Ohmsv2}.
\end{equation}
% (Note that any one of $\Sigma$, $\r$, $S$, $\tilde \r$, and $\tilde S$ decides all the others.) 

Now we apply the Ohm's Law in new coordinate system with the choice $(\xi,\eta) = (\psi_2,-\psi_1)$. One of the main technique in this paper is the use of stream functions as independent variables introducing new coordinate system. We simply obtain
\begin{align*}
  \begin{pmatrix} -\partial_\eta \tilde{\psi}_1 & \partial_\xi \tilde{\psi}_1 \end{pmatrix} = \begin{pmatrix} \partial_\eta \eta & -\partial_\xi \eta \end{pmatrix} = \begin{pmatrix} 1 & 0 \end{pmatrix}, \\
  \begin{pmatrix} -\partial_\eta \tilde{\psi}_2 & \partial_\xi \tilde\psi _2 \end{pmatrix} = \begin{pmatrix} -\partial_\eta \xi & \partial_\xi \xi \end{pmatrix} = \begin{pmatrix} 0 & 1 \end{pmatrix},
\end{align*}
and the relation \eqref{eqn:Ohmsv2} is written as
\begin{equation}\label{tilder2}
\tilde \r
=\begin{pmatrix} 1 & 0 \\ 0 & 1 \end{pmatrix}\tilde \r
=-\begin{pmatrix} \partial_\xi \tilde u _1 & \partial_\eta \tilde u _1 \\ \partial_\xi \tilde u _2 & \partial_\eta \tilde u _2\end{pmatrix}.
\end{equation}
Since $\tilde \r$ is symmetric we obtain that 
\begin{equation} \label{eq:firsteq}
\partial_\eta \tilde u _1= \partial_\xi \tilde u _2.
\end{equation}
 This in turn enables us to define a scalar function $\phi$ so  that $\tilde u _1 = \partial_\xi\phi$ and $\tilde u _2=\partial_\eta\phi$. Finally, the resistivity tensor $\tilde\r$ is the negative of Hessian of $\phi$ with respect to the $\xi,\eta$ variables, i.e.,
$$
\tilde\r=-D^2_{\xi,\eta}\phi.
$$


Ohm's law \eqref{eqn:Ohmsv2} for the third current density $\F_3$ is then written as
$$
-\begin{pmatrix} -\partial_\eta\tilde \psi_3 & \partial_\xi\tilde \psi_3 \end{pmatrix} D^2_{\xi,\eta}\phi
= -\begin{pmatrix} \partial_\xi \tilde u _3 & \partial_\eta \tilde u _3 \end{pmatrix}.
$$
The application of the curl operator $\nabla_{(\xi,\eta)} \times$ to both sides gives
% \begin{equation}
%   \partial_\eta^2\tilde\psi _3\partial^2_{\xi}\phi -2\partial_\eta\partial_\xi\tilde\psi _3\partial_\eta\partial_\xi\phi + \partial_\xi^2\tilde\psi _3 \partial^2_{\eta} \phi=0. \label{eqn:anisotropic}
% \end{equation}
\begin{equation}
  \tilde\psi _{3\eta\eta}\phi_{\xi\xi} -2\tilde\psi _{3\xi\eta}\phi_{\xi\eta} + \tilde\psi _{3\xi\xi} \phi_{\eta\eta}=0. \label{eqn:anisotropic}
\end{equation}
Because $\tilde\r = -D^2_{\xi,\eta}\phi$ is positive definite, so is the matrix $-\begin{pmatrix} \phi_{\eta\eta} & -\phi_{\xi\eta}\\ -\phi_{\xi\eta} & \phi_{\xi\xi}\end{pmatrix}$. Therefore, \eqref{eqn:anisotropic}, considering the second derivatives of $\phi$ as coefficients and $\tilde\psi_3$ as unknown, is an elliptic equation. In particular, \eqref{eqn:anisotropic} does not have any low order terms. We quote the following Lemma.

\begin{Lem}[Gilbarg and Trudinger {\cite[p. 256]{gilbarg_elliptic_2001}}] \label{lemma:scalarcurvature}
Let $\begin{pmatrix} a & b \\ b & c \end{pmatrix}$ be uniformly positive on $\Omega$ and $\psi$ satisfies
\begin{equation} \label{eq:phi2nd}
a\partial_x^2\psi + 2b\partial_x\partial_y\psi + c\partial_y^2\psi = 0.
\end{equation}
Then, $\partial_x^2\psi\partial_y^2\psi - (\partial_x\partial_y\psi)^2 \le 0$ and the equality holds only when $\partial_x^2\psi = \partial_y^2\psi = \partial_x\partial_y\psi = 0$.
\end{Lem}
\begin{proof}
The uniform ellipticity gives a constant $\mu_0>0$ that satisfies
  \begin{align*}
    \mu_0((\partial_x^2\psi)^2 + (\partial_x\partial_y\psi)^2) & \le a(\partial_x^2\psi)^2 + 2b\partial_x^2\psi\partial_x\partial_y\psi + c(\partial_x\partial_y\psi)^2 \\
    &= (-2b\partial_x\partial_y\psi-c\partial_y^2\psi)\partial_x^2\psi + 2b\partial_x^2\psi\partial_x\partial_y\psi + c(\partial_x\partial_y\psi)^2 \\
    &= -c(\partial_x^2\psi\partial_y^2\psi-(\partial_x\partial_y\psi)^2).
  \end{align*}
Similarly, we obtain
\begin{align*}
\mu_0((\partial_y^2\psi)^2 + (\partial_x\partial_y\psi)^2) & \le -a(\partial_x^2\psi\partial_y^2\psi-(\partial_x\partial_y\psi)^2).
\end{align*}
Therefore, since the trace of the matrix is $a+c>0$, we have
\begin{align*}
\partial_x^2\psi\partial_y^2\psi-(\partial_x\partial_y\psi)^2 \le -\frac{\mu_0}{a+c}((\partial_x^2\psi)^2 + 2(\partial_x\partial_y\psi)^2 + (\partial_y^2\psi)^2)\le0.
\end{align*}
\end{proof}

In summary, that $\psi_3\circ \Psi^{-1}$ has non-positive scalar curvature is not restrictive. Our admissibility \ref{def:adm3} (3) requires the scalar curvature of $\psi_3\circ \Psi^{-1}$ to be strictly negative in $\overline\Omega$ and thus it will attain the negative maximum in $\overline\Omega$. This certainly is stronger condition. Nevertheless, in Appendix we show that there is a choice of boundary condition for the third data that gives rise to   \ref{def:adm3} (3).

The last condition \ref{def:adm3} (4) is a sufficient condition for the global reconstruction at all points of $\overline\Omega$, and one can recover the resistivity in a possibly proper subset of $\Omega$ without the condition. This needs a technical explanation: In the following section we formulate a system of hyperbolic equation, whose integration gives rise to resistivity. Because the problem is hyperbolic, it is to be solved as an initial boundary value problem, or our two dimensional domain $U=\Psi(\Omega)$ is to be equipped with a metric that tells us which is timelike direction and which is spacelike direction. Accordingly, for the boundary $\partial U$ we have to  decide which portion is the initial spacelike curve, the finial spacelike curve, and boundary timelike curves respectively. Global integrability of a hyperbolic equation thus relies on whether characteristic curves emanated from those initial and boundary curves covers whole domain in a consistent manner. This will be resolved by \ref{def:adm3} (4). As of now, we have not been successful in giving a sufficient condition to have \ref{def:adm3} (4).

\subsection{Inverse problem of Ohm's Law} \label{sec:inv}
In this section, we precisely define the inverse problem we want to solve. We denote the following assumptions by $(A)$.
\begin{enumerate} 
 \item $\Omega \subset \mathbb{R}^2$ is a bounded open set that is simply connected and with smooth boundary.
 \item We are given $\{\F_1, \F_2, \F_3\}$ an admissible data.
\end{enumerate}
In the below $\r$ is a map defined on $\overline\Omega$ that takes values in symmetric matrices, and $u_k$ for $k=1,2,3$ are real-valued functions on $\overline\Omega$. We define our problem $(P)$: Assuming $(A)$, find the four maps $(\r,u_1,u_2,u_3)$ at the same time satisfying the Ohm's Law 
\begin{equation}
\r\F_k = -\nabla u_k \quad \text{in $\overline\Omega$} \quad \text{for $k=1,2,3$}. \label{myohms} 
\end{equation}
We prefer writing the Ohm's Law as in \eqref{myohms} to writing $\F_k = -\Sigma\nabla u_k$ for our specific inverse problem because the right-hand-side of the latter is a product of two unknowns $\Sigma$ and $\nabla u_k$, making the equation nonlinear. Note that \eqref{myohms} is linear.


\section{hyperbolic system definition} \label{sec:hprob}
The purpose of this section is to formulate the inverse problem as a hyperbolic pde problem \eqref{heqn3}-\eqref{hBC} in the below. That a hyperbolic equation appears in a formulation of an inverse problem for elliptic problem has been reported by many authors. For the problems directly related to our problem, the works of Richter \cite{richter_inverse_1981, richter_numerical_1981} seems to be the earliest. Bal \cite{bal_2013} also introduced a formulation where hyperbolic equation appears for ultrasound modulated EIT problem. Lee et al \cite{lee_virtual_2014} and Kim and Lee \cite{lee_well-posedness_2015} studied the isotropic conductivity problem with a formulation similar to that in \cite{richter_inverse_1981, richter_numerical_1981}. A system of hyperbolic equations were employed in Lee et al \cite{lee_orthotropic_2015} to study the orthotropic conductivity problem. In this article, we pursue this idea formulating  a system of hyperbolic equations to study the anisotropic conductivity problem.


If an admissible data is given, then we define the coordinate map $\Psi$ and a simply connected open set $U$ with smooth boundary as 
\begin{align} \label{maps}
 \Psi&=(\xi,\eta):=(\psi_2,-\psi_1), \quad U:=\Psi(\Omega).
\end{align}
and we also define $C^\infty(\bar U)$ functions
\begin{equation} \label{coeff}
 \tilde\psi _3:= \psi_3\circ\Psi^{-1}, \quad a := \tilde\psi _ {3\eta\eta}, \quad b :=- \tilde\psi _ {3\xi\eta}, \quad c := \tilde\psi _ {3\xi\xi}.
\end{equation}
We consider an equation 
\begin{equation} \label{heqn}
 a\phi_{\xi\xi} + 2b\phi_{\xi\eta} + c \phi_{\eta\eta} = 0 \quad \text{in $U$}.
\end{equation}
The role has been interchanged: $\phi$ is regarded as unknown while terms out of $\tilde\psi _3$ are regarded as given coefficients. From the negative curvature condition \ref{def:adm3} (3), the equation is second order hyperbolic. %If $\tilde u _1=\partial_\xi \phi$ and $\tilde u _2=\partial_\eta \phi$ are to be unknowns instead of $\phi$, the first order system consists of  
% \begin{equation} \label{heqn2}
% \begin{aligned}
% \partial_\eta \tilde u_1 - \partial_\xi \tilde u_2 &=0,\\
%  a \partial_\xi\tilde u_1 + b (\partial_\eta \tilde u_1 +\partial_\xi \tilde u_2) + c\partial_\eta \tilde u_2 &= 0.
%  \end{aligned}
% \end{equation}

% The problem definition certainly is data dependent, and this in particular includes the following considerations. for the boundary $\partial U$ we have to  decide which portion is the initial spacelike curve, the finial spacelike curve, and boundary timelike curves respectively, and this structure is dependent on $\F_k$ $k=1,2,3$.
% In the below, we specified the hyperbolic problem definition. To retain the Jacobian matrix in the form $\begin{pmatrix} \partial_x\xi & \partial_y\xi \\ \partial_x\eta & \partial_y\eta\end{pmatrix}$, we write the gradient vector as a row vector onward.

\subsection{Characteristic vector fields}\label{sect.char}

In this section we first show that the admissibility gives that there are nowhere vanishing characteristic vector fields for the equation \eqref{heqn}. This allows us to integrate the equation \eqref{heqn} on the whole domain and to impose consistent initial boundary conditions. 

For the notational convenience, for any function $h$ on $\overline\Omega$, we abuse to write $h\circ \Psi^{-1}$ as $h$ onwards. A generic point in $\U$ will be denoted by $z$.

We say a vector $\N=(f,g)$ is a characteristic vector at $z\in \U$ to \eqref{heqn} if at $z$
$$a g^2 - 2b fg + c f^2 = 0 \quad \left( \Longleftrightarrow \quad \left<\N(z), D^2\psi_3(z) \N(z)\right> = 0\right).$$
By admissibility condition \ref{def:adm3} (3), \eqref{heqn} admits the two linearly independent characteristic vectors everywhere in $\overline{U}$. To have a smooth choice of characteristic vectors over the domain, we quote a result for a two dimensional Lorentzian manifold in the following. If we equip $\overline U$ with the metric $\mathscr{G}:=D^2 \psi_3$, \ref{def:adm3} (3) implies that one of the two eigenvalues is positive and the other is negative. Therefore, $(\U,\mathscr{G})$ is a simply connected Lorentzian manifold. The characteristic vector is called the {\it null vector} in this context.% from the view point of Lorentzian manifold.
\begin{Lem}\cite[Proposition 3.37]{MR1384756}\label{lemma:GHL}
Let $(M,\mathscr{G})$ be a simply connected Lorentzian manifold of dimension two. Then, there exist two linearly independent smooth null vector fields $\N_1$ and $\N_2$ defined on $M$.
\end{Lem}
We thus have the following proposition:
\begin{Prop}\label{prop:separability} Suppose $(\F_1,\F_2,\F_3)$ is admissible and $\Psi$, $U$, $a$, $b$, and $c$ are defined as in \eqref{maps} and \eqref{coeff}. Then there exist two smooth linearly independent vector fields $\N_1$ and $\N_2$ on $\overline U$ which are characteristic everywhere for the equation \eqref{heqn}. We may assume $\N_1 \times \N_2 > 0$ in $\overline U$.
\end{Prop}

%   % and we have two linearly independent smooth null vector fields denoted by $\N_1$ and $\N_2$. 
% By definition,  The two null vectors will be given by the formula,
% \begin{equation} \label{eqn:charformula}
%     \left<\N_j, D^2\psi_3\N_j\right> = 0.\qquad j=1,2.
% \end{equation}
% We introduce a Lemma that works on two dimensional domain.
% 
% 
% 
% The two vector fields $\N_j$, $j=1,2$, are called null vector fields  metric $D^2\psi_3$. From the hyperbolic wave equation view point of \eqref{eqn:anisotropic}, they are called \emph{characteristic} vector fields.
% Therefore, the next Proposition immediately follows the lemma.

\subsection{Definition of $\Gamma^-_1$ and $\Gamma^-_2$}

% \subsection{Data dependent definition of boundary and boundary data}
Given a hyperbolic equation defined on a general shaped {\it space-time} domain, the integrability of the equation depends on the characteristic curves emanated from the space-time boundary and the data assigned on them. One feasible case will be the one where we can specify an initial spacelike curve and boundary timelike curves so that characteristics emanated from them cover the whole domain. The possibility of doing this is of topological in nature and this is where we make use of the \ref{def:adm3} (4). In the followings, $\n$ denotes the outward unit normal along $\partial U$.

\begin{Prop} \label{lemma:anisogeom}
Suppose $(\F_1,\F_2,\F_3)$ is admissbile, and $\Psi$, $U$, $a$, $b$, and $c$ are defined as in \eqref{maps} and \eqref{coeff}. Let $\N_1$ and $\N_2$ be the vector fields defined in Proposition \ref{prop:separability}. 
Then $P:=\left\{ z\in \partial U ~|~ \N_1(z)\cdot \n(z) = 0\right\}$ has two elements, $Q:=\left\{ z\in \partial U ~|~ \N_2(z)\cdot \n(z) = 0\right\}$ has two elements, and their union contains exactly four distinct elements. If $t \mapsto \gamma(t) \in \partial U$ is an embedding for $\partial U$, then elements of $P$ and elements of $Q$ appear alternatively as $t$ increases.
\end{Prop}
\begin{proof}
 Because $U$ is simply connected, $\partial U$ is homeomorphic to a circle. Let $\partial U$ has the length $L$, and let $\gamma : [0,L] \mapsto \partial U$ be an arc length parametrization of $\partial U$ rotating counter-clock-wisely.  Over this closed path parametrized by $\gamma$,  we consider angle differential 
 $$\theta_T'(t) = \frac{\gamma(t) \times \gamma(t)'}{|\gamma(t)|^2},$$
 and those of vector fields $\N_1$, $\N_2$
\begin{align*}
  \theta_1'(t) = \frac{\N_1\big(\gamma(t)\big) \times \N_1\big(\gamma(t)\big)'}{|\N_1\big(\gamma(t)\big)|^2}, \quad  \theta_2'(t) = \frac{\N_2\big(\gamma(t)\big) \times \N_2\big(\gamma(t)\big)'}{|\N_2\big(\gamma(t)\big)|^2}.
 \end{align*}
If $\arg(\mathbf{V})$ is the polar angle of a vector $\mathbf{V}$, ranging in $[0,2\pi)$, we define initial angles 
$$\theta_T(0) = \arg(\gamma(0)), \quad \theta_1(0) = \arg\Big(\N_1\big(\gamma(0)\big)\Big),\quad \theta_2(0) = \arg\Big(\N_2\big(\gamma(0)\big)\Big).$$
Without loss, we may assume $\theta_T(0) = 0$, and also $\theta_1(0) \in [0,\pi)$ by replacing $(\N_1,\N_2)$ by $(-\N_1,-\N_2)$ if necessary. It is immediate that $\theta_T(L) = 2\pi$ and from that $\N_1$ and $\N_2$ are nowhere vanishing we also have $\theta_1(0)=\theta_1(L)$ and $\theta_2(0)=\theta_2(L)$.

Now, because $\N_1 \times \N_2 > 0$ in $\overline U$, i.e., $(\N_1,\N_2)$ is positively oriented, $\theta_1(0) < \theta_2(0) < \theta_1(0) + \pi$. We claim this strict inequality must hold for any $t\in[0,L]$, i.e.,
\begin{equation} \label{strinq}
\theta_1(t) < \theta_2(t) < \theta_1(t) + \pi, \quad \forall t\in[0,L]. 
\end{equation}
Suppose on the contrary that there is a point $t'$ where the first or the second inequality does not hold. If $\theta_1(t') \ge \theta_2(t')$, then by intermediate value theorem, there is $t''$ where $\theta_1(t'') = \theta_2(t'')$, which contradicts to that $\N_1\times \N_2>0$ in $\overline U$. Similarly the second inequality must hold. 

Let 
$$g_1(t) = \theta_1(t) - \theta_T(t), \quad g_2(t) = \theta_1(t) + \pi - \theta_T(t).$$ We have 
$$g_1(0)\ge0, \quad g_1(L)<0, \quad g_2(t)>0, \quad g_2(0)>0, \quad g_2(L)<0.$$
Therefore by intermediate value theorem, there exist $a,b \in[0,L]$ such that 
$$\theta_1(a) = \theta_T(a), \quad \theta_1(b) = -\pi + \theta_T(b)$$
and the two points are distinct. Similarly, we let 
$$g_3(t) = \theta_2(t) - \theta_T(t), \quad g_4(t) = \left\{\begin{aligned} \theta_2(t)+\pi - \theta_T(t), \quad &\text{if $\theta_2(0) \in [0,\pi)$,}\\ \theta_2(t)-\pi - \theta_T(t), \quad &\text{if $\theta_2(0) \in [\pi,2\pi)$.} \end{aligned}\right.$$
We have
$$g_3(0)>0, \quad g_3(L)<0, \quad g_4(0)\ge0, \quad g_4(L)<0.$$
Therefore by intermediate value theorem, there exist $c, d\in[0,L]$ such that 
$$\theta_2(c) = \theta_T(c), \quad \theta_2(d) = \left\{\begin{aligned} - \pi+ \theta_T(d), \quad &\text{if $\theta_2(0) \in [0,\pi)$,}\\ +\pi + \theta_T(d), \quad &\text{if $\theta_2(0) \in [\pi,2\pi)$,} \end{aligned}\right.$$
Because $\theta_1(t)\ne \theta_2(t)$ for all $t\in [0,L]$, the points $a$, $b$, $c$, and $d$ are all distinct. From the \ref{def:adm3} (4), there are exactly four points on $\partial U$ where the tangent vector is characteristic, and thus $a$, $b$, $c$, and $d$ are uniquely defined.

Finally, using \eqref{strinq} we have 
$$g_3(a)>0, \quad g_3(b)<0$$
and thus $c$ is between $a$ and $b$. Also
\begin{align*}
 &g_2(c)>0, \quad g_2(d)<0 \quad \text{if $\theta_2(0) \in [0,\pi)$,}\\
 &g_1(c)<0, \quad g_1(d)>0 \quad \text{if $\theta_2(0) \in [\pi,2\pi)$}
\end{align*}
and thus $a$ is between $c$ and $d$ for the former case, and $b$ is between $c$ and $d$ for the latter case.
\end{proof}

\begin{Rem} \label{defs}
$\partial U \setminus P$ has two connected components, and $\partial U\setminus Q$ has two connected components. They are
\begin{equation}\label{Gammas}
\begin{aligned}
\Gamma_1^+&:=\{z\in\partial U\,|\, \N_1(z)\cdot\n(z)>0\}, \quad \Gamma_1^-:=\{z\in\partial U\,|\, \N_1(z)\cdot\n(z)<0\},\\
\Gamma_2^+&:=\{z\in\partial U\,|\, \N_2(z)\cdot\n(z)>0\}, \quad \Gamma_2^-:=\{z\in\partial U\,|\, \N_2(z)\cdot\n(z)<0\}. 
\end{aligned}
\end{equation}
By the last assertion of Proposition \ref{lemma:anisogeom}, $\Gamma_1^- \cap \Gamma_2^-$ is nonempty and connected. One end point of $\Gamma_1^- \cap \Gamma_2^-$ is in $P$, and the other end point is in $Q$, and we denote the points by $p_0$ and $q_0$ respectively. The remainder in $P$ and $Q$ are then denoted by $p_1$ and $q_1$ respectively. From onwards we assume without loss of generality that the embedding $\gamma :[0,L] \rightarrow \partial U$ is such that 
$\gamma$ is counter-clock-wise and $(q_0,p_0,q_1,p_1)$ is in the order of appearance in this embedding. See Figure \ref{fig1}.
\end{Rem}
\begin{figure}[ht]
  \center
  \psfrag{G2}{\hskip -5pt$\Gamma_2^-$}
  \psfrag{G1}{\hskip -5pt$\Gamma_1^-$}
  \psfrag{G3}{$\Gamma_1^- \cap \Gamma_2^-$}
  \psfrag{p0}{$p_0$}\psfrag{p1}{$p_1$}\psfrag{q0}{$q_0$}\psfrag{q1}{$q_1$}
  \includegraphics[width=6cm]{Gamma.eps}
  \caption{Configuration of $\Gamma^-_j$ $j=1,2$ on $\partial U$.} \label{fig1}
\end{figure}

\subsection{Hyperbolic system definition} 
In this section, we finally pose a first order $2\times 2$ hyperbolic system. We consider a first order formulation of \eqref{heqn}: We let $u _1=\partial_\xi \phi$ and $u_2=\partial_\eta \phi$.
% , and consider 
% \begin{equation} \label{heqn2}
% \begin{aligned}
% \partial_\eta u_1 - \partial_\xi  u_2 &=0,\\
%  a \partial_\xi u_1 + b (\partial_\eta  u_1 +\partial_\xi  u_2) + c\partial_\eta  u_2 &= 0.
%  \end{aligned}
% \end{equation}
To make use of the characteristic vector fields, we consider a linear transformation
\begin{align} \label{formula}
 \begin{pmatrix} v_1 \\ v_2 \end{pmatrix} = \begin{pmatrix} \bar f & \bar g \\ f & g \end{pmatrix} \begin{pmatrix} u_1 \\ u_2 \end{pmatrix}, \quad \text{where $(f,g)=\N_1$ and  $(\bar f,\bar g)=\N_2$.}
\end{align}
Note that $\N_1\times\N_2>0$ gives the bijectivity of the transformation. 
Then the directional derivatives
\begin{equation} \label{pdev}
\begin{aligned}
   \N_1\cdot\nabla v_1 &= \N_1\cdot (u_1\nabla \bar{f}  + u_2\nabla \bar{g} ) + \N_1 \cdot (\bar{f}\nabla u_1  + \bar{g}\nabla u_2 ), \\
   \N_2\cdot\nabla v_2 &= \N_2\cdot (u_1\nabla {f}  + u_2\nabla {g} ) + \N_2 \cdot ({f}\nabla u_1  + {g}\nabla u_2 ).
\end{aligned}
\end{equation}
% the term $\N_1\cdot (\nabla \bar{f} u_1 + \nabla \bar{g} u_2)$ and $\N_2\cdot (\nabla {f} u_1 + \nabla {g} u_2)$ can be calculated from the data and \eqref{formula}. Thanks to the transformation we have that $\N_1 \cdot (\nabla u_1 \bar{f} + \nabla u_2 \bar{g}) = \N_2 \cdot (\nabla u_1 {f} + \nabla u_2 {g})= 0$ which is calculated in the below.
\begin{Lem} \label{lem:zero} Suppose $\phi$ is a smooth solution of \eqref{heqn}, $u_1 = \phi_\xi$, and $u_2=\phi_\eta$. Suppose $\N_1=({f},{g})$ and $\N_2=(\bar{f},\bar{g})$ are linearly independent and characteristic at each point of $\U$. Then $\N_1 \cdot (\bar{f}\nabla u_1  + \bar{g}\nabla u_2 ) = \N_2 \cdot ({f}\nabla u_1  + {g}\nabla u_2 )= 0.$
\end{Lem}
\begin{proof}
Using that $u_1 = \phi_\xi$ and $u_2=\phi_\eta$,
 $$\N_1 \cdot (\bar{f}\nabla u_1  + \bar{g}\nabla u_2 ) = \N_2 \cdot ( {f}\nabla u_1 + {g}\nabla u_2 )=  {f}\bar{f} \phi_{\xi\xi} + ({f}\bar{g} + {g} \bar{f}) \phi_{\xi\eta} + {g}\bar{g} \phi_{\eta\eta}.$$
 
 By definition,  $({f},{g})$ and $(\bar{f},\bar{g})$ are the two distinct solutions (unique up to multiplication of nonzero constant) of the following quadratic equation on $({f}_*,{g}_*)$
 \begin{equation} \label{quadratic}
 c{f}_*^2  - 2b{f}_*{g}_* + a{g}_*^2   = 0.
 \end{equation}
 In case $c\ne0$, we solve the quadratic equation for ${f}_*$. This gives that 
 $${f}\bar{f} =  \frac{a{g}\bar{g}}{c}, \quad  {f}\bar{g} + {g} \bar{f} =  \frac{2b{g}\bar{g}}{c}.$$
  In case $a\ne0$, we solve the quadratic equation for ${g}_*$. This gives that 
 $${g}\bar{g} =  \frac{c{f}\bar{f}}{a}, \quad  {f}\bar{g} + {g} \bar{f} =  \frac{2b{f}\bar{f}}{a}.$$
 In case $a=c=0$, then we set $({f},{g})=(1,0)$, $(\bar{f},\bar{g}) = (0,1)$. 
 For all three cases, ${f}\bar{f} \phi_{\xi\xi} + ({f}\bar{g} + {g} \bar{f}) \phi_{\xi\eta} + {g}\bar{g} \phi_{\eta\eta}$ is a constant multiple of $a\phi_{\xi\xi} + 2b\phi_{\xi\eta} + c\phi_{\eta\eta}$ that is identically $0$.
\end{proof}
In view of Lemma \ref{lem:zero} and the transformation \eqref{formula}, if $\phi$ solves \eqref{heqn} the right-hand-sides of \eqref{pdev} are combinations of $u_1$ and $u_2$, equivalently combinations of $v_1$ and $v_2$. We divide equations in \eqref{pdev} by $|\N_1|$ and $|\N_2|$ respectively to have
\begin{equation} \label{heqn3} \frac{\N_1}{|\N_1|}\cdot\nabla v_1 = pv_1 + qv_2, \quad \frac{\N_2}{|\N_2|}\cdot\nabla v_2 = rv_1 + sv_2 \quad \text{in $U$},
\end{equation}
where $p$, $q$, $r$, and $s$ are the coefficients. They are given by $\N_j$ and derivatives of $\N_j$, $j=1,2$. $\N_j$, $j=1,2$ are computed by the quadratic equation \eqref{quadratic}, where $a$, $b$, and $c$ are decided by data $\F_k$ $k=1,2,3$. 

In the next section, we study the problem $(H)$ consists of \eqref{heqn3} and the following boundary condition 
\begin{equation} \label{hBC}
v_1 = v_{10} \quad \text{on $\Gamma_1^-$,} \quad v_2 = v_{20} \quad \text{on $\Gamma_2^-$}.
\end{equation}
Assigning the values of $v_1$ (resp. $v_2$) on $\Gamma^-_1$ (resp. $\Gamma^-_2$) is to measure a combination $\N_2\cdot(u_1,u_2)=\bar{{f}}u_1 + \bar{g} u_2$ (resp. $\N_1\cdot(u_1,u_2)={{f}}u_1 + {g} u_2$) of two potentials $u_1$ and $u_2$. %If more information is included, then the problem is overdetermined.
% 
% \begin{equation} \label{hprob}\tag{H}
% \begin{aligned}
%  \frac{\N_1}{|\N_1|}\cdot\nabla v_1 = pv_1 + qv_2, \quad \frac{\N_2}{|\N_2|}\cdot\nabla v_2 &= rv_1 + sv_2 \quad \text{in $U$}, \\
% \end{aligned}
% \end{equation}
% where $p,q,r,s$ are decided by the data in the below and $v_{10}$ and $v_{20}$ are to be assigned suitably.
% 
% We finally remark on the boundary measurements $v_{10}: = \N_2 \cdot (u_1,u_2)$ on $\Gamma^-_1$ and $v_{20}: = \N_1 \cdot (u_1,u_2)$ on $\Gamma^-_2$. They are linear combinations of potentials $u_1$ and $u_2$ precisely at $\Gamma_1^-$ and $\Gamma_2^-$ respectively. 

\section{Existence and uniqueness}\label{sect.EU}

% The purpose of this section is to show the existence and uniqueness of anisotropic resistivity distribution $\r$ that satisfies \eqref{Faraday}--\eqref{FaradayBC}.

The purpose of this section is to prove the global integrability of \eqref{heqn3}-\eqref{hBC} which in turn gives rise to the uniqueness and existence of an anisotropic conductivity  and potentials that satisfies \eqref{myohms}. Before we proceed further, we summarize how \eqref{heqn3}-\eqref{hBC} was formulated from the admissible data, and also outline how the rest of this section will proceed.

The data $(\F_1,\F_2,\F_3)$ is assumed, which satisfies the admissibility \ref{def:adm3} (1)-(4). 
\begin{enumerate}
 \item[1.] $\F_1$ and $\F_2$ gives rise to the coordinate map $\Psi$, and the domain $U$ of our problem, as in \eqref{maps}.
 \item[2.] Out of $\F_3$, the Hessian $D^2\psi_3$ in a new coordinate system is computed, whose determinant is negative everywhere in $\U$.
 \item[3.] That $\U$ is simply connected subset of $\mathbb{R}^2$ and that $\det D^2\psi_3 < 0$ everywhere give rise to the computation of the vector fields $\N_1$ and $\N_2$. Accordingly, the boundary portions $\Gamma^\pm_j$ are defined.
 \item[4.] The hyperbolic problem consists of a system of equation \eqref{heqn3} and boundary condition \eqref{hBC} is formulated.
 \item[5.] In Section \ref{causal}, we give a {\it causal structure} on $\U$, revealing the topological properties of $1$-characteristic family of curves and $2$-characteristic family of curves. It turns out that $\overline{\Gamma^-_1 \cap \Gamma^-_2}$  (See Figure \ref{fig3}) is our initial spacelike curve, and $\overline{\Gamma^-_1} \setminus \Gamma^-_2$ and $\overline{\Gamma^-_2} \setminus \Gamma^-_1$ are the two boundary timelike curves. \eqref{heqn3}-\eqref{hBC} is thus understood as an initial boundary value problem, and the integration is to be done up to the final spacelike curve $\overline{\Gamma^+_1 \cap \Gamma^+_2}$. 
 \item[6.] In Appendix \ref{local}, local solvabilities of the {\it Goursat problem} and the {\it Triangle problem} are introduced. 
 \item[7.] In Proposition \ref{prop:final}, the integration procedure explained in step 5. is achieved to obtain $(v_1,v_2)$  in the wholse set $\U$. 
 \item[8.] $(u_1,u_2)$ is recovered from $(v_1,v_2)$ in $\U$ by \eqref{formula}. Two potentials then give rise to the computation of $\tilde {r}$ in \eqref{tilder2}.
 \item[9.] $r$ in $\overline{\Omega}$ is recovered by \eqref{tilder}.
\end{enumerate}





\subsection{Characteristic families of curves and causal structure on $\U$} \label{causal}

Solving the linear hyperbolic system \eqref{heqn3}-\eqref{hBC} is of classical theory, and it can be done by a so-called the Hodograph transformation, constructing a coordinate map $(\nu_1,\nu_2)$ whose level lines are the integral curves of two characteristic vector fields. Constructing a local coordinate map is immediate by inverse function theorem, but contructing a global one is quite elaborate. Instead of relying on the existence of global coordinate map, we here work with two families of characteristic curves. 

In the following Lemma, the simplicity of topological properties of integral curves by $\N_1$ and $\N_2$ are presented. The family of integral curves of $\N_1$ is called the $1$-characteristic family, or $1$-family for brevity, and that of $\N_2$ is called the $2$-characteristic family. We estabilish that curves in $1$-family are layered, not intersecting each other, and the same is true for $2$-family. A curve from $1$-family and another curve from $2$-family must intersect, transversally, only once. That $\N_1$ and $\N_2$ are nowhere vanishing vector fields, and that $\N_1\times \N_2$ is nowhere vanishing too give the results.

The main tool is a phase space analysis for a planar dynamical system; the results follows mainly from the Poincar\'e-Bendixson Theorem. For notions in the area of dynamical systems, we refer a standard textbook such as Perko \cite{perko_differential_2001}.  We introduce a few notations: For a given smooth vector field $\N$ and a point $\bar z$, we consider an integral curve of $\N$ passing $\bar z$ that is a solution of orinary differential equations $z'(s) = \N(z(s))$, $z(0)=\bar z$. The forward trajectory by $\N$ from $\bar{z}$ is the portion where $s\ge0$, and the backward trajectory is the one for $s\le0$. With these, $C_1(\bar z)$ denotes the maximally defined closed integral curve passing $\bar z$ by $\N_1$ flow. $C_2(\bar z)$ denotes the similar by $\N_2$ flow. $C^+_1(\bar z)$ (resp. $C^-_1(\bar z)$) denotes the maximally defined closed forward (resp. backward) trajectory by $\N_1$ out of $\bar z$. $C^\pm_2(\bar z)$ denote the similars by $\N_2$. 

\begin{Prop} \label{lemma:2familychar}
Suppose $(\F_1,\F_2,\F_3)$ is admissbile, and $\Psi$, $U$, $a$, $b$, and $c$ are defined as in \eqref{maps} and \eqref{coeff}. Let $\N_1$ and $\N_2$ be the vector fields defined in Proposition \ref{prop:separability}. Let $\Gamma^\pm_1$, $\Gamma^\pm_2$, $P=\{p_0,p_1\}$, and $Q=\{q_0,q_1\}$ be the sets defined in Proposition \ref{lemma:anisogeom} and in Remark \ref{defs}. Then followings hold.
\begin{enumerate}
 \item For every $z \in {\overline{U}\setminus P}$, there exists a unique integral curve of $\N_1$ passing $z$ and ending up forwardly on some $\alpha^+(z)\in\Gamma_1^+$ and backwardly on some $\alpha^-(z)\in\Gamma_1^-$. For every $z\in {\overline{U}\setminus Q}$, there exists a unique integral curve of $\N_2$ passing $z$ and ending up forwardly on some $\beta^+(z)\in\Gamma_2^+$ and backwardly on some $\beta^-(z)\in\Gamma_2^-$. Every integral curve is a homeomorphic image of a finite line segment.
 \item Conversely, if $\alpha^-\in \Gamma^-_1$ and $\beta^-\in \Gamma^-_2$ then $C^+_1(\alpha^-)$ and $C^+_2(\beta^-)$ intersects exactly at one point in $\overline{U}\setminus (P \cup Q)$ transversally.
\end{enumerate}
\end{Prop}
\begin{proof}
 Arguments in \cite[Lemma 3,4]{lee_well-posedness_2015} can be used to show assertion (1) but for completeness, we present its proof here.
 
 We first make a simple observation that for the flow defined by $\N_1$, there cannot be  a critical point, or a periodic orbit, inside of which a critical point is assumed, or a separatrix cycle, in which critical points are already assumed. This in turn by Poincar\'e-Bendixson Theorem, leads to the fact that no nonempty subset $A$ of $\overline{U}$ is positively invariant or negatively invariant. If there is, then its $\omega$-limit set or $\alpha$-limit set is nonempy but by Poincar\'e-Bendixson Theorem, the set must be either a critical point, or a periodic orbit, or a separatrix cycle, contradicting to the above observation.
 
 As a consequence, for a given $z\in \overline{U}\setminus P$ its forward trajectory by $\N_1$ must escape $\overline{U}$ in finite time through $\alpha^+(z)\in\partial U$. By the same reason, its backward trajectory escapes $\overline{U}$ at finite time through $\alpha^-(z)\in\partial U$. %We write in the below $\alpha^\pm = \alpha^\pm(z)$ suppressing the dependency on $z$.
%  it impossible to have any subset of $\overline{U}$ that is either positively invariant or negatively invariant. 
%  
%  there is no critical point, no periodic orbit, 
%  For given $(\xi,\eta)\in U$ we consider a trajectory by the vector field $\N_1$ passing $(\xi,\eta)$ at initial time. Suppose its forward flow does not escape $\overline{U}$ at finite time. Then its forward trajectory is contained in a compact set $\overline{U}$ and its omega limit set $\omega$ is nonempty. By Poincar\'e-Bendixson theorem, $\omega$ is a critical point, or a periodic orbit, or a separatrix cycle, all of which assume the existence of a critical point of $\N_1$. Since $\N_1$ is nowhere vanishing this is impossible, and its forward flow must escape $\overline{U}$ at finite time through 

Now, we claim that $\alpha^+(z) \in \Gamma_1^+$ and $\alpha^-(z) \in \Gamma_1^-$. $\alpha^+(z)\ne \alpha^-(z)$ because the trajectory cannot be a cycle. Consider two connected curves along $\partial U$ joining $\alpha^-(z)$ and $\alpha^+(z) $. Suppose on the contrary that  $\alpha^+(z) $ or $\alpha^-(z)$ is in the set $P$. Since there are only two elements in $P$, one of the connected curves along $\partial U$ does not intersect $P$ in its relative interior. If $K$ is the compact set enclosed by this portion of boundary and $C_1(z)$, then $K$ is either positively invariant or negatively invariant by $\N_1$, which contradicts to the observation. We do the same argument for $\N_2$ and assertion (1) follows.

% Now, for any $(\xi,\eta) \in U$, there exists the unique integral curve of $\N_1$ passing $(\xi,\eta)$ emanated from $z^- \in \Gamma_1^-$, and there exists the unique integral curve of $\N_2$ passing $(\xi,\eta)$ emanated from $w^- \in \Gamma_2^-$. 

Now, we prove assertion (2). Let $\alpha^-\in \Gamma^-_1$ and $\beta^-\in \Gamma^-_2$. Since at $\alpha^-$,  $\N_1\cdot\n < 0$ where $\n$ is outward normal, $C^+_1(\alpha^-)$ continues into $int(\overline{U})$, and by assertion (1) it ends up at some $\alpha^+\in \Gamma^+_1$. We denote this trajectory by $D$. By the embedding $\gamma: [0,L] \rightarrow \partial U$ as in the proof of Proposition \ref{lemma:anisogeom}, let $(t_1,t_2)$ be the inverse image of $\Gamma^-_2$. At each point on $D \cap int(\U)$, by assertion (1) for some $\tilde{\beta}^-\in \Gamma^-_2$, $C^+_2(\tilde \beta^-)$ passes the point. Therefore the set $I \subset (t_1,t_2)$ of point $t$ such that out of $\gamma(t)$, $C^+_2(\gamma(t))$ intersects $D$ is nonempty. We show that $I$ is open and closed relatively to $(t_1,t_2)$. If $t \in I$ and $z_*$ is an intersection point of $D$ and $C^+_2(\gamma(t))$, the intersection at $z_*$ is transversal  because $\N_1 \times \N_2 \ne 0$. Because the finite length trajectory $C^+_2(\gamma(t))$ smoothly changes as $t$ changes, and the transversal intersection is stable under perturbation, intersection persists for some neighborhood of $t$. On the other hand, if $t' \in (t_1,t_2) \setminus I$, $C^+_2(\gamma(t'))$ and $D$ are two disjoint closed sets, and the distance between two is a finite positive number. Therefore, for some neighborhood of $t'$ intersection does not occur. Thus $I$ is closed relatively to $I$, and $I = (t_1,t_2)$. Intersection must uccur for every $\beta^-\in \Gamma^-_2$.

Now suppose $D$ and $E:=C^+_2(\beta^-)$ intersects more than once. $\overline{U}$ is divided into two sides by the curve $E$. Pick one side and let $\nu$ be the inward normal to the side along $E$. Since $\N_2$ is tangent on $E$ and $\N_1\times \N_2>0$, $\N_1\cdot \nu$ has definite sign on $E$. We assume $\N_1\cdot \nu>0$ without loss. We let $p$ be an intersection point such that another intersection point appears at later time by $\N_1$ flow. Since $C^+_1(p)$ is continued into interior of the side from $p$, if $q$ is the intersection point appears for the first time after $p$, then the portion $F$ of $C^+_1(p)$ between $p$ and $q$ is entirely in the side. The curves $E$ and $F$ then enclose a simply connected set $K$, and $K$ is positively invariant by $\N_1$, which is a contradiction. 

Finally, the trajectory $D$ has been continued into $int(\overline{U})$, and by assertion (1) each of its two end points cannot be in $P$, and thus $D$ is contained in $\overline{U}\setminus P$. $E$ is contained in $\overline{U}\setminus Q$ for the similar reason. Therefore the unique intersection point is in $\overline{U}\setminus (P \cup Q)$.
\end{proof}


\subsubsection{Causal structure on $\U$}

 At each point $z\in \U$, the {\it causal structure} can be introduced on the tangent space $T_z$. Two subspaces $\textsf{Span}(\N_1)$ and $\textsf{Span}(\N_2)$ divide $T_z$ into four open sectors. Vectors in $\textsf{Span}(\N_1) \cup \textsf{Span}(\N_2)$ are said to be {\it null}. Because $\N_1 \times \N_2 >0$, rotating from $\N_1$ counter-clock-wisely, the open sectors are the ones consisting of {\it future-pointing timelike, negative-pointing spacelike, past-pointing timelike, and positive-pointing spacelike} tangent vectors. See Figure \ref{Tz}.
 
\begin{figure}[ht]
 \centering
%  \begin{subfigure}[t]{2in}
 \centering
 \psfrag{z}{${z}$}
 \psfrag{F}{{Future}}
 \psfrag{P2}{Past}
 \psfrag{P1}{Positive}
 \psfrag{N}{Negative}
 \psfrag{N1}{$\N_1$}
 \psfrag{N2}{$\N_2$}
 \includegraphics[width=4cm]{cone2.eps} 
  \caption{Causal structure on $T_z$ is illustrated. The closure of the open positive sector is the nonnegative non-timelike sector $K_z$.} \label{Tz}
%  \end{subfigure} \quad \quad \quad
\end{figure}

 
 Regarding the boundary portions $\Gamma^\pm_j$ $j=1,2$, we will have to consider a closed curve nowhere characteristic with exceptions at the end points. Precisely, a $C^1$ closed curve $G$ in $\U$ is said to be timelike in its relative interior if there is an embedding $\kappa$ of $G$ from $[s_0,s_1]$ with unit speed such that at every point $s\in(s_0,s_1)$ $\kappa'(s)$ is future-pointing timelike. A $C^1$ closed curve $H$ in $\U$ is said to be spacelike in its relative interior if there is an embedding $\kappa$ of $H$ from $[s_0,s_1]$ with unit speed such that at every point $s\in(s_0,s_1)$ $\kappa'(s)$ is positive-pointing spacelike. It is not difficult to see that $E_*=\overline{\Gamma_1^-\cap \Gamma_2^-}$ and $E^*=\overline{\Gamma_1^+\cap \Gamma_2^+}$ are two examples of curves spacelike in relative interior, and $\overline{\Gamma_1^-} \setminus \Gamma_2^-$ and $\overline{\Gamma_2^-} \setminus \Gamma_1^-$ are two curves timelike in relative interior.
 
 \begin{figure}[ht]
%  \begin{subfigure}[t]{2in}
 \centering
 \psfrag{G1}{\hskip -2em $\overline{\Gamma_1^- \cap \Gamma_2^-}$}
 \psfrag{G2}{$\overline{\Gamma_2^-} \setminus \Gamma_1^-$}
 \psfrag{G3}{$\overline{\Gamma_1^+ \cap \Gamma_2^+}$}
 \psfrag{G4}{\hskip -3em $\overline{\Gamma_1^-} \setminus \Gamma_2^-$}
 \psfrag{z1}{$z_1$}
 \psfrag{z2}{$z_2$}
 \psfrag{z3}{$z_3$}
 \psfrag{z4}{$z_4$}
 \psfrag{p0}{$p_0$}\psfrag{p1}{$p_1$}\psfrag{q0}{\hskip -1em$q_0$}\psfrag{q1}{$q_1$}
 \includegraphics[width=6cm]{curves} 
 \caption{An example of a monotone and non-timelike curve that consists of portions of characteristic curves $C^+_1(z_1)$, $C^-_2(z_2)$, $C^+_1(z_3)$, $C^-_2(z_4)$. Red arrows are parallel to $\N_1$ and blue arrows are parallel to $\N_2$ at respective points.} \label{fig3}
%  \end{subfigure}
\end{figure} 
 
We will also have to consider generalized spacelike curves, for example the staircase pattern consists of sequence of  alternating pieces of forward $1$-family and backward $2$-family trajectories $C^+_1(z_1)$, $C^-_2(z_2)$, $C^+_1(z_3)$, $C^-_2(z_4)$, $\cdots$. We first define the closed cone $K_z$ (see Figure \ref{Tz})  of nonnegative and non-timelike  vectors that is 
 \begin{equation} \label{Kz}
  K_z:= \text{closure of } \left\{ v \in T_z ~|~ v \times \N_1>0 \quad \text{and} \quad v \times \N_2>0\right\}.
 \end{equation}
  We say a closed Lipschitz curve $E$ in $\U$ is monotone and non-timelike if there is an embedding of $\kappa$ of $E$ from $[s_0, s_1]$ with unit speed such that $\kappa'(s)$ is a BV function and at every point $s\in(s_0,s_1)$ the left and right limit $\kappa'(s\pm)$ is in the cone $K_z$, and $\kappa'$ is continuous at the end points $s_0$ and $s_1$. %In particular, it is not dificult to see that $E_*=\overline{\Gamma_1^-\cap \Gamma_2^-}$ and $E^*=\overline{\Gamma_1^+\cap \Gamma_2^+}$ are such examples, where at each end points tangent vector is null. $\Gamma^- \setminus E_*$ and $\Gamma^-_2$
%  $\kappa'(s)$ belongs to $K_{\kappa(s)}$ for every $s\in[s_0,s_1]$. We say a Lipschitz and piecewise $C^1$ curve $E$ is monotone and non-timelike if the curve is piecewisely monotone and non-timelike, and globally Lipschitz. In particular, it is not difficult to see that $\overline{\Gamma_1^-\cap \Gamma_2^-}$ is monotone and non-timelike.







\subsection{Main theorem}
Finally we prove Proposition \ref{prop:final} and Theorem \ref{thm:main}. Materials on Goursat and Triangle problem are necessary in the proof, and are contained in Appendix \ref{local}.
\begin{Prop}\label{prop:final} Suppose $\varphi_1,\varphi_2 \in C^1(\overline U)$ and assign $v_{01}=\varphi_1$ on $\Gamma_1^-$ and $v_{02}=\varphi_2$ on $\Gamma_2^-$. Then there exists a unique solution $(v_1,v_2)\in [C^1(\overline U)]^2$ of \eqref{heqn3}-\eqref{hBC}.
\end{Prop}
\begin{proof} 
First, we set $E_* = \overline{\Gamma_1^- \cap \Gamma_2^-}$, where the value of $(v_1,v_2)$ is known by \eqref{hBC}. Because $v_{0j}$ is uniformly continuous relative to $\Gamma^-_j$, $v_{0j}$ is uniquely extended to $\overline{\Gamma^-_j}$ for $j=1,2$. 

Now, we consider a finite length closed non-timelike curve $E$ in $\U$. We say a closed curve $E$ satisfies the condition $(EC)$ if the following holds:
\begin{enumerate} 
 \item $E$ is homeomorphic to a line segment.
 \item $E$ is entirely contained in $\U$.
 \item One end point of $E$ is in $\overline{\Gamma^-_1} \setminus (\Gamma_1^- \cap \Gamma_2^-)$ and the other is in $\overline{\Gamma^-_2} \setminus (\Gamma_1^- \cap \Gamma_2^-)$.
 \item $E$ is Lipschitz, monotone and non-timelike as defined in Section \ref{causal}.
\end{enumerate}
% \begin{equation} \label{E}\tag{$EC$}
%  \begin{aligned}
% (1) \quad & \text{$E$ is homeomorphic to a line segment.}\\
% (2) \quad & \text{$E$ is entirely contained in $\U$.}\\
% (3) \quad & \text{One end point of $E$ is in $\overline{\Gamma^-_1} \setminus (\Gamma_1^- \cap \Gamma_2^-)$ and the other is in $\overline{\Gamma^-_2} \setminus (\Gamma_1^- \cap \Gamma_2^-)$.}\\
% (4) \quad & \text{$E$ is Lipschitz, monotone and non-timelike as in Remark \ref{nontimelike}.}\\
%  \end{aligned}
% \end{equation}
We define $\mathcal{E}_0$ as the collection of all $E$ satisfying $(EC)$. $\mathcal{E}_0$ is nonempty; $E_*$ and $E^*:=\overline{\Gamma^+_1 \cap \Gamma^+_2}$ are such sets.

For each $E\in\mathcal{E}_0$, we define the closed subset $A(E)$ of $\U$, {\it the past of $E$}: If $E$ intersects $int(\U)$, then $E$ divides $\U$ into two sides. We define $A(E)$ by the side containing $E_*$, as a closed set. If $E$ is entirely on $\partial U$, there are only two possibilities, either $E$ contains $E_*$ or disjoint from $E_*$. For the former case we define $A(E)=E$ and for the latter case $A(E)=\U$. 

Having defined the condition $(EC)$ and $A(E)$, we consider the collection
$$\mathcal{E}:=\left\{ E \in \mathcal{E}_0 ~|~ \text{a solution $(v_1,v_2)$ of \eqref{heqn3}-\eqref{hBC} exists in $A(E)$} \right\}.$$ 
Then $E_* \in \mathcal{E}$ and thus $\mathcal{E}$ is nonempty. We give a partial order $E_1 \preceq E_2$ if $A(E_1)\subseteq A(E_2)$. We now claim: If $E\in \mathcal{E}$ and $A(E)\ne \U$, then there is $E'\in \mathcal{E}$ such that $A(E)$ is a proper subset of $A(E')$. Once the claim is established, a maximal element is in $\mathcal{E}$ and the existence of solution on $\U$ is proved. Uniqueness is a practice of the typical energy method. 

We are left to prove the claim. Suppose $E \in \mathcal{E}$ is given and $A(E)\ne \U$. We divide the proof into two cases. In the first case, a point $z_0$ on $E$ is assumed that is not characteristic, i.e., the tangent vector of $E$ at $z_0$ is parallel neither to $\N_1$ nor to $\N_2$. The remaining case is then everywhere of $E$ is characteristic. If the first case happens, then by continuity there is a neighborhood $\U\cap V$ of $z_0$ such that $E\cap \U \cap V$ is nowhere characteristic. If so, by definition of $E$, $E\cap \U \cap V$ must be spacelike. Let $z \in \U \cap V \setminus A(E)$. Because $C^-_1(z)$ and $C^-_2(z)$ restricted in $\U \cap V \setminus A(E)$ collapse down to $z_0$  as $z$ approaches to $z_0$, for some $z$ close enough to $z_0$, both $C^-_1(z)$ and $C^-_2(z)$ intersect the curve $E\cap \U \cap V$ at $\alpha$ and $\beta$ respectively. Then on the set $T$ enclosed by the curves joining points $z$, $\alpha$, and $\beta$, one can pose a Triangle problem (See Figure \ref{T} in Appendix \ref{local}) of sufficiently small size, and the solution is extended to $T$. Now, $E'$ is the curve obtained from $E$ with the portion $E\cap \U \cap V$ replaced by the two characteristic curves, one joining $z$ and $\alpha$ and the other joining $z$ and $\beta$.

Now, we assume $E$ is everywhere characteristic. Consider an embedding $\kappa:[s_0,s_1] \rightarrow E$ with $\kappa(s_0) \in \overline{\Gamma^-_1} \setminus \Gamma_2^-$ and $\kappa(s_1) \in \overline{\Gamma^-_2} \setminus \Gamma_1^-$. Then there are only three possibilities:
\begin{enumerate}
 \item For some $s_a \in (s_0,s_1)$, $\kappa\big([s_0,s_a]\big) \subset C^+_1(\kappa(s_0))$ and $\kappa(s_0)\ne p_1$.
 \item The entire set $E = C^-_2(\kappa(s_0))$ and $\kappa(s_1)\ne q_1$.
 \item For some $s_a, s_b$ such that $s_0 < s_a < s_b <s_1$, $\kappa\big([s_0,s_a]\big) \subset C^-_2(\kappa(s_0))$  and $\kappa\big([s_a,s_b]\big) \subset C^+_1(\kappa(s_a))$.
\end{enumerate}
It is justified by following reasons: In (1), if $\kappa(s_0)=p_1$, $E$ is tangent to $\partial U$ at $p_1$ and $E$ cannot be a curve everywhere characteristic. Similarly in (2) that $\kappa(s_1)\ne q_1$ is justified.

Suppose (1) happens and let $z_0=\kappa(s_0)$. Because $z_0\ne p_1$ there is a neighborhood $\U \cap V$ of $z_0$ such that $p_1\notin V$. Let $F=\Gamma^-_1 \cap V \setminus A(E)$. Let $z \in \U\cap V \setminus A(E)$. Because $C^-_1(z)$ and $C^-_2(z)$ restricted in $\U\cap V \setminus A(E)$ collapse down to $z_0$ as $z$ approaches to $z_0$, for some $z$ close enough to $z_0$, $C^-_1(z)$ intersects $F$ at $\alpha$ and $C^-_2(z)$ intersects $E$ at $\beta$ respectively. Then $R$ is defined by curves joining $z$, $\alpha$, $\beta$, and $z_0$. One can pose a modified Goursat problem (See Figure \ref{G1} in Appendix \ref{local}) on $R$ of sufficiently small size, and the solution is extended to $R$. Now, $E'$ is the curve obtained from $E$ with the portion $[\alpha,z_0]$ and $[\beta,z_0]$ replaced by $[z,\alpha]$ and $[z,\beta]$.

If (2) happens we do the similar at $\kappa(s_1)\ne q_1$ (See Figure \ref{G2} in Appendix \ref{local} but left-and-right reversed). If (3) happens, we pose a Goursat problem at the corner $\kappa(s_a)$ (See Figure \ref{G} in Appendix \ref{local}).

% Now, applying $\frac{\N_2}{|\N_2|}\cdot \nabla$ and $\frac{\N_1}{|\N_1|}\cdot \nabla$ respectively on the first and the second equation in \eqref{hprob}, we obtain
% \begin{align*}
%  \frac{\N_1}{|\N_1|}\cdot \nabla \Big(\frac{\N_2}{|\N_2|}\cdot \nabla v_1\Big) &= \Big(\frac{\N_1}{|\N_1|}\cdot \nabla \N_2\Big) \cdot \nabla v_1 -  \Big(\frac{\N_2}{|\N_2|}\cdot \nabla \N_1\Big) \cdot \nabla v_1 \\
%  & + \frac{\N_2}{|\N_2|}\cdot \nabla (pv_1 + qv_2), \\
%  \frac{\N_2}{|\N_2|}\cdot \nabla \Big(\frac{\N_1}{|\N_1|}\cdot \nabla v_2\Big) &= \Big(\frac{\N_2}{|\N_2|}\cdot \nabla \N_1\Big) \cdot \nabla v_2 -  \Big(\frac{\N_1}{|\N_1|}\cdot \nabla \N_2\Big) \cdot \nabla v_2 \\
%  & + \frac{\N_1}{|\N_1|}\cdot \nabla (pv_1 + qv_2). 
% \end{align*}
% 
% 

\end{proof}

\begin{Thm}[Existence and Uniqueness] \label{thm:main} Let $\Omega\subset\R^2$ be a simply connected bounded domain with a smooth boundary, $\F_k$, $k=1,2,3$, be admissible vector fields. Suppose the boundary data $v_{01}$ and $v_{02}$ are assigned as in Proposition \ref{prop:final}. Then, there exists a unique anisotropic conductivity $\r\in C(\overline\Omega)$ and potential functions $(u_1,u_2,u_3) \in [C^1(\overline\Omega)]^3$ that satisfies \eqref{myohms}.
\end{Thm}
\begin{proof}
 From $(v_1,v_2)$, the potentials $(u_1,u_2)$, the resistivity $\r$, and then $u_3$ are defined sequencially in $\overline U$ satisfying Ohm's Law in $\overline U$. Via smooth diffeomorphism, the Ohm's Law \eqref{myohms} holds in $\overline \Omega$.
\end{proof}


\section{Conclusions}\label{sect.con}

The conductivity of an animal body obviously has an anisotropic structure because of muscles and nerve fibers in the body (see \cite{Nicholson1965386,Roth2000}). Considering the sensitivity nature of conductivity reconstruction problems isotropic models may give only a limited information for the conductivity of a human body and all the meaningful data related to the anisotropic structure will be forgotten and treated as noise. Therefore, it is not surprising that anisotropic conductivity reconstruction algorithms draw more attention recently. The main purpose of this paper was to develop an anisotropic conductivity reconstruction model in two space dimensions that gives the uniqueness and the existence together. In general, obtaining the uniqueness for an overdetermined problem or the existence for an underdetermined one is easier than obtaining them at the same time. 

In this section we compare the method suggested in the series papers to other methods. The electrical impedance tomography (EIT for brevity) is one of the most actively studied inverse problems (see \cite{ammari_mathematical_2009, ammari_reconstruction_2004,0266-5611-25-12-123011}). The conductivity $\Sigma=\r^{-1}$ is the inverse tensor of the resistivity one and the electrical potential $u$ satisfies
\begin{equation}\label{eqn:div}
\begin{array}{ll}
\nabla\cdot(\Sigma\nabla u)=0  &\quad \mbox{in~~} \Omega,\\
-\Sigma\nabla u\cdot\n=g&\quad \mbox{on~} \partial\Omega,\\
\end{array}
\end{equation}
where $\n$ is the outward unit normal vector to the boundary $\partial\Omega$ and the normal component $g$ of the boundary current density satisfies $\int_{\partial\Omega}gds=0$. It is well known that the mapping that connects the Neumann boundary value $g$ to the Dirichlet boundary potential $u|_{\partial\Omega}$ decides the isotropic conductivity uniquely (see \cite{nachman_global_1996,sylvester_global_1987}). However, the uniqueness holds only in equivalence classes by diffeomorphisms if anisotropic conductivity is allowed (see \cite{MR1955896}). This observation shows the limitation of boundary measurement methods in the construction of anisotropic conductivity distribution.

It is clear that using internal data is unavoidable to obtain anisotropic conductivity tensor and such a method is actually considered on isotropic cases first (see \cite{alessandrini_identification_1986, richter_inverse_1981, richter_numerical_1981}). More recently, MRI technology enabled us to find the current density inside the body by measuring internal magnetic field and several reconstruction algorithms using internal current density have been developed to obtain isotropic conductivity. The uniqueness of the reconstructed isotropic conductivity has been shown in various cases (see \cite{ider_uniqueness_2003,kim_uniqueness_2003, kwon_equipotential_2002,nachman_conductivity_2007,nachman_recovering_2009, seo_magnetic_2011}). The study of anisotropic conductivity has been recently started and, in particular, Bal and his collaborators showed uniqueness of their anisotropic conductivities reconstruction method \cite{bal_inverse_2011, doi:10.1137/140961754, bal_inverse_2014, monard_inverse_2012-1,monard_inverse_2012}.  

The biggest difference of the method we developed is that our method is based on Faraday's law
\begin{equation}\label{FaradayLaw2}
\nabla\times(\r\F)=0,
\end{equation}
which gives a direct connection between the resistivity $\r$ and the current density $\F$. One may consider this choice of the equation as a simplification process that cancels out the unknown variable $u$. It is this simplification that allows us to construct a correctly determined system and obtain the existence, the uniqueness and the stability together.

The curl free equation also allows us to construct numerical algorithms based on loop integrals. In fact, we have developed a numerical scheme using such loop integrals in a mimetic way which turns out virtual resistive network (VRN for brevity) method for resistivity reconstruction. There are many ways to construct numerical schemes using VRN (see \cite{lee_virtual_2014,lee_orthotropic_2015,lee_reconstruction_2010}). In particular, a few explicit methods with local computations are developed in this series of papers. However, these inexpensive local computation methods work only for isotropic and orthotropic cases and a different approach is needed for anisotropic case.

This series of papers on Faraday's law based two dimensional conductivity reconstruction consists of three parts. First, the isotropic conductivity reconstruction has been studied theoretically in \cite{lee_well-posedness_2015} and numerically in \cite{lee_virtual_2014}. The uniqueness, existence and stability were obtained using single set of internal current density and a part of boundary resistivity. The resistivity reconstruction for orthotropic conductivity has been studied theoretically and numerically in \cite{lee_orthotropic_2015}. The well-posedness of the problem has been obtained using two sets of internal current densities. Finally, the anisotropic conductivity has been reconstructed in this paper employing the technique used for the orthotropic case. The newly added part is a construction of new coordinate system that makes the anisotropic structure into an orthotropic one in terms of the new coordinate system. To do that three sets of internal current densities are used. However, the numerical algorithm based on local computations does not work for the anisotropic case and a different approach seems to be needed. In the same spirit, one may also consider a problem in three dimensions. However, due to that the curl operator in three dimensions is a vector-valued, problem becomes quite different in nature. To come up with the right numbers of unknowns and data does not seem to be straightforward. There are results on anisotropic conductivity reconstruction in three dimensions \cite{doi:10.1080/03605302.2013.787089, FR_2018}, where more data are employed.



\appendix

\section{Construction of admissible vector fields}\label{sect.cons}

If there is no admissible current densities, the theory of this paper ends in vain. In the proof of the following proposition we introduce appropriate boundary conditions that may produce admissible set of current densities that satisfy the assumptions (1), (2) and (3) in Definition \ref{def:adm3}. The divergence free equation in \eqref{Eqn3.2} has been studied intensively for a long time, where $\Sigma$ is the \emph{conductivity} tensor corresponding to the resistivity one $\r$, i.e., $\Sigma=\r^{-1}$. There are rich theories related to the solution of this divergence free equation, and, in fact, the first part of the proposition can be found from Bauman \emph{et al.} \cite{MR1871388}. We will fully prove the proposition for completeness using the following lemma. The lemma is related to the interior critical points of an elliptic problem, which is used in Proposition \ref{thm:anisoadm}.

\begin{Lem}[Alessandrini \cite{alessandrini_critical_1987}] \label{lem:aless}
Let $\Omega \subset \R^2$ be a bounded simply connected domain with a smooth boundary, $g \in C(\partial\Omega)$, $a_{ij} \in C^1(\Omega)$, and $a_i \in C(\Omega)$ for $i,j=1,2$. Let $u \in W^2_{loc}(\Omega) \cap C(\overline\Omega)$ satisfy
\begin{align*}
  \sum_{i,j=1}^2 a_{ij} \partial_{x_i}\partial_{x_j}u + \sum_{i=1}^2 a_i \partial_{x_i}u &= 0, \quad \text{in $\Omega$,} \\
  u&=g, \quad \text{on $\partial \Omega$}.
\end{align*}
If $g|_{\partial\Omega}$ has $N$ maxima (and hence it has $N$ minima), then the interior critical points of $u$ are of a finite number and
$$
\sum_{i=1}^K m_i \le N-1,
$$
where $m_1,\cdots,m_K$ are the multiplicities of the corresponding maxima.
\end{Lem}




\begin{Prop} \label{thm:anisoadm}
Suppose that $\Omega\subset\R^2$ is a bounded simply connected domain and $\Sigma=(\sigma^{ij})\in C^1(\Omega)$ is a positive conductivity tensor on it. There exist boundary values $g_k:\partial\Omega\mapsto\R$, $k=1,2,3$, such that the current density fields $\F_k = -\Sigma \nabla u_k $ satisfy (1),(2) and (3) of Definition \ref{def:adm3}, where $u_k$ satisfy
\begin{equation}\label{Eqn3.2}
\begin{array}{cc}\ds
\nabla\cdot(\,\Sigma\nabla u_k)  =0, &\quad \text{in $\Omega$},\\
\Sigma\nabla u_k\cdot \n=g_k, &\quad \text{on $\partial\Omega$}.
\end{array}
\end{equation}
\end{Prop}
\begin{proof}  We will construct three Dirichlet boundary condition $G_k$'s satisfied by stream functions $\psi_k$, which is equivalent to constructing the Neumann data. (See Section \ref{sec:review}.) Let $\gamma:[0, L] \mapsto \partial\Omega$ be an embedding curve on $\partial\Omega$. For a notational convenience, we assume $L=2\pi$ and let $G_1(\gamma(t))=-\sin t$ and $G_2(\gamma(t))=\cos t$. Both $G_1$ and $G_2$ have a single local maximum along the boundary. Let $\r=\Sigma^{-1}$ be the corresponding resistivity tensor and $S$ be given as in Section \ref{sec:review}. Consider $\psi_k$, $k=1,2$, which are the solutions of
\begin{equation}\label{Eqn3.3}
\begin{array}{cc}\ds
 \nabla\cdot(\,S\nabla \psi_k)  =0, &\quad \text{in $\Omega$},\\
 \psi_k=G_k, &\quad \text{on $\partial\Omega$}.
\end{array}
\end{equation}
Then, $\psi_k$ is a stream function of the current density field $\F_k = -\Sigma\nabla u_k$ with the corresponding Neumann boundary value $g_k$.  

Let $\Psi(x,y)= \big(\xi(x,y),\eta(x,y)\big):=\big(\psi_2(x,y), -\psi_1(x,y)\big)$. Since the boundary value is continuous on the smooth boundary $\partial\Omega$, the solution is continuous on $\overline\Omega$. Since $\psi_k|_{\partial\Omega}$ has one local maximum on the boundary, by Lemma \ref{lem:aless}, $\psi_1$ and $\psi_2$ have no critical point in $\Omega$. By the Hopf lemma, $\nabla\psi_k \ne 0$ along the boundary, neither. Suppose that there is a point $\x_0 \in \Omega$ such that $\nabla\psi_1(\x_0) \times \nabla\psi_2(\x_0) =0$. Then, $\nabla\psi_1(\x_0) = c \nabla\psi_2(\x_0)$ for a constant $c\ne0$. Then, $\psi := \psi_1 - c\psi_2$  is also a solution with a boundary condition $\psi\big(\gamma(t)\big) = \sin t + c\cos t =\sqrt{1+c^2} \sin(t+t^*)$ for some $t^*$, and $\x_0$ is an interior critical point of $\psi$. However, this boundary condition also has one local maximum point on the boundary and hence $\psi$ does not have an interior critical point, which is a contradiction. Therefore $\nabla\psi_1 \times \nabla\psi_2$  has no interior zero point, i.e., $\det D\Psi\ne0$ in $\Omega$. Furthermore, since $(G_2(\gamma(t)),-G_1(\gamma(t))) = (\cos t, \sin t)$, the mapping $(\psi_2,-\psi_1)|_{\partial \Omega}$ is one-to-one from $\partial\Omega$ to the unit circle. Therefore, by Lemma \ref{lemma:bijection}, the mapping $\Psi$ is bijective in $\overline\Omega$. The differentiability of the mapping and its inverse one comes from the inverse function theorem. Therefore, we conclude that $\F_1$ and $\F_2$ satisfy the first two admissibility conditions.

Next, we prove the third admissibility condition. The diffeomorphism $\Psi$ gives a new coordinate system $(\xi,\eta):=(\psi_2,-\psi_1)$ where the domain $\Omega$ is  transformed to the unit disk. The third stream function $\psi_3$ is taken as the solution of the uniformly elliptic equation in \eqref{eqn:anisotropic} with a boundary condition $\psi_3(\gamma(t))=\cos 2t$. If we show the Hessian $D^2\tilde{\psi}_3(\xi,\eta)\ne0$ ($\tilde{\psi}_3 = \psi_3 \circ \Psi^{-1}$) for all $(\xi,\eta)$ in the unit disk, the strict inequality in \eqref{strictInquality} holds by Lemma \ref{lemma:scalarcurvature}.

Suppose that $D^2\tilde{\psi}_3(\xi_0,\eta_0)=0$ at a point $(\xi_0,\eta_0)$ and let $\nabla \tilde{\psi}_3 (\xi_0,\eta_0) = (c_1,c_2)$. Then, the Hessian of $\phi:=\tilde{\psi}_3 - c_1\xi - c_2\eta$ is still zero at $(\xi_0,\eta_0)$ and it has a critical point at $(\xi_0,\eta_0)$ with multiplicity $2$. By the linearity of the problem, $\phi$ is a solution with a boundary condition $\phi(\gamma(t)) = \cos 2t - c_1\cos t - c_2\sin t$. In order to investigate the local maxima along the boundary, differentiate $\phi(\gamma(t))$ with respect to $t$ and obtain
$$
{\frac{d}{dt}}\phi(\gamma(t))= -2\sin 2t + c_1\sin t - c_2\cos t =-4\cos t  \sin  t  + c_1\sin t  - c_2\cos t .
$$
One may easily see that this derivative has four zero points. For example, if $c_1=c_2=0$, it has zeros at $t=0,{\pi/2},\pi$ and $3\pi/2$. If not, let $\alpha(\xi,\eta):=-4\xi\eta + c_1\eta-c_2\xi$ which is identical to ${\frac{d}{dt}}\phi(\gamma(t))$ in $\xi,\eta$ variables on the boundary. The zeros of $\alpha$ are hyperbolas and hence there are 4 critical points on the unit circle. In other words, there are at most two local maxima of $\phi$ on the boundary. This contradict Lemma \ref{lem:aless} in Appendix since $\phi$ has an interior critical point of multiplicity $2$. Therefore, there is no such interior point $(\xi_0,\eta_0)$ that makes the Hessian of $\tilde{\psi}_3$ be zero matrix, and the strict inequality in \eqref{strictInquality} is obtained by Lemma \ref{lemma:scalarcurvature}.
\end{proof}



% \begin{Rem}
% We will use this lemma to claim that there is no interior critical point if the boundary value $g$ has only one local maximum point on the boundary.
% \end{Rem}

\section{Quadrilateral and Triangle problem} \label{local}
% Now, having $1$-family and $2$-family of characteristic curves, we are to prove the existence of the linear hyperbolic system. To this ends, we first 
We introduce two building blocks that are the {\it Goursat problem} and the {\it Triangle problem}. For solvabilities of Goursat and Triangle problem, and smoothness results, one is referred to a classical material on a linear hyperbolic system for example \cite[Chapter V. Section 6]{CH_2008}. We briefly explain what they are, adapted to \eqref{heqn3}-\eqref{hBC}.

Consider four points $\bar z\in \overline{U} \setminus (P \cup Q)$, $\bar\alpha$ on $C^-_1(\bar z)$, and $\bar\beta$ on $C^-_2(\bar z)$, and $z_0$ that is the intersection of $C^-_2(\bar\alpha)$ and $C^-_1(\bar\beta)$. It is not difficult to see that the set $R$ enclosed by the four characteristic curves is a diffeomorphic image of a rectangle, where straight lines are mapped to characteristic curves. The portion of $C^-_1(\bar z)$ joining $\bar z$ and $\bar \alpha$ is denoted by $[\bar z, \bar\alpha]$, and other three boundary curves are denoted by $[\bar z,\bar\beta]$, $[\bar\alpha, z_0]$, and $[\bar\beta, z_0]$. We say $R$ is of size $\delta$ for some $\delta>0$, if every characteristic curves of any family intersected by $R$ is of length not greater than $\delta$.
% \captionsetup[subfigure]{labelfont=rm}
\begin{figure}[ht]
 \centering
 \begin{subfigure}[t]{2in}
 \centering
 \psfrag{zb}{\scriptsize$\bar{z}$}
 \psfrag{ab}{\scriptsize$\bar\alpha$}
 \psfrag{bb}{\scriptsize$\bar\beta$}
 \psfrag{z0}{\scriptsize$z0$}
 \psfrag{C1}{\scriptsize\hskip -17pt$C^-_1(\bar z)$}
 \psfrag{C2}{\scriptsize\hskip -20pt$C^-_1(\bar \beta)$}
 \psfrag{C3}{\scriptsize$C^-_2(\bar z)$}
 \psfrag{C4}{\scriptsize$C^-_2(\bar \alpha)$}
 \psfrag{R1}{\scriptsize$[\bar z,\bar\alpha]$}
 \psfrag{R2}{\scriptsize$[\bar z,\bar\beta]$}
 \psfrag{R3}{\scriptsize$[\bar \beta,z_0]$}
 \psfrag{R4}{\scriptsize$[\bar \alpha,z_0]$}
 \includegraphics[width=4.5cm]{Goursat} 
 \caption{Goursat problem} \label{G}
 \end{subfigure} \quad \quad \quad 
 \begin{subfigure}[t]{2in}
 \centering
 \psfrag{zb}{\scriptsize$\bar{z}$}
 \psfrag{ab}{\scriptsize$\bar\alpha$}
 \psfrag{aab}{\scriptsize$\bar\alpha'$}
 \psfrag{abb}{\scriptsize$\bar\beta'$}
 \psfrag{bb}{\scriptsize$\bar\beta$}
 \psfrag{z0}{\scriptsize$z0$}
 \psfrag{C1}{\scriptsize\hskip -17pt$C^-_1(\bar z)$}
 \psfrag{C2}{\scriptsize\hskip -20pt$C^-_1(\bar \beta)$}
 \psfrag{C3}{\scriptsize$C^-_2(\bar z)$}
 \psfrag{C4}{\scriptsize$C^-_2(\bar \alpha)$}
 \psfrag{R1}{\scriptsize$[\bar z,\bar\alpha']$}
 \psfrag{R2}{\scriptsize$[\bar z,\bar\beta']$}
 \psfrag{R3}{\scriptsize$[\bar\alpha',\bar\beta']$}
 \psfrag{R4}{}
 \includegraphics[width=4.5cm]{Triangle} 
 \caption{Triangle problem} \label{T}
 \end{subfigure}\\
 \vskip 10pt
 \begin{subfigure}[t]{2in}
 \centering
 \psfrag{zb}{\scriptsize$\bar{z}$}
 \psfrag{ab}{\scriptsize$\bar\alpha$}
 \psfrag{aab}{\scriptsize$\bar\alpha'$}
 \psfrag{bb}{\scriptsize$\bar\beta$}
 \psfrag{z0}{\scriptsize$z0$}
 \psfrag{C1}{\scriptsize\hskip -17pt$C^-_1(\bar z)$}
 \psfrag{C2}{\scriptsize\hskip -20pt$C^-_1(\bar \beta)$}
 \psfrag{C3}{\scriptsize$C^-_2(\bar z)$}
 \psfrag{C4}{\scriptsize$C^-_2(\bar \alpha)$}
 \psfrag{R1}{\scriptsize$[\bar z,\bar\alpha']$}
 \psfrag{R2}{\scriptsize$[\bar z,\bar\beta']$}
 \psfrag{R3}{\scriptsize$[\bar\beta,z_0]$}
 \psfrag{R4}{\scriptsize$[\bar\alpha',z_0]$}
 \includegraphics[width=4.5cm]{Goursat1} 
 \caption{Modified Goursat problem 1} \label{G1}
 \end{subfigure} \quad \quad \quad 
 \begin{subfigure}[t]{2in}
 \centering
 \psfrag{zb}{\scriptsize$\bar{z}$}
 \psfrag{ab}{\scriptsize$\bar\alpha$}
 \psfrag{abb}{\scriptsize$\bar\beta'$}
 \psfrag{bb}{\scriptsize$\bar\beta$}
 \psfrag{z0}{\scriptsize$z0$}
 \psfrag{C1}{\scriptsize\hskip -17pt$C^-_1(\bar z)$}
 \psfrag{C2}{\scriptsize\hskip -20pt$C^-_1(\bar \beta)$}
 \psfrag{C3}{\scriptsize$C^-_2(\bar z)$}
 \psfrag{C4}{\scriptsize$C^-_2(\bar \alpha)$}
 \psfrag{R1}{\scriptsize$[\bar z,\bar\alpha]$}
 \psfrag{R2}{\scriptsize$[\bar z,\bar\beta']$}
 \psfrag{R3}{\scriptsize$[\bar \beta',z_0]$}
 \psfrag{R4}{\scriptsize$[\bar \alpha,z_0]$}
 \includegraphics[width=4.5cm]{Goursat2}
 \caption{Modified Goursat problem 2} \label{G2}
 \end{subfigure} 
 \caption{Configurations for the Goursat problem and the Triangle problem. The red curves are in $1$-characteristic family and the blue curves are in $2$-characteristic family.} \label{GT}
\end{figure}


The {\it Goursat problem} consists of \eqref{heqn} restricted on $R$ and data $v_1$ on $[\bar\alpha,z_0]$ and $v_2$ on $[\bar\beta,z_0]$. See Figure \ref{GT} (A). For each point $z\in R$, the solution admits the integral representation 
\begin{equation} \label{integral}
\begin{aligned}
 v_1(z) &= v_{1}(\alpha(z)) + \int_0^{\ell(z)} (pv_1 + qv_2)(\ell) \; d\ell, \\
 v_2(z) &= v_{2}(\beta(z)) + \int_0^{\tau(z)} (rv_1 + sv_2)(\tau) \; d\tau, 
\end{aligned}
\end{equation}
where $\alpha(z)$ is the intersection of $C^-_1(z)$ and the $[\bar\alpha,z_0]$, $\beta(z)$ is the intersection of the $C^-_2(z)$ and the $[\bar\beta,z_0]$, $\ell$ and $\tau$ are the arc length parameters so that $z$ is reached at $\ell(z)$ and $\tau(z)$.

The right-hand-sides of \eqref{integral} defines an operator $J$ from $L^\infty(R)$ to itself such that a solution $(v_1,v_2)$ is a fixed point of $J$. Because $p$, $q$, $r$, and $s$ are uniformly bounded in $\U$, there is a $\delta>0$ such that if $R$ is of size $\delta$, then $J$ is a contraction mapping. In conclusion, the Goursat problem on $R$ of size $\delta$ is solvable.

The Goursat problem can be modified in a way, the boundary portion $[\bar\alpha, z_0$] is replaced by $[\bar\alpha', z_0]$ that is an arc joining $\bar\alpha'$ and $z_0$ entirely in $R$, where $\bar\alpha' \in [\bar z, \bar\alpha]$ and the arc is timelike in its relative interior. New region $R'$ is the one enclosed by $[\bar\alpha', z_0]$ and the other three. If data $v_1$ on $[\bar\alpha',z_0]$ and $v_2$ on $[\bar\beta,z_0]$ are assigned, the modified problem is also solvable using the contraction mapping principle. See Figure \ref{GT} (C). A configuration illustrated in Figure \ref{GT} (D) is another way of modification, and we omit the description. The Goursat problem and the modified Goursat problems will be called the Quadrilateral problem. %For the smoothness of solution, we refer discussions in \cite[Chapter V. Section 6]{CH_2008} where the subdivision of $R$ into even smaller regions and successive applications of picard type iteration may be needed.

The Triangle problem can be also considered. Let $R$ be the set as in the above. Suppose $\bar\alpha' \in [\bar z, \bar\alpha]$, $\bar\beta' \in [\bar z, \bar\beta]$ and we join $\bar\alpha'$ and $\bar\beta'$ by an arc entirely in $R$ that is spacelike in its relative interior. The set enclosed by $[\bar z,\bar\alpha']$, $[\bar z,\bar\beta']$ and the arc $[\bar\alpha',\bar\beta']$ is denoted by $T$. The Triangle problem consists of \eqref{heqn} restricted on $T$ and the {\it initial} data $(v_1,v_2)$ on the arc $[\bar\alpha',\bar\beta']$. For each point $z\in T$, the solution admits the integral representation in the same form \eqref{integral} but now $\alpha(z)$ is the intersection of $C^-_1(z)$ and $[\bar \alpha', \bar\beta']$, and $\beta(z)$ is the intersection of $C^-_2(z)$ and $[\bar \alpha', \bar\beta']$. If $R\supset T$ is of size $\delta$, then the Triangle problem is solvable using the contraction mapping. See Figure \ref{GT} (B).

%\bibliographystyle{unsrt}
\bibliographystyle{amsplain}
\begin{thebibliography}{10}

    
       

\bibitem{ammari_mathematical_2009}
        {H. Ammari, Y. Capdeboscq, H. Kang, and A. Kozhemyak},
        {Mathematical models and reconstruction methods in magneto-acoustic imaging}, 
        European Journal of Applied Mathematics \textbf{20} (2009), no. 3  303--317.% \MR{2511278 (2010f:35428)}

\bibitem{ammari_reconstruction_2004}
        {H. Ammari and H. Kang}, 
        {Reconstruction of small inhomogeneities from boundary measurements}, 
        Lecture Notes in Mathematics, vol. 1846 (Springer-Verlag, Berlin, 2004). %\MR{2168949 (2006k:35295)}

\bibitem{bal_2013}
        {G. Bal}, 
        {Cauchy problem for ultrasound-modulated EIT}, 
        Analysis \& PDE \textbf{6} (2013), no. 4, 751--775.

\bibitem{bal_inverse_2011}
        {G. Bal, E. Bonnetier, F. Monard, and F. Triki}, 
        {Inverse diffusion from knowledge of power densities},
        arXiv preprint arXiv:1110.4577 (2011).

\bibitem{doi:10.1137/140961754}
        {G. Bal, C. Guo, and F. Monard}, 
        {Imaging of anisotropic conductivities from current densities in two dimensions}, 
        SIAM Journal on Imaging Sciences \textbf{7} (2014), no. 4 2538--2557.

\bibitem{bal_inverse_2014}
        {G. Bal, C. Guo, and F. Monard}, 
        {Inverse anisotropic conductivity from internal current densities}, 
        Inverse Problems \textbf{30} (2014), no. 2, 025001.

\bibitem{MR1871388}
        {P. Bauman, A Marini, and V Nesi}, 
        {Univalent solutions  of an elliptic system of partial differential equations arising in homogenization}, Indiana Univ. Math. J. \textbf{50} (2001), no. 2, 747--757.

\bibitem{MR1384756}
        {J.K. Beem, P.E. Ehrlich, and K.L. Easley}, 
        {Global Lorentzian geometry 2nd. ed.}, 
        Monographs and Textbooks in Pure and Applied Mathematics, vol. 202, 
        (Marcel Dekker, Inc., New York, 1996). %\MR{1384756  (97f:53100)}
  
\bibitem{MR1955896}
L. Borcea, {Electrical impedance tomography}, Inverse Problems
  \textbf{18} (2002), no.~6, R99--R136. %\MR{1955896}

  \bibitem{CH_2008}
        {R. Courant and D. Hilbert},
        {Methods of mathematical physics vol 2: partial differential equations.},
        {John Wiley \& Sons 2008}.

  
\bibitem{Gamba}
H.E. Gamba, D. Bayford, and D. Holder, {Measurement of electrical current
  density distribution in a simple head phantom with magnetic resonance
  imaging}, Phys. Med. Biol. \textbf{44} (1999), 281--91.

  
\bibitem{gilbarg_elliptic_2001}
D. Gilbarg and N. Trudinger, {Elliptic partial differential
  equations of second order}, Classics in Mathematics, Springer-Verlag, Berlin,
  2001, Reprint of the 1998 edition.

        
        
\bibitem{alessandrini_identification_1986}
        {A. Giovanni}, 
        {An identification problem for an elliptic equation in two variables}, 
        Annali di matematica pura ed applicata \textbf{145} (1986), no. 1, 265--295.

\bibitem{alessandrini_critical_1987}
        {A. Giovanni}, 
        {Critical points of solutions of elliptic equations in two  variables}, 
        Annali della Scuola Normale Superiore di Pisa. Classe di Scienze. Serie {IV} \textbf{14} (1987), no. 2, 229--256.
  
  
  
\bibitem{MR3206987}
N. Hoell, A. Moradifam, and A. Nachman, {Current density
  impedance imaging of an anisotropic conductivity in a known conformal class},
  SIAM J. Math. Anal. \textbf{46} (2014), no.~3, 1820--1842. %\MR{3206987}

\bibitem{ider_uniqueness_2003}
Y.Z. Ider, S. Onart, and W. Lionheart, {Uniqueness and reconstruction in
  magnetic resonance-electrical impedance tomography ({MR}-{EIT})},
  Physiological measurement \textbf{24} (2003), 591--604.

\bibitem{Joy}
M.L.G. Joy, G.C. Scott, and R.M. Henkelman, {In vivo detection of applied
  electric currents by magnetic resonance imaging}, Magn. Reson. Imaging
  \textbf{7} (1989), 89--94.

\bibitem{kim_uniqueness_2003}
Y.-J. Kim, O. Kwon, J. Seo, and E. Woo, {Uniqueness and
  convergence of conductivity image reconstruction in magnetic resonance
  electrical impedance tomography}, Inverse Problems \textbf{19} (2003), no.~5,
  1213--1225.

\bibitem{lee_well-posedness_2015}
Y.-J. Kim and M.-G. Lee, {Well-posedness of the conductivity
  reconstruction from an interior current density in terms of schauder theory},
  Quart. Appl. Math. \textbf{73} (2015), 419--433.

\bibitem{kwon_equipotential_2002}
O. Kwon, J.-Y. Lee, and J.-R. Yoon, {Equipotential line method
  for magnetic resonance electrical impedance tomography}, Inverse Problems
  \textbf{18} (2002), no.~4, 1089--1100.

\bibitem{lee_virtual_2014}
M.-G. Lee, M.-S. Ko, and Y.-J. Kim, {Virtual resistive network and
  conductivity reconstruction with faraday's law}, Inverse Problems
  \textbf{30} (2014), no.~12, 125009.

\bibitem{lee_orthotropic_2015}
M.-G. Lee, M.-S. Ko, and Y.-J. Kim, {Orthotropic conductivity reconstruction with virtual-resistive
  network and faraday's law}, Mathematical Methods in the Applied Sciences \textbf{39}
  (2015), no. 5, 1183--1196.% (http://dx.doi.org/10.1002/mma.3564).

\bibitem{lee_reconstruction_2010}
T. Lee, H. Nam, M.-G. Lee, Y.-J. Kim, E. Woo, and O.Kwon, {Reconstruction of conductivity using the dual-loop method with one injection current in {MREIT}}, Physics in Medicine and Biology
  \textbf{55} (2010), no.~24, 7523.

\bibitem{meisters_locally_1963}
G. H. Meisters and C. Olech, {Locally one-to-one mappings and a classical
  theorem on schlicht functions}, Duke Mathematical Journal \textbf{30} (1963),
  63--80.

  
\bibitem{monard_inverse_2012-1}
F. Monard and G. Bal, {Inverse anisotropic diffusion from
  power density measurements in two dimensions}, Inverse Problems \textbf{28}
  (2012), no.~8, 084001, 20.

\bibitem{monard_inverse_2012}
F. Monard and G. Bal, {Inverse diffusion problems with
  redundant internal information}, Inverse Problems and Imaging \textbf{6}
  (2012), no.~2, 289--313.

\bibitem{doi:10.1080/03605302.2013.787089}
F. Monard and G. Bal, {Inverse anisotropic conductivity from
  power densities in dimension $n \ge 3$}, Communications in Partial
  Differential Equations \textbf{38} (2013), no.~7, 1183--1207.

\bibitem{FR_2018}
        {F. Monard and D. Rim},
        {Imaging of isotropic and anisotropic conductivities from power densities in three dimensions},
        Inverse Problems \textbf{34} (2018), no. 7,  075005.
  
\bibitem{nachman_conductivity_2007}
A. Nachman, A. Tamasan, and A. Timonov, {Conductivity
  imaging with a single measurement of boundary and interior data}, Inverse
  Problems \textbf{23} (2007), no.~6, 2551--2563.

\bibitem{nachman_recovering_2009}
A. Nachman, A. Tamasan, and A. Timonov, {Recovering the conductivity from a single measurement of
  interior data}, Inverse Problems. An International Journal on the Theory and
  Practice of Inverse Problems, Inverse Methods and Computerized Inversion of
  Data \textbf{25} (2009), no.~3, 035014. 

\bibitem{nachman_global_1996}
A. Nachman, {Global uniqueness for a two-dimensional inverse boundary
  value problem}, Annals of Mathematics (1996), 71--96.

\bibitem{Nicholson1965386}
P.W. Nicholson, {Specific impedance of cerebral white matter},
  Experimental Neurology \textbf{13} (1965), no.~4, 386 -- 401.

\bibitem{perko_differential_2001}
        {L. Perko}, 
        {Differential equations and dynamical systems 3rd. ed.}, 
        TAM {\bf 7} (Springer-Verlag New York 2001).
 
  
\bibitem{richter_inverse_1981}
G.R. Richter, {An inverse problem for the steady state diffusion
  equation}, {SIAM} Journal on Applied Mathematics \textbf{41} (1981), no.~2,
  210--221.

\bibitem{richter_numerical_1981}
G.R. Richter, {Numerical identification of a spatially varying diffusion
  coefficient}, mathematics of computation \textbf{36} (1981), no.~154,
  375--386.

\bibitem{Roth2000}
B.J. Roth, {The electrical conductivity of tissues, in the biomedical
  engineering handbook}, CRC Press, Boca Raton, FL, USA, 2000.

\bibitem{Scott}
G.C. Scott, M.L.G. Joy, R.L. Armstrong, and R.M. Henkelman, {Measurement
  of nonuniform current density by magnetic resonance}, IEEE Trans. Med.
  Imaging \textbf{10} (1991), 362--74.

\bibitem{seo_magnetic_2011}
J. Seo and E. Woo, {Magnetic resonance electrical impedance
  tomography ({MREIT})}, {SIAM} Review \textbf{53} (2011), no.~1, 40--68.

\bibitem{sylvester_global_1987}
J. Sylvester and G. Uhlmann, {A global uniqueness theorem for an inverse
  boundary value problem}, Annals of Mathematics (1987), 153--169.

\bibitem{0266-5611-25-12-123011}
G. Uhlmann, {Electrical impedance tomography and calderón's problem},
  Inverse Problems \textbf{25} (2009), no.~12, 123011.

\end{thebibliography}

\end{document}

