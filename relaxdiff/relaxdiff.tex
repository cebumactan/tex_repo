%%%%%%%%%%%%%%%%%%%%%%%%%%%%%%%%%%%%%%%%%%%%%%%
%
%    thermalelectric equation
%
%                                                      
%
%                                       
%
%                                          version Aug 2017
%
%
%%%%%%%%%%%%%%%%%%%%%%%%%%%%%%%%%%%%%%%%%%%%%%%
\documentclass[a4paper,11pt]{article}

\usepackage[margin=3cm]{geometry}
\usepackage{setspace}
\onehalfspacing
%\doublespacing
%\usepackage{authblk}
\usepackage{amsmath}
\usepackage{amssymb}
\usepackage{amsthm}
\usepackage{calrsfs}
%\usepackage[notcite,notref]{showkeys}

\usepackage{psfrag}
\usepackage{graphicx,subfigure}
\usepackage{color}
\def\red{\color{red}}
\def\blue{\color{blue}}
%\usepackage{verbatim}
% \usepackage{alltt}
%\usepackage{kotex}



\usepackage{enumerate}

%%%%%%%%%%%%%% MY DEFINITIONS %%%%%%%%%%%%%%%%%%%%%%%%%%%

\def\tr{\,\textrm{tr}\,}
\def\div{\,\textrm{div}\,}
\def\sgn{\,\textrm{sgn}\,}

\newcommand{\tcr}{\textcolor{red}}
\newcommand{\tcb}{\textcolor{blue}}
\newcommand{\ubar}[1]{\text{\b{$#1$}}}
\newtheorem{theorem}{Theorem}
\newtheorem{lemma}{Lemma}[section]
\newtheorem{proposition}{Proposition}[section]
\newtheorem{corollary}{Corollary}[section]
\newtheorem{definition}{Definition}[section]
\newtheorem{claim}{Claim}

\theoremstyle{remark}
\newtheorem{remark}{Remark}[section]


%%%%%%%%%%%%%%%%%%%%%%%%%%%%%%%%%%%%%%%%%%%%%%%%%%%%%%%%%%
\begin{document}
\title{Diffusion controlled by ode relaxation:\\stable regime $n-m>0$}
% \author{Athanasios Tzavaras\footnotemark[1] \and Min-Gi Lee\footnotemark[2]} %\ \footnotemark[3]  % \footnotemark[4]}
\date{}

\maketitle
% \renewcommand{\thefootnote}{\fnsymbol{footnote}}
% \footnotetext[2]{Min-Gi Lee \\ King Abdullah University of Science and Technology (KAUST), Computer, Electrical and Mathematical Sciences \& Engineering Division, KAUST, Thuwal, Saudi Arabia, \\ e-mail: mingi.lee@kaust.edu.sa}%{Computer, Electrical and Mathematical Sciences \& Engineering Division, King Abdullah University of Science and Technology (KAUST), Thuwal, Saudi Arabia}
% \footnotetext[1]{Athanasios Tzavaras \\ King Abdullah University of Science and Technology (KAUST), Computer, Electrical and Mathematical Sciences \& Engineering Division, KAUST, Thuwal, Saudi Arabia,\\  e-mail: {athanasios.tzavaras@kaust.edu.sa}}

\renewcommand{\thefootnote}{\arabic{footnote}}

\maketitle

\begin{abstract}
\end{abstract}

\tableofcontents
\pagebreak 
 
We study system of equations
\begin{equation}\label{ug_system}
\begin{aligned}
 \partial_t u &= \Delta \Big(\frac{u^n}{\gamma}\Big),\\
 \partial_t \gamma &= u^m - \gamma
\end{aligned}
\end{equation}
in a bounded $C^{2,\alpha}$ domain $\Omega\subset \mathbb{R}^d$
with initial conditions
\begin{equation}
 \gamma(0,x)=\gamma_0(x), \quad u(0,x)=u_0(x), \quad x\in \Omega,
\end{equation}
and boundary conditions
\begin{equation}
 \nabla\Big(\frac{u^n}{\gamma}\Big)\cdot \mathbf{n} = 0, \quad \partial\Omega. \label{bdrycond}
\end{equation}
\eqref{bdrycond} implies
\begin{equation}
 \frac{d}{dt} \int_\Omega u \: dx = 0. \quad \text{We normalize so that}  \int_\Omega u \: dx=1.
\end{equation}
We define $\sigma\triangleq\frac{|u|^{n-1}u}{\gamma}$. 
We can formulate the problem with variables $\sigma$ and $\gamma$ by the relation 
$\displaystyle u = \sigma^{\frac{1}{n}}\gamma^{\frac{1}{n}}$:
\begin{align}
%  \partial_t \big(\sigma^{\frac{1}{n}}\big) &= \gamma^{-\frac{1}{n}} \Delta \sigma - \frac{1}{n} \sigma^{\frac{1}{n}} \gamma^{-1+\frac{m}{n}}\Big(\sigma^{\frac{m}{n}}-\gamma^{1-\frac{m}{n}}\Big), \label{eq:s}\\
 \partial_t \sigma &= n\gamma^{-\frac{1}{n}}\sigma^{\frac{n-1}{n}} \Delta \sigma - \gamma^{-1}\sigma\Big(\gamma^{\frac{m}{n}}\sigma^{\frac{m}{n}}-\gamma\Big) \label{eq:s},\\
 \partial_t \gamma &= \gamma^{\frac{m}{n}}\sigma^{\frac{m}{n}} - \gamma. \label{eq:g}
\end{align}
Initial conditions:
\begin{align}
 \gamma(0,x)&=\gamma_0(x), \quad \sigma(0,x)=\sigma_0(x), \quad x\in \Omega. \label{initcond}
\end{align}
In particular, \eqref{eq:g} gives rise to the integral representation of $\gamma(t,x)$,
\begin{equation}
 \gamma(t,x)^{\frac{n-m}{n}} = e^{-\frac{n-m}{n}t}\gamma_0(x)^{\frac{n-m}{n}} + e^{-\frac{n-m}{n}t}\int_0^t \frac{n-m}{n} e^{\frac{n-m}{n}s} \sigma(s,x)^{\frac{m}{n}} \: ds. \label{eq:rep}
\end{equation}
In this paper, we study the stable regime, namely
\begin{equation}
 n-m>0. \label{stable}
\end{equation}
\section{Global Existence}
The notations from the study of parabolic problem is used as in the below:
\begin{align*}
 \Omega \subset \mathbb{R}^d& \text{ is a bounded $C^{2,\alpha}$ domain with $0<\alpha <1$},\\
 Q_T &\triangleq (0,T)\times \Omega,\\
 Q &\triangleq (0,\infty)\times \Omega,\\
 \Gamma_T &\triangleq (0,T)\times \partial\Omega \cup \{0\}\times\Omega,\\
 \Gamma &\triangleq (0,\infty)\times \partial\Omega \cup \{0\}\times\Omega. 
\end{align*}

\begin{lemma}
Suppose $\sigma,\gamma\in C^{2+\alpha,1+\alpha/2}(\bar{Q})$ is a solution of \eqref{eq:s}-\eqref{eq:g} such that $\gamma^- \le \gamma(0,x) \le \gamma^+$ and $\sigma^- \le \sigma(0,x) \le \sigma^+$, where $\gamma^-,\gamma^+,\sigma^-,\sigma^+$ are positive constants. Then 
\begin{enumerate}
 \item $\gamma^- \le \gamma(t,x) \le \gamma^+, \quad \sigma^- \le \sigma(t,x) \le \sigma^+ \quad \text{for all $(t,x)\in Q$}.$
 \item $||\nabla\sigma||_{L^2(\Omega)} \le C(n,m,\gamma^-,\gamma^+,\sigma^-,\sigma^+).$
\end{enumerate}
\end{lemma}
\begin{proof}
 {\blue For the first assertion, we use the invariant region method.}
 
 The second assertion is from the Energy dissipation equality
\begin{align}
 \partial_t\Big( \frac{u^{n+1}}{\gamma} &+ \frac{m}{n+1-m}\gamma^{\frac{n+1-m}{m}}\Big) + (n+1) \left|\nabla \big(\frac{u^n}{\gamma}\big)\right|^2 \nonumber \\
 &= \div \Big( (n+1)\big(\frac{u^n}{\gamma}\big)\nabla \big(\frac{u^n}{\gamma}\big)\Big) - \frac{1}{\gamma^2}\Big( (u^m)^{\frac{n+1}{m}} - \gamma^{\frac{n+1}{m}}\Big)(u^m-\gamma). \label{eq:energy}
\end{align}
Note that $\displaystyle - \frac{1}{\gamma^2}\Big( (u^m)^{\frac{n+1}{m}} - \gamma^{\frac{n+1}{m}}\Big)(u^m-\gamma)\le0$. %From the energy equality, we have a uniform bound of $||\nabla\sigma||_{H^1(\Omega)}$.
\end{proof}
{\red 
\begin{proposition}
Suppose $\sigma,\gamma\in C^{2+\alpha,1+\alpha/2}(\bar{Q})$ is a solution of \eqref{eq:s}-\eqref{eq:g}. Then 
$$ ||\sigma||_{C^{2+\alpha,1+\alpha/2}(\bar{Q})}, ||\gamma||_{C^{2+\alpha,1+\alpha/2}(\bar{Q})} \le C,$$
where $C$ depends only on $d,\gamma^-,\gamma^+, \sigma^-, \sigma^+, ||\sigma(0,x)||_{C^{2,\alpha}(\Omega)}$, and $||\gamma(0,x)||_{C^{2,\alpha}(\Omega)}$.
\end{proposition}
\begin{proof}
Krylov-Safonov theory needs the VMO or Holder coefficients, but $||\nabla\sigma||_{L^2(Q)}$ is not enough for that...
%  Let us write the equation \eqref{eq:g} in the form
%  \begin{equation}
%   \partial_t \sigma = a(\gamma,\sigma) \Delta \sigma + f(\gamma,\sigma). \label{eq:abbrev}
%  \end{equation}
%  From the the lower and upper bounds of $\gamma,\sigma$, the $L^2$ bound of the $\nabla \sigma$ and the Poincar\'e inequality, $(i)$ the coefficient $a(\gamma,\sigma)$ has a uniform ellipticity constant; $(ii)$ $a(\gamma,\sigma)$ is in the space of VMO. By the parabolic regularity theory by the Krlov-Safonov, we obtain the $||\sigma||_{W^{2,p}(\Omega)}$ bound for any $1<p<\infty$ by the constant $C$ that only depends on $d$, the ellipticity constant, $||\nabla\sigma||_{L^2(\Omega)}$ and $p$. With a high enough $p$, this gives rise to the H\"older estimates of $\sigma$ and in turn that of $\gamma$, $\partial_t\sigma$, and $\partial_t\gamma$. The parabolic regularity theory with the coefficient in H\"older then gives $\sigma\in C^{2+\alpha,1+\alpha/2}(\bar{Q})$ bound as stated.
\end{proof}
}

\begin{theorem} \label{thm:existence}
 \eqref{eq:s}-\eqref{eq:g} with initial boundary conditions \eqref{bdrycond} and \eqref{initcond} has a unique globally defined classical solution.
\end{theorem}
 
% \begin{corollary} Suppose $\sigma,\gamma\in C^{2+\alpha,1+\alpha/2}(\bar{Q}_T)$ is a solution of \eqref{aux}.
%  \begin{enumerate}
% %   \item $\gamma^- \le \gamma(t,x) \le \gamma^+$ and $\sigma^- \le \sigma(t,x) \le \sigma^+$ for all $(t,x)\in Q_T$. %[0,\infty)\times \bar\Omega$.
%   \item $||\sigma||_{C^{2,1,\alpha}(Q_T)} \le C\Big(\min_{t,x}a(t,x), \gamma^-,\gamma^+, \sigma^-, \sigma^+,||\sigma_0||_{C^{2,\alpha}(\bar\Omega)},||\gamma_0||_{C^{2,\alpha}(\bar\Omega)}\Big).$
%   \item $||\gamma||_{C^{2,1,\alpha}(Q_T)} \le C\Big(\min_{t,x}a(t,x), \gamma^-,\gamma^+, \sigma^-, \sigma^+, ||\sigma_0||_{C^{2,\alpha}(\bar\Omega)},||\gamma_0||_{C^{2,\alpha}(\bar\Omega)}\Big).$
%  \end{enumerate}
% \end{corollary}

% \begin{proposition}
%  The auxilary system has a unique solution.
% \end{proposition}
\begin{proof}
We first proceed with the short time existence in $\bar{Q}_{T}$ with $T>0$ small enough, with cruder estimate
% $$ \underline{\gamma} \le \gamma(t,x) \le \overline{\gamma}$$
\begin{equation}\underline\gamma \le \gamma(t,x) \le \overline\gamma, \quad \underline\sigma \le \sigma(t,x) \le \overline\sigma \quad \text{for all $(t,x)\in Q_T$}.\label{eq:cruder} \end{equation}
$\underline\gamma$, $\overline\gamma$, $\underline\sigma$, and $\overline\sigma$ are positive constants. This lower bound is in particular necessary because the nonlinearities $a(\gamma,\sigma)$ and $f(\gamma,\sigma)$ are unbounded as $\gamma \rightarrow 0$ and are not Lipschitz as $\sigma \rightarrow 0$. 

1. We choose $\underline{\sigma}\le \sigma^-$ and $\overline{\sigma}\ge 2\sigma^+$. Then from the integral expression \eqref{eq:rep}, $\gamma(t,x) \in [\underline{\gamma},\overline\gamma]$, where $\underline\gamma^{n-m} = \underline\sigma^m$; $(ii)$ $\overline\gamma^{n-m} = \overline\sigma^m$. Because $n-m>0$, we can have them so that the cube $K\triangleq[\underline{\gamma},\overline{\gamma}]\times[\underline\sigma,\overline\sigma] \supset [\gamma^-,2\gamma^+]\times[\sigma^-,2\sigma^+]$.
% 
% 
% $\underline\gamma,\underline\sigma$ small enough and $\overline\gamma,\overline\sigma$ large enough so that the cube $K\triangleq[\underline{\gamma},\overline{\gamma}]\times[\underline\sigma,\overline\sigma] \supset [\gamma^-,2\gamma^+]\times[\sigma^-,2\sigma^+]$. We can choose them such that $(i)$ $\underline\gamma^{n-m} = \underline\sigma^m$; $(ii)$ $\overline\gamma^{n-m} = \overline\sigma^m$. This is possible because $n-m>0$. 


% $\underline\gamma \le \gamma^-$, $\underline\sigma \le \sigma^-$, $\overline\gamma \ge 2\gamma^+$, $\overline\sigma \ge 2\sigma^+$. 

% Now, let $M>0$ be bigger than the Lipschitz constant of $a(\gamma,\sigma)$ and $f(\gamma,\sigma)$ 
2. Now, let $$B\triangleq\left\{\sigma \in C^{2+\alpha,1+\alpha/2}(\bar{Q}_T) ~\Big|~ \sigma(t,x) \in [\underline\sigma, \overline\sigma] \quad \text{for all $(t,x)\in Q_T$}\right\},$$ 
We define the transformation $\Phi: B \mapsto B$ in the following way. For a given $\sigma_j \in B$, we first define $\gamma_j(t,x)$ in $Q_T$ via the integral expression \eqref{eq:rep}. It is straightforward to check that $\sigma(t,x)\in [\underline\sigma,\overline\sigma]$ implies $\gamma(t,x) \in [\underline\gamma,\overline\gamma]$ and 
{\blue$$||\gamma||_{C^{2+\alpha,1+\alpha/2}(\bar{Q}_T)} \le C(blabla)||\sigma||_{C^{2+\alpha,1+\alpha/2}(\bar{Q}_T)}.$$}

3. By the maximum principle we have
$$||\sigma_{j+1}||_{L^\infty} \le T ||\tilde f||_{L^\infty}+ ||\sigma_{j+1}(0,x)||_{L^\infty},$$
so we can choose $T$ small enough that $\sigma_{j+1} \le 2\sigma^+ \le \overline{\sigma}.$ Since $|\tilde f|$ is bounded in $K$ and the bound depends only on the $\underline{\gamma}$, $\overline{\gamma}$, $\underline{\sigma}$, and $\overline{\sigma}$, so is $T$.



where $M>0$ is chosen later.   

Now we define $\sigma_{j+1}$ by the solution of 
\begin{equation}
\partial_t \sigma_{j+1} + M_1\sigma_{j+1} = a(\gamma_j,\sigma_j)\Delta \sigma_{j+1} + M_1\sigma_j + f(\gamma_j,\sigma_j), \label{eq:iteration}
\end{equation}
with the same initial boundary conditions. Let $M_1>0$ very large so that 
$$ M|a(\gamma_j,\sigma_j)-a(\gamma_{j-1},\sigma_{j-1})|+ |f(\gamma_j,\sigma_j)-f(\gamma_{j-1},\sigma_{j-1})| \le M_1|\sigma_j - \sigma_{j-1}|.$$
In the above the Lipschitz constant could be taken because $(\gamma(t,x),\sigma(t,x))\in K$. $|\gamma_j - \gamma_{j-1}|$ was estimated by $|\sigma_j-\sigma_{j+1}|$ via the integral expression. If $\tilde a(t,x)\triangleq a(\gamma_j(t,x),\sigma_j(t,x))$ and $\tilde f(t,x)\triangleq f(\gamma_j(t,x),\sigma_j(t,x))$ then the linear elliptic problem is solvable because $(\gamma_j(t,x),\sigma_j(t,x))\in K$ and thus $(i)$ $a(\gamma_j,\sigma_j)$ has the uniform ellipticity constant; $(ii)$ $\tilde a(t,x),\tilde f(t,x) \in C^{\alpha,\alpha/2}(\bar{Q}_T)$. Therefore $\sigma_{j+1}$ is well-defined and we must prove that $\sigma_{j+1} \in B$.% thanks to the lower bounds of $\gamma$ and $\sigma$. 

3.  Let $\sigma_0 \equiv \underline\sigma$, $\gamma_0\equiv \underline\gamma$, $w_{j+1}=\sigma_{j+1} - \sigma_j$. We prove that $w_{j+1}\ge0$ by the induction. Suppose $w_j\ge0$. $w_{j+1}$ satisfies
\begin{equation}
\begin{aligned}
\partial_t w_{j+1} + M_1w_{j+1} &= a(\gamma_j,\sigma_j)\Delta w_{j+1} + M_1w_j + \Delta \sigma_j\big( a(\gamma_j,\sigma_j)-a(\gamma_{j-1},\sigma_{j-1}\big) + f(\gamma_j,\sigma_j)-f(\gamma_{j-1},\sigma_{j-1})\\
&=a(\gamma_j,\sigma_j)\Delta w_{j+1} + M_1w_j + F_j.
\end{aligned}\label{eq:contraction}
\end{equation}
We have $|F_j| \le M_1|w_j|$ and because $w_j\ge0$, 
$$\partial_t w_{j+1} + M'w_{j+1} - a(\gamma_j,\sigma_j)\Delta w_{j+1} \ge 0$$
to conclude $w_{j+1} \ge 0$, or $\sigma_{j+1} \ge \sigma_{j}$. On account of the integral expression of $\gamma_{j+1}$, then $\gamma_{j+1}\ge \gamma_{j}$. On the other hand,  Since $(\gamma_{j+1}(t,x),\sigma_{j+1})$ is again in $K$, we have the same lower bound and H\"older estimate for $a(\gamma_{j+1},\sigma_{j+1})$ and $f(\gamma_{j+1},\sigma_{j+1})$. From the elliptic regularity theory, we can take $M>0$ that is preserved by the iteration. In conclusion, the map $\Phi$ is well-defined.

4. Since the space $C^{\alpha,\alpha/2}(\bar{Q}_T)$ is compact, we have the existence of the solution. We show that $\Phi$ is a contraction in $L^\infty$ norm. From \eqref{eq:contraction} we have by the Maximum principle,
$$||w_{j+1}||_{L^\infty(Q_T)} \le T M_1||w_{j+1}||_{L^\infty(Q_T)}.$$

5. $T= \frac{1}{2} M_1$ only depends on $d$, the ellipticity constant, $||\tilde a||_{C^{\alpha,\alpha/2}}$, and $||\tilde f||_{C^{\alpha,\alpha/2}}$. In particular the latter norms depends on the bounds of the initial data $\gamma^\pm$ and $\sigma^\pm$. By the a priori estimate, the solution constructed has the same lower and upper bounds as before, or we can continue the solution in the same slab size $\frac{1}{2} M_1$ again. This gives the global existence of the unique solution.
\end{proof}

Now, because the elliptic regularity theory In conclusion, the map $\Phi$ is well-defined.



Then, we note that $C^{\alpha,\alpha/2}$ norm of $a$ and $f$ is taken in the same cube $K$ again. are bounded by the same constant, which gives rise to the uniform bound of $\sigma$. This makes $\sigma_{j+1} \in B$.
that will be chosen later. . Thanks to the lower bound of the $\gamma$ and $\sigma$,  Then by {\blue the solvability Theorem} of the linear elliptic problem, $\sigma_{j+1}$ is well-defined in $C^{2+\alpha,1+\alpha/2}(\bar{Q}_T)$ and we have $$||\sigma_{j+1}||_{C^{2+\alpha,1+\alpha/2}(\bar{Q}_T)} \le C,$$ where $C$ only depends on $d$, the ellipticity constant, $||\tilde a||_{C^{\alpha,\alpha/2}}$, and $||\tilde f||_{C^{\alpha,\alpha/2}}$ 

If $(\gamma,\sigma)(t,x) \in K$, then $\tilde a(t,x)\triangleq a(\gamma(t,x),\sigma(t,x))$ and $\tilde f(t,x)\triangleq f(\gamma(t,x),\sigma(t,x))$ in $C^{\alpha,\alpha/2}(\bar{Q}_T)$ and the elliptic regularity theory will give


below. $C>0$ is the constant from the elliptic regularity theory that $||\sigma||_{C^{2+\alpha,1+\alpha/2}(\bar{Q}_T)}$ would be bounded by, for given  This constant only depends on $d$, the ellipticity constant, $||\tilde a||_{C^{\alpha,\alpha/2}}$, and $||\tilde f||_{C^{\alpha,\alpha/2}}$. 



Now, 






\pagebreak


Now, we define the transformation $B:\sigma_j\in C^{2+\alpha,1+\alpha/2}(\bar{Q}_T) \mapsto \sigma_{j+1}\in C^{2+\alpha,1+\alpha/2}(\bar{Q}_T)$ in the following way.

$\underline\gamma \le \gamma^-$, $\underline\sigma \le \sigma^-$, $\overline\gamma \ge 2\gamma^+$, $\overline\sigma \ge 2\sigma^+$. We can choose them 



% Let $V\triangleq \left\{(\sigma(t,x),\gamma(t,x)) \in C^{2+\alpha,1+\alpha/2}(\bar{Q}_\infty) ~|~ 
1. Let $\sigma_{j},\gamma_{j}\in C^{2+\alpha,1+\alpha/2}(\bar{Q}_T)$ be given functions with the property $\gamma^- \le \gamma_j(t,x) \le \gamma^+$ and $\sigma^- \le \sigma_j(t,x) \le \sigma^+$ for all $(t,x)\in Q_T$. Then we define $\sigma_{j+1},\gamma_{j+1}\in C^{2+\alpha,1+\alpha/2}(\bar{Q}_T)$ in such a way that 
$\gamma_{j+1}$ is given by the integral expression \eqref{eq:rep} with $\sigma_{j}(t,x)$ in the integral, or equivalently,  $\partial_t\gamma_{j+1} = \gamma_{j+1}^{\frac{m}{n}}\sigma_j^{\frac{m}{n}} - \gamma_{j+1}$ and $\sigma_{j+1}$ is defined by the solution of the linear equation
 \begin{equation}\label{aux2}
 \begin{aligned}
 \partial_t \sigma_{j+1} &= a(t,x) \Delta \sigma_{j+1} -\sigma_{j+1}\frac{\gamma_{j+1}^{\frac{m}{n}}\sigma_j^{\frac{m}{n}} - \gamma_{j+1}}{\gamma_{j+1}}, \\
 \end{aligned}
 \end{equation}
 subjected to the initial boundary conditions \eqref{initcond}, \eqref{bdrycond}. Denote $f(\gamma_{j+1},\sigma_j)\triangleq\frac{\gamma_{j+1}^{\frac{m}{n}}\sigma_j^{\frac{m}{n}} - \gamma_{j+1}}{\gamma_{j+1}}$ and $\tilde{f}(t,x) = f(\gamma_{j+1}(t,x),\sigma_j(t,x))$. From the integral expression of $\gamma_{j+1}$, $\gamma_{j+1}\ge \sigma^- >0$ so that $\tilde{f}(t,x)$ is bounded from above and $\tilde f (t,x), a(t,x)\in C^{\ell,\ell/2}(\bar{Q}_T)$. Then we can solve the linear equation for $\sigma_{j+1}\in C^{2+\alpha,1+\alpha/2}(\bar{Q}_T)$. 
 
 2. Note that the maximum and minimum of $\sigma_{j+1}$ is attained from the initial data. Therefore, the transformation $R:(\sigma_{j},\gamma_{j})\in C^{2+\alpha,1+\alpha/2}(\bar{Q}_T) \mapsto (\sigma_{j+1},\gamma_{j+1})\in C^{2+\alpha,1+\alpha/2}(\bar{Q}_T)$ is well-defined keeping the bounds
 $$ \sigma^-\le \sigma_{j+1}(t,x),\gamma_{j+1}(t,x) \le \sigma^+\quad \text{for all $(t,x)\in Q_T$}$$
 is kept.
 

 4. Since $f$ is smooth where $\gamma_{j+1}$ is away from $0$, there is a $L>0$
 $$ |f(\gamma_{j+1},\sigma_j)- f(\gamma_{j},\sigma_{j-1})| \le L\Big(|\gamma_{j+1}-\gamma_{j}| + |\sigma_{j}-\sigma_{j-1}|\Big).$$
 On account of the integral expression,
 $$ |f(\gamma_{j+1},\sigma_j)- f(\gamma_{j},\sigma_{j-1})| \le L(n,m)|\sigma_{j}-\sigma_{j-1}|.$$
 
 5. Now, we show that this map is a contraction in $L^\infty(\bar{Q}_{T_*})$-norm provided $T_* < c_0$. {\blue specify this $c_0$.} If $w_{j+1}\triangleq \sigma_{j+1}-\sigma_{j}$, then it satisfies an equation
 $$ \partial_t w_{j+1} = a(t,x)\Delta w_{j+1} - w_{j+1}\tilde{f}(t,x) - w_{j}g(t,x),$$
 where $|g(t,x)| \le \sigma^+ L(n,m)$.
 Since $w_j$ for any $j$ has vanishing initial condition, by the maximum principle, we have
 $|w_{j+1}| \le T_*|w_j g(t,x)|$ and thus if $T_*$ is taken less than $\sigma^+ L(n,m)$, then the map is contraction in $L^\infty$-norm.
 


% Claim 1. Let $R:(\sigma_{j+1},\gamma_{j+1})\in C^{2+\alpha,1+\alpha/2}(\bar{Q}_T) \mapsto (\sigma_{j},\gamma_{j})\in C^{2+\alpha,1+\alpha/2}(\bar{Q}_T)$ be the solution operator of the linear equation
% 
%  This is well-defined by the solvability of the linear parabolic equation \eqref{aux2} with


% \end{proof}




\section{Convergence to the globally asymptotically stable constant state}
We have the formal Energy dissipation equality
\begin{align}
 \partial_t\Big( \frac{u^{n+1}}{\gamma} &+ \frac{m}{n+1-m}\gamma^{\frac{n+1-m}{m}}\Big) + (n+1) \left|\nabla \big(\frac{u^n}{\gamma}\big)\right|^2 \nonumber \\
 &= \div \Big( (n+1)\big(\frac{u^n}{\gamma}\big)\nabla \big(\frac{u^n}{\gamma}\big)\Big) - \frac{1}{\gamma^2}\Big( (u^m)^{\frac{n+1}{m}} - \gamma^{\frac{n+1}{m}}\Big)(u^m-\gamma). \label{eq:energy}
\end{align}
Note that $\displaystyle - \frac{1}{\gamma^2}\Big( (u^m)^{\frac{n+1}{m}} - \gamma^{\frac{n+1}{m}}\Big)(u^m-\gamma)\le0$.
\begin{theorem} Let $\gamma(t,x)$, $\sigma(t,x)$, and $u(t,x)$ be the global classical solution constructed by the Theorem \ref{thm:existence}.
 \begin{enumerate}
  \item $u(t,x) \rightarrow |\Omega|^{-1}$ as $t \rightarrow \infty$, %$|u(t,x) -|\Omega|^{-1}|=o\big(\frac{1}{\sqrt{t}}\big)$  
  \item $\gamma(t,x) \rightarrow |\Omega|^{-m}$ as $t \rightarrow \infty$,
  \item $\sigma(t,x) \rightarrow |\Omega|^{n-m}$ as $t \rightarrow \infty$.
 \end{enumerate}
\end{theorem}
\begin{proof}
 Seen from the energy estimate \eqref{eq:energy}, $||\nabla\sigma||_{L^2(\Omega)}(t)$ is $L^2( \mathbb{R}^+)$ in time, or $||\nabla\sigma||^2_{L^2(\Omega)}(t) = o(\frac{1}{t})$ as $t \rightarrow \infty$. Now from the integral representation \eqref{eq:rep} of $\gamma$, 
 $$ \nabla\gamma^{\frac{n-m}{n}}(\tau,x) = e^{-\frac{n-m}{n}t}\nabla\gamma_0(x)^{\frac{n-m}{n}} + e^{-\frac{n-m}{n}t}\int_0^t \frac{n-m}{n} e^{\frac{n-m}{n}s} \nabla\sigma(s,x)^{\frac{m}{n}} \: ds.$$
 \begin{align*}
  \int_0^T \int_\Omega \left|\nabla\gamma^{\frac{n-m}{n}}(s,x)\right|^2 \: dxds &= \int_\Omega \left|\nabla\gamma_0(x)^{\frac{n-m}{n}}\right|^2 \: dx \int_0^T e^{-\frac{2(n-m)}{n}s}\:ds \\
  &+ \int_0^T \int_0^t \frac{n-m}{n} e^{-\frac{2(n-m)}{n}t}e^{\frac{2(n-m)}{n}s} \int_\Omega\left|\nabla\sigma(s,x)^{\frac{m}{n}}\right|^2 \: dxdsdt\\
  &+ 2\int_0^T \int_0^t \frac{n-m}{n} e^{-\frac{2(n-m)}{n}t}e^{\frac{(n-m)}{n}s} \int_\Omega \nabla\sigma(s,x)^{\frac{m}{n}}\cdot\nabla\gamma_0(x)^{\frac{n-m}{n}} \: dxdsdt\\
  &\le C\Big(||\nabla\sigma^{\frac{m}{n}}||_{L^2(Q_T)}^2 + ||\nabla\gamma_0^{\frac{n-m}{n}}||_{L^2(\Omega)}^2\Big).
 \end{align*}
 Combined with the positive maximum and minimum estimates of $\sigma$ and $\gamma$, $||\nabla\gamma||_{L^2(\Omega)}$ is again an $L^2( \mathbb{R}^+)$ in time. In turn, $u=\sigma^{\frac{1}{n}}\gamma^{\frac{1}{n}}$ and $||\nabla u||_{L^2(\Omega)}$ is  an $L^2( \mathbb{R}^+)$ in time.

 Now, 
 \begin{align*}
  u(t,x) &= u(t,y) + \int_0^1 \frac{d}{ds} u(t,x + s(y-x))\: ds,\\
  \left|\int_\Omega u(t,x) -u(t,y) \: dy\right| &= \Big| ~|\Omega|u(t,x) -1~\Big| = \left|\int_\Omega \int_0^1 \frac{d}{ds} u(t,x + s(y-x))\: dsdy\right|\\
  &\le\int_\Omega |\nabla u||y-x| \: dy \le (\textrm{diam }\Omega)|\Omega|^{\frac{1}{2}} \left( \int_\Omega |\nabla u|^2 \: dy\right)^{\frac{1}{2}} = o(\frac{1}{\sqrt{t}}).
 \end{align*}
 
 From the \eqref{eq:g},
 $$ \partial_t \gamma + \gamma = \Big(|\Omega|^{-1} + o\big(\frac{1}{\sqrt{t}}\big)\Big)^m = |\Omega|^{-m} + o\big(\frac{1}{\sqrt{t}}\big).$$

\end{proof}


\section{Radial self-similar solutions}
\subsection{Fundamental solution}

\section{Relaxation limit to the effective equation}
\subsection{energy structure}

\appendix
\section{Solvability of Linear Neumann boundary value problem}
$Q_T\triangleq (0,T)\times\Omega$, $S_T\triangleq [0,T]\times\partial\Omega$.

\begin{theorem}
 $C^{\ell,\ell/2}(\bar{Q}_T)$ is the Banach space of functions $u(x,t)$ that are continuous in $\bar{Q}_T$, together with all derivatives of the form $D^r_t D^s_x$ for $2r+s<\ell$.
 \begin{equation}
  Lu\triangleq \partial_t u - \sum_{i,j=1}^d a_{ij}(t,x) \frac{\partial^2 u}{\partial x^i \partial x^j} + \sum_{i=1}^d a_i(t,x) \frac{\partial u}{\partial x^i} + a(t,x) u.
 \end{equation}
 We consider the Neumann boundary value problem
 \begin{equation} \label{eq:lin_para}
  \begin{aligned}
   Lu &= f(t,x), \quad Q_T\\
   \nabla u\cdot \mathbf{n} &= 0, \quad S_T\\
   u(0,x) &= u_0(x).
  \end{aligned}
 \end{equation}
 Suppose $\ell>0$ is a nonintegral number, $\partial\Omega \in C^{\ell+2}$, the coefficients in the operator belong to the class $C^{\ell,\ell/2}(\bar{Q}_T)$. Then for any $f \in C^{\ell,\ell/2}(\bar{Q}_T)$, $u_0\in C^{\ell+2}(\bar\Omega)$ with the {\blue compatibility conditions}, \eqref{eq:lin_para} has a unique solution from the class $C^{\ell+2,\ell/2+1}(\bar{Q}_T)$, with
 \begin{equation}
  ||u||_{C^{\ell+2,\ell/2+1}(\bar{Q}_T)} \le C\Big( ||f||_{C^{\ell,\ell/2}(\bar{Q}_T)} + ||u_0||_{C^{\ell+2,\ell/2+1}(\bar\Omega)}\Big). \label{regularity}
 \end{equation}
\end{theorem}





\begin{thebibliography}{10}

% \bibitem{bertsch_effect_1991}
% {\sc M.~Bertsch, L.~Peletier, and S.~Verduyn~Lunel},
% The effect of temperature dependent viscosity on shear flow of  incompressible fluids,
% {\it SIAM J. Math. Anal.} {\bf 22 } (1991), 328--343.
% 
% \bibitem{clifton_rev_1990}
% {\sc R.J. Clifton},  High strain rate behaviour of metals,
% % {\it Applied Mechanics Review}
% {\it Appl. Mech. Rev.}
% {\bf 43} (1990), S9-S22.
% 
% \bibitem{DH_1983}
% {\sc C.M. Dafermos and L.~Hsiao},
% Adiabatic shearing of incompressible fluids with temperature-dependent viscosity.
% {\it Quart.  Applied Math.} {\bf 41} (1983), 45--58.
% 
% % \bibitem{fenichel_persistence_1972}
% % {\sc N.~Fenichel},
% % Persistence and smoothness of invariant manifolds for  flows,
% % {\it Indiana Univ. Math. J.} {\bf 21} (1972) 193--226.
% 
% \bibitem{fenichel_asymptotic_1974}
% {\sc N.~Fenichel},
% Asymptotic stability with rate conditions,
% {\it Indiana Univ. Math. J.} {\bf 23} (1974) 1109--1137.
% 
% \bibitem{fenichel_asymptotic_1977}
% {\sc N.~Fenichel},
% Asymptotic stability with rate conditions \textrm{II},
% {\it Indiana Univ. Math. J.} {\bf 26} (1977) 81--93.
% 
% \bibitem{fenichel_geometric_1979}
% {\sc N.~Fenichel},
% Geometric singular perturbation theory for ordinary differential equations,
% {\it J. Differ. Equations} {\bf 31} (1979), 53--98.
% 
% %\bibitem{FM87}
% %{\sc C.~Fressengeas and A.~Molinari},
% %Instability and localization of plastic flow in shear at high strain rates,
% %{\it J.  Mech. Physics of Solids} {\bf 35} (1987), 185--211.
% 
% \bibitem{HPS_1977}
% {\sc M.W. Hirsch, C.C. Pugh, and M. Shub},
% {\it Invariant Manifolds}, LNM {\bf 583}, (Springer-Verlag, New York/Heidelberg/Berlin 1977)
% 
% \bibitem{HN77}
% {\sc J.W.~Hutchinson and K.W.~Neale},
% Influence of strain-rate sensitivity on necking under uniaxial tension,
% {\it  Acta Metallurgica} {\bf 25} (1977), 839-846.
% 
% \bibitem{KOT14}
% {\sc Th.~Katsaounis, J.~Olivier, and A.E.~Tzavaras},
% Emergence of coherent localized structures in shear deformations of temperature dependent fluids,
% {\it Archive for Rational Mechanics and Analysis} {\bf 224} (2017), 173--208.
% 
% \bibitem{KT09}
% {\sc Th. Katsaounis and A.E.~Tzavaras},
% Effective equations for localization and shear band formation,
% {\it SIAM J. Appl. Math.}  {\bf 69} (2009), 1618--1643.
% 
% \bibitem{KLT_2016}
% {\sc Th. Katsaounis, M.-G. Lee, and A.E. Tzavaras},
% Localization in inelastic rate dependent shearing deformations,
% {\it J. Mech. Phys. of Solids} {\bf 98} (2017), 106--125.
% 
% \bibitem{KUEHN_2015}
% {\sc C.~ Kuehn}, 
% {\it Multiple time scale dynamics}, Applied Mathematical Sciences, Vol. {\bf 191} (Springer Basel 2015).
% 
% \bibitem{LT16}
% {\sc M.-G.~Lee and A.E.~Tzavaras},
% Existence of localizing solutions in plasticity via the geometric singular perturbation theory,
% {\it Siam J. Appl. Dyn. Systems} {\bf 16} (2017), 337--360.
% 
% \bibitem{KLT_HYP2016}
% {\sc M.-G. Lee, Th. Katsaounis, and A.E. Tzavaras},
% Localization of Adiabatic Deformations in Thermoviscoplastic Materials, In Proceedings of the 16th International Conference on Hyperbolic Problems: Theory, Numerics, Applications (HYP2016), to appear.
% 
% %
% % \bibitem{jones_geometric_1995}
% % {\sc C.~K. R.~T. Jones},
% % Geometric singular perturbation theory, in {\it Dynamical systems}, LNM {\bf 1609} (Springer Berlin Heidelberg 1995) 44--118.
% %
% %
% 
% %
% % % \bibitem{perko_differential_2001}
% % % {\sc L.~Perko},
% % % {\it Differential equations and dynamical systems 3rd. ed.}, TAM {\bf 7} (Springer-Verlag New York 2001).
% %
% 
% \bibitem{shawki_shear_1989}
% {\sc T.G. Shawki and R.J. Clifton},
% Shear band formation in thermal viscoplastic materials,
% % {\it Mechanics of Materials}
% {\it Mech. Mater.}
% {\bf 8 } (1989), 13--43.
% 
% \bibitem{Sz1991}
% {\sc P.~Szmolyan},
% Transversal heteroclinic and homoclinic orbits in singular perturbation problems,
% {\it J. Differ. Equations}
% {\bf 92} (1991), 252--281.
% 
% \bibitem{Tz_1986}
% {\sc A.E. Tzavaras},
% Shearing of materials exhibiting thermal softening or temperature dependent viscosity,
% {\em Quart.  Applied Math.} {\bf 44} (1986), 1--12.
% 
% \bibitem{Tz_1987}
% {\sc A.E. Tzavaras},
% Effect of thermal softening in shearing of strain-rate dependent materials.
% {\em Archive for Rational Mechanics and Analysis}, {\bf 99} (1987), 349--374.
% 
% \bibitem{tzavaras_plastic_1986}
% {\sc A.E. Tzavaras},
% Plastic shearing of materials exhibiting strain hardening or strain softening,
% % {\it Archive for Rational Mechanics and  Analysis}
% {\it Arch. Ration. Mech. Anal.}
% {\bf 94} (1986), 39--58.
% 
% %\bibitem{tzavaras_strain_1991}
% %{\sc A.E. Tzavaras},
% %Strain softening in viscoelasticity of the rate type.
% %{\it J. Integral Equations Appl.} {\bf  3}  (1991), 195--238.
% 
% \bibitem{tzavaras_nonlinear_1992}
% %\leavevmode\vrule height 2pt depth -1.6pt width 23pt,
% {\sc A.E. Tzavaras},
% Nonlinear analysis techniques for shear band formation at high strain-rates,
% % {\it Applied Mechanics Reviews}
% {\it Appl. Mech. Rev.}
% {\bf  45} (1992), S82--S94.
% 
% 
% 
% %
% % \bibitem{clifton_critical_1984}
% % {\sc R.~J. Clifton, J.~Duffy, K.~A. Hartley, and T.~G. Shawki},
% % On critical conditions for shear band formation at high strain rates.
% % % {\it Scripta Metallurgica}
% % {\it Scripta. Metall. Mater.}
% % {\bf 18} (1984), 443--448.
% %
% 
% 
% %
% % \bibitem{freistuhler_spectral_2002}
% % {\sc H.~Freistühler and P.~Szmolyan},
% % {Spectral stability of small shock waves},
% % {\it Arch. Ration. Mech. Anal.}
% % {\bf 164} (2002), 287--309.
% % %   \href{http://dx.doi.org/10.1007/s00205-002-0215-8}{doi:\nolinkurl{10.1007/s00205-002-0215-8}},
% % %   \url{http://dx.doi.org/10.1007/s00205-002-0215-8}.
% % \bibitem{fressengeas_instability_1987}
% % {\sc C.~Fressengeas, A.~Molinari},
% % {Instability and localization of plastic flow in shear at high strain rates},
% % {\it J. Mech. Phys. of Solids}
% % {\bf 35} (1987), 185--211.
% %
% % \bibitem{gasser_geometric_1993}
% % {\sc I.~Gasser and P.~Szmolyan},
% % {A geometric singular perturbation analysis of detonation and deflagration waves},
% % {\it {SIAM} J. Math. Anal.}
% % {\bf 24} (1993), 968--986.
% % %   \href{http://dx.doi.org/10.1137/0524058}{doi:\nolinkurl{10.1137/0524058}},
% % %   \url{http://dx.doi.org/10.1137/0524058}.
% % \bibitem{ghazaryan_traveling_2007}
% % {\sc A.~Ghazaryan, P.~Gordon, and C.~K. R.~T. Jones},
% % {Traveling waves in porous media combustion: uniqueness of waves for small thermal diffusivity},
% % {\it J. Dyn. Differ. Equ.}
% % {\bf 19} (2007), 951--966.
% % %   \href{http://dx.doi.org/10.1007/s10884-007-9079-9}{doi:\nolinkurl{10.1007/s10884-007-9079-9}},
% % %   \url{http://dx.doi.org/10.1007/s10884-007-9079-9}.
% %
% %
% 
% % \bibitem{MC_1987}
% % {\sc A.~Molinari and R.~J. Clifton},
% % Analytical characterization of shear localization in thermoviscoplastic materials,
% % {\it Journal of Applied Mechanics}
% % {\it J. Appl. Mech.}
% % {\bf 54} (1987), 806--812.
% %
% %
% % \bibitem{jones_geometric_1995}
% % {\sc C.~K. R.~T. Jones},
% % Geometric singular perturbation theory, in {\it Dynamical systems}, LNM {\bf 1609} (Springer Berlin Heidelberg 1995) 44--118.
% %
% %
% %
% %
% %
% %
% 
% %
% % \bibitem{KUEHN_2015}
% % {\sc C.~ Kuehn},
% % {\it Multiple time scale dynamics}, Applied Mathematical Sciences, Vol. {\bf 191} (Springer Basel 2015).
% %
% 
% %
% % \bibitem{perko_differential_2001}
% % {\sc L.~Perko},
% % {\it Differential equations and dynamical systems 3rd. ed.}, TAM {\bf 7} (Springer-Verlag New York 2001).
% %
% %
% 
% 
% 
% %
% % \bibitem{shawki_energy_1994}
% % {\sc T.~G. Shawki},
% % {An Energy Criterion for the Onset of Shear Localization in Thermal Viscoplastic Materials, Part II: Applications and Implications}, {\it ASME. J. Appl. Mech.}
% % {\bf 61} (1994), 538--547.
% %
% %
% % \bibitem{SS_2004}
% % {\sc S. Schecter and P. Szmolyan}
% % Composite waves in the Dafermos regularization.
% % {\it J. Dynamics Diff. Equations} {\bf 16} (2004), 847-867.
% %
% 
% %
% % \bibitem{wiggins_normally_1994}
% % {\sc S.~Wiggins},
% % {\it Normally hyperbolic invariant manifolds in dynamical  systems}, AMS {\bf 105} (Springer-Verlag New York 1994).
% %
% \bibitem{wiggins_normally_1994}
% {\sc S.~Wiggins}, 
% {\it Normally hyperbolic invariant manifolds in dynamical  systems}, AMS {\bf 105} (Springer-Verlag New York 1994).
% 
% \bibitem{wright_survey_2002}
% {\sc T.W. Wright},
% {\it The Physics and Mathematics of Shear Bands.} (Cambridge Univ. Press 2002).
% %
% % \bibitem{xiao_stability_2003}
% % {\sc L.~Xiao-Biao and S.~ Schecter},
% % {Stability of self-similar solutions of the {D}afermos regularization of a system of conservation laws},
% % {SIAM J. Math. Anal.}
% % {\bf 35} (2003), 884--921.
% 
% %\bibitem{WF83}
% %{\sc F.H. Wu and L.B. Freund},
% %Deformation trapping due to thermoplastic instability in one-dimensional wave propagation,
% %{\it J. Mech. Phys. of Solids} {\bf  32} (1984), 119-132.
% 
% \bibitem{zener_effect_1944}
% {\sc C.~Zener and J.~H. Hollomon},
% Effect of strain rate upon plastic flow of steel,
% % {\it  Journal of Applied Physics}
% {\it J. Appl. Phys.}
% {\bf 15} (1944), 22--32.

\end{thebibliography}
\end{document}