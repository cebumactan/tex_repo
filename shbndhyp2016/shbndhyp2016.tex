%%%%%%%%%%%%%%%%%%%%%%%%%%%%%%%%%%%%%%%%%%%%%%%
%
%    Localization in the adiabatic process of shear deformation for Hyp2016
%
%                                                      by
%
%                                       Min-Gi Lee   
%
%                                          version Sep 2016
%
%
%%%%%%%%%%%%%%%%%%%%%%%%%%%%%%%%%%%%%%%%%%%%%%%




% RECOMMENDED %%%%%%%%%%%%%%%%%%%%%%%%%%%%%%%%%%%%%%%%%%%%%%%%%%%
\documentclass[graybox]{svmult}

% choose options for [] as required from the list
% in the Reference Guide

\usepackage{mathptmx}       % selects Times Roman as basic font
\usepackage{helvet}         % selects Helvetica as sans-serif font
\usepackage{courier}        % selects Courier as typewriter font
\usepackage{type1cm}        % activate if the above 3 fonts are
                            % not available on your system
%
\usepackage{makeidx}         % allows index generation
\usepackage{graphicx}        % standard LaTeX graphics tool
                             % when including figure files
\usepackage{multicol}        % used for the two-column index
\usepackage[bottom]{footmisc}% places footnotes at page bottom


\usepackage{amsmath}
\usepackage{amssymb}
\usepackage{color,subfigure}
\def\red{\color{red}}
\def\blue{\color{blue}}
%\usepackage{verbatim}
% \usepackage{alltt}
%\usepackage{kotex}

%\usepackage{showkeys}
%%%%%%%%%%%%%% MY DEFINITIONS %%%%%%%%%%%%%%%%%%%%%%%%%%%

\def\tr{\,\textrm{tr}\,}
\def\div{\,\textrm{div}\,}
\def\sgn{\,\textrm{sgn}\,}

\def\th{\tilde{h}}
\def\tx{\tilde{x}}
\def\tk{\tilde{\kappa}}


\def\bg{{\bar{\gamma}}}
\def\bv{{\bar{v}}}
\def\bth{{\bar{\theta}}}
\def\bs{{\bar{\tau}}}
\def\bu{{\bar{u}}}
\def\bph{{\bar{\varphi}}}


\def\tg{{\tilde{\gamma}}}
\def\tv{{\tilde{v}}}
\def\tth{{\tilde{\theta}}}
\def\ts{{\tilde{\tau}}}
\def\tu{{\tilde{u}}}
\def\tph{{\tilde{\varphi}}}

\def\dtg{{\dot{\tilde{\gamma}}}}
\def\dtv{{\dot{\tilde{v}}}}
\def\dtth{{\dot{\tilde{\theta}}}}
\def\dts{{\dot{\tilde{\tau}}}}
\def\dtu{{\dot{\tilde{u}}}}
\def\dtph{{\dot{\tilde{\varphi}}}}

\def\dpp{\dot{p}}
\def\dqq{\dot{q}}
\def\drr{\dot{r}}
\def\dss{\dot{s}}

\def\ta{{\tilde{a}}}
\def\tb{{\tilde{b}}}
\def\tc{{\tilde{c}}}
\def\td{{\tilde{d}}}

\def\BO{{\mathcal{O}}}
\def\lio{{\mathcal{o}}}
\def\R{{\mathbb{R}}}


\def\bx{\bar{x}}
\def\bm{\bar{\mathbf{m}}}
\def\K{\mathcal{K}}
\def\E{\mathcal{E}}
\def\del{\partial}
\def\eps{\varepsilon}

% see the list of further useful packages
% in the Reference Guide

\makeindex             % used for the subject index
                       % please use the style svind.ist with
                       % your makeindex program

%%%%%%%%%%%%%%%%%%%%%%%%%%%%%%%%%%%%%%%%%%%%%%%%%%%%%%%%%%%%%%%%%%%%%%%%%%%%%%%%%%%%%%%%%

\begin{document}

\title*{Localization of Adiabatic Deformations in Thermoviscoplastic Materials }
% Use \titlerunning{Short Title} for an abbreviated version of
% your contribution title if the original one is too long
\author{Min-Gi Lee, Theodoros Katsaounis and Athanasios E. Tzavaras}
% Use \authorrunning{Short Title} for an abbreviated version of
% your contribution title if the original one is too long
\institute{Min-Gi Lee \at King Abdullah University of Science and Technology (KAUST), Computer, Electrical and Mathematical Sciences \& Engineering Division, KAUST, Thuwal, Saudi Arabia, \\ \email{mingi.lee@kaust.edu.sa}
\and Theodoros Katsaounis \at King Abdullah University of Science and Technology (KAUST), Computer, Electrical and Mathematical Sciences \& Engineering Division, KAUST, Thuwal, Saudi Arabia and  \at IACM, FORTH, Heraklion, Greece,  \\ \email{theodoros.katsaounis@kaust.edu.sa}
\and Athanasios Tzavaras \at King Abdullah University of Science and Technology (KAUST), Computer, Electrical and Mathematical Sciences \& Engineering Division, KAUST, Thuwal, Saudi Arabia,\\  \email{athanasios.tzavaras@kaust.edu.sa}}
%
% Use the package "url.sty" to avoid
% problems with special characters
% used in your e-mail or web address
%
\maketitle

\abstract{ We study an instability occurring at high strain-rate deformations,  induced by thermal softening properties of metals, and leading to the formation of shear bands. 
We consider adiabatic shear deformations of thermoviscoplastic materials and establish the existence of a family of focusing self-similar solutions that capture this instability.
The self-similar solutions emerge as the net response resulting from the  competition between Hadamard instability and viscosity.
Their existence is turned into a problem of constructing a heteroclinic orbit for an associated dynamical system, which is achieved with the help of geometric singular perturbation theory. }


\section{Introduction}
Shear bands are regions of intensely localized shear deformation appearing when metals are deforming at high strain-rates.  This type of material instability has attracted attention in the mechanics and mathematical literature, \cite{FM87,SC89,ZH_1944,dafermos_adiabatic_1983,Tz_1986,tzavaras_nonlinear_1992}. In the mechanics literature, such material instability,  is
often called \emph{Hadamard instability} and is associated with an ill-posed initial value problem. However, it should be noted that although Hadamard instability indicates the catastrophic growth of oscillations around a mean state, coherent localized structures, \emph{the shear bands}, emerge in an orderly fashion. This is a highly nonlinear phenomenon resulting from the competition between Hadamard instability and viscosity. 


Under isothermal conditions, metals, in general, strain harden and exhibit a stable response. As the deformation speed increases, the heat produced by the plastic work causes an increase in the temperature. For certain metals, the tendency for thermal softening may outweigh the tendency for strain hardening and deliver net softening. A destabilizing feedback mechanism is then induced, which operates as follows (see \cite{CDHS}): Nonuniformities in the strain rate result in nonuniform heating. If heat diffusion is too weak to equalize the temperatures, the initial nonuniformities in the strain rate are, in turn, amplified. This mechanism tends to localize the total deformation into narrow regions(shear bands). On the other hand, there is opposition to this process by ``viscous effects'' induced by strain-rate sensitivity. The outcome of the competition depends mainly on the relative weights of thermal softening, strain hardening, and strain-rate sensitivity, as well as the loading circumstances. This qualitative scenario is widely accepted as the mechanism of shear band formation. 


We work with a simple model that captures the mechanism of shear band formation : we consider the  adiabatic shear deformation  of a  thermoviscoplastic material that occupies the slab between two parallel plates. The relevant quantities are: the velocity $v(t,x)$, the shear strain $\gamma(t,x)$, 
 the shear strain rate $u(t,x)$, the shear stress $\tau(t,x)$,  and the temperature $\theta(t,x)$. The system of equations describing the motion takes the form
\begin{equation} \label{eq:A}\tag{A}
\begin{aligned}
 \gamma_t &= u \quad \text{(kinematic compatibility)}, 	\\
 v_t &= \tau_x \quad \text{(momentum conservation)}, 	\\
 \theta_t &= \tau u \quad \text{(energy equation)},	\\
 \tau &=\tau(\theta,\gamma,u) \quad \text{(constitutive law)},			
\end{aligned}
\end{equation}
with $(t,x) \in \mathbb{R}^+\times \mathbb{R}$. In terms of classification, the model \eqref{eq:A} belongs to the framework of one-dimensional thermo-visco-elasticity. 
It is also instructive to interpret $\eqref{eq:A}_4$ as a constitutive law for thermo-visco-plastic materials 
viewing $\gamma \equiv \gamma_p$ as the plastic strain, see the hierarchy of models in \cite{KT09,tzavaras_nonlinear_1992}.
This context suggests the terminology: the material exhibits thermal softening at $(\theta,\gamma,u)$ when $\tau_\theta(\theta,\gamma,u)<0$, strain hardening if $\tau_\gamma(\theta,\gamma,u)>0$, and strain softening if $\tau_\gamma(\theta,\gamma,u)<0$.

In this study we focus on a constitutive hypothesis in the form of a \emph{power} law 
\begin{equation}
 \tau = \varphi(\theta,\gamma)u^n = \theta^{-\alpha}\gamma^m u^n, \label{eq:power} 
\end{equation}
where $n$ is the strain-rate sensitivity which is assumed to be very small $0<n\ll1$, $\alpha$ measures the degree of thermal softening, while $m$ measures the degree of strain hardening. We further introduce two subclasses of \eqref{eq:A}, where $\varphi (\theta, \gamma)$ is independent 
of either $\theta$ or $\gamma$, respectively. These are: the strain independent model $(B)$ ($m = 0$)  consising of 
\begin{equation} \label{eq:B}\tag{B}
 v_t = \tau_x, \quad \theta_t = \tau u, \quad \tau = \mu(\theta)u^n 	\, , 	
\end{equation}
and the temperature independent model $(C)$ ($\alpha = 0$) consisting of
\begin{equation} \label{eq:C}\tag{C}
 \gamma_t = u, \quad  v_t = \tau_x, \quad \tau = \varphi(\gamma)u^n.			
\end{equation}
Model \eqref{eq:A} with power law \eqref{eq:power} admits a special class of solutions describing uniform shearing,  which can be written explicitly 
\begin{equation}
\label{USS}
 \gamma_s = t+\gamma_0, \ v_s=x, \ \theta_s=\left[\theta_0^{1+\alpha}+\frac{1+\alpha}{1+m}\left[ \gamma_s^{1+m} - \gamma_0^{1+m}\right] \right]^{\frac1{1+\alpha}}, \  \tau_s = \theta_s^{-\alpha}\gamma_s^m \, ,
\end{equation}
where $\gamma_0, \ \theta_0$ denote  initial values of shear strain and temperature respectively. 

The linear stability analysis, \cite{FM87} around the \emph{uniform shearing solutions} \eqref{USS} indicates that system \eqref{eq:A} becomes unstable in the regime 
$q:=-\alpha + m + n < 0$ while it is asymptotically stable in the complementary region $q > 0$. Using the Champan-Enskog expansion and a relaxation theory approach, the authors in \cite{KT09}, obtained an effective equation for the shear strain rate that changes type from forward parabolic when $q>0$ to backward parabolic when $q<0$. 

Non-linear stability of model \eqref{eq:B} ($m=0$) in the regime $q>0$ has been studied in \cite{dafermos_adiabatic_1983} when $n=1$, and in \cite{Tz_1986} for $n\ne1$. System \eqref{eq:B} has an interpretation that the model describes a fluid with temperature dependent viscosity $\mu(\theta)$ in the rect-linear shear motion. Similar result for the problem \eqref{eq:C} ($\alpha=0$) in the regime $q>0$  is obtained in \cite{tzavaras_nonlinear_1992}. For the problem \eqref{eq:B} in the regime $q<0$, the failure of the asymptotic stability is treated in \cite{bertsch_effect_1991} when $n=1$ and for $n\ne1$ in \cite{KT09}.  

The main result of this paper is the construction of a family of self-similar solutions of {\it focusing type} to models \eqref{eq:B}, and \eqref{eq:C} that capture the underlying instability. It provides a survey of several related results concerning the construction of focusing solutions that
have recently appeared in the literature \cite{KOT14,KLT_2016,LT16}. The exposition is organized mostly around the general system of three equations 
 \eqref{eq:A}. The last step of the construction of focusing self-similar solutions for the general power law \eqref{eq:power} is work in progress \cite{LT16_2}. 

The paper is organized as follows: Section 2 contains the main results of this work. The existence of focusing type self similar solutions for systems \eqref{eq:B} and \eqref{eq:C} is stated precisely in Theorems \ref{thm1} and \ref{thm2}. Figures \ref{fig:theta}, \ref{fig:u}, \ref{fig:v} and \ref{fig:sigma} depict the typical behavior of such focusing solutions.  
The existence of focusing self-similar solutions is turned into a problem of constructing a heteroclinic orbit for an associated dynamical system. 
The main tool of proving the latter is the theory of geometric singular perturbations, \cite{fenichel_geometric_1979} which is  discussed briefly in Section 3. 


%%%%%%%%%%%%%%%%%%%%%%
%% SECTION
%%%%%%%%%%%%%%%%%%%%%%
\section{Main results}
\subsection{Self-similar structure}
We investigate the scale invariance property of the system (A), and consequently that of (B) and (C) too.
Suppose $(\gamma,u,v,\theta,\tau)$ is a solution of system (A). Then a rescaled version of it $(\gamma_\rho,u_\rho,v_\rho,\theta_\rho,\tau_\rho)$ given by
\begin{align*}
 \gamma_\rho(t,x) &= \rho^a\gamma(\rho^{-1}t,\rho^\lambda x), &
 v_\rho(t,x) &= \rho^bv(\rho^{-1}t,\rho^\lambda x),\\
 \theta_\rho(t,x) &= \rho^c\theta(\rho^{-1}t,\rho^\lambda x), &
 \tau_\rho(t,x) &= \rho^d\tau(\rho^{-1}t,\rho^\lambda x),\\
 u_\rho(t,x) &= \rho^{b+\lambda}\gamma(\rho^{-1}t,\rho^\lambda x),
\end{align*}
is also a solution of \eqref{eq:A}  provided that 
\begin{align*}
 a&= \frac{2+2\alpha-n}{D} + \frac{2+2\alpha}{D}\lambda =: a_0 + a_1 \lambda, \\
 b&=\frac{1+m}{D} + \frac{1+m+n}{D}\lambda =: b_0 + b_1\lambda,\\
 c&=\frac{2(1+m)}{D} + \frac{2(1+m+n)}{D}\lambda =: c_0 + c_1\lambda, \\
 d&=\frac{-2\alpha + 2m +n}{D} + \frac{-2\alpha+2m+2n}{D}\lambda =: d_0 + d_1\lambda,
\end{align*}
for each $\lambda \in \mathbb{R}$, where  $D = 1+2\alpha-m-n$. 
Motivated by the scale invariance property parametrized by $\lambda$, we look for the solutions of the form 
\begin{align*}
 \gamma(t,x) &= t^a\Gamma(t^\lambda x), & v(t,x) &= t^b V(t^\lambda x), &\theta(t,x) &= t^c \Theta(t^\lambda x),\\
 \tau(t,x) &= t^d \Sigma(t^\lambda x), & u(t,x) &= t^{b+\lambda} U(t^\lambda x),
\end{align*}
and set $\xi = t^\lambda x$. In this format, $\lambda>0$ accounts for the focusing behavior as time increases,  whereas $\lambda<0$ accounts for the de-focusing behavior. This family includes the uniform shearing solution at $\lambda = -\frac{1+m}{2(1+\alpha)}$. Since we are interested in the focusing solutions, we consider $\lambda>0$ in the rest of the paper.


Using this ansatz to the system \eqref{eq:A} we obtain a  system of ordinary differential and algebraic equations that $\big(\Gamma(\xi), V(\xi), \Theta(\xi), \Sigma(\xi), U(\xi)\big)$ satisfies, 
\begin{equation}
\begin{aligned}
 a \Gamma(\xi) + \lambda \xi \Gamma'(\xi) &= U(\xi),\\
 b V(\xi) + \lambda \xi V'(\xi) &= \Sigma'(\xi),\\
 c \Theta(\xi) + \lambda \xi \Theta'(\xi)&=\Sigma(\xi) U(\xi),\\
 \Sigma(\xi) &= \Theta(\xi)^{-\alpha} \Gamma(\xi)^m U(\xi)^n,\\
 V'(\xi)&=U(\xi).
\end{aligned} \label{eq:ss-odes}
\end{equation}

\subsection{Main theorem}

We first state the existence of two parameters family of solutions for \eqref{eq:B}  where $m=0$. See \cite{LT16_2} for the detailed discussion.
\begin{theorem} \label{thm1}
Let $\alpha,n>0$, $\alpha\ne2n+1$ the given material parameters and fix $U_0>0$ and $\Theta_0>0$. Suppose that
\begin{equation} \label{eq:restriction}
 \frac{2}{1+2\alpha-n} < \frac{U_0^{1+n}}{\Theta_0^{1+\alpha}} < \frac{2}{1+n},
\end{equation}
$-\alpha+n<0$, and $n$ is sufficiently small. Then there is a focusing self-similar solution to system \eqref{eq:B} of the form
\begin{equation*}
\begin{aligned}
 v(t,x) &= (t+1)^b V((t+1)^\lambda x), &\theta(t,x) &= (t+1)^c \Theta((t+1)^\lambda x),\\
 \tau(t,x) &= (t+1)^d \Sigma((t+1)^\lambda x), & u(t,x) &= (t+1)^{b+\lambda} U((t+1)^\lambda x)
\end{aligned}
\end{equation*}
where the focusing rate is 
\begin{equation}
 \lambda = \frac{1+2\alpha-n}{2+2n}\frac{U_0^{1+n}}{\Theta_0^{1+\alpha}} - \frac{2}{2+2n}>0. \label{eq:lambda}
\end{equation}
Furthermore, the self-similar profile $\big(V(\xi),\Theta(\xi),\Sigma(\xi),U(\xi)\big), \ \xi = (t+1)^\lambda x$,  has the  following properties :
 \begin{enumerate}
  \item[(i)] Satisfies the boundary condition at $\xi=0$,
    \begin{equation*}
    {V}(0) = \Theta_\xi(0)=\Sigma_\xi(0) = {U}_\xi(0)=0, \quad U(0)=U_0, \Theta(0)=\Theta_0.
  \end{equation*}
  \item[(ii)] Its asymptotic behavior as $\xi \rightarrow 0$ is given by 
  \begin{equation} \label{eq:ss_asymp0}
  \begin{aligned}
%     \Gamma(\xi) &= \frac{1}{a}U(0) + \Gamma^{''}(0)\frac{\xi^2}{2} + o(\xi^2), & \Gamma^{''}(0)&<0,\\
    \Theta(\xi) &= \Theta(0) + \Theta^{''}(0)\frac{\xi^2}{2} + o(\xi^2), & \Theta^{''}(0)&<0,\\
    \Sigma(\xi) &= \Theta(0)^{-\alpha}{U(0)^n}+ \Sigma^{''}(0)\frac{\xi^2}{2} + o(\xi^2), & \Sigma^{''}(0)&>0, \\
    U(\xi) &= U(0) + U^{''}(0)\frac{\xi^2}{2} + o(\xi^2), & U^{''}(0)&<0,\\
    V(\xi) &= U(0)\xi + U^{''}(0)\frac{\xi^3}{6} + o(\xi^3), & U^{''}(0)&<0.
  \end{aligned}
  \end{equation}
  \item[(iii)] Its asymptotic behavior as $\xi \rightarrow \infty$ is given by
  \begin{equation} \label{eq:ss_asymp1}
  \begin{aligned}
%     \Gamma(\xi) &= O(\xi^{-\frac{1+\alpha}{\alpha-n}}), & 
    V(\xi) &= O(1), &    \Theta(\xi) &= O(\xi^{-\frac{1+n}{\alpha-n}}),\\
   \Sigma(\xi) &= O(\xi), &   U(\xi) &= O(\xi^{-\frac{1+\alpha}{\alpha-n}}).
  \end{aligned}
  \end{equation}
 \end{enumerate}
\end{theorem}

In case of system \eqref{eq:C} , where  $\alpha=0$,  there exists a two parameter family of solutions,  see \cite{KLT_2016}, \cite{LT16} for the detailed discussion. 
\begin{theorem} \label{thm2}
Let $-1\le m<0$ and $n>0$, $m+n\ne \frac{1}{2}$ the given material parameters and fix $U_0>0$ and $\Gamma_0>0$. Suppose that
\begin{equation*}
 \frac{2-n}{1-m-n} \;<\; \frac{U_0}{\Gamma_0} \;<\; \frac{2-n}{1+m+n} \, ,
\end{equation*}
$m+n<0$, and $n$ is sufficiently small. Then there is a focusing self-similar solution to system \eqref{eq:C} of the form
\begin{equation*}
\begin{aligned}
 \gamma(t,x) &= (t+1)^a\Gamma((t+1)^\lambda x), & v(t,x) &= (t+1)^b V((t+1)^\lambda x), \\
 \tau(t,x) &= (t+1)^d \Sigma((t+1)^\lambda x), & u(t,x) &= (t+1)^{b+\lambda} U((t+1)^\lambda x),
\end{aligned}
\end{equation*}
where the focusing rate is
\begin{equation}
 \lambda = \frac{1-m-n}{2}\Big(\frac{U_0}{\Gamma_0} - \frac{2-n}{1-m-n}\Big)>0. \label{eq:lambdaC}
\end{equation}
Furthermore, the self-similar profile $\big(V(\xi),\Theta(\xi),\Sigma(\xi),U(\xi)\big), \ \xi = (t+1)^\lambda x$,  has the  following properties :
 \begin{enumerate}
  \item[(i)] Satisfies the boundary condition at $\xi=0$,
    \begin{equation*}
    {V}(0) = \Gamma_\xi(0) = \Sigma_\xi(0) = {U}_\xi(0)=0, \quad U(0)=U_0, \Gamma(0)=\Gamma_0.
  \end{equation*}
  \item[(ii)] Its asymptotic behavior as $\xi \rightarrow 0$ is given by 
  \begin{equation} \label{eq:ss_asympC0}
  \begin{aligned}
    \Gamma(\xi) &= \frac{1}{a}U(0) + \Gamma^{''}(0)\frac{\xi^2}{2} + o(\xi^2), & \Gamma^{''}(0)&<0,\\
%     \Theta(\xi) &= \Theta(0) + \Theta^{''}(0)\frac{\xi^2}{2} + o(\xi^2), & \Theta^{''}(0)&<0,\\
    \Sigma(\xi) &= \Gamma(0)^{m}{U(0)^n}+ \Sigma^{''}(0)\frac{\xi^2}{2} + o(\xi^2), & \Sigma^{''}(0)&>0, \\
    U(\xi) &= U(0) + U^{''}(0)\frac{\xi^2}{2} + o(\xi^2), & U^{''}(0)&<0,\\
    V(\xi) &= U(0)\xi + U^{''}(0)\frac{\xi^3}{6} + o(\xi^3), & U^{''}(0)&<0.
  \end{aligned}
  \end{equation}
  \item[(iii)] Its asymptotic behavior as $\xi \rightarrow \infty$ is given by
  \begin{equation} \label{eq:ss_asympC1}
  \begin{aligned}
    \Gamma(\xi) &= O(\xi^{\frac{1}{m+n}}), & V(\xi) &= O(1), \\%&    \Theta(\xi) &= O(\xi^{-\frac{1+n}{\alpha-n}}),\\
   \Sigma(\xi) &= O(\xi), &   U(\xi) &= O(\xi^{\frac{1}{m+n}}).
  \end{aligned}
  \end{equation}
 \end{enumerate}
\end{theorem}

\subsection{Emergence of localization}
We describe the emergence of localization of the family of solutions for system \eqref{eq:B}   constructed by Theorem \ref{thm1}. The corresponding localized solutions for system \eqref{eq:C}  constructed by Theorem \ref{thm2} is similar, thus omitted. 

In both cases we replace $t \leftarrow t+1$,
\begin{equation*}
\begin{aligned}
 v(t,x) &= (t+1)^b V((t+1)^\lambda x),  & \theta(t,x) &= (t+1)^c \Theta((t+1)^\lambda x), \\
 \tau(t,x) &= (t+1)^d \Sigma((t+1)^\lambda x), & u(t,x) &= (t+1)^{b+\lambda} U((t+1)^\lambda x),
\end{aligned}
\end{equation*}
so that we interpret 
$$\big(V(\xi),\Theta(\xi),\Sigma(\xi),U(\xi)\big)=\big(v(0,x),\theta(0,x),\tau(0,x),u(0,x)\big)|_{x=\xi},$$  the initial states of the self-similar solutions. 

\begin{itemize}
 \item Initial non-uniformities : The profile $\big(V(\xi),\Theta(\xi),\Sigma(\xi),U(\xi)\big)$ is the initial profile of the self-similar solution. $\Theta(\xi)$ and $U(\xi)$ have a small bump at the origin from the asymptotically flat state. The tip sizes at the origin $\Theta_0$ and $U_0$ are the two parameters that fixes the solution. The velocity $V(\xi)$ is an odd function of $\xi$ that connects $-V_\infty$ and $V_\infty$ as $\xi$ spans from $-\infty$ to $\infty$, where $V_\infty \triangleq \lim_{\xi \rightarrow \infty} V(\xi)$. The slope near the origin is slightly steeper, which reflects the initial non-uniformity in the velocity.
\item Temperature : The temperature is an increasing function of $t$ for a fixed $x$. The growth rate at the origin is faster than any other $x$, which dictates the localization near the origin, see Figure \ref{fig:theta}
\begin{align*}
 \theta(t,0) &= (1+t)^{\frac{2}{D} + \frac{2+2n}{D}\lambda}\Theta(0),&
 \theta(t,x) &\sim t^{\frac{2}{D} - \frac{(1+n)^2}{D(\alpha-n)}\lambda}|x|^{-\frac{1+\alpha}{\alpha-n}}, \quad \text{as $t \rightarrow \infty$, $x\ne0$.}
\end{align*}

\item Strain rate : The growth rate at the origin is faster than the rest of the points, which dictates the localization near the origin, see Figure \ref{fig:u}
\begin{align*}
 u(t,0) &= (1+t)^{\frac{1}{D} + \frac{2+2\alpha}{D}\lambda}U(0), \\
 u(t,x) &\sim t^{\frac{1}{D} - \frac{(1+\alpha)(1+n)}{D(\alpha-n)}\lambda}|x|^{-\frac{1+\alpha}{\alpha-n}}, \quad \text{as $t \rightarrow \infty$, $x\ne0$.}
\end{align*}
\item Stress : The stress is a decreasing function of $t$ for  fixed $x$. However, the decay rate at the origin is faster than the rest of the points, see Figure \ref{fig:sigma}
\begin{align*}
 \tau(t,0) &= (1+t)^{\frac{-2\alpha+n}{D} + \frac{-2\alpha+2n}{D}\lambda}\Sigma(0), \\
 \tau(t,x) &\sim t^{\frac{-2\alpha+n}{D} +\frac{1+n}{D}\lambda}|x|^{-\frac{1+\alpha}{\alpha-n}}, \quad \text{as $t \rightarrow \infty$, $x\ne0$.}
\end{align*}
Note that the rate of the latter is always less than $-\frac{n}{1+n}$ in the valid range of  $\lambda$.
\item Velocity : The velocity $v(x,t)$ is an odd function of $x$. It connects $-v_\infty$ to $v_\infty$, as $x$ runs from $-\infty$ to $\infty$, where $v_\infty\triangleq \lim_{x \rightarrow \infty} v(t,x)$. Because of the scaling law $\xi=(1+t)^\lambda x$, the transition from $-v_\infty$ to $v_\infty$ localizes around the origin as time increases. The slope becomes steeper and steeper and develops a step function type singularity, see Figure \ref{fig:v}. The far field velocity 
$$v_\infty(t)=(1+t)^{b}V_\infty = (1+t)^{\frac{1}{D} + \frac{1+n}{D}\lambda}V_\infty$$
itself grows at a polynomial rate. This is not in agreement with the uniform shearing motion. This deviation is a consequence of our simplifying assumption for the self-similarity.
\end{itemize}

\begin{figure}[ht]
  \centering
  \subfigure[temperature ($\theta$)]{
  \includegraphics[scale=0.25]{temperature_log} \label{fig:theta}
  }
  \subfigure[strain rate ($u$)]{
  \includegraphics[scale=0.25]{strain_rate_log} \label{fig:u}
  }
  \subfigure[velocity ($v$)] {
  \includegraphics[scale=0.25]{velocity} \label{fig:v}
  }
  \subfigure[stress ($\sigma$)] {
  \includegraphics[scale=0.25]{stress_log} \label{fig:sigma}
  }
\caption{The localizing solutions for system \eqref{eq:B}, for $\mu(\theta)=\theta^{-\alpha}$, $\alpha=1.4$, $n=0.3$ and $\lambda =0.39$. 
 All graphs except $v$ are in logarithmic scale. See \cite{KLT_2016} for the system \eqref{eq:C}.} 
\end{figure}


\section{Existence via Geometric singular perturbation theory} \label{sec:idea}

The main step in proving Theorem \ref{thm1} and \ref{thm2}, is to show the existence of a heteroclinic orbit to an associated dynamical system, which, in the case of  system  \eqref{eq:B}, reads as follows,
\begin{equation} \label{eq:pqrsys}\tag{P}
\begin{aligned}
 \frac{\dpp}{p}&=\Big[\frac{1+\alpha}{1+n}\,\frac{1}{\lambda }\Big(r^{1+n}-c_0\Big)\Big] -\Big[d_1 + q + \lambda pr\Big],\\
 \frac{\dqq}{q}&=\Big[b_1 +\frac{bpr}{q}\Big] -\Big[d_1 + q + \lambda pr\Big],\\
 n\frac{\drr}{r}&=\Big[\frac{\alpha-n}{\lambda(1+n)}\Big(r^{1+n}-c_0\Big)\Big]+\Big[d_1 + q + \lambda pr\Big].
\end{aligned}
\end{equation}
The objective is to construct the heteroclinic orbit that connects equilibrium points
\begin{align*}
 M_0=\Big(0,0,\big(\frac{2}{D} + \frac{2(1+n)}{D} \lambda\big)^{\frac{1}{1+n}}\Big), \quad M_1=\Big(0,1,\big(\frac{2}{D} -\frac{(1+n)^2}{D(\alpha-n)} \lambda\big)^{\frac{1}{1+n}}\Big).
\end{align*}

First, we describe now briefly how system \eqref{eq:pqrsys} is derived.  The technique we employ was mainly developed in  \cite{KOT14}. Following \cite{KOT14,LT16,LT16_2},  we introduce a series of non-linear transformations described by  \eqref{eq:CAPtoBAR} and \eqref{eq:BARtoTIL} while the definition of $(p,q,r)$-variables is given by \eqref{eq:pqrdef}, 
\begin{equation} \label{eq:CAPtoBAR}
\begin{aligned}
 \bg(\xi)&=\xi^{a_1}\Gamma(\xi), &
 \bv(\xi)&=\xi^{b_1}V(\xi), &
 \bth(\xi)&=\xi^{c_1}\Theta(\xi), \\
 \bs(\xi)&=\xi^{d_1}\Sigma(\xi), &
 \bu(\xi)&=\xi^{b_1+1}U(\xi).
\end{aligned}
\end{equation}
\begin{equation} \label{eq:BARtoTIL}
\begin{aligned}
 \tg(\log\xi)&=\bg(\xi), &
 \tv(\log\xi)&=\bv(\xi), &
 \tth(\log\xi)&=\bth(\xi), \\
 \ts(\log\xi)&=\bs(\xi), &
 \tu(\log\xi)&=\bu(\xi), 
\end{aligned}
\end{equation}
\begin{equation}\label{eq:pqrdef}
 \begin{aligned}
  p \triangleq \frac{\tth^{\,\frac{1+\alpha}{1+n}}}{\ts}, \quad q \triangleq b \frac{\tv}{\ts},  \quad r \triangleq \frac{\tu}{\tth^{\,\frac{1+\alpha}{1+n}}},
 \end{aligned}
\end{equation}
with $\eta\triangleq \log\xi$ being the new independent variable $\Big(\frac{df}{d\eta}=\dot{f}\Big)$. 
The system \eqref{eq:pqrsys} has a {\it fast-slow} structure due to parameter $n$ in front of $\dot{r}$. We conduct a Chapman-Enskog type reduction via geometric singular perturbation theory \cite{fenichel_geometric_1979, KUEHN_2015}. The reduced problem becomes a planar dynamical system and  the heteroclinic orbit is obtained by phase space analysis, \cite{LT16_2}. 
\subsection{Critical manifold}
The surface the orbit relaxes is near the zero set of the right-hand-side of $\eqref{eq:pqrsys}_3$. The zero set,  that is away from $r=0$ plane,  is the surface specified by
\begin{align*}
 r=\frac{ \frac{\alpha c_0}{\lambda} - d_1 -q }{ \frac{\alpha}{\lambda} + \lambda p}\triangleq h(p,q;n=0), \quad \text{or} \quad  q + \lambda {r}p + \frac{\alpha}{\lambda} \Big( r-r_0\Big)=0.
\end{align*}
%
%
We take the triangle $T$ in the first quadrant enclosed by $p$-axis, $q$-axis and the contour line $\underbar{r} =  h(p,q;n=0)$ and a compact set $K\supset\supset T$. We set the critical manifold 
\begin{equation}
 G(\lambda,\alpha,n=0) \triangleq \Big\{\, (p,q,r) \;|\; (p,q) \in K, \text{ and } r=\frac{ \frac{\alpha c_0}{\lambda} - d_1 -q }{ \frac{\alpha}{\lambda} + \lambda p} \,\Big\}.
\end{equation}
%
The system in {\it fast scale} with the independent variable $\tilde{\eta} = \eta/n$ is
\begin{equation}\label{eq:pqr_fast} \tag*{($\tilde{P}$)}
\begin{aligned}
 p^\prime &=np\Big( \Big[\frac{1+\alpha}{1+n}\,\frac{1}{\lambda }\Big(r^{1+n}-c_0\Big)\Big] -\Big[d_1 + q + \lambda pr\Big]\Big), \\
 q^\prime &=nq\Big(\Big[b_1 +\frac{bpr}{q}\Big] -\Big[d_1 + q + \lambda pr\Big]\Big), \\
 r^\prime &=r\Big( \Big[\frac{\alpha-n}{\lambda(1+n)}\Big(r^{1+n}-c_0\Big)\Big]+\Big[d_1 + q + \lambda pr\Big]\Big),
\end{aligned}
\end{equation}
where we denoted $\displaystyle(\cdot)^\prime = \frac{d}{d\tilde{\eta}}(\cdot)$. When $n=0$, $(\tilde{P})|_{n=0}$ reads
\begin{align*}
 p^\prime =0, \quad q^\prime =0, \quad r^\prime=r\Big( \Big[\frac{\alpha}{\lambda}\Big(r-c_0\Big)\Big]+\Big[d_1 + q + \lambda pr\Big]\Big). %\label{eq:fastn0}
\end{align*}
\begin{lemma} \label{lem:normal_hyper}
 $G(\lambda,\alpha,0)$ is a normally hyperbolic invariant manifold with respect to the system $(\tilde{P})|_{n=0}$.
\end{lemma}

\subsection{Chapman-Enskog type reduction}
By the theorem of geometric singular perturbation theory, \cite{fenichel_geometric_1979, KUEHN_2015}, if $n$ is sufficiently small, there exists  the locally invariant manifold $G(\lambda,\alpha,n)$ with respect to \eqref{eq:pqrsys}. Then, on this manifold, $\big(p(\eta),q(\eta)\big)$ satisfies the planar system
\begin{equation} \tag*{(${R}$)} \label{eq:reduced}
\begin{aligned}
 {\dpp}&=p\bigg\{\Big[\frac{1+\alpha}{1+n}\,\frac{1}{\lambda }\Big(h^{1+n}-c_0\Big)\Big] -\Big[d_1 + q + \lambda ph\Big]\bigg\},\\
 {\dqq}&=q\Big(1-q-\lambda p h\Big) + bph,\\
\end{aligned}
\end{equation}
where $h$ stands for $h(p,q;n)$.


\subsection{Confinement of the orbit}
\begin{lemma}
The triangle $T$  is positively invariant for the system $(R)$ when $n=0$.
\end{lemma}
We can compute the inward normal component of $(\dot{p},\dot{q})$ on the boundary of the triangle $T$ for $(R)|_{n=0}$:
\begin{align*}
 \dot{p} &= -\frac{D}{\alpha} p\Big(d_1+q+\lambda p h\Big),\\
 \dot{q} &= q\Big(1-q-\lambda p h\Big) + b p h.
\end{align*}
Essential calculation is on the hypotenuse and the fact that it is the contour line $\underbar{r}=h(p,q,n=0)$ helps us obtain the estimate. Define $\underbar{p}$ and $\underbar{q}$ to be the $p$-intercept and $q$-intercept of the contour line respectively : $\underbar{q} = \lambda \underbar{p}\underbar{r} = \frac{\alpha}{\lambda}(r_0-\underbar{r})$. With  $(-\underbar{q},-\underbar{p})$ being the inward normal vector, the  inward normal component on the hypotenuse is 
\begin{align*}
    (\dot{p},\dot{q}) \cdot (-\underbar{q},-\underbar{p}) &= \frac{D}{\alpha}\underbar{q}p\Big(d_1 + q + \lambda p \underbar{r}\Big) -\underbar{p}\bigg\{q\Big(1-q-\lambda p \underbar{r}\Big) + b p \underbar{r}\bigg\}\\
    &=\frac{D}{\alpha}\underbar{q}p\Big(d_1 +\underbar{q}\Big) -\underbar{p}\bigg\{\Big(\underbar{q}-\lambda p \underbar{r}\Big)\Big(1-\underbar{q}\Big) + b p \underbar{r}\bigg\}\\
    &=-\underbar{p}\underbar{q}(1-\underbar{q}) + \underbar{q}p\Big(\frac{D}{\alpha}d_1 + \frac{D}{\alpha}\underbar{q} + (1-\underbar{q}) - \frac{b}{\lambda}\Big)\\
    &=-\underbar{p}\underbar{q}(1-\underbar{q}) + \underbar{q}p\frac{1+\alpha}{\lambda}\Big(\frac{1}{1+\alpha}-\underbar{r}\Big)\\
    &\ge -\underbar{p}\underbar{q}(1-\underbar{q}) \quad \text{for $\underbar{r} < \frac{1}{1+\alpha}$}\\
    &\ge \delta_0 > 0.
\end{align*}

\begin{lemma}
The triangle $T$  is positively invariant for the system $(R)$ provided $n$ is sufficiently small.
\end{lemma}
Now the hypotenuse is not anymore a contour line of the function $h(p,q;\lambda,\alpha,n)$. We arrange terms of right-hand-sides of $(R)$ in the form
\begin{align*}
 {\dpp}&=p\bigg\{\Big[\frac{1+\alpha}{\lambda }\Big(\underbar{r}-c_0\Big)\Big] -\Big[d_1 + q + \lambda p\underbar{r}\Big]\bigg\} \\
 &+ \underbrace{p\bigg\{\Big[\frac{1+\alpha}{1+n}\,\frac{1}{\lambda }\Big(h^{1+n}-c_0\Big)\Big]-\Big[\frac{1+\alpha}{\lambda }\Big(\underbar{r}-c_0\Big)\Big] -\lambda p(h-\underbar{r})\Big]\bigg\}}_\text{$\triangleq g_1(p,q,n)$},\\
 {\dqq}&=q\Big(1-q-\lambda p \underbar{r}\Big) + bp\underbar{r} + \underbrace{(-q\lambda p+b) (h-\underbar{r})}_\text{$\triangleq g_2(p,q,n)$}.
\end{align*}
Since $h$ is a smooth function of $n$ and $D$ is compact, provided $n$ is sufficiently small, we have an estimate
\begin{equation*}
 |g_1(p,q,n)| + |g_2(p,q,n)| \le C_0 n, \quad \text{where $C_0$ does not depend on $p$, $q$, and $n$.}
\end{equation*}
Therefore
$$ (\dot{p},\dot{q}) \cdot(-\underbar{q},-\underbar{p}) \ge \delta_0 + C_0'n \quad \text{for another uniform constant $C_0'$}.$$
For $n$ sufficiently small the last expression is positive. After having the orbit confined in the positive invariant set $T$, we further conduct the phase space analysis to capture the heteroclinic orbit, for details we refer to \cite{LT16, LT16_2}. 
\subsection*{Acknowledgements}
This research was supported by King Abdullah University of Science and Technology (KAUST).

\begin{thebibliography}{10}

% \bibitem{baxevanis_adaptive_2010}
% {\sc Th.~Baxevanis, Th~Katsaounis, and A.~E. Tzavaras},  
% Adaptive finite element computations of shear band formation, 
%   {\em Math. Models  Methods Appl. Sci.} {\bf 20}  (2010),  423--448.

\bibitem{bertsch_effect_1991}
{\sc M.~Bertsch, L.~Peletier, and S.~Verduyn~Lunel}, 
The effect of temperature dependent viscosity on shear flow of  incompressible fluids,
{\em SIAM J. Math. Anal.} {\bf 22 } (1991), 328--343.

\bibitem{CDHS}
{\sc R.J. Clifton, J. Duffy, K.A. Hartley, and T.G. Shawki}, 
On critical conditions for shear band formation at high strain rates, 
{\em Scripta Met.}, {\bf 18} (1984), 443?448.

% \bibitem{CB99} 
% {\sc L.~ Chen and R.C.~Batra },
% The asymptotic structure of a shear band in mode-II deformations.
% {\em International Journal of Engineering Science} {\bf 37} (1999),  895--919.

\bibitem{dafermos_adiabatic_1983}
{\sc C.~M. Dafermos and L.~Hsiao}, 
Adiabatic shearing of incompressible fluids with temperature-dependent viscosity.
{\em Quart.  Applied Math.} {\bf 41} (1983), 45--58.

% \bibitem{estep_2001}
% {\sc Donald~J Estep, Sjoerd M~Verduyn Lunel, and Roy~D Williams}, 
% {Analysis of Shear Layers in a Fluid with Temperature-Dependent Viscosity},
%  {\em  J. Comp. Physics}  {\bf 173} (2001), 17--60.

\bibitem{fenichel_geometric_1979}
{\sc N.~Fenichel}, 
Geometric singular perturbation theory for ordinary differential equations, 
{\it J. Differ. Equations} {\bf 31} (1979), 53--98.

\bibitem{FM87}
{\sc C.~Fressengeas and A.~Molinari}, 
Instability and localization of plastic flow in shear at high strain rates, 
  {\em J.  Mech. Physics of Solids} {\bf 35} (1987), 185--211.

% \bibitem{jones_geometric_1995}
% {\sc C.~K. R.~T. Jones}, 
% Geometric singular perturbation theory, in {\it Dynamical systems}, LNM {\bf 1609} (Springer Berlin Heidelberg 1995) 44--118.
  
\bibitem{KT09}
{\sc Th.~Katsaounis and A.E.~Tzavaras}, 
 Effective equations for localization and shear band formation, 
 {\em SIAM J. Appl. Math.}  {\bf 69} (2009), 1618--1643.
  
\bibitem{KOT14}
{\sc Th.~Katsaounis, J.~Olivier, and A.E.~Tzavaras}, 
Emergence of coherent localized structures in shear deformations of temperature dependent fluids, {\em Archive for Rational Mechanics and Analysis}, (online). doi:10.1007/s00205-016-1071-2.

\bibitem{KUEHN_2015}
{\sc C.~ Kuehn}, 
{\it Multiple time scale dynamics}, Applied Mathematical Sciences, Vol. {\bf 191} (Springer Basel 2015).
  
\bibitem{LT16}
{\sc M-G.~Lee and A.E.~Tzavaras},
Existence of localizing solutions in plasticity via the geometric singular perturbation theory, 
{\em SIAM J. Appl. Dyn. Syst.}, (to appear). (arXiv:1608.00198).
%(https://arxiv.org/pdf/1608.00198.pdf)

\bibitem{LT16_2}
{\sc M-G.~Lee, Th. Katsaounis and A.E.~Tzavaras},
Localization and the formation of shear bands in thermoviscoplastic deformations. (in preparation).

\bibitem{KLT_2016}
{\sc Th. Katsaounis, M-G. Lee, and A.E. Tzavaras}, 
Localization in inelastic rate dependent shearing deformations, 
{\em J.  Mech. Physics of Solids} {\bf 98} (2017), 106--125.

% \bibitem{perko_differential_2001}
% {\sc L.~Perko}, 
% {\it Differential equations and dynamical systems 3rd. ed.}, TAM {\bf 7} (Springer-Verlag New York 2001).  

\bibitem{SC89}
{\sc T.~G. Shawki and R.~J. Clifton}, 
Shear band formation in thermal viscoplastic materials, 
{\em Mechanics of Materials} {\bf 8 } (1989), 13--43.

\bibitem{Tz_1986}
{\sc A.E. Tzavaras},
Shearing of materials exhibiting thermal softening or temperature dependent viscosity,
{\em Quart.  Applied Math.} {\bf 44} (1986), 1--12.

% \bibitem{Tz_1987}
% {\sc A.E. Tzavaras}, 
% Effect of thermal softening in shearing of strain-rate dependent materials.
% {\em Archive for Rational Mechanics and Analysis}, {\bf 99} (1987), 349--374.

\bibitem{tzavaras_nonlinear_1992}
%\leavevmode\vrule height 2pt depth -1.6pt width 23pt,
{\sc A.E. Tzavaras}, 
Nonlinear analysis techniques for shear band formation at high strain-rates, 
% {\it Applied Mechanics Reviews} 
{\it Appl. Mech. Rev.}
{\bf  45} (1992), S82--S94.

% \bibitem{walter_1992}
% {\sc J.W.~Walter}, 
% Numerical experiments on adiabatic shear band formation
%   in one dimension.
%   {\em International Journal of Plasticity} {\bf  8} (1992), 657--693.

\bibitem{ZH_1944}
{\sc C.~Zener and J.~H. Hollomon}, 
Effect of strain rate upon plastic flow of steel,
% {\it  Journal of Applied Physics} 
{\it J. Appl. Phys.}
{\bf 15} (1944), 22--32.
\end{thebibliography}

\end{document}	
