%%%%%%%%%%%%%%%%%%%%%%%%%%%%%%%%%%%%%%%%%%%%%%%
%
%    Localization in the adiabatic process of shear deformation for Hyp2016
%
%                                                      by
%
%                                       Min-Gi Lee   
%
%                                          version Sep 2016
%
%
%%%%%%%%%%%%%%%%%%%%%%%%%%%%%%%%%%%%%%%%%%%%%%%




% RECOMMENDED %%%%%%%%%%%%%%%%%%%%%%%%%%%%%%%%%%%%%%%%%%%%%%%%%%%
\documentclass[graybox]{svmult}

% choose options for [] as required from the list
% in the Reference Guide

\usepackage{mathptmx}       % selects Times Roman as basic font
\usepackage{helvet}         % selects Helvetica as sans-serif font
\usepackage{courier}        % selects Courier as typewriter font
\usepackage{type1cm}        % activate if the above 3 fonts are
                            % not available on your system
%
\usepackage{makeidx}         % allows index generation
\usepackage{graphicx}        % standard LaTeX graphics tool
                             % when including figure files
\usepackage{multicol}        % used for the two-column index
\usepackage[bottom]{footmisc}% places footnotes at page bottom


\usepackage{amsmath}
\usepackage{amssymb}
\usepackage{color,subfigure}
\def\red{\color{red}}
\def\blue{\color{blue}}
%\usepackage{verbatim}
% \usepackage{alltt}
%\usepackage{kotex}


%%%%%%%%%%%%%% MY DEFINITIONS %%%%%%%%%%%%%%%%%%%%%%%%%%%

\def\tr{\,\textrm{tr}\,}
\def\div{\,\textrm{div}\,}
\def\sgn{\,\textrm{sgn}\,}

\def\th{\tilde{h}}
\def\tx{\tilde{x}}
\def\tk{\tilde{\kappa}}


\def\bg{{\bar{\gamma}}}
\def\bv{{\bar{v}}}
\def\bth{{\bar{\theta}}}
\def\bs{{\bar{\tau}}}
\def\bu{{\bar{u}}}
\def\bph{{\bar{\varphi}}}


\def\tg{{\tilde{\gamma}}}
\def\tv{{\tilde{v}}}
\def\tth{{\tilde{\theta}}}
\def\ts{{\tilde{\tau}}}
\def\tu{{\tilde{u}}}
\def\tph{{\tilde{\varphi}}}

\def\dtg{{\dot{\tilde{\gamma}}}}
\def\dtv{{\dot{\tilde{v}}}}
\def\dtth{{\dot{\tilde{\theta}}}}
\def\dts{{\dot{\tilde{\tau}}}}
\def\dtu{{\dot{\tilde{u}}}}
\def\dtph{{\dot{\tilde{\varphi}}}}

\def\dpp{\dot{p}}
\def\dqq{\dot{q}}
\def\drr{\dot{r}}
\def\dss{\dot{s}}

\def\ta{{\tilde{a}}}
\def\tb{{\tilde{b}}}
\def\tc{{\tilde{c}}}
\def\td{{\tilde{d}}}

\def\BO{{\mathcal{O}}}
\def\lio{{\mathcal{o}}}
\def\R{{\mathbb{R}}}


\def\bx{\bar{x}}
\def\bm{\bar{\mathbf{m}}}
\def\K{\mathcal{K}}
\def\E{\mathcal{E}}
\def\del{\partial}
\def\eps{\varepsilon}

% see the list of further useful packages
% in the Reference Guide

\makeindex             % used for the subject index
                       % please use the style svind.ist with
                       % your makeindex program

%%%%%%%%%%%%%%%%%%%%%%%%%%%%%%%%%%%%%%%%%%%%%%%%%%%%%%%%%%%%%%%%%%%%%%%%%%%%%%%%%%%%%%%%%

\begin{document}

\title*{Localization of Thermoviscoplastic Materials In Adiabatic Deformations }
% Use \titlerunning{Short Title} for an abbreviated version of
% your contribution title if the original one is too long
\author{Min-Gi Lee, Theodoros Katsaounis and Athanasios E. Tzavaras}
% Use \authorrunning{Short Title} for an abbreviated version of
% your contribution title if the original one is too long
\institute{Min-Gi Lee \at King Abdullah University of Science and Technology (KAUST), Computer, Electrical and Mathematical Sciences \& Engineering Division, KAUST, Thuwal, Saudi Arabia, \\ \email{mingi.lee@kaust.edu.sa}
\and Theodoros Katsaounis \at King Abdullah University of Science and Technology (KAUST), Computer, Electrical and Mathematical Sciences \& Engineering Division, KAUST, Thuwal, Saudi Arabia,\\  \email{theodoros.katsaounis@kaust.edu.sa}
\and Athanasios Tzavaras \at King Abdullah University of Science and Technology (KAUST), Computer, Electrical and Mathematical Sciences \& Engineering Division, KAUST, Thuwal, Saudi Arabia,\\  \email{athanasios.tzavaras@kaust.edu.sa}}
%
% Use the package "url.sty" to avoid
% problems with special characters
% used in your e-mail or web address
%
\maketitle

%\abstract*{We study a focusing instability occuring during deformations of metals at high strain rates  and leading to the formation of shear bands. 
%We consider shear motions of a thermal-viscoplastic material exhibting thermal softening and strain hardening.
%Under adiabatic conditions thermal softening induces a destabilizing mechanism that may lead to  localization of shear strain in a narrow band.
%The main result of the paper is the existence of the family of focusing self-similar solutions. This reveals the nature of the instability, in particular the special role of the viscosity. In the absence of the viscosity, the system without hyperbolicity exhibits Hadamard Instability. The presence of the viscosity alters the nature of the instability; the focusing self-similar solutions we obtain can grow only at the polynomial order and the rate of focusing is bounded above. To construct the solutions, we introduce the non-linear transformations to turn the problem into that of finding heteroclinic orbit of the associated dynamical system. The key step is to conduct the Chapman-Enskog type reduction via geometric singular perturbation theory. The part of the latter step is included.}

\abstract{We study a focusing instability occurring during high strain-rate deformations of metals  and leading to the formation of shear bands. 
We consider shear motions of a thermal-viscoplastic material exhibting thermal softening and strain hardening.
Under adiabatic conditions thermal softening induces a destabilizing mechanism that may lead to  localization of shear strain in a narrow band.
We show the existence of a family of focusing self-similar solutions, that capture the nature of this instability, in particular its
emergence as a net result of the competition between Hadamard instability and viscosity.
The focusing self-similar solutions we obtain can grow at a  polynomial order. To construct the solutions, we introduce a non-linear transformation and to turn the 
existence problem into that of finding a heteroclinic orbit of an associated dynamical system. The mechanism of instability is captured by a Chapman-Enskog type asymptotic reduction. The existence of the heteroclinic orbit is obtained via the geometric singular perturbation theory. }


\section{Introduction}
We study a focusing type instability occurring in one dimensional adiabatic shear flows. We consider shear motions of a slab  located at in the $x-y$ plane,
and moving in the $y$-direction. We are interested in two situations: (a) the material  occupies the slab  between two parallel plates,
$(x,y) \in (0,1) \times \R$,  and is sheared by assigned velocities at $x=0,1$. 
(b) The material occupies the entire $x-y$-space and is sheared in the $y$-direction. 
As the material keeps loading, the shear deformation and the internal energy increase in time. Materials typically is {\it thermally softening} 
and this, in some instances, can lead to the loss of hyperbolicity in the corresponding thermo-mechanical system. As the process is adiabatic, energy accumulates at each of point $x$ possibly non-uniformly and this imbalance of the temperature distribution can lead to the localization of shear.
%This problem also has attracted attentions in mechanics literature, see \cite{FM87,ZH_1944} and references therein.

To understand this phenomenon, we study the following one dimensional thermo-mechanical model. $y(t,x)\in \mathbb{R}$ is the spatial coordinate in the $y$-direction at time $t$ of the material initially at $x$ on the $x$-axis. The motion is described by following quantities,
\begin{equation} \label{eq:vars}
\begin{aligned}
 \gamma(t,x) \triangleq \partial_x y(t,x)&: \text{shear strain},\\
 v(t,x)\triangleq \partial_t y(t,x) &: \text{vertical velocity},\\
 u(t,x)\triangleq \partial_t\gamma(t,x) &: \text{shear strain rate},\\
 \theta(t,x) &: \text{temperature},\\
 \tau(t,x) &: \text{shear stress}.
\end{aligned}
\end{equation}
Here, we focus on the problem in the entire space $(t,x) \in \mathbb{R}^+\times \mathbb{R}$.
The system of equations describing the motion is
\begin{equation} \label{eq:A}\tag{A}
\begin{aligned}
 \gamma_t &= u \quad \text{(kinematic compatibility)}, 	\\
 v_t &= \tau_x \quad \text{(momentum conservation)}, 	\\
 \theta_t &= \tau u \quad \text{(energy equation that is adiabatic)},	\\
 \tau &=\tau(\theta,\gamma,u) \quad \text{(constitutive law)}.			
\end{aligned}
\end{equation}
In terms of classification, the model \eqref{eq:A} belongs to the framework of one-dimensional thermo-visco-elasticity. 
It is also instructive to interprete $\eqref{eq:A}_4$ as a constitutive law for thermo-visco-plastic materials viewing $\gamma=\gamma_p$ the plastic strain. (See the hierarchy of models in \cite{KT09,tzavaras_nonlinear_1992}.)
The latter context suggests the terminology: The material exhibits thermal softening at $(\theta,\gamma,u)$ where $f_\theta(\theta,\gamma,u)<0$, strain hardening where $f_\gamma(\theta,\gamma,u)>0$, and strain softening where $f_\gamma(\theta,\gamma,u)<0$.

To carry on the parametric study accordingly to the stiffness of these slopes, we focus on the law of form
\begin{equation}
 \tau = \varphi(\theta,\gamma)u^n = \theta^{-\alpha}\gamma^m u^n, \label{eq:power} 
\end{equation}
where $n$ is the strain-rate sensitivity and is assumed to be very small, $0<n\ll1$. We further introduce two subclasses of \eqref{eq:A}, where $\varphi$ is independent of either $\theta$ or $\gamma$ respectively.
% \begin{align}
%  \tau &= \varphi(\gamma)u^n, \quad \text{or} \quad \tau = \mu(\theta)u^n.
% \end{align}
The strain independent model $(B)$ consists of 
\begin{equation} \label{eq:B}\tag{B}
 v_t = \tau_x, \quad \theta_t = \tau u, \quad \tau = \mu(\theta)u^n 		
\end{equation}
and the temperature indenpendent model $(C)$ consists of
\begin{equation} \label{eq:C}\tag{C}
 \gamma_t = u, \quad  v_t = \tau_x \quad \tau, = \varphi(\gamma)u^n.			
\end{equation}

The main results of this paper is the construction of the family of self-similar solutions of {\it focusing type} to the problems (B), and (C). The same for the problem (A) is in progress. This study on the focusing behavior is motivated by the phenomena of shear bands, which is the narrow zones shear localizes intensely. It was observed by Zener and Hollomon \cite{ZH_1944} during the high speed of deformation of metals. For further mechanical interests on the problem, we refer to \cite{FM87, SC89} and references therein. %, where it was recognized that the effects of high strain rate are twofolds: (a) It alters the deformation conditions from isothermal to nearly adiabatic, (b) strain rate has an effect {\it per se}, and needs to be included in the constitutive law. 

In the study of the stability, one finds the solution of uniform shearing motion as the referential motion, which arises universally to (A). At the simplest normalization constants, 
$$ \gamma = t, \quad v=x, \quad u=1, \quad \text{$\theta$ such that $\theta_t = \varphi(\theta,t)$}$$
constitutes a solution. More concrete form is available for (B),
$$ h(\theta) = \int_{\Theta_0}^\theta \frac{d\theta'}{\mu(\theta')}, \quad \theta = h^{-1}(t).$$
Note that the growth rate of the strain is linear and uniform in $x$. In the focusing solutions we construct, the growth rate is not uniform; it grows faster at the focusing core and slower at the rest of the points.

The linear stability around the uniform shearing solution \cite{FM87} and the study of Chapman-Enskog type relaxation to the effective equation \cite{KT09} indicate the system becomes unstable in the regime
\begin{equation}
 -\alpha + m + n < 0. \label{eq:constraint}
\end{equation}
In the other regime $-\alpha + m + n > 0$, it is expected that the uniform shearing solution is the globally asymptotically stable solution.
Non-linear stability of the problem (B) ($m=0$) in the regime $-\alpha+n>0$ has been studied from early stage of the research, by Dafermos and Hsiao \cite{dafermos_adiabatic_1983} when $n=1$, and by Tzavaras \cite{Tz_1986} when $n\ne1$. The system (B) has an interpretation that the model describes a fluid with temperature dependent viscosity $\mu(\theta)$ in the rect-linear shear motion. Similar result for the problem (C) ($\alpha=0$) in the regime $m+n>0$  is obtained by Tzavaras \cite{tzavaras_nonlinear_1992}. For the problem (B), the failure of the asymptotic stability is treated by Bertsch et. al. \cite{bertsch_effect_1991} when $n=1$ and by Katsaounis and Tzavaras \cite{KT09} when $n\ne1$. Our objective is to show the onset of localization by constructing solutions directly.

The special role of the viscosity in the model is of our central insterest. In the absence of the viscosity $(n=0)$ for the regime \eqref{eq:constraint}, the system loses hyperbolicity and the instability system exhibits has coined the term {\it Hadamard Instability}. This refers to that the oscillations will grow exponentially, i.e., the systems are catastrophically ill-posed. In the presence of the viscosity on the other hand, it alters the nature of the instability. There is a competition between the viscosity and the inelastic softening. Even when the viscosity effects merely in the way \eqref{eq:constraint} holds, it is one of our main objective to state that, among the self-similar family, the growth and focusing rates are polynomial and the rate is bounded above according to the material property.
% The main results of this paper is the construction of the family of self-similar solutions to the problems (B), and (C). The same for the problem (A) is in progress. For this studies, the thermal softening effect and the strain hardening effect are parametrized by the exponents for the following power law 
% \begin{equation}
%  \tau = \varphi(\theta,\gamma)u^n = \theta^{-\alpha}\gamma^m u^n. \label{eq:pw}
% \end{equation}
% Linear stability around the uniform shearing solution in the slab of material \cite{FM87} has revealed that the system exhibits instability for the regime 
% \begin{equation}
%  -\alpha + m + n < 0. \label{eq:constraint}
% \end{equation}
% In this parametrization, the problem (A) refers to as the one with \eqref{eq:pw}, the problem (B) as the one with with \eqref{eq:pw} but $m=0$, and lastly (C) as the one with \eqref{eq:pw} but $\alpha=0$. In all cases, we work in the regime satisfying \eqref{eq:constraint}. The technique we employ for the study was mainly developed in Katsaounis et. al. \cite{KOT14}, where the authors studies the problem (B) with exponential law $\mu(\theta) = e^{-\alpha\theta}$. 



%%%%%%%%%%%%%%%%%%%%%%
%% SECTION
%%%%%%%%%%%%%%%%%%%%%%
\section{Main results}
\subsection{Self-similar structure}
We investigate the scale invariance property of the system (A), and consequently that of (B) and (C) too.
Suppose $(\gamma,u,v,\theta,\tau)$ is a solution of system (A). Then a rescaled version of it $(\gamma_\rho,u_\rho,v_\rho,\theta_\rho,\tau_\rho)$ given by
\begin{align*}
 \gamma_\rho(t,x) &= \rho^a\gamma(\rho^{-1}t,\rho^\lambda x), &
 v_\rho(t,x) &= \rho^bv(\rho^{-1}t,\rho^\lambda x),\\
 \theta_\rho(t,x) &= \rho^c\theta(\rho^{-1}t,\rho^\lambda x), &
 \tau_\rho(t,x) &= \rho^d\tau(\rho^{-1}t,\rho^\lambda x),\\
 u_\rho(t,x) &= \rho^{b+\lambda}\gamma(\rho^{-1}t,\rho^\lambda x),
\end{align*}
is also a solution of (A) provided
\begin{align*}
 a&= \frac{2+2\alpha-n}{D} + \frac{2+2\alpha}{D}\lambda =: a_0 + a_1 \lambda, \\
 b&=\frac{1+m}{D} + \frac{1+m+n}{D}\lambda =: b_0 + b_1\lambda,\\
 c&=\frac{2(1+m)}{D} + \frac{2(1+m+n)}{D}\lambda =: c_0 + c_1\lambda, \\
 d&=\frac{-2\alpha + 2m +n}{D} + \frac{-2\alpha+2m+2n}{D}\lambda =: d_0 + d_1\lambda,
\end{align*}
for each $\lambda \in \mathbb{R}$, where  $D = 1+2\alpha-m-n$. 
Motivated by the scale invariance property parametrized by $\lambda$, we look for the solutions of the form 
\begin{align*}
 \gamma(t,x) &= t^a\Gamma(t^\lambda x), & v(t,x) &= t^b V(t^\lambda x), &\theta(t,x) &= t^c \Theta(t^\lambda x),\\
 \tau(t,x) &= t^d \Sigma(t^\lambda x), & u(t,x) &= t^{b+\lambda} U(t^\lambda x)
\end{align*}
and set $\xi = t^\lambda x$. In this format, $\lambda>0$ accounts for the focusing behavior as time proceeds whereas $\lambda<0$ accounts for the de-focusing behavior. This family includes the uniform shearing solution at $\lambda = -\frac{1+m}{2(1+\alpha)}$. Since we are interested in the focusing solutions, we consider $\lambda>0$ in the rest of the paper.


Plugging in the ansatz to the system (A) we obtain a  system of ordinary differential and algebraic equations that $\big(\Gamma(\xi), V(\xi), \Theta(\xi), \Sigma(\xi), U(\xi)\big)$ satisfies.

\begin{equation}
\begin{aligned}
 a \Gamma(\xi) + \lambda \xi \Gamma'(\xi) &= U(\xi),\\
 b V(\xi) + \lambda \xi V'(\xi) &= \Sigma'(\xi),\\
 c \Theta(\xi) + \lambda \xi \Theta'(\xi)&=\Sigma(\xi) U(\xi),\\
 \Sigma(\xi) &= \Theta(\xi)^{-\alpha} \Gamma(\xi)^m U(\xi)^n,\\
 V'(\xi)&=U(\xi).
\end{aligned} \label{eq:ss-odes}
\end{equation}

\subsection{Main theorem}

We first state the existence of two parameters family of solutions for (B) where $m=0$. See \cite{LT16_2} for the detailed discussion.

\begin{theorem} \label{thm1}
Let $\alpha,n>0$, $\alpha\ne2n+1$ the given material parameters and fix $U_0>0$ and $\Theta_0>0$. Suppose that
\begin{equation} \label{eq:restriction}
 \frac{2}{1+2\alpha-n} < \frac{U_0^{1+n}}{\Theta_0^{1+\alpha}} < \frac{2}{1+n},
\end{equation}
$-\alpha+n<0$, and $n$ is sufficiently small. Then there is a focusing self-similar solution of the form
\begin{equation*}
\begin{aligned}
 v(t,x) &= (t+1)^b V((t+1)^\lambda x), &\theta(t,x) &= (t+1)^c \Theta((t+1)^\lambda x),\\
 \tau(t,x) &= (t+1)^d \Sigma((t+1)^\lambda x), & u(t,x) &= (t+1)^{b+\lambda} U((t+1)^\lambda x)
\end{aligned}
\end{equation*}
to the system (B), where the focusing rate is 
\begin{equation}
 \lambda = \frac{1+2\alpha-n}{2+2n}\frac{U_0^{1+n}}{\Theta_0^{1+\alpha}} - \frac{2}{2+2n}>0. \label{eq:lambda}
\end{equation}
Furthermore, the self-similar profile $\big(V(\xi),\Theta(\xi),\Sigma(\xi),U(\xi)\big), \ \xi = (t+1)^\lambda x$,  has the  following properties :
 \begin{enumerate}
  \item[(i)] Satisfies the boundary condition at $\xi=0$,
    \begin{equation*}
    {V}(0) = \Theta_\xi(0)=\Sigma_\xi(0) = {U}_\xi(0)=0, \quad U(0)=U_0, \Theta(0)=\Theta_0.
  \end{equation*}
  \item[(ii)] Its asymptotic behavior as $\xi \rightarrow 0$ is given by 
  \begin{equation} \label{eq:ss_asymp0}
  \begin{aligned}
%     \Gamma(\xi) &= \frac{1}{a}U(0) + \Gamma^{''}(0)\frac{\xi^2}{2} + o(\xi^2), & \Gamma^{''}(0)&<0,\\
    \Theta(\xi) &= \Theta(0) + \Theta^{''}(0)\frac{\xi^2}{2} + o(\xi^2), & \Theta^{''}(0)&<0,\\
    \Sigma(\xi) &= \Theta(0)^{-\alpha}{U(0)^n}+ \Sigma^{''}(0)\frac{\xi^2}{2} + o(\xi^2), & \Sigma^{''}(0)&>0, \\
    U(\xi) &= U(0) + U^{''}(0)\frac{\xi^2}{2} + o(\xi^2), & U^{''}(0)&<0,\\
    V(\xi) &= U(0)\xi + U^{''}(0)\frac{\xi^3}{6} + o(\xi^3), & U^{''}(0)&<0.
  \end{aligned}
  \end{equation}
  \item[(iii)] Its asymptotic behavior as $\xi \rightarrow \infty$ is given by
  \begin{equation} \label{eq:ss_asymp1}
  \begin{aligned}
%     \Gamma(\xi) &= O(\xi^{-\frac{1+\alpha}{\alpha-n}}), & 
    V(\xi) &= O(1), &    \Theta(\xi) &= O(\xi^{-\frac{1+n}{\alpha-n}}),\\
   \Sigma(\xi) &= O(\xi), &   U(\xi) &= O(\xi^{-\frac{1+\alpha}{\alpha-n}}).
  \end{aligned}
  \end{equation}
 \end{enumerate}
\end{theorem}

In case of system (C), where  $\alpha=0$,  there exists a two parameter family of solutions,  see \cite{KLT_2016}, \cite{LT16} for the detailed discussion. 

\begin{theorem} \label{thm2}
Let $-1\le m<0$ and $n>0$, $m+n\ne \frac{1}{2}$ the given material parameters and fix $U_0>0$ and $\Gamma_0>0$. Suppose that
\begin{equation*}
 \frac{2-n}{1-m-n} \;<\; \frac{U_0}{\Gamma_0} \;<\; \frac{2-n}{1+m+n} \, ,
\end{equation*}
$m+n<0$, and $n$ is sufficiently small. Then there is a focusing self-similar solution of the form
\begin{equation*}
\begin{aligned}
 \gamma(t,x) &= (t+1)^a\Gamma((t+1)^\lambda x), & v(t,x) &= (t+1)^b V((t+1)^\lambda x), \\
 \tau(t,x) &= (t+1)^d \Sigma((t+1)^\lambda x), & u(t,x) &= (t+1)^{b+\lambda} U((t+1)^\lambda x),
\end{aligned}
\end{equation*}
to the system (C), where the focusing rate is
\begin{equation}
 \lambda = \frac{1-m-n}{2}\Big(\frac{U_0}{\Gamma_0} - \frac{2-n}{1-m-n}\Big)>0. \label{eq:lambdaC}
\end{equation}
Furthermore, the self-similar profile $\big(V(\xi),\Theta(\xi),\Sigma(\xi),U(\xi)\big), \ \xi = (t+1)^\lambda x$,  has the  following properties :
 \begin{enumerate}
  \item[(i)] Satisfies the boundary condition at $\xi=0$,
    \begin{equation*}
    {V}(0) = \Gamma_\xi(0) = \Sigma_\xi(0) = {U}_\xi(0)=0, \quad U(0)=U_0, \Gamma(0)=\Gamma_0.
  \end{equation*}
  \item[(ii)] Its asymptotic behavior as $\xi \rightarrow 0$ is given by 
  \begin{equation} \label{eq:ss_asympC0}
  \begin{aligned}
    \Gamma(\xi) &= \frac{1}{a}U(0) + \Gamma^{''}(0)\frac{\xi^2}{2} + o(\xi^2), & \Gamma^{''}(0)&<0,\\
%     \Theta(\xi) &= \Theta(0) + \Theta^{''}(0)\frac{\xi^2}{2} + o(\xi^2), & \Theta^{''}(0)&<0,\\
    \Sigma(\xi) &= \Gamma(0)^{m}{U(0)^n}+ \Sigma^{''}(0)\frac{\xi^2}{2} + o(\xi^2), & \Sigma^{''}(0)&>0, \\
    U(\xi) &= U(0) + U^{''}(0)\frac{\xi^2}{2} + o(\xi^2), & U^{''}(0)&<0,\\
    V(\xi) &= U(0)\xi + U^{''}(0)\frac{\xi^3}{6} + o(\xi^3), & U^{''}(0)&<0.
  \end{aligned}
  \end{equation}
  \item[(iii)] Its asymptotic behavior as $\xi \rightarrow \infty$ is given by
  \begin{equation} \label{eq:ss_asympC1}
  \begin{aligned}
    \Gamma(\xi) &= O(\xi^{\frac{1}{m+n}}), & V(\xi) &= O(1), \\%&    \Theta(\xi) &= O(\xi^{-\frac{1+n}{\alpha-n}}),\\
   \Sigma(\xi) &= O(\xi), &   U(\xi) &= O(\xi^{\frac{1}{m+n}}).
  \end{aligned}
  \end{equation}
 \end{enumerate}
\end{theorem}

\subsection{Emergence of localization}
We describe the emergence of localization of the family of solutions for system (B)  constructed by Theorem \ref{thm1}. The corresponding localized solutions for system (C) constructed by Theorem \ref{thm2} is similar, thus omitted. 

In both cases we replace $t \leftarrow t+1$,
\begin{equation*}
\begin{aligned}
 v(t,x) &= (t+1)^b V((t+1)^\lambda x),  & \theta(t,x) &= (t+1)^c \Theta((t+1)^\lambda x), \\
 \tau(t,x) &= (t+1)^d \Sigma((t+1)^\lambda x), & u(t,x) &= (t+1)^{b+\lambda} U((t+1)^\lambda x),
\end{aligned}
\end{equation*}
so that we interpret 
$$\big(V(\xi),\Theta(\xi),\Sigma(\xi),U(\xi)\big)=\big(v(0,x),\theta(0,x),\tau(0,x),u(0,x)\big)|_{x=\xi},$$  the initial states of the self-similar solutions. 

\begin{itemize}
 \item Initial non-uniformities : The profile $\big(V(\xi),\Theta(\xi),\Sigma(\xi),U(\xi)\big)$ is the initial profile of the self-similar solution. $\Theta(\xi)$ and $U(\xi)$ have a small bump at the origin from the asymptotically flat state. The tip sizes at the origin $\Theta_0$ and $U_0$ are the two parameters that fixes the solution. The velocity $V(\xi)$ is an odd function of $\xi$ that connects $-V_\infty$ and $V_\infty$ as $\xi$ runs from $-\infty$ to $\infty$, where $V_\infty \triangleq \lim_{\xi \rightarrow \infty} V(\xi)$. The slope near the origin is slightly steeper, which reflects the initial non-uniformity in the velocity.
\item Temperature : The temperature is an increasing function of $t$ for a fixed $x$. The growth rate at the origin is faster than any other $x$, which dictates the localization near the origin.
\begin{align*}
 \theta(t,0) &= (1+t)^{\frac{2}{D} + \frac{2+2n}{D}\lambda}\Theta(0),&
 \theta(t,x) &\sim t^{\frac{2}{D} - \frac{(1+n)^2}{D(\alpha-n)}\lambda}|x|^{-\frac{1+\alpha}{\alpha-n}}, \quad \text{as $t \rightarrow \infty$, $x\ne0$.}
\end{align*}

\item Strain rate : The growth rate at the origin is faster than the rest of the points, which dictates the localization near the origin.
\begin{align*}
 u(t,0) &= (1+t)^{\frac{1}{D} + \frac{2+2\alpha}{D}\lambda}U(0), \\
 u(t,x) &\sim t^{\frac{1}{D} - \frac{(1+\alpha)(1+n)}{D(\alpha-n)}\lambda}|x|^{-\frac{1+\alpha}{\alpha-n}}, \quad \text{as $t \rightarrow \infty$, $x\ne0$.}
\end{align*}
\item Stress : The stress is a decreasing function of $t$ for  fixed $x$. However, the decay rate at the origin is faster than the rest of the points.
\begin{align*}
 \tau(t,0) &= (1+t)^{\frac{-2\alpha+n}{D} + \frac{-2\alpha+2n}{D}\lambda}\Sigma(0), \\
 \tau(t,x) &\sim t^{\frac{-2\alpha+n}{D} +\frac{1+n}{D}\lambda}|x|^{-\frac{1+\alpha}{\alpha-n}}, \quad \text{as $t \rightarrow \infty$, $x\ne0$.}
\end{align*}
Note that the rate of the latter is always less than $-\frac{n}{1+n}$ in the valid range of  $\lambda$.
\item Velocity : The velocity $v(x,t)$ is an odd function of $x$. It connects $-v_\infty$ to $v_\infty$, as $x$ runs from $-\infty$ to $\infty$, where $v_\infty\triangleq \lim_{x \rightarrow \infty} v(t,x)$. Because of the scaling law $\xi=(1+t)^\lambda x$, the transition from $-v_\infty$ to $v_\infty$ localizes around the origin as time increases. The slope becomes steeper and steeper and develops a step function type singularity, see Figure 1(c). The far field velocity 
$$v_\infty(t)=(1+t)^{b}V_\infty = (1+t)^{\frac{1}{D} + \frac{1+n}{D}\lambda}V_\infty$$
itself grows at a polynomial rate. This is not in agreement with the uniform shearing motion. This deviation is a consequence of our simplifying assumption for the self-similarity.
\end{itemize}

\begin{figure}[ht]
  \centering
  \subfigure[temperature ($\theta$)]{
  \includegraphics[scale=0.25]{temperature_log} \label{fig:gamma}
  }
  \subfigure[strain rate ($u$)]{
  \includegraphics[scale=0.25]{strain_rate_log} \label{fig:u}
  }
  \subfigure[velocity ($v$)] {
  \includegraphics[scale=0.25]{velocity} \label{fig:v}
  }
  \subfigure[stress ($\sigma$)] {
  \includegraphics[scale=0.25]{stress_log} \label{fig:sigma}
  }
\caption{The localizing solutions for system (B), for $n=0.3$ and $\lambda =0.39$. 
 All graphs except $v$ are in logarithmic scale. See \cite{KLT_2016} for the system (C).} 
\end{figure}


\section{Existence via Geometric singular perturbation theory} \label{sec:idea}

Among steps to prove Theorem \ref{thm1} and \ref{thm2}, we provide the one key step of the existence proof, for system (B). The problem boils down to show the existence of a heteroclinic orbit for the  system. 
\begin{equation} \label{eq:pqrsys}\tag{P}
\begin{aligned}
 \frac{\dpp}{p}&=\Big[\frac{1+\alpha}{1+n}\,\frac{1}{\lambda }\Big(r^{1+n}-c_0\Big)\Big] -\Big[d_1 + q + \lambda pr\Big],\\
 \frac{\dqq}{q}&=\Big[b_1 +\frac{bpr}{q}\Big] -\Big[d_1 + q + \lambda pr\Big],\\
 n\frac{\drr}{r}&=\Big[\frac{\alpha-n}{\lambda(1+n)}\Big(r^{1+n}-c_0\Big)\Big]+\Big[d_1 + q + \lambda pr\Big].
\end{aligned}
\end{equation}
The objective is to construct the heteroclinic orbit that connects equilibrium points
\begin{align*}
 M_0=\Big(0,0,\big(\frac{2}{D} + \frac{2(1+n)}{D} \lambda\big)^{\frac{1}{1+n}}\Big), \quad M_1=\Big(0,1,\big(\frac{2}{D} -\frac{(1+n)^2}{D(\alpha-n)} \lambda\big)^{\frac{1}{1+n}}\Big).
\end{align*}

We describe now briefly how system \eqref{eq:pqrsys} is derived.  The technique we employ was mainly developed in Katsaounis et. al. \cite{KOT14}. Following \cite{KOT14,LT16,LT16_2},  we introduce a series of non-linear transformations described by  \eqref{eq:CAPtoBAR}, \eqref{eq:BARtoTIL} while the definition of $(p,q,r)$-variables is given in \eqref{eq:pqrdef}, 
\begin{equation} \label{eq:CAPtoBAR}
\begin{aligned}
 \bg(\xi)&=\xi^{a_1}\Gamma(\xi), &
 \bv(\xi)&=\xi^{b_1}V(\xi), &
 \bth(\xi)&=\xi^{c_1}\Theta(\xi), \\
 \bs(\xi)&=\xi^{d_1}\Sigma(\xi), &
 \bu(\xi)&=\xi^{b_1+1}U(\xi).
\end{aligned}
\end{equation}
\begin{equation} \label{eq:BARtoTIL}
\begin{aligned}
 \tg(\log\xi)&=\bg(\xi), &
 \tv(\log\xi)&=\bv(\xi), &
 \tth(\log\xi)&=\bth(\xi), \\
 \ts(\log\xi)&=\bs(\xi), &
 \tu(\log\xi)&=\bu(\xi), 
\end{aligned}
\end{equation}
\begin{equation}\label{eq:pqrdef}
 \begin{aligned}
  p \triangleq \frac{\tth^{\,\frac{1+\alpha}{1+n}}}{\ts}, \quad q \triangleq b \frac{\tv}{\ts},  \quad r \triangleq \frac{\tu}{\tth^{\,\frac{1+\alpha}{1+n}}},
 \end{aligned}
\end{equation}
and $\eta\triangleq \log\xi$ the new independent variable $\Big(\frac{df}{d\eta}=\dot{f}\Big)$. 
The system \eqref{eq:pqrsys} has a {\it fast-slow} structure due to $n$ in front of $\dot{r}$. We conduct a Chapman-Enskog type reduction via geometric singular perturbation theory \cite{fenichel_geometric_1979, KUEHN_2015}. The reduced problem becomes a planar dynamical system and  the heteroclinic orbit is obtained by phase space analysis, \cite{LT16_2}. 
\subsection{Critical manifold}
The surface the orbit relaxes is the zero set of the right-hand-side of $\eqref{eq:pqrsys}_3$. The zero set,  that is away from $r=0$ plane,  is the surface specified by
\begin{align*}
 r=\frac{ \frac{\alpha c_0}{\lambda} - d_1 -q }{ \frac{\alpha}{\lambda} + \lambda p}\triangleq h(p,q;n=0), \quad \text{or} \quad  q + \lambda {r}p + \frac{\alpha}{\lambda} \Big( r-r_0\Big)=0.
\end{align*}

 
We take the triangle $T$ in the first quadrant enclosed by $p$-axis, $q$-axis and the contour line $\underbar{r} =  h(p,q;n=0)$ and a compact set $K\supset\supset T$. We set the critical manifold 
\begin{equation}
 G(\lambda,\alpha,n=0) \triangleq \Big\{\, (p,q,r) \;|\; (p,q) \in K, \text{ and } r=\frac{ \frac{\alpha c_0}{\lambda} - d_1 -q }{ \frac{\alpha}{\lambda} + \lambda p} \,\Big\}.
\end{equation}

The system in {\it fast scale} with the independent variable $\tilde{\eta} = \eta/n$ is
\begin{equation}\label{eq:pqr_fast} \tag*{($\tilde{P}$)}
\begin{aligned}
 p^\prime &=np\Big( \Big[\frac{1+\alpha}{1+n}\,\frac{1}{\lambda }\Big(r^{1+n}-c_0\Big)\Big] -\Big[d_1 + q + \lambda pr\Big]\Big), \\
 q^\prime &=nq\Big(\Big[b_1 +\frac{bpr}{q}\Big] -\Big[d_1 + q + \lambda pr\Big]\Big), \\
 r^\prime &=r\Big( \Big[\frac{\alpha-n}{\lambda(1+n)}\Big(r^{1+n}-c_0\Big)\Big]+\Big[d_1 + q + \lambda pr\Big]\Big),
\end{aligned}
\end{equation}
where we denoted $\displaystyle(\cdot)^\prime = \frac{d}{d\tilde{\eta}}(\cdot)$. When $n=0$, $(\tilde{P})|_{n=0}$ reads
\begin{align*}
 p^\prime =0, \quad q^\prime =0, \quad r^\prime=r\Big( \Big[\frac{\alpha}{\lambda}\Big(r-c_0\Big)\Big]+\Big[d_1 + q + \lambda pr\Big]\Big). %\label{eq:fastn0}
\end{align*}
\begin{lemma} \label{lem:normal_hyper}
 $G(\lambda,\alpha,0)$ is a normally hyperbolic invariant manifold with respect to the system $(\tilde{P})|_{n=0}$.
\end{lemma}

\subsection{Chapman-Enskog type reduction}
By the theorem of geometric singular perturbation theory, \cite{fenichel_geometric_1979, KUEHN_2015}, if $n$ is sufficiently small, there exists  the locally invariant manifold $G(\lambda,\alpha,n)$ with respect to \eqref{eq:pqrsys}. Then, on this manifold, $\big(p(\eta),q(\eta)\big)$ satisfies the planar system
\begin{equation} \tag*{(${R}$)} \label{eq:reduced}
\begin{aligned}
 {\dpp}&=p\bigg\{\Big[\frac{1+\alpha}{1+n}\,\frac{1}{\lambda }\Big(h^{1+n}-c_0\Big)\Big] -\Big[d_1 + q + \lambda ph\Big]\bigg\},\\
 {\dqq}&=q\Big(1-q-\lambda p h\Big) + bph,\\
\end{aligned}
\end{equation}
where $h$ stands for $h(p,q;n)$.


\subsection{Confinement of the orbit}
\begin{lemma}
The triangle $T$  is positively invariant for the system $(R)$ when $n=0$.
\end{lemma}
We can compute the inward normal component of $(\dot{p},\dot{q})$ on the boundary of the triangle $T$ for $(R)|_{n=0}$:
\begin{align*}
 \dot{p} &= -\frac{D}{\alpha} p\Big(d_1+q+\lambda p h\Big),\\
 \dot{q} &= q\Big(1-q-\lambda p h\Big) + b p h.
\end{align*}
Essential calculation is on the hypotenuse and the fact that it is the contour line $\underbar{r}=h(p,q,n=0)$ helps us obtain the estimate. Define $\underbar{p}$ and $\underbar{q}$ to be the $p$-intercept and $q$-intercept of the contour line respectively : $\underbar{q} = \lambda \underbar{p}\underbar{r} = \frac{\alpha}{\lambda}(r_0-\underbar{r})$. With  $(-\underbar{q},-\underbar{p})$ being the inward normal vector, the  inward normal component on the hypotenuse is 
\begin{align*}
    (\dot{p},\dot{q}) \cdot (-\underbar{q},-\underbar{p}) &= \frac{D}{\alpha}\underbar{q}p\Big(d_1 + q + \lambda p \underbar{r}\Big) -\underbar{p}\bigg\{q\Big(1-q-\lambda p \underbar{r}\Big) + b p \underbar{r}\bigg\}\\
    &=\frac{D}{\alpha}\underbar{q}p\Big(d_1 +\underbar{q}\Big) -\underbar{p}\bigg\{\Big(\underbar{q}-\lambda p \underbar{r}\Big)\Big(1-\underbar{q}\Big) + b p \underbar{r}\bigg\}\\
    &=-\underbar{p}\underbar{q}(1-\underbar{q}) + \underbar{q}p\Big(\frac{D}{\alpha}d_1 + \frac{D}{\alpha}\underbar{q} + (1-\underbar{q}) - \frac{b}{\lambda}\Big)\\
    &=-\underbar{p}\underbar{q}(1-\underbar{q}) + \underbar{q}p\frac{1+\alpha}{\lambda}\Big(\frac{1}{1+\alpha}-\underbar{r}\Big)\\
    &\ge -\underbar{p}\underbar{q}(1-\underbar{q}) \quad \text{for $\underbar{r} < \frac{1}{1+\alpha}$}\\
    &\ge \delta_0 > 0.
\end{align*}

\begin{lemma}
The triangle $T$  is positively invariant for the system $(R)$ provided $n$ is sufficiently small.
\end{lemma}
Now the hypotenuse is not anymore a contour line of the function $h(p,q;\lambda,\alpha,n)$. We arrange terms of right-hand-sides of $(R)$ in the form
\begin{align*}
 {\dpp}&=p\bigg\{\Big[\frac{1+\alpha}{\lambda }\Big(\underbar{r}-c_0\Big)\Big] -\Big[d_1 + q + \lambda p\underbar{r}\Big]\bigg\} \\
 &+ \underbrace{p\bigg\{\Big[\frac{1+\alpha}{1+n}\,\frac{1}{\lambda }\Big(h^{1+n}-c_0\Big)\Big]-\Big[\frac{1+\alpha}{\lambda }\Big(\underbar{r}-c_0\Big)\Big] -\lambda p(h-\underbar{r})\Big]\bigg\}}_\text{$\triangleq g_1(p,q,n)$},\\
 {\dqq}&=q\Big(1-q-\lambda p \underbar{r}\Big) + bp\underbar{r} + \underbrace{(-q\lambda p+b) (h-\underbar{r})}_\text{$\triangleq g_2(p,q,n)$}.
\end{align*}
Since $h$ is a smooth function of $n$ and $D$ is compact, provided $n$ is sufficiently small, we have an estimate
\begin{equation*}
 |g_1(p,q,n)| + |g_2(p,q,n)| \le C_0 n, \quad \text{where $C_0$ does not depend on $p$, $q$, and $n$.}
\end{equation*}
Therefore
$$ (\dot{p},\dot{q}) \cdot(-\underbar{q},-\underbar{p}) \ge \delta_0 + C_0'n \quad \text{for another uniform constant $C_0'$}.$$
Take $n$ sufficiently small so that the last expression is positive. After having the orbit confined in the positive invariant set $T$, we further conduct the phase space analysis to capture the heteroclinic orbit but is omitted here.
\subsection*{Acknowledgements}
This research was supported by King Abdullah University of Science and Technology (KAUST).

\begin{thebibliography}{10}

% \bibitem{baxevanis_adaptive_2010}
% {\sc Th.~Baxevanis, Th~Katsaounis, and A.~E. Tzavaras},  
% Adaptive finite element computations of shear band formation, 
%   {\em Math. Models  Methods Appl. Sci.} {\bf 20}  (2010),  423--448.

\bibitem{bertsch_effect_1991}
{\sc M.~Bertsch, L.~Peletier, and S.~Verduyn~Lunel}, 
The effect of temperature dependent viscosity on shear flow of  incompressible fluids,
{\em SIAM J. Math. Anal.} {\bf 22 } (1991), 328--343.

% \bibitem{CB99} 
% {\sc L.~ Chen and R.C.~Batra },
% The asymptotic structure of a shear band in mode-II deformations.
% {\em International Journal of Engineering Science} {\bf 37} (1999),  895--919.

\bibitem{dafermos_adiabatic_1983}
{\sc C.~M. Dafermos and L.~Hsiao}, 
Adiabatic shearing of incompressible fluids with temperature-dependent viscosity.
{\em Quart.  Applied Math.} {\bf 41} (1983), 45--58.

% \bibitem{estep_2001}
% {\sc Donald~J Estep, Sjoerd M~Verduyn Lunel, and Roy~D Williams}, 
% {Analysis of Shear Layers in a Fluid with Temperature-Dependent Viscosity},
%  {\em  J. Comp. Physics}  {\bf 173} (2001), 17--60.

\bibitem{fenichel_geometric_1979}
{\sc N.~Fenichel}, 
Geometric singular perturbation theory for ordinary differential equations, 
{\it J. Differ. Equations} {\bf 31} (1979), 53--98.

\bibitem{FM87}
{\sc C.~Fressengeas and A.~Molinari}, 
Instability and localization of plastic flow in shear at high strain rates, 
  {\em J.  Mech. Physics of Solids} {\bf 35} (1987), 185--211.

% \bibitem{jones_geometric_1995}
% {\sc C.~K. R.~T. Jones}, 
% Geometric singular perturbation theory, in {\it Dynamical systems}, LNM {\bf 1609} (Springer Berlin Heidelberg 1995) 44--118.
  
\bibitem{KT09}
{\sc Th.~Katsaounis and A.E.~Tzavaras}, 
 Effective equations for localization and shear band formation, 
 {\em SIAM J. Appl. Math.}  {\bf 69} (2009), 1618--1643.
  
\bibitem{KOT14}
{\sc Th.~Katsaounis, J.~Olivier, and A.E.~Tzavaras}, 
Emergence of coherent localized structures in shear deformations of
  temperature dependent fluids, {\em Archive for Rational Mechanics and Analysis}, (to appear).

\bibitem{KUEHN_2015}
{\sc C.~ Kuehn}, 
{\it Multiple time scale dynamics}, Applied Mathematical Sciences, Vol. {\bf 191} (Springer Basel 2015).
  
\bibitem{LT16}
{\sc M-G.~Lee and A.E.~Tzavaras},
Existence of localizing solutions in plasticity via the geometric singular perturbation theory, 
{\em SIAM J. Appl. Dyn. Syst.}, (to appear).

\bibitem{LT16_2}
{\sc M-G.~Lee and A.E.~Tzavaras},
Localization in adiabatic process of viscous shear flow via geometric theory of singular perturbation, (in preparation).

\bibitem{KLT_2016}
{\sc Th. Katsaounis, M-G. Lee, and A.E. Tzavaras}, 
Localization in inelastic rate dependent shearing deformations, 
{\em J.  Mech. Physics of Solids} {\bf 98} (2017), 106--125.

% \bibitem{perko_differential_2001}
% {\sc L.~Perko}, 
% {\it Differential equations and dynamical systems 3rd. ed.}, TAM {\bf 7} (Springer-Verlag New York 2001).  

\bibitem{SC89}
{\sc T.~G. Shawki and R.~J. Clifton}, 
Shear band formation in thermal viscoplastic materials, 
{\em Mechanics of Materials} {\bf 8 } (1989), 13--43.

\bibitem{Tz_1986}
{\sc A.E. Tzavaras},
Shearing of materials exhibiting thermal softening or temperature dependent viscosity,
{\em Quart.  Applied Math.} {\bf 44} (1986), 1--12.

% \bibitem{Tz_1987}
% {\sc A.E. Tzavaras}, 
% Effect of thermal softening in shearing of strain-rate dependent materials.
% {\em Archive for Rational Mechanics and Analysis}, {\bf 99} (1987), 349--374.

\bibitem{tzavaras_nonlinear_1992}
%\leavevmode\vrule height 2pt depth -1.6pt width 23pt,
{\sc A.E. Tzavaras}, 
Nonlinear analysis techniques for shear band formation at high strain-rates, 
% {\it Applied Mechanics Reviews} 
{\it Appl. Mech. Rev.}
{\bf  45} (1992), S82--S94.

% \bibitem{walter_1992}
% {\sc J.W.~Walter}, 
% Numerical experiments on adiabatic shear band formation
%   in one dimension.
%   {\em International Journal of Plasticity} {\bf  8} (1992), 657--693.

\bibitem{ZH_1944}
{\sc C.~Zener and J.~H. Hollomon}, 
Effect of strain rate upon plastic flow of steel,
% {\it  Journal of Applied Physics} 
{\it J. Appl. Phys.}
{\bf 15} (1944), 22--32.
\end{thebibliography}

\end{document}	


\end{document}















\documentclass[a4paper,11pt]{article}

\usepackage[margin=2cm]{geometry}
\usepackage{setspace}
%\onehalfspacing
\doublespacing
%\usepackage{authblk}
\usepackage{amsmath}
\usepackage{amssymb}
\usepackage{amsthm}

\usepackage[notcite,notref]{showkeys}

% \usepackage{psfrag}
\usepackage{graphicx,subfigure}
\usepackage{color}
\def\red{\color{red}}
\def\blue{\color{blue}}
%\usepackage{verbatim}
% \usepackage{alltt}
%\usepackage{kotex}

\usepackage{enumerate}

%%%%%%%%%%%%%% MY DEFINITIONS %%%%%%%%%%%%%%%%%%%%%%%%%%%

\def\tr{\,\textrm{tr}\,}
\def\div{\,\textrm{div}\,}
\def\sgn{\,\textrm{sgn}\,}

\def\th{\tilde{h}}
\def\tx{\tilde{x}}
\def\tk{\tilde{\kappa}}

\def\tg{{\tilde{\gamma}}}
\def\tv{{\tilde{v}}}
\def\tth{{\tilde{\theta}}}
\def\ts{{\tilde{\tau}}}
\def\tu{{\tilde{u}}}

\def\dtg{\dot{\tilde{\gamma}}}
\def\dtv{\dot{\tilde{v}}}
\def\dtth{\dot{\tilde{\theta}}}
\def\dts{\dot{\tilde{\tau}}}
\def\dtu{\dot{\tilde{u}}}

\def\dpp{\dot{p}}
\def\dqq{\dot{q}}
\def\drr{\dot{r}}
\def\dss{\dot{s}}

\def\ta{{\tilde{a}}}
\def\tb{{\tilde{b}}}
\def\tc{{\tilde{c}}}
\def\td{{\tilde{d}}}




\def\bx{\bar{x}}
\def\bm{\bar{\mathbf{m}}}
\def\K{\mathcal{K}}
\def\E{\mathcal{E}}
\def\del{\partial}
\def\eps{\varepsilon}

\newcommand{\tcr}{\textcolor{red}}
\newcommand{\tcb}{\textcolor{blue}}

\newcommand{\ubar}[1]{\text{\b{$#1$}}}
\newtheorem{theorem}{Theorem}
\newtheorem{lemma}{Lemma}[section]
\newtheorem{proposition}{Proposition}[section]
%\newtheorem{definition}{Definition}[section]
\newtheorem{remark}{Remark}[section]

%%%%%%%%%%%%%%%%%%%%%%%%%%%%%%%%%%%%%%%%%%%%%%%%%%%%%%%%%%
\begin{document}
\title{Note for the Self-similar shear bands problem}
\author{Min-Gi Lee\footnotemark[1]}
% \and Athanasios Tzavaras\footnotemark[1]\  \footnotemark[3]  \footnotemark[4]}
\date{}

\maketitle
\renewcommand{\thefootnote}{\fnsymbol{footnote}}
% \footnotetext[1]{Computer, Electrical and Mathematical Sciences \& Engineering Division, King Abdullah University of Science and Technology (KAUST), Thuwal, Saudi Arabia}
% \footnotetext[2]{Department of Mathematics and Applied Mathematics, University of Crete, Heraklion, Greece}
% \footnotetext[3]{Institute of Applied and Computational Mathematics, FORTH, Heraklion, Greece}
% \footnotetext[4]{Corresponding author : \texttt{athanasios.tzavaras@kaust.edu.sa}}
%\footnotetext[4]{Research supported by the King Abdullah University of Science and Technology (KAUST) }
\renewcommand{\thefootnote}{\arabic{footnote}}


\maketitle

\tableofcontents
% \begin{abstract}
% abstract
% \end{abstract}

\section{The model description}
We consider a 1-d shear deformation of a material whose material law of stress depends on 1) temperature, 2) strain, 3) strain rate. The motion is described by following field variables,
\begin{equation} \label{eq:vars}
\begin{aligned}
 \gamma(t,x) &: \text{strain}\\
 u(t,x)=\gamma_t &: \text{strain rate}\\
 v(t,x) &: \text{vertical velocity}\\
 \theta(t,x) &: \text{temperature}\\
 \tau(t,x) &: \text{stress}
\end{aligned}
\end{equation}
The material exhibits 1) temperature-softening, 2) strain-hardening, 3) rate-hardening. we denote the shear stress
$$ \tau = \tau(\theta,\gamma,u). $$
and study a model
\begin{equation}
 \tau = \theta^{-\alpha}\gamma^m u^n. \label{eq:stresslaw}
\end{equation}

A few forehand perspectives :
\begin{enumerate}
 \item The regime where $-\alpha+m+n <0$ will exhibit the localization, whereas the regime $-\alpha+m+n > 0$ will exhibit stabilization. {\blue Can we rigorously study the linearize stability to the uniform shearing solution?}
 \item The uniform shearing solution will appear as one of the self-similar solution by a specific $\lambda$ that is negative.
\end{enumerate}
\subsection{A system of conservation laws}
For the field variables \eqref{eq:vars}, equations describing the deformation are given by
\begin{align}
 \gamma_t &= u, \quad \text{(kinematic compatibility)} 	\label{eq:g}\\
 v_t &= \tau_x, \quad \text{(momentum conservation)} 	\label{eq:v}\\
 \theta_t &= \tau u \quad \text{(energy conservation)}	\label{eq:th}\\
 \tau &=\theta^{-\alpha}\gamma^m u^n.			\label{eq:tau}
\end{align}

\subsubsection{temperature softening model}
We first focus on the problem where 
$$ \tau = \tau(\theta,u) = \theta^{-\alpha}u^n.$$
Then the first equation \eqref{eq:g} drops out from the system, and we focus on the system
\begin{equation} \label{eq:orisys}
 \begin{aligned}
  v_t &= \tau_x,\\
  \theta_t &= \tau u,\\
  \tau &=\theta^{-\alpha}u^n.
 \end{aligned}
\end{equation}


\subsection{Scale invariance property of the system}
The system \eqref{eq:orisys} admits a scale invariance property. Suppose $(v,u,\theta,\tau)$ is a solution. Then a rescaled version of it 
\begin{align*}
 v_\rho(t,x) &= \rho^bv(\rho^{-1}t,\rho^\lambda x),\\
 u_\rho(t,x) &= \rho^{b+\lambda}u(\rho^{-1}t,\rho^\lambda x),\\
 \theta_\rho(t,x) &= \rho^c\theta(\rho^{-1}t,\rho^\lambda x),\\
 \tau_\rho(t,x) &= \rho^d\tau(\rho^{-1}t,\rho^\lambda x)
\end{align*}

Calculations :
\begin{align*}
 &\text{let} \;f(t,x) = \rho^k F(\rho^{-1}t,\rho^\lambda x), \\
 &\partial_t f(t,x) = \rho^k \partial_{t'} F(\rho^{-1}t,\rho^\lambda x) \rho^{-1} = \rho^{k-1} \partial_{t'}F, \\
 &\partial_x f(t,x) = \rho^k \partial_{x'} F(\rho^{-1}t,\rho^\lambda x) \rho^\lambda = \rho^{k+\lambda} \partial_{x'}F.
\end{align*}
Relations for invariance :
\begin{align*}
 b-1 = d+\lambda, \quad c-1 = d+b+\lambda, \quad d = -\alpha c + m a + n (b+\lambda)
\end{align*}
From above, we reach to exponents
\begin{align*}
 D & = 1+2\alpha-m-n,\\
 a&= \frac{2+2\alpha-n}{D} + \frac{2+2\alpha}{D}\lambda =: a_0 + a_1 \lambda, & b&=\frac{1+m}{D} + \frac{1+m+n}{D}\lambda =: b_0 + b_1\lambda,\\
 c&=\frac{2(1+m)}{D} + \frac{2(1+m+n)}{D}\lambda =: c_0 + c_1\lambda, & d&=\frac{-2\alpha + 2m +n}{D} + \frac{-2\alpha+2m+2n}{D}\lambda =: d_0 + d_1\lambda
\end{align*}
for each $\lambda \in \mathbb{R}$. For localization, $\lambda>0$. But uniform shearing solution takes a negative $\lambda$.
\subsection{Self-Similar variables}
We try the solutions of type, i.e. put $\rho =t$ in the rescaling,
\begin{align*}
 \gamma(t,x) &= t^a\Gamma(t^\lambda x),\\
 v(t,x) &= t^b V(t^\lambda x),\\
 \theta(t,x) &= t^c \Theta(t^\lambda x),\\
 \tau(t,x) &= t^d \Sigma(t^\lambda x),\\
 u(t,x) &= t^{b+\lambda} U(t^\lambda x)
\end{align*}
and set $\xi = t^\lambda x$.

Calculations:
\begin{align*}
 &\text{Suppose } \; f(t,x) = t^k F(t^\lambda x),\\
 &\partial_t f = k t^{k-1} F + t^k F' \lambda t^{\lambda-1} x = t^{k-1} (kF + \lambda\xi F'),\\
 &\partial_x f = t^k F' t^\lambda = t^{k+\lambda} F',
\end{align*}

At \eqref{eq:g}-\eqref{eq:tau}:

\begin{align*}
 t^{a-1}(a \Gamma(\xi) + \lambda \xi \Gamma'(\xi)) &= t^{b+ \lambda} U(\xi),\\
 t^{b-1}(b V(\xi) + \lambda \xi V'(\xi)) &= t^{d+ \lambda} \Sigma'(\xi)\\
 t^{c-1}(c \Theta(\xi) + \lambda \xi \Theta'(\xi))&=t^{b+d+\lambda} \Sigma U(\xi),\\
 t^d\Sigma(\xi) &= t^{-\alpha c +ma +n(b+ \lambda)} \Theta(\xi)^{-\alpha} \Gamma(\xi)^m U(\xi)^n,\\
 t^{b+\lambda}V'(\xi)&=t^{b+\lambda}U(\xi)
\end{align*}
\begin{equation}
\begin{aligned}
 a \Gamma(\xi) + \lambda \xi \Gamma'(\xi) &= U(\xi),\\
 b V(\xi) + \lambda \xi V'(\xi) &= \Sigma'(\xi)\\
 d \Theta(\xi) + \lambda \xi \Theta'(\xi)&=\Sigma(\xi) U(\xi),\\
 \Sigma(\xi) &= \Theta(\xi)^{-\alpha} \Gamma(\xi)^m U(\xi)^n,\\
 V'(\xi)&=U(\xi)
\end{aligned} \label{eq:ss-odes}
\end{equation}
\subsection{de-singularization}
Introduce new field variables 
\begin{equation}
\begin{aligned}
 \Gamma(\xi) &= \xi^\ta \tg(\xi),\\
 V(\xi)&=\xi^\tb \tv(\xi),\\
 \Theta(\xi)&=\xi^\tc \tth(\xi),\\
 \Sigma(\xi)&=\xi^\td \ts(\xi),\\
 U(\xi)&=\xi^{\tb-1} \tu(\xi). 
\end{aligned}
\end{equation}
Then at \eqref{eq:ss-odes}:
\begin{align*}
 \xi^\ta\Big( a\tg + \lambda \ta \tg + \lambda\xi\tg'\Big) &=\xi^{\tb-1} \tu,\\
 \xi^\tb\Big( b\tv + \lambda \tb \tv + \lambda\xi\tv'\Big) &=\xi^{\td-1} \Big(\td\ts + \xi\ts'\Big),\\
 \xi^\tc\Big( c\tth+ \lambda \tc \tth+ \lambda\xi\tth'\Big)&=\xi^{\td+\tb-1} \ts\tu,\\
 \xi^\td\ts &= \xi^{-\alpha \tc +m\ta +n(\tb-1)} \tth^{-\alpha} \tg^m \tu^n,\\
 \xi^{\tb-1}\Big(\tb\tv + \xi \tv'\Big)&= \xi^{\tb-1} \tu.
\end{align*}

$\ta, \tb, \tc, \td$ such that
\begin{align*}
 &\ta=\tb-1, \quad \tb=\td-1, \quad \tc=\td+\tb-1,\quad \td = -\alpha \tc + m\ta +n(\tb-1) \\
 \Longrightarrow \quad&\ta = -a_1, \quad \td = -d_1, \quad \tc = -c_1, \quad \tb=-b_1.
\end{align*}

\begin{equation}
 \begin{aligned}
  a_0\tg + \lambda\xi\tg' &=\tu,\\
  b_0\tv + \lambda\xi\tv' &=-d_1 \ts + \xi\ts',\\
  c_0\tth+ \lambda\xi\tth'&=\ts\tu,\\
  \ts &=\tth^{-\alpha}\tg^m\tu^n,\\
  -b_1\tv+\xi\tv' &= \tu.
 \end{aligned}
\end{equation}

Introduce the new independent variable $\eta = \log\xi$.
\section{$(p,q,r,s)$-system derivation}

\begin{enumerate}
 \item The equation \eqref{eq:th} can be rewritten in the form
 $$ \Big(\frac{1}{1+\alpha} \tth^{1+\alpha}\Big)_t = \frac{\tu^n}{\tg^n} \Big(\frac{1}{1+m+n} \tg^{1+m+n}\Big)_t$$
 and we expect, at least for the self-similar solutions, that
 $$ \frac{ \frac{1}{1+\alpha} \tth^{1+\alpha} }{ \frac{1}{1+m+n} \tg^{1+m+n} }  = 1 + \mathcal{O}(n) \overset{put}{=} s^n, \quad \text{for some $s$}. $$
 \item If so, we can define the variable 
 $$r = \Big(\Big(\frac{1+m+n}{1+\alpha}\Big)^{\frac{\alpha}{(1+\alpha)}}\tau \gamma^{\frac{\alpha-m-n}{1+\alpha}}\Big)^{\frac{1}{n}} = \frac{\tu}{\tg}\,s^{-\frac{\alpha}{1+\alpha}} \sim \mathcal{O}(1) $$
 and when $\alpha=0$, it reduces to $\frac{\tu}{\tg}$. The advantage of this definition to the $\frac{\tu}{\tg}$ is that the latter expression couples to the $\tth$ too whereas $r$ here does not. 
\end{enumerate}

Auxiliary calculations:
\begin{align*}
 \frac{\dtg}{\tg} &= \frac{1}{\lambda }\Big(\frac{\tu}{\tg}-a_0\Big),\\
 \frac{\dts}{\ts} &= d_1+ b_0\frac{\tv}{\ts} + \lambda \frac{\dtv}{\ts} = d_1+ b_0\frac{\tv}{\ts} + \lambda \Big(b_1 + \frac{\tu}{\tv}\Big)\frac{\tv}{\ts} = d_1 + b\frac{\tv}{\ts} + \lambda\frac{\tu}{\tv}\frac{\tv}{\ts} ,\\
 \frac{\dtth}{\tth}&=\frac{1}{\lambda }\Big(\frac{\ts\tu}{\tth}-c_0\Big),\\
 0&=\frac{\dts}{\ts} +\alpha \frac{\dtth}{\tth} - m \frac{\dtg}{\tg} - n\frac{\dtu}{\tu},\\
 \frac{\dtv}{\tv}&= b_1 +\frac{\tu}{\tv}
\end{align*}

Define 
\begin{equation}\label{eq:pqrsdef}
 \begin{aligned}
  p &= \frac{\tg}{\ts}, & q&=b \frac{\tv}{\ts},\\
  r &= \Big(\Big(\frac{1+m+n}{1+\alpha}\Big)^{\frac{\alpha}{(1+\alpha)}}\tau \gamma^{\frac{\alpha-m-n}{1+\alpha}}\Big)^{\frac{1}{n}} , & s&=z^{\frac{1}{n}}, \quad z=\frac{ \frac{1}{1+\alpha} \tth^{1+\alpha} }{ \frac{1}{1+m+n} \tg^{1+m+n} }.
 \end{aligned}
\end{equation}

We have
\begin{align*}
 \frac{\dpp}{p}&=\frac{\dtg}{\tg} - \frac{\dts}{\ts}& &=\left[\frac{1}{\lambda }\Big(\frac{\tu}{\tg}-a_0\Big)\right] & &-\left[d_1 + b\frac{\tv}{\ts} + \lambda\frac{\tu}{\tv}\frac{\tv}{\ts}\right]\\
 \frac{\dqq}{q}&=\frac{\dtv}{\tv} - \frac{\dts}{\ts}& &=\left[b_1 +\frac{\tu}{\tv}\right] & &-\left[d_1 + b\frac{\tv}{\ts} + \lambda\frac{\tu}{\tv}\frac{\tv}{\ts}\right]\\
 n\frac{\drr}{r}&=n\frac{\dtu}{\tu} -n\frac{\dtg}{\tg}& &=\left[\frac{\alpha-m-n}{\lambda(1+\alpha) }\Big(\frac{\tu}{\tg}-a_0\Big)\right]& &+\left[d_1 + b\frac{\tv}{\ts} + \lambda\frac{\tu}{\tv}\frac{\tv}{\ts}\right]\\
 & & & & &+ \Big[\frac{\alpha}{1+\alpha}\, \Big(\frac{1+m+n}{\lambda} \Big(\frac{r^n}{z}-1\Big)r + \frac{n}{\lambda}\Big)\Big]\\
 \frac{\dot{z}}{z} &= (1+\alpha)\frac{\dtth}{\tth} - (1+m+n)\frac{\dtg}{\tg} & &=\left[\frac{-1-m-n}{\lambda }\Big(\frac{\tu}{\tg}-a_0\Big)\right] & &+ \left[\frac{1+\alpha}{\lambda }\Big(\frac{\ts\tu}{\tth}-c_0\Big)\right].%\\
%  \dot{s} &=\frac{\partial s}{\partial (z-1)} \dot{z} &&= \frac{1+m}{\lambda}\frac{\partial s}{\partial (z-1)} \bigg\{z\big[-r - \frac{n}{D}\Big]+ ru^n\bigg\}.
\end{align*}
Noticing that $\displaystyle a_0(1+m+n)-c_0(1+\alpha)=n$ and that
\begin{align*}
 \frac{\ts\tu}{\tth} = \frac{\tg^{1+m+n}}{\tth^{1+\alpha}}\Big(\frac{\tu}{\tg}\Big)^{1+n} = \frac{1+m+n}{1+\alpha} \frac{1}{z}\,\Big(\frac{\tu}{\tg}\Big)^{1+n} =  \frac{1+m+n}{1+\alpha} \frac{r^{1+n}}{z},
\end{align*}
\begin{align*}
\frac{\dot{z}}{z} = \frac{1+m+n}{\lambda}\Big(\frac{r^{1+n}}{z} - r \Big) + \frac{n}{\lambda} = \frac{1+m+n}{\lambda}\,r\Big(\frac{r^{n}}{z} - 1 \Big) + \frac{n}{\lambda}
\end{align*}

\begin{enumerate}
 \item  Equation on $z$ in the form of
 $$ \dot{z-1} = -\frac{1+m+n}{\lambda}\, r (z-1) + \frac{1+m+n}{\lambda}\, r(r^n-1) + n\frac{z}{\lambda} = -\frac{1+m+n}{\lambda}\, r (z-1) + \mathcal{O}(n)$$
 dictates that the $z$ will relax to $z=1 + \mathcal{O}(n)$. When $n=0$, it exactly relaxes to $z\equiv1$, provided $r>0$.
 \item Furthermore, once $z = 1+\mathcal{O}(n)$ at some point, then it will remain to be so (using gronwall's inequality), at least for the finite time, suggesting the invariant manifold that comprises these orbits around $z\equiv1$.
 \item Let us suppose the existence of the invariant manifold $\bar{z}(p,q)$ that is of $1+\mathcal{O}(n)$. Since we are only interested in the orbits that are on the invariant manifold or that are close enough to the manifold, we can safely assume that
 $$ |z-1| < \sqrt{n}, \quad 0<n\ll1.$$
 We observe that the relaxation time of the variable $z$ is of $\mathcal{O}(1)$, and when it is on the manifold, $\dot{z} \sim O(n)$. Compare this to the fast variable $r$: $r$ relaxes with speed $\mathcal{O}(1/n)$ and when it is on the manifold, $\dot{r} \sim O(1)$.
 
 We may technically define a variable that has the fast relaxation time.  When $|z-1| < \sqrt{n}$, $s$ defined by the relation 
 $$ s = \sgn{(z-1)}\Big(\frac{|z-1|}{\sqrt{n}}\Big)^{\frac{1}{n}}, \quad \Longleftrightarrow z = 1+\sqrt{n}s^n, \quad \text{for $|z-1|<\sqrt{n}.$}$$
 is uniformly bounded and is invertible for fixed $n$. If $z$ relaxes to the manifold $\bar{z}=1+\mathcal{O}(n)$ then $s$ relaxes to the manifold $\bar{s}\sim \sqrt{n}^{\frac{1}{n}}$ for $n>0$, and $\displaystyle \lim_{n \rightarrow 0+} \sqrt{n}^{\frac{1}{n}}=0$, which is consistent with $z\equiv1$.
\end{enumerate}




$(p,q,r,s)$-system:
\begin{align*}
 \dpp&=p\bigg\{\left[\frac{1}{\lambda }\Big(r-a_0\Big)\right] & &-\left[d_1 + q + \lambda pr\right]\bigg\}\\
 \dqq&=q\bigg\{\left[b_1 +\frac{bpr}{q}\right] & &-\left[d_1 + q + \lambda pr\right]\bigg\}\\ 
 n\drr&=r\bigg\{\left[\frac{\alpha-m-n}{\lambda(1+\alpha) }\Big(r-a_0\Big)\right]& &+\left[d_1 + q + \lambda pr\right]+ \Big[\frac{\alpha}{1+\alpha}\, \Big(\frac{1+m+n}{\lambda} \Big(\frac{r^n}{1+\sqrt{n}s^n}-1\Big)r + \frac{n}{\lambda}\Big)\Big]\bigg\}\\
 n\dss&=-\frac{1+m+n}{\lambda}rs & &+ \sqrt{n}s^{1-n}\Big( \frac{1+m+n}{\lambda} \, r\frac{r^n-1}{n} + \frac{1+\sqrt{n}s^n}{\lambda}\Big).
\end{align*}

\begin{remark}
 \begin{enumerate}
  \item 
  \item Note that $s$ is not a fast variable. Even though $\displaystyle\frac{ \frac{1}{1+\alpha} \tth^{1+\alpha} }{ \frac{1}{1+m+n} \tg^{1+m+n} }$ relaxes to the manifold $1 + \mathcal{O}(n)$, the relaxation time is not of $\mathcal{O}(\frac{1}{n})$ but is of $\mathcal{O}(1)$.
  
 \end{enumerate}
\end{remark}



$(p,q,r,s)$-system:

\begin{equation}
\begin{aligned}
  {\dpp}&=p\bigg\{\Big[\frac{1}{\lambda }\Big(r-a_0\Big)\Big] -\Big[d_1 + q + \lambda p r\Big]\bigg\}\\
  {\dqq}&=q\bigg\{b_1-d_1 + \lambda p r\bigg\} +bpr,\\
 n{\drr}&=r\bigg\{\left[\frac{\alpha-m-n}{\lambda(1+\alpha) }\Big(\frac{\tu}{\tg}-a_0\Big)\right]& &+\left[d_1 + b\frac{\tv}{\ts} + \lambda\frac{\tu}{\tv}\frac{\tv}{\ts}\right]\bigg\}\\
 n\frac{\dot{s}}{s} &= \frac{1+m+n}{\lambda}\Big(r^{1+n}s^{\frac{\alpha-n}{1+\alpha}} - r \Big) + \frac{n}{\lambda}
\end{aligned}
\end{equation}

\begin{remark}
 In the case of the variable $r$, assuming $\tg, \ts,\tth \sim \mathcal{O}(1)$, the exponent $n$ is natural and the relaxation time scale $\drr \sim \mathcal{O}(\frac{1}{n})$. For the case of $\displaystyle\left(\frac{ \frac{1}{1+\alpha}\tth^{1+\alpha}}{ \frac{1}{1+m}\tg^{1+m} } -1 \right)$, there is no preferred relaxation time scale. For the time being, we do not specify the function $s(n,z-1)$ in the upcoming calculations.
\end{remark}
\begin{align*}
 \dot{s} =\frac{\partial s}{\partial (z-1)} \dot{z}
 &= \frac{1+m}{\lambda}\frac{\partial s}{\partial (z-1)} z\bigg\{-r - \frac{n}{D}+ \frac{r}{z}u^n\bigg\}\\
 &=\frac{1+m}{\lambda}\frac{\partial s}{\partial (z-1)} \bigg\{r(u^n-1) +r(1-z) -n\frac{z}{D}\bigg\}.
\end{align*}
We choose $s(n,z)$ such that
\begin{enumerate}
 \item As $(n,z-1) \rightarrow (0,0)$, $s(n,z-1) \rightarrow 0$, 
 \item For fixed $n$, the map $z \mapsto s$ is invertible, and for inverse $z=z(n,s)\rightarrow 1$ as $(n,s) \rightarrow (0,0)$.
\end{enumerate}
We set 
$$ s(n,z-1) = \frac{(z-1)^{\frac{1}{n}}}{n}, \quad \text{or} \quad z= 1+ns^n$$
and look for solutions of this form. Then we have


\begin{align*}
 \frac{\ts\tu}{\tth} &= \frac{1+m}{1+\alpha} \frac{r}{z}\,u^n = \frac{1+m}{1+\alpha} r + n\frac{1+m}{1+\alpha} \frac{r}{1+ns^n}\Big(\frac{u^n-1}{n}-s^n\Big),%\\
%  \frac{\tu}{\tv}\frac{\tv}{\ts}&=\frac{\ts}{\tv} \frac{\tg}{\ts} \frac{\tu}{\tg} \frac{\tv}{\ts} = pr.
\end{align*}





\begin{equation}
\begin{aligned}
  {\dpp}&=p\bigg\{\Big[\frac{1}{\lambda }\Big(r-a_0\Big)\Big] -\Big[d_1 + q + \lambda p r\Big]\bigg\}\\
  {\dqq}&=q\bigg\{b_1-d_1 + \lambda p r\bigg\} +bpr,\\
 n{\drr}&=r\bigg\{\Big[\frac{-m-n}{\lambda }\Big(r-a_0\Big)\Big]+\Big[d_1 + q + \lambda p r\Big]+\Big[\frac{\alpha}{\lambda }\Big(\frac{1+m}{1+\alpha}r-c_0\Big)\Big] + n\Big[\frac{\alpha}{\lambda }\frac{1+m}{1+\alpha} \frac{r}{1+ns^n}\Big(\frac{u^n-1}{n}-s^n\Big)\Big]\bigg\}\\
 n\dot{s}&=\frac{1+m}{\lambda}s^{1-n} \left\{r\frac{u^n-1}{n} - rs^n -\frac{1+ns^n}{D}\right\}.
\end{aligned}
\end{equation}

\section{Normally hyperbolic invariant manifold}
\section{Equilibrium points, Linear structure}
\subsection{Characterization of the heteroclinic orbit : why and how}
\subsection{Asymptotic behavior of self-similar variables in $\xi$}
\section{A $k$-parameter family of shear banding solutions}
\subsection{Asymptotic behavior of field variables in $t$ and $x$}
\section{Existence via Geometric theory of singular perturbation}



\end{document}