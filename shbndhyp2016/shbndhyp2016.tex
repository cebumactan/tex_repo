%%%%%%%%%%%%%%%%%%%%%%%%%%%%%%%%%%%%%%%%%%%%%%%
%
%    Localization in the adiabatic process of shear deformation for Hyp2016
%
%                                                      by
%
%                                       Min-Gi Lee   
%
%                                          version Sep 2016
%
%
%%%%%%%%%%%%%%%%%%%%%%%%%%%%%%%%%%%%%%%%%%%%%%%




% RECOMMENDED %%%%%%%%%%%%%%%%%%%%%%%%%%%%%%%%%%%%%%%%%%%%%%%%%%%
\documentclass[graybox]{svmult}

% choose options for [] as required from the list
% in the Reference Guide

\usepackage{mathptmx}       % selects Times Roman as basic font
\usepackage{helvet}         % selects Helvetica as sans-serif font
\usepackage{courier}        % selects Courier as typewriter font
\usepackage{type1cm}        % activate if the above 3 fonts are
                            % not available on your system
%
\usepackage{makeidx}         % allows index generation
\usepackage{graphicx}        % standard LaTeX graphics tool
                             % when including figure files
\usepackage{multicol}        % used for the two-column index
\usepackage[bottom]{footmisc}% places footnotes at page bottom


\usepackage{amsmath}
\usepackage{amssymb}
\usepackage{color}
\def\red{\color{red}}
\def\blue{\color{blue}}
%\usepackage{verbatim}
% \usepackage{alltt}
%\usepackage{kotex}

\usepackage{enumerate}

%%%%%%%%%%%%%% MY DEFINITIONS %%%%%%%%%%%%%%%%%%%%%%%%%%%

\def\tr{\,\textrm{tr}\,}
\def\div{\,\textrm{div}\,}
\def\sgn{\,\textrm{sgn}\,}

\def\th{\tilde{h}}
\def\tx{\tilde{x}}
\def\tk{\tilde{\kappa}}


\def\bg{{\bar{\gamma}}}
\def\bv{{\bar{v}}}
\def\bth{{\bar{\theta}}}
\def\bs{{\bar{\sigma}}}
\def\bu{{\bar{u}}}
\def\bph{{\bar{\varphi}}}


\def\tg{{\tilde{\gamma}}}
\def\tv{{\tilde{v}}}
\def\tth{{\tilde{\theta}}}
\def\ts{{\tilde{\sigma}}}
\def\tu{{\tilde{u}}}
\def\tph{{\tilde{\varphi}}}

\def\dtg{{\dot{\tilde{\gamma}}}}
\def\dtv{{\dot{\tilde{v}}}}
\def\dtth{{\dot{\tilde{\theta}}}}
\def\dts{{\dot{\tilde{\sigma}}}}
\def\dtu{{\dot{\tilde{u}}}}
\def\dtph{{\dot{\tilde{\varphi}}}}

\def\dpp{\dot{p}}
\def\dqq{\dot{q}}
\def\drr{\dot{r}}
\def\dss{\dot{s}}

\def\ta{{\tilde{a}}}
\def\tb{{\tilde{b}}}
\def\tc{{\tilde{c}}}
\def\td{{\tilde{d}}}

\def\BO{{\mathcal{O}}}
\def\lio{{\mathcal{o}}}



\def\bx{\bar{x}}
\def\bm{\bar{\mathbf{m}}}
\def\K{\mathcal{K}}
\def\E{\mathcal{E}}
\def\del{\partial}
\def\eps{\varepsilon}

% see the list of further useful packages
% in the Reference Guide

\makeindex             % used for the subject index
                       % please use the style svind.ist with
                       % your makeindex program

%%%%%%%%%%%%%%%%%%%%%%%%%%%%%%%%%%%%%%%%%%%%%%%%%%%%%%%%%%%%%%%%%%%%%%%%%%%%%%%%%%%%%%%%%

\begin{document}

\title*{On Localization of Thermo-Elastic Materials In Adiabatic Process  }
% Use \titlerunning{Short Title} for an abbreviated version of
% your contribution title if the original one is too long
\author{Min-Gi Lee, Katsaonious Theodoros and Athanasios Tzavaras}
% Use \authorrunning{Short Title} for an abbreviated version of
% your contribution title if the original one is too long
\institute{Min-Gi Lee \at KAUST, Computer, Electrical and Mathematical Sciences \& Engineering Division, King Abdullah University of Science and Technology (KAUST), Thuwal, Saudi Arabia, \email{mingi.lee@kaust.edu.sa}
\and Katsaonious Theodoros \at KAUST, Computer, Electrical and Mathematical Sciences \& Engineering Division, King Abdullah University of Science and Technology (KAUST), Thuwal, Saudi Arabia \email{theodoros.katsaounis@kaust.edu.sa}
\and Athanasios Tzavaras \at KAUST, Computer, Electrical and Mathematical Sciences \& Engineering Division, King Abdullah University of Science and Technology (KAUST), Thuwal, Saudi Arabia \email{athanasios.tzavaras@kaust.edu.sa}}
%
% Use the package "url.sty" to avoid
% problems with special characters
% used in your e-mail or web address
%
\maketitle

\abstract*{Each chapter should be preceded by an abstract (10--15 lines long) that summarizes the content. The abstract will appear \textit{online} at \url{www.SpringerLink.com} and be available with unrestricted access. This allows unregistered users to read the abstract as a teaser for the complete chapter. As a general rule the abstracts will not appear in the printed version of your book unless it is the style of your particular book or that of the series to which your book belongs.
Please use the 'starred' version of the new Springer \texttt{abstract} command for typesetting the text of the online abstracts (cf. source file of this chapter template \texttt{abstract}) and include them with the source files of your manuscript. Use the plain \texttt{abstract} command if the abstract is also to appear in the printed version of the book.}

\abstract{Each chapter should be preceded by an abstract (10--15 lines long) that summarizes the content. The abstract will appear \textit{online} at \url{www.SpringerLink.com} and be available with unrestricted access. This allows unregistered users to read the abstract as a teaser for the complete chapter. As a general rule the abstracts will not appear in the printed version of your book unless it is the style of your particular book or that of the series to which your book belongs.\newline\indent
Please use the 'starred' version of the new Springer \texttt{abstract} command for typesetting the text of the online abstracts (cf. source file of this chapter template \texttt{abstract}) and include them with the source files of your manuscript. Use the plain \texttt{abstract} command if the abstract is also to appear in the printed version of the book.}

\section{Introduction}
We study instability in adiabatic one dimensional shear flow. Consider a slab of a material situated on the $(x,y)$-plane between the interval $0<x<1$ that is under shear deformation in the $y$-direction. As the flow is adiabatic, the energy, which is ceaselessly supplied by the movement of the boundary, accumulates at each point and once the imbalance of the temperature distribution occurs then it persists and does not diffuse out. Typically, material exhibits {\it thermal softening} and {\it strain hardening}. When the temperature distribution and the strain distribution are non-uniform in $x$, the stress response is the competition between the softening and the hardening and thus according to the precise constitutive law, the nature of the system has many possibilities.

If the decay rate of the stress as temperature increases is faster enough than the growth rate as strain increases, then the net effect makes the point of material softer than before and this can  trigger the positive-feedback mechanism within the increment of the temperature and the increment of the strain. This is manifested by loss of hyperbolicity in the corresponding thermo-elastic model.

In the regime that the system loses hyperbolicity, it is acutely pointed out in \cite{KLT_2016} that the original thermo-elastic model does not fully account for the shear localization phenomena: The instability system exhibits has coined the term {\it Hadamard Instability} that typically occurs in the initial problem of an elliptic equations and that we have Hadamard Instability indicates the exponential growth of oscillation, i.e., the system is catastrophically ill-posed.

On the other hand, Zenor and Solomon \cite{ZH_1944} conducted the experiments of shear deformation of metals in high rate of deformation, which let the process close to the adiabatic. What is observed in the experiments is the coherent localization of the shear strain, the {\it shear band}. There, oscillations are suppressed and the process is ``orderly'' albeit the development of the singularity.

\cite{tzavaras_nonlinear_1992,Tz_1986,Tz_1987} the efforts has been focused on the role of viscosity in the model. In the linearized analysis in \cite{KLT_2016}, the presence of small viscosity alters the nature of the instability from Hadamard Instability to the Turing Instability. The positive eigenvalues of the linearized system are unbounded without the viscosity whereas they are bounded with the viscosity.

In this paper, we study the instability by constructing a family of localizing self-similar solutions for the variety of thermo-visco-elastic models for the one dimensional shear flow. We explain the hierarchy of variety of models as in \cite{tzavaras_nonlinear_1992}.

The family of solutions reveals the structure of the coherent shear bands. With in the family, the rate of focusing and growing is of polynomial order and it is bounded by the value associated to the viscosity. The family of solutions are parametrized by the initial heights of the small bumps in temperature, strain, and strain rate distributions, $\Theta(0)$, $\Gamma(0)$, and $U(0)$ respectively. It turns out that the solution does not exists if the ratio $\displaystyle \frac{U(0)^{1+n}}{\Theta(0)^{1+\alpha}}$, where $n$ and $\alpha$ accounts for the  $\displaystyle \frac{U(0)}{\Gamma(0)}$ (resp. $\displaystyle \frac{U(0)^{1+n}}{\Theta(0)^{1+\alpha}}$) is outside of a certain interval, which indicates the coherent localization is a consequence of the sophisticated competition between the viscosity and the instability due to inelastic response of the material.
\section{Problem Description}
We study the one dimensional adiabatic shear flow in $(t,x) \in \mathbb{R}^+\times \mathbb{R}$. The direction of shear is $y$-direction and the motion is described by following field variables,
\begin{equation} \label{eq:vars}
\begin{aligned}
 \gamma(t,x) &: \text{shear strain}\\
 u(t,x)=\gamma_t &: \text{shear strain rate}\\
 v(t,x) &: \text{vertical velocity}\\
 \theta(t,x) &: \text{temperature}\\
 \tau(t,x) &: \text{shear stress}
\end{aligned}
\end{equation}
The shear stress
$$ \tau = \tau(\theta,\gamma,u) $$
the thermo-visco-elastic constitutive law. In particular we consider the nonlinear viscous contribution in the following form
\begin{equation}
 \tau = \varphi(\theta,\gamma) u^n, \label{eq:stresslaw}
\end{equation}
where $n$ is the strain-rate sensitivity and is assumed to be very small, $0<n\ll1$.

A system of equations describing the deformation are given by
\begin{equation} \label{eq:A}\tag{A}
\begin{aligned}
 \gamma_t &= u\triangleq v_x, \quad \text{(kinematic compatibility)} 	\\
 v_t &= \tau_x, \quad \text{(momentum conservation)} 	\\
 \theta_t &= \tau u \quad \text{(adiabatic energy equation)}	\\
 \tau &=\varphi(\theta,\gamma) u^n.			
\end{aligned}
\end{equation}
We look for a self-similar solution of the system. 

The material keeps loading in $y$-direction so we expect the temperature and the strain at each point increase to infinity as $t \rightarrow \infty$,
$$ \theta(t,x), \; \gamma(t,x) \rightarrow \infty, \quad \text{as $t \rightarrow \infty$, for each $x$}.$$




From \eqref{eq:A}, the cases where $\varphi$ is independent on either of $\theta$ and $\gamma$ forms two subclasses of models. In fact, two subclasses have their own interpretations respectively, but they share the exactly same set of equations. See also the model hierarchy explained in \cite{tzavaras_nonlinear_1992}. 
\subsubsection{strain independent model}
We first consider the subclass where
$$ \varphi(\theta,\gamma) = \varphi(\theta), \quad \tau = \varphi(\theta)u^n.$$
In this subclass, the model describes a rect-linear shear flow of a viscous fluid. $\tau = \varphi(\theta)u^n$ is the non-newtonian viscous stress and the $\varphi(\theta)$ is the temperature dependent viscosity.

The system is closed without the the first equation of \eqref{eq:A}, having
\begin{equation} \label{eq:B1}\tag{B1}
 \begin{aligned}
  v_t &= \tau_x,\\
  \theta_t &= \tau u,\\
  \tau &=\varphi(\theta)u^n.
 \end{aligned}
\end{equation}



\subsubsection{temperature independent model}
Next, we consider the subclass where
$$ \varphi(\theta,\gamma) = \varphi(\gamma), \quad \varphi'(\gamma)<0, \quad \tau = \varphi(\gamma)u^n.$$
In this subclass, the model describes the isothermal plastic viscous shear deformation.

The system is closed without the temperature equation of \eqref{eq:A}, having
\begin{equation} \label{eq:B2}\tag{B2}
 \begin{aligned}
  \gamma_t &= u,\\
  v_t &= \tau_x,\\
  \tau &=\varphi(\gamma)u^n.
 \end{aligned}
\end{equation}

\section{Main results}

\section{Idea of proof} \label{sec:idea}

\section{Self-similar structure}

\subsection{Scale invariance property of the system}
The system \eqref{eq:g}-\eqref{eq:tau} admits a scale invariance property. Suppose $(\gamma,u,v,\theta,\tau)$ is a solution. Then a rescaled version of it $(\gamma_\rho,u_\rho,v_\rho,\theta_\rho,\tau_\rho)$ that is given by
\begin{align*}
 \gamma_\rho(t,x) &= \rho^a\gamma(\rho^{-1}t,\rho^\lambda x), &
 v_\rho(t,x) &= \rho^bv(\rho^{-1}t,\rho^\lambda x),\\
 \theta_\rho(t,x) &= \rho^c\theta(\rho^{-1}t,\rho^\lambda x), &
 \tau_\rho(t,x) &= \rho^d\tau(\rho^{-1}t,\rho^\lambda x),\\
 u_\rho(t,x) &= \rho^{b+\lambda}\gamma(\rho^{-1}t,\rho^\lambda x),
\end{align*}
again is a solution provided
\begin{align*}
 a&= \frac{2+2\alpha-n}{D} + \frac{2+2\alpha}{D}\lambda =: a_0 + a_1 \lambda, & b&=\frac{1+m}{D} + \frac{1+m+n}{D}\lambda =: b_0 + b_1\lambda,\\
 c&=\frac{2(1+m)}{D} + \frac{2(1+m+n)}{D}\lambda =: c_0 + c_1\lambda, & d&=\frac{-2\alpha + 2m +n}{D} + \frac{-2\alpha+2m+2n}{D}\lambda =: d_0 + d_1\lambda,
\end{align*}
for each $\lambda \in \mathbb{R}$, where the denominator $D = 1+2\alpha-m-n$. Being $\lambda$ negative makes this scaling spread the original profiles spatially as $\rho$ grows, and this is the sign that is used for instance in the parabolic problems. On the other hand, we are interested in the {\it localizing} phenomena and thus we focus only on the opposite, with $\lambda>0$, throughout this paper.

 But uniform shearing solution takes a negative $\lambda$.
\subsection{Self-Similar field variables}
Motivated by the scale invariance property parametrized by $\lambda>0$, we look for the solutions of type
\begin{align*}
 \gamma(t,x) &= t^a\Gamma(t^\lambda x), & v(t,x) &= t^b V(t^\lambda x), &\theta(t,x) &= t^c \Theta(t^\lambda x),\\
 \tau(t,x) &= t^d \Sigma(t^\lambda x), & u(t,x) &= t^{b+\lambda} U(t^\lambda x)
\end{align*}
and set $\xi = t^\lambda x$. Plugging in the ansatz to the system \eqref{eq:g}-\eqref{eq:tau} gives us the system of ordinary differential and algebraic equations that $\big(\Gamma(\xi), V(\xi), \Theta(\xi), \Sigma(\xi), U(\xi)\big)$ satisfies.

\begin{equation}
\begin{aligned}
 a \Gamma(\xi) + \lambda \xi \Gamma'(\xi) &= U(\xi),\\
 b V(\xi) + \lambda \xi V'(\xi) &= \Sigma'(\xi),\\
 c \Theta(\xi) + \lambda \xi \Theta'(\xi)&=\Sigma(\xi) U(\xi),\\
 \Sigma(\xi) &= \Theta(\xi)^{-\alpha} \Gamma(\xi)^m U(\xi)^n,\\
 V'(\xi)&=U(\xi).
\end{aligned} \label{eq:ss-odes}
\end{equation}


\section{Section Heading}
\label{sec:1}
Use the template \emph{chapter.tex} together with the Springer document class SVMono (monograph-type books) or SVMult (edited books) to style the various elements of your chapter content in the Springer layout.

Instead of simply listing headings of different levels we recommend to
let every heading be followed by at least a short passage of text.
Further on please use the \LaTeX\ automatism for all your
cross-references and citations. And please note that the first line of
text that follows a heading is not indented, whereas the first lines of
all subsequent paragraphs are.

\section{Section Heading}
\label{sec:2}
% Always give a unique label
% and use \ref{<label>} for cross-references
% and \cite{<label>} for bibliographic references
% use \sectionmark{}
% to alter or adjust the section heading in the running head
Instead of simply listing headings of different levels we recommend to
let every heading be followed by at least a short passage of text.
Further on please use the \LaTeX\ automatism for all your
cross-references and citations.

Please note that the first line of text that follows a heading is not indented, whereas the first lines of all subsequent paragraphs are.

Use the standard \verb|equation| environment to typeset your equations, e.g.
%
\begin{equation}
a \times b = c\;,
\end{equation}
%
however, for multiline equations we recommend to use the \verb|eqnarray| environment\footnote{In physics texts please activate the class option \texttt{vecphys} to depict your vectors in \textbf{\itshape boldface-italic} type - as is customary for a wide range of physical subjects}.
\begin{eqnarray}
a \times b = c \nonumber\\
\vec{a} \cdot \vec{b}=\vec{c}
\label{eq:01}
\end{eqnarray}

\subsection{Subsection Heading}
\label{subsec:2}
Instead of simply listing headings of different levels we recommend to
let every heading be followed by at least a short passage of text.
Further on please use the \LaTeX\ automatism for all your
cross-references\index{cross-references} and citations\index{citations}
as has already been described in Sect.~\ref{sec:2}.

\begin{quotation}
Please do not use quotation marks when quoting texts! Simply use the \verb|quotation| environment -- it will automatically render Springer's preferred layout.
\end{quotation}


\subsubsection{Subsubsection Heading}
Instead of simply listing headings of different levels we recommend to
let every heading be followed by at least a short passage of text.
Further on please use the \LaTeX\ automatism for all your
cross-references and citations as has already been described in
Sect.~\ref{subsec:2}, see also Fig.~\ref{fig:1}\footnote{If you copy
text passages, figures, or tables from other works, you must obtain
\textit{permission} from the copyright holder (usually the original
publisher). Please enclose the signed permission with the manuscript. The
sources\index{permission to print} must be acknowledged either in the
captions, as footnotes or in a separate section of the book.}

Please note that the first line of text that follows a heading is not indented, whereas the first lines of all subsequent paragraphs are.

% For figures use
%
\begin{figure}[b]
\sidecaption
% Use the relevant command for your figure-insertion program
% to insert the figure file.
% For example, with the graphicx style use
\includegraphics[scale=.65]{figure}
%
% If no graphics program available, insert a blank space i.e. use
%\picplace{5cm}{2cm} % Give the correct figure height and width in cm
%
\caption{If the width of the figure is less than 7.8 cm use the \texttt{sidecapion} command to flush the caption on the left side of the page. If the figure is positioned at the top of the page, align the sidecaption with the top of the figure -- to achieve this you simply need to use the optional argument \texttt{[t]} with the \texttt{sidecaption} command}
\label{fig:1}       % Give a unique label
\end{figure}


\paragraph{Paragraph Heading} %
Instead of simply listing headings of different levels we recommend to
let every heading be followed by at least a short passage of text.
Further on please use the \LaTeX\ automatism for all your
cross-references and citations as has already been described in
Sect.~\ref{sec:2}.

Please note that the first line of text that follows a heading is not indented, whereas the first lines of all subsequent paragraphs are.

For typesetting numbered lists we recommend to use the \verb|enumerate| environment -- it will automatically render Springer's preferred layout.

\begin{enumerate}
\item{Livelihood and survival mobility are oftentimes coutcomes of uneven socioeconomic development.}
\begin{enumerate}
\item{Livelihood and survival mobility are oftentimes coutcomes of uneven socioeconomic development.}
\item{Livelihood and survival mobility are oftentimes coutcomes of uneven socioeconomic development.}
\end{enumerate}
\item{Livelihood and survival mobility are oftentimes coutcomes of uneven socioeconomic development.}
\end{enumerate}


\subparagraph{Subparagraph Heading} In order to avoid simply listing headings of different levels we recommend to let every heading be followed by at least a short passage of text. Use the \LaTeX\ automatism for all your cross-references and citations as has already been described in Sect.~\ref{sec:2}, see also Fig.~\ref{fig:2}.

For unnumbered list we recommend to use the \verb|itemize| environment -- it will automatically render Springer's preferred layout.

\begin{itemize}
\item{Livelihood and survival mobility are oftentimes coutcomes of uneven socioeconomic development, cf. Table~\ref{tab:1}.}
\begin{itemize}
\item{Livelihood and survival mobility are oftentimes coutcomes of uneven socioeconomic development.}
\item{Livelihood and survival mobility are oftentimes coutcomes of uneven socioeconomic development.}
\end{itemize}
\item{Livelihood and survival mobility are oftentimes coutcomes of uneven socioeconomic development.}
\end{itemize}

\begin{figure}[t]
\sidecaption[t]
% Use the relevant command for your figure-insertion program
% to insert the figure file.
% For example, with the option graphics use
\includegraphics[scale=.65]{figure}
%
% If no graphics program available, insert a blank space i.e. use
%\picplace{5cm}{2cm} % Give the correct figure height and width in cm
%
%\caption{Please write your figure caption here}
\caption{If the width of the figure is less than 7.8 cm use the \texttt{sidecapion} command to flush the caption on the left side of the page. If the figure is positioned at the top of the page, align the sidecaption with the top of the figure -- to achieve this you simply need to use the optional argument \texttt{[t]} with the \texttt{sidecaption} command}
\label{fig:2}       % Give a unique label
\end{figure}

\runinhead{Run-in Heading Boldface Version} Use the \LaTeX\ automatism for all your cross-references and citations as has already been described in Sect.~\ref{sec:2}.

\subruninhead{Run-in Heading Italic Version} Use the \LaTeX\ automatism for all your cross-refer\-ences and citations as has already been described in Sect.~\ref{sec:2}\index{paragraph}.
% Use the \index{} command to code your index words
%
% For tables use
%
\begin{table}
\caption{Please write your table caption here}
\label{tab:1}       % Give a unique label
%
% Follow this input for your own table layout
%
\begin{tabular}{p{2cm}p{2.4cm}p{2cm}p{4.9cm}}
\hline\noalign{\smallskip}
Classes & Subclass & Length & Action Mechanism  \\
\noalign{\smallskip}\svhline\noalign{\smallskip}
Translation & mRNA$^a$  & 22 (19--25) & Translation repression, mRNA cleavage\\
Translation & mRNA cleavage & 21 & mRNA cleavage\\
Translation & mRNA  & 21--22 & mRNA cleavage\\
Translation & mRNA  & 24--26 & Histone and DNA Modification\\
\noalign{\smallskip}\hline\noalign{\smallskip}
\end{tabular}
$^a$ Table foot note (with superscript)
\end{table}
%
\section{Section Heading}
\label{sec:3}
% Always give a unique label
% and use \ref{<label>} for cross-references
% and \cite{<label>} for bibliographic references
% use \sectionmark{}
% to alter or adjust the section heading in the running head
Instead of simply listing headings of different levels we recommend to
let every heading be followed by at least a short passage of text.
Further on please use the \LaTeX\ automatism for all your
cross-references and citations as has already been described in
Sect.~\ref{sec:2}.

Please note that the first line of text that follows a heading is not indented, whereas the first lines of all subsequent paragraphs are.

If you want to list definitions or the like we recommend to use the Springer-enhanced \verb|description| environment -- it will automatically render Springer's preferred layout.

\begin{description}[Type 1]
\item[Type 1]{That addresses central themes pertainng to migration, health, and disease. In Sect.~\ref{sec:1}, Wilson discusses the role of human migration in infectious disease distributions and patterns.}
\item[Type 2]{That addresses central themes pertainng to migration, health, and disease. In Sect.~\ref{subsec:2}, Wilson discusses the role of human migration in infectious disease distributions and patterns.}
\end{description}

\subsection{Subsection Heading} %
In order to avoid simply listing headings of different levels we recommend to let every heading be followed by at least a short passage of text. Use the \LaTeX\ automatism for all your cross-references and citations citations as has already been described in Sect.~\ref{sec:2}.

Please note that the first line of text that follows a heading is not indented, whereas the first lines of all subsequent paragraphs are.

\begin{svgraybox}
If you want to emphasize complete paragraphs of texts we recommend to use the newly defined Springer class option \verb|graybox| and the newly defined environment \verb|svgraybox|. This will produce a 15 percent screened box 'behind' your text.

If you want to emphasize complete paragraphs of texts we recommend to use the newly defined Springer class option and environment \verb|svgraybox|. This will produce a 15 percent screened box 'behind' your text.
\end{svgraybox}


\subsubsection{Subsubsection Heading}
Instead of simply listing headings of different levels we recommend to
let every heading be followed by at least a short passage of text.
Further on please use the \LaTeX\ automatism for all your
cross-references and citations as has already been described in
Sect.~\ref{sec:2}.

Please note that the first line of text that follows a heading is not indented, whereas the first lines of all subsequent paragraphs are.

\begin{theorem}
Theorem text goes here.
\end{theorem}
%
% or
%
\begin{definition}
Definition text goes here.
\end{definition}

\begin{proof}
%\smartqed
Proof text goes here.
\qed
\end{proof}

\paragraph{Paragraph Heading} %
Instead of simply listing headings of different levels we recommend to
let every heading be followed by at least a short passage of text.
Further on please use the \LaTeX\ automatism for all your
cross-references and citations as has already been described in
Sect.~\ref{sec:2}.

Note that the first line of text that follows a heading is not indented, whereas the first lines of all subsequent paragraphs are.
%
% For built-in environments use
%
\begin{theorem}
Theorem text goes here.
\end{theorem}
%
\begin{definition}
Definition text goes here.
\end{definition}
%
\begin{proof}
\smartqed
Proof text goes here.
\qed
\end{proof}
%
\begin{acknowledgement}
If you want to include acknowledgments of assistance and the like at the end of an individual chapter please use the \verb|acknowledgement| environment -- it will automatically render Springer's preferred layout.
\end{acknowledgement}
%
\section*{Appendix}
\addcontentsline{toc}{section}{Appendix}
%
%
When placed at the end of a chapter or contribution (as opposed to at the end of the book), the numbering of tables, figures, and equations in the appendix section continues on from that in the main text. Hence please \textit{do not} use the \verb|appendix| command when writing an appendix at the end of your chapter or contribution. If there is only one the appendix is designated ``Appendix'', or ``Appendix 1'', or ``Appendix 2'', etc. if there is more than one.

\begin{equation}
a \times b = c
\end{equation}

\input{referenc}
\begin{thebibliography}{10}

\bibitem{bertsch_effect_1991}
{\sc M.~Bertsch, L.~Peletier, and S.~Verduyn~Lunel}, 
The effect of temperature dependent viscosity on shear flow of  incompressible fluids,
{\em SIAM J. Math. Anal.} {\bf 22 } (1991), 328--343.

\bibitem{dafermos_adiabatic_1983}
{\sc C.~M. Dafermos and L.~Hsiao}, 
Adiabatic shearing of incompressible fluids with temperature-dependent viscosity.
{\em Quart.  Applied Math.} {\bf 41} (1983), 45--58.

\bibitem{fenichel_geometric_1979}
{\sc N.~Fenichel}, 
Geometric singular perturbation theory for ordinary differential equations, 
{\it J. Differ. Equations} {\bf 31} (1979), 53--98.

\bibitem{FM87}
{\sc C.~Fressengeas and A.~Molinari}, 
Instability and localization of plastic flow in shear at high strain rates, 
  {\em J.  Mech. Physics of Solids} {\bf 35} (1987), 185--211.

\bibitem{jones_geometric_1995}
{\sc C.~K. R.~T. Jones}, 
Geometric singular perturbation theory, in {\it Dynamical systems}, LNM {\bf 1609} (Springer Berlin Heidelberg 1995) 44--118.
  
  
\bibitem{KOT14}
{\sc Th.~Katsaounis, J.~Olivier, and A.E.~Tzavaras}, 
Emergence of coherent localized structures in shear deformations of
  temperature dependent fluids, arXiv preprint arXiv:1411.6131,  (2014).

\bibitem{LT16}
{\sc M.-G.~Lee and A.E.~Tzavaras},
Existence of localizing solutions in plasticity via the geometric singular perturbation theory, arXiv preprint arXiv:xxxx.xxxxx,  (2016)

\bibitem{KLT_2016}
{\sc Th. Katsaounis, M.-G. Lee, and A.E. Tzavaras}, 
Localization in inelastic rate dependent shearing deformations, arXiv preprint arXiv:1605.04564,  (2016).

% \bibitem{perko_differential_2001}
% {\sc L.~Perko}, 
% {\it Differential equations and dynamical systems 3rd. ed.}, TAM {\bf 7} (Springer-Verlag New York 2001).  

\bibitem{Tz_1986}
{\sc A.E. Tzavaras},
Shearing of materials exhibiting thermal softening or temperature dependent viscosity,
{\em Quart.  Applied Math.} {\bf 44} (1986), 1--12.

\bibitem{Tz_1987}
{\sc A.E. Tzavaras}, 
Effect of thermal softening in shearing of strain-rate dependent materials.
{\em Archive for Rational Mechanics and Analysis}, {\bf 99} (1987), 349--374.

\bibitem{tzavaras_nonlinear_1992}
%\leavevmode\vrule height 2pt depth -1.6pt width 23pt,
{\sc A.E. Tzavaras}, 
Nonlinear analysis techniques for shear band formation at high strain-rates, 
% {\it Applied Mechanics Reviews} 
{\it Appl. Mech. Rev.}
{\bf  45} (1992), S82--S94.

\bibitem{ZH_1944}
{\sc C.~Zener and J.~H. Hollomon}, 
Effect of strain rate upon plastic flow of steel,
% {\it  Journal of Applied Physics} 
{\it J. Appl. Phys.}
{\bf 15} (1944), 22--32.
\end{thebibliography}

\end{document}	


\end{document}















\documentclass[a4paper,11pt]{article}

\usepackage[margin=2cm]{geometry}
\usepackage{setspace}
%\onehalfspacing
\doublespacing
%\usepackage{authblk}
\usepackage{amsmath}
\usepackage{amssymb}
\usepackage{amsthm}

\usepackage[notcite,notref]{showkeys}

% \usepackage{psfrag}
\usepackage{graphicx,subfigure}
\usepackage{color}
\def\red{\color{red}}
\def\blue{\color{blue}}
%\usepackage{verbatim}
% \usepackage{alltt}
%\usepackage{kotex}

\usepackage{enumerate}

%%%%%%%%%%%%%% MY DEFINITIONS %%%%%%%%%%%%%%%%%%%%%%%%%%%

\def\tr{\,\textrm{tr}\,}
\def\div{\,\textrm{div}\,}
\def\sgn{\,\textrm{sgn}\,}

\def\th{\tilde{h}}
\def\tx{\tilde{x}}
\def\tk{\tilde{\kappa}}

\def\tg{{\tilde{\gamma}}}
\def\tv{{\tilde{v}}}
\def\tth{{\tilde{\theta}}}
\def\ts{{\tilde{\tau}}}
\def\tu{{\tilde{u}}}

\def\dtg{\dot{\tilde{\gamma}}}
\def\dtv{\dot{\tilde{v}}}
\def\dtth{\dot{\tilde{\theta}}}
\def\dts{\dot{\tilde{\tau}}}
\def\dtu{\dot{\tilde{u}}}

\def\dpp{\dot{p}}
\def\dqq{\dot{q}}
\def\drr{\dot{r}}
\def\dss{\dot{s}}

\def\ta{{\tilde{a}}}
\def\tb{{\tilde{b}}}
\def\tc{{\tilde{c}}}
\def\td{{\tilde{d}}}




\def\bx{\bar{x}}
\def\bm{\bar{\mathbf{m}}}
\def\K{\mathcal{K}}
\def\E{\mathcal{E}}
\def\del{\partial}
\def\eps{\varepsilon}

\newcommand{\tcr}{\textcolor{red}}
\newcommand{\tcb}{\textcolor{blue}}

\newcommand{\ubar}[1]{\text{\b{$#1$}}}
\newtheorem{theorem}{Theorem}
\newtheorem{lemma}{Lemma}[section]
\newtheorem{proposition}{Proposition}[section]
%\newtheorem{definition}{Definition}[section]
\newtheorem{remark}{Remark}[section]

%%%%%%%%%%%%%%%%%%%%%%%%%%%%%%%%%%%%%%%%%%%%%%%%%%%%%%%%%%
\begin{document}
\title{Note for the Self-similar shear bands problem}
\author{Min-Gi Lee\footnotemark[1]}
% \and Athanasios Tzavaras\footnotemark[1]\  \footnotemark[3]  \footnotemark[4]}
\date{}

\maketitle
\renewcommand{\thefootnote}{\fnsymbol{footnote}}
% \footnotetext[1]{Computer, Electrical and Mathematical Sciences \& Engineering Division, King Abdullah University of Science and Technology (KAUST), Thuwal, Saudi Arabia}
% \footnotetext[2]{Department of Mathematics and Applied Mathematics, University of Crete, Heraklion, Greece}
% \footnotetext[3]{Institute of Applied and Computational Mathematics, FORTH, Heraklion, Greece}
% \footnotetext[4]{Corresponding author : \texttt{athanasios.tzavaras@kaust.edu.sa}}
%\footnotetext[4]{Research supported by the King Abdullah University of Science and Technology (KAUST) }
\renewcommand{\thefootnote}{\arabic{footnote}}


\maketitle

\tableofcontents
% \begin{abstract}
% abstract
% \end{abstract}

\section{The model description}
We consider a 1-d shear deformation of a material whose material law of stress depends on 1) temperature, 2) strain, 3) strain rate. The motion is described by following field variables,
\begin{equation} \label{eq:vars}
\begin{aligned}
 \gamma(t,x) &: \text{strain}\\
 u(t,x)=\gamma_t &: \text{strain rate}\\
 v(t,x) &: \text{vertical velocity}\\
 \theta(t,x) &: \text{temperature}\\
 \tau(t,x) &: \text{stress}
\end{aligned}
\end{equation}
The material exhibits 1) temperature-softening, 2) strain-hardening, 3) rate-hardening. we denote the shear stress
$$ \tau = \tau(\theta,\gamma,u). $$
and study a model
\begin{equation}
 \tau = \theta^{-\alpha}\gamma^m u^n. \label{eq:stresslaw}
\end{equation}

A few forehand perspectives :
\begin{enumerate}
 \item The regime where $-\alpha+m+n <0$ will exhibit the localization, whereas the regime $-\alpha+m+n > 0$ will exhibit stabilization. {\blue Can we rigorously study the linearize stability to the uniform shearing solution?}
 \item The uniform shearing solution will appear as one of the self-similar solution by a specific $\lambda$ that is negative.
\end{enumerate}
\subsection{A system of conservation laws}
For the field variables \eqref{eq:vars}, equations describing the deformation are given by
\begin{align}
 \gamma_t &= u, \quad \text{(kinematic compatibility)} 	\label{eq:g}\\
 v_t &= \tau_x, \quad \text{(momentum conservation)} 	\label{eq:v}\\
 \theta_t &= \tau u \quad \text{(energy conservation)}	\label{eq:th}\\
 \tau &=\theta^{-\alpha}\gamma^m u^n.			\label{eq:tau}
\end{align}

\subsubsection{temperature softening model}
We first focus on the problem where 
$$ \tau = \tau(\theta,u) = \theta^{-\alpha}u^n.$$
Then the first equation \eqref{eq:g} drops out from the system, and we focus on the system
\begin{equation} \label{eq:orisys}
 \begin{aligned}
  v_t &= \tau_x,\\
  \theta_t &= \tau u,\\
  \tau &=\theta^{-\alpha}u^n.
 \end{aligned}
\end{equation}


\subsection{Scale invariance property of the system}
The system \eqref{eq:orisys} admits a scale invariance property. Suppose $(v,u,\theta,\tau)$ is a solution. Then a rescaled version of it 
\begin{align*}
 v_\rho(t,x) &= \rho^bv(\rho^{-1}t,\rho^\lambda x),\\
 u_\rho(t,x) &= \rho^{b+\lambda}u(\rho^{-1}t,\rho^\lambda x),\\
 \theta_\rho(t,x) &= \rho^c\theta(\rho^{-1}t,\rho^\lambda x),\\
 \tau_\rho(t,x) &= \rho^d\tau(\rho^{-1}t,\rho^\lambda x)
\end{align*}

Calculations :
\begin{align*}
 &\text{let} \;f(t,x) = \rho^k F(\rho^{-1}t,\rho^\lambda x), \\
 &\partial_t f(t,x) = \rho^k \partial_{t'} F(\rho^{-1}t,\rho^\lambda x) \rho^{-1} = \rho^{k-1} \partial_{t'}F, \\
 &\partial_x f(t,x) = \rho^k \partial_{x'} F(\rho^{-1}t,\rho^\lambda x) \rho^\lambda = \rho^{k+\lambda} \partial_{x'}F.
\end{align*}
Relations for invariance :
\begin{align*}
 b-1 = d+\lambda, \quad c-1 = d+b+\lambda, \quad d = -\alpha c + m a + n (b+\lambda)
\end{align*}
From above, we reach to exponents
\begin{align*}
 D & = 1+2\alpha-m-n,\\
 a&= \frac{2+2\alpha-n}{D} + \frac{2+2\alpha}{D}\lambda =: a_0 + a_1 \lambda, & b&=\frac{1+m}{D} + \frac{1+m+n}{D}\lambda =: b_0 + b_1\lambda,\\
 c&=\frac{2(1+m)}{D} + \frac{2(1+m+n)}{D}\lambda =: c_0 + c_1\lambda, & d&=\frac{-2\alpha + 2m +n}{D} + \frac{-2\alpha+2m+2n}{D}\lambda =: d_0 + d_1\lambda
\end{align*}
for each $\lambda \in \mathbb{R}$. For localization, $\lambda>0$. But uniform shearing solution takes a negative $\lambda$.
\subsection{Self-Similar variables}
We try the solutions of type, i.e. put $\rho =t$ in the rescaling,
\begin{align*}
 \gamma(t,x) &= t^a\Gamma(t^\lambda x),\\
 v(t,x) &= t^b V(t^\lambda x),\\
 \theta(t,x) &= t^c \Theta(t^\lambda x),\\
 \tau(t,x) &= t^d \Sigma(t^\lambda x),\\
 u(t,x) &= t^{b+\lambda} U(t^\lambda x)
\end{align*}
and set $\xi = t^\lambda x$.

Calculations:
\begin{align*}
 &\text{Suppose } \; f(t,x) = t^k F(t^\lambda x),\\
 &\partial_t f = k t^{k-1} F + t^k F' \lambda t^{\lambda-1} x = t^{k-1} (kF + \lambda\xi F'),\\
 &\partial_x f = t^k F' t^\lambda = t^{k+\lambda} F',
\end{align*}

At \eqref{eq:g}-\eqref{eq:tau}:

\begin{align*}
 t^{a-1}(a \Gamma(\xi) + \lambda \xi \Gamma'(\xi)) &= t^{b+ \lambda} U(\xi),\\
 t^{b-1}(b V(\xi) + \lambda \xi V'(\xi)) &= t^{d+ \lambda} \Sigma'(\xi)\\
 t^{c-1}(c \Theta(\xi) + \lambda \xi \Theta'(\xi))&=t^{b+d+\lambda} \Sigma U(\xi),\\
 t^d\Sigma(\xi) &= t^{-\alpha c +ma +n(b+ \lambda)} \Theta(\xi)^{-\alpha} \Gamma(\xi)^m U(\xi)^n,\\
 t^{b+\lambda}V'(\xi)&=t^{b+\lambda}U(\xi)
\end{align*}
\begin{equation}
\begin{aligned}
 a \Gamma(\xi) + \lambda \xi \Gamma'(\xi) &= U(\xi),\\
 b V(\xi) + \lambda \xi V'(\xi) &= \Sigma'(\xi)\\
 d \Theta(\xi) + \lambda \xi \Theta'(\xi)&=\Sigma(\xi) U(\xi),\\
 \Sigma(\xi) &= \Theta(\xi)^{-\alpha} \Gamma(\xi)^m U(\xi)^n,\\
 V'(\xi)&=U(\xi)
\end{aligned} \label{eq:ss-odes}
\end{equation}
\subsection{de-singularization}
Introduce new field variables 
\begin{equation}
\begin{aligned}
 \Gamma(\xi) &= \xi^\ta \tg(\xi),\\
 V(\xi)&=\xi^\tb \tv(\xi),\\
 \Theta(\xi)&=\xi^\tc \tth(\xi),\\
 \Sigma(\xi)&=\xi^\td \ts(\xi),\\
 U(\xi)&=\xi^{\tb-1} \tu(\xi). 
\end{aligned}
\end{equation}
Then at \eqref{eq:ss-odes}:
\begin{align*}
 \xi^\ta\Big( a\tg + \lambda \ta \tg + \lambda\xi\tg'\Big) &=\xi^{\tb-1} \tu,\\
 \xi^\tb\Big( b\tv + \lambda \tb \tv + \lambda\xi\tv'\Big) &=\xi^{\td-1} \Big(\td\ts + \xi\ts'\Big),\\
 \xi^\tc\Big( c\tth+ \lambda \tc \tth+ \lambda\xi\tth'\Big)&=\xi^{\td+\tb-1} \ts\tu,\\
 \xi^\td\ts &= \xi^{-\alpha \tc +m\ta +n(\tb-1)} \tth^{-\alpha} \tg^m \tu^n,\\
 \xi^{\tb-1}\Big(\tb\tv + \xi \tv'\Big)&= \xi^{\tb-1} \tu.
\end{align*}

$\ta, \tb, \tc, \td$ such that
\begin{align*}
 &\ta=\tb-1, \quad \tb=\td-1, \quad \tc=\td+\tb-1,\quad \td = -\alpha \tc + m\ta +n(\tb-1) \\
 \Longrightarrow \quad&\ta = -a_1, \quad \td = -d_1, \quad \tc = -c_1, \quad \tb=-b_1.
\end{align*}

\begin{equation}
 \begin{aligned}
  a_0\tg + \lambda\xi\tg' &=\tu,\\
  b_0\tv + \lambda\xi\tv' &=-d_1 \ts + \xi\ts',\\
  c_0\tth+ \lambda\xi\tth'&=\ts\tu,\\
  \ts &=\tth^{-\alpha}\tg^m\tu^n,\\
  -b_1\tv+\xi\tv' &= \tu.
 \end{aligned}
\end{equation}

Introduce the new independent variable $\eta = \log\xi$.
\section{$(p,q,r,s)$-system derivation}

\begin{enumerate}
 \item The equation \eqref{eq:th} can be rewritten in the form
 $$ \Big(\frac{1}{1+\alpha} \tth^{1+\alpha}\Big)_t = \frac{\tu^n}{\tg^n} \Big(\frac{1}{1+m+n} \tg^{1+m+n}\Big)_t$$
 and we expect, at least for the self-similar solutions, that
 $$ \frac{ \frac{1}{1+\alpha} \tth^{1+\alpha} }{ \frac{1}{1+m+n} \tg^{1+m+n} }  = 1 + \mathcal{O}(n) \overset{put}{=} s^n, \quad \text{for some $s$}. $$
 \item If so, we can define the variable 
 $$r = \Big(\Big(\frac{1+m+n}{1+\alpha}\Big)^{\frac{\alpha}{(1+\alpha)}}\tau \gamma^{\frac{\alpha-m-n}{1+\alpha}}\Big)^{\frac{1}{n}} = \frac{\tu}{\tg}\,s^{-\frac{\alpha}{1+\alpha}} \sim \mathcal{O}(1) $$
 and when $\alpha=0$, it reduces to $\frac{\tu}{\tg}$. The advantage of this definition to the $\frac{\tu}{\tg}$ is that the latter expression couples to the $\tth$ too whereas $r$ here does not. 
\end{enumerate}

Auxiliary calculations:
\begin{align*}
 \frac{\dtg}{\tg} &= \frac{1}{\lambda }\Big(\frac{\tu}{\tg}-a_0\Big),\\
 \frac{\dts}{\ts} &= d_1+ b_0\frac{\tv}{\ts} + \lambda \frac{\dtv}{\ts} = d_1+ b_0\frac{\tv}{\ts} + \lambda \Big(b_1 + \frac{\tu}{\tv}\Big)\frac{\tv}{\ts} = d_1 + b\frac{\tv}{\ts} + \lambda\frac{\tu}{\tv}\frac{\tv}{\ts} ,\\
 \frac{\dtth}{\tth}&=\frac{1}{\lambda }\Big(\frac{\ts\tu}{\tth}-c_0\Big),\\
 0&=\frac{\dts}{\ts} +\alpha \frac{\dtth}{\tth} - m \frac{\dtg}{\tg} - n\frac{\dtu}{\tu},\\
 \frac{\dtv}{\tv}&= b_1 +\frac{\tu}{\tv}
\end{align*}

Define 
\begin{equation}\label{eq:pqrsdef}
 \begin{aligned}
  p &= \frac{\tg}{\ts}, & q&=b \frac{\tv}{\ts},\\
  r &= \Big(\Big(\frac{1+m+n}{1+\alpha}\Big)^{\frac{\alpha}{(1+\alpha)}}\tau \gamma^{\frac{\alpha-m-n}{1+\alpha}}\Big)^{\frac{1}{n}} , & s&=z^{\frac{1}{n}}, \quad z=\frac{ \frac{1}{1+\alpha} \tth^{1+\alpha} }{ \frac{1}{1+m+n} \tg^{1+m+n} }.
 \end{aligned}
\end{equation}

We have
\begin{align*}
 \frac{\dpp}{p}&=\frac{\dtg}{\tg} - \frac{\dts}{\ts}& &=\left[\frac{1}{\lambda }\Big(\frac{\tu}{\tg}-a_0\Big)\right] & &-\left[d_1 + b\frac{\tv}{\ts} + \lambda\frac{\tu}{\tv}\frac{\tv}{\ts}\right]\\
 \frac{\dqq}{q}&=\frac{\dtv}{\tv} - \frac{\dts}{\ts}& &=\left[b_1 +\frac{\tu}{\tv}\right] & &-\left[d_1 + b\frac{\tv}{\ts} + \lambda\frac{\tu}{\tv}\frac{\tv}{\ts}\right]\\
 n\frac{\drr}{r}&=n\frac{\dtu}{\tu} -n\frac{\dtg}{\tg}& &=\left[\frac{\alpha-m-n}{\lambda(1+\alpha) }\Big(\frac{\tu}{\tg}-a_0\Big)\right]& &+\left[d_1 + b\frac{\tv}{\ts} + \lambda\frac{\tu}{\tv}\frac{\tv}{\ts}\right]\\
 & & & & &+ \Big[\frac{\alpha}{1+\alpha}\, \Big(\frac{1+m+n}{\lambda} \Big(\frac{r^n}{z}-1\Big)r + \frac{n}{\lambda}\Big)\Big]\\
 \frac{\dot{z}}{z} &= (1+\alpha)\frac{\dtth}{\tth} - (1+m+n)\frac{\dtg}{\tg} & &=\left[\frac{-1-m-n}{\lambda }\Big(\frac{\tu}{\tg}-a_0\Big)\right] & &+ \left[\frac{1+\alpha}{\lambda }\Big(\frac{\ts\tu}{\tth}-c_0\Big)\right].%\\
%  \dot{s} &=\frac{\partial s}{\partial (z-1)} \dot{z} &&= \frac{1+m}{\lambda}\frac{\partial s}{\partial (z-1)} \bigg\{z\big[-r - \frac{n}{D}\Big]+ ru^n\bigg\}.
\end{align*}
Noticing that $\displaystyle a_0(1+m+n)-c_0(1+\alpha)=n$ and that
\begin{align*}
 \frac{\ts\tu}{\tth} = \frac{\tg^{1+m+n}}{\tth^{1+\alpha}}\Big(\frac{\tu}{\tg}\Big)^{1+n} = \frac{1+m+n}{1+\alpha} \frac{1}{z}\,\Big(\frac{\tu}{\tg}\Big)^{1+n} =  \frac{1+m+n}{1+\alpha} \frac{r^{1+n}}{z},
\end{align*}
\begin{align*}
\frac{\dot{z}}{z} = \frac{1+m+n}{\lambda}\Big(\frac{r^{1+n}}{z} - r \Big) + \frac{n}{\lambda} = \frac{1+m+n}{\lambda}\,r\Big(\frac{r^{n}}{z} - 1 \Big) + \frac{n}{\lambda}
\end{align*}

\begin{enumerate}
 \item  Equation on $z$ in the form of
 $$ \dot{z-1} = -\frac{1+m+n}{\lambda}\, r (z-1) + \frac{1+m+n}{\lambda}\, r(r^n-1) + n\frac{z}{\lambda} = -\frac{1+m+n}{\lambda}\, r (z-1) + \mathcal{O}(n)$$
 dictates that the $z$ will relax to $z=1 + \mathcal{O}(n)$. When $n=0$, it exactly relaxes to $z\equiv1$, provided $r>0$.
 \item Furthermore, once $z = 1+\mathcal{O}(n)$ at some point, then it will remain to be so (using gronwall's inequality), at least for the finite time, suggesting the invariant manifold that comprises these orbits around $z\equiv1$.
 \item Let us suppose the existence of the invariant manifold $\bar{z}(p,q)$ that is of $1+\mathcal{O}(n)$. Since we are only interested in the orbits that are on the invariant manifold or that are close enough to the manifold, we can safely assume that
 $$ |z-1| < \sqrt{n}, \quad 0<n\ll1.$$
 We observe that the relaxation time of the variable $z$ is of $\mathcal{O}(1)$, and when it is on the manifold, $\dot{z} \sim O(n)$. Compare this to the fast variable $r$: $r$ relaxes with speed $\mathcal{O}(1/n)$ and when it is on the manifold, $\dot{r} \sim O(1)$.
 
 We may technically define a variable that has the fast relaxation time.  When $|z-1| < \sqrt{n}$, $s$ defined by the relation 
 $$ s = \sgn{(z-1)}\Big(\frac{|z-1|}{\sqrt{n}}\Big)^{\frac{1}{n}}, \quad \Longleftrightarrow z = 1+\sqrt{n}s^n, \quad \text{for $|z-1|<\sqrt{n}.$}$$
 is uniformly bounded and is invertible for fixed $n$. If $z$ relaxes to the manifold $\bar{z}=1+\mathcal{O}(n)$ then $s$ relaxes to the manifold $\bar{s}\sim \sqrt{n}^{\frac{1}{n}}$ for $n>0$, and $\displaystyle \lim_{n \rightarrow 0+} \sqrt{n}^{\frac{1}{n}}=0$, which is consistent with $z\equiv1$.
\end{enumerate}




$(p,q,r,s)$-system:
\begin{align*}
 \dpp&=p\bigg\{\left[\frac{1}{\lambda }\Big(r-a_0\Big)\right] & &-\left[d_1 + q + \lambda pr\right]\bigg\}\\
 \dqq&=q\bigg\{\left[b_1 +\frac{bpr}{q}\right] & &-\left[d_1 + q + \lambda pr\right]\bigg\}\\ 
 n\drr&=r\bigg\{\left[\frac{\alpha-m-n}{\lambda(1+\alpha) }\Big(r-a_0\Big)\right]& &+\left[d_1 + q + \lambda pr\right]+ \Big[\frac{\alpha}{1+\alpha}\, \Big(\frac{1+m+n}{\lambda} \Big(\frac{r^n}{1+\sqrt{n}s^n}-1\Big)r + \frac{n}{\lambda}\Big)\Big]\bigg\}\\
 n\dss&=-\frac{1+m+n}{\lambda}rs & &+ \sqrt{n}s^{1-n}\Big( \frac{1+m+n}{\lambda} \, r\frac{r^n-1}{n} + \frac{1+\sqrt{n}s^n}{\lambda}\Big).
\end{align*}

\begin{remark}
 \begin{enumerate}
  \item 
  \item Note that $s$ is not a fast variable. Even though $\displaystyle\frac{ \frac{1}{1+\alpha} \tth^{1+\alpha} }{ \frac{1}{1+m+n} \tg^{1+m+n} }$ relaxes to the manifold $1 + \mathcal{O}(n)$, the relaxation time is not of $\mathcal{O}(\frac{1}{n})$ but is of $\mathcal{O}(1)$.
  
 \end{enumerate}
\end{remark}



$(p,q,r,s)$-system:

\begin{equation}
\begin{aligned}
  {\dpp}&=p\bigg\{\Big[\frac{1}{\lambda }\Big(r-a_0\Big)\Big] -\Big[d_1 + q + \lambda p r\Big]\bigg\}\\
  {\dqq}&=q\bigg\{b_1-d_1 + \lambda p r\bigg\} +bpr,\\
 n{\drr}&=r\bigg\{\left[\frac{\alpha-m-n}{\lambda(1+\alpha) }\Big(\frac{\tu}{\tg}-a_0\Big)\right]& &+\left[d_1 + b\frac{\tv}{\ts} + \lambda\frac{\tu}{\tv}\frac{\tv}{\ts}\right]\bigg\}\\
 n\frac{\dot{s}}{s} &= \frac{1+m+n}{\lambda}\Big(r^{1+n}s^{\frac{\alpha-n}{1+\alpha}} - r \Big) + \frac{n}{\lambda}
\end{aligned}
\end{equation}

\begin{remark}
 In the case of the variable $r$, assuming $\tg, \ts,\tth \sim \mathcal{O}(1)$, the exponent $n$ is natural and the relaxation time scale $\drr \sim \mathcal{O}(\frac{1}{n})$. For the case of $\displaystyle\left(\frac{ \frac{1}{1+\alpha}\tth^{1+\alpha}}{ \frac{1}{1+m}\tg^{1+m} } -1 \right)$, there is no preferred relaxation time scale. For the time being, we do not specify the function $s(n,z-1)$ in the upcoming calculations.
\end{remark}
\begin{align*}
 \dot{s} =\frac{\partial s}{\partial (z-1)} \dot{z}
 &= \frac{1+m}{\lambda}\frac{\partial s}{\partial (z-1)} z\bigg\{-r - \frac{n}{D}+ \frac{r}{z}u^n\bigg\}\\
 &=\frac{1+m}{\lambda}\frac{\partial s}{\partial (z-1)} \bigg\{r(u^n-1) +r(1-z) -n\frac{z}{D}\bigg\}.
\end{align*}
We choose $s(n,z)$ such that
\begin{enumerate}
 \item As $(n,z-1) \rightarrow (0,0)$, $s(n,z-1) \rightarrow 0$, 
 \item For fixed $n$, the map $z \mapsto s$ is invertible, and for inverse $z=z(n,s)\rightarrow 1$ as $(n,s) \rightarrow (0,0)$.
\end{enumerate}
We set 
$$ s(n,z-1) = \frac{(z-1)^{\frac{1}{n}}}{n}, \quad \text{or} \quad z= 1+ns^n$$
and look for solutions of this form. Then we have


\begin{align*}
 \frac{\ts\tu}{\tth} &= \frac{1+m}{1+\alpha} \frac{r}{z}\,u^n = \frac{1+m}{1+\alpha} r + n\frac{1+m}{1+\alpha} \frac{r}{1+ns^n}\Big(\frac{u^n-1}{n}-s^n\Big),%\\
%  \frac{\tu}{\tv}\frac{\tv}{\ts}&=\frac{\ts}{\tv} \frac{\tg}{\ts} \frac{\tu}{\tg} \frac{\tv}{\ts} = pr.
\end{align*}





\begin{equation}
\begin{aligned}
  {\dpp}&=p\bigg\{\Big[\frac{1}{\lambda }\Big(r-a_0\Big)\Big] -\Big[d_1 + q + \lambda p r\Big]\bigg\}\\
  {\dqq}&=q\bigg\{b_1-d_1 + \lambda p r\bigg\} +bpr,\\
 n{\drr}&=r\bigg\{\Big[\frac{-m-n}{\lambda }\Big(r-a_0\Big)\Big]+\Big[d_1 + q + \lambda p r\Big]+\Big[\frac{\alpha}{\lambda }\Big(\frac{1+m}{1+\alpha}r-c_0\Big)\Big] + n\Big[\frac{\alpha}{\lambda }\frac{1+m}{1+\alpha} \frac{r}{1+ns^n}\Big(\frac{u^n-1}{n}-s^n\Big)\Big]\bigg\}\\
 n\dot{s}&=\frac{1+m}{\lambda}s^{1-n} \left\{r\frac{u^n-1}{n} - rs^n -\frac{1+ns^n}{D}\right\}.
\end{aligned}
\end{equation}

\section{Normally hyperbolic invariant manifold}
\section{Equilibrium points, Linear structure}
\subsection{Characterization of the heteroclinic orbit : why and how}
\subsection{Asymptotic behavior of self-similar variables in $\xi$}
\section{A $k$-parameter family of shear banding solutions}
\subsection{Asymptotic behavior of field variables in $t$ and $x$}
\section{Existence via Geometric theory of singular perturbation}



\end{document}