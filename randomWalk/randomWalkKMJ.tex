\documentclass{beamer}

\usepackage{mathtools}
\usepackage{amsmath,amssymb,amsthm,psfrag,subfigure}
\usepackage{setspace}
\usepackage{color}
\usepackage{cancel}

%%%% macros from beamer.cls %%%%%%%%%%%%%%%%
  \usetheme{Boadilla}
  \usecolortheme{whale}
  \setbeamertemplate{footline}[frame number]
  \setbeamercovered{transparent}


%%%% my macros %%%%%%%%%%%%%%%%%%%%%%%%%%%%
  \def\red{\color{red}}
  \def\blue{\color{blue}}
  \def\green{\color{green}}
\def\Lip{{\mathrm{Lip}}}
\def\R{\mathbb{R}}
\def\Z{\mathbb{Z}}
\def\div{\,\textrm{div}\,}

\newtheorem{proposition}{Proposition}

\addtobeamertemplate{navigation symbols}{}{ \hspace{1em}    \usebeamerfont{footline}%
    \insertframenumber /\inserttotalframenumber }

\begin{document}
\title[Random walk in spatially heterogeneous media ]{\LARGE A model of aggregation via \\ random walk in spatially heterogeneous media }
\author{\vskip 5pt \small {Min-Gi Lee}\\ \vskip 5pt \scriptsize Department of Mathmatics\\Kyungpook National University, Korea\\}
%\author{\vskip 5pt\Large {\underline{Thesis Defense}}}
\institute{\vskip 5pt Yong-Jung Kim, KAIST \\ Jaywan Chung, Korea Electrotechnology Research Institute}
\date{~}
\begin{frame}
  \titlepage
  \vskip -60pt
  \center
  {\small
  {\scriptsize KMJ Conference, Daegu, Korea}\\
  %   {\scriptsize University of Vienna, Vienna, Austria}\\
  %{\scriptsize Universit\"at Wien, Austria}\\
  %{\scriptsize Universit\"at Stuttgart, Stuttgart, Germany}\\
  %{\scriptsize  Texas A\&M University, U.S.}\\
  {\scriptsize February 14th, 2019}\\}
%   {\scriptsize Group Meeting, KAUST}\\
%   {\scriptsize  Thuwal, Saudi Arabia}\\
%   {\scriptsize Octobor 13th, 2015}\\}
\end{frame}
% 
% 
% \title[Localization ]{Localization in dynamic plasticity: \\ A study of parabolic regularizations  \\
% of elliptic initial value problems}
% % \title{{\LARGE Two-parameters family of focusing self-similar solutions in 1-d thermo-visco-plasticity}\\ {\small(Introduction to Geometric Singular Perturbation Theory)}}
% \author{\vskip 5pt \small {Min-Gi Lee}\\ \vskip 5pt \scriptsize Applied Maths. \& Computational Science, KAUST, SA\\}
% %\author{\vskip 5pt\Large {\underline{Thesis Defense}}}
% 
% \vskip 15pt
% \institute{\scriptsize
% Joint works with  \\ \textbf{\color{red}J. Olivier}, Universite de Marseilles, FR\\
% \textbf{\color{cyan} Th. Katsaounis}, KAUST, SA \\
% \textbf{\color{blue} A. Tzavaras}, KAUST, SA}
% 
% % \institute{\vskip 5pt joint with Athanasios Tzavaras \quad and \quad Thodoros Katsaounis\\}
% \date{~}
% \begin{frame}
%   \titlepage
%   \vskip -60pt
%   \center
%   {\small
%   %{\scriptsize Summer School in Nonlinear Partial Differential Equations}\\
%   {\scriptsize September 28th, 2017, Group Meeting}\\}
%   %{\scriptsize Universit\"at Wien, Austria}\\
%   %{\scriptsize Universit\"at Stuttgart, Stuttgart, Germany}\\
%   %{\scriptsize  Texas A\&M University, U.S.}\\
% %   {\scriptsize June 20th, 2017}\\}
% %   {\scriptsize Group Meeting, KAUST}\\
% %   {\scriptsize  Thuwal, Saudi Arabia}\\
% %   {\scriptsize Octobor 13th, 2015}\\}
% \end{frame}

\section{Introduction}
\begin{frame}
 \frametitle{Mass migration system}
 \setstretch{2}
 
%  \begin{align*}
%   \rho : \text{a density function of a certain mass (chemicals, population, ...)}\\
%   \int_\Omega \rho \; dx = M_0 : \text{Total mass}\\
%   \int_E \rho \; dx : \text{Mass residing in a set $E\subset \Omega$}
%  \end{align*}

 $\rho$ : a {\green density} function of a certain mass (chemicals, populations, ...), \pause
 
 i.e., $\displaystyle \int_\Omega \rho \; dx = M_0$ is the {\blue total mass},
 
 $\displaystyle \int_E \rho \; dx $ is the {\blue mass residing in a set $E\subset \Omega$}. \pause
 \vfill
 We consider a model where mass moves around\\
 by the law given by the {\blue flux $\mathbf{f}$}
 $$ \partial_t \rho = -\div( \mathbf{f} ).$$%, \quad \text{where $\mathbf{f}$ is a flux}.$$
\end{frame}

\begin{frame}
 \frametitle{Mass migration system: convection}
 \setstretch{1.5}
 Suppose that $ \mathbf{f} = \rho V$, so that
 $$ \partial_t \rho = -\div( \rho V(t,x) ), \quad \text{where $V(t,\cdot)$ is a velocity vector field}. $$
%  \begin{align*}
%   \rho : \text{a density function of a certain mass (chemicals, population, ...)}\\
%   \int_\Omega \rho \; dx = M_0 : \text{Total mass}\\
%   \int_E \rho \; dx : \text{Mass residing in a set $E\subset \Omega$}
%  \end{align*}
    \begin{minipage}{0.6\linewidth}
    \begin{figure}
    \centering
    \psfrag{xt}{\scriptsize $x$ at $t=t_1$}
    \psfrag{x0}{\scriptsize $x$ at $t=0$}
%     \psfrag(x){\scriptsize $x$}
    \includegraphics[width=5cm]{mediaflow.eps}
    \end{figure}
    \end{minipage}    
    \hfill
    \begin{minipage}{0.3\linewidth}
    Solve o.d.e., 
    \begin{equation*}
        \left\{\begin{array}{l}
        \frac{d \mathbf{x}}{dt} = V(t,x),\\
        \mathbf{x}(0)= \mathbf{x}_0.
        \end{array}\right.
    \end{equation*}
    \end{minipage}

    \vspace{1em}
    In this case, $ \displaystyle\rho(t, \mathbf{x}(t)) = \frac{\rho(t, \mathbf{x}(0))}{ \det \frac{\partial \mathbf{x}}{\partial \mathbf{x}_0}}$ solves the equation.
    
    This describes a mass flow passively conveyed by the background media deformation from $ \mathbf{x}(0)$ to $ \mathbf{x}(t) $.
    
\end{frame}

\begin{frame}
 \frametitle{Mass migration system: convection}
 \setstretch{1.5}
 Its self-governing nonlinear version can be considered:
 $$ \partial_t \rho = -\div( \rho V(\rho) )$$

 In this case, the velocity field is a non-linear function of the density itself. 
\end{frame}

\begin{frame}
 \frametitle{Mass migration system: convection}
 \setstretch{1.5}
 Interaction with another component also can be considered. For instance, %if $c$ is a certain chemical that attract or repell the mass,
 $$ \partial_t \rho = -\div( \rho V(c,\nabla c) )$$

 $V = \nabla c$ : $c$ is a chemo-attractant \: / \:  
 $V = -\nabla c$ : $c$ is a chemo-repellent.
 
equipped with the equation for $c$ as well:

$$\partial_t c = f(\rho) - g(c).$$

(chemical secretion and degradation)
\end{frame}

\begin{frame}
 \frametitle{Mass migration system: diffusion}
 \setstretch{1.5}
    \begin{minipage}{0.6\linewidth}
        $$\partial_t \rho = - \div( \mathbf{f}), \quad \mathbf{f} = -\mu \nabla \rho$$
        describes the diffusion. This flux law is called the Fick's law.
    \end{minipage}
    \begin{minipage}{0.3\linewidth}
    \hfill
    \begin{figure}
    \centering
%     \psfrag(x){\scriptsize $x$}
    \includegraphics[width=1.5cm]{diffusionink.eps}
    \end{figure}
    \end{minipage}    
    Diffusion is a {\blue macroscopic} phenomenon, which we can observe.
    
    It is well-known that the {\blue microscopic model of} {\red Random Walk} gives rise to the Fick'w law and thus explains the diffusion phenomenon.

\end{frame}

\begin{frame}
 \frametitle{Microscopic model: Random Walk}
 \setstretch{1.5}

    \begin{figure}
    \centering
     \psfrag{dx}{\scriptsize $\Delta x$}
     \psfrag{dt}{\scriptsize $\Delta t$}
     \psfrag{pl}{\scriptsize $p_\ell = \frac{1}{2}$,}
     \psfrag{pr}{\scriptsize $p_r = \frac{1}{2}$}
    \includegraphics[width=5cm]{randomwalkp.eps}
    \end{figure}
  
    Random walk is parameterized by $\Delta x$, the walk lenth, and $\Delta t$, the time interval between two walks, which are of particle size scale.
    
    Let $p^k_n$ be a probabibility mass function at $k$-th site and at time $t=t_n$. Then, easy to see 
    $$ p^k_{n+1} = \frac{1}{2}p^{k-1}_{n} + \frac{1}{2}p^{k+1}_n.$$
    
    Once the initial p.m.f. $p^k_0$ is known, $p^k_n$ for all $n\ge0$ is computable.
\end{frame}

\begin{frame}
 \frametitle{Microscopic model: Random Walk}
 \setstretch{1.5}

%     \begin{figure}
%     \centering
%      \psfrag{dx}{\scriptsize $\Delta x$}
%      \psfrag{dt}{\scriptsize $\Delta t$}
%      \psfrag{pl}{\scriptsize $p_\ell = \frac{1}{2}$,}
%      \psfrag{pr}{\scriptsize $p_r = \frac{1}{2}$}
%     \includegraphics[width=5cm]{randomwalkp.eps}
%     \end{figure}


    One consider the {\blue probability density} $\rho^k_n:= \frac{p^k_n}{\Delta x}$ at each interval midpoint. This may be interpolated smoothly to form a density function $\rho(x)$. This is parametrized by $\Delta x$. \pause
    
    Since $\Delta x$ and $\Delta t$ are {\red very small}, we may want the limit $\Delta x \rightarrow 0$, $\Delta t \rightarrow 0$.
    
    For the phenomenon to be visible even in {\blue macroscopic scale}, 
    
    the ratio $d = \frac{(\Delta x)^2}{\Delta t}$ is kept constant in the limit. (i.e., $\Delta t$ follows from $d$ and $\Delta x$.)
    
    \vfill
    
    It is verifiable that {\red $\rho_{\Delta x}$ converges to a function $\rho^*$} that solves
    $$\partial_t \rho = d \partial^2_{xx}\rho$$
    in an appropriate sense.
\end{frame}


\begin{frame}
 \frametitle{Convection revisited}
 \setstretch{1.5}

 Having those all said, despite it is unnecessary, one can ask what happens microscopically for the convection case
 \vspace{-1em}
 $$ \partial_t \rho = - \div (\rho V(t,x)).$$
 
  \vspace{-1em}
 It is, of course, necessary to have a well-defined velocity $V(t,x)$ at every location $x$ at time $t$. \pause
 
 Let us consider a chemotaxis model with chemo-attractant $c$ for a population of micro-organism (bacteria, for instance).
  \vspace{-1em}
 $$ \partial_t \rho = d\Delta \rho - \div (\rho \nabla c).$$
 
 \vspace{-1em}
 This assumes bacteria who are capable to measure the {\blue gradient of concentration of $c$} at a location.
\end{frame}

\begin{frame}
 \frametitle{Convection revisited}
 \setstretch{1.3}
 
 If oscillations in $c$ are not sufficiently small, compared to the size of the micro-organism, the micro-organism never be able to measure the {\blue gradient of $c$}.

    \begin{figure}
    \centering
     \psfrag{x}{\scriptsize $x$}
     \psfrag{c}{\scriptsize $c$}
    \includegraphics[width=3.5cm]{measuregradient.eps}
    \end{figure}
    
 While it can measure or approximate the density quite accurately by averaging 
 $$\frac{1}{2\epsilon} \int_{x-\epsilon}^{x+\epsilon} c \; dx.$$
\end{frame}

\begin{frame}
 \frametitle{Task}
 \setstretch{1.5}

 It'd nice to have a model where
 
 {\blue the only mechanism} for a micro-organism to move is the {\red random walk}, not measureing any gradient.
 
 Instead, we let the micro-organism can changes the random walk behavior ($\Delta x$ and $\Delta t$) based on the measurements of $\rho$ and $c$. 
 
 ({\blue \scriptsize More specifically, we let it be slow when it feels satisfactory and be fast when it does not.})
 
 \vfill
 This indicates, as the first stage to its study, that we need to understand the {\red random walk in spatially heterogeneous media where $\Delta x$ and $\Delta t$ vary}.
 
 \end{frame}

 
\begin{frame}
 \frametitle{Heterogeneous environment}
 \setstretch{1.5}
 The simplest heterogeneous environment is considered:
 
 \begin{figure}
  \centering
  \psfrag{x}{\scriptsize $x$}
  \psfrag{O}{\scriptsize $O$}
  \psfrag{dtl}{\scriptsize $\Delta t_L$}
  \psfrag{dxl}{\scriptsize $\Delta x_L$}
  \psfrag{dtr}{\scriptsize $\Delta t_R$}
  \psfrag{dxr}{\scriptsize $\Delta x_R$}
  \includegraphics[width=5cm]{Twosubstrates.eps}
 \end{figure}
 Physical instance: semiconductor exposed in water vapor.

 \vfill
 We'd let $d_L = \frac{(\Delta x_L)^2}{\Delta t_L}$ and $d_R = \frac{(\Delta x_R)^2}{\Delta t_R}$.
 
 
 Would the macroscopic diffusion model that the random walk gives rise to be the 
 $$ \partial_t \rho = \div (d(x) \nabla \rho) \quad \quad?$$
 
 \end{frame} 
 
 \begin{frame}
 \frametitle{Macroscopic diffusion model}
 \setstretch{1.2}
The model in the below will raise the jump discontinuity,
$$ \partial_t \rho = \div (\nabla (d(x) \rho) ).$$\pause

Then, why not 
$$ \partial_t \rho = \div (\sqrt{d(x)}\nabla (\sqrt{d(x)} \rho) ) \quad \quad ?$$
or
$$ \partial_t \rho = \div (d(x)^q\nabla (d(x)^{1-q} \rho) ) \quad \quad ?$$
More complicated:
$$ \partial_t \rho = \div (\Delta x^q\nabla (\frac{\Delta x^{2-q}}{\Delta t} \rho) ) \quad \quad ?$$

 
 \end{frame} 
 
  \begin{frame}
 \frametitle{Partial answer}
 \setstretch{1.5}
The partial anwser to the question which macroscopic model is right is that 

{\red ``different microscopic model can give rise to different macroscopic diffusion''.} \pause

Indeed, the types of diffusion are diverse, nonlinear, nonlocal, .... Mostly are from stochastic process study. \pause

In the rest of the talk, we will pick a very simple microscopic model and we assert that this microscopic random walk gives rise to macroscopically
$$\partial_t \rho = \div (\Delta x \nabla \big( \frac{\Delta x}{\Delta t} \rho\big) )$$
which suffices to raise the jump discontinuity.
 
 \end{frame} 
 
  \begin{frame}
 \setstretch{1.5}

  First, we examine the effect of having heterogeneous $\Delta t$; we let $\Delta x = h$ to be constant, while we let
 \begin{equation*}\label{alpha}
\Delta t=\alpha(x)\tau,\qquad
\alpha(x) =
\left\{\begin{array}{ll}
        1, & \text{if } x<0,\\
        2, & \text{if } x\ge0.
        \end{array}\right.
\end{equation*}

\vspace{-1em}
We also let ${h^2\over 2\tau}=d.$

In our setup, $\Delta t$ is interpreted as the {\blue traveling time} for the particle that is defined on the middle of each space {\blue interval}.
 \begin{figure}
  \centering
  \psfrag{x}{\scriptsize $x$}
  \psfrag{O}{\scriptsize $O$}
  \psfrag{dtl}{\scriptsize $\Delta t_L=\tau$}
  \psfrag{dtr}{\scriptsize \hskip-2em $\Delta t_R=2\tau$}
  \psfrag{time}{\scriptsize it takes twice as much time here}
  \includegraphics[width=5cm]{Twosubstrates2.eps}
 \end{figure}
\end{frame} 
 \begin{frame}
 \frametitle{Random walk in heterogeneous media}
 \setstretch{1.5}
 Insomuch as we did before, this random walk is described by
 \begin{equation} \label{RW1}\tag{RW1}
\begin{aligned}
    2p^k_n = \left\{\begin{array}{lr}
        p^{k-1} _{n-1} + p ^{k+1} _{n-1}, & \text{if } k<0,\\
        p^{-1} _{n-1} + p ^{1} _{n-2}, & \text{if } k=0,\\
        p^{k-1} _{n-2} + p ^{k+1} _{n-2}, & \text{if } k>0.
        \end{array}\right.
\end{aligned}
\end{equation} 
Note that the state at $n$-th step depends two previous steps. \pause Alternatively, we expand $(n-1)$-th step to write

\begin{equation} \label{RW2}\tag{RW2}
\begin{aligned}
    &p^{k}_n = \left\{\begin{array}{ll}
        \frac{1}{4}p^{k-2} _{n-2} + \frac{1}{2}p ^{k} _{n-2} + \frac{1}{4} p^{k+2}_{n-2}, & \text{if $k<0$ is even},\\
        \frac{1}{4}p^{-2} _{n-2} + \frac{1}{4}p^{0} _{n-2} + \frac{1}{2}p ^{1} _{n-2}, & \text{if $k=0$},\\
        \frac{1}{2}p^{k-1} _{n-2} + \frac{1}{2}p ^{k+1} _{n-2}, & \text{if } k>0.
        \end{array}\right. 
\end{aligned}
\end{equation}
 \end{frame}  

  \begin{frame}
  \frametitle{Connection to macroscopic equation}
   \setstretch{1.5}
   We assert that $\partial_t \rho = d \partial^2_{xx} \big(\frac{\rho}{\alpha(x)}\big)$. By defining $w=\frac{\rho}{\alpha(x)}$, we prove instead $w$ solves
   \begin{equation} \label{eqnw}\alpha(x)\partial_t w = d \partial^2_{xx} w. \end{equation}\pause
 \begin{definition}[Weak solution] We call $L^1_{loc}\big(\R^+;W^{1,1}(\R)\big)$ a weak solution of
\eqref{eqnw} for an initial data $w_0 \in L^1(\R)$ if, for any test function $\phi \in
C_c^1(\R^+\times \R)$,
 \begin{equation}\label{DefWeak}
  \int_0^T\int_\R \alpha w\phi_t\,dxdt
  +\int_\R\alpha w_0\phi(0,x)\,dx = d\int_0^T\int_\R w_x\phi_x\, dxdt.
 \end{equation}
\end{definition}
\end{frame}

\begin{frame}
\begin{lemma} \label{misc}
Let $w_0\in C^{\infty}_c(\R)$ and $w^h$ be defined by {\red (* linear interpolation)}. We assume that $\tau$ and $h$ are related by
$$
{h^2\over 2\tau}=d,
$$
where $d>0$ is fixed. Then, for any $h>0$ small and any $t>0$, 
$$\|\partial_x w^h(t,\cdot)\|_{L^\infty(\R)} + \|\partial_t w^h(t,\cdot)\|_{L^\infty(\R)} \le C,$$
where $C$ depends only on $d$ and $\|w_0\|_{W^{2,\infty}(\R)}.$
\end{lemma}

\begin{theorem}[Kim, L]
  The interpolation $w^{h}(t,x)$ converges to $w\in C^{0,1}(\R^+\times\R)$ and the limit is a weak solution of \eqref{eqnw}.
\end{theorem}
\end{frame}
\begin{frame}
   \setstretch{1.5}
   To turn back to the aggregation model, it is not strange to consider a model
   \begin{align*}
    \partial_t \rho &= d \div \nabla ({\red \gamma(c,\rho)} \rho),\\
    \partial_t c &= f(\rho) - g(c) + \epsilon \Delta c.
   \end{align*}
   
   Global existence \& Numerical evidence of aggregation : Desvillettes, Kim, Trescases, Yoon. \pause

   {\it Interpretation of the induced mass flow: a determined person among purposelessly wandering people?}

\end{frame}


\begin{frame}
 \center
 \vfill
 
 {\bf \huge \blue  Thank you very much.}
 
 \vfill
\end{frame}


\end{document}

