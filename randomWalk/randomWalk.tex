\documentclass[11pt]{amsart}
%\usepackage{amsmath,amsfonts,amssymb,mathrsfs}
%\usepackage{amssymb,mathrsfs}

\usepackage{amsfonts,amssymb,amsmath,mathrsfs,graphicx}
\usepackage{float}
%\usepackage{srcltx}
\usepackage{epsfig}
\usepackage{graphicx}
\usepackage{caption}
\usepackage{subcaption}
%\usepackage{refcheck}

\hoffset=-30mm \voffset=-32mm
%\usepackage[left=35mm,right=35mm,top=35mm,bottom=40mm,a4paper]{geometry}

\title[Random walk with location dependent travelling time]{Diffusion equation for spatially heterogeneous random walk}% with location dependent travelling time}

\author[Jaywan Chung]{Jaywan Chung}
\address[Jaywan Chung]{\newline Department of Mathematics, Dankook University \newline 119 Dandae-ro, Dongnam-gu, Cheonan-si, Chungnam 330-714, Korea}
\email{jaywan.chung@gmail.com}


\author[Yong-Jung Kim]{Yong-Jung Kim}
\address[Yong-Jung Kim]{\newline Department of Mathematical Sciences, KAIST \newline 291 Daehak-ro, Yuseong-gu, Daejeon, 305-701, Korea}
\email{yongkim@kaist.edu}


\author[Mingi Lee]{Mingi Lee}
\address[Mingi Lee]{\newline ...}
\email{...}

\def\eps{\epsilon}
\def\R{{\bf R}}


\setcounter{tocdepth}{3}

%%%%%%%%%%%%%% MY DEFINITIONS %%%%%%%%%%%%%%%%%%%%%%%%%%%%
\def\d{d}
\def\v{{\bf v}}
\def\e{{\bf e}}
\def\x{{x}}
\def\k{{\bf k}}
\def\p{{\bf p}}
\def\q{{\bf q}}
\def\R{{\bf R}}
\def\Z{{\bf Z}}
\def\N{{\bf N}}
\def\bfR{{\bf R}}
\def\bfZ{{\bf Z}}
\def\f{{\bf f}}
\def\div{{\rm div}}
\def\tp{\Sigma}
\def\erf{\mathrm{erf}}
\def\erfc{\mathrm{erfc}}
%%%%%%%%%%%%%%%%%%%%%%%%%%%%%%%%%%%%%%%%%%%
\newtheorem{theorem}{Theorem}[section]
\newtheorem{lemma}{Lemma}[section]
\newtheorem{corollary}{Corollary}[section]
\newtheorem{proposition}{Proposition}[section]
\newtheorem{remark}{Remark}[section]
\newtheorem{definition}{Definition}[section]
\newtheorem{example}{Example}[section]

\begin{document}
% The correct dates will be entered by the editor
\maketitle
%
\begin{abstract}
  Random walk is a mathematical counterpart of Brownian motion and has been used as a theoretical tool to understand random dispersal phenomena. If temperature is not spatially constant, the mean free path and the speed of a Brownian particle are heterogeneous in space. To include such a spatial heterogeneity in a random walk system, we consider a case that the walk length and jumping time are spatially nonconstant. We derive corresponding diffusion equation and construct a fundamental solution of a special case that the jumping time has discontinuity. One of interesting findings is that we cannot sense the change of the walk length of the other side, but can sense the change of the jumping time.
\end{abstract}

%\tableofcontents



\section{Introduction}

The purpose of this paper is to derive the partial differential equation satisfied by the diffusion limit of the probability density distribution of a random walk system. We are interested in a random dispersal phenomenon which is spatially heterogeneous. For example, it is known well that the homogeneous diffusion theory fails if temperature is spatially heterogeneous. The phenomenon is called thermal diffusion and steady states are not constant in the case. If temperature is spatially heterogeneous, the mean free path and the speed of a Brownian particle are not constant. To include such a spatial heterogeneity in our random walk system, we assume that the walk length $\Delta x$ and the jumping time $\Delta t$ are spatially nonconstant.

If $\Delta x=h$ and $\Delta t=\tau$ are constant, the diffusion limit $u$ of the probability density functions in one space dimension satisfies a diffusion equation,
\begin{equation}\label{HomoDiffusion}
u_t=d_0u_{xx},\qquad d_1:={h^2\over 2\tau},
\end{equation}
where the sub-indexes indicate partial derivatives. The diffusivity constant $d_1$ is given by the Einstein-Smoluchowski relation in one space dimension. The multi-dimensional case is similarly obtained, where the diffusivity becomes $d_n:={h^2\over 2n\tau}$ in $\R^n$.

The diffusion equation \eqref{HomoDiffusion} is valid only when the diffusivity is constant. There have been many efforts to find a correct diffusion equation when the diffusivity is non-constant, i.e., when $d=d(\x)$. For example, three diffusion models are often considered:
\begin{eqnarray}
% \nonumber % Remove numbering (before each equation)
\label{Fick} u_t &=& (d(\x)u_x)_x, \\
\label{Wereide} u_t &=& ( \sqrt{d(\x)}(\sqrt{d(\x)}u_x))_x, \\
\label{Chapman} u_t &=& (d(\x)u)_{xx}.
\end{eqnarray}
Eq. \eqref{Fick} is called Fick's law. Eq. \eqref{Wereide} and \eqref{Chapman} have been by Wereide \cite{Wereide1914} and Chapman \cite{Chapman1928}, respectively. These three equation are identical to each other if the diffusivity $d$ is constant. However, any two of them does not agree to each other. In particular, the Fickian type diffusion \eqref{Fick} and the Fokker-Planck type one \eqref{Chapman} are often taken and compared as physical or biological dispersal models in heterogeneous environment (see \cite{Bengfort2016,Milligen}).

These three classical diffusion equations also appear as model equations satisfied by the probability density functions of stochastic processes when the variation of the random variable (i.e., $|\Delta x|^2$ for the random walk case) depends on the terminal, middle, or starting point of each jump, respectively. In other words, \eqref{Wereide} and \eqref{Chapman} are obtained by the Stratonovich and Ito type integrals of a stochastic process, respectively. Note that in the three cases of stochastic processes, the spatial heterogeneity is in variation of the random variable (i.e., in $|\Delta x|^2$) only and the frequency of events (i.e., ${1\over\Delta t}$) is assumed to be spatially homogeneous. Including a spatial heterogeneity in the frequency of events is one of key contributions of this paper.

Let
\begin{equation}\label{ab}
\Delta x=ha(x)>0\ \ \text{and}\ \ \Delta t=\tau b(x)>0,
\end{equation}
where $h>0$ and $\tau>0$ denote the microscopic scale of walk length and jumping time, and $a(x)$ and $b(x)$ denote their macroscopic scale spatial heterogeneity. To denote the spatial heterogeneity in the frequency of events (or turning frequency) we take
\begin{equation}\label{mu}
\mu(x):=b(x)^{-1}.
\end{equation}
The claim of this paper is that the corresponding diffusion equation is
\begin{equation}\label{Kim}
u_t=d_1\Big(a(x)\Big({a(x)\over b(x)}u\Big)_x\Big)_x.
\end{equation}
If $a(x)=1$ and $b(x)=1$, the diffusivity becomes constant $d=d_1$ and hence the four cases, \eqref{Fick}-\eqref{Chapman} and \eqref{Kim}, are identical to each other. If $b(x)=1$ and $a(x)$ is not constant, \eqref{Kim} is identical to \eqref{Wereide}, the case of Wereide's diffusion or of Stratonovich integral. If $a(x)=1$ and $b(x)$ is not constant, \eqref{Kim} is identical to \eqref{Chapman}, the case of Chapman's diffusion or of Ito integral. We show that the diffusion limit of a probability density of a discrete random walk system satisfies \eqref{Kim} for a general spatial heterogeneity in $a(x)$. However, for $b(x)$, we show for a spacial case given by \eqref{b(x)}.

Random walk is mathematical counterpart of Brownian motion of particles suspended in a fluid. The parameters $\Delta x$ and $\Delta t$ respectively represent the mean free path and the mean collision time. Spatial heterogeneity in such a motion appears often, where an obvious source is the temperature gradient. The suspended particles move more actively in warm regions. The effect of temperature gradient has been studied for a long time under the name of thermal diffusion or Ludwig-Soret effect \cite{Ludwig1856,Soret1879}. There are enormous amount of researches related such phenomena (see \cite{Braun2004, Duhr2006, Eslamian2009, Harstad2009, Huang2010,14,Putnam2007,Srinivasan2011}). However, there is no comprehensive or generic thermal diffusion model such as Einstein's molecular level explanation for the homogeneous case. The study of spatially heterogeneous random walk can provide a theoretical background for the thermal diffusion theory.

The spatial variation in the jumping time interval $\Delta t$ makes a more profound difference than the one in $\Delta x$. In Section 2, we consider such a case when the jumping time is discontinuous at the origin and consists of two values:
\begin{equation}\label{b(x)}
\Delta x=h,\qquad
\Delta t=\tau b(x),\quad b(x)=
\begin{cases}
1 & \text{if $x<0$,}\\
2 & \text{if $x>0$}.
\end{cases}
\end{equation}
This is one of simplest cases for the heterogeneity in jumping time $\Delta t$ and may serve as the basic component of a general case. However, this simplicity allows us to see the effect of spatial heterogeneity in $\Delta t$. We show that the diffusion limit of discrete probability density distribution satisfies
\begin{equation}\label{ux1}
u_t=d_1\Big({u\over b(x)}\Big)_{xx}.
\end{equation}


In Section 3, we introduce a discrete random walk model when
$$
 \Delta x=ha(x),\qquad\Delta t=\tau.
$$
The spatial heterogeneity in the walk length $\Delta x$ can be easily handled by employing a new space variable,
$$
y(x):=\int_0^x \frac{1}{a(s)}\,ds.
$$
Then, the spatially heterogeneous random walk system is reduced to a classical homogenous one. It is classical that the diffusion limit of the homogenous random walk system gives the homogeneous diffusion equation \eqref{HomoDiffusion}. If we return back to the original space variable, we obtain a diffusion equation of the Wereide type,
$$
u_t=d_1(a(x)(a(x)u)_x)_x.
$$
Note that, even if this equation is not autonomous, we may solve it with a general initial value by using the new space variable.

For general case of \eqref{ab}, the corresponding diffusion equation is
\begin{equation}\label{FullDiffusion}
u_t=d_1\Big(a(x)\Big({a(x)\over b(x)}u\Big)_{x}\Big)_{x},
\end{equation}
where the proof of this paper is valid only for the $b(x)$ in \eqref{b(x)}.

\newpage

\section{Random walk with a spatially non-constant jumping time}

We consider a random walk system that a particle jumps to one of two adjacent grid points with the equal probability ${1\over2}$. The special feature of the random walk system in this paper is that the jumping time interval $\Delta t$ is different for two regions $x<0$ and $x>0$. To provide the meaning of situation more clearly, we consider $\Delta t$ as a traveling time to move to the next grid point. We let $\Delta t=\tau$ in the region $x<0$ and $\Delta t=2\tau$ in the other region $x<0$, i.e.,
\begin{equation}\label{alpha}
\Delta x=h,\quad
\Delta t= b(x)\tau,\qquad
 b(x) =
\left\{\begin{array}{ll}
        1, & \text{if } x<0,\\
        2, & \text{if } x\ge0.
        \end{array}\right.
\end{equation}
Note that the step function $b(x)$ is to denote the spatial heterogeneity in $\Delta t$ and $\tau$ is the microscopic scale of it.

The convergence proof depends on the fact that the traveling time in one region is exactly two times larger than the other. For a general case this method seems not to work. However, this case shows interesting diffusion phenomena when the jumping time is spatially heterogeneous.

\subsection{Time-space mesh and random walk algorithm}

We denote by $p_n^k$ the probability of a particle to be placed at $x=x^k$ at time $t=t_n$. The time-space mesh used in this paper is denoted by
$$
(t_n,x^k):=(n\tau,kh),\qquad n\in\Z_+,\ k\in\Z,
$$
where the time step size $\tau$ and the space mesh size $h$ are small and $n$ and $k$ are integers. Then, the probability $p_n^k$ satisfies
\begin{equation} \label{RW1}%\tag{RW1}
\begin{aligned}
    2p^k_n = \left\{\begin{array}{lr}
        p^{k-1} _{n-1} + p ^{k+1} _{n-1}, & \text{if } k<0,\\
        p^{-1} _{n-1} + p ^{1} _{n-2}, & \text{if } k=0,\\
        p^{k-1} _{n-2} + p ^{k+1} _{n-2}, & \text{if } k>0.
        \end{array}\right.
\end{aligned}
\end{equation}
Find that the probability $p^k_n$ is the average of two probabilities at two adjacent grid points at two moments, $t_{n-1}$ or $t_{n-2}$, depending on $k$. By computing the probabilities at  $t=t_{n-1}$ using the ones at $t=t_{n-2}$, we can rewrite \eqref{RW1} using even numbered time steps $n=2\ell$ only, i.e.,
\begin{equation} \label{RW2}%\tag{RW2}
\begin{aligned}
    &p^{k}_{2\ell} = \left\{\begin{array}{ll}
        \frac{1}{4}p^{k-2} _{2(\ell-1)} + \frac{1}{2}p ^{k} _{2(\ell-1)} +
        \frac{1}{4} p^{k+2}_{2(\ell-1)}, & \text{if $k<0$ is even},\\
        \frac{1}{4}p^{-2} _{2(\ell-1)} + \frac{1}{4}p^{0} _{2(\ell-1)} +
        \frac{1}{2}p ^{1} _{2(\ell-1)}, & \text{if $k=0$},\\
        \frac{1}{2}p^{k-1} _{2(\ell-1)} + \frac{1}{2}p ^{k+1} _{2(\ell-1)},
        & \text{if } k>0.
        \end{array}\right.
\end{aligned}
\end{equation}
Notice that the odd numbered indexes for $k<0$ never appears in this formula. Hence, the even numbered time steps, even numbered space grids for $k<0$, and all grids for $k>0$ are self-contained. This is the special property when $b(x)$ in one region is two times greater than the one in the other region. The proof of diffusion limit of this section depends on this structure.

Suppose that an initial probability density function be given by $v_0(x)$ for $x\in\R$. We assume that
\begin{equation} \label{initial}
 v_0 \in L^1( \R)\cap L^\infty(\R), \quad \int_ \R v_0(x)\, dx = 1, \quad
 w_0:=\frac{v_0(x)}{ b(x)} \in C^2(\R).
\end{equation}
Note that the regularity of the problem is on ${v(x)\over b(x)}$ and the smoothness assumption of the initial value is given on $\frac{v_0(x)}{ b(x)}$, not on $v_0(x)$. For a given mesh size $h>0$, we discretize the initial probability density function $v_0$ by
\begin{equation}\label{p0k}
p_{0}^{k} = \left\{\begin{array}{ll}
               0 & \text{if } k<0\ \text{is odd},\\
               \int_{x^k}^{x^{k}+2h} v_0(x)dx & \text{if } k<0\ \text{is
               even},\\
                \int_{x^k}^{x^{k}+h} v_0(x)dx  & \text{if } k\ge0.
               \end{array}\right.
\end{equation}
Then, $\sum_{j=-1}^{-\infty}p_{0}^{2j}+\sum_{j=0}^{\infty}p_{0}^{j}=1$ and, since \eqref{RW2} is self-contained, $\sum_{j=-1}^{-\infty}p_{2\ell}^{2j}+\sum_{j=0}^{\infty}p_{2\ell}^{j}=1$ for all $\ell>0$. Note that missing probabilities are all zero, $p_{2\ell}^{2j+1}=0$ if $j<0$, and \eqref{RW2} show the relations of the others. To denote the space grids with non-zero probability, we introduce
\begin{equation}\label{yj}
y^j=x^{2j}\quad\text{if}\quad j<0,\quad\text{and}\quad
y^j=x^{j}\quad\text{if}\quad j\ge0.
\end{equation}
Note that $p_{n}^{k}$ does not approximate the probability density function but the probability for the particle to be placed between two grid points. After dividing them with the domain size of the integrals in \eqref{p0k}, the probability density function is approximated by
\begin{equation}\label{vj}
v^{j}_\ell:={p^{2j}_{2\ell} \over2h}\quad\text{for}\quad
j<0\quad\text{and}
\quad v^{j}_\ell:={p_{2\ell}^{j}\over h}\quad\text{for}\quad j\ge0.
\end{equation}
The probability density distribution $v(t,x)$ will be the limit of $v^j_\ell$ as $h\to0$. However, we cannot use the relation \eqref{RW2} for $v^{j}_\ell$'s due to the non-symmetry in \eqref{vj}. Instead, we will take
\begin{equation}\label{pandw}
w^{j}_\ell:={p_{2\ell}^{2j}\over2h}\quad\text{if}\quad j<0,\quad\text
{and} \quad w^{j}_\ell=:{p_{2\ell}^{j}\over2h}\quad\text{if}\quad j\ge0.
\end{equation}
Then, $v^{j}_\ell=w^{j}_\ell$ for $j<0$ and $v^{j}_\ell=2w^{j}_\ell$ for
$j\ge0$, i.e.,
$$
v^j_\ell=w_\ell^j b(y^j).
$$
Divide the relation \eqref{RW2} by $2h$ and obtain a relation for $w^{j}_\ell$'s:
\begin{equation} \label{RW3}%\tag{RW3}
\begin{aligned}
    &w^{j}_\ell = \left\{\begin{array}{ll}
        \frac{1}{4}w^{j-1} _{\ell-1} + \frac{1}{2}w ^{j} _{\ell-1} +
        \frac{1}{4} w^{j+1}_{\ell-1}, & \text{if $j<0$},\\
        \frac{1}{4}w^{-1} _{\ell-1} + \frac{1}{4}w^{0} _{\ell-1} +
        \frac{1}{2}w ^{1} _{\ell-1}, & \text{if $j=0$},\\
        \frac{1}{2}w^{j-1} _{\ell-1} + \frac{1}{2}w ^{j+1} _{\ell-1}, &
        \text{if } j>0.
        \end{array}\right.
\end{aligned}
\end{equation}
Realize that $w^{j}_0$ discretizes $w_0(x)$ given in \eqref{initial}, i.e.,
\begin{equation} \label{vol_aver+pmf}
w_{0}^{j} ={1\over y^{j+1}-y^{j}}\int_{y^j}^{y^{j+1}} w_0(x)dx.
\end{equation}
In the rest of the section we will find the diffusion limit $w(t,x)$ of $w_\ell^j$'s as $h\to0$. Then, the probability density distribution $v(t,x)$ is given by the relation,
$$
v(x,t)=b(x)w(t,x),
$$
and the initial value becomes $v(0,x)=v_0(x)$ as given in \eqref{initial}.

\subsection{Difference quotients and continuum interpolation}

Difference quotients are useful in finding the regularity of a weak solution and the relation satisfied by the limit of finite difference schemes. Let
\begin{equation}\label{Ljl}
L^j_\ell := \frac{w_\ell^{j+1}-w_\ell^j}{y^{j+1}-y^j}
        =\left\{\begin{array}{ll}
        \frac{w^{j+1}_{\ell} - w^{j}_{\ell}}{2\l}, & \text{if $j<0$},\\
        \frac{w^{j+1}_{\ell} - w^{j}_{\ell}}{\l}, & \text{if }  j\ge0,
        \end{array}\right.
\end{equation}
which is an approximation of the gradient $w_x$. Let
\begin{equation}\label{Qjl}
Q^j_\ell := \frac{L^j_\ell-L^{j-1}_\ell}{2h}= \left\{\begin{array}{ll}
        \frac{1}{2\l^2}\big( \frac{1}{2} w^{j-1} _{\ell} - w ^{j} _{\ell} +
        \frac{1}{2} w^{j+1} _{\ell}\big), & \text{if $j<0$},\\
        \frac{1}{2\l^2}\big( \frac{1}{2} w^{-1} _{\ell} - \frac{3}{2}w
        ^{0} _{\ell} +w^{1} _{\ell}\big), & \text{if $j=0$},\\
        \frac{1}{2\l^2}\big(w^{j-1} _{\ell} - 2w^j_\ell + w ^{j+1}
        _{\ell}\big), & \text{if $j>0$}.
        \end{array}\right.
\end{equation}
Notice that $y^{j+1}-y^{j}=2h$ for $j<0$ and $y^{j+1}-y^{j}=h$ for $j\ge0$. Therefore, $Q^j_\ell$ is an approximation of $w_{xx}$ for $j<0$ and of ${1\over2}w_{xx}$ for $j\ge0$, i.e., of ${w_{xx}\over b}$.\footnote{Remember that it is NOT an approximation of $({w_x\over b})_x$. There is no delta measure like phenomenon in the relation. This observation will be useful when we consider a general case in the next paper.} From \eqref{RW3}, we obtain
$$
\begin{aligned}
    &w^{j}_\ell-w^{j}_{\ell-1} = \left\{\begin{array}{ll}
        \frac{1}{4}w^{j-1} _{\ell-1} - \frac{1}{2}w ^{j} _{\ell-1} +
        \frac{1}{4} w^{j+1}_{\ell-1}, & \text{if $j<0$},\\
        \frac{1}{4}w^{-1} _{\ell-1} - \frac{3}{4}w^{0} _{\ell-1} +
        \frac{1}{2}w ^{1} _{\ell-1}, & \text{if $j=0$},\\
        \frac{1}{2}w^{j-1} _{\ell-1}-w^{j}_{\ell-1} + \frac{1}{2}w ^{j+1}
        _{\ell-1}, & \text{if } j>0.
        \end{array}\right.
\end{aligned}
$$
The above three cases are written in the same form using $Q_\ell^j$, which is
\begin{equation} \label{RW1w}
\begin{aligned}
 \frac{w^j_\ell - w^j_{\ell-1}}{2\tau} &= \left(\frac{h^2}{2\tau}\right)
 \frac{1}{2h}\left( \Big(\tfrac{w^{j+1}_{\ell-1} -
 w^{j}_{\ell-1}}{y^{j+1}-y^j}\Big) -\Big(\tfrac{w^{j}_{\ell-1} -
 w^{j-1}_{\ell-1}}{y^{j}-y^{j-1}} \Big)\right)= d_1Q^j_{\ell-1}.
  \end{aligned}
\end{equation}
This is the main relation which will give the diffusion equation satisfied by the probability density function.

The relations in \eqref{RW3} can be used to find relations for $L^j_\ell$ and $Q^j_\ell$ which are
\begin{equation} \label{eq:L}
L^j_\ell = \left\{\begin{array}{ll}
        \frac{1}{4}L_{\ell-1}^{j-1} + \frac{1}{2}L_{\ell-1}^j +
        \frac{1}{4}L_{\ell-1}^{j+1}, & \text{if $j<0$},\\
        \frac{1}{2}L_{\ell-1}^{j-1} + \frac{1}{2}L_{\ell-1}^{j+1}, &
        \text{if $j\ge0$},
        \end{array}\right.
\end{equation}
and
\begin{equation}\label{eq:Q}
\begin{aligned}
Q^j_\ell %&= \frac{L^{j+1}_\ell - L^j_\ell}{\l}\\
&=\left\{\begin{array}{ll}
        \frac{1}{4}Q_{\ell-1}^{j-1} + \frac{1}{2}Q_{\ell-1}^{j} +
        \frac{1}{4}Q_{\ell-1}^{j+1}, & \text{if $j<0$},\\
        \frac{1}{4}Q_{\ell-1}^{-1} + \frac{1}{4}Q_{\ell-1}^{0} +
        \frac{1}{2}Q_{\ell-1}^{1}, & \text{if $j=0$},\\
        \frac{1}{2}Q_{\ell-1}^{j-1} + \frac{1}{2}Q_{\ell-1}^{j+1}, &
        \text{if $j>0$}.
        \end{array}\right.
 \end{aligned}
\end{equation}
Note that $L^j_\ell$ and $Q^j_\ell$ are obtained by averaging relations of previous time steps. This allows key estimates needed.

\begin{lemma}[Uniform estimates] \label{uniform_est} ~

\begin{enumerate}\item Total sums of $w_\ell^j$, $L_\ell^j$, and
$Q_\ell^j$ are constant, i.e.,
$$
\sum_{j} w^j_\ell = \sum_{j} w^j_0,\quad \sum_{j} L^j_\ell = \sum_{j}
L^j_0,\quad \sum_{j} Q^j_\ell = \sum_{j} Q^j_0.
$$
\item Sequences $w_\ell^j$, $L_\ell^j$, and $Q_\ell^j$ are bounded by the
    initial maximums, i.e.,
  $$
  |w^j_\ell| \le \sup _{j} |w^j_0|,\quad  |L^j_\ell| \le \sup _{j}
  |L^j_0|,\quad |Q^j_\ell| \le \sup _{j} |Q^j_0|.
  $$
\item We also have, for $p\ge1$,
$$
\sum _{j} |w^j_\ell|^p \le \sum _{j} |w^j_0|^p,\quad\sum _{j} |L^j_\ell|^p \le \sum _{j} |L^j_0|^p,\quad \sum _{j} |Q^j_\ell|^p \le
\sum _{j} |Q^j_0|^p.
$$
\end{enumerate}
\end{lemma}
\begin{proof}
The first two are obvious since the relations in \eqref{RW3},\eqref{eq:L}, and \eqref{eq:Q} are averaging processes. The third one is also obvious since $f(x)=|x|^p$ is a convex function for all $p\ge1$. One may apply Jensen's inequality to the averaging processes, \eqref{RW3},\eqref{eq:L}, and \eqref{eq:Q}. For example, \eqref{eq:L} turns into
$$
|L^j_\ell|^p \le \left\{\begin{array}{ll}
        \frac{1}{4}|L_{\ell-1}^{j-1}|^p + \frac{1}{2}|L_{\ell-1}^j|^p +
        \frac{1}{4}|L_{\ell-1}^{j+1}|^p, & \text{if $j<0$},\\
        \frac{1}{2}|L_{\ell-1}^{j-1}|^p + \frac{1}{2}|L_{\ell-1}^{j+1}|^p, &
        \text{if $j\ge0$}.
        \end{array}\right.
$$
Therefore, $\sum _{j} |L^j_\ell|^p \le\sum _{j} |L^j_{\ell-1}|^p\le\cdots\le\sum _{j} |L^j_0|^p$.
\end{proof}

Note that this lemma implies that we have enough regularity in $x$ variables. For example, the second assertion implies that the approximation of the second order derivative, $Q^j_\ell$, is uniformly bounded by the initial condition. Even higher order derivatives can be controlled in a similar way and these are possible since the system is an averaging process. Technical difficulty is in controlling the smoothness in $t$ variables. The third assertion implies that $W^{k,p}(\R)$ regularity is passed to the limit.

Now we construct functions defined on the continuum space $\R^+\times\R$ which approximates the discrete values $w^j_\ell$ defined at grid points.
We show the convergence of the continuum approximations as $h\to0$. We construct four different interpolations,
\begin{equation} \label{linear0}
\begin{aligned}
 w_1^h(t,x)&:= w^j_\ell,\\
 w_2^h(t,x)&:= \left(\tfrac{t-t_{2\ell}}{t_{2\ell+2}-t_{2\ell}} \right)
 w_{\ell+1}^{j} + \left(\tfrac{t_{2\ell+2}-t}{t_{2\ell+2}-t_{2\ell}} \right)
 w_{\ell}^{j},\\
 w_3^h(t,x)&:= \left(\tfrac{x-y^j}{y^{j+1}-y^j} \right) w_{\ell}^{j+1}
 + \left(\tfrac{y^{j+1}-x}{y^{j+1}-y^j} \right) w_{\ell}^{j},\\
  w^h(t,x)&:= \left(\tfrac{x-y^j}{y^{j+1}-y^j}
  \right)\left(\tfrac{t-t_{2\ell}}{t_{2\ell+2}-t_{2\ell}} \right)
  w_{\ell+1}^{j+1} + \left(\tfrac{y^{j+1}-x}{y^{j+1}-y^j}
  \right)\left(\tfrac{t-t_{2\ell}}{t_{2\ell+2}-t_{2\ell}} \right)
  w_{\ell+1}^{j}\\
 &+ \left(\tfrac{x-y^j}{y^{j+1}-y^j}
 \right)\left(\tfrac{t_{2\ell+2}-t}{t_{2\ell+2}-t_{2\ell}} \right)
 w_{\ell}^{j+1} + \left(\tfrac{y^{j+1}-x}{y^{j+1}-y^j}
 \right)\left(\tfrac{t_{2\ell+2}-t}{t_{2\ell+2}-t_{2\ell}} \right)
 w_{\ell}^{j}
  \end{aligned}
\end{equation}
when $(t,x)\in D_\ell^j:=[t_{2\ell}, t _{2\ell+2}) \times [y^j, y^{j+1})$. Note that $w_1^h$ is constant on $D_\ell^j$, and $w_2^h$ and $w_3^h$ are linear on $D_\ell^j$. The last one, $w^h(t,x)$, is continuous on $\R^+\times\R$. All of them has the value $w^j_\ell$ at $t=t_{2\ell}$ and $x=y^j$. We now see in the following lemma that $w^h(t,x)$ is Holder continuous with exponent $\gamma={1\over2}$.

\begin{lemma} \label{misc}
Let $w_0\in W^{1,1}(\R)\cap W^{1,\infty}(\R)$ and $w^h$ be defined by \eqref{linear0}. We assume that $\tau$ and $h$ are related by
$$
{h^2\over 2\tau}=d,
$$
where $d>0$ is fixed. Then,
\begin{enumerate}
 \item For any $t>0$ and $h>0$, $\|\partial_x w^h(t,\cdot)\|_{L^\infty(\R)} \le \|\partial_x w_0\|_{L^\infty(\R)}$ in a weak sense. Hence, $w^h(t,\cdot)$ is uniformly Lipschitz continuous in the $x$-variable.
 \item The approximations $w^h(\cdot,x)$ are uniformly Holder continuous in the $t$-variable with exponent ${1\over2}$ with respect to $h>0$ and $x\in\R$. In other words, there exists a constant $C>0$ such that, for any $x\in\R$ and $h>0$,
     \begin{equation}\label{Holder}
     |w^h(t,x)-w^h(s,x)|\le C\sqrt{|t-s|} \quad\text{for all } |t-s|\text{
     small}.
     \end{equation}
 \end{enumerate}
\end{lemma}
\begin{proof}
1. Since $w_0^j$ is the mean of $w_0(x)$ in the interval $(y^j,y^{j+1})$
(see \eqref{vol_aver+pmf}) and $w^h(0,\cdot)$ is the linear interpolation of
$(y^j,y^{j+1})$, the first assertion is clear by Lemma \ref{uniform_est}.

2. Let $L_0:=\sup |w_0'(x)|$. Then, $|L_0^{j}|\le L_0$ for any mesh size
$h>0$. It is enough to consider an estimate along mesh grids and let
$x=y^j$. By \eqref{Qjl} and \eqref{RW1w}, we have
$$
 \frac{|w^j_\ell - w^j_{\ell-1}|}{2\tau} \le d\sup|Q^j_{\ell}|\le {d\over
 h}L_0.
$$
Multiply $\sqrt{2\tau}$ both sides and obtain
$$
 {|w^h(t_{2\ell},y^j)-w^h(t_{2\ell-2},y^j)|\over\sqrt{2\tau}}=\frac{|w^j_\ell
 - w^j_{\ell-1}|}{\sqrt{2\tau}} \le d\sqrt{2\tau\over
 h^2}L_0=\sqrt{d}\,L_0.
$$
This inequality \eqref{Holder} holds for any $|s-t|<\tau$ and hence $w^h$ is Holder continuous with exponent ${1\over2}$. Since the upper bound $C:=\sqrt{d}\,L_0$ is independent of $h$ and $x$, $w^h$'s are uniformly Holder continuous.
\end{proof}

The functions $w^h(t,x)$ are uniformly Lipschitz continuous in the $x$-variable and uniformly Holder continuous with the exponent ${1\over2}$ in the $t$-variable. Therefore, $w^h$'s are equicontinuous and uniformly bounded. Therefore, by the Arzela-Ascoli theorem, there exists a subsequence $w^{h_i}(t,x)$ such that $w^{h_i}(t,x)\to w$ uniformly on any compact set for a limit function $w\in C^1_{1/2}(\R^+\times \R)$. Therefore, due to the way of construction, $w_1^{h_i},w_2^{h_i}$ and $w_3^{h_i}$ also converge to the same limit.

\begin{lemma}\label{compactness} There exists a subsequence $\{w^{h_i}\}$ and a function $w\in C^1_{1/2}\big([0,T)\times \R\big)$ such that $w_1^{h_i},w_2^{h_i},w_3^{h_i},w^{h_i}\rightarrow w$ uniformly on any compact set $K \subset [0,T)\times \R$.
\end{lemma}


\subsection{Convergence to the weak solution}

Now we show that the subsequential limit is a weak solution to a
Cauchy problem of a non-autonomous parabolic equation,
\begin{equation}\label{eqnw}
   b(x) w_t = d w_{xx},\qquad w(0,x)=w_0(x).
\end{equation}
We also show that there is an energy functional such that the energy level of the weak solution decreases as $t\to\infty$. This implies the uniqueness of the solution in the class of admissible ones and the convergence of $w^h$ as $h\to0$.

Let $w$ be a smooth solution of \eqref{eqnw} and define an energy
functional by
$$
e(t;w)={1\over 2}\int  b(x) w^2(t,x)\,dx.
$$
Then,
$$
e'(t;w)=\int  b ww_t\,dx=d\int ww_{xx}\,dx=-d\int w_x^2\,dx\le 0.
$$
We show that the energy of a subsequential limit decreases in time.
\begin{theorem}
Let $w$ be a subsequential limit, i.e., $w^{h_i}\to w$.
\begin{enumerate}
 \item For all $t\in[0,T)$, $\int_\R b(x) w(t,x)\;dx = 1$.
 \item For all $t>0$, $e(t;w)\le e(0;w)$.
\end{enumerate}
\end{theorem}
\begin{proof}
For $t\in [t_{2\ell}, t_{2\ell+2})$, we have
\begin{align*}
\int_\R  b(x)w_1^h(t,x) \; dx &= \sum_{j\ge0}2w^j_\ell h +
\sum_{j<0}w^j_\ell2h \\
&=\sum_{j\ge0}p^j_{2\ell} +
\sum_{j<0}p^{2j}_{2\ell}=\sum_{j\ge0}p^j_{0} +
\sum_{j<0}p^{2j}_{0} = 1.
\end{align*}
Therefore, since $w_1^h(t,\cdot) \rightarrow w(t,\cdot)$ in $L^1(\R)$,
$\int_\R b(x) w(t,x) \; dx = 1$.

Next consider the energy of the approximation $w^h_1$,
\begin{align*}
e(t_{2\ell};w^h_1)&=\int_\R  b(x) (w^h_1(t_{2\ell},x))^2dx\\
&=\sum_{j<0}(w_\ell^{j})^2(2h) +\sum_{j\ge0}2(w_\ell^{j})^2h =\sum_{j<0}(p_{2\ell}^{2j})^2 +\sum_{j\ge0}(p_{2\ell}^{j})^2 .
\end{align*}
Since $f(x)=x^2$ is a convex function, \eqref{RW2} gives that
$$
\begin{aligned}
    &(p^{k}_{2\ell})^2 \le \left\{\begin{array}{ll}
        \frac{1}{4}(p^{k-2} _{2(\ell-1)})^2 + \frac{1}{2}(p^{k}_{2(\ell-1)})^2 +\frac{1}{4} (p^{k+2}_{2(\ell-1)})^2, & \text{if $k<0$ is even},\\
        \frac{1}{4}(p^{-2} _{2(\ell-1)})^2 + \frac{1}{4}(p^{0} _{2(\ell-1)})^2 +\frac{1}{2}(p ^{1} _{2(\ell-1)})^2, & \text{if $k=0$},\\
        \frac{1}{2}(p^{k-1} _{2(\ell-1)})^2 + \frac{1}{2}(p ^{k+1} _{2(\ell-1)})^2,
        & \text{if } k>0.
        \end{array}\right.
\end{aligned}
$$
Therefore, $e(t_{2\ell};w^h_1)\le e(0;w^h_1)$ for all $\ell>0$. Since $w^{h_i}\to w$ as $h_i\to0$, we have $e(t;w)\le e(0;w)$ for all $t>0$.
\end{proof}


Finally, we define the weak solution of the limiting equation \eqref{eqnw} and show that the subsequential limit is a weak solution.

\begin{definition}[Weak solution] We call $w \in L^2\Big([0,T];H^1(
\R)\Big) \cap L^\infty\Big([0,T];L^2(\R)\Big)$ a weak solution of
\eqref{eqnw} if, for any test function $\phi \in
C_c^1([0,T)\times \R)$,
 \begin{equation}\label{DefWeak}
  \int_0^T\int_\R  b w\phi_t\,dxdt
  +\int_\R b w_0\phi(0,x)\,dx = d\int_0^T\int_\R w_x\phi_x\, dxdt.
 \end{equation}
\end{definition}

\begin{theorem}
  The interpolation $w^{h}(t,x)$ converges in $C^1_{1/2}([0,T]\times\R)$ as $h\to0$ and the limit is a weak solution of \eqref{eqnw}.
\end{theorem}
\begin{proof}
It is enough to show that the subsequential limit $w^{h_i}\to w$ is a weak solution of \eqref{eqnw}. Then, the uniqueness of energy decreasing solution gives the rest. The regularity of the limit $w$ is given in Lemma \ref{uniform_est}. Consider the two terms in the left side of \eqref{DefWeak} with $w_2^{h_i}$ in the place of $w$. Then,
\begin{align*}
 \int_0^T\int_\R& w_2^{h_i}(t,x)  b(x)\partial_t\phi(t,x)\; dx dt
 +\int_\R w_2^{h_i}(0,x) b(x)\phi(t,x) \; dx\hskip 5mm \Big(=:A_i\Big)\\
 &=-\sum_{\ell}\int_{t_{2\ell}}^{t_{2\ell+2}}\int_\R \partial_t
 w_2^{h_i}(t,x)  b(x)\phi(t,x)\; dx dt \\
 &=-\sum_{\ell}\sum_{j}\int_{t_{2\ell}}^{t_{2\ell+2}}\int_\R
 \frac{w^j_{2\ell+2}-w^j_{\ell}}{2\tau}  b \phi \chi_j(x)\; dx dt \hskip
 1.5cm (\text{with } \chi_j=\chi_{[y^j,y^{j+1})})\\
 &=-\sum_{\ell}\sum_{j}d\int_{t_{2\ell}}^{t_{2\ell+2}}\int_\R
 \tfrac{1}{2h} \left( \Big(\tfrac{w^{j+1}_{\ell} -
 w^{j}_{\ell}}{y^{j+1}-y^j}\Big) -\Big(\tfrac{w^{j}_{\ell} -
 w^{j-1}_{\ell}}{y^{j}-y^{j-1}} \Big)\right)  b \phi \chi_j(x)\; dx
 dt\quad\ (\text{by } \eqref{RW1w}\,)\\
 &=-\sum_{\ell}\sum_{j}d\int_{t_{2\ell}}^{t_{2\ell+2}}\int_\R
 \tfrac{1}{2h} \Big(\tfrac{w^{j}_{\ell} -
 w^{j-1}_{\ell}}{y^{j-1}-y^j}\Big)
 \big(\chi_{j-1}(x)-\chi_{j}(x)\big) b \phi \; dx \;dt\\
 &=-\sum_{\ell}\sum_{j}d\int_{t_{2\ell}}^{t_{2\ell+2}}B_\ell^j\,dt\hskip 5mm\Big(B_\ell^j:=\int_\R \tfrac{1}{2h} \Big(\tfrac{w^{j}_{\ell} -
w^{j-1}_{\ell}}{y^{j-1}-y^j}\Big)
\big(\chi_{j-1}(x)-\chi_{j}(x)\big) b \phi \; dx\Big).
\end{align*}
Since $w_2^{h_i} \rightarrow w$ uniformly on each compact set and
$w_2^{h_i}(0,\cdot) \rightarrow w_0(\cdot)$ in $L^1$, the left side converges as
\begin{align*}
\lim_{h_i\to0}A_i=\int_0^T\int_\R w(t,x)  b(x)\partial_t\phi(t,x)\; dx
dt+\int_\R w_0(x) b(x)\phi(t,x) \; dx.
\end{align*}
The $B^j_\ell$ in the right side becomes
\begin{align*}
 B_\ell^j &= \Big(\tfrac{w^{j}_{\ell} - w^{j-1}_{\ell}}{y^{j}-y^{j-1}}\Big)
        \left\{\frac{1}{2h}\int_{y^{j-1}}^{y^{j}}  b\phi  \; dx -
        \frac{1}{2h}\int_{y^{j}}^{y^{j+1}}  b\phi  \; dx\right\}\\
     &= \left\{\begin{array}{ll}
        \Big(\tfrac{w^{j}_{\ell} - w^{j-1}_{\ell}}{y^{j}-y^{j-1}}\Big)
        \bigg( \frac{1}{y^j-y^{j-1}}\int_{y^{j-1}}^{y^{j}} \phi  \; dx-
        \frac{1}{y^{j+1}-y^{j}}\int_{y^{j}}^{y^{j+1}} \phi  \; dx \bigg)
        & \text{if } j<0,\\
        \Big(\tfrac{w^{0}_{\ell} - w^{-1}_{\ell}}{y^{0}-y^{-1}}\Big)
        \bigg( \frac{1}{y^j-y^{j-1}}\int_{y^{j-1}}^{y^{j}} \phi  \; dx-
        \frac{1}{2(y^{j+1}-y^{j})}\int_{y^{j}}^{y^{j+1}} 2\phi  \; dx
        \bigg) & \text{if } j=0,\\
        \Big(\tfrac{w^{j}_{\ell} - w^{j-1}_{\ell}}{y^{j}-y^{j-1}}\Big)
        \bigg( \frac{1}{2(y^j-y^{j-1})}\int_{y^{j-1}}^{y^{j}} 2\phi  \; dx-
        \frac{1}{2(y^{j+1}-y^{j})}\int_{y^{j}}^{y^{j+1}} 2\phi  \; dx
        \bigg) & \text{if } j>0,
       \end{array}\right.\\
     &=\Big(\tfrac{w^{j}_{\ell} - w^{j-1}_{\ell}}{y^{j}-y^{j-1}}\Big)
        \bigg( \frac{1}{(y^j-y^{j-1})}\int_{y^{j-1}}^{y^{j}} \phi  \; dx-
        \frac{1}{(y^{j+1}-y^{j})}\int_{y^{j}}^{y^{j+1}} \phi  \; dx
        \bigg).
\end{align*}
Therefore,
\begin{align*}
 B_\ell^j =&-\int_{y^{j-1}}^{y^{j}}\Big(\tfrac{w^{j}_{\ell} -
 w^{j-1}_{\ell}}{y^{j}-y^{j-1}}\Big)\partial_x\phi \;dx \\
     & + \Big(\tfrac{w^{j}_{\ell} -
     w^{j-1}_{\ell}}{y^{j}-y^{j-1}}\Big)\left\{
        \frac{\int_{y^{j-1}}^{y^{j}} \phi -\phi(t,y^{j-1}) \;
        dx}{y^j-y^{j-1}} - \frac{\int_{y^{j}}^{y^{j+1}} \phi -\phi(t,y^{j})
        \; dx}{y^{j+1}-y^j} \right\}.
\end{align*}
Since the test function $\phi$ is continuous, the second term goes to zero as $h_i\to0$.
Therefore, after summing $B_\ell^j$, and taking $h_i\to0$, we obtain
$$
-\sum_j  B_\ell^j = \int_\R \partial_x w_3^{h_i}(t,x) \partial_x \phi(t,x) \;
dx + o(1) \rightarrow \int_\R \partial_x w(t,x) \partial_x \phi(t,x) \; dx.
$$
Therefore, combining the limits shows that $w$ is the weak solution.
\end{proof}


The probability density function is $v= b\, w$. Then, $v$ satisfies
$$
v_t=d_1\Big({v\over b(x)}\Big)_{xx},\qquad
v(0,x)={ b(x)w_0(x)}=v_0(x),
$$
where the initial value $v_0(x)$ is the original initial value given in
\eqref{initial}.


\newpage
\section{Random walk with a spatially non-constant walk length}

We consider a random walk system of the case,
\begin{equation}\label{case1}
\Delta x=ha(x),\quad \Delta t=\tau,
\end{equation}
where the spatial heterogeneity $a(x)$ is a given function and bounded by
$0<m\le a(x)<M$ for some constants $m$ and $M$. It is classical that the diffusion limit satisfies linear diffusion equation $u_t=d_1u_{xx}$ if $a(X)=1$. The spatial heterogeneity in $\Delta x$ is easily reduced to a homogeneous case after a change of variable. Therefore, the diffusion equation in the original variable is obtained by changing back the variable. We introduce the process in this section.

Introduce a new space variable $y$ given by
\begin{equation}\label{y}
y(x):=\int_0^x \frac{1}{a(s)}\,ds,\quad\text{and}\quad y^k=hk\quad k\in\Z.
\end{equation}
Then, since $a$ is positive and bounded away from zero, this relation is invertible and we may write the inverse function by $x=x(y)$. This inverse relation is given by
$$
x(y)=\int_0^y a(x(s))ds.
$$
Now we denote by $(t_n,x^k)$ the space-time mesh,
\begin{equation}\label{x^k}
t_n=n\tau,\quad x^k=x(y^k)=\int_0^{y^k}a(x(s))ds,\quad n\in\Z_+,\ k\in\Z.
\end{equation}
Then, $x^0=0$ and, if $a(x)$ is continuous,
$$
x^{k+1}-x^{k}=\int_{y^k}^{y^k+h}a(x(s))ds=ha(\bar x), \quad\bar x\in(x^{k},x^{k+1}).
$$
Therefore, the space mesh grids $\{x^k\}$ are of the type \eqref{case1}. Let $p_n^k$ be the probability for the random walk particle is placed at $x=x^k$ at time step $t=t_n$. Then, for all $n\in\Z_+$ and $k\in\Z$,
\begin{equation}\label{pnk}
2p_{n+1}^k = p_{n}^{k-1} + p_{n}^{k+1}.
\end{equation}
Suppose that the probability distribution for the initial position of the particle is given by a function in $\R$, $u_0(x)$. The probability $p^k_0$ is given by
\begin{equation}\label{p0k}
p_0^k:=\int_{x_{k}}^{x_{k+1}}u_0(x)dx.
\end{equation}






Note that the probability for a particle placed at $x^k$ is same as the one at $y^k$ since the random walk is independent of the distance between two grid points and only the order of the grid points matters. Furthermore, they are actually the same points after the change of the space variable. Hence we may consider the same probability $p^k_n$ as the one for the random walk on $y^k$'s. However, the probability density are different for the two cases. Let $u^k_n$ and $v^k_n$ be probability densities of the random walk system taking $x^k$ and $y^k$ as the grid points, respectively. Then, they are given by
$$
v^k_n:={p^k_n\over y^{k+1}-y^k}={p^k_n\over h},\quad
u^k_n:={p^k_n\over x^{k+1}-x^k}={p^k_n\over a(\bar x)h}.
$$
Therefore, the probability density with the two space variables satisfy the relation,
\begin{equation}\label{u-v}
  a(\bar x)u^k_n=v^k_n,\qquad \bar x\in (x^k,x^{k+1}).
\end{equation}

Next we introduce continuum interpolations,
$$
u^h(t,x):= u^k_n,\quad v^h(t,x)=v^k_n,\quad (t,x)\in D^k_n:=[t_n,t_{n+1})\times [x^k,x^{k+1}).
$$



\begin{theorem}\label{thm1} Consider the random walk system $p_n^k$ given in (\ref{x^k})--(\ref{p0k}) with ${h^2\over2\tau}=d_1$. Let $u_{n}^{k}$ be the probability density of the random particle defined by
$$
u_{n}^{k}={p_{n}^{k}\over x^{k+1}-x^{k}}
$$
and $\bar u$ be the solution of
\begin{equation}\label{DiffusionEqn2}
\bar u_t=\d\big(a(x) (a(x)\,\bar u)_x \big)_x,\quad\bar u(x,0)=\bar f(x).
\end{equation}
If the initial value $\bar f(x)$ is bounded,
\begin{equation}\label{Assumptions}
\sup_k | b(x^{k})-\bar b(x^{k}) | \to 0, \quad\text{and}\quad
\sup_k \Big| u^{k}_0 -\bar f(x^{k}) \Big| \to 0 \quad \text{as $\Delta t,\|\pi\|\to 0$,}
\end{equation}
then, for any $T>0$,
\[
\sup_{2n\Delta t\leq T,\,k\in\Z} \big| u^{2k}_{2n} - \bar u(x^{2k},t_{2n}) \big| \to 0 \quad \text{as $\Delta t, \|\pi\| \to 0.$}
\]
\end{theorem}

--

For a homogeneous random walk system, the distance between two neighboring grid points is constant. In this paper we consider a non-uniform case when $x^{k+1}-x^{k}$ is not necessarily constant. We introduce a partition consisting of odd numbered grid points and its norm:
\begin{equation}\label{pi}
\pi:=\{x^{2k+1}:k\in\Z\},\qquad\|\pi\|:=\sup_{k\in\Z}(x^{2k+1}-x^{2k-1}).
\end{equation}
Let $\beta$ be defined at even numbered grid points as
\begin{equation}\label{betax2k}
\beta(x^{2k}):=x^{2k+1}-x^{2k-1},
\end{equation}
which measures the distance between odd numbered grid points. We will see in the followings that even numbered grid points and odd numbered ones are separated. The theory of this section will be developed in terms of the even numbered grid points and that is the reason for defining $\beta$ at even numbered ones only. Similarly, we define
\begin{equation}\label{bx2k}
b(x^{2k}):=\frac{x^{2k+1}-x^{2k-1}}{\|\pi\|}=\frac{\beta(x^{2k})}{\|\pi\|},
\end{equation}
which measures heterogeneity in the walk length. We assume this ratio is bounded away from zero, i.e, there exists $c>0$ such that
\begin{equation}\label{c}
0<c\le b(x^{2k})\le 1,\qquad k\in\Z.
\end{equation}
We will keep this lower bound $c>0$ uniformly when we take the limit as $\|\pi\|\to0$. We may extend these functions over the real line by defining
\begin{equation}\label{continuation}
\beta(x):=\beta(x^{2k})\ \
 \text{and}\ \ a(x):=b(x^{2k})\quad\text{for}\ \ x^{2k-1}\leq x<x^{2k+1}.
\end{equation}
Notice that $\beta(x)$ is not the walk length, but the sum of two adjacent walk lengths. The reason for this is to use it in computing the probability density as in (\ref{ProbabilityDensity}). If we return to a uniform random walk case with equally distanced grid points, we have $c=1$, $a(x)\equiv 1$, and $\Delta x =\frac{1}{2}\|\pi\|$.

\subsection{Derivation of a PDE model}









Now we derive a diffusion equation that approximates the probability distribution function. First we introduce uniform grid points $y^k := \frac{k}{2}\|\pi\|$, and let $p(y,t)$ be a smooth function that satisfies $p(y^k,t_n)=p^k_n$. Then, by (\ref{p2n2k}),
$$
{p(y^{2k},t_{2n})-p(y^{2k},t_{2n-2})} ={1\over 4} \Big(p(y^{2k-2},t_{2n-2})-2p(y^{2k},t_{2n-2})+p(y^{2k+2},t_{2n-2})\Big).
$$
Divide this equation by $2\Delta t$ and obtain
\begin{equation}\label{FDR}
\begin{split}
{p(y^{2k},t_{2n})-p(y^{2k},t_{2n-2})\over 2\Delta t}
={\|\pi\|^2\over 8\Delta t} {p(y^{2k-2},t_{2n-2})-2p(y^{2k},t_{2n-2})+p(y^{2k+2},t_{2n-2}) \over \|\pi\|^2}.
\end{split}
\end{equation}
The left-hand side of (\ref{FDR}) is a first order forward approximation of $p_t:={\partial\over\partial t}p$ and the right side is a second order central approximation of $p_{yy}:={\partial^2\over\partial y^2}p$. Hence, we may write
\begin{equation}\label{EqnForPinY}
p_t(y,t)+O(\Delta t)=\d\,p_{yy}(y,t)+O(\d\|\pi\|^2),\quad \d:={\|\pi\|^2\over 8\Delta t}.
\end{equation}
To return to the original variable $x$ we change the space variable by using $y(x^{2k+1})=y^{2k+1}$. Then,
\begin{equation}\label{a(x)}
\frac{dx}{dy} \cong \frac{x^{2k+1}-x^{2k-1}}{y^{2k+1}-y^{2k-1}} =\frac{\beta(x^{2k})}{\|\pi\|}=b(x^{2k})
\end{equation}
and the equation (\ref{EqnForPinY}) turns into
\begin{equation*}\label{EqnForPinX}
p_t(x,t)=\d\,a(x) \big( a(x)p_x(x,t) \big)_x+O(\Delta t)+O(\d\|\pi\|^2).
\end{equation*}
Here we abuse some notations for simplicity by writing $p(x,t)$ instead of $p(y(x),t)$. We will use this convention throughout the paper. Let $u(x,t)$ be a smooth function given by
\begin{equation}\label{ProbabilityDensity}
u(x,t):={p(x,t)\over\beta(x)} ={p(x,t)\over \|\pi\|a(x)},
\end{equation}
which is called the \emph{probability density} function. Then, it satisfies
\[
u_t(x,t)=\d \big(a(x)(a(x)u)_x)_x+O(\Delta t/\|\pi\|)+O(\d\|\pi\|).
\]
Since $\d={\|\pi\|^2\over 8\Delta t}$, the remainder terms can be written as
\[
u_t(x,t)=\d\big(a(x)(a(x)u)_x)_x+O(\|\pi\|(\d+{1/\d})\,).
\]
Therefore, it is important to take
\begin{equation}\label{diffusivity}
\d={\|\pi\|^2\over 8\Delta t}=O(1)\quad\mbox{as}\quad\Delta t\to0,
\end{equation}
so that, for $\|\pi\|$ small, the probability density function $u(x,t)$ approximately satisfies a \emph{non-uniform} diffusion equation,
\begin{equation}\label{DiffusionEqn1}
u_t(x,t)=\d\big(a(x)(a(x)u)_x)_x.
\end{equation}

\begin{remark}
The diffusivity of the diffusion equation (\ref{DiffusionEqn1}) at $x\in\R$ is $d(x):=a(x)^2\d$. Since $c\le a(x)\le1$, the diffusivity is in the range between $c^2\d$ and $\d$. Hence we may call $\d$ the maximum diffusivity for a heterogeneous case and the diffusivity for a homogeneous case.
\end{remark}



\subsection{Convergence to the solution of the PDE model}

We have derived the diffusion equation (\ref{DiffusionEqn1}) as an approximation of the difference equation (\ref{FDR}) when the mesh grids $x^k$'s and the jumping time $\Delta t$ are fixed. In the following theorem we will show that the solution of the difference equation converges to the solution of the non-uniform diffusion equation as $\|\pi\|$ and $\Delta t$ vanish. In this approach, the limit should be taken to the probability density function since the probability at each grid point just vanishes as the mesh becomes finer. Furthermore, one cannot take the limit as $\|\pi\|,\Delta t\to0$ arbitrarily; the relation (\ref{diffusivity}) should be satisfied. This kind of convergence is classical with a uniform mesh case (see Lin and Segel \cite[Section 3.3]{MR982711}). Hence the contribution of the following theorem is its extension to a non-uniform random walk system.

In the following theorem we denote the quantities obtained from a limiting process by symbols with a bar such as $\bar u$ or $\bar w$. The quantities that depend on the choice of spatial and temporal mesh grids are denoted without it.
\begin{theorem}\label{thm1} Consider the random walk system $p_n^k$ given in (\ref{x^k})--(\ref{p0Odd}) with a relation $\|\pi\|^2=8\d\Delta t$ for a given constant $\d>0$. Let $u_{2n}^{2k}$ be the probability density of the random particle defined by
$$
u_{2n}^{2k}={p_{2n}^{2k}\over x^{2k+1}-x^{2k-1}}={p_{2n}^{2k}\over\beta(x^{2k})}
$$
and $\bar u$ be the solution of
\begin{equation}\label{DiffusionEqn2}
\bar u_t=\d\big(a(x) (a(x)\,\bar u)_x \big)_x,\quad\bar u(x,0)=\bar f(x).
\end{equation}
If the initial value $\bar f(x)$ is bounded,
\begin{equation}\label{Assumptions}
\sup_k | b(x^{2k})-\bar b(x^{2k}) | \to 0, \quad\text{and}\quad
\sup_k \Big| u^{2k}_0 -\bar f(x^{2k}) \Big| \to 0 \quad \text{as $\Delta t,\|\pi\|\to 0$,}
\end{equation}
then, for any $T>0$,
\[
\sup_{2n\Delta t\leq T,\,k\in\Z} \big| u^{2k}_{2n} - \bar u(x^{2k},t_{2n}) \big| \to 0 \quad \text{as $\Delta t, \|\pi\| \to 0.$}
\]
\end{theorem}

\begin{proof}
Introduce new grid points $y^{2k+1}:=\int_0^{x^{2k+1}} \frac{1}{a(s)}\,ds$ and $y^{2k}=\frac{1}{2}(y^{2k+1}+y^{2k-1})$. Then,
\[
y^{2k+1}-y^{2k-1} =\int_{x^{2k-1}}^{x^{2k+1}}\frac{\|\pi\|}{\beta(s)}\,ds =\int_{x^{2k-1}}^{x^{2k+1}}\frac{\|\pi\|}{x^{2k+1}-x^{2k-1}}\,ds  =\|\pi\|
\]
and hence $\{y^k:k\in\Z\}$ is a uniform mesh grid with a constant walk length $|\Delta y|=\frac{1}{2}\|\pi\|$. Let $p_n^k$ be the probability for a particle to be placed at spatial grid point $y^k$ and at time step $t_n$. Then, we have
\[
p^{2k}_{2n} ={1\over 4}\, \big( p^{2(k-1)}_{2(n-1)} +2p^{2k}_{2(n-1)} +p^{2(k+1)}_{2(n-1)} \big)
\]
as in (\ref{p2n2k}). Since the probability at odd numbered spatial grid points at even numbered time step is zero, we consider even numbered spatial grid points at even numbered time steps only. Then, the probability density $w^{2k}_{2n}$ for this uniform random walk is given by
\[
w^{2k}_{2n}:={p^{2k}_{2n}\over y^{2k+1}-y^{2k-1}} = \frac{p^{2k}_{2n}}{ \|\pi\| }.
\]
Therefore, the finite difference relation for $w^{2k}_{2n}$ is the same as the one for $p^{2k}_{2n}$, i.e.,
\[
w^{2k}_{2n} ={1\over 4}\, \big( w^{2(k-1)}_{2(n-1)} +2w^{2k}_{2(n-1)} +w^{2(k+1)}_{2(n-1)} \big).
\]
Now let $\bar w$ be the solution of
\begin{equation} \label{barw}
\bar w_t =\d\,\bar w_{yy},\qquad\bar w(y,0) =a(x)\bar f(x).
\end{equation}
Here the coordinate $x$ in the initial value is determined by
\begin{equation} \label{relation-y(x)}
y(x):=\int_0^x \frac{1}{a(s)}\,ds.
\end{equation}
Then, by Taylor's theorem and the relation $2\Delta y=\|\pi\|$, we have
\[\begin{split}
\bar{w}^{2k}_{2n} -\bar{w}^{2k}_{2(n-1)} +O(|\Delta t|^2)
=\frac{1}{4} \big( \bar{w}^{2(k-1)}_{2(n-1)} -2\bar{w}^{2k}_{2(n-1)} + \bar{w}^{2(k+1)}_{2(n-1)} \big) +O(\Delta t\|\pi\|^{2}),
\end{split}\]
where $\bar{w}^{2k}_{2n}:=\bar{w}(y^{2k},t_{2n})$. Therefore, the difference $e^{2k}_{2n}:=w^{2k}_{2n}-\bar{w}^{2k}_{2n}$ satisfies
\[
\big| e^{2k}_{2n} \big| = \frac{1}{4} \big| e^{2(k-1)}_{2(n-1)} +2e^{2k}_{2(n-1)} +e^{2(k+1)}_{2(n-1)} \big| +O(|\Delta t|^2)\leq \sup_k \big| e^{2k}_{2(n-1)} \big| + O(|\Delta t|^2).
\]
Repeatedly using this relation, we obtain
\begin{equation} \label{error-between-w's}
\big| w^{2k}_{2n}-\bar{w}^{2k}_{2n} \big| \leq \sup_k \big| w^{2k}_0-\bar{w}^{2k}_0 \big| + nO(|\Delta t|^2).
\end{equation}

Define $\bar u$ as the one satisfying $\bar{w}(y,t) =a(x) \,\bar u(x,t)$, where the coordinates $x$ and $y$ are related by \eqref{relation-y(x)}. Then,
$$
\bar w_t=a(x)\bar u_t,\quad \d\,\bar w_{yy}= \d\,a(x)(a(x)(a(x) \,\bar u)_x)_x,\quad\text{and}\quad \bar u(x,0)=\bar f(x) .
$$
Therefore, this $\bar u$ is the solution of (\ref{DiffusionEqn2}). On the other hand, the probability density $u$ for the non-uniform random walk is
\[
u^{2k}_{2n}:={p^{2k}_{2n}\over x^{2k+1}-x^{2k-1}} = {\|\pi\|w_{2n}^{2k}\over\beta(x^{2k})}={w_{2n}^{2k}\over b(x^{2k})}.
\]
If we set
\[\bar u^{2k}_{2n} :=\bar u(x^{2k},t_{2n}) = {\bar{w}^{2k}_{2n}\over\bar b(x^{2k})}, \]
we have
\[
\begin{split}
\big| u^{2k}_{2n}-\bar{u}^{2k}_{2n} \big|=\Big|{w_{2n}^{2k}\over b(x^{2k})}-{\bar{w}^{2k}_{2n}\over\bar b(x^{2k})}\Big| &\le \Big|{w_{2n}^{2k}\over b(x^{2k})}-{\bar{w}^{2k}_{2n}\over b(x^{2k})}
\Big|+\Big|{\bar w_{2n}^{2k}\over b(x^{2k})}-{\bar{w}^{2k}_{2n}\over\bar b(x^{2k})}  \Big|\ \\
&\le \frac{1}{c} \,\big| w^{2k}_{2n} -\bar{w}^{2k}_{2n} \big| + \frac{\bar{w}^{2k}_{2n}}{c^2} \,\Big| b(x^{2k})-\bar b(x^{2k})  \Big|.
\end{split}
\]
The last term vanishes uniformly in $d$ because of the assumption (\ref{Assumptions}). Let $2n\Delta t<T$. Then, due to \eqref{error-between-w's}, the other term in the last line is bounded by
\[\begin{split}
\frac{1}{c} \,\big| w^{2k}_{2n} -\bar{w}^{2k}_{2n} \big| &\le
\frac{1}{c} \, \sup_k \big| w^{2k}_0 -\bar{w}^{2k}_0 \big| +nO(|\Delta t|^2)= \frac{1}{c} \, \sup_k \big| b(x^{2k})u^{2k}_0 -\bar b(x^{2k})\bar{u}_0(x^{2k}) \big| +O(\Delta t)\\
&\le\frac{1}{c}\sup_k\Big(\big|b(x^{2k})\big|\big|u^{2k}_0-\bar{u}_0(x^{2k})\big| +\big|\bar{u}_0(x^{2k})\big|\,\big|b(x^{2k}) -\bar b(x^{2k})\big|\Big)+O(\Delta t).
\end{split}\]
This bound vanishes uniformly in $d$ as $\Delta t \to 0$ because of the assumption in (\ref{Assumptions}) and the initial condition $\bar u(x,0)=\bar f(x)$. Therefore the proof is complete.
\end{proof}


\subsection{Numerical comparison between the random walk and the PDE model}

In this section we compare the discrete random walk and the solution of the PDE model. First we consider the models using the normalized variable $y$. The PDE for the probability density function $w$ in $y$ variable is given by
\begin{equation}\label{EqnForWy}
w_t=\d \,w_{yy},\quad w(y,0)=w_0(y),\quad -\infty<y<\infty,
\end{equation}
where the initial value is nonnegative and $\d=\frac{|\Delta y|^2}{2\Delta t}$ for the constant walk length $\Delta y>0$ and the jumping time $\Delta t>0$. The solution is simply given by
\begin{equation}\label{ExplicitP}
w(y,t) = \int \phi(y-a;t\d)\, w_0(a)\,da,
\end{equation}
where the heat kernel
\begin{equation*}
\phi(y;t\d):=\frac{1}{\sqrt{4\pi t\d\,}}e^{-y^2/4t\d}
\end{equation*}
is the solution when the initial value is given by the Dirac delta distribution, $p_0=\delta$. If $t\d=1/2$, then $\phi(\cdot;t\d)$ is called the standard normal distribution.  Note that the formula (\ref{ExplicitP}) holds since the heat equation (\ref{EqnForWy}) is autonomous and hence Green's function is simply $G(y,a,t)=\phi(y-a,t)$. In Figure \ref{fig1} the probability density distribution of a random walk, $w^{2k}_{2n}=p^{2k}_{2n}/(2\Delta y)$, is plotted with the standard normal distribution. In this simulation we set $\Delta y=0.1$ and $\Delta t=0.1$. Then the diffusivity becomes $\d=0.05$ and hence $t\d=0.5$ if the final time is $t=10$. The particle was initially placed at the origin, i.e., $p^k_0=\delta_{k0}$, where $\delta_{ij}$ is the Kronecker delta.

\begin{figure}[ht]
\centering
\begin{minipage}[t]{0.48\textwidth}
\centering
\includegraphics[width=\textwidth]{fig1a}

(a) Gaussian and ramdom walk
\end{minipage}
\begin{minipage}[t]{0.48\textwidth}
 \centering
 \includegraphics[width=\textwidth]{fig1b}

(b) Error $=w^{2k}_{2n}-\phi(y^{2k},0.5)$
\end{minipage}
 \caption{{\bf Random walk with a normalized variable.} Probability distribution of discrete random walk is given in (a) with dots. The curve is the standard normal distribution. Even using a small number of grid points gives a good match of the normal distribution. The difference is given in (b). The parameters of this random walk are $\Delta y=0.1,\Delta t=0.1,t=10,$ and $t\d=0.5$.} \label{fig1}
\end{figure}

Next we consider the probability density distribution of a random walk system with a constant jumping time $\Delta t=0.1$ and non-uniform grid points:
\[
x^0=0\quad\text{and}\quad x^k-x^{k-1}=
\begin{cases}
0.1 & \text{if $k\leq 0$,}\\
0.05 & \text{if $k>0$.}
\end{cases}\]
The corresponding diffusion equation is
\begin{equation*}\label{EqnForU2}
u_t=\d \big(a(x)(a(x)u)_x \big)_x,\quad u(x,0)=u_0(x),\quad -\infty<x<\infty,
\end{equation*}
where
$$
\|\pi\|=0.2, \quad \d=\frac{\|\pi\|^2}{8\Delta t}=0.05, \quad b(x^{2k})=
\begin{cases}
1 & \text{if $k<0$,}\\
0.75 & \text{if $k=0$,}\\
0.5 & \text{if $x>0$.}
\end{cases}
$$
This equation is not autonomous so that there is no explicit formula such as (\ref{ExplicitP}). However, if we introduce a new variable,
\[y(x) := \int_0^{x}\frac{1}{a(s)}\,ds,\]
then $\{y^k=y(x^k):k\in\Z\}$ becomes a uniform grid and the probability density $w(y,t)$ in the new variable satisfies (\ref{EqnForWy}), where the corresponding initial value is $w_0(y(x))=a(x)u_0(x)$. Hence
\begin{equation*}\label{000}
u(x,t)={1\over a(x)} \frac{1}{\sqrt{4\pi t\d\,}} \int e^{-(y(x)-z)^2/4t\d} w_0(z)\,dz.
\end{equation*}

\begin{figure}[ht]
\centering
\begin{minipage}[t]{0.48\textwidth}
\centering
\includegraphics[width=\textwidth]{fig2a}

(a) $\Delta x=0.1, x<0$; $\Delta x=0.05, x>0$; $\Delta t=0.1$
\end{minipage}
\begin{minipage}[t]{0.48\textwidth}
 \centering
 \includegraphics[width=\textwidth]{fig2b}

(b) $\Delta x=0.05, x<0$; $\Delta x=0.025, x>0$; $\Delta t=0.1$
\end{minipage}
\caption{{\bf Random walk with a non-constant walk length.} Probability density distribution of the random walk, $u^{2k}_{2n}:=p^{2k}_{2n}/(x^{2k+1}-x^{2k-1})$, is given with the rescaled Gaussian. (a) The maximum diffusivity is $\d=0.05$ and the final time is given by $t\d=0.5$.  (b) The diffusivity is $\d=0.0125$ and the final time is given by $t\d=0.5$. } \label{fig2}
\end{figure}
The heat kernel for this non-autonomous problem is given by a rescaled Gaussian,
\begin{equation*}
\phi_b(x;t\d) := {1\over a(x)}\frac{1}{\sqrt{4\pi t\d\,}}e^{-y(x)^2/4t\d},\quad y(x) := \int_0^{x}\frac{1}{a(s)}\,ds.
\end{equation*}
In Figure \ref{fig2} this rescaled Gaussian and the probability distribution of some discrete random walk systems are compared. These two agree well even with relatively small number of grid points. Also one may observe a discontinuity at the origin which was generated by the discontinuity in the walk length $\Delta x$.

\begin{remark}[Invariance of the median]\label{Answer1}
The probability density function of a random walk system may have different shapes depending on heterogeneity in the walk length. However, the median of the distribution cannot be changed. For example, consider a random walk system that the particle started at the origin. Then, the probability density is given by a rescaled Gaussian and satisfies
\begin{equation}\label{100}
\begin{split}
\int_0^\infty u(x,t)\,dx&=\frac{1}{\sqrt{4\pi t\d\,}}\int_0^\infty {1\over a(x)}e^{-y(x)^2/4t\d}\,dx
=\frac{1}{\sqrt{4\pi t\d\,}}\int_0^\infty e^{-s^2/4t\d}\,ds=0.5.
\end{split}
\end{equation}
Hence the probability for a particle to be placed on the region where $x<0$ is identical to the one where $x>0$. This is the case when there is heterogeneity in $\Delta x$ only. However, we will see in the next section that is not the case if there is heterogeneity in $\Delta t$.
\end{remark}



\newpage
\section{Random walk with a spatially non-constant travelling time}

In this section we develop a random walk theory with a spatial heterogeneity in the time step $\Delta t$. Since heterogeneity in the walk length could be handled by rescaling the space variable as shown in the previous section, we will focus on a constant walk length case:
\begin{equation*}\label{y^k}
y^k=k|\Delta y|,\qquad k\in\Z.
\end{equation*}
Now we will work with heterogeneity in the time step $\Delta t$, which is the main contribution of this paper. We consider a simplest spatial heterogeneity in $\Delta t$ given by
\begin{equation*}\label{gamma}
\Delta t(y^{k+1/2}):=
\begin{cases}
\tau & \text{if $y^{k+1/2}<0$,}\\
2\tau & \text{if $y^{k+1/2}\geq 0$,}
\end{cases}
\end{equation*}
where $\tau>0$ is a constant and $y^{k+1/2}:=(y^k+y^{k+1})/2$. We will consider $\Delta t(y^{k+1/2})$ as the \emph{travelling time} which it takes for a particle to move from $y^k$ to $y^{k+1}$ or vice versa\footnote{If we consider a non-constant $\Delta t$ as the waiting time for the next jump, there may exist two different particles or probabilities with different jumping moments. This makes a presentation complicate. Hence we consider it as a travelling time and the next walk starts immediately after arrival. Hence, in the region where $y>0$, one should count the probability that the particle does not arrive at a grid point yet.}. As before, we may extend  the domain of the function $\Delta t$ over the real line by defining $\Delta t(y):=\Delta t(y^{k+1/2})$ for $y^k\leq y<y^{k+1}$. Note that the value of $\Delta t$ at a spatial grid point $y^k$ has no meaning in our setting. We will focus on this two-time step random walk in this section. This simple case is actually a challenging one due to the singularity at the origin and can be used as a building block for general cases with arbitrarily heterogeneous time steps.

\subsection{Derivation of a PDE model}

Define
\begin{equation}\label{alpha}
\mu(y):=\frac{\tau}{\Delta t},\quad\text{i.e.,}\quad
\mu(y)=
\begin{cases}
1 & \text{if $y<0$,}\\
1/2 & \text{if $y\geq 0$.}
\end{cases}
\end{equation}
We consider a particle randomly walking along the grid points. Let $p_0^k$ be the probability that the particle is placed at the grid point $y^k$ and at the initial time $t=0$. Similarly, let $p_n^k$ be the probability at time $t_n=n\tau$, $n>0$. Then, the probability $p^k_n$ for $n\ge2$ is computed by
\begin{equation}\label{interior2}
2p_n^k=
\begin{cases}
p_{n-1}^{k-1}+p_{n-1}^{k+1} & \text{if $k<0$,}\\
p_{n-1}^{-1}+p_{n-2}^{1} & \text{if $k=0$,}\\
p_{n-2}^{k-1}+p_{n-2}^{k+1} & \text{if $k>0$.}
\end{cases}
\end{equation}
Applying this relation twice obtain
\[
4p_n^k=
\begin{cases}
p_{n-2}^{k-2}+2p_{n-2}^{k}+p_{n-2}^{k+2} & \text{if $k<-1$,}\\
p_{n-2}^{-3}+2p_{n-2}^{-1}+\boxed{p_{n-3}^1} & \text{if $k=-1$,}\\
p_{n-2}^{-2}+p_{n-2}^{0}+p_{n-4}^{0}+p_{n-4}^{2} & \text{if $k=0$,}\\
\boxed{p_{n-3}^{-1}}+2p_{n-4}^{1}+p_{n-4}^3 & \text{if $k=1$,}\\
p_{n-4}^{k-2}+2p_{n-4}^{k}+p_{n-4}^{k+2} & \text{if $k>1$.}
\end{cases}
\]
Notice that even numbered grid points and odd numbered ones are not separated in this case because of the boxed terms in the second and the fourth cases. Now we repeat the derivation process as in the previous section. Let
$$
w_n^k={p_n^k\over \Delta y}.
$$
Then, the relation in (\ref{interior2}) gives
\begin{equation}\label{w}
2w_n^k=
\begin{cases}
w_{n-1}^{k-1}+w_{n-1}^{k+1} & \text{if $k<0$,}\\
w_{n-1}^{-1}+w_{n-2}^{1} & \text{if $k=0$,}\\
w_{n-2}^{k-1}+w_{n-2}^{k+1} & \text{if $k>0$.}
\end{cases}
\end{equation}
Fix
$$
\d={|\Delta y|^2\over 2\tau}
$$
as a constant $\d>0$ and consider the limit as $\tau\to0$. The argument on convergence in the previous section shows that $w_n^k$ converges to a limit $w(x,t)$ as $\tau\to0$ and satisfies
$$
w_t=\d w_{yy}\quad\text{for $y<0,$}\qquad
w_t={1\over2}\d w_{yy}\quad\text{for $y>0$.}
$$
Using the definition of $\mu(y)$ in \eqref{alpha} we may write it as
\begin{equation}\label{EqnForPinY2}
w_t=\mu(y)\d w_{yy} \qquad \text{for $y\ne0$.}
\end{equation}
This equation has meaning only if $y\ne0$ due to the discontinuity of $\mu$ at the origin $y=0$. Notice that the three cases with $k=-1,0,1$ are forgotten since $\Delta y$ is of microscopic scale, which is also the reason for the disconnected domain of the equation. In the following proposition we show that the limit $w$ is continuously diffrentiable even at the origin, i.e.,
\begin{equation*}\label{continuity}
\lim_{y\to0-}w(y,t_0)=\lim_{y\to0+}w(y,t_0),\qquad
\lim_{y\to0-}w_y(y,t_0)=\lim_{y\to0+}w_y(y,t_0).
\end{equation*}


\begin{proposition}\label{proposition} Let $w_\pm$ be the limit of $w_n^{\pm1}$ as $\tau \to0$ and $n\to\infty$, where the limit is taken under fixed $\d={|\Delta y|^2\over 2\tau}$ and fixed $t_0=n\tau$. Then, $w_-=w_+$. Furthermore, if
$$
\lim_{\tau \to0}{w_n^{1}-w_n^0\over\Delta y}=w_+',\qquad \lim_{\tau \to0}{w_n^{0}-w_n^{-1}\over\Delta y}=w_-',
$$
then $w_-'=w_+'$. Therefore, the limit function $w$ is in $C^1(\R)\cap C^\infty(\R\setminus\{0\})$.
\end{proposition}
\begin{proof} From the relation for $k=-1$ in (\ref{w}) we obtain
$$
4w_n^{-1}=2w_{n-1}^{-2}+2w_{n-1}^{0} =2w_{n-1}^{-2}+w_{n-1}^{-1}+w_{n-2}^{1}.
$$
Taking the limit  as $\tau \to0$ gives
$$
4w_-=2w_-+w_-+w_+.
$$
Therefore, $w_-=w_+$. Next subtract the relation for $k=0$ in (\ref{w}) from the one for $k=-1$ to obtain
\begin{equation}\label{difference-relation-w'}
{2(w^{-1}_n-w_n^0)\over\Delta y} ={w^{-2}_{n-1}-w_{n-1}^{-1}\over\Delta y} +{w^{0}_{n-1}-w_{n-1}^{1}\over\Delta y} +{w^{1}_{n-1}-w_{n-2}^{1}\over\Delta y}.
\end{equation}
The last term vanishes since
\begin{equation}\label{w'}
{w^{1}_{n-1}-w_{n-2}^{1}\over\Delta y}={w^{1}_{n-1}-w_{n-2}^{1}\over2\d\tau}\ {|\Delta y|^2\over\Delta y}\to 0
\end{equation}
as $\tau \to0$. Hence, taking the limit as $\tau\to0$ in \eqref{difference-relation-w'} gives
$$
2w_-'=w_-'+w_+',
$$
or $w_-'=w_+'$.
\end{proof}

\begin{remark}
Notice that the vanishing limit in (\ref{w'}) is invalid for second order derivative since ${\tau\over|\Delta y|^2}$ does not vanish. Therefore continuity in the proposition is valid only up to the first order derivative and $w_{yy}$ has a discontinuity at $y=0$.
\end{remark}

Next we consider the probability density function that includes the probability for a particle to be placed between grid points, but not at a grid point. Notice that $w_n^k$ is the probability density that a particle arrives at $y^k$ at time $t_n=n\tau$. However, the travelling time in the region where $y>0$ is $2\tau$ and hence there is a chance for a particle to be in the middle of its way. Including such a chance, the probability \emph{density} function is given by
\begin{equation*}\label{density}
v(y,t)=
\begin{cases}
w(y,t) & \text{if $y<0$,}\\
2w(y,t) & \text{if $y>0$,}
\end{cases}
\quad \text{or}\quad v(y,t)=\frac{1}{\mu(y)}w(y,t).
\end{equation*}
This relation can be justified only when the probability at grid points and the probability in the middle of the way are identical in the region where $y>0$. They are almost identical if $\Delta t$ and $\Delta x$ are small enough. Substitute $w=\mu(y)v$ into (\ref{EqnForPinY2}) and obtain
\begin{equation*}\label{EqnForUinY2}
v_t=\d (\mu v)_{yy},\qquad y\in\bfR,\ \d:={|\Delta y|^2\over 2\tau}.
\end{equation*}
Now we change the variable using (\ref{relation-y(x)}) as in the previous section. Then the density becomes $u(x,t)=v(y(x),t)/a(x)$ and satisfies
\begin{equation}\label{EqnForUinX2}
u_t=\d (b(b\mu u)_{x})_{x},\qquad x\in\bfR,\ \d:={\|\Delta x\|^2\over 2\tau}.
\end{equation}
Remember that $w(y,t)=b(x(y))\mu(y)u(x(y),t)$ is the relation between $w$ and $u$ and it is $w$ that has $C^1(\R)$ regularity.

\subsection{Green's function for the two-time step PDE model} \label{sect.explicit}

The purpose of this section is to find Green's function for the diffusion model (\ref{EqnForUinX2}) in an explicit form. For simplicity we consider the case when there is no heterogeneity in $\Delta x$, i.e., $a(x)\equiv 1$, and the diffusivity is $\d=1$. Going back to (\ref{EqnForPinY2}), we will actually find  $G=G(y,a,t)$ that solves
\[\left\{
\begin{aligned}
&G_t= \mu(y) \,G_{yy},\\
&G(y,0)=\mu(y) \,\delta(y-a).
\end{aligned}
\right.\]
In the followings we will look for Green's function such that $G(\cdot,a,t)\in C^1(\R)\cap C^\infty(\R\setminus\{0\})$ for each $a\in\bfR$ and $t>0$ since a meaningful solution satisfies this regularity as shown in Proposition \ref{proposition}.


\subsubsection{When $a=0$.}
In this section, we will find $w(y,t):=G(y,0,t)$ explicitly that solves
\begin{equation}\label{w0}
\left\{
\begin{aligned}
&w_t= \mu(y) \,w_{yy},\\
&w(y,0)=\mu(y) \,\delta(y).
\end{aligned}
\right.
\end{equation}
It is tricky to find a solution due to the discontinuity of the coefficient $\mu$ at the origin. We employ a similar technique used in \cite{Chung20142520}. First, we split the real line $\textbf{R}$ into two regions, $\{ y>0 \}$ and $\{ y<0 \}$. In the region $\{y>0\}$, we solve an initial-boundary value problem:
\[
\begin{cases}
w_t = \frac{1}{2} w_{yy} & \text{if $y > 0, ~t > 0$,} \\
w(y,0) = \frac{1}{2} f(y) & \text{if $y > 0$,} \\
w(0,t) = g(t) & \text{if $t > 0$,}
\end{cases}
\]
where the value $g(t)$ at the boundary $y=0$ is unknown and will be determined later. Notice that we have replaced the initial data $\delta$ by an arbitrary function $f(y)$ for now. This will be replaced by a Dirac delta sequence and the unknown boundary value $g$ will be decided from a limiting process. One may find the solution to the initial-boundary value problem in \cite[p18 and p22]{MR1728947}, which is
\[ \begin{split}
w(y,t) &= \frac{1}{\sqrt{2 \pi t}} \int_0^\infty \frac{1}{2} f(\xi) \big( e^{-(\xi-y)^2/2t} - e^{-(\xi+y)^2/2t} \big)\,d\xi \\
& \qquad + \int_0^t g'(\eta) \,\erfc\Big( \frac{y}{\sqrt{2(t-\eta)}} \Big) \,d\eta + g(0) \,\erfc\Big( \frac{y}{\sqrt{2t}} \Big) \quad \text{if $y > 0$.}
\end{split} \]
For $y < 0$, $w$ satisfies
\[
\begin{cases}
w_t = w_{yy} & \text{if $y<0, ~t > 0$,} \\
w(y,0) = f(y) & \text{if $y < 0$,} \\
w(0,t) = g(t) & \text{if $t > 0$.}
\end{cases}
\]
Notice that, by choosing the same boundary value $g(t)$, the solution $w$ is automatically continuous. One can similarly find the solution
\[ \begin{split}
w(y,t) &= \frac{-1}{2\sqrt{\pi t}} \int_0^\infty f(-\xi) \big( e^{-(\xi-y)^2/4t} - e^{-(\xi+y)^2/4t} \big) \,d\xi \\
& \qquad + \int_0^t g'(\tau) \,\erfc\Big( \frac{-y}{2\sqrt{t-\tau}} \Big) + g(0) \,\erfc\Big( \frac{-y}{2\sqrt{t}} \Big) \quad \text{if $y < 0$.}
\end{split} \]

To determine $g(t)$, we use the assumption that the solution $w$ is $C^1(\bfR)$ in the $y$ variable, i.e.,
\[ w_y(0+,t) = w_y(0-,t) \qquad \text{for all $t > 0$.} \]
Since
\[w_y(y,t)=
\begin{cases}
\begin{split}
& \textstyle \frac{1}{2t \sqrt{2\pi t}} \int_0^\infty f(\xi) \big( (\xi-y) e^{-(\xi-y)^2/2t} + (\xi+y) e^{-(\xi+y)^2/2t} \big) \,d\xi \\
& \textstyle \qquad - \sqrt{\frac{2}{\pi}} \int_0^t g'(\tau) \, \frac{e^{-y^2/2(t-\tau)}}{\sqrt{t-\tau}} \,d\tau - \sqrt{\frac{2}{\pi t}} \,g(0) \,e^{-y^2/2t}
\end{split}
& \text{if $y>0$,} \\
\begin{split}
& \textstyle \frac{-1}{4t \sqrt{\pi t}} \int_0^\infty f(-\xi) \big( (\xi-y) e^{-(\xi-y)^2/4t} + (\xi+y) e^{-(\xi+y)^2/4t} \big) \,d\xi \\
& \textstyle \qquad + \frac{1}{\sqrt{\pi}} \int_0^t g'(\tau) \,\frac{e^{-y^2/4(t-\tau)}}{\sqrt{t-\tau}} \,d\tau + \frac{1}{\sqrt{\pi t}} \,g(0) \,e^{-y^2/4t}
\end{split}
& \text{if $y<0$,}
\end{cases}\]
the continuous differentiability of $w$ implies that
\[ \begin{split}
& \frac{1}{t \sqrt{2t}} \int_0^\infty f(\xi) \,\xi e^{-\xi^2/2t}\,d\xi + \frac{1}{2t \sqrt{t}} \int_0^\infty f(-\xi) \, \xi e^{-\xi^2/4t} \,d\xi \\
& \qquad\qquad = (\sqrt{2}+1) \int_0^t g'(\tau) \, \frac{1}{\sqrt{t-\tau}} \,d\tau + (\sqrt{2} + 1) \,\frac{g(0)}{\sqrt{t}} \quad \text{for all $t > 0$.}
\end{split} \]
This relation gives the boundary value $g(t)$ implicitly for any given initial value $f(y)$. For Green's function case, the corresponding initial value is the Dirac delta distribution and we may find $g$ explicitly. First, take a Dirac delta sequence in the place of the initial value: for example, $f(y) = \frac{n}{2} \,\chi_{[-1/n,1/n]}(y)$. Then
\[ \begin{split}
&\frac{n}{2\sqrt{2t}} (1 - e^{-\frac{1}{2n^2 t}}) + \frac{n}{2\sqrt{t}} (1 - e^{-\frac{1}{4n^2 t}}) \\
& \qquad\qquad = (\sqrt{2} + 1) \int_0^t g'(\tau) \,\frac{1}{\sqrt{t-\tau}} \,d\tau + (\sqrt{2} + 1) \, \frac{g(0)}{\sqrt{t}}.
\end{split} \]
Taking the Laplace transformation yields
\[ \begin{split}
& \frac{n}{2\sqrt{2}} \sqrt{\frac{\pi}{s}} (1 - e^{-\sqrt{2s}/n}) + \frac{n}{2} \sqrt{\frac{\pi}{s}} (1 - e^{-\sqrt{s}/n} ) \\
& \qquad = (\sqrt{2} + 1) (s \mathcal{G} (s) - g(0)) \sqrt{\frac{\pi}{s}} + (\sqrt{2} + 1) \,g(0) \sqrt{\frac{\pi}{s}}\ ,
\end{split} \]
or
\[ \mathcal{G} (s) = \frac{n}{2(\sqrt{2}+1) s} \Big( \frac{1}{\sqrt{2}} (1 - e^{-\sqrt{2s}/n}) + (1 - e^{-\sqrt{s}/n}) \Big),\]
and the inverse Laplace transform gives
\[ g(t) = \frac{n}{2} (\sqrt{2} - 1) \Big( \frac{1}{\sqrt{2}} \erf\big( \frac{\sqrt{2}}{2n\sqrt{t}} \big) + \erf \big( \frac{1}{2n \sqrt{t}} \big) \Big). \]
Therefore, the boundary condition $g(t)$ for Green's function is obtained by taking the limit as $n \to \infty$:
\[ g(t) = \frac{\sqrt{2} - 1}{\sqrt{\pi t}}. \]
Therefore the solution $w$ of (\ref{w0}) is given by
\[w(y,t)=\begin{cases}
\frac{\sqrt{2}-1}{\sqrt{2} \pi} \int_0^t \frac{y e^{-y^2/2(t-\tau)}}{\sqrt{\tau} (t-\tau)^{3/2}} \,d\tau = \frac{2 (\sqrt{2}-1)}{\pi} \int_{y/\sqrt{2t}}^\infty \frac{e^{-\eta^2}}{\sqrt{t - y^2/2\eta^2}} \,d\eta & \text{if $y > 0$,} \\
\frac{\sqrt{2}-1}{2\pi} \int_0^t \frac{(-y) e^{-y^2/4(t-\tau)}}{\sqrt{\tau} (t-\tau)^{3/2}} \,d\tau = \frac{2 (\sqrt{2}-1)}{\pi} \int_{-y/2\sqrt{t}}^\infty \frac{e^{-\eta^2}}{\sqrt{t-y^2/4\eta^2}}\,d\eta & \text{if $y < 0$.}
\end{cases}\]
Using the definition of $\mu(y)$ in \eqref{alpha}, Green's function $G(y,a=0,t)=w(y,t)$ can be written in a combined form
\begin{equation}\label{G0}
G(y,a=0,t)=\frac{2(\sqrt{2}-1)}{\pi} \int_{|y|/\sqrt{4\mu(y)t}}^\infty \frac{e^{-\eta^2}}{\sqrt{t-y^2/4\mu(y)\eta^2}}\,d\eta,\qquad y\in\R.
\end{equation}



\subsubsection{When $a\ne0$}


Now we move the initial Dirac delta distribution from the origin to some other point. Then the corresponding equation becomes
\[ \left\{ \begin{aligned}
&w_t = \mu(y) \,w_{yy}, \\
&w(y,0) = \mu(y) \delta(y-a),
\end{aligned} \right. \]
where $a\ne0$. One may solve this problem in the same manner by dividing the spatial domain into two parts,  $\{y>0\}$ and $\{y<0\}$. For the case when $a > 0$, the corresponding boundary value is
\[
g(t)= \frac{\sqrt{2} - 1}{\sqrt{\pi t}} \,e^{-a^2/2t}
\]
and Green's function is
\begin{equation}\label{G+}
G(y,a>0,t):=
\begin{cases}
\begin{split}
& \textstyle \frac{2(\sqrt{2}-1)}{\pi} \int_{|y|/\sqrt{2 t}}^\infty \exp\Big(- \frac{a^2/2}{t - y^2/2\eta^2}\Big)  \, \frac{e^{-\eta^2}}{\sqrt{t - y^2/2\eta^2}} \,d\eta\\
& \textstyle \qquad + \frac{1}{2\sqrt{2\pi t}} \big( e^{-(y-a)^2/2t} - e^{-(y+a)^2/2t} \big)
\end{split}
& \text{if $y>0$,} \\
\frac{2(\sqrt{2}-1)}{\pi} \int_{|y|/2\sqrt{t}}^\infty \exp\Big( -\frac{a^2/2}{t-y^2/4\eta^2} \Big) \, \frac{e^{-\eta^2}}{\sqrt{t - y^2/4\eta^2}} \,d\eta & \text{if $y < 0$.}
\end{cases}
\end{equation}
For the case when $a < 0$, we have
\[ g(t) = \frac{\sqrt{2} - 1}{\sqrt{\pi t}} \,e^{-a^2/4t} \]
and
\begin{equation}\label{G-}
G(y,a<0,t)=\begin{cases}
\frac{2(\sqrt{2}-1)}{\pi} \int_{|y|/\sqrt{2t}}^\infty \exp\Big( -\frac{a^2/4}{t-y^2/2\eta^2} \Big) \, \frac{e^{-\eta^2}}{\sqrt{t - y^2/2\eta^2}} \,d\eta & \text{if $y > 0$,} \\
\begin{split}
& \textstyle \frac{2(\sqrt{2}-1)}{\pi} \int_{|y|/2\sqrt{t}}^\infty \exp\Big( - \frac{a^2/4}{t - y^2/4\eta^2} \Big) \, \frac{e^{-\eta^2}}{\sqrt{t - y^2/4\eta^2}} \,d\eta \\
& \textstyle \quad + \frac{1}{2\sqrt{\pi t}} \big( e^{-(y-a)^2/4t} - e^{-(y+a)^2/4t} \big)
\end{split}
& \text{if $y < 0$.}
\end{cases}
\end{equation}
Note that the formulas for $a > 0$ and $a < 0$ are almost identical; especially, the integrands for the case when $a < 0$ is achieved just by simply replacing `$a^2/2$' in the exponent for the case when $a > 0$ by `$a^2/4$'.



\subsection{Numerical comparison between the random walk and the PDE model}

We compare the probability density function of the two-time step discrete random walk system and the explicit Green's function in (\ref{G0}). In Figure \ref{fig3} Green's function and the probability distribution of the discrete random walk system are compared. One may observe a discontinuity at the origin which comes from the discontinuity in the travelling time $\Delta t$. These two agree well even with a small number of grid points. The both cases with $\Delta x=0.1$ or $\Delta x=0.05$ are given in the figures.


\begin{figure}[ht]
\centering
\begin{minipage}[t]{0.49\textwidth}
 \centering
 \includegraphics[width=\textwidth]{fig3a}

(a) $\Delta t=0.1, x<0$, $\Delta t=0.2, x>0$, $\Delta x=0.1$
\end{minipage}
\begin{minipage}[t]{0.49\textwidth}
\centering
 \includegraphics[width=\textwidth]{fig3b}
(b) $\Delta t=0.1, x<0$, $\Delta t=0.2, x>0$, $\Delta x=0.05$
\end{minipage}
\caption{{\bf Random walk with a non-constant time step.} Probability distribution of the random walk is given with the explicit solution $u=w/\mu$, where $w$ is given by (\ref{ExplicitP}). The time step is $\Delta t=0.1$ for $x<0$ and  $\Delta t=0.2$ for $x>0$. The diffusivity is $\d=0.05$ on $x<0$ and the final time is given by a relation $t\d=0.5$.} \label{fig3}
\end{figure}
Notice difference between Figure \ref{fig2} and Figure \ref{fig3}. The probability density distribution in Figure \ref{fig2} are basically obtained by gluing two normal distributions of different variations. Hence the probability of one side, $P\{x<0\}$, is the same as the other side $P\{x>0\}$. However, the profiles in Figure \ref{fig3} do not follow the Gaussian distribution and even the probability for $x<0$ is different from the one of the other side, i.e., $P\{x<0\}\ne P\{x>0\}$, which will be shown in the following section.


\section{Conclusion and discussion}

Under presence of spatial heterogeneity in the walk length $\Delta x$ and in the jumping time $\Delta t$, random walk systems and the corresponding PDE models have been studied. Thermal diffusion of Brownian particles or starvation driven dispersal of biological organisms are possible applications. In this paper we have considered essential differences among uniform and non-uniform random walk systems. We also observed that the spatial heterogeneity in the jumping time $\Delta t$ has a profound difference from the one in $\Delta x$ and they are unexchangeable.

\subsection{Numerical comparison with Fick's law}

The diffusivity is given by the relation $k=\frac{|\Delta x|^2}{2|\Delta t|}$ in one space dimension. For the case with a constant jumping time interval $\Delta t$, the diffusivity can be written as $k=\d\,a(x)^2$ using the notation in the diffusion model (\ref{DiffusionEqn1}). The original Fick's law \cite{Fick1855} is for the constant diffusivity case. For a non-constant diffusivity $k=d(x)$, Fick's diffusion law usually refers to
\begin{equation}\label{FicksLaw}
u_t=(d(x) u_x)_x
\end{equation}
which is different from the diffusion model (\ref{DiffusionEqn1}) we have derived in this paper. We will numerically compare the solutions of these diffusion equations to the probability density distribution of the discrete random walk system. For numerical simulations we take an initial value
\[
u_0(x)=
\begin{cases}
1 & \text{if $-2<x<2$,}\\
0 & \text{otherwise.}
\end{cases}
\]
The time step is fixed with $\Delta t=0.1$. We consider two sets of random walk grids:
\begin{equation}\label{betaEx1}
x^k=
\begin{cases}
~0.1k & \text{if $k\le0$,}\\
0.05k & \text{if $k>0$,}
\end{cases}
\end{equation}
and
\begin{equation}\label{betaEx2}
x^0=0,\qquad
x^k=
\begin{cases}
x^{k-1}+0.08\times(1.5+\sin(x^{k-1})) & \text{if $k>0$,}\\
x^{k+1}-0.08\times(1.5+\sin(x^{k+1})) & \text{if $k<0$.}
\end{cases}
\end{equation}
There is a discontinuity in the variation of the walk length in the case of (\ref{betaEx1}). The other case of (\ref{betaEx2}) has gradually changing walking length.

For the above random walk systems one may easily compute the diffusivity $d(x)$ and the coefficient $a(x)$ in (\ref{a(x)}). In Figure \ref{fig4} solutions of these two diffusion models are given with the probability density distribution of the discrete random walk system. From these examples one may find that the solutions of the new diffusion law (\ref{DiffusionEqn1}) gives the probability density distribution of the discrete random walk system correctly. However, Fick's law gives something else. In conclusion the new diffusion law (\ref{DiffusionEqn1}) is the one that explains the non-uniform random walk system.

\begin{figure}[ht]
\begin{minipage}[t]{0.49\textwidth}
 \centering
 \includegraphics[width=\textwidth]{fig4a}
(a) Discontinuous walk length in (\ref{betaEx1})
\end{minipage}
\begin{minipage}[t]{0.49\textwidth}
\centering
 \includegraphics[width=\textwidth]{fig4b}
(b) Continuous walk length in (\ref{betaEx2})
\end{minipage}
\caption{{\bf Comparison with Fick's law.} Final time is $t=40$. $\Delta t=0.1$. It is the solution of the new diffusion law (\ref{DiffusionEqn1}) that shows the correct probability density distribution of discrete random systems, but not Fick's law (\ref{FicksLaw}).} \label{fig4}
\end{figure}



\subsection{Sensibility of $\Delta t$ versus insensibility of $\Delta x$}\label{sect.Answer2}
Let $w(y,t)=G(y,0,t)$ be Green's function with a Dirac delta distribution placed at $a=0$, which is given by the formula (\ref{G0}),
\begin{equation*}
w(y,t)=\frac{2(\sqrt{2}-1)}{\pi} \int_{|y|/\sqrt{4\mu(y)t}}^\infty \frac{e^{-\eta^2}}{\sqrt{t-y^2/4\mu(y)\eta^2}}\,d\eta,\qquad y\in\R.
\end{equation*}
By Fubini's theorem, we may observe that
\begin{align*}
\int_0^\infty w(y,t)\,dy &= \frac{\sqrt{2}-1}{\sqrt{2} \pi} \int_0^t \frac{1}{\sqrt{t-\tau} \,\tau^{3/2}} \int_0^\infty y e^{-y^2/2\tau} \,dy \,d\tau \\
&= \frac{\sqrt{2}-1}{\sqrt{2} \pi} \int_0^t \frac{1}{\sqrt{t-\tau} \,\sqrt{\tau}} \,d\tau = \frac{\sqrt{2} - 1}{\sqrt{2}}.
\end{align*}
We similarly compute
$$
\int_{-\infty}^0 w(y,t)\,dx = \sqrt{2} - 1.
$$
The probability density function in $y$ variable is given by $v(y,t)=w(y,t)/\mu(y)$ and satisfies
\begin{equation}\label{101}
\sqrt{2}\int_{-\infty}^0 v(y,t)\,dy =\int_0^\infty v(y,t)\,dy.
\end{equation}
The relation (\ref{101}) implies that the probability for a particle to be placed on $x>0$ is $\sqrt{2}$ times greater than the one on $x<0$. Remember that, for the case with spatial heterogeneity in walk length only, the two probabilities are identical as shown in (\ref{100}). One may clearly observe this phenomenon by comparing Figure \ref{fig2} and Figure \ref{fig3}.

In conclusion we may say that people in the region where $x<0$ may sense a change in jumping time $\Delta t$ of the other side where $x>0$. However, they cannot sense a change in walk length $\Delta x$. This observation implies that heterogeneity in time step cannot be reduced to one in walk length through any kinds of rescaling.


%
%
%\providecommand{\bysame}{\leavevmode\hbox to3em{\hrulefill}\thinspace}
%\providecommand{\MR}{\relax\ifhmode\unskip\space\fi MR }
%% \MRhref is called by the amsart/book/proc definition of \MR.
%\providecommand{\MRhref}[2]{%
%  \href{http://www.ams.org/mathscinet-getitem?mr=#1}{#2}
%}
%\providecommand{\href}[2]{#2}
%\begin{thebibliography}{10}
%
%\bibitem{Braun2004}
%D. Braun and A. Libchaber, \emph{Thermal force approach to molecular
%  evolution}, Phys. Biol. \textbf{1}, 1--8 (2004)
%
%\bibitem{Chapman1928}
%S. Chapman, \emph{On the {Brownian} displacements and thermal diffusion of
%  grains suspended in a non-uniform fluid}, Proc. Roy. Soc. Lond. A
%  \textbf{119}, 34--54 (1928)
%
%\bibitem{MR3050058}
%E. Cho and Y.-J. Kim, \emph{Starvation driven diffusion as a survival
%  strategy of biological organisms}, Bull. Math. Biol. \textbf{75},
%  845--870 (2013)
%
%\bibitem{ChoiKim}
%S. Choi and Y.-J. Kim, \emph{Chemotactic traveling waves by the metric
%  of food}, preprint (2014)
%
%\bibitem{Chung20142520}
%J. Chung, Y.-J. Kim, and M. Slemrod, \emph{An explicit solution
%  of Burgers equation with stationary point source}, J. Differ.
%  Equations \textbf{257}, 2520--2542 (2014)
%
%\bibitem{Duhr2006}
%S. Duhr and D. Braun, \emph{Why molecules move along a temperature
%  gradient}, Proc. Natl. Acad. Sci. USA \textbf{103}, 19678--19682 (2006)
%
%\bibitem{Einstein1905}
%A. Einstein, \emph{\"{U}ber die von der molekularkinetischen theorie
%  derw\"{a}rme geforderte bewegung von in ruhenden fl\"{u}ssigkeiten
%  suspendierten teilchen (on the movement of small particles suspended in a
%  stationary liquid demanded by the kinetic molecular theory of heat.)}, Ann.
%  Phys. \textbf{17}, 549--560 (1905)
%
%\bibitem{Einstein1906}
%A. Einstein, \emph{\emph{Zur theorie der brownschen bewegung} (on the theory of
%  brownian movement)}, Ann. Phys. \textbf{19}, 371--381 (1906)
%
%\bibitem{Eslamian2009}
%M. Eslamian and M. Z. Saghir, \emph{A critical review of thermodiffusion
%  models: Role and significance of the heat of transport and the activation
%  energy of viscous flow}, J. Non-Equilib. Thermodyn. \textbf{34}, 97--131 (2009)
%
%\bibitem{Fick1855}
%A. Fick, \emph{\"{U}ber diffusion ({O}n diffusion)}, Poggendorff's. Annalen.
%  \textbf{94}, 59--86 (1855)
%
%\bibitem{Harstad2009}
%K. Harstad, \emph{Modeling the Soret effect in dense media mixtures},
%  Ind. Eng. Chem. Res. \textbf{48}, 6907--6915 (2009)
%
%\bibitem{Huang2010}
%F. Huang, P. Chakraborty, C. C. Lundstrom, C. Holmden, J. J. G. Glessner, S. W.
%  Kieffer, and C. E. Lesher, \emph{Isotope fractionation in silicate melts by
%  thermal diffusion}, Nature \textbf{464}, 396--400 (2010)
%
%\bibitem{MR1728947}
%J. Kevorkian, Partial differential equations (2nd ed.), Texts in
%  Applied Mathematics, vol.~35, Springer-Verlag, New York (2000)
%
%\bibitem{MR3128024}
%Y.-J. Kim, O. Kwon, and F. Li, \emph{Evolution of dispersal toward
%  fitness}, Bull. Math. Biol. \textbf{75}, 2474--2498 (2013)
%
%\bibitem{MR3189110}
%Y.-J. Kim, O. Kwon, and F. Li, \emph{Global asymptotic stability and the ideal free distribution in a starvation driven diffusion}, J. Math. Biol. \textbf{68},  1341--1370 (2014)
%
%\bibitem{MR982711}
%C. C. Lin and L. A. Segel, Mathematics applied to deterministic problems
%  in the natural sciences (2nd ed.), Classics in Applied Mathematics,
%  vol.~1, Society for Industrial and Applied Mathematics (SIAM), Philadelphia,
%  PA (1988)
%
%\bibitem{Ludwig1856}
%C. Ludwig, \emph{Diffusion zwischen ungleich erw\"{a}rmten orten gleich zusammengesetzter l\"{o}sungen} (Diffusion of homogeneous fluids between regions of different temperature), Sitz. Ber. Akad. Wiss. Wien Math-Naturw. Kl. \textbf{20}, 539 (1856)
%
%\bibitem{14}
%J. C. Maxwell, \emph{On stresses in rarefied gases arising from inequalities of
%  temperature}, Phil. Trans. R. Soc. Lond. \textbf{170}, 231--256 (1879)
%
%\bibitem{MR1895041}
%A. Okubo and S. A. Levin, Diffusion and ecological problems: modern
%  perspectives (2nd ed.), Interdisciplinary Applied Mathematics, vol.~14,
%  Springer-Verlag, New York (2001)
%
%\bibitem{Putnam2007}
%S. A. Putnam, D. G. Cahill, and G. C. L. Wong, \emph{Temperature dependence of
%  thermodiffusion in aqueous suspensions of charged nanoparticles}, Langmuir
%  \textbf{23}, 9221--9228 (2007)
%
%\bibitem{Skellam72}
%J. G. Skellam, Some phylosophical aspects of mathematical modelling in
%  empirical science with special reference to ecology, Mathematical Models in
%  Ecology, Blackwell Sci. Publ., London (1972)
%
%\bibitem{Skellam73}
%J. G. Skellam, The formulation and interpretation of mathematical models of
%  diffusionary processes in population biology, The mathematical theory of the
%  dynamics of biological populations, Academic Press, New York (1973)
%
%\bibitem{Soret1879}
%Ch. Soret, \emph{Sur l'\'{e}tat d'\'{e}quilibre que prend, au point de vue de sa
%concentration, une dissolution saline primitivement homog\`{e}ne, dont deux parties sont port\'{e}es \`{a} des temp\'{e}ratures diff\'{e}rentes}, Archives de Gen\`{e}ve \textbf{3}, 48--61 (1879)
%
%\bibitem{Srinivasan2011}
%S. Srinivasan and M. Z. Saghir, \emph{Experimental approaches to study
%  thermodiffusion - a review}, Int. J. Therm. Sci. \textbf{50}, 1125--1137 (2011)
%
%\bibitem{YoonKim}
%C. Yoon and Y.-J. Kim, \emph{Bacterial chemotaxis without gradient-sensing}, J. Math. Biol. (2014). doi:10.1007/s00285-014-0790-y
%\end{thebibliography}
%
%\end{document}



%%%%%%%%%%%%%%%%%%%%%%%%%%%%%%%%%%%%%%%%
%\bibliographystyle{unsrt}
\bibliographystyle{amsplain}
\bibliography{ChungKim}


\end{document}
--

\begin{thebibliography}{99}



\bibitem{DuhrBraun2006}
S. Duhr and D. Braun,
\newblock \emph{Why molecules move along a temperature gradient},
\newblock Proc. Natl. Acad. Sci. USA \textbf{103} (2006), pp. 19678--82.



\bibitem{Enskog1912}
D. Enskog,
\newblock \emph{Zur Elektronentheorie der Dispersion und Absorption der Metalle} (For the electron theory of dispersion and absorption of metals),
\newblock Ann. Phys. \textbf{38} (1912), pp. 731--763.


\bibitem{vanKampen1988}
N.G. van Kampen,
\newblock \emph{Diffusion in inhomogeneous media},
\newblock J. Phys. Chem. Solids \textbf{49} (1988), pp. 673--677.



\bibitem{Milligen2005}
B. Ph. van Milligen, P. D. Bons, B.A. Carreras and R. S\'{a}nchez,
\newblock \emph{On the applicability of Fick’s law to diffusion in inhomogeneous systems},
\newblock Eur. J. Phys. \textbf{26} (2005), 913--925.


\bibitem{Philibert2006}
J. Philibert,
\newblock \emph{One and a half century of diffusion: Fick, Einstein,
before and beyond},
\newblock Diffusion Fundamentals \textbf{4} (2006), pp. 6.1--6.19.

\bibitem{Selmeczi2007}
D. Selmeczi, S.F. Toli\'{c}-N{\o}rrelykke, E. Sch\"{a}ffer, P.H. Hagedorn, S. Mosler, K. Berg-S{\o}rensen, N.B. Larsen and H. Flyvbjerg,
\newblock \emph{Brownian motion after Einstein and Smoluchowski: some new applications and new experiments},
\newblock Acta Phys. Pol. B \textbf{38} (2007), pp. 2407--31.






\end{thebibliography}




\end{document}
