\documentclass[a4paper,11pt]{article}
%\hoffset=-20mm
%\voffset=-20mm
\usepackage[margin=3cm]{geometry}
\usepackage{setspace}
\onehalfspacing
%\doublespacing
%\usepackage{authblk}
\usepackage{amsmath}
\usepackage{amssymb}
\usepackage{amsthm}
% \usepackage{calrsfs}
%\usepackage[notcite,notref]{showkeys}

\usepackage{psfrag}
\usepackage{graphicx,subfigure}
\usepackage{color}
\def\red{\color{red}}
\def\blue{\color{blue}}
%\usepackage{verbatim}
% \usepackage{alltt}
%\usepackage{kotex}

 \special{papersize=150mm,230mm}
 \setlength{\oddsidemargin}{-22mm}
 \setlength{\evensidemargin}{-22mm}
 \setlength{\topmargin}{ -33mm}
 \setlength{\textheight}{210mm}
 \setlength{\textwidth}{130mm}




\usepackage{enumerate}


%%%%%%%%%%%%%%%%%%
\def\BO{{\mathcal{O}}}
\def\lio{{\mathcal{o}}}
\def\t{{\tau}}
% \def\l{{\|\pi\|}}
\def\l{{h}}
\def\Lip{{\mathrm{Lip}}}
\def\R{\mathbb{R}}
\def\Z{\mathbb{Z}}

\newcommand{\tcr}{\textcolor{red}}
\newcommand{\tcb}{\textcolor{blue}}
\newcommand{\ubar}[1]{\text{\b{$#1$}}}
\newtheorem{theorem}{Theorem}
\newtheorem{lemma}{Lemma}[section]
\newtheorem{proposition}{Proposition}[section]
\newtheorem{corollary}{Corollary}[section]
\newtheorem{definition}{Definition}[section]
\newtheorem{claim}{Claim}

\newcounter{mycounter}
\newtheorem{step}{Step}[mycounter]

\theoremstyle{remark}
\newtheorem{remark}{Remark}[section]


%%%%%%%%%%%%%%%%%%%%%%%%%%%%%%%%%%%%%%%%
\begin{document}
\title{Random walk with spatially heterogenous jumping time}
\date{\today}

\maketitle

\section{Introduction}
\begin{align*}
 u_t &= \frac{1}{2} \left( \Delta x(x) \left( \frac{\Delta x(x)}{\Delta t(x)} u \right)_x\right)_x= \frac{1}{2} \left( \tfrac{\Delta x(x)}{\l} \left( \frac{\tfrac{\Delta x(x)}{\l}}{\tfrac{\Delta t(x)}{\tau}} u \right)_x\right)_x \frac{\l^2}{\t} =  \frac{\l^2}{2\t} \left( \beta(x) \left( \frac{\beta(x)}{\alpha(x)} u \right)_x\right)_x
\end{align*}

\begin{definition}[Weak solution] We call $L^1_{loc}\big(\R^+;W^{1,1}(\R)\big)$ a weak solution of
\eqref{eqnw} for an initial data $w_0 \in L^1(\R)$ if, for any test function $\phi \in
C_c^1(\R^+\times \R)$,
 \begin{equation}\label{DefWeak}
  \int_0^T\int_\R \alpha w\phi_t\,dxdt
  +\int_\R\alpha w_0\phi(0,x)\,dx = d\int_0^T\int_\R w_x\phi_x\, dxdt.
 \end{equation}
\end{definition}


\section{Convergence of a discrete scheme}

\subsection{Time-space mesh and random walk algorithm}

We denote by $p_n^k$ the probability of a particle to be placed at $x=x^k$ at time $t=t^n$. The time-space mesh used in this paper is denoted by
$$
(t^n,x^k):=(n\tau,kh),\qquad (n,k)\in \Z^+_0\times\Z,
$$
where the time step size $\tau$ and space mesh size $h$ are small and $n$ and $k$ are integers. We consider a random walk system that a particle jumps to one of two adjacent grid points with the same probability ${1\over2}$. The special feature of the random walk system in this paper is that the jumping time interval $\Delta t$ is different for two regions $x<0$ and $x>0$. To make the situation physically clear, we consider $\Delta t$ as a traveling time to move to the next grid point. We let $\Delta t=\tau$ in the region $x<0$ and $\Delta t=2\tau$ in the other region $x\ge0$, i.e.,
\begin{equation}\label{alpha}
\Delta t=\alpha(x)\tau,\qquad
\alpha(x) =
\left\{\begin{array}{ll}
        1, & \text{if } x<0,\\
        2, & \text{if } x\ge0.
        \end{array}\right.
\end{equation}
Note that the step function $\alpha$ is to denote the spatial heterogeneity in $\Delta t$ and $\tau$ is the microscopic scale of it. In this case the probability $p_n^k$ satisfies
\begin{equation} \label{RW1}\tag{RW1}
\begin{aligned}
    2p^k_n = \left\{\begin{array}{lr}
        p^{k-1} _{n-1} + p ^{k+1} _{n-1}, & \text{if } k<0,\\
        p^{-1} _{n-1} + p ^{1} _{n-2}, & \text{if } k=0,\\
        p^{k-1} _{n-2} + p ^{k+1} _{n-2}, & \text{if } k>0.
        \end{array}\right.
\end{aligned}
\end{equation}
Note that the probability $p^k_n$ is the average of probabilities at two adjacent grid points at two moments, $t^{n-1}$ or $t^{n-2}$, depending on $k$. By removing from the equations the probabilities at  $t=t^{n-1}$ using the ones at $t=t_{n-2}$, we can rewrite \eqref{RW1} 
\begin{equation} \label{RW2}\tag{RW2}
\begin{aligned}
    &p^{k}_n = \left\{\begin{array}{ll}
        \frac{1}{4}p^{k-2} _{n-2} + \frac{1}{2}p ^{k} _{n-2} + \frac{1}{4} p^{k+2}_{n-2}, & \text{if $k<0$ is even},\\
        \frac{1}{4}p^{-2} _{n-2} + \frac{1}{4}p^{0} _{n-2} + \frac{1}{2}p ^{1} _{n-2}, & \text{if $k=0$},\\
        \frac{1}{2}p^{k-1} _{n-2} + \frac{1}{2}p ^{k+1} _{n-2}, & \text{if } k>0.
        \end{array}\right. 
\end{aligned}
\end{equation}
% 
% using even numbered time steps $n=2\ell$ only and obtain
% \begin{equation} \label{RW3}\tag{RW3}
% \begin{aligned}
%     &p^{k}_{2\ell} = \left\{\begin{array}{ll}
%         \frac{1}{4}p^{k-2} _{2(\ell-1)} + \frac{1}{2}p ^{k} _{2(\ell-1)} +
%         \frac{1}{4} p^{k+2}_{2(\ell-1)}, & \text{if $k<0$ is even},\\
%         \frac{1}{4}p^{-2} _{2(\ell-1)} + \frac{1}{4}p^{0} _{2(\ell-1)} +
%         \frac{1}{2}p ^{1} _{2(\ell-1)}, & \text{if $k=0$},\\
%         \frac{1}{2}p^{k-1} _{2(\ell-1)} + \frac{1}{2}p ^{k+1} _{2(\ell-1)},
%         & \text{if } k>0.
%         \end{array}\right.
% \end{aligned}
% \end{equation}


\subsection{Discretization}

Suppose that an initial probability density function be given by $v_0(x)$ for $x\in\R$. We assume that
\begin{equation} \label{initial}
 v_0 \in L^1( \R)\cap L^\infty(\R), \quad \int_ \R v_0(x)\, dx = 1, \quad
 w_0:=\frac{v_0(x)}{\alpha(x)} \in C^\infty_c(\R).
\end{equation}
The purpose of this paper is to support the hypothesis that what enjoys the propagation or the gain of regularities is the variable $w$, not $v$, whose actual distinction is only relevant when the medium is heterogeneous. This is why the regularity assumption of the problem is on the initial value of $w_0={v_0\over\alpha}$.

For a given mesh size $h>0$, we discretize the initial probability density function $v_0$ by
\begin{equation}\label{p0k}
\bar{p}_{0}^{k} = \left\{\begin{array}{ll}
               0 & \text{if } k<0\ \text{is odd},\\
               \int_{x^k}^{x^{k}+2h} v_0(x)dx & \text{if } k<0\ \text{is
               even},\\
                \int_{x^k}^{x^{k}+h} v_0(x)dx  & \text{if } k\ge0.
               \end{array}\right.
\end{equation}
This discretization is motivated from that a random walk process defined by \eqref{RW2} stands independently of \eqref{RW1} and is identical to that by \eqref{RW1} with the initial data such that $\bar{p}^k_0=0$ if $k<0$ is odd. Indeed \eqref{RW2} is self-contained; $\sum_{k=-1}^{-\infty}\bar{p}_{0}^{2k}+\sum_{k=0}^{\infty}\bar{p}_{0}^{k}= 1$ and $\sum_{k=-1}^{-\infty}\bar{p}_{2\ell}^{2k}+\sum_{k=0}^{\infty}\bar{p}_{2\ell}^{k}=1$ for all $\ell>0$. Note that missing probabilities are all zero, $\bar{p}_{2\ell}^{2k+1}=0$ if $k<0$. The justification that this discretization solely  accounts for the general case of \eqref{RW1} is not treated in this paper. 

Besides, we keep labeling a quantity defined over the interval by its leftmost grid point index, as was how $\bar{p}^k_0$ is labeled, at our convenience.


To denote the space grids with non-zero probability, we introduce another labelling, namely $(\ell,j)\in \mathbb{Z}_0^+ \times \mathbb{Z}$. The inverse map $(\ell,j)\mapsto (n,k): \mathbb{Z}_0^+ \times \mathbb{Z} \rightarrow \mathbb{Z}_0^+ \times \mathbb{Z}$ defines the labelling,
\begin{align*}
 n(\ell) &= 2\ell, \quad 
 k(j)  =\left\{\begin{array}{lr}
        2j, & \text{if } j<0,\\
        j, & \text{if }  j\ge0.
        \end{array}\right.
\end{align*}
Having that, we consider the grid points of $x^{k(j)}_{n(\ell)}$ and the probability $\bar{p}_{n(\ell)}^{k(j)}$. In particular, we define $y^j:=x^{k(j)}$.
% and $p^j_\ell$, without assigning different symbol, is understood as $p^{k(j)}_{n(\ell)}$. 

% \begin{equation}\label{yj}
% y^j=x^{2j}\quad\text{if}\quad j<0,\quad\text{and}\quad
% y^j=x^{j}\quad\text{if}\quad j\ge0.
% \end{equation}
To approximate the probability density function, not the total probability mass for the particle to be placed between two grid points, we divide $\bar{p}_{n(\ell)}^{k(j)}$ by the domain size of the integrals in \eqref{p0k}. 
\begin{equation}\label{vj}
\bar{v}^{j}_\ell:={\bar{p}_{n(\ell)}^{k(j)} \over2h}\quad\text{for}\quad
j<0\quad\text{and}
\quad \bar{v}^{j}_\ell:={\bar{p}_{n(\ell)}^{k(j)}\over h}\quad\text{for}\quad j\ge0.
\end{equation}
The activity function $w$ are respectively approximated by
% The probability density distribution $v(t,x)$ will be the limit of $v^j_\ell$ as $h\to0$. However, we cannot use the relation \eqref{RW2} for $v^{j}_\ell$'s due to the non-symmetry in \eqref{vj}. Instead, we will take
\begin{equation}\label{pandw}
\bar{w}^{j}_\ell:={\bar{p}_{n(\ell)}^{k(j)}\over2h}\quad\text{if}\quad j<0,\quad\text
{and} \quad \bar{w}^{j}_\ell:={\bar{p}_{n(\ell)}^{k(j)} \over2h}\quad\text{if}\quad j\ge0.
\end{equation}
Then, $\bar{v}^{j}_\ell=\bar{w}^{j}_\ell$ for $j<0$ and $\bar{v}^{j}_\ell=2\bar{w}^{j}_\ell$ for
$j\ge0$, i.e.,
$$
\bar{v}^j_\ell=\bar{w}_\ell^j\alpha(y^j).
$$
Realize further that $\bar{w}^{j}_\ell$ is a finite volume discretization of $w_0(x)$ given in \eqref{initial}, i.e.,
\begin{equation} \label{vol_aver+pmf}
\bar{w}_{0}^{j} ={1\over y^{j+1}-y^{j}}\int_{y^j}^{y^{j+1}} w_0(x)dx.
\end{equation}
\eqref{RW2}, after simply dividing it by $2h$, defines a dimensional version of random walk system for $\bar{w}^{j}_\ell$. 

Despite the inexact total probability, it is expedient to consider also the finite difference discretization of $w_0(x)$,
\begin{equation}\label{w_fdm}
w^{j}_0:=w_0(y^j),
\end{equation}
and to reversely define  
\begin{align*}
 v^j_0:=w^j_0 \quad \text{for $j<0$}, \quad &v^j_0:=2w^j_0 \quad \text{for $j\ge0$},\\
 p^j_0:=2hv^j_0\quad \text{for $j<0$}, \quad &p^j_0:=hv^j_0 \quad \text{for $j\ge0$}.
\end{align*}
On ahead, the random walk system with the discrete initial data \eqref{w_fdm} described by 
\begin{equation} \label{RW3}\tag{RW3}
\begin{aligned}
    &w^{j}_\ell = \left\{\begin{array}{ll}
        \frac{1}{4}w^{j-1} _{\ell-1} + \frac{1}{2}w ^{j} _{\ell-1} +
        \frac{1}{4} w^{j+1}_{\ell-1}, & \text{if $j<0$},\\
        \frac{1}{4}w^{-1} _{\ell-1} + \frac{1}{4}w^{0} _{\ell-1} +
        \frac{1}{2}w ^{1} _{\ell-1}, & \text{if $j=0$},\\
        \frac{1}{2}w^{j-1} _{\ell-1} + \frac{1}{2}w ^{j+1} _{\ell-1}, &
        \text{if } j>0.
        \end{array}\right.
\end{aligned}
\end{equation}
is analyzed. We will find the limit function $w(t,x)$ of $w_\ell^j$'s as $h\to0$. Then, the probability density distribution $v(t,x)$ is recovered by $\alpha(x)w(t,x)$ with the total probability $1$ and the initial value becomes $v(0,x)=v_0(x)$ as given in \eqref{initial}.

\subsection{Difference quotients}

Difference quotients are useful to the regularity study and finding the relation satisfied by the limit of finite difference schemes. Let
\begin{equation}\label{Ljl}
L^j_\ell := \frac{w_\ell^{j+1}-w_\ell^j}{y^{j+1}-y^j}
        =\left\{\begin{array}{ll}
        \frac{w^{j+1}_{\ell} - w^{j}_{\ell}}{2\l}, & \text{if $j<0$},\\
        \frac{w^{j+1}_{\ell} - w^{j}_{\ell}}{\l}, & \text{if }  j\ge0,
        \end{array}\right.
\end{equation}
which is an approximation of the gradient $w_x$. Let
\begin{equation}\label{Qjl}
Q^j_\ell := \frac{L^j_\ell-L^{j-1}_\ell}{2h}= \left\{\begin{array}{ll}
        \frac{1}{2\l^2}\Big( \frac{1}{2} w^{j-1} _{\ell} - w ^{j} _{\ell} +
        \frac{1}{2} w^{j+1} _{\ell}\Big), & \text{if $j<0$},\\
        \frac{1}{2\l^2}\Big( \frac{1}{2} w^{-1} _{\ell} - \frac{3}{2}w
        ^{0} _{\ell} +w^{1} _{\ell}\Big), & \text{if $j=0$},\\
        \frac{1}{2\l^2}\Big(w^{j-1} _{\ell} - 2w^j_\ell + w ^{j+1}
        _{\ell}\Big), & \text{if $j>0$}.
        \end{array}\right.
\end{equation}
Notice that $y^{j+1}-y^{j}=2h$ for $j<0$ and $y^{j+1}-y^{j}=h$ for $j\ge0$. Therefore, $Q^j_\ell$ is an approximation of $w_{xx}$ for $j<0$ and of ${1\over2}w_{xx}$ for $j>0$, i.e., of ${w_{xx}\over\alpha}$.\footnote{Remember that it is NOT an approximation of $({w_x\over\alpha})_x$. There is no delta measure like phenomenon in the relation. This observation will be useful when we consider a general case in the next paper.} From \eqref{RW3}, we obtain
$$
\begin{aligned}
    &w^{j}_\ell-w^{j}_{\ell-1} = \left\{\begin{array}{ll}
        \frac{1}{4}w^{j-1} _{\ell-1} - \frac{1}{2}w ^{j} _{\ell-1} +
        \frac{1}{4} w^{j+1}_{\ell-1}, & \text{if $j<0$},\\
        \frac{1}{4}w^{-1} _{\ell-1} - \frac{3}{4}w^{0} _{\ell-1} +
        \frac{1}{2}w ^{1} _{\ell-1}, & \text{if $j=0$},\\
        \frac{1}{2}w^{j-1} _{\ell-1}-w^{j}_{\ell-1} + \frac{1}{2}w ^{j+1}
        _{\ell-1}, & \text{if } j>0.
        \end{array}\right.
\end{aligned}
$$
The three cases of the above are written in the same form using $Q_\ell^j$,
which is
\begin{equation} \label{RW1w}
\begin{aligned}
 \frac{w^j_\ell - w^j_{\ell-1}}{2\tau} &= \left(\frac{h^2}{2\tau}\right)
 \frac{1}{2h}\left( \Big(\tfrac{w^{j+1}_{\ell-1} -
 w^{j}_{\ell-1}}{y^{j+1}-y^j}\Big) -\Big(\tfrac{w^{j}_{\ell-1} -
 w^{j-1}_{\ell-1}}{y^{j}-y^{j-1}} \Big)\right)=
 \left(\frac{h^2}{2\tau}\right)Q^j_{\ell-1}.
  \end{aligned}
\end{equation}
This is the main relation which will give the diffusion equation satisfied by the probability density function.


\subsection{Uniform estimates}
The relations in \eqref{RW3} can be used to find relations for $L^j_\ell$ and $Q^j_\ell$ which are
\begin{equation} \label{eq:L}
L^j_\ell = \left\{\begin{array}{ll}
        \frac{1}{4}L_{\ell-1}^{j-1} + \frac{1}{2}L_{\ell-1}^j +
        \frac{1}{4}L_{\ell-1}^{j+1}, & \text{if $j<0$},\\
        \frac{1}{2}L_{\ell-1}^{j-1} + \frac{1}{2}L_{\ell-1}^{j+1}, &
        \text{if $j\ge0$},
        \end{array}\right.
\end{equation}
and
\begin{equation}\label{eq:Q}
\begin{aligned}
Q^j_\ell %&= \frac{L^{j+1}_\ell - L^j_\ell}{\l}\\
&=\left\{\begin{array}{ll}
        \frac{1}{4}Q_{\ell-1}^{j-1} + \frac{1}{2}Q_{\ell-1}^{j} +
        \frac{1}{4}Q_{\ell-1}^{j+1}, & \text{if $j<0$},\\
        \frac{1}{4}Q_{\ell-1}^{-1} + \frac{1}{4}Q_{\ell-1}^{0} +
        \frac{1}{2}Q_{\ell-1}^{1}, & \text{if $j=0$},\\
        \frac{1}{2}Q_{\ell-1}^{j-1} + \frac{1}{2}Q_{\ell-1}^{j+1}, &
        \text{if $j>0$}.
        \end{array}\right.
 \end{aligned}
\end{equation}
Note that $L^j_\ell$ and $Q^j_\ell$ are again averaging relations of previous time steps, and this points to our theme monitoring the $w$ variable, not $v$ variable. This allows key estimates needed.

\begin{lemma}[Uniform estimates by the initial value] \label{uniform_est} ~

\begin{enumerate}\item Total sums of $w_\ell^j$, $L_\ell^j$, and
$Q_\ell^j$ are constant, i.e.,
$$
\sum_{j} w^j_\ell = \sum_{j} w^j_0,\quad \sum_{j} L^j_\ell = \sum_{j}
L^j_0,\quad \sum_{j} Q^j_\ell = \sum_{j} Q^j_0.
$$
\item Sequences $w_\ell^j$, $L_\ell^j$, and $Q_\ell^j$ are bounded by the
    initial maximums, i.e.,
  $$
  |w^j_\ell| \le \sup _{j} |w^j_0|,\quad  |L^j_\ell| \le \sup _{j}
  |L^j_0|,\quad |Q^j_\ell| \le \sup _{j} |Q^j_0|.
  $$
\item We also have, for $p\ge1$,
$$
\sum _{j} |w^j_\ell|^p \le \sum _{j} |w^j_0|^p,\quad\sum _{j} |L^j_\ell|^p \le \sum _{j} |L^j_0|^p,\quad \sum _{j} |Q^j_\ell|^p \le
\sum _{j} |Q^j_0|^p.
$$
\end{enumerate}
\end{lemma}
\begin{proof}
The first two are obvious since the relations in \eqref{RW3},\eqref{eq:L}, and \eqref{eq:Q} are averaging processes. The third one is also obvious since $f(x)=|x|^p$ is a convex function for all $p\ge1$. One may apply Jensen's inequality to the averaging processes, \eqref{RW3},\eqref{eq:L}, and \eqref{eq:Q}. For example, \eqref{eq:L} turns into
$$
|L^j_\ell|^p \le \left\{\begin{array}{ll}
        \frac{1}{4}|L_{\ell-1}^{j-1}|^p + \frac{1}{2}|L_{\ell-1}^j|^p +
        \frac{1}{4}|L_{\ell-1}^{j+1}|^p, & \text{if $j<0$},\\
        \frac{1}{2}|L_{\ell-1}^{j-1}|^p + \frac{1}{2}|L_{\ell-1}^{j+1}|^p, &
        \text{if $j\ge0$}.
        \end{array}\right.
$$
Therefore, $\sum _{j} |L^j_\ell|^p \le\sum _{j} |L^j_{\ell-1}|^p\le\cdots\le\sum _{j} |L^j_0|^p$.
\end{proof}

Note that this lemma implies that we have enough regularity in $x$ variables. For example, the second assertion implies that the approximation of the second order derivative, $Q^j_\ell$, is uniformly bounded by the initial condition. 

\subsection{Linear interpolation and subsequential convergence}
In this section, we construct functions defined on the continuum space $\R^+\times\R$ which approximates the discrete values $w^j_\ell$ defined at grid points. The reconstruction is simply piecewise linear one, or we merely work in the framework of Lipschitz functions. This will turn out to be enough for the uniqueness and the existence of the limit. We show the convergence of the continuum approximations as $h\to0$. As typical, we construct four different interpolations,
\begin{equation} \label{linear0}
\begin{aligned}
 w_1^h(t,x)&:= w^j_\ell,\\
 w_2^h(t,x)&:= \left(\tfrac{t-t_{2\ell}}{t_{2\ell+2}-t_{2\ell}} \right)
 w_{\ell+1}^{j} + \left(\tfrac{t_{2\ell+2}-t}{t_{2\ell+2}-t_{2\ell}} \right)
 w_{\ell}^{j},\\
 w_3^h(t,x)&:= \left(\tfrac{x-y^j}{y^{j+1}-y^j} \right) w_{\ell}^{j+1}
 + \left(\tfrac{y^{j+1}-x}{y^{j+1}-y^j} \right) w_{\ell}^{j},\\
  w^h(t,x)&:= \left(\tfrac{x-y^j}{y^{j+1}-y^j}
  \right)\left(\tfrac{t-t_{2\ell}}{t_{2\ell+2}-t_{2\ell}} \right)
  w_{\ell+1}^{j+1} + \left(\tfrac{y^{j+1}-x}{y^{j+1}-y^j}
  \right)\left(\tfrac{t-t_{2\ell}}{t_{2\ell+2}-t_{2\ell}} \right)
  w_{\ell+1}^{j}\\
 &+ \left(\tfrac{x-y^j}{y^{j+1}-y^j}
 \right)\left(\tfrac{t_{2\ell+2}-t}{t_{2\ell+2}-t_{2\ell}} \right)
 w_{\ell}^{j+1} + \left(\tfrac{y^{j+1}-x}{y^{j+1}-y^j}
 \right)\left(\tfrac{t_{2\ell+2}-t}{t_{2\ell+2}-t_{2\ell}} \right)
 w_{\ell}^{j}
  \end{aligned}
\end{equation}
when $(t,x)\in D_\ell^j:=[t_{2\ell}, t _{2\ell+2}) \times [y^j, y^{j+1})$. Note that $w_1^h$ is constant on $D_\ell^j$, and $w_2^h$ and $w_3^h$ are linear on $D_\ell^j$. The last one, $w^h(t,x)$, is continuous on $\R^+\times\R$. All of them has the value $w^j_\ell$ at $t=t_{2\ell}$ and $x=y^j$. We now see in the following lemma that the functions $w^h(t,x)$ are uniformly bounded in the Lipschitz space $C^{0,1}\big(\R^+\times \mathbb{R}\big)$ where the norm is defined by
$$\|u\|_{C^{0,1}\big(\R^+\times \mathbb{R}\big)} = \sup_{(t,y)\in \R^+\times \mathbb{R}}|u(t,y)| + \sup_{(t,y)\ne(s,z) \in \R^+\times \mathbb{R}} \frac{|u(t,y)-u(s,z)|}{|(t,y)-(s,z)|}.$$
\begin{lemma} \label{misc}
Let $w_0\in C^{\infty}_c(\R)$ and $w^h$ be defined by \eqref{linear0}. We assume that $\tau$ and $h$ are related by
$$
{h^2\over 2\tau}=d,
$$
where $d>0$ is fixed. Then, for any $h>0$ small and any $t>0$, 
$$\|\partial_x w^h(t,\cdot)\|_{L^\infty(\R)} + \|\partial_t w^h(t,\cdot)\|_{L^\infty(\R)} \le C,$$
where $C$ depends only on $d$ and $\|w_0\|_{W^{2,\infty}(\R)}.$
\end{lemma}
\begin{proof}
It is obvious that
$$|L_0^{j;h}| + |Q_0^{j;h}| \le C\|w_0\|_{W^{2,\infty}(\R)}.$$
from the Taylor Theorem. The linear interpolation $w^h$ has weak first 
derivatives for a.e. $(t,y)$. By Lemma \ref{uniform_est},
 \begin{align*}
 |\partial_t w^h(t,y)| \le d\sup_j |Q^{j;h}_{\ell}|\le d\sup_j |Q^{j;h}_{0}|,\quad \quad
 |\partial_y w^h(t,y)| \le \sup_j |L_0^{j;h}|.
 \end{align*}
\end{proof}
% \begin{lemma} \label{misc}
% Let $w_0\in C^{\infty}(\R)$ and $w^h$ be defined by \eqref{linear0}. We assume that $\tau$ and $h$ are related by
% $$
% {h^2\over 2\tau}=d,
% $$
% where $d>0$ is fixed. Then,
% \begin{enumerate}
%  \item For any $t>0$ and $h>0$, $\|\partial_x w^h(t,\cdot)\|_{L^\infty(\R)} \le \|\partial_x w_0\|_{L^\infty(\R)}$. Hence, $w^h(t,\cdot)$ is uniformly Lipschitz continuous in the $x$-variable.
%  \item {\red need the correction} The approximations $w^h(\cdot,x)$ are uniformly Holder continuous in the $t$-variable with exponent ${1\over2}$ with respect to $h>0$ and $x\in\R$. In other words, there exists a constant $C>0$ such that, for any $x\in\R$ and $h>0$,
%      \begin{equation}\label{Holder}
%      |w^h(t,x)-w^h(s,x)|\le C\sqrt{|t-s|} \quad\text{for all } |t-s|\text{
%      small}.
%      \end{equation}
%  \end{enumerate}
% \end{lemma}
% \begin{proof}
% 1. Since $w_0^j$ is the mean of $w_0(x)$ in the interval $(y^j,y^{j+1})$
% (see \eqref{vol_aver+pmf}) and $w^h(0,\cdot)$ is the linear interpolation of
% $(y^j,y^{j+1})$, the first assertion is clear by Lemma \ref{uniform_est}.
% 
% 2. Let $L_0:=\sup |w_0'(x)|$. Then, $|L_0^{j}|\le L_0$ for any mesh size
% $h>0$. It is enough to consider an estimate along mesh grids and let
% $x=y^j$. By \eqref{Qjl} and \eqref{RW1w}, we have
% $$
%  \frac{|w^j_\ell - w^j_{\ell-1}|}{2\tau} \le d\sup|Q^j_{\ell}|\le {d\over
%  h}L_0.
% $$
% Multiply $\sqrt{2\tau}$ both sides and obtain
% $$
%  {|w^h(t_{2\ell},y^j)-w^h(t_{2\ell-2},y^j)|\over\sqrt{2\tau}}=\frac{|w^j_\ell
%  - w^j_{\ell-1}|}{\sqrt{2\tau}} \le d\sqrt{2\tau\over
%  h^2}L_0=\sqrt{d}\,L_0.
% $$
% This inequality \eqref{Holder} holds for any $|s-t|<\tau$ and hence $w^h$ is Holder continuous with exponent ${1\over2}$. Since the upper bound $C:=\sqrt{d}\,L_0$ is independent of $h$ and $x$, $w^h$'s are uniformly Holder continuous.
% \end{proof}

By the Arzela-Ascoli theorem, there exists a subsequence $w^{h_i}(t,x)$ such that $w^{h_i}(t,x)\to w$ uniformly on any compact set for a limit function $w$ in the Lipschitz space. Furthermore, due to the way of construction, $w_1^{h_i},w_2^{h_i}$ and $w_3^{h_i}$ also converge to the same limit in the uniform norm.

\begin{lemma}\label{compactness} There exists a subsequence $\{w^{h_i}\}$ and a function $w\in C^{0,1}\big(\R^+\times \R\big)$ such that $w_1^{h_i},w_2^{h_i},w_3^{h_i},w^{h_i}\rightarrow w$ uniformly on any compact set $K \subset \R^+\times \R$ and $\partial_x w_3^{h_i}, \partial_x w^{h_i} \rightharpoonup \partial_y w$ weakly-* in $L^\infty(\R^+\times \mathbb{R})$.
\end{lemma}
\begin{proof}
 We prove only the assertion that $\partial_x w_3^{h_i}$ weakly-* converges to the same limit $\partial_x w^{h_i}$ converges.  %First we zero extend the $\partial_x {w}^h(t,x)$, $\partial_x w_3^h(t,x)$ and the test function $g\in L^1(\R^+\times \mathbb{R})$ for all $t\in \mathbb{R}$. Then
 For a test function $g\in L^1(\R^+\times \mathbb{R})$,
 \begin{align*}
  &\int_0^\infty\int_\mathbb{R} \Big|\partial_x {w}^{h_i} - \partial_x w_3^{h_i}\Big| \Big|g(t,x)\Big| \; dxdt  \\%=\int_\mathbb{R}\int_\mathbb{R} \Big|\partial_x {w}^{h_i} - \partial_x w_3^{h_i}\Big| \Big|g(t,x)\Big| \; dydt\\
  &= \sum_{\ell\ge0}\int_{t_{2\ell}}^{t_{2(\ell+1)}}\int_\mathbb{R} \Big|\partial_x w_3^{h_i}(t+2\tau,x)-\partial_x w_3^{h_i}(t,x) \Big| \left|\frac{t-t_{2\ell}}{t_{2(\ell+1)}-t_{2\ell}}\right|~\Big|g(t,x)\Big| \;dxdt\\
%   &= \sum_{\ell\ge0}\int_{t_\ell}^{t_{\ell+1}}\int_\mathbb{R} \Big|\partial_y \hat{w}^{h(m)}(t+2\tau,y)-\partial_y \hat{w}^{h(m)}(t,y) \Big| \left|\frac{t-t_\ell}{t_{\ell+1}-t_{\ell}}g(t,y) \right|\;dydt\\
  &\le \int_0^\infty \int_\mathbb{R} \Big|\partial_x w_3^{h_i}(t+2\tau,x)-\partial_x w_3^{h_i}(t,x) \Big| \Big|g(t,x)\Big| \; dxdt.
 \end{align*}
 Because the integrand pointwise converges to $0$, by Lebesgue dominated convergence theorem the error converges to $0$.
\end{proof}

\subsection{Convergence to a weak solution}

In this section we show that the subsequential limit is a weak solution to a
Cauchy problem of a non-autonomous parabolic equation,
\begin{equation}\label{eqnw}
  \alpha(x) w_t = d w_{xx},\qquad w(0,x)=w_0(x).
\end{equation}
We also show that there is an energy functional such that the energy level of the weak solution decreases as $t\to\infty$. This implies the uniqueness of the solution in the class of admissible ones and the convergence of $w^h$ as $h\to0$.

Let $w$ be a smooth solution of \eqref{eqnw} and define an energy
functional by
$$
e(t;w)={1\over 2}\int \alpha(x) w^2(t,x)\,dx.
$$
Then formally,
$$
e'(t;w)=\int \alpha ww_t\,dx=d\int ww_{xx}\,dx=-d\int w_x^2\,dx\le 0.
$$
We show that the energy of a subsequential limit decreases in time.
\begin{theorem}
Let $w$ be a subsequential limit, i.e., $w^{h_i}\to w$.
\begin{enumerate}
 \item  $\int_\R\alpha(x) w(t,x)\;dx = 1$.
 \item For all $t>0$, $e(t;w)\le e(0;w)$.
\end{enumerate}
\end{theorem}
\begin{proof}
It is straightforward to see that the $L^p$ norm of $w_0(x)$ and that of $w_0^h(x)$ are bounded for any $p\in [1,\infty]$. Further, by the Lemma \ref{uniform_est} and that $w_1^h(t,\cdot) \rightarrow w(t,\cdot)$ uniformly in any compact set of $\R$ for all $t\in\R^+$, $w_1^h(t,\cdot) \rightarrow w(t,\cdot)$ strongly in $L^p(\R)$ for any $p \in [1,\infty)$.

Now, for $t\in [t_{2\ell}, t_{2\ell+2})$, we have
\begin{align*}
\int_\R \alpha(x)w_1^h(t,x) \; dx &= \sum_{j<0}w^j_\ell2h + 
\sum_{j\ge0}2w^j_\ell h = \sum_{j<0}p^{2j}_{2\ell} +
\sum_{j\ge0}p^j_{2\ell} = \sum_{j<0}p^{2j}_{0} + \sum_{j\ge0}p^j_{0} \rightarrow 1
\end{align*}
since $w_1^h(t,\cdot) \rightarrow w(t,\cdot)$ in $L^1(\R)$ for all $t\ge0$ and $\int_\R\alpha(x) w_0(x) \; dx=1$.

Next consider the energy of the approximation $w^h_1$. If $t\in [t_{2\ell}, t_{2\ell+2})$,
\begin{align*}
e(0;w^h_1) &= \frac{1}{2} \int_\R \alpha(x) \big(w^h_1(0,x)\big)^2dx = \frac{1}{2} \Big( \sum_{j<0}(w_0^{j})^2(2h) +\sum_{j\ge0}2(w_0^{j})^2h \Big)=\sum_{j<0}(w_0^{j})^2\,h\\
&\ge \sum_{j<0}(w_\ell^{j})^2\,h =\frac{1}{2} \Big( \sum_{j<0}(w_\ell^{j})^2(2h) +\sum_{j\ge0}2(w_\ell^{j})^2h \Big) = e(t;w^h_1).
\end{align*}
% \begin{align*}
% e(t;w^h_1) &= \frac{1}{2} \int_\R \alpha(x) \big(w^h_1(t,x)\big)^2dx= \frac{1}{2} \Big( \sum_{j<0}(w_\ell^{j})^2(2h) +\sum_{j\ge0}2(w_\ell^{j})^2h \Big)\\
% &= \sum_{j<0}(w_\ell^{j})^2\,h\le \sum_{j<0}(w_0^{j})^2\,h = \frac{1}{2} \Big( \sum_{j<0}(w_0^{j})^2(2h) +\sum_{j\ge0}2(w_0^{j})^2h \Big)=e(0;w^h_1).
% \end{align*}
Therefore, since $w_1^h(t,\cdot) \rightarrow w(t,\cdot)$ in $L^2(\R)$ for all $t\ge0$, $e(t;w)\le e(0;w)$ for all $t\ge0$.
% Since $f(x)=x^2$ is a convex function, \eqref{RW2} gives that
% $$
% \begin{aligned}
%     &(p^{k}_{2\ell})^2 \le \left\{\begin{array}{ll}
%         \frac{1}{4}(p^{k-2} _{2(\ell-1)})^2 + \frac{1}{2}(p^{k}_{2(\ell-1)})^2 +\frac{1}{4} (p^{k+2}_{2(\ell-1)})^2, & \text{if $k<0$ is even},\\
%         \frac{1}{4}(p^{-2} _{2(\ell-1)})^2 + \frac{1}{4}(p^{0} _{2(\ell-1)})^2 +\frac{1}{2}(p ^{1} _{2(\ell-1)})^2, & \text{if $k=0$},\\
%         \frac{1}{2}(p^{k-1} _{2(\ell-1)})^2 + \frac{1}{2}(p ^{k+1} _{2(\ell-1)})^2,
%         & \text{if } k>0.
%         \end{array}\right.
% \end{aligned}
% $$
% Therefore, $e(t^{2\ell};w^h_1)\le e(0;w^h_1)$ for all $\ell>0$. Since $w^{h_i}\to w$ as $h_i\to0$, we have $e(t;w)\le e(0;w)$ for all $t>0$.
\end{proof}

Finally, we show that the sequence converges to a weak solution of \eqref{eqnw}.
\begin{theorem}
  The interpolation $w^{h}(t,x)$ converges to $w\in C^{0,1}(\R^+\times\R)$ and the limit is a weak solution of \eqref{eqnw}.
\end{theorem}
\begin{proof}
It is enough to show that the subsequential limit $w^{h_i}\to w$ is a weak solution of \eqref{eqnw}. Then, the uniqueness of energy decreasing solution gives the rest. The regularity of the limit $w$ is a consequence of the Arzela-Ascoli Theorem. Consider the two terms in the left side of \eqref{DefWeak} with $w_2^{h_i}$ in the place of $w$,
\begin{align*}
 \int_0^\infty\int_\R& w_2^{h_i}(t,x) \alpha(x)\partial_t\phi(t,x)\; dx dt
 +\int_\R w_2^{h_i}(0,x)\alpha(x)\phi(t,x) \; dx\hskip 5mm \Big(=:A_i\Big).
\end{align*}
Since $w_2^{h_i} \rightarrow w$ uniformly on each compact set of $\R^+\times \R$, the left side converges as
\begin{align*}
\lim_{h_i\to0}A_i=\int_0^\infty\int_\R w(t,x) \alpha(x)\partial_t\phi(t,x)\; dx
dt+\int_\R w_0(x)\alpha(x)\phi(t,x) \; dx.
\end{align*}
Now,
\begin{align*}
%  \int_0\infty\int_\R& w_2^{h_i}(t,x) \alpha(x)\partial_t\phi(t,x)\; dx dt
%  +\int_\R w_2^{h_i}(0,x)\alpha(x)\phi(t,x) \; dx\hskip 5mm \Big(=:A_i\Big)\\
 A_i&=-\sum_{\ell}\int_{t_{2\ell}}^{t_{2\ell+2}}\int_\R \partial_t
 w_2^{h_i}(t,x) \alpha(x)\phi(t,x)\; dx dt \\
 &=-\sum_{\ell}\sum_{j}\int_{t_{2\ell}}^{t_{2\ell+2}}\int_\R
 \frac{w^j_{2\ell+2}-w^j_{\ell}}{2\tau} \alpha(x) \phi(t,x) \chi_j(x)\; dx dt \hskip
 1.5cm (\text{with } \chi_j=\chi_{[y^j,y^{j+1})})\\
 &=-\sum_{\ell}\sum_{j}d\int_{t_{2\ell}}^{t_{2\ell+2}}\int_\R
 \tfrac{1}{2h} \left( \Big(\tfrac{w^{j+1}_{\ell} -
 w^{j}_{\ell}}{y^{j+1}-y^j}\Big) -\Big(\tfrac{w^{j}_{\ell} -
 w^{j-1}_{\ell}}{y^{j}-y^{j-1}} \Big)\right) \alpha(x) \phi(t,x) \chi_j(x)\; dx
 dt\quad\ (\text{by } \eqref{RW1w}\,)\\
 &=-\sum_{\ell}\sum_{j}d\int_{t_{2\ell}}^{t_{2\ell+2}}\int_\R
 \tfrac{1}{2h} \Big(\tfrac{w^{j}_{\ell} -
 w^{j-1}_{\ell}}{y^{j-1}-y^j}\Big)
 \big(\chi_{j-1}(x)-\chi_{j}(x)\big)\alpha(x) \phi(t,x) \; dx \;dt\\
 &=-\sum_{\ell}\sum_{j}d\int_{t_{2\ell}}^{t_{2\ell+2}}B_\ell^j\,dt\hskip 5mm\Big(B_\ell^j:=\int_\R \tfrac{1}{2h} \Big(\tfrac{w^{j}_{\ell} -
w^{j-1}_{\ell}}{y^{j-1}-y^j}\Big)
\big(\chi_{j-1}(x)-\chi_{j}(x)\big)\alpha(x) \phi(t,x) \; dx\Big).
\end{align*}

The $B^j_\ell$ in the right side becomes
\begin{align*}
 B_\ell^j &= \Big(\tfrac{w^{j}_{\ell} - w^{j-1}_{\ell}}{y^{j}-y^{j-1}}\Big)
        \left\{\frac{1}{2h}\int_{y^{j-1}}^{y^{j}} \alpha\phi  \; dx -
        \frac{1}{2h}\int_{y^{j}}^{y^{j+1}} \alpha\phi  \; dx\right\}\\
     &= \left\{\begin{array}{ll}
        \Big(\tfrac{w^{j}_{\ell} - w^{j-1}_{\ell}}{y^{j}-y^{j-1}}\Big)
        \bigg( \frac{1}{y^j-y^{j-1}}\int_{y^{j-1}}^{y^{j}} \phi  \; dx-
        \frac{1}{y^{j+1}-y^{j}}\int_{y^{j}}^{y^{j+1}} \phi  \; dx \bigg)
        & \text{if } j<0,\\
        \Big(\tfrac{w^{0}_{\ell} - w^{-1}_{\ell}}{y^{0}-y^{-1}}\Big)
        \bigg( \frac{1}{y^j-y^{j-1}}\int_{y^{j-1}}^{y^{j}} \phi  \; dx-
        \frac{1}{2(y^{j+1}-y^{j})}\int_{y^{j}}^{y^{j+1}} 2\phi  \; dx
        \bigg) & \text{if } j=0,\\
        \Big(\tfrac{w^{j}_{\ell} - w^{j-1}_{\ell}}{y^{j}-y^{j-1}}\Big)
        \bigg( \frac{1}{2(y^j-y^{j-1})}\int_{y^{j-1}}^{y^{j}} 2\phi  \; dx-
        \frac{1}{2(y^{j+1}-y^{j})}\int_{y^{j}}^{y^{j+1}} 2\phi  \; dx
        \bigg) & \text{if } j>0,
       \end{array}\right.\\
     &=\Big(\tfrac{w^{j}_{\ell} - w^{j-1}_{\ell}}{y^{j}-y^{j-1}}\Big)
        \bigg( \frac{1}{(y^j-y^{j-1})}\int_{y^{j-1}}^{y^{j}} \phi  \; dx-
        \frac{1}{(y^{j+1}-y^{j})}\int_{y^{j}}^{y^{j+1}} \phi  \; dx
        \bigg).
\end{align*}
Therefore,
\begin{align*}
 B_\ell^j =&-\int_{y^{j-1}}^{y^{j}}\Big(\tfrac{w^{j}_{\ell} -
 w^{j-1}_{\ell}}{y^{j}-y^{j-1}}\Big)\partial_x\phi \;dx \\
     & + \Big(\tfrac{w^{j}_{\ell} -
     w^{j-1}_{\ell}}{y^{j}-y^{j-1}}\Big)\left\{
        \frac{\int_{y^{j-1}}^{y^{j}} \phi -\phi(t,y^{j-1}) \;
        dx}{y^j-y^{j-1}} - \frac{\int_{y^{j}}^{y^{j+1}} \phi -\phi(t,y^{j})
        \; dx}{y^{j+1}-y^j} \right\}.
\end{align*}
Since the test function $\phi$ is $C^1$, the second term goes to zero faster than $h_i\to0$; the only exception is for $j=0$ where the error is $O(h_i)$ as $h_i\to0$.
Since the support of $\phi$ is compact, after summing $B_\ell^j$ up and taking $h_i\to0$, we obtain
\begin{align*}
-\sum_{\ell}\sum_{j}d\int_{t_{2\ell}}^{t_{2\ell+2}}B_\ell^j\,dt &= d\int_0^\infty\int_\R \partial_x w_3^{h_i}(t,x) \partial_x \phi(t,x) \;
dxdt + o(1) \\
&\rightarrow d\int_0^\infty\int_\R \partial_x w(t,x) \partial_x \phi(t,x) \; dxdt,
\end{align*}
where the weak-* convergence of $\partial_x w_3^{h_i}$ to $\partial_x w$ has been used.
Therefore, combining the limits shows that $w$ is a weak solution.
\end{proof}

\subsection{Conclusion}

The probability density function is $v=\alpha\, w$. Then, $v$ satisfies
$$
v_t=d\Big({v\over\alpha(x)}\Big)_{xx},\qquad
v(0,x)={\alpha(x)w_0(x)}=v_0(x),
$$
where the initial value $v_0(x)$ is the original initial value given in
\eqref{initial}.

If we include the spatial heterogeneity in the walk length
$\Delta x=\beta(x)\epsilon$, then the probability density function $v$
satisfies
$$
v_t=d\Big(\beta(x)\Big({\beta(x)\over\alpha(x)}v\Big)_x\Big)_x,\qquad
v(0,x)=v_0(x).
$$

\end{document}
