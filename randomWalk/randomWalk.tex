\documentclass[11pt]{amsart}
%\usepackage{amsmath,amsfonts,amssymb,mathrsfs}
%\usepackage{amssymb,mathrsfs}

\usepackage{amsfonts,amssymb,amsmath,mathrsfs,graphicx}
\usepackage{float}
%\usepackage{srcltx}
\usepackage{epsfig}
\usepackage{graphicx}
\usepackage{caption}
\usepackage{subcaption}
%\usepackage{refcheck}

\hoffset=-25mm \voffset=-25mm
\usepackage[left=35mm,right=35mm,top=35mm,bottom=40mm,a4paper]{geometry}

\title[Random walk with location dependent travelling time]{Spatially heterogeneous random walk in one space dimension}% with location dependent travelling time}

\author[Jaywan Chung]{Jaywan Chung}
\address[Jaywan Chung]{\newline Department of Mathematics, Dankook University \newline 119 Dandae-ro, Dongnam-gu, Cheonan-si, Chungnam 330-714, Korea}
\email{jaywan.chung@gmail.com}


\author[Yong-Jung Kim]{Yong-Jung Kim}
\address[Yong-Jung Kim]{\newline Department of Mathematical Sciences, KAIST \newline 291 Daehak-ro, Yuseong-gu, Daejeon, 305-701, Korea}
\email{yongkim@kaist.edu}


\author[Mingi Lee]{Mingi Lee}
\address[Mingi Lee]{\newline ...}
\email{...}

\def\eps{\epsilon}
\def\R{{\bf R}}


\setcounter{tocdepth}{3}

%%%%%%%%%%%%%% MY DEFINITIONS %%%%%%%%%%%%%%%%%%%%%%%%%%%%
\def\d{d}
\def\v{{\bf v}}
\def\e{{\bf e}}
\def\x{{x}}
\def\k{{\bf k}}
\def\p{{\bf p}}
\def\q{{\bf q}}
\def\R{{\bf R}}
\def\Z{{\bf Z}}
\def\bfR{{\bf R}}
\def\bfZ{{\bf Z}}
\def\f{{\bf f}}
\def\div{{\rm div}}
\def\tp{\Sigma}
\def\erf{\mathrm{erf}}
\def\erfc{\mathrm{erfc}}
%%%%%%%%%%%%%%%%%%%%%%%%%%%%%%%%%%%%%%%%%%%
\newtheorem{theorem}{Theorem}[section]
\newtheorem{lemma}{Lemma}[section]
\newtheorem{corollary}{Corollary}[section]
\newtheorem{proposition}{Proposition}[section]
\newtheorem{remark}{Remark}[section]
\newtheorem{definition}{Definition}[section]
\newtheorem{example}{Example}[section]

\begin{document}
% The correct dates will be entered by the editor
\maketitle
%
\begin{abstract}
  The random walk is a mathematical counterpart of a Brownian particle system and has been used as a powerful theoretical tool to understand random dispersal phenomena. If temperature is not spatially constant, the mean free path and the traveling time for the path has variation in space. In this paper we investigate behavior of a simple random walk having spatially dependent walk length and jumping time. One of the main conclusions is that one does not sense the change of the walk length on the other side, but does sense the change of the jumping time. This also indicates that a random walk system with a non-constant time step cannot be reduced to a constant time step one even if the walk length is rescaled.
\end{abstract}

%\tableofcontents



\section{Introduction}

In this paper we study the diffusion limit of a random walk system and find the corresponding diffusion equation. Let $\Delta x$ be the walk length and $\Delta t$ be the jumping time interval of a random walk system. ${1\over\Delta t}$ is also called a turning frequency. It is well understood that, if $\Delta x$ and $\Delta t$ are constant, the probability density $u$ in one space dimension satisfies a diffusion equation,
\begin{equation}\label{HomoDiffusion}
u_t=ku_{xx},\qquad k={|\Delta x|^2\over 2\Delta t},
\end{equation}
where the sub-indexes indicate partial derivatives. The diffusivity constant $k$ is given by the Einstein-Smoluchowski relation. The multi-dimensional case is similarly obtained, where the diffusivity becomes $k={|\Delta x|^2\over 2n\Delta t}$ in $\R^n$. The purpose of this paper is to develop a diffusion model when $\Delta x$ and $\Delta t$ are spatially non-constant.

The diffusion model \eqref{HomoDiffusion} is valid only when the diffusivity $k$ is constant. There have been many efforts and discussions to find a correct diffusion equation when the diffusivity is non-constant, i.e., $k=k(\x)$. For example, three diffusion models are often considered:
\begin{eqnarray}
% \nonumber % Remove numbering (before each equation)
\label{Fick} u_t &=& (k(\x)u_x)_x, \\
\label{Wereide} u_t &=& ( \sqrt{k(\x)}(\sqrt{k(\x)}u_x))_x, \\
\label{Chapman} u_t &=& (k(\x)u)_{xx}.
\end{eqnarray}
Eq. \eqref{Fick} is called Fick's law. Eq. \eqref{Wereide} and \eqref{Chapman} have been by Wereide \cite{Wereide1914} and Chapman \cite{Chapman1928}, respectively. However, none of them are agreed to be the correct diffusion model. For example, the Fickian type diffusion \eqref{Fick} and the Fokker-Planck type one \eqref{Chapman} are often compared as mathematical dispersal models for physical particles and biological species in heterogeneous environment (see \cite{Bengfort2016,Milligen}).

The three equations are satisfied by the probability density function of a stochastic process when the variation of jumping distance, i.e., $|\Delta x|^2$, depends on the terminal, middle, or starting point of each jump, respectively. In other words, \eqref{Wereide} and \eqref{Chapman} are obtained by the Stratonovich and Ito integrals of a stochastic process, respectively. Note that in the three cases of stochastic processes, the spatial heterogeneity is in $\Delta x$ only and $\Delta t$ is assumed to be spatially homogeneous. However, to understand a general spatially heterogeneous diffusion, it is essential to consider a case both $\Delta x$ and $\Delta t$ are spatially heterogeneous. For example, consider
\begin{equation*}
{1\over \Delta t}={\eps^2\over2}\mu(x)>0,\quad \Delta x=\eps b(x)>0,
\end{equation*}
where $\mu(x)$ and $b(x)$ contain the information of spatial heterogeneity in the turning frequency and walk length, respectively, and ${\eps^2\over2}$ and $\eps$ their microscopic scales. The claim of this paper is that the corresponding diffusion equation is
\begin{equation}\label{Kim}
u_t=(b(x)(b(x)\mu(x)u)_x)_x.
\end{equation}
In this paper we show that the diffusion limit of a random walk system satisfies the equation. If $\mu(x)$ and $b(x)$ are constant, then the diffusivity is $k={b^2\mu}$ and the four cases, \eqref{Fick}-\eqref{Kim}, are identical. If $\mu(x)=1$ and $b=b(x)$, \eqref{Kim} is identical to \eqref{Wereide}, the case of Stratonovich integral. We consider general spatial heterogeneity in $b(x)$. However, for $\mu(x)$, we consider a spacial case given by \eqref{mu(x)}.

A random walk system is a mathematical counterpart of Brownian motion of particles suspended in a fluid and the parameters $\Delta x$ and $\Delta t$ respectively represent the mean free path and the mean collision time of Brownian particles. Spatial heterogeneity in such a motion appears in naturally and an obvious source of it is the spatial difference in temperature. The Brownian particles move more actively in warm regions. The effect of temperature gradient such as thermal diffusion or Ludwig-Soret effect  \cite{Ludwig1856,Soret1879} has been well known for a long time. There are enormous amount of researches related such phenomena (see \cite{Braun2004, Duhr2006, Eslamian2009, Harstad2009, Huang2010,14,Putnam2007,Srinivasan2011}). However, the literatures agree that there is no comprehensive or generic thermal diffusion model such as Einstein's molecular level explanation for the homogeneous case.

In Section 2, we first consider a random walk model that the walk length is given as a spatial function $\Delta x={\|\pi\|\over2}b(x)$ for a constant $\|\pi\|$ in \eqref{pi} and the jumping time interval $\Delta t=\tau$ as a constant. It is shown in Theorem \ref{thm1} that the probability density function of the discrete random walk model converges to the solution of
$$
u_t=d(b(x)(b(x)u)_x)_x,\qquad d={\|\pi\|^2\over 8\tau}.
$$
Note that, even if this equation is not autonomous, we may solve it with a general initial value since, after changing the variable with
$$
y(x):=\int_0^x \frac{1}{b(s)}\,ds,
$$
the equation returns to the heat equation $u_t=du_{yy}$.

The spatial variation in the jumping time interval $\Delta t$ makes a more profound theoretical difference in comparison with the one in $\Delta x$. In particular the corresponding random walk model is not even a Markov process. In Section 3, we consider a simplest case when the jumping time consists of two values:
$$
\Delta t=
\begin{cases}
\tau & \text{if $x<0$,}\\
2\tau & \text{if $x>0$},
\end{cases}
$$
where $\tau>0$ is a constant. However, due to the discontinuity at the origin, it is challenging to develop and analyze a correspoding PDE model; it will be shown that the probability density function of the discrete random walk model converges to the solution of
\begin{equation}\label{ux1}
u_t=d(b(x)(b(x)\mu(x)u)_x)_x,\qquad d={\|\pi\|^2\over 8\tau},
\end{equation}
where
\begin{equation}\label{mu(x)}
\mu(x)=\begin{cases}
1 & \text{if $x<0$,}\\
0.5& \text{if $x>0$}.
\end{cases}
\end{equation}
Then, after the same variable change as above, we obtain $u_t=d(\mu(x)u)_{yy}$. If we set $w=\mu u$ and $d=1$, then $w$ satisfies
$$
w_t=\mu w_{yy},\qquad y\in\R.
$$
Notice that this equation is meaningless at $y=0$ due to the discontinuity of $\mu$. However, it is shown in Proposition \ref{proposition} that the limit of the probability density function of the corresponding random walk system has a continuous first order derivative at the origin. Hence we are looking for a solution which is $C^1(\R)$. Green's function of this PDE with a discontinuous coefficient has been constructed in Section \ref{sect.explicit} explicitly. Green's function corresponding to the diffusion equation (\ref{ux1}) is obtained by going back to the original variable $x$.


\newpage
\section{Random walk with a spatially non-constant walk length}

In this section we develop a random walk system with a spatial heterogeneity in the walk length $\Delta x$. The time step $\Delta t$ is assumed to be constant in this section. We first introduce the notation. Let spatial grid points $x^k$ be ordered by $x^0=0$ and\footnote{We use the super-indices for the spatial discretization and sub-indices for the temporal one.}
\begin{equation}\label{x^k}
\cdots<x^{k-1}<x^k<x^{k+1}<\cdots \quad\text{with}\quad k\in\Z.
\end{equation}
For a homogeneous random walk system, the distance between two neighboring grid points is constant. In this paper we consider a non-uniform case when $x^{k+1}-x^{k}$ is not necessarily constant. We introduce a partition consisting of odd numbered grid points and its norm:
\begin{equation}\label{pi}
\pi:=\{x^{2k+1}:k\in\Z\},\qquad\|\pi\|:=\sup_{k\in\Z}(x^{2k+1}-x^{2k-1}).
\end{equation}
Let $\beta$ be defined at even numbered grid points as
\begin{equation}\label{betax2k}
\beta(x^{2k}):=x^{2k+1}-x^{2k-1},
\end{equation}
which measures the distance between odd numbered grid points. We will see in the followings that even numbered grid points and odd numbered ones are separated. The theory of this section will be developed in terms of the even numbered grid points and that is the reason for defining $\beta$ at even numbered ones only. Similarly, we define
\begin{equation}\label{bx2k}
b(x^{2k}):=\frac{x^{2k+1}-x^{2k-1}}{\|\pi\|}=\frac{\beta(x^{2k})}{\|\pi\|},
\end{equation}
which measures heterogeneity in the walk length. We assume this ratio is bounded away from zero, i.e, there exists $c>0$ such that
\begin{equation}\label{c}
0<c\le b(x^{2k})\le 1,\qquad k\in\Z.
\end{equation}
We will keep this lower bound $c>0$ uniformly when we take the limit as $\|\pi\|\to0$. We may extend these functions over the real line by defining
\begin{equation}\label{continuation}
\beta(x):=\beta(x^{2k})\ \
 \text{and}\ \ b(x):=b(x^{2k})\quad\text{for}\ \ x^{2k-1}\leq x<x^{2k+1}.
\end{equation}
Notice that $\beta(x)$ is not the walk length, but the sum of two adjacent walk lengths. The reason for this is to use it in computing the probability density as in (\ref{ProbabilityDensity}). If we return to a uniform random walk case with equally distanced grid points, we have $c=1$, $b(x)\equiv 1$, and $\Delta x =\frac{1}{2}\|\pi\|$.

\subsection{Derivation of a PDE model}

Consider a particle randomly walking along the grid points. Let $p_0^k$ be the probability that the particle is to be placed at a grid point $x^k$ and at initial time $t=0$. Similarly, let $p_n^k$ be such probability at time $t_n=n\Delta t$. In a one dimensional random walk system, a particle walks to one of the two adjacent grid points with the same probability. So, for all $n\geq 1$ and $k\in\Z$,
\begin{equation*}\label{pnk}
2p_n^k = p_{n-1}^{k-1} + p_{n-1}^{k+1}.
\end{equation*}
Applying this relation twice, we obtain
\begin{equation}\label{p2n2k}
4p_{2n}^{2k}=2(p_{2n-1}^{2k-1} + p_{2n-1}^{2k+1}) =p_{2n-2}^{2k-2}+2p_{2n-2}^{2k}+p_{2n-2}^{2k+2}.
\end{equation}
This relation shows that the probability at even numbered grid points and odd numbered ones evolve independently. Hence it suffices to consider the case when the particle is initially placed at one of the even numbered grid points by assuming
\begin{equation}\label{p0Odd}
\sum_{k=-\infty}^\infty p_0^k=1\quad\text{and}\quad p^{2k+1}_0=0 \text{~for all $k\in\Z$}.
\end{equation}
Then it holds that for all $n\geq 1$,
\begin{equation*}\label{p0Even}
\sum_{k=-\infty}^\infty p_n^k=1\quad\text{and}\quad p^{2k}_{2n-1}=p^{2k+1}_{2n}=0 \text{~for all $k\in\Z$.}
\end{equation*}
Therefore, if only the even numbered time steps $t_{2n}=2n\Delta t$ are counted, then it is enough to consider the probability at the even numbered grid points only.

Now we derive a diffusion equation that approximates the probability distribution function. First we introduce uniform grid points $y^k := \frac{k}{2}\|\pi\|$, and let $p(y,t)$ be a smooth function that satisfies $p(y^k,t_n)=p^k_n$. Then, by (\ref{p2n2k}),
$$
{p(y^{2k},t_{2n})-p(y^{2k},t_{2n-2})} ={1\over 4} \Big(p(y^{2k-2},t_{2n-2})-2p(y^{2k},t_{2n-2})+p(y^{2k+2},t_{2n-2})\Big).
$$
Divide this equation by $2\Delta t$ and obtain
\begin{equation}\label{FDR}
\begin{split}
{p(y^{2k},t_{2n})-p(y^{2k},t_{2n-2})\over 2\Delta t}
={\|\pi\|^2\over 8\Delta t} {p(y^{2k-2},t_{2n-2})-2p(y^{2k},t_{2n-2})+p(y^{2k+2},t_{2n-2}) \over \|\pi\|^2}.
\end{split}
\end{equation}
The left-hand side of (\ref{FDR}) is a first order forward approximation of $p_t:={\partial\over\partial t}p$ and the right side is a second order central approximation of $p_{yy}:={\partial^2\over\partial y^2}p$. Hence, we may write
\begin{equation}\label{EqnForPinY}
p_t(y,t)+O(\Delta t)=\d\,p_{yy}(y,t)+O(\d\|\pi\|^2),\quad \d:={\|\pi\|^2\over 8\Delta t}.
\end{equation}
To return to the original variable $x$ we change the space variable by using $y(x^{2k+1})=y^{2k+1}$. Then,
\begin{equation}\label{b(x)}
\frac{dx}{dy} \cong \frac{x^{2k+1}-x^{2k-1}}{y^{2k+1}-y^{2k-1}} =\frac{\beta(x^{2k})}{\|\pi\|}=b(x^{2k})
\end{equation}
and the equation (\ref{EqnForPinY}) turns into
\begin{equation*}\label{EqnForPinX}
p_t(x,t)=\d\,b(x) \big( b(x)p_x(x,t) \big)_x+O(\Delta t)+O(\d\|\pi\|^2).
\end{equation*}
Here we abuse some notations for simplicity by writing $p(x,t)$ instead of $p(y(x),t)$. We will use this convention throughout the paper. Let $u(x,t)$ be a smooth function given by
\begin{equation}\label{ProbabilityDensity}
u(x,t):={p(x,t)\over\beta(x)} ={p(x,t)\over \|\pi\|b(x)},
\end{equation}
which is called the \emph{probability density} function. Then, it satisfies
\[
u_t(x,t)=\d \big(b(x)(b(x)u)_x)_x+O(\Delta t/\|\pi\|)+O(\d\|\pi\|).
\]
Since $\d={\|\pi\|^2\over 8\Delta t}$, the remainder terms can be written as
\[
u_t(x,t)=\d\big(b(x)(b(x)u)_x)_x+O(\|\pi\|(\d+{1/\d})\,).
\]
Therefore, it is important to take
\begin{equation}\label{diffusivity}
\d={\|\pi\|^2\over 8\Delta t}=O(1)\quad\mbox{as}\quad\Delta t\to0,
\end{equation}
so that, for $\|\pi\|$ small, the probability density function $u(x,t)$ approximately satisfies a \emph{non-uniform} diffusion equation,
\begin{equation}\label{DiffusionEqn1}
u_t(x,t)=\d\big(b(x)(b(x)u)_x)_x.
\end{equation}

\begin{remark}
The diffusivity of the diffusion equation (\ref{DiffusionEqn1}) at $x\in\R$ is $k(x):=b(x)^2\d$. Since $c\le b(x)\le1$, the diffusivity is in the range between $c^2\d$ and $\d$. Hence we may call $\d$ the maximum diffusivity for a heterogeneous case and the diffusivity for a homogeneous case.
\end{remark}



\subsection{Convergence to the solution of the PDE model}

We have derived the diffusion equation (\ref{DiffusionEqn1}) as an approximation of the difference equation (\ref{FDR}) when the mesh grids $x^k$'s and the jumping time $\Delta t$ are fixed. In the following theorem we will show that the solution of the difference equation converges to the solution of the non-uniform diffusion equation as $\|\pi\|$ and $\Delta t$ vanish. In this approach, the limit should be taken to the probability density function since the probability at each grid point just vanishes as the mesh becomes finer. Furthermore, one cannot take the limit as $\|\pi\|,\Delta t\to0$ arbitrarily; the relation (\ref{diffusivity}) should be satisfied. This kind of convergence is classical with a uniform mesh case (see Lin and Segel \cite[Section 3.3]{MR982711}). Hence the contribution of the following theorem is its extension to a non-uniform random walk system.

In the following theorem we denote the quantities obtained from a limiting process by symbols with a bar such as $\bar u$ or $\bar w$. The quantities that depend on the choice of spatial and temporal mesh grids are denoted without it.
\begin{theorem}\label{thm1} Consider the random walk system $p_n^k$ given in (\ref{x^k})--(\ref{p0Odd}) with a relation $\|\pi\|^2=8\d\Delta t$ for a given constant $\d>0$. Let $u_{2n}^{2k}$ be the probability density of the random particle defined by
$$
u_{2n}^{2k}={p_{2n}^{2k}\over x^{2k+1}-x^{2k-1}}={p_{2n}^{2k}\over\beta(x^{2k})}
$$
and $\bar u$ be the solution of
\begin{equation}\label{DiffusionEqn2}
\bar u_t=\d\big(\bar b(x) (\bar b(x)\,\bar u)_x \big)_x,\quad\bar u(x,0)=\bar f(x).
\end{equation}
If the initial value $\bar f(x)$ is bounded,
\begin{equation}\label{Assumptions}
\sup_k | b(x^{2k})-\bar b(x^{2k}) | \to 0, \quad\text{and}\quad
\sup_k \Big| u^{2k}_0 -\bar f(x^{2k}) \Big| \to 0 \quad \text{as $\Delta t,\|\pi\|\to 0$,}
\end{equation}
then, for any $T>0$,
\[
\sup_{2n\Delta t\leq T,\,k\in\Z} \big| u^{2k}_{2n} - \bar u(x^{2k},t_{2n}) \big| \to 0 \quad \text{as $\Delta t, \|\pi\| \to 0.$}
\]
\end{theorem}

\begin{proof}
Introduce new grid points $y^{2k+1}:=\int_0^{x^{2k+1}} \frac{1}{b(s)}\,ds$ and $y^{2k}=\frac{1}{2}(y^{2k+1}+y^{2k-1})$. Then,
\[
y^{2k+1}-y^{2k-1} =\int_{x^{2k-1}}^{x^{2k+1}}\frac{\|\pi\|}{\beta(s)}\,ds =\int_{x^{2k-1}}^{x^{2k+1}}\frac{\|\pi\|}{x^{2k+1}-x^{2k-1}}\,ds  =\|\pi\|
\]
and hence $\{y^k:k\in\Z\}$ is a uniform mesh grid with a constant walk length $|\Delta y|=\frac{1}{2}\|\pi\|$. Let $p_n^k$ be the probability for a particle to be placed at spatial grid point $y^k$ and at time step $t_n$. Then, we have
\[
p^{2k}_{2n} ={1\over 4}\, \big( p^{2(k-1)}_{2(n-1)} +2p^{2k}_{2(n-1)} +p^{2(k+1)}_{2(n-1)} \big)
\]
as in (\ref{p2n2k}). Since the probability at odd numbered spatial grid points at even numbered time step is zero, we consider even numbered spatial grid points at even numbered time steps only. Then, the probability density $w^{2k}_{2n}$ for this uniform random walk is given by
\[
w^{2k}_{2n}:={p^{2k}_{2n}\over y^{2k+1}-y^{2k-1}} = \frac{p^{2k}_{2n}}{ \|\pi\| }.
\]
Therefore, the finite difference relation for $w^{2k}_{2n}$ is the same as the one for $p^{2k}_{2n}$, i.e.,
\[
w^{2k}_{2n} ={1\over 4}\, \big( w^{2(k-1)}_{2(n-1)} +2w^{2k}_{2(n-1)} +w^{2(k+1)}_{2(n-1)} \big).
\]
Now let $\bar w$ be the solution of
\begin{equation} \label{barw}
\bar w_t =\d\,\bar w_{yy},\qquad\bar w(y,0) =\bar b(x)\bar f(x).
\end{equation}
Here the coordinate $x$ in the initial value is determined by
\begin{equation} \label{relation-y(x)}
y(x):=\int_0^x \frac{1}{\bar b(s)}\,ds.
\end{equation}
Then, by Taylor's theorem and the relation $2\Delta y=\|\pi\|$, we have
\[\begin{split}
\bar{w}^{2k}_{2n} -\bar{w}^{2k}_{2(n-1)} +O(|\Delta t|^2)
=\frac{1}{4} \big( \bar{w}^{2(k-1)}_{2(n-1)} -2\bar{w}^{2k}_{2(n-1)} + \bar{w}^{2(k+1)}_{2(n-1)} \big) +O(\Delta t\|\pi\|^{2}),
\end{split}\]
where $\bar{w}^{2k}_{2n}:=\bar{w}(y^{2k},t_{2n})$. Therefore, the difference $e^{2k}_{2n}:=w^{2k}_{2n}-\bar{w}^{2k}_{2n}$ satisfies
\[
\big| e^{2k}_{2n} \big| = \frac{1}{4} \big| e^{2(k-1)}_{2(n-1)} +2e^{2k}_{2(n-1)} +e^{2(k+1)}_{2(n-1)} \big| +O(|\Delta t|^2)\leq \sup_k \big| e^{2k}_{2(n-1)} \big| + O(|\Delta t|^2).
\]
Repeatedly using this relation, we obtain
\begin{equation} \label{error-between-w's}
\big| w^{2k}_{2n}-\bar{w}^{2k}_{2n} \big| \leq \sup_k \big| w^{2k}_0-\bar{w}^{2k}_0 \big| + nO(|\Delta t|^2).
\end{equation}

Define $\bar u$ as the one satisfying $\bar{w}(y,t) =\bar b(x) \,\bar u(x,t)$, where the coordinates $x$ and $y$ are related by \eqref{relation-y(x)}. Then,
$$
\bar w_t=\bar b(x)\bar u_t,\quad \d\,\bar w_{yy}= \d\,\bar b(x)(\bar b(x)(\bar b(x) \,\bar u)_x)_x,\quad\text{and}\quad \bar u(x,0)=\bar f(x) .
$$
Therefore, this $\bar u$ is the solution of (\ref{DiffusionEqn2}). On the other hand, the probability density $u$ for the non-uniform random walk is
\[
u^{2k}_{2n}:={p^{2k}_{2n}\over x^{2k+1}-x^{2k-1}} = {\|\pi\|w_{2n}^{2k}\over\beta(x^{2k})}={w_{2n}^{2k}\over b(x^{2k})}.
\]
If we set
\[\bar u^{2k}_{2n} :=\bar u(x^{2k},t_{2n}) = {\bar{w}^{2k}_{2n}\over\bar b(x^{2k})}, \]
we have
\[
\begin{split}
\big| u^{2k}_{2n}-\bar{u}^{2k}_{2n} \big|=\Big|{w_{2n}^{2k}\over b(x^{2k})}-{\bar{w}^{2k}_{2n}\over\bar b(x^{2k})}\Big| &\le \Big|{w_{2n}^{2k}\over b(x^{2k})}-{\bar{w}^{2k}_{2n}\over b(x^{2k})}
\Big|+\Big|{\bar w_{2n}^{2k}\over b(x^{2k})}-{\bar{w}^{2k}_{2n}\over\bar b(x^{2k})}  \Big|\ \\
&\le \frac{1}{c} \,\big| w^{2k}_{2n} -\bar{w}^{2k}_{2n} \big| + \frac{\bar{w}^{2k}_{2n}}{c^2} \,\Big| b(x^{2k})-\bar b(x^{2k})  \Big|.
\end{split}
\]
The last term vanishes uniformly in $k$ because of the assumption (\ref{Assumptions}). Let $2n\Delta t<T$. Then, due to \eqref{error-between-w's}, the other term in the last line is bounded by
\[\begin{split}
\frac{1}{c} \,\big| w^{2k}_{2n} -\bar{w}^{2k}_{2n} \big| &\le
\frac{1}{c} \, \sup_k \big| w^{2k}_0 -\bar{w}^{2k}_0 \big| +nO(|\Delta t|^2)= \frac{1}{c} \, \sup_k \big| b(x^{2k})u^{2k}_0 -\bar b(x^{2k})\bar{u}_0(x^{2k}) \big| +O(\Delta t)\\
&\le\frac{1}{c}\sup_k\Big(\big|b(x^{2k})\big|\big|u^{2k}_0-\bar{u}_0(x^{2k})\big| +\big|\bar{u}_0(x^{2k})\big|\,\big|b(x^{2k}) -\bar b(x^{2k})\big|\Big)+O(\Delta t).
\end{split}\]
This bound vanishes uniformly in $k$ as $\Delta t \to 0$ because of the assumption in (\ref{Assumptions}) and the initial condition $\bar u(x,0)=\bar f(x)$. Therefore the proof is complete.
\end{proof}


\subsection{Numerical comparison between the random walk and the PDE model}

In this section we compare the discrete random walk and the solution of the PDE model. First we consider the models using the normalized variable $y$. The PDE for the probability density function $w$ in $y$ variable is given by
\begin{equation}\label{EqnForWy}
w_t=\d \,w_{yy},\quad w(y,0)=w_0(y),\quad -\infty<y<\infty,
\end{equation}
where the initial value is nonnegative and $\d=\frac{|\Delta y|^2}{2\Delta t}$ for the constant walk length $\Delta y>0$ and the jumping time $\Delta t>0$. The solution is simply given by
\begin{equation}\label{ExplicitP}
w(y,t) = \int \phi(y-a;t\d)\, w_0(a)\,da,
\end{equation}
where the heat kernel
\begin{equation*}
\phi(y;t\d):=\frac{1}{\sqrt{4\pi t\d\,}}e^{-y^2/4t\d}
\end{equation*}
is the solution when the initial value is given by the Dirac delta distribution, $p_0=\delta$. If $t\d=1/2$, then $\phi(\cdot;t\d)$ is called the standard normal distribution.  Note that the formula (\ref{ExplicitP}) holds since the heat equation (\ref{EqnForWy}) is autonomous and hence Green's function is simply $G(y,a,t)=\phi(y-a,t)$. In Figure \ref{fig1} the probability density distribution of a random walk, $w^{2k}_{2n}=p^{2k}_{2n}/(2\Delta y)$, is plotted with the standard normal distribution. In this simulation we set $\Delta y=0.1$ and $\Delta t=0.1$. Then the diffusivity becomes $\d=0.05$ and hence $t\d=0.5$ if the final time is $t=10$. The particle was initially placed at the origin, i.e., $p^k_0=\delta_{k0}$, where $\delta_{ij}$ is the Kronecker delta.

\begin{figure}[ht]
\centering
\begin{minipage}[t]{0.48\textwidth}
\centering
\includegraphics[width=\textwidth]{fig1a}

(a) Gaussian and ramdom walk
\end{minipage}
\begin{minipage}[t]{0.48\textwidth}
 \centering
 \includegraphics[width=\textwidth]{fig1b}

(b) Error $=w^{2k}_{2n}-\phi(y^{2k},0.5)$
\end{minipage}
 \caption{{\bf Random walk with a normalized variable.} Probability distribution of discrete random walk is given in (a) with dots. The curve is the standard normal distribution. Even using a small number of grid points gives a good match of the normal distribution. The difference is given in (b). The parameters of this random walk are $\Delta y=0.1,\Delta t=0.1,t=10,$ and $t\d=0.5$.} \label{fig1}
\end{figure}

Next we consider the probability density distribution of a random walk system with a constant jumping time $\Delta t=0.1$ and non-uniform grid points:
\[
x^0=0\quad\text{and}\quad x^k-x^{k-1}=
\begin{cases}
0.1 & \text{if $k\leq 0$,}\\
0.05 & \text{if $k>0$.}
\end{cases}\]
The corresponding diffusion equation is
\begin{equation*}\label{EqnForU2}
u_t=\d \big(b(x)(b(x)u)_x \big)_x,\quad u(x,0)=u_0(x),\quad -\infty<x<\infty,
\end{equation*}
where
$$
\|\pi\|=0.2, \quad \d=\frac{\|\pi\|^2}{8\Delta t}=0.05, \quad b(x^{2k})=
\begin{cases}
1 & \text{if $k<0$,}\\
0.75 & \text{if $k=0$,}\\
0.5 & \text{if $x>0$.}
\end{cases}
$$
This equation is not autonomous so that there is no explicit formula such as (\ref{ExplicitP}). However, if we introduce a new variable,
\[y(x) := \int_0^{x}\frac{1}{b(s)}\,ds,\]
then $\{y^k=y(x^k):k\in\Z\}$ becomes a uniform grid and the probability density $w(y,t)$ in the new variable satisfies (\ref{EqnForWy}), where the corresponding initial value is $w_0(y(x))=b(x)u_0(x)$. Hence
\begin{equation*}\label{000}
u(x,t)={1\over b(x)} \frac{1}{\sqrt{4\pi t\d\,}} \int e^{-(y(x)-z)^2/4t\d} w_0(z)\,dz.
\end{equation*}

\begin{figure}[ht]
\centering
\begin{minipage}[t]{0.48\textwidth}
\centering
\includegraphics[width=\textwidth]{fig2a}

(a) $\Delta x=0.1, x<0$; $\Delta x=0.05, x>0$; $\Delta t=0.1$
\end{minipage}
\begin{minipage}[t]{0.48\textwidth}
 \centering
 \includegraphics[width=\textwidth]{fig2b}

(b) $\Delta x=0.05, x<0$; $\Delta x=0.025, x>0$; $\Delta t=0.1$
\end{minipage}
\caption{{\bf Random walk with a non-constant walk length.} Probability density distribution of the random walk, $u^{2k}_{2n}:=p^{2k}_{2n}/(x^{2k+1}-x^{2k-1})$, is given with the rescaled Gaussian. (a) The maximum diffusivity is $\d=0.05$ and the final time is given by $t\d=0.5$.  (b) The diffusivity is $\d=0.0125$ and the final time is given by $t\d=0.5$. } \label{fig2}
\end{figure}
The heat kernel for this non-autonomous problem is given by a rescaled Gaussian,
\begin{equation*}
\phi_b(x;t\d) := {1\over b(x)}\frac{1}{\sqrt{4\pi t\d\,}}e^{-y(x)^2/4t\d},\quad y(x) := \int_0^{x}\frac{1}{b(s)}\,ds.
\end{equation*}
In Figure \ref{fig2} this rescaled Gaussian and the probability distribution of some discrete random walk systems are compared. These two agree well even with relatively small number of grid points. Also one may observe a discontinuity at the origin which was generated by the discontinuity in the walk length $\Delta x$.

\begin{remark}[Invariance of the median]\label{Answer1}
The probability density function of a random walk system may have different shapes depending on heterogeneity in the walk length. However, the median of the distribution cannot be changed. For example, consider a random walk system that the particle started at the origin. Then, the probability density is given by a rescaled Gaussian and satisfies
\begin{equation}\label{100}
\begin{split}
\int_0^\infty u(x,t)\,dx&=\frac{1}{\sqrt{4\pi t\d\,}}\int_0^\infty {1\over b(x)}e^{-y(x)^2/4t\d}\,dx
=\frac{1}{\sqrt{4\pi t\d\,}}\int_0^\infty e^{-s^2/4t\d}\,ds=0.5.
\end{split}
\end{equation}
Hence the probability for a particle to be placed on the region where $x<0$ is identical to the one where $x>0$. This is the case when there is heterogeneity in $\Delta x$ only. However, we will see in the next section that is not the case if there is heterogeneity in $\Delta t$.
\end{remark}




\section{Random walk with a spatially non-constant travelling time}

In this section we develop a random walk theory with a spatial heterogeneity in the time step $\Delta t$. Since heterogeneity in the walk length could be handled by rescaling the space variable as shown in the previous section, we will focus on a constant walk length case:
\begin{equation*}\label{y^k}
y^k=k|\Delta y|,\qquad k\in\Z.
\end{equation*}
Now we will work with heterogeneity in the time step $\Delta t$, which is the main contribution of this paper. We consider a simplest spatial heterogeneity in $\Delta t$ given by
\begin{equation*}\label{gamma}
\Delta t(y^{k+1/2}):=
\begin{cases}
\tau & \text{if $y^{k+1/2}<0$,}\\
2\tau & \text{if $y^{k+1/2}\geq 0$,}
\end{cases}
\end{equation*}
where $\tau>0$ is a constant and $y^{k+1/2}:=(y^k+y^{k+1})/2$. We will consider $\Delta t(y^{k+1/2})$ as the \emph{travelling time} which it takes for a particle to move from $y^k$ to $y^{k+1}$ or vice versa\footnote{If we consider a non-constant $\Delta t$ as the waiting time for the next jump, there may exist two different particles or probabilities with different jumping moments. This makes a presentation complicate. Hence we consider it as a travelling time and the next walk starts immediately after arrival. Hence, in the region where $y>0$, one should count the probability that the particle does not arrive at a grid point yet.}. As before, we may extend  the domain of the function $\Delta t$ over the real line by defining $\Delta t(y):=\Delta t(y^{k+1/2})$ for $y^k\leq y<y^{k+1}$. Note that the value of $\Delta t$ at a spatial grid point $y^k$ has no meaning in our setting. We will focus on this two-time step random walk in this section. This simple case is actually a challenging one due to the singularity at the origin and can be used as a building block for general cases with arbitrarily heterogeneous time steps.

\subsection{Derivation of a PDE model}

Define
\begin{equation}\label{alpha}
\mu(y):=\frac{\tau}{\Delta t},\quad\text{i.e.,}\quad
\mu(y)=
\begin{cases}
1 & \text{if $y<0$,}\\
1/2 & \text{if $y\geq 0$.}
\end{cases}
\end{equation}
We consider a particle randomly walking along the grid points. Let $p_0^k$ be the probability that the particle is placed at the grid point $y^k$ and at the initial time $t=0$. Similarly, let $p_n^k$ be the probability at time $t_n=n\tau$, $n>0$. Then, the probability $p^k_n$ for $n\ge2$ is computed by
\begin{equation}\label{interior2}
2p_n^k=
\begin{cases}
p_{n-1}^{k-1}+p_{n-1}^{k+1} & \text{if $k<0$,}\\
p_{n-1}^{-1}+p_{n-2}^{1} & \text{if $k=0$,}\\
p_{n-2}^{k-1}+p_{n-2}^{k+1} & \text{if $k>0$.}
\end{cases}
\end{equation}
Applying this relation twice obtain
\[
4p_n^k=
\begin{cases}
p_{n-2}^{k-2}+2p_{n-2}^{k}+p_{n-2}^{k+2} & \text{if $k<-1$,}\\
p_{n-2}^{-3}+2p_{n-2}^{-1}+\boxed{p_{n-3}^1} & \text{if $k=-1$,}\\
p_{n-2}^{-2}+p_{n-2}^{0}+p_{n-4}^{0}+p_{n-4}^{2} & \text{if $k=0$,}\\
\boxed{p_{n-3}^{-1}}+2p_{n-4}^{1}+p_{n-4}^3 & \text{if $k=1$,}\\
p_{n-4}^{k-2}+2p_{n-4}^{k}+p_{n-4}^{k+2} & \text{if $k>1$.}
\end{cases}
\]
Notice that even numbered grid points and odd numbered ones are not separated in this case because of the boxed terms in the second and the fourth cases. Now we repeat the derivation process as in the previous section. Let
$$
w_n^k={p_n^k\over \Delta y}.
$$
Then, the relation in (\ref{interior2}) gives
\begin{equation}\label{w}
2w_n^k=
\begin{cases}
w_{n-1}^{k-1}+w_{n-1}^{k+1} & \text{if $k<0$,}\\
w_{n-1}^{-1}+w_{n-2}^{1} & \text{if $k=0$,}\\
w_{n-2}^{k-1}+w_{n-2}^{k+1} & \text{if $k>0$.}
\end{cases}
\end{equation}
Fix
$$
\d={|\Delta y|^2\over 2\tau}
$$
as a constant $\d>0$ and consider the limit as $\tau\to0$. The argument on convergence in the previous section shows that $w_n^k$ converges to a limit $w(x,t)$ as $\tau\to0$ and satisfies
$$
w_t=\d w_{yy}\quad\text{for $y<0,$}\qquad
w_t={1\over2}\d w_{yy}\quad\text{for $y>0$.}
$$
Using the definition of $\mu(y)$ in \eqref{alpha} we may write it as
\begin{equation}\label{EqnForPinY2}
w_t=\mu(y)\d w_{yy} \qquad \text{for $y\ne0$.}
\end{equation}
This equation has meaning only if $y\ne0$ due to the discontinuity of $\mu$ at the origin $y=0$. Notice that the three cases with $k=-1,0,1$ are forgotten since $\Delta y$ is of microscopic scale, which is also the reason for the disconnected domain of the equation. In the following proposition we show that the limit $w$ is continuously diffrentiable even at the origin, i.e.,
\begin{equation*}\label{continuity}
\lim_{y\to0-}w(y,t_0)=\lim_{y\to0+}w(y,t_0),\qquad
\lim_{y\to0-}w_y(y,t_0)=\lim_{y\to0+}w_y(y,t_0).
\end{equation*}


\begin{proposition}\label{proposition} Let $w_\pm$ be the limit of $w_n^{\pm1}$ as $\tau \to0$ and $n\to\infty$, where the limit is taken under fixed $\d={|\Delta y|^2\over 2\tau}$ and fixed $t_0=n\tau$. Then, $w_-=w_+$. Furthermore, if
$$
\lim_{\tau \to0}{w_n^{1}-w_n^0\over\Delta y}=w_+',\qquad \lim_{\tau \to0}{w_n^{0}-w_n^{-1}\over\Delta y}=w_-',
$$
then $w_-'=w_+'$. Therefore, the limit function $w$ is in $C^1(\R)\cap C^\infty(\R\setminus\{0\})$.
\end{proposition}
\begin{proof} From the relation for $k=-1$ in (\ref{w}) we obtain
$$
4w_n^{-1}=2w_{n-1}^{-2}+2w_{n-1}^{0} =2w_{n-1}^{-2}+w_{n-1}^{-1}+w_{n-2}^{1}.
$$
Taking the limit  as $\tau \to0$ gives
$$
4w_-=2w_-+w_-+w_+.
$$
Therefore, $w_-=w_+$. Next subtract the relation for $k=0$ in (\ref{w}) from the one for $k=-1$ to obtain
\begin{equation}\label{difference-relation-w'}
{2(w^{-1}_n-w_n^0)\over\Delta y} ={w^{-2}_{n-1}-w_{n-1}^{-1}\over\Delta y} +{w^{0}_{n-1}-w_{n-1}^{1}\over\Delta y} +{w^{1}_{n-1}-w_{n-2}^{1}\over\Delta y}.
\end{equation}
The last term vanishes since
\begin{equation}\label{w'}
{w^{1}_{n-1}-w_{n-2}^{1}\over\Delta y}={w^{1}_{n-1}-w_{n-2}^{1}\over2\d\tau}\ {|\Delta y|^2\over\Delta y}\to 0
\end{equation}
as $\tau \to0$. Hence, taking the limit as $\tau\to0$ in \eqref{difference-relation-w'} gives
$$
2w_-'=w_-'+w_+',
$$
or $w_-'=w_+'$.
\end{proof}

\begin{remark}
Notice that the vanishing limit in (\ref{w'}) is invalid for second order derivative since ${\tau\over|\Delta y|^2}$ does not vanish. Therefore continuity in the proposition is valid only up to the first order derivative and $w_{yy}$ has a discontinuity at $y=0$.
\end{remark}

Next we consider the probability density function that includes the probability for a particle to be placed between grid points, but not at a grid point. Notice that $w_n^k$ is the probability density that a particle arrives at $y^k$ at time $t_n=n\tau$. However, the travelling time in the region where $y>0$ is $2\tau$ and hence there is a chance for a particle to be in the middle of its way. Including such a chance, the probability \emph{density} function is given by
\begin{equation*}\label{density}
v(y,t)=
\begin{cases}
w(y,t) & \text{if $y<0$,}\\
2w(y,t) & \text{if $y>0$,}
\end{cases}
\quad \text{or}\quad v(y,t)=\frac{1}{\mu(y)}w(y,t).
\end{equation*}
This relation can be justified only when the probability at grid points and the probability in the middle of the way are identical in the region where $y>0$. They are almost identical if $\Delta t$ and $\Delta x$ are small enough. Substitute $w=\mu(y)v$ into (\ref{EqnForPinY2}) and obtain
\begin{equation*}\label{EqnForUinY2}
v_t=\d (\mu v)_{yy},\qquad y\in\bfR,\ \d:={|\Delta y|^2\over 2\tau}.
\end{equation*}
Now we change the variable using (\ref{relation-y(x)}) as in the previous section. Then the density becomes $u(x,t)=v(y(x),t)/b(x)$ and satisfies
\begin{equation}\label{EqnForUinX2}
u_t=\d (b(b\mu u)_{x})_{x},\qquad x\in\bfR,\ \d:={\|\Delta x\|^2\over 2\tau}.
\end{equation}
Remember that $w(y,t)=b(x(y))\mu(y)u(x(y),t)$ is the relation between $w$ and $u$ and it is $w$ that has $C^1(\R)$ regularity.

\subsection{Green's function for the two-time step PDE model} \label{sect.explicit}

The purpose of this section is to find Green's function for the diffusion model (\ref{EqnForUinX2}) in an explicit form. For simplicity we consider the case when there is no heterogeneity in $\Delta x$, i.e., $b(x)\equiv 1$, and the diffusivity is $\d=1$. Going back to (\ref{EqnForPinY2}), we will actually find  $G=G(y,a,t)$ that solves
\[\left\{
\begin{aligned}
&G_t= \mu(y) \,G_{yy},\\
&G(y,0)=\mu(y) \,\delta(y-a).
\end{aligned}
\right.\]
In the followings we will look for Green's function such that $G(\cdot,a,t)\in C^1(\R)\cap C^\infty(\R\setminus\{0\})$ for each $a\in\bfR$ and $t>0$ since a meaningful solution satisfies this regularity as shown in Proposition \ref{proposition}.


\subsubsection{When $a=0$.}
In this section, we will find $w(y,t):=G(y,0,t)$ explicitly that solves
\begin{equation}\label{w0}
\left\{
\begin{aligned}
&w_t= \mu(y) \,w_{yy},\\
&w(y,0)=\mu(y) \,\delta(y).
\end{aligned}
\right.
\end{equation}
It is tricky to find a solution due to the discontinuity of the coefficient $\mu$ at the origin. We employ a similar technique used in \cite{Chung20142520}. First, we split the real line $\textbf{R}$ into two regions, $\{ y>0 \}$ and $\{ y<0 \}$. In the region $\{y>0\}$, we solve an initial-boundary value problem:
\[
\begin{cases}
w_t = \frac{1}{2} w_{yy} & \text{if $y > 0, ~t > 0$,} \\
w(y,0) = \frac{1}{2} f(y) & \text{if $y > 0$,} \\
w(0,t) = g(t) & \text{if $t > 0$,}
\end{cases}
\]
where the value $g(t)$ at the boundary $y=0$ is unknown and will be determined later. Notice that we have replaced the initial data $\delta$ by an arbitrary function $f(y)$ for now. This will be replaced by a Dirac delta sequence and the unknown boundary value $g$ will be decided from a limiting process. One may find the solution to the initial-boundary value problem in \cite[p18 and p22]{MR1728947}, which is
\[ \begin{split}
w(y,t) &= \frac{1}{\sqrt{2 \pi t}} \int_0^\infty \frac{1}{2} f(\xi) \big( e^{-(\xi-y)^2/2t} - e^{-(\xi+y)^2/2t} \big)\,d\xi \\
& \qquad + \int_0^t g'(\eta) \,\erfc\Big( \frac{y}{\sqrt{2(t-\eta)}} \Big) \,d\eta + g(0) \,\erfc\Big( \frac{y}{\sqrt{2t}} \Big) \quad \text{if $y > 0$.}
\end{split} \]
For $y < 0$, $w$ satisfies
\[
\begin{cases}
w_t = w_{yy} & \text{if $y<0, ~t > 0$,} \\
w(y,0) = f(y) & \text{if $y < 0$,} \\
w(0,t) = g(t) & \text{if $t > 0$.}
\end{cases}
\]
Notice that, by choosing the same boundary value $g(t)$, the solution $w$ is automatically continuous. One can similarly find the solution
\[ \begin{split}
w(y,t) &= \frac{-1}{2\sqrt{\pi t}} \int_0^\infty f(-\xi) \big( e^{-(\xi-y)^2/4t} - e^{-(\xi+y)^2/4t} \big) \,d\xi \\
& \qquad + \int_0^t g'(\tau) \,\erfc\Big( \frac{-y}{2\sqrt{t-\tau}} \Big) + g(0) \,\erfc\Big( \frac{-y}{2\sqrt{t}} \Big) \quad \text{if $y < 0$.}
\end{split} \]

To determine $g(t)$, we use the assumption that the solution $w$ is $C^1(\bfR)$ in the $y$ variable, i.e.,
\[ w_y(0+,t) = w_y(0-,t) \qquad \text{for all $t > 0$.} \]
Since
\[w_y(y,t)=
\begin{cases}
\begin{split}
& \textstyle \frac{1}{2t \sqrt{2\pi t}} \int_0^\infty f(\xi) \big( (\xi-y) e^{-(\xi-y)^2/2t} + (\xi+y) e^{-(\xi+y)^2/2t} \big) \,d\xi \\
& \textstyle \qquad - \sqrt{\frac{2}{\pi}} \int_0^t g'(\tau) \, \frac{e^{-y^2/2(t-\tau)}}{\sqrt{t-\tau}} \,d\tau - \sqrt{\frac{2}{\pi t}} \,g(0) \,e^{-y^2/2t}
\end{split}
& \text{if $y>0$,} \\
\begin{split}
& \textstyle \frac{-1}{4t \sqrt{\pi t}} \int_0^\infty f(-\xi) \big( (\xi-y) e^{-(\xi-y)^2/4t} + (\xi+y) e^{-(\xi+y)^2/4t} \big) \,d\xi \\
& \textstyle \qquad + \frac{1}{\sqrt{\pi}} \int_0^t g'(\tau) \,\frac{e^{-y^2/4(t-\tau)}}{\sqrt{t-\tau}} \,d\tau + \frac{1}{\sqrt{\pi t}} \,g(0) \,e^{-y^2/4t}
\end{split}
& \text{if $y<0$,}
\end{cases}\]
the continuous differentiability of $w$ implies that
\[ \begin{split}
& \frac{1}{t \sqrt{2t}} \int_0^\infty f(\xi) \,\xi e^{-\xi^2/2t}\,d\xi + \frac{1}{2t \sqrt{t}} \int_0^\infty f(-\xi) \, \xi e^{-\xi^2/4t} \,d\xi \\
& \qquad\qquad = (\sqrt{2}+1) \int_0^t g'(\tau) \, \frac{1}{\sqrt{t-\tau}} \,d\tau + (\sqrt{2} + 1) \,\frac{g(0)}{\sqrt{t}} \quad \text{for all $t > 0$.}
\end{split} \]
This relation gives the boundary value $g(t)$ implicitly for any given initial value $f(y)$. For Green's function case, the corresponding initial value is the Dirac delta distribution and we may find $g$ explicitly. First, take a Dirac delta sequence in the place of the initial value: for example, $f(y) = \frac{n}{2} \,\chi_{[-1/n,1/n]}(y)$. Then
\[ \begin{split}
&\frac{n}{2\sqrt{2t}} (1 - e^{-\frac{1}{2n^2 t}}) + \frac{n}{2\sqrt{t}} (1 - e^{-\frac{1}{4n^2 t}}) \\
& \qquad\qquad = (\sqrt{2} + 1) \int_0^t g'(\tau) \,\frac{1}{\sqrt{t-\tau}} \,d\tau + (\sqrt{2} + 1) \, \frac{g(0)}{\sqrt{t}}.
\end{split} \]
Taking the Laplace transformation yields
\[ \begin{split}
& \frac{n}{2\sqrt{2}} \sqrt{\frac{\pi}{s}} (1 - e^{-\sqrt{2s}/n}) + \frac{n}{2} \sqrt{\frac{\pi}{s}} (1 - e^{-\sqrt{s}/n} ) \\
& \qquad = (\sqrt{2} + 1) (s \mathcal{G} (s) - g(0)) \sqrt{\frac{\pi}{s}} + (\sqrt{2} + 1) \,g(0) \sqrt{\frac{\pi}{s}}\ ,
\end{split} \]
or
\[ \mathcal{G} (s) = \frac{n}{2(\sqrt{2}+1) s} \Big( \frac{1}{\sqrt{2}} (1 - e^{-\sqrt{2s}/n}) + (1 - e^{-\sqrt{s}/n}) \Big),\]
and the inverse Laplace transform gives
\[ g(t) = \frac{n}{2} (\sqrt{2} - 1) \Big( \frac{1}{\sqrt{2}} \erf\big( \frac{\sqrt{2}}{2n\sqrt{t}} \big) + \erf \big( \frac{1}{2n \sqrt{t}} \big) \Big). \]
Therefore, the boundary condition $g(t)$ for Green's function is obtained by taking the limit as $n \to \infty$:
\[ g(t) = \frac{\sqrt{2} - 1}{\sqrt{\pi t}}. \]
Therefore the solution $w$ of (\ref{w0}) is given by
\[w(y,t)=\begin{cases}
\frac{\sqrt{2}-1}{\sqrt{2} \pi} \int_0^t \frac{y e^{-y^2/2(t-\tau)}}{\sqrt{\tau} (t-\tau)^{3/2}} \,d\tau = \frac{2 (\sqrt{2}-1)}{\pi} \int_{y/\sqrt{2t}}^\infty \frac{e^{-\eta^2}}{\sqrt{t - y^2/2\eta^2}} \,d\eta & \text{if $y > 0$,} \\
\frac{\sqrt{2}-1}{2\pi} \int_0^t \frac{(-y) e^{-y^2/4(t-\tau)}}{\sqrt{\tau} (t-\tau)^{3/2}} \,d\tau = \frac{2 (\sqrt{2}-1)}{\pi} \int_{-y/2\sqrt{t}}^\infty \frac{e^{-\eta^2}}{\sqrt{t-y^2/4\eta^2}}\,d\eta & \text{if $y < 0$.}
\end{cases}\]
Using the definition of $\mu(y)$ in \eqref{alpha}, Green's function $G(y,a=0,t)=w(y,t)$ can be written in a combined form
\begin{equation}\label{G0}
G(y,a=0,t)=\frac{2(\sqrt{2}-1)}{\pi} \int_{|y|/\sqrt{4\mu(y)t}}^\infty \frac{e^{-\eta^2}}{\sqrt{t-y^2/4\mu(y)\eta^2}}\,d\eta,\qquad y\in\R.
\end{equation}



\subsubsection{When $a\ne0$}


Now we move the initial Dirac delta distribution from the origin to some other point. Then the corresponding equation becomes
\[ \left\{ \begin{aligned}
&w_t = \mu(y) \,w_{yy}, \\
&w(y,0) = \mu(y) \delta(y-a),
\end{aligned} \right. \]
where $a\ne0$. One may solve this problem in the same manner by dividing the spatial domain into two parts,  $\{y>0\}$ and $\{y<0\}$. For the case when $a > 0$, the corresponding boundary value is
\[
g(t)= \frac{\sqrt{2} - 1}{\sqrt{\pi t}} \,e^{-a^2/2t}
\]
and Green's function is
\begin{equation}\label{G+}
G(y,a>0,t):=
\begin{cases}
\begin{split}
& \textstyle \frac{2(\sqrt{2}-1)}{\pi} \int_{|y|/\sqrt{2 t}}^\infty \exp\Big(- \frac{a^2/2}{t - y^2/2\eta^2}\Big)  \, \frac{e^{-\eta^2}}{\sqrt{t - y^2/2\eta^2}} \,d\eta\\
& \textstyle \qquad + \frac{1}{2\sqrt{2\pi t}} \big( e^{-(y-a)^2/2t} - e^{-(y+a)^2/2t} \big)
\end{split}
& \text{if $y>0$,} \\
\frac{2(\sqrt{2}-1)}{\pi} \int_{|y|/2\sqrt{t}}^\infty \exp\Big( -\frac{a^2/2}{t-y^2/4\eta^2} \Big) \, \frac{e^{-\eta^2}}{\sqrt{t - y^2/4\eta^2}} \,d\eta & \text{if $y < 0$.}
\end{cases}
\end{equation}
For the case when $a < 0$, we have
\[ g(t) = \frac{\sqrt{2} - 1}{\sqrt{\pi t}} \,e^{-a^2/4t} \]
and
\begin{equation}\label{G-}
G(y,a<0,t)=\begin{cases}
\frac{2(\sqrt{2}-1)}{\pi} \int_{|y|/\sqrt{2t}}^\infty \exp\Big( -\frac{a^2/4}{t-y^2/2\eta^2} \Big) \, \frac{e^{-\eta^2}}{\sqrt{t - y^2/2\eta^2}} \,d\eta & \text{if $y > 0$,} \\
\begin{split}
& \textstyle \frac{2(\sqrt{2}-1)}{\pi} \int_{|y|/2\sqrt{t}}^\infty \exp\Big( - \frac{a^2/4}{t - y^2/4\eta^2} \Big) \, \frac{e^{-\eta^2}}{\sqrt{t - y^2/4\eta^2}} \,d\eta \\
& \textstyle \quad + \frac{1}{2\sqrt{\pi t}} \big( e^{-(y-a)^2/4t} - e^{-(y+a)^2/4t} \big)
\end{split}
& \text{if $y < 0$.}
\end{cases}
\end{equation}
Note that the formulas for $a > 0$ and $a < 0$ are almost identical; especially, the integrands for the case when $a < 0$ is achieved just by simply replacing `$a^2/2$' in the exponent for the case when $a > 0$ by `$a^2/4$'.



\subsection{Numerical comparison between the random walk and the PDE model}

We compare the probability density function of the two-time step discrete random walk system and the explicit Green's function in (\ref{G0}). In Figure \ref{fig3} Green's function and the probability distribution of the discrete random walk system are compared. One may observe a discontinuity at the origin which comes from the discontinuity in the travelling time $\Delta t$. These two agree well even with a small number of grid points. The both cases with $\Delta x=0.1$ or $\Delta x=0.05$ are given in the figures.


\begin{figure}[ht]
\centering
\begin{minipage}[t]{0.49\textwidth}
 \centering
 \includegraphics[width=\textwidth]{fig3a}

(a) $\Delta t=0.1, x<0$, $\Delta t=0.2, x>0$, $\Delta x=0.1$
\end{minipage}
\begin{minipage}[t]{0.49\textwidth}
\centering
 \includegraphics[width=\textwidth]{fig3b}
(b) $\Delta t=0.1, x<0$, $\Delta t=0.2, x>0$, $\Delta x=0.05$
\end{minipage}
\caption{{\bf Random walk with a non-constant time step.} Probability distribution of the random walk is given with the explicit solution $u=w/\mu$, where $w$ is given by (\ref{ExplicitP}). The time step is $\Delta t=0.1$ for $x<0$ and  $\Delta t=0.2$ for $x>0$. The diffusivity is $\d=0.05$ on $x<0$ and the final time is given by a relation $t\d=0.5$.} \label{fig3}
\end{figure}
Notice difference between Figure \ref{fig2} and Figure \ref{fig3}. The probability density distribution in Figure \ref{fig2} are basically obtained by gluing two normal distributions of different variations. Hence the probability of one side, $P\{x<0\}$, is the same as the other side $P\{x>0\}$. However, the profiles in Figure \ref{fig3} do not follow the Gaussian distribution and even the probability for $x<0$ is different from the one of the other side, i.e., $P\{x<0\}\ne P\{x>0\}$, which will be shown in the following section.


\section{Conclusion and discussion}

Under presence of spatial heterogeneity in the walk length $\Delta x$ and in the jumping time $\Delta t$, random walk systems and the corresponding PDE models have been studied. Thermal diffusion of Brownian particles or starvation driven dispersal of biological organisms are possible applications. In this paper we have considered essential differences among uniform and non-uniform random walk systems. We also observed that the spatial heterogeneity in the jumping time $\Delta t$ has a profound difference from the one in $\Delta x$ and they are unexchangeable.

\subsection{Numerical comparison with Fick's law}

The diffusivity is given by the relation $k=\frac{|\Delta x|^2}{2|\Delta t|}$ in one space dimension. For the case with a constant jumping time interval $\Delta t$, the diffusivity can be written as $k=\d\,b(x)^2$ using the notation in the diffusion model (\ref{DiffusionEqn1}). The original Fick's law \cite{Fick1855} is for the constant diffusivity case. For a non-constant diffusivity $k=k(x)$, Fick's diffusion law usually refers to
\begin{equation}\label{FicksLaw}
u_t=(k(x) u_x)_x
\end{equation}
which is different from the diffusion model (\ref{DiffusionEqn1}) we have derived in this paper. We will numerically compare the solutions of these diffusion equations to the probability density distribution of the discrete random walk system. For numerical simulations we take an initial value
\[
u_0(x)=
\begin{cases}
1 & \text{if $-2<x<2$,}\\
0 & \text{otherwise.}
\end{cases}
\]
The time step is fixed with $\Delta t=0.1$. We consider two sets of random walk grids:
\begin{equation}\label{betaEx1}
x^k=
\begin{cases}
~0.1k & \text{if $k\le0$,}\\
0.05k & \text{if $k>0$,}
\end{cases}
\end{equation}
and
\begin{equation}\label{betaEx2}
x^0=0,\qquad
x^k=
\begin{cases}
x^{k-1}+0.08\times(1.5+\sin(x^{k-1})) & \text{if $k>0$,}\\
x^{k+1}-0.08\times(1.5+\sin(x^{k+1})) & \text{if $k<0$.}
\end{cases}
\end{equation}
There is a discontinuity in the variation of the walk length in the case of (\ref{betaEx1}). The other case of (\ref{betaEx2}) has gradually changing walking length.

For the above random walk systems one may easily compute the diffusivity $k(x)$ and the coefficient $b(x)$ in (\ref{b(x)}). In Figure \ref{fig4} solutions of these two diffusion models are given with the probability density distribution of the discrete random walk system. From these examples one may find that the solutions of the new diffusion law (\ref{DiffusionEqn1}) gives the probability density distribution of the discrete random walk system correctly. However, Fick's law gives something else. In conclusion the new diffusion law (\ref{DiffusionEqn1}) is the one that explains the non-uniform random walk system.

\begin{figure}[ht]
\begin{minipage}[t]{0.49\textwidth}
 \centering
 \includegraphics[width=\textwidth]{fig4a}
(a) Discontinuous walk length in (\ref{betaEx1})
\end{minipage}
\begin{minipage}[t]{0.49\textwidth}
\centering
 \includegraphics[width=\textwidth]{fig4b}
(b) Continuous walk length in (\ref{betaEx2})
\end{minipage}
\caption{{\bf Comparison with Fick's law.} Final time is $t=40$. $\Delta t=0.1$. It is the solution of the new diffusion law (\ref{DiffusionEqn1}) that shows the correct probability density distribution of discrete random systems, but not Fick's law (\ref{FicksLaw}).} \label{fig4}
\end{figure}



\subsection{Sensibility of $\Delta t$ versus insensibility of $\Delta x$}\label{sect.Answer2}
Let $w(y,t)=G(y,0,t)$ be Green's function with a Dirac delta distribution placed at $a=0$, which is given by the formula (\ref{G0}),
\begin{equation*}
w(y,t)=\frac{2(\sqrt{2}-1)}{\pi} \int_{|y|/\sqrt{4\mu(y)t}}^\infty \frac{e^{-\eta^2}}{\sqrt{t-y^2/4\mu(y)\eta^2}}\,d\eta,\qquad y\in\R.
\end{equation*}
By Fubini's theorem, we may observe that
\begin{align*}
\int_0^\infty w(y,t)\,dy &= \frac{\sqrt{2}-1}{\sqrt{2} \pi} \int_0^t \frac{1}{\sqrt{t-\tau} \,\tau^{3/2}} \int_0^\infty y e^{-y^2/2\tau} \,dy \,d\tau \\
&= \frac{\sqrt{2}-1}{\sqrt{2} \pi} \int_0^t \frac{1}{\sqrt{t-\tau} \,\sqrt{\tau}} \,d\tau = \frac{\sqrt{2} - 1}{\sqrt{2}}.
\end{align*}
We similarly compute
$$
\int_{-\infty}^0 w(y,t)\,dx = \sqrt{2} - 1.
$$
The probability density function in $y$ variable is given by $v(y,t)=w(y,t)/\mu(y)$ and satisfies
\begin{equation}\label{101}
\sqrt{2}\int_{-\infty}^0 v(y,t)\,dy =\int_0^\infty v(y,t)\,dy.
\end{equation}
The relation (\ref{101}) implies that the probability for a particle to be placed on $x>0$ is $\sqrt{2}$ times greater than the one on $x<0$. Remember that, for the case with spatial heterogeneity in walk length only, the two probabilities are identical as shown in (\ref{100}). One may clearly observe this phenomenon by comparing Figure \ref{fig2} and Figure \ref{fig3}.

In conclusion we may say that people in the region where $x<0$ may sense a change in jumping time $\Delta t$ of the other side where $x>0$. However, they cannot sense a change in walk length $\Delta x$. This observation implies that heterogeneity in time step cannot be reduced to one in walk length through any kinds of rescaling.


%
%
%\providecommand{\bysame}{\leavevmode\hbox to3em{\hrulefill}\thinspace}
%\providecommand{\MR}{\relax\ifhmode\unskip\space\fi MR }
%% \MRhref is called by the amsart/book/proc definition of \MR.
%\providecommand{\MRhref}[2]{%
%  \href{http://www.ams.org/mathscinet-getitem?mr=#1}{#2}
%}
%\providecommand{\href}[2]{#2}
%\begin{thebibliography}{10}
%
%\bibitem{Braun2004}
%D. Braun and A. Libchaber, \emph{Thermal force approach to molecular
%  evolution}, Phys. Biol. \textbf{1}, 1--8 (2004)
%
%\bibitem{Chapman1928}
%S. Chapman, \emph{On the {Brownian} displacements and thermal diffusion of
%  grains suspended in a non-uniform fluid}, Proc. Roy. Soc. Lond. A
%  \textbf{119}, 34--54 (1928)
%
%\bibitem{MR3050058}
%E. Cho and Y.-J. Kim, \emph{Starvation driven diffusion as a survival
%  strategy of biological organisms}, Bull. Math. Biol. \textbf{75},
%  845--870 (2013)
%
%\bibitem{ChoiKim}
%S. Choi and Y.-J. Kim, \emph{Chemotactic traveling waves by the metric
%  of food}, preprint (2014)
%
%\bibitem{Chung20142520}
%J. Chung, Y.-J. Kim, and M. Slemrod, \emph{An explicit solution
%  of Burgers equation with stationary point source}, J. Differ.
%  Equations \textbf{257}, 2520--2542 (2014)
%
%\bibitem{Duhr2006}
%S. Duhr and D. Braun, \emph{Why molecules move along a temperature
%  gradient}, Proc. Natl. Acad. Sci. USA \textbf{103}, 19678--19682 (2006)
%
%\bibitem{Einstein1905}
%A. Einstein, \emph{\"{U}ber die von der molekularkinetischen theorie
%  derw\"{a}rme geforderte bewegung von in ruhenden fl\"{u}ssigkeiten
%  suspendierten teilchen (on the movement of small particles suspended in a
%  stationary liquid demanded by the kinetic molecular theory of heat.)}, Ann.
%  Phys. \textbf{17}, 549--560 (1905)
%
%\bibitem{Einstein1906}
%A. Einstein, \emph{\emph{Zur theorie der brownschen bewegung} (on the theory of
%  brownian movement)}, Ann. Phys. \textbf{19}, 371--381 (1906)
%
%\bibitem{Eslamian2009}
%M. Eslamian and M. Z. Saghir, \emph{A critical review of thermodiffusion
%  models: Role and significance of the heat of transport and the activation
%  energy of viscous flow}, J. Non-Equilib. Thermodyn. \textbf{34}, 97--131 (2009)
%
%\bibitem{Fick1855}
%A. Fick, \emph{\"{U}ber diffusion ({O}n diffusion)}, Poggendorff's. Annalen.
%  \textbf{94}, 59--86 (1855)
%
%\bibitem{Harstad2009}
%K. Harstad, \emph{Modeling the Soret effect in dense media mixtures},
%  Ind. Eng. Chem. Res. \textbf{48}, 6907--6915 (2009)
%
%\bibitem{Huang2010}
%F. Huang, P. Chakraborty, C. C. Lundstrom, C. Holmden, J. J. G. Glessner, S. W.
%  Kieffer, and C. E. Lesher, \emph{Isotope fractionation in silicate melts by
%  thermal diffusion}, Nature \textbf{464}, 396--400 (2010)
%
%\bibitem{MR1728947}
%J. Kevorkian, Partial differential equations (2nd ed.), Texts in
%  Applied Mathematics, vol.~35, Springer-Verlag, New York (2000)
%
%\bibitem{MR3128024}
%Y.-J. Kim, O. Kwon, and F. Li, \emph{Evolution of dispersal toward
%  fitness}, Bull. Math. Biol. \textbf{75}, 2474--2498 (2013)
%
%\bibitem{MR3189110}
%Y.-J. Kim, O. Kwon, and F. Li, \emph{Global asymptotic stability and the ideal free distribution in a starvation driven diffusion}, J. Math. Biol. \textbf{68},  1341--1370 (2014)
%
%\bibitem{MR982711}
%C. C. Lin and L. A. Segel, Mathematics applied to deterministic problems
%  in the natural sciences (2nd ed.), Classics in Applied Mathematics,
%  vol.~1, Society for Industrial and Applied Mathematics (SIAM), Philadelphia,
%  PA (1988)
%
%\bibitem{Ludwig1856}
%C. Ludwig, \emph{Diffusion zwischen ungleich erw\"{a}rmten orten gleich zusammengesetzter l\"{o}sungen} (Diffusion of homogeneous fluids between regions of different temperature), Sitz. Ber. Akad. Wiss. Wien Math-Naturw. Kl. \textbf{20}, 539 (1856)
%
%\bibitem{14}
%J. C. Maxwell, \emph{On stresses in rarefied gases arising from inequalities of
%  temperature}, Phil. Trans. R. Soc. Lond. \textbf{170}, 231--256 (1879)
%
%\bibitem{MR1895041}
%A. Okubo and S. A. Levin, Diffusion and ecological problems: modern
%  perspectives (2nd ed.), Interdisciplinary Applied Mathematics, vol.~14,
%  Springer-Verlag, New York (2001)
%
%\bibitem{Putnam2007}
%S. A. Putnam, D. G. Cahill, and G. C. L. Wong, \emph{Temperature dependence of
%  thermodiffusion in aqueous suspensions of charged nanoparticles}, Langmuir
%  \textbf{23}, 9221--9228 (2007)
%
%\bibitem{Skellam72}
%J. G. Skellam, Some phylosophical aspects of mathematical modelling in
%  empirical science with special reference to ecology, Mathematical Models in
%  Ecology, Blackwell Sci. Publ., London (1972)
%
%\bibitem{Skellam73}
%J. G. Skellam, The formulation and interpretation of mathematical models of
%  diffusionary processes in population biology, The mathematical theory of the
%  dynamics of biological populations, Academic Press, New York (1973)
%
%\bibitem{Soret1879}
%Ch. Soret, \emph{Sur l'\'{e}tat d'\'{e}quilibre que prend, au point de vue de sa
%concentration, une dissolution saline primitivement homog\`{e}ne, dont deux parties sont port\'{e}es \`{a} des temp\'{e}ratures diff\'{e}rentes}, Archives de Gen\`{e}ve \textbf{3}, 48--61 (1879)
%
%\bibitem{Srinivasan2011}
%S. Srinivasan and M. Z. Saghir, \emph{Experimental approaches to study
%  thermodiffusion - a review}, Int. J. Therm. Sci. \textbf{50}, 1125--1137 (2011)
%
%\bibitem{YoonKim}
%C. Yoon and Y.-J. Kim, \emph{Bacterial chemotaxis without gradient-sensing}, J. Math. Biol. (2014). doi:10.1007/s00285-014-0790-y
%\end{thebibliography}
%
%\end{document}



%%%%%%%%%%%%%%%%%%%%%%%%%%%%%%%%%%%%%%%%
%\bibliographystyle{unsrt}
\bibliographystyle{amsplain}
\bibliography{ChungKim}


\end{document}
--

\begin{thebibliography}{99}



\bibitem{DuhrBraun2006}
S. Duhr and D. Braun,
\newblock \emph{Why molecules move along a temperature gradient},
\newblock Proc. Natl. Acad. Sci. USA \textbf{103} (2006), pp. 19678--82.



\bibitem{Enskog1912}
D. Enskog,
\newblock \emph{Zur Elektronentheorie der Dispersion und Absorption der Metalle} (For the electron theory of dispersion and absorption of metals),
\newblock Ann. Phys. \textbf{38} (1912), pp. 731--763.


\bibitem{vanKampen1988}
N.G. van Kampen,
\newblock \emph{Diffusion in inhomogeneous media},
\newblock J. Phys. Chem. Solids \textbf{49} (1988), pp. 673--677.



\bibitem{Milligen2005}
B. Ph. van Milligen, P. D. Bons, B.A. Carreras and R. S\'{a}nchez,
\newblock \emph{On the applicability of Fick’s law to diffusion in inhomogeneous systems},
\newblock Eur. J. Phys. \textbf{26} (2005), 913--925.


\bibitem{Philibert2006}
J. Philibert,
\newblock \emph{One and a half century of diffusion: Fick, Einstein,
before and beyond},
\newblock Diffusion Fundamentals \textbf{4} (2006), pp. 6.1--6.19.

\bibitem{Selmeczi2007}
D. Selmeczi, S.F. Toli\'{c}-N{\o}rrelykke, E. Sch\"{a}ffer, P.H. Hagedorn, S. Mosler, K. Berg-S{\o}rensen, N.B. Larsen and H. Flyvbjerg,
\newblock \emph{Brownian motion after Einstein and Smoluchowski: some new applications and new experiments},
\newblock Acta Phys. Pol. B \textbf{38} (2007), pp. 2407--31.






\end{thebibliography}




\end{document}
