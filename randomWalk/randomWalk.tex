\documentclass[a4paper,11pt]{article}

\usepackage[margin=3cm]{geometry}
\usepackage{setspace}
\onehalfspacing
%\doublespacing
%\usepackage{authblk}
\usepackage{amsmath}
\usepackage{amssymb}
\usepackage{amsthm}
% \usepackage{calrsfs}
%\usepackage[notcite,notref]{showkeys}

\usepackage{psfrag}
\usepackage{graphicx,subfigure}
\usepackage{color}
\def\red{\color{red}}
\def\blue{\color{blue}}
%\usepackage{verbatim}
% \usepackage{alltt}
%\usepackage{kotex}



\usepackage{enumerate}


%%%%%%%%%%%%%%%%%%
\def\BO{{\mathcal{O}}}
\def\lio{{\mathcal{o}}}
\def\t{{\tau}}
% \def\l{{\|\pi\|}}
\def\l{{h}}
\def\Lip{{\mathrm{Lip}}}


\newcommand{\tcr}{\textcolor{red}}
\newcommand{\tcb}{\textcolor{blue}}
\newcommand{\ubar}[1]{\text{\b{$#1$}}}
\newtheorem{theorem}{Theorem}
\newtheorem{lemma}{Lemma}[section]
\newtheorem{proposition}{Proposition}[section]
\newtheorem{corollary}{Corollary}[section]
\newtheorem{definition}{Definition}[section]
\newtheorem{claim}{Claim}

\newcounter{mycounter}
\newtheorem{step}{Step}[mycounter]

\theoremstyle{remark}
\newtheorem{remark}{Remark}[section]


%%%%%%%%%%%%%%%%%%%%%%%%%%%%%%%%%%%%%%%%%%%%%%%%%%%%%%%%%%
\begin{document}
\title{A note on a heterogeneous random walk model}
\date{}

\maketitle




\subsection{Notations}
\begin{align*}
 t &: \text{macroscopic time coordinate}\\
 y &: \text{macroscopic spatial coordinate}\\
 \Delta t(y) &: \text{a function for the heterogeneous random walk time interval}\\
 \Delta y &: \text{random walk length and {\blue put to be constant}}\\
 \tau &: \text{typical time scale of random walk system}\\
 h &: \text{typical length scale of random walk system}\\
 d \triangleq \frac{\l^2}{2\tau} &: \text{typical diffusion scale of random walk system}\\
 \alpha(y) \triangleq \frac{\Delta t(y)}{\tau} &: \text{non-dimensional microscopic random walk time interval}\\
 & \text{({\red note $\alpha(y)$ was reversly defined in the paper!})}\\
 \beta(y) \triangleq \frac{\Delta y}{\l} = 1 &: \text{non-dimensional microscipic random walk length which is put to be $1$.}\\
 p^k_n &: \text{probability at space grid point $k$ and time grid point $n$}\\
 v(t,y) &: \text{a density function for a continuum model}\\
 w(t,y) &: \text{a conjugate variable to $v(t,y)$}\\
\end{align*}

\subsection{Fully general continuum model}
This shall be
\begin{align*}
 u_t &= \frac{1}{2} \left( \Delta x(x) \left( \frac{\Delta x(x)}{\Delta t(x)} u \right)_x\right)_x\\
 &= \frac{1}{2} \left( \Delta x(x)/\l \left( \frac{\Delta x(x)/\l}{\Delta t(x)/\t} u \right)_x\right)_x \frac{\l^2}{\t} \\
 &=  \frac{\l^2}{2\t} \left( \beta(x) \left( \frac{\beta(x)}{\alpha(x)} u \right)_x\right)_x
\end{align*}

\subsection{temporally heterogeneous continuum model}
We consider the temporally heterogeneous case, i.e.,
\begin{align*}
     \alpha(y) = \left\{\begin{array}{ll}
        1, & \text{if } y<0,\\
        2, & \text{if } y>0
        \end{array}\right. 
        \quad \Longleftrightarrow \quad
     \Delta t(y) = \left\{\begin{array}{ll}
        \tau, & \text{if } y<0,\\
        2\tau, & \text{if } y>0
        \end{array}\right.         
\end{align*}
and $\beta(y)\equiv 1$. (Using the coordinate $y$, instead of $x$, is to indicate the spatial homogeneity.)

A few equivalent expressions:
\begin{equation} \label{eq:continuum}
\begin{aligned}
 v_t &= d (\frac{v}{\alpha(y)})_{yy},\\
 v_t &= d w_{yy},\\
 (\alpha(y) w)_t &= d w_{yy},\\
 w_t &= \frac{d}{\alpha(y)} w_{yy}.
\end{aligned}
\end{equation}
\begin{definition}[Energy weak solution]
 $w \in L^2\Big([0,T];H^1( \mathbb{R})\Big) \cap L^\infty\Big([0,T];L^2( \mathbb{R})\Big)$ is an energy weak solution if
 \begin{align*}
  \int_0^T \int_\mathbb{R} -\alpha(y)w(t,y)\phi_t(t,y) + d w_y(t,y)\phi_y(t,y)\; dydt - \int_\mathbb{R} \alpha(y)w(0,y)\phi(0,y) \;dy = 0
 \end{align*}
 for all $\phi \in C_c^1([0,T)\times \mathbb{R})$ and satisfies the energy identity
\begin{align*}
 \int_\mathbb{R} \alpha(y) \frac{1}{2} |w|^2(T,y)\; dy + d \int_0^T\int_\mathbb{R} |w_y|^2 \; dydt = \int_\mathbb{R} \alpha(y) \frac{1}{2} |w|^2(0,y)\; dy.
\end{align*}
\end{definition}
Note the energy weak solution is unique for this linear equation. We also consider a solution with better regularity in the parabolic H\"older space $C^{\frac{1}{2},1}\big([0,T)\times \mathbb{R}\big)$. The norm is defined by
$$\|u\|_{C^{\frac{1}{2},1}\big([0,T)\times \mathbb{R}\big)} = \sup_{(t,y)\in [0,T)\times \mathbb{R}}|u(t,y)| + \sup_{(t,y)\ne(s,z) \in [0,T)\times \mathbb{R}} \frac{|u(t,y)-u(s,z)|}{|t-s|^{1/2} + |y-z|}.$$
\begin{definition}
$w$ is a $C^{\frac{1}{2},1}$ solution if $w$ is an energy weak solution and $w \in C^{\tfrac{1}{2},1}([0,T]\times\mathbb{R})$. 
\end{definition}

\subsection{temporally heterogeneous random walk model}
We consider a random walk with grid points $(n,k)$
\begin{equation} \label{RW0}\tag{RW0}
\begin{aligned}
    2p^k_n = \left\{\begin{array}{lr}
        p^{k-1} _{n-1} + p ^{k+1} _{n-1}, & \text{if } k<0,\\
        p^{-1} _{n-1} + p ^{1} _{n-2}, & \text{if } k=0,\\
        p^{k-1} _{n-2} + p ^{k+1} _{n-2}, & \text{if } k>0.
        \end{array}\right.
\end{aligned}
\end{equation}
\begin{remark}
 As a discrete model, \eqref{RW0} is a Markov chain {\blue of order 2}. Regarded as a difference scheme, this suggests the appearance of the second order time derivative in the continuum equation while \eqref{eq:continuum} does not contain it. To clarify this aspects  {\it that the discrete model being a Markov chain of order 2 essentially is hidden and ignored in the continuum limit} will be the critical piece in understanding the underlying structures of models.
\end{remark}
{\blue
\begin{equation*} %\label{RW1_}\tag{RW1}
\begin{aligned}
    \vdots &\\
    2p^{-4}_n &= \frac{1}{2}p^{-6} _{n-2} + p ^{-4} _{n-2} + \frac{1}{2} p^{-2}_{n-2},\\
    2p^{-2}_n &= \frac{1}{2}p^{-4} _{n-2} + p ^{-2} _{n-2} + \frac{1}{2} p^0_{n-2},\\
    2p^0_n &= \frac{1}{2}p^{-2} _{n-2} + \frac{1}{2}p^{0} _{n-2} + p ^{1} _{n-2},\\
    2p^1_n &= p^{0} _{n-2} + p ^{2} _{n-2},\\
    2p^2_n &= p^{1} _{n-2} + p ^{3} _{n-2},\\
    \vdots &
\end{aligned}
\end{equation*}
}
From \eqref{RW0} we can derive a decoupled discrete system, i.e., for $n=2,4,\cdots$,
\begin{equation} \label{RW1}\tag{RW1}
\begin{aligned}
    &p^{k}_n = \left\{\begin{array}{ll}
        \frac{1}{4}p^{k-2} _{n-2} + \frac{1}{2}p ^{k} _{n-2} + \frac{1}{4} p^{k+2}_{n-2}, & \text{if $k<0$ and even},\\
        \frac{1}{4}p^{-2} _{n-2} + \frac{1}{4}p^{0} _{n-2} + \frac{1}{2}p ^{1} _{n-2}, & \text{if $k=0$},\\
        \frac{1}{2}p^{k-1} _{n-2} + \frac{1}{2}p ^{k+1} _{n-2}, & \text{if } k>0.
        \end{array}\right. 
\end{aligned}
\end{equation}

\subsubsection{referential label system $(\ell,j)$}
To define the derived random walk model \eqref{RW1} in a systematic way, those sites appeared in \eqref{RW1} are given another labelling, namely $(\ell,j)\in \mathbb{Z}_0^+ \times \mathbb{Z}$.

The inverse map $(\ell,j)\mapsto (n,k): \mathbb{Z}_0^+ \times \mathbb{Z} \rightarrow \mathbb{Z}_0^+ \times \mathbb{Z}$ defines the labelling,
\begin{align*}
 n(\ell) &= 2\ell, \quad 
 k(j)  =\left\{\begin{array}{lr}
        2j, & \text{if } j\le0,\\
        j, & \text{if }  j>0.
        \end{array}\right.
\end{align*}
The grid point $(n,k)$ will be coordinated at $(t^n,y^k)=(n\tau,k\l)$ in the later sections. Thus we regard $k$ as a spatial label while $j$ as a referential label. We will use both  labelling system $(n,k)$ and $(\ell,j)$ at our convenience. %The random walk \eqref{RW1} in the new label is simply written as
% \begin{equation} \label{RW1w}\tag{$\textrm{RW1}^\prime$}
% \begin{aligned}
%     &p^{k}_n = \left\{\begin{array}{ll}
%         \frac{1}{4}p^{k-2} _{n-2} + \frac{1}{2}p ^{k} _{n-2} + \frac{1}{4} p^{k+2}_{n-2}, & \text{if $k<0$ and even},\\
%         \frac{1}{4}p^{-2} _{n-2} + \frac{1}{4}p^{0} _{n-2} + \frac{1}{2}p ^{1} _{n-2}, & \text{if $k=0$},\\
%         \frac{1}{2}p^{k-1} _{n-2} + \frac{1}{2}p ^{k+1} _{n-2}, & \text{if } k>0.
%         \end{array}\right. 
% \end{aligned}
% \end{equation}
% \begin{equation}
%  p^j_\ell - p^j_{\ell-1} = \frac{h}{2} \left( \Big(\tfrac{p^{j+1}_{\ell-1} - p^{j}_{\ell-1}}{y^{j+1}-y^j}\Big) -\Big(\tfrac{p^{j}_{\ell-1} - p^{j-1}_{\ell-1}}{y^{j}-y^{j-1}} \Big)\right) \quad \text{for all $j$.}
% \end{equation}


%The set of appeared grid points in \eqref{RW1}, temporally $\{(n(\ell)\}$, spatially $\{k(j)\}$, and pairs of $\{(n(\ell),k(j)\}$ will be denoted by $\mathcal{N}$, $\mathcal{K}$, and $\mathcal{N}\times\mathcal{K}$ respectively.
% 
% 
% 
% : $(i)$ For time, we simply let $n=2\ell$, where $\ell$ runs over the nonnegative integer; $(ii)$ for space,
% $$ k(j)  =\left\{\begin{array}{lr}
%         2j, & \text{if } j\le0,\\
%         j, & \text{if }  j>0,
%         \end{array}\right. \quad \text{where $j$ runs over the integer.}$$
%  We may regard $j$ as {\it referential} label while $k$ as {\it spatial} label that we can go back and forth. 
% 
% \begin{remark}
%  As a discrete model, \eqref{RW1} is %, by re-labelling the time index $n=2\ell$, $\ell=0,1,2,3\cdots$, 
%  associated to a Markov chain ({\blue of order 1}) with the label $(\ell,j)$, while \eqref{RW0} is associated to a Markov chain {\blue of order 2}. %while  by re-labelling the time index $n=2\ell$, $\ell=0,1,2,3\cdots$.
% \end{remark}

\subsection{Dimensional consideration}
\subsubsection{Initial data preparation}
The assumptions on the initial probability density function $v_0$ is that
\begin{equation*}
 v_0 \in L^1( \mathbb{R}), \quad \int_ \mathbb{R} v_0(y)\, dy = 1, \quad \frac{v_0(y)}{\alpha(y)} \in C^{0,1}(\mathbb{R}).
\end{equation*}
The volume average on an interval $\tfrac{1}{b-a}\int_a^b v_0(y)\, dy$ for $a\le b$ is denoted by $\big(v_0\big)_{[a,b]}$. For a fixed $h$ and $v_0$, the probability mass $p_{0}^{j;h}$  at each of the grid point $y^j$ is defined by
\begin{equation} \label{vol_aver+pmf}
 \begin{aligned}
 p_{0}^{j;h} = \left\{\begin{array}{ll}
                2h~\big(v_0\big)_{[y_j-h,y_j+h]} & \text{if } j<0,\\
                \frac{3}{2}h~\big(v_0\big)_{[-h,\frac{h}{2}]} & \text{if } j=0,\\                
                h~\big(v_0\big)_{[y_j-\frac{h}{2},y_j+\frac{h}{2}]} & \text{if } j>0,
               \end{array}\right.
 \quad \frac{p_{0}^{j;h}}{h}\triangleq w_{0}^{j;h} \left(= \left\{\begin{array}{ll}
                \big(w_0\big)_{[y_j-h,y_j+h]} & \text{if } j<0,\\
                \frac{\big(w_0\big)_{[-h,0]}+ \big(w_0\big)_{[0,\frac{h}{2}]}}{2} & \text{if } j=0,\\                
                \big(w_0\big)_{[y_j-\frac{h}{2},y_j+\frac{h}{2}]} & \text{if } j>0,
               \end{array}\right.\right)
 \end{aligned}
\end{equation}
where $w_0(y)\triangleq \frac{2v_0(y)}{\alpha(y)}$ is the normalized Lipschiz function. 


\subsubsection{\eqref{RW1} as a finite difference scheme}
With $p_{0}^{j;h}$ as the initial probability mass, \eqref{RW1} defines $\left\{p^{j;h}_{\ell}\right\}$ and $\left\{w^{j;h}_{\ell}\right\}$ in $\mathbb{Z}_0^+\times\mathbb{Z}$ all bounded uniformly. (See the Lemma below). In the point of view where the coordinate system $(t,y)$ is fixed while $\tau$ or $h$ is varing, $p^j_\ell$ is of $O(h)$, whereas $\left\{w^{j;h}_{\ell}\right\}$ does not scale. Thus $\left\{w^{j;h}_{\ell}\right\}$ is the one we monitor in the below with respect to the limit process.

Now, we define
$$L^j_\ell \triangleq \frac{w^{j+1}-w^j}{y^{j+1}-y^j}, \quad Q^j_\ell = \frac{L^j_\ell-L^{j-1}_\ell}{2h}.$$
Note that the denominator in the latter is not $y^{j+1/2}-y^{j-1/2}$. Seen from the spatial label $k(j)$,
\begin{align*}
 L^{k(j)}_\ell &\triangleq \left\{\begin{array}{ll}
        \frac{p^{k+2} _{\ell} - p ^{k} _{\ell}}{2\l}, & \text{if $j<0$},\\
        \frac{p^{k+1} _{\ell} - p ^{k} _{\ell}}{\l}, & \text{if }  j\ge0,
        \end{array}\right. \quad Q^{k(j)}_\ell \triangleq \left\{\begin{array}{ll}
        \frac{1}{2\l^2}\Big( \frac{1}{2} p^{k-2} _{\ell} - p ^{k} _{\ell} + \frac{1}{2} p^{k+2} _{\ell}\Big), & \text{if $j<0$},\\
        \frac{1}{2\l^2}\Big( \frac{1}{2} p^{-2} _{\ell} - \frac{3}{2}p ^{0} _{\ell} +p^{1} _{\ell}\Big), & \text{if $j=0$},\\
        \frac{1}{2\l^2}\Big(p^{k-1} _{\ell} - 2p^k_n + p ^{k+1} _{\ell}\Big), & \text{if $j>0$},
        \end{array}\right.
% \Lip(n)&\triangleq\sup_{k \in \mathcal{K}} |L(n,k)|, \quad \bar Q(n)\triangleq\sup_{k \in \mathcal{K}} |Q(n,k)|.        
\end{align*}
and $Q^j_\ell$ is an approximation of $\frac{w_{yy}}{\alpha}$. More precisely, the right-hand-side expressions for $Q^{k(j)}_\ell$ are seen as taylor expansions expanded at $\big(2\ell\tau,k(j)\l\big)$. It approximates $w_{yy}$ multiplied respectively by $1$; $\frac{3}{4}\left(=\tfrac{1+1/2}{2}\right)$; and $\frac{1}{2}$, respectively for $j<0$; $j=0$; and $j>0$.
% \begin{align*}
%  Q^{k(j)}_\ell \sim \left\{\begin{array}{ll}
%         w_{yy}(2\ell\tau,k(j)\l), & \text{if $j<0$},\\
%         \frac{1}{2}\left(1+\frac{1}{2}\right)w_{yy}(2\ell\tau,k(j)\l), & \text{if $j=0$},\\
%         \frac{1}{2} w_{yy}(2\ell\tau,k\l), & \text{if $j>0$}.
%         \end{array}\right.
% \end{align*}
Now, \eqref{RW1} where $p$ replaced by $w$ is written as
\begin{equation} \label{RW1w}
\begin{aligned}
%  w^j_\ell - w^j_{\ell-1} &= \frac{h}{2} \left( \Big(\tfrac{w^{j+1}_{\ell-1} - w^{j}_{\ell-1}}{y^{j+1}-y^j}\Big) -\Big(\tfrac{w^{j}_{\ell-1} - w^{j-1}_{\ell-1}}{y^{j}-y^{j-1}} \Big)\right) \quad \text{or }
 \frac{w^j_\ell - w^j_{\ell-1}}{2\tau} &= \left(\frac{h^2}{2\tau}\right) \frac{1}{2h}\left( \Big(\tfrac{w^{j+1}_{\ell-1} - w^{j}_{\ell-1}}{y^{j+1}-y^j}\Big) -\Big(\tfrac{w^{j}_{\ell-1} - w^{j-1}_{\ell-1}}{y^{j}-y^{j-1}} \Big)\right)= \left(\frac{h^2}{2\tau}\right)Q^j_{\ell-1}
  \end{aligned}
\end{equation}
% $$\frac{w^j_\ell - w^j_{\ell-1}}{2\tau} = \left(\frac{h^2}{2\tau}\right) Q^j_{\ell-1}, \quad \text{where } Q^j_{\ell-1} = \frac{1}{2h} \left( \Big(\frac{w^{j+1}_{\ell-1} - w^{j}_{\ell-1}}{y^{j+1}-y^j}\Big) -\Big(\frac{w^{j}_{\ell-1} - w^{j-1}_{\ell-1}}{y^{j}-y^{j-1}} \Big)\right).$$% \Longleftrightarrow \frac{w^j_\ell - w^j_{\ell-1}}{2\tau} =  \quad \text{with $d=\frac{h^2}{2\tau}$}.$$
% 
% we define the volume average $v_{0j}^h$ and the probability mass function 
% \begin{equation} \label{vol_aver+pmf}
%  \begin{aligned}
%  v_{0j}^h = \left\{\begin{array}{ll}
%                 \big(v_0\big)_{[y_j-h,y_j+h]} & \text{if } j<0,\\
%                 \big(v_0\big)_{[-h,\frac{h}{2}]} & \text{if } j=0,\\                
%                 \big(v_0\big)_{[y_j-\frac{h}{2},y_j+\frac{h}{2}]} & \text{if } j>0.
%                \end{array}\right.
%  \quad
%  p_{0j}^h = \left\{\begin{array}{ll}
%                 v_{0j}^h(2h) & \text{if } j<0,\\
%                 v_{0j}^h(\frac{3}{2}h) & \text{if } j=0,\\                
%                 v_{0j}^h(h) & \text{if } j>0.
%                \end{array}\right.
%  \end{aligned}
% \end{equation}
%We also define the normalized quantity $\frac{p_{0j}^h}{h}\triangleq w_{0j}^h$. 
% If the Lipschitz function $\frac{v_0(y)}{\alpha(y)}$ is normalized so that $w_0(y)\triangleq \frac{2v_0(y)}{\alpha(y)}$, then \eqref{vol_aver+pmf}, in turn, gives the relations that
% \begin{equation} \label{wj}
%  \begin{aligned}
%  w_{0j}^h = \left\{\begin{array}{ll}
%                 \big(w_0\big)_{[y_j-h,y_j+h]} & \text{if } j<0,\\
%                 \frac{\big(w_0\big)_{[-h,0]}+ \big(w_0\big)_{[0,\frac{h}{2}]}}{2} & \text{if } j=0,\\                
%                 \big(w_0\big)_{[y_j-\frac{h}{2},y_j+\frac{h}{2}]} & \text{if } j>0.
%                \end{array}\right.
%  \end{aligned}
% \end{equation}
\subsection{Propagation of $C^{1,1}$ regularity}
\eqref{RW1} induces formulas that 
\begin{equation} \label{eq:L}
L^j_\ell = \left\{\begin{array}{ll}
        \frac{1}{4}L_{\ell-1}^{j-1} + \frac{1}{2}L_{\ell-1}^j + \frac{1}{4}L_{\ell-1}^{j+1}, & \text{if $j<0$},\\
        \frac{1}{2}L_{\ell-1}^{j-1} + \frac{1}{2}L_{\ell-1}^{j+1}, & \text{if $j\ge0$},
        \end{array}\right.
\end{equation}
\begin{equation}\label{eq:Q}
\begin{aligned}
Q^j_\ell %&= \frac{L^{j+1}_\ell - L^j_\ell}{\l}\\
&=\left\{\begin{array}{ll}
        \frac{1}{4}Q_{\ell-1}^{j-1} + \frac{1}{2}Q_{\ell-1}^{j} + \frac{1}{4}Q_{\ell-1}^{j+1}, & \text{if $j<0$},\\
        \frac{1}{4}Q_{\ell-1}^{j-1} + \frac{1}{4}Q_{\ell-1}^{j} + \frac{1}{2}Q_{\ell-1}^{j+1}, & \text{if $j=0$},\\
        \frac{1}{2}Q_{\ell-1}^{j-1} + \frac{1}{2}Q_{\ell-1}^{j+1}, & \text{if $j>0$}.
        \end{array}\right.
 \end{aligned}
\end{equation}
\begin{lemma} \label{uniform_est}
%  Let $|L|_\ell \triangleq \displaystyle\sup _{j} |L^j_\ell|$ and $|Q|_\ell \triangleq \displaystyle\sup _{j} |Q^j_\ell|$. $|L|_\ell \le |L|_0$ and $|Q|_\ell \le |Q|_0$.
 $ \displaystyle\sup _{j} |w^j_\ell| \le \sup _{j} |w^j_0|$, $ \displaystyle\sup _{j} |L^j_\ell| \le \sup _{j} |L^j_0|$, and $\displaystyle \sup _{j} |Q^j_\ell| \le \sup _{j} |Q^j_0|$.
\end{lemma}
\begin{proof}
 This is clear from \eqref{RW1} ($p$ replaced by $w$), \eqref{eq:L}, and \eqref{eq:Q}.
\end{proof}
% 
% 
% \begin{lemma} \label{lem:Lip}
% % Let $\displaystyle \Lip(n) \triangleq \sup _{k\in \mathcal{K}} \left\{\begin{array}{lr}
% %         \frac{p^{k} _{n} - p ^{k-2} _{n}}{2\l}, & \text{if } k\le0,\\
% %          \frac{p^{k} _{n} - p ^{k-1} _{n}}{\l}, & \text{if }  k>0.
% %        \end{array}\right.$
%  $\Lip(n) \le \Lip(0)$ for all $n=2\ell$, $\ell=0,1,2,\cdots$.
% \end{lemma}
% \begin{proof}
%  We observe that
% %  \begin{align*}
% %   2(p^k_n - p^{k-1}_n) &= \big(p^{k+1}_{n-2} - p^k_{n-2}\big) + \big(p^{k-1}_{n-2} - p^{k-2}_{n-2}\big),& &\text{if $k\ge2$},\\
% %   2(p^1_n - p^0_n) &= \big(p^{2}_{n-2} - p^{1}_{n-2}\big) + \frac{1}{2}\big(p^0_{n-2}-p^{-2}_{n-2}\big), & &\text{if $k=1$},\\
% %   2(p^0_n - p^{-2}_n) &= \big(p^{1}_{n-2} - p^{0}_{n-2}\big) + \big(p^{0}_{n-2} - p^{-2}_{n-2}\big)+\frac{1}{2}\big(p^{-2}_{n-2}-p^{-4}_{n-2}\big), & &\text{if $k=0$},\\
% %   2(p^k_n - p^{k-2}_n) &= \frac{1}{2}\big(p^{k+2}_{n-2}-p^{k}_{n-2}\big) + \big(p^{k}_{n-2} - p^{k-2}_{n-2}\big) + \frac{1}{2}\big(p^{k-2}_{n-2} - p^{k-4}_{n-2}\big),& &\text{if $k\le -2$},\  
% %  \end{align*}
% %  or
%   \begin{align*}
%   \left(\frac{2}{2}\right)\left|\frac{2(p^k_n - p^{k-2}_n)}{\l}\right| &= \left|\frac{p^{k+2}_{n-2}-p^{k}_{n-2}}{2\l} + \left(\frac{2}{2}\right)\frac{p^{k}_{n-2} - p^{k-2}_{n-2}}{\l} + \frac{p^{k-2}_{n-2} - p^{k-4}_{n-2}}{2\l}\right|\\
%   &\le 4\Lip(n-2), \quad\text{if $k\le -2$ even},\\
%   \left(\frac{2}{2}\right)\left|\frac{2(p^0_n - p^{-2}_n)}{\l}\right| &= \left|\frac{p^{1}_{n-2} - p^{0}_{n-2}}{\l} + \left(\frac{2}{2}\right)\frac{p^{0}_{n-2} - p^{-2}_{n-2}}{\l}+\frac{p^{-2}_{n-2}-p^{-4}_{n-2}}{2\l}\right| \\
%   &\le 4\Lip(n-2), \quad\text{if $k=0$},\\
%   \left|\frac{2(p^1_n - p^0_n)}{\l}\right| &= \left|\frac{p^{2}_{n-2} - p^{1}_{n-2}}{\l} + \frac{p^0_{n-2}-p^{-2}_{n-2}}{2\l}\right| \le 2\Lip(n-2), \quad\text{if $k=1$},\\
%   \left|\frac{2(p^k_n - p^{k-1}_n)}{\l}\right| &= \left|\frac{p^{k+1}_{n-2} - p^k_{n-2}}{\l} + \frac{p^{k-1}_{n-2} - p^{k-2}_{n-2}}{\l}\right| \le 2\Lip(n-2), \quad\text{if $k\ge2$}.
%  \end{align*}
% \end{proof}
\begin{lemma}[time regularity] \label{uniform_est_time}
 For all $j$, 
 \begin{enumerate}
  \item $\left|w^j_\ell - w^j_{\ell-1}\right| \le 2\displaystyle \sup _{j} |w^j_\ell| \le 2\displaystyle \sup _{j} |w^j_0|$, 
  \item $\left|w^j_\ell - w^j_{\ell-1}\right| \le \sqrt{2d\tau}\displaystyle \frac{|L^{j+1}_{\ell-1}| + |L^{j}_{\ell-1}|}{2}\le \sqrt{2d\tau}\displaystyle\sup _{j} |L^j_{0}|,$
  \item $\left|w^j_\ell - w^j_{\ell-1}\right| = 2d\tau |Q^j_{\ell-1}| \le 2d\tau\displaystyle\sup _{j} |Q^j_{0}|.$
 \end{enumerate}
\end{lemma}
\begin{proof}
This follows from \eqref{RW1} and \eqref{RW1w}. %the observation
% $$w^j_\ell - w^j_{\ell-1} = h\left( \frac{1}{2} L^{j}_{\ell-1}- \frac{1}{2}L^{j-1}_{\ell-1}\right) = \frac{h^2}{2} Q^j_{\ell-1} \quad \text{with $d=\frac{h^2}{2\tau}$}.$$
%  \begin{align*}
% \left|\frac{2(p^k_n - p^k_{n-2})}{\l}\right| &= \left\{\begin{array}{ll}
%         \left|\frac{p^{k+1} _{n-2} - p ^{k} _{n-2}}{\l} - \frac{p^{k} _{n-2} - p ^{k-1} _{n-2}}{\l}\right|\le 2\Lip(n-2), & \text{if } k\ge1,\\
%         \left|\frac{p^{1}_{n-2} - p^{0}_{n-2}}{\l} - \frac{\big(p^{0}_{n-2} - p^{-2}_{n-2}\big)}{2\l}\right|\le 2\Lip(n-2), & \text{if }  k=0,\\
%         \left|\frac{\big(p^{k+2}_{n-2} - p^{k}_{n-2}\big)}{2\l}-\frac{\big(p^{k}_{n-2} - p^{k-2}_{n-2}\big)}{2\l}\right|\le 2\Lip(n-2), & \text{if $k\le -2$ even},
%         \end{array}\right.
%  \end{align*}
%  or $\frac{|p^k_n - p^k_{n-2}|}{\l}\le \Lip(n-2)$. Multiplying $\sqrt{d}=\frac{\l}{\sqrt{2\tau}}$ and Lemma \ref{lem:Lip} prove the claim.
\end{proof}



\subsection{Convergence to an energy weak solution of \eqref{eq:continuum}}
Out of $\left\{w^j_{\ell;h}\right\}$, we define a function $w^h(t,x)$ by linear interpolation.
% Consider a point $(t,y)$ in a (half open) rectangle . %With notations of $\lambda^j(y)=\frac{y-y_j}{y_{j+1}-y_j}$ and $\mu^\ell(t)=\frac{t-t_\ell}{t_{\ell+1}-t_\ell}$, 
For $(t,y)$ in the half open rectangle $[y_j, y _{j+1}) \times [t_\ell, t _{\ell+1})$,
\begin{equation} \label{linear}
\begin{aligned}
 w^h(t,y)&\triangleq \left(\tfrac{y-y_j}{y_{j+1}-y_j} \right)\left(\tfrac{t-t_\ell}{t_{\ell+1}-t_\ell} \right) w_{\ell+1}^{j+1} + \left(\tfrac{y_{j+1}-y}{y_{j+1}-y_j} \right)\left(\tfrac{t-t_\ell}{t_{\ell+1}-t_\ell} \right) w_{\ell+1}^{j}\\
 &+ \left(\tfrac{y-y_j}{y_{j+1}-y_j} \right)\left(\tfrac{t_{\ell+1}-t}{t_{\ell+1}-t_\ell} \right) w_{\ell}^{j+1} + \left(\tfrac{y_{j+1}-y}{y_{j+1}-y_j} \right)\left(\tfrac{t_{\ell+1}-t}{t_{\ell+1}-t_\ell} \right) w_{\ell}^{j}. 
 \end{aligned}
\end{equation}
It is convenient to have the space or time only interpolations such that in the half open rectangle $(t,y)\in[y_j, y _{j+1}) \times [t_\ell, t _{\ell+1})$
\begin{equation} \label{linear0}
\begin{aligned}
 \bar w^h(t,y)&\triangleq \left(\tfrac{t-t_\ell}{t_{\ell+1}-t_\ell} \right) w_{\ell+1}^{j} + \left(\tfrac{t_{\ell+1}-t}{t_{\ell+1}-t_\ell} \right) w_{\ell}^{j},\\
 \hat w^h(t,y)&\triangleq \left(\tfrac{y-y_j}{y_{j+1}-y_j} \right) w_{\ell}^{j+1} + \left(\tfrac{y_{j+1}-y}{y_{j+1}-y_j} \right) w_{\ell}^{j}.
 \end{aligned}
\end{equation}
From \eqref{RW1w}, we can write for $(t,y) \in [y_j, y _{j+1}) \times [t_\ell, t _{\ell+1})$,
\begin{equation}
 \partial_t \bar{w}(t,y) = \left\{\begin{array}{ll}
                            \frac{d}{2h}\big(\partial_y\hat{w}(t,y)-\partial_y\hat{w}(t,y-2h)\big) & \text{if } j<0,\\
%                             \frac{d}{2h}\big(\partial_y\hat{w}(t,y)-\partial_y\hat{w}(t,y-h)\big)=\frac{d}{2h}\big(\partial_y\hat{w}(t,y)-\partial_y\hat{w}(t,y-2h)\big) & \text{if } j=0, \\
                            \frac{d}{2h}\big(\partial_y\hat{w}(t,y)-\partial_y\hat{w}(t,y-h)\big) & \text{if } j>0.
                           \end{array}\right.
\end{equation}


{\red
$$ \| w^h\|_{C^{\frac{1}{2},1}\big([0,T)\times \mathbb{R}\big)} \le C\sup _{j}\left(|w^j_0|+|L^j_0|\right)$$
with the uniform constant $C$. Now, by the compactness theorem, 
 $\exists$ subsequence $\left\{w^{h(m)}\right\}$ and a function $w(t,x)\in C^{\frac{1}{2},1}\big([0,T)\times \mathbb{R}\big)$ such that for each compact set $K \subset [0,T)\times \mathbb{R}$
 \begin{enumerate}
  \item $\bar{w}^{h(m)}, w^{h(m)}\rightarrow w$ uniformly in $K$,
  \item $\partial_y \hat{w}^{h(m)}, \partial_y w^{h(m)} \rightharpoonup \partial_y w$ weakly-* in $L^\infty(K)$.
 \end{enumerate}
}

Now,
\begin{align*}
 &\overbrace{\int_0^T\int_\mathbb{R} -\bar w^h(t,y) \alpha(y)\partial_t\phi(t,y)\; dy dt}^{\triangleq I} -\overbrace{\int_\mathbb{R} \bar{w}^h(0,y)\alpha(y)\phi(t,y) \; dydt}^{\triangleq II} \\
 %&=\sum_{\ell\ge0}\int_{t_\ell}^{t_{\ell+1}}\int_\mathbb{R} -\bar w^h(t,y) \alpha(y)\partial_t\phi(t,y)\; dy dt - \int_\mathbb{R} \bar{w}^h(0,y)\alpha(y)\phi(t,y) \; dydt \\
 &=\sum_{\ell\ge0}\int_{t_\ell}^{t_{\ell+1}}\int_\mathbb{R} \partial_t\bar w^h(t,y) \alpha(y)\phi(t,y)\; dy dt \\
 &=\sum_{\ell\ge0}\sum_{j}\int_{t_\ell}^{t_{\ell+1}}\int_\mathbb{R} \frac{w^j_{\ell+1}-w^j_{\ell}}{2\tau} \alpha(y)\phi(t,y)\chi_j(y)\; dy dt \quad (\text{with } \chi_j=\chi_{[y_j,y_{j+1})})\\
 &=\sum_{\ell\ge0}\sum_{j}d\int_{t_\ell}^{t_{\ell+1}}\int_\mathbb{R} \tfrac{1}{2h} \left( \Big(\tfrac{w^{j+1}_{\ell} - w^{j}_{\ell}}{y^{j+1}-y^j}\Big) -\Big(\tfrac{w^{j}_{\ell} - w^{j-1}_{\ell}}{y^{j}-y^{j-1}} \Big)\right) \alpha(y)\phi(t,y)\chi_j(y)\; dy dt\\
 &=\sum_{\ell\ge0}\sum_{j}d\int_{t_\ell}^{t_{\ell+1}}\underbrace{\int_\mathbb{R} \tfrac{1}{2h} \Big(\tfrac{w^{j}_{\ell} - w^{j}_{\ell}}{y^{j-1}-y^j}\Big)\alpha(y)\big(\chi_{j-1}(y)-\chi_{j}(y)\big) \phi(t,y)\; dy}_{\triangleq A_j} \;dt.
\end{align*}
\begin{align*}
 A_j &= \Big(\tfrac{w^{j}_{\ell} - w^{j-1}_{\ell}}{y^{j}-y^{j-1}}\Big)
        \left\{\frac{1}{2h}\int_{y^{j-1}}^{y^{j}} \alpha(y)\phi(t,y) \; dy -  \frac{1}{2h}\int_{y^{j}}^{y^{j+1}} \alpha(y)\phi(t,y) \; dy\right\}\\
     &= \left\{\begin{array}{ll} 
        \Big(\tfrac{w^{j}_{\ell} - w^{j-1}_{\ell}}{y^{j}-y^{j-1}}\Big)
        \bigg( \frac{1}{y^j-y^{j-1}}\int_{y^{j-1}}^{y^{j}} \phi(t,y) \; dy- \frac{1}{y^{j+1}-y^{j}}\int_{y^{j}}^{y^{j+1}} \phi(t,y) \; dy \bigg) & \text{if } j<0,\\
        \Big(\tfrac{w^{0}_{\ell} - w^{-1}_{\ell}}{y^{0}-y^{-1}}\Big)
        \bigg( \frac{1}{y^j-y^{j-1}}\int_{y^{j-1}}^{y^{j}} \phi(t,y) \; dy- \frac{1}{2(y^{j+1}-y^{j})}\int_{y^{j}}^{y^{j+1}} 2\phi(t,y) \; dy \bigg) & \text{if } j=0,\\
        \Big(\tfrac{w^{j}_{\ell} - w^{j-1}_{\ell}}{y^{j}-y^{j-1}}\Big)
        \bigg( \frac{1}{2(y^j-y^{j-1})}\int_{y^{j-1}}^{y^{j}} 2\phi(t,y) \; dy- \frac{1}{2(y^{j+1}-y^{j})}\int_{y^{j}}^{y^{j+1}} 2\phi(t,y) \; dy \bigg) & \text{if } j>0,        
       \end{array}\right.\\
     &=-\int_{y^{j-1}}^{y^{j}}\Big(\tfrac{w^{j}_{\ell} - w^{j-1}_{\ell}}{y^{j}-y^{j-1}}\Big)\partial_y\phi(t,y) \\
     & + \Big(\tfrac{w^{j}_{\ell} - w^{j-1}_{\ell}}{y^{j}-y^{j-1}}\Big)\left\{
        \frac{\int_{y^{j-1}}^{y^{j}} \phi(t,y)-\phi(t,y^{j-1}) \; dy}{y^j-y^{j-1}} - \frac{\int_{y^{j}}^{y^{j+1}} \phi(t,y)-\phi(t,y^{j}) \; dy}{y^{j+1}-y^j} \right\}.
\end{align*}
The latter error term, in particular, is $o(h)$ if $j\ne0$ and $O(h)$ if $j=0$. Hence in total,
$$\sum_j  A_j = -\int_\mathbb{R} \partial_y \hat{w}(t,y) \partial_y \phi(t,y) \; dy + o(1) \rightarrow -\int_\mathbb{R} \partial_y w(t,y) \partial_y \phi(t,y) \; dy.$$

Since $\bar{w}^h \rightarrow w$ uniformly for each compact set,
\begin{align*}
I &\rightarrow \int_0^T\int_\mathbb{R} -w(t,y) \alpha(y)\partial_t\phi(t,y)\; dy dt, \quad 
II \rightarrow \int_\mathbb{R} w(0,y)\alpha(y)\phi(t,y) \; dydt. 
\end{align*}

% 
% 
% 
% Now noticing that $\chi_{j-1}(y) = \chi_{j}(y+h)$ if $j>0$ and $\chi_{j-1}(y) = \chi_{j}(y+2h)$ if $j<0$,
% \begin{align*}
%  A_j &=\left\{\begin{array}{ll} 
%         \int_\mathbb{R} \Big(\tfrac{w^{j+1}_{\ell} - w^{j}_{\ell}}{y^{j+1}-y^j}\Big) \chi_{j}(y) \tfrac{2\big(\phi(t,y)-\phi(t,y-h)\big)}{2h} \; dy & \text{if } j>0,\\
%         \int_\mathbb{R} \Big(\tfrac{w^{j+1}_{\ell} - w^{j}_{\ell}}{y^{j+1}-y^j}\Big) \chi_{j}(y) \tfrac{\big(\phi(t,y)-\phi(t,y-2h)\big)}{2h} \; dy & \text{if } j<0.
%        \end{array}\right.
% \end{align*}
% For $j=0$, 
% \begin{align*}
% A_0 &= \Big(\tfrac{w^{j+1}_{\ell} - w^{j}_{\ell}}{y^{j+1}-y^j}\Big)\left\{ \tfrac{1}{2h}\int_0^h 2\phi(t,y)\; dy-\tfrac{1}{2h} \int_{-2h}^0\phi(t,y)\;dy\right\}\\
% &= \Big(\tfrac{w^{j+1}_{\ell-1} - w^{j}_{\ell-1}}{y^{j+1}-y^j}\Big)\int_0^h \partial_y \phi(t,y)\; dy+o(h) \quad \text{as $h \rightarrow 0$}
% \end{align*}
% since $\phi$ is $C^1$. %With the same reason, expressions $\frac{\phi(t,y)-\phi(t,y-h)}{h}$ and $\frac{\phi(t,y)-\phi(t,y-2h)}{2h}$ converges uniformly to $\partial_y\phi(t,y)$ in the compact support of $\phi$.
% Now $\Big(\tfrac{w^{j+1}_{\ell-1} - w^{j}_{\ell-1}}{y^{j+1}-y^j}\Big)= \partial_y \hat{w}^h(t,y)$ and using $\partial_y \hat{w}^h(t,y)\rightharpoonup \partial_y w$ weakly-* in $L^\infty$ and that $\phi\in C^1$ again
% \begin{align*}
%  \sum_{\ell\ge0} \sum_j A_j = \int_0^T\int_ \mathbb{R} \partial_y \hat{w}(t,y)\partial_y\phi(t,y)\; dydt + o(1)\rightarrow \int_{0}^{T}\int_\mathbb{R} \partial_y{w}(t,y)\partial_y\phi(t,y) \;dy dt.
% \end{align*}

% or $w(t,x)$ is a weak solution.
% 
% In each of the rectangle, $w$ has a well-defined right gradient that are simply
% \begin{equation}
% \begin{aligned}
%  \partial_t w(t,y) &= \Big(\tfrac{y-y_j}{y_{j+1}-y_j} \Big) \Big(\tfrac{w^{j+1}_{\ell+1} -w^{j+1}_{\ell}}{t_{\ell+1}-t_\ell}\Big) + \Big(\tfrac{y_{j+1}-y}{y_{j+1}-y_j} \Big) \Big(\tfrac{w^{j}_{\ell+1} -w^{j}_{\ell}}{t_{\ell+1}-t_\ell}\Big),\\
% \partial_y w(t,y) &= \Big(\tfrac{t-t_\ell}{t_{\ell+1}-t_\ell} \Big) \Big(\tfrac{w^{j+1}_{\ell+1} -w^{j}_{\ell+1}}{y_{j+1}-y_j}\Big) + \Big(\tfrac{t_{\ell+1}-t}{t_{\ell+1}-t_\ell} \Big) \Big(\tfrac{w^{j+1}_{\ell} -w^{j}_{\ell}}{y_{j+1}-y_j}\Big).
% \end{aligned}
% \end{equation}


% 
% \begin{proposition}
%  Let the function $w(t,y;\tau)$ on $[0,T]\times \mathbb{R}$ such that 
% \begin{equation} \label{eq:w_def}
% \begin{aligned}
%     w(t,y;\tau) \triangleq \left\{\begin{array}{ll}
%         p^{k}_{n} , & \text{if $(t,y)=(n\tau,k\l)$ and $(n,k)\in \mathcal{N}\times\mathcal{K}$},\\
%         \text{linear interpolation}, & \text{otherwise }.
%         \end{array}\right.
% \end{aligned}
% \end{equation} 
% %Let $\Lip(t)$ be the Lipschitz constant of the function $w(t,x)$ at time $t$. 
%  Then for all $t_1,t_2\in[0,T]$ and all $y_1,y_2\in \mathbb{R}$,
%  \begin{enumerate}
%   \item $|w(t,y_1)-w(t,y_2)| \le \Lip(0)|y_1-y_2|$,
%   \item $|w(t_1,y)-w(t_2,y)| \le \sqrt{d}\Lip(0)\sqrt{|t_1-t_2|}$.
%  \end{enumerate}
% \end{proposition}
% \begin{proof}
%  {\red proof!}
% \end{proof}
% 
% \begin{proposition}
%  Let $\tau_m \searrow 0$. Then $\exists \tau_{m(i)}$ such that $w^{\tau_{m(i)}}$ converges to a weak solution $w$. Furthermore $w\in C^{\frac{1}{2},1}$ or $w$ is a strong solution.
% \end{proposition}
% 
% \begin{theorem}
%  $||w_0^{\tau_m}(x)-w_0(x)||_{C^{0,1}( \mathbb{R})} \rightarrow 0$ as $m \rightarrow \infty$.
% \end{theorem}
% 
% \subsection{perspectives}
% Note that \eqref{eq:Q} of $Q^k_n$ coincides with \eqref{RW1} of $p^k_n$. Therefore, all the higher order difference quotients can be estimated inductively.

\end{document}
