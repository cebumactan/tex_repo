%%%%%%%%%%%%%%%%%%%%%%%%%%%%%%%%%%%%%%%%%%%%%%%
%
%    Self-Similar shear bands, Existence, Numerics, Asymptotics
%
%                                                      by
%
%                                       Min-Gi Lee   
%
%                                          version Sep 2016
%
%
%%%%%%%%%%%%%%%%%%%%%%%%%%%%%%%%%%%%%%%%%%%%%%%
\documentclass[a4paper,11pt]{article}

\usepackage[margin=2cm]{geometry}
\usepackage{setspace}
%\onehalfspacing
\doublespacing
%\usepackage{authblk}
\usepackage{amsmath}
\usepackage{amssymb}
\usepackage{amsthm}

\usepackage[notcite,notref]{showkeys}

% \usepackage{psfrag}
\usepackage{graphicx,subfigure}
\usepackage{color}
\def\red{\color{red}}
\def\blue{\color{blue}}
%\usepackage{verbatim}
% \usepackage{alltt}
%\usepackage{kotex}

\usepackage{enumerate}

%%%%%%%%%%%%%% MY DEFINITIONS %%%%%%%%%%%%%%%%%%%%%%%%%%%

\def\tr{\,\textrm{tr}\,}
\def\div{\,\textrm{div}\,}
\def\sgn{\,\textrm{sgn}\,}

\def\th{\tilde{h}}
\def\tx{\tilde{x}}
\def\tk{\tilde{\kappa}}

\def\tg{{\tilde{\gamma}}}
\def\tv{{\tilde{v}}}
\def\tth{{\tilde{\theta}}}
\def\ts{{\tilde{\tau}}}
\def\tu{{\tilde{u}}}

\def\dtg{\dot{\tilde{\gamma}}}
\def\dtv{\dot{\tilde{v}}}
\def\dtth{\dot{\tilde{\theta}}}
\def\dts{\dot{\tilde{\tau}}}
\def\dtu{\dot{\tilde{u}}}

\def\dpp{\dot{p}}
\def\dqq{\dot{q}}
\def\drr{\dot{r}}
\def\dss{\dot{s}}

\def\ta{{\tilde{a}}}
\def\tb{{\tilde{b}}}
\def\tc{{\tilde{c}}}
\def\td{{\tilde{d}}}




\def\bx{\bar{x}}
\def\bm{\bar{\mathbf{m}}}
\def\K{\mathcal{K}}
\def\E{\mathcal{E}}
\def\del{\partial}
\def\eps{\varepsilon}

\newcommand{\tcr}{\textcolor{red}}
\newcommand{\tcb}{\textcolor{blue}}

\newcommand{\ubar}[1]{\text{\b{$#1$}}}
\newtheorem{theorem}{Theorem}
\newtheorem{lemma}{Lemma}[section]
\newtheorem{proposition}{Proposition}[section]
\newtheorem{corollary}{Corollary}[section]
\newtheorem{definition}{Definition}[section]
\newtheorem{remark}{Remark}[section]
\newtheorem{claim}{Claim}
%%%%%%%%%%%%%%%%%%%%%%%%%%%%%%%%%%%%%%%%%%%%%%%%%%%%%%%%%%
\begin{document}
\title{Note for the Self-similar shear bands problem}
\author{Min-Gi Lee\footnotemark[1]}
% \and Athanasios Tzavaras\footnotemark[1]\  \footnotemark[3]  \footnotemark[4]}
\date{}

\maketitle
\renewcommand{\thefootnote}{\fnsymbol{footnote}}
% \footnotetext[1]{Computer, Electrical and Mathematical Sciences \& Engineering Division, King Abdullah University of Science and Technology (KAUST), Thuwal, Saudi Arabia}
% \footnotetext[2]{Department of Mathematics and Applied Mathematics, University of Crete, Heraklion, Greece}
% \footnotetext[3]{Institute of Applied and Computational Mathematics, FORTH, Heraklion, Greece}
% \footnotetext[4]{Corresponding author : \texttt{athanasios.tzavaras@kaust.edu.sa}}
%\footnotetext[4]{Research supported by the King Abdullah University of Science and Technology (KAUST) }
\renewcommand{\thefootnote}{\arabic{footnote}}


\maketitle

\tableofcontents
% \begin{abstract}
% abstract
% \end{abstract}

\section{The model description}
We consider a 1-d shear deformation of a material whose material law of stress depends on 1) temperature, 2) strain, 3) strain rate. The motion is described by following field variables,
\begin{equation} \label{eq:vars}
\begin{aligned}
 \gamma(t,x) &: \text{strain}\\
 u(t,x)=\gamma_t &: \text{strain rate}\\
 v(t,x) &: \text{vertical velocity}\\
 \theta(t,x) &: \text{temperature}\\
 \tau(t,x) &: \text{stress}
\end{aligned}
\end{equation}
The material exhibits 1) temperature-softening, 2) strain-hardening, 3) rate-hardening. we denote the shear stress
$$ \tau = \tau(\theta,\gamma,u). $$
and study a model
\begin{equation}
 \tau = \theta^{-\alpha}\gamma^m u^n. \label{eq:stresslaw}
\end{equation}

A few forehand perspectives :
\begin{enumerate}
 \item The regime where $-\alpha+m+n <0$ will exhibit the localization, whereas the regime $-\alpha+m+n > 0$ will exhibit stabilization. {\blue Can we rigorously study the linearize stability to the uniform shearing solution?}
 \item The uniform shearing solution will appear as one of the self-similar solution by a specific $\lambda$ that is negative.
\end{enumerate}
\subsection{A system of conservation laws}
For the field variables \eqref{eq:vars}, equations describing the deformation are given by
\begin{align}
 \gamma_t &= u, \quad \text{(kinematic compatibility)} 	\label{eq:g}\\
 v_t &= \tau_x, \quad \text{(momentum conservation)} 	\label{eq:v}\\
 \theta_t &= \tau u \quad \text{(energy conservation)}	\label{eq:th}\\
 \tau &=\theta^{-\alpha}\gamma^m u^n.			\label{eq:tau}
\end{align}
\subsection{Scale invariance property of the system}
The system \eqref{eq:g}-\eqref{eq:tau} admits a scale invariance property. Suppose $(\gamma,u,v,\theta,\tau)$ is a solution. Then a rescaled version of it 
\begin{align*}
 \gamma_\rho(t,x) &= \rho^a\gamma(\rho^{-1}t,\rho^\lambda x), &
 v_\rho(t,x) &= \rho^bv(\rho^{-1}t,\rho^\lambda x),\\
 \theta_\rho(t,x) &= \rho^c\theta(\rho^{-1}t,\rho^\lambda x) &
 \tau_\rho(t,x) &= \rho^d\tau(\rho^{-1}t,\rho^\lambda x),\\
 u_\rho(t,x) &= \rho^{b+\lambda}\gamma(\rho^{-1}t,\rho^\lambda x). 
\end{align*}

Calculations :
\begin{align*}
 \text{let} \;f(t,x) = \rho^k F(\rho^{-1}t,\rho^\lambda x), \\
 \partial_t f(t,x) = \rho^k \partial_{t'} F(\rho^{-1}t,\rho^\lambda x) \rho^{-1} = \rho^{k-1} \partial_{t'}F, \\
 \partial_x f(t,x) = \rho^k \partial_{x'} F(\rho^{-1}t,\rho^\lambda x) \rho^\lambda = \rho^{k+\lambda} \partial_{x'}F.
\end{align*}
Relations for invariance :
\begin{align*}
 a-1 = b+\lambda, \quad b-1 = d+\lambda, \quad c-1 = d+b+\lambda, \quad d = -\alpha c + m a + n (b+\lambda)
\end{align*}
From above, we reach to exponents
\begin{align*}
 D & = 1+2\alpha-m-n,\\
 a&= \frac{2+2\alpha-n}{D} + \frac{2+2\alpha}{D}\lambda =: a_0 + a_1 \lambda, & b&=\frac{1+m}{D} + \frac{1+m+n}{D}\lambda =: b_0 + b_1\lambda,\\
 c&=\frac{2(1+m)}{D} + \frac{2(1+m+n)}{D}\lambda =: c_0 + c_1\lambda, & d&=\frac{-2\alpha + 2m +n}{D} + \frac{-2\alpha+2m+2n}{D}\lambda =: d_0 + d_1\lambda
\end{align*}
for each $\lambda \in \mathbb{R}$. For localization, $\lambda>0$. But uniform shearing solution takes a negative $\lambda$.

\begin{align*}
 &a_0-1 = b_0,  \quad b_0-1=d_0,  \quad c_0-1=d_0 +b_0, \quad c_1-1=d_1+b_1, \quad d_0 = -\alpha c_0 + ma_0 +nb_0,\\
 &a_1-1=b_1, \quad b_1-1=d_1, \quad c_1-1=d_1+b_1,  \quad d_1 = -\alpha c_1 + ma_1 +n(b_1+1).
\end{align*}


\subsection{Self-Similar variables}
We try the solutions of type, i.e. put $\rho =t$ in the rescaling,
\begin{align*}
 \gamma(t,x) &= t^a\Gamma(t^\lambda x),\\
 v(t,x) &= t^b V(t^\lambda x),\\
 \theta(t,x) &= t^c \Theta(t^\lambda x),\\
 \tau(t,x) &= t^d \Sigma(t^\lambda x),\\
 u(t,x) &= t^{b+\lambda} U(t^\lambda x)
\end{align*}
and set $\xi = t^\lambda x$.

Calculations:
\begin{align*}
 &\text{Suppose } \; f(t,x) = t^k F(t^\lambda x),\\
 &\partial_t f = k t^{k-1} F + t^k F' \lambda t^{\lambda-1} x = t^{k-1} (kF + \lambda\xi F'),\\
 &\partial_x f = t^k F' t^\lambda = t^{k+\lambda} F',
\end{align*}

At \eqref{eq:g}-\eqref{eq:tau}:

\begin{align*}
 t^{a-1}(a \Gamma(\xi) + \lambda \xi \Gamma'(\xi)) &= t^{b+ \lambda} U(\xi),\\
 t^{b-1}(b V(\xi) + \lambda \xi V'(\xi)) &= t^{d+ \lambda} \Sigma'(\xi)\\
 t^{c-1}(c \Theta(\xi) + \lambda \xi \Theta'(\xi))&=t^{b+d+\lambda} \Sigma U(\xi),\\
 t^d\Sigma(\xi) &= t^{-\alpha c +ma +n(b+ \lambda)} \Theta(\xi)^{-\alpha} \Gamma(\xi)^m U(\xi)^n,\\
 t^{b+\lambda}V'(\xi)&=t^{b+\lambda}U(\xi)
\end{align*}
\begin{equation}
\begin{aligned}
 a \Gamma(\xi) + \lambda \xi \Gamma'(\xi) &= U(\xi),\\
 b V(\xi) + \lambda \xi V'(\xi) &= \Sigma'(\xi)\\
 c \Theta(\xi) + \lambda \xi \Theta'(\xi)&=\Sigma(\xi) U(\xi),\\
 \Sigma(\xi) &= \Theta(\xi)^{-\alpha} \Gamma(\xi)^m U(\xi)^n,\\
 V'(\xi)&=U(\xi)
\end{aligned} \label{eq:ss-odes}
\end{equation}

\begin{equation}
\begin{aligned}
 a \Gamma(\xi) + \lambda \xi \Gamma'(\xi) &= U(\xi),\\
 (b+\lambda) U(\xi) + \lambda \xi U'(\xi) &= \Sigma^{''}(\xi)\\
 c \Theta(\xi) + \lambda \xi \Theta'(\xi)&=\Sigma(\xi) U(\xi),\\
 \Sigma(\xi) &= \Theta(\xi)^{-\alpha} \Gamma(\xi)^m U(\xi)^n,\\
 V'(\xi)&=U(\xi)
\end{aligned} \label{eq:ss-odes2}
\end{equation}

\subsection{de-singularization}
% {\red This part cannot be done : constants are inconsistent
% We find the following monomials constitute the special solution.
% \begin{equation} \label{eq:monomials}
% \begin{aligned}
%  \Gamma(\xi)&=\xi^{-a_1}\Gamma_0, & \Theta(\xi)&=\xi^{-c_1}\Theta_0,\\
%  \Sigma(\xi)&=\xi^{-d_1}\Sigma_0, & U(\xi)&=\xi^{-b_1-1}U_0, \\
%  V(\xi)&=\int_{\xi_0}^\xi U(\xi')\, d\xi',
% \end{aligned}
% \end{equation}
% where the constants $\big(\Gamma_0, U_0, \Theta_0, \Sigma_0, U_0\big)$ is in relations
% \begin{align*}
%  \Gamma_0 &= \frac{U_0}{a_0}, \quad \Theta_0 = a_0^{-\frac{m}{1+\alpha}} c_0^{-\frac{1}{1+\alpha}} U_0^{\frac{1+m+n}{1+\alpha}}, \quad \Sigma_0=a_0^{-\frac{m}{1+\alpha}} c_0^{\frac{\alpha}{1+\alpha}} U_0^{-\frac{\alpha-m-n}{1+\alpha}} \\
% %  U_0&=\Big(b_1\,d_1\,a_0^{-\frac{m}{1+\alpha}} b_0^{-1}c_0^{\frac{\alpha}{1+\alpha}}\Big)^{\frac{1+\alpha}{D}}.
% \end{align*}
% }

These arise from the consideration of the scale invariance property of the system \eqref{eq:ss-odes}: Provided $\big(\Gamma(\xi), V(\xi), \Theta(\xi), \Sigma(\xi), U(\xi)\big)$ is a solution of it, then the scaled version $\big(\Gamma_\rho(\xi), V_\rho(\xi), \Theta_\rho(\xi), \Sigma_\rho(\xi), U_\rho(\xi)\big)$, where
\begin{align*}
 \Gamma_\rho(\xi)&=\rho^{a_1}\Gamma(\rho\xi), & V_\rho(\xi)&=\rho^{b_1}V(\rho\xi), & \Theta_\rho(\xi)&=\rho^{c_1}\Theta(\rho\xi),\\
 \Sigma_\rho(\xi)&=\rho^{d_1}\Sigma(\rho\xi), & U_\rho(\xi)&=\rho^{b_1+1}U(\rho\xi),
\end{align*}
again is a solution. 

Introduce new field variables 
\begin{equation}
\begin{aligned}
 \Gamma(\xi) &= \xi^\ta \tg(\xi),\\
 V(\xi)&=\xi^\tb \tv(\xi),\\
 \Theta(\xi)&=\xi^\tc \tth(\xi),\\
 \Sigma(\xi)&=\xi^\td \ts(\xi),\\
 U(\xi)&=\xi^{\tb-1} \tu(\xi). 
\end{aligned}
\end{equation}
Then at \eqref{eq:ss-odes}:
\begin{align*}
 \xi^\ta\Big( a\tg + \lambda \ta \tg + \lambda\xi\tg'\Big) &=\xi^{\tb-1} \tu,\\
 \xi^\tb\Big( b\tv + \lambda \tb \tv + \lambda\xi\tv'\Big) &=\xi^{\td-1} \Big(\td\ts + \xi\ts'\Big),\\
 \xi^\tc\Big( c\tth+ \lambda \tc \tth+ \lambda\xi\tth'\Big)&=\xi^{\td+\tb-1} \ts\tu,\\
 \xi^\td\ts &= \xi^{-\alpha \tc +m\ta +n(\tb-1)} \tth^{-\alpha} \tg^m \tu^n,\\
 \xi^{\tb-1}\Big(\tb\tv + \xi \tv'\Big)&= \xi^{\tb-1} \tu.
\end{align*}

$\ta, \tb, \tc, \td$ such that
\begin{align*}
 &\ta=\tb-1, \quad \tb=\td-1, \quad \tc=\td+\tb-1,\quad \td = -\alpha \tc + m\ta +n(\tb-1) \\
 \Longrightarrow \quad&\ta = -a_1, \quad \td = -d_1, \quad \tc = -c_1, \quad \tb=-b_1.
\end{align*}

\begin{equation}
 \begin{aligned}
  a_0\tg + \lambda\xi\tg' &=\tu,\\
  b_0\tv + \lambda\xi\tv' &=-d_1 \ts + \xi\ts',\\
  c_0\tth+ \lambda\xi\tth'&=\ts\tu,\\
  \ts &=\tth^{-\alpha}\tg^m\tu^n,\\
  -b_1\tv+\xi\tv' &= \tu.
 \end{aligned}
\end{equation}
Introduce the new independent variable $\eta = \log\xi$.

At $\xi=0$:
$$V(0)=\Gamma'(0)=U'(0)=\Sigma'(0)=\Theta'(0)=0.$$

As $\xi \rightarrow \infty$:
$$ \frac{U(\xi)}{\Gamma(\xi)} \rightarrow r_1 >0.$$
\pagebreak
\section{$(p,q,r,s)$-system derivation}
Auxiliary calculations:
\begin{align*}
 \frac{\dtg}{\tg} &= \frac{1}{\lambda }\Big(\frac{\tu}{\tg}-a_0\Big),\\
 \frac{\dts}{\ts} &= d_1+ b_0\frac{\tv}{\ts} + \lambda \frac{\dtv}{\ts} = d_1+ b_0\frac{\tv}{\ts} + \lambda \Big(b_1 + \frac{\tu}{\tv}\Big)\frac{\tv}{\ts} = d_1 + b\frac{\tv}{\ts} + \lambda\frac{\tu}{\tv}\frac{\tv}{\ts} ,\\
 \frac{\dtth}{\tth}&=\frac{1}{\lambda }\Big(\frac{\ts\tu}{\tth}-c_0\Big),\\
 0&=\frac{\dts}{\ts} +\alpha \frac{\dtth}{\tth} - m \frac{\dtg}{\tg} - n\frac{\dtu}{\tu},\\
 \frac{\dtv}{\tv}&= b_1 +\frac{\tu}{\tv}
\end{align*}

Define 
\begin{equation}\label{eq:pqrsdef}
 \begin{aligned}
  p &= \frac{\tg}{\ts}, & q&=b \frac{\tv}{\ts}, &  r &= \frac{\tu}{\tg}, & s&=\frac{\ts\tg}{\tth}.
 \end{aligned}
\end{equation}

We have
\begin{align*}
 \frac{\dpp}{p}&=\frac{\dtg}{\tg} - \frac{\dts}{\ts}& &=\left[\frac{1}{\lambda }\Big(\frac{\tu}{\tg}-a_0\Big)\right] & &-\left[d_1 + b\frac{\tv}{\ts} + \lambda\frac{\tu}{\tv}\frac{\tv}{\ts}\right]\\
 \frac{\dqq}{q}&=\frac{\dtv}{\tv} - \frac{\dts}{\ts}& &=\left[b_1 +\frac{\tu}{\tv}\right] & &-\left[d_1 + b\frac{\tv}{\ts} + \lambda\frac{\tu}{\tv}\frac{\tv}{\ts}\right]\\
 n\frac{\drr}{r}&=\frac{\dts}{\ts} + \alpha\frac{\dtth}{\tth} -(m+n)\frac{\dtg}{\tg}& &=\left[\frac{-(m+n)}{\lambda}\Big(\frac{\tu}{\tg}-a_0\Big)\right]& &+\left[d_1 + b\frac{\tv}{\ts} + \lambda\frac{\tu}{\tv}\frac{\tv}{\ts}\right] + \left[\frac{\alpha}{\lambda }\Big(\frac{\ts\tu}{\tth}-c_0\Big)\right]\\
 \frac{\dot{s}}{s} &= \frac{\dts}{\ts} + \frac{\dtg}{\tg} - \frac{\dtth}{\tth} & &=\left[\frac{1}{\lambda }\Big(\frac{\tu}{\tg}-a_0\Big)\right] & &+\left[d_1 + b\frac{\tv}{\ts} + \lambda\frac{\tu}{\tv}\frac{\tv}{\ts}\right] -\left[\frac{1}{\lambda }\Big(\frac{\ts\tu}{\tth}-c_0\Big)\right].%\\
%  \dot{s} &=\frac{\partial s}{\partial (z-1)} \dot{z} &&= \frac{1+m}{\lambda}\frac{\partial s}{\partial (z-1)} \bigg\{z\big[-r - \frac{n}{D}\Big]+ ru^n\bigg\}.
\end{align*}
\begin{equation}\label{eq:slow}
 \begin{aligned}
 \dot{p} &=p\Big(\frac{1}{\lambda}(r-a) + 2- \lambda p r -q\Big),\\
 \dot{q} &=q\Big(1 -\lambda p r -q\Big) + b p r,\\
 n\dot{r} &=r\Big(\frac{\alpha-m-n}{\lambda(1+\alpha)}(r-a) + \lambda pr + q +\frac{\alpha}{\lambda}r\big(s- \frac{1+m+n}{1+\alpha}\big) + \frac{n\alpha}{\lambda(1+\alpha)}\Big),\\
 \dot{s} &=s\Big(\frac{\alpha-m-n}{\lambda(1+\alpha)}(r-a) + \lambda pr + q - \frac{1}{\lambda}r\big(s- \frac{1+m+n}{1+\alpha}\big) - \frac{n}{\lambda(1+\alpha)}\Big).
 \end{aligned}
\end{equation}

\section{Equilibrium points and linear stability}

\begin{align*}
 M_0 &= \Big(0,0,r_0,\frac{1+m+n}{1+\alpha} - \frac{n}{(1+\alpha)r_0}\Big), & r_0 &=\frac{2+2\alpha-n}{D} + \frac{2+2\alpha}{D}\lambda,\\
 M_1 &= \Big(0,1,r_1,\frac{1+m+n}{1+\alpha} - \frac{n}{(1+\alpha)r_1}\Big), & r_1 &= r_0-\frac{1+\alpha}{\alpha-m-n}\lambda.
\end{align*}

Other equilibrium points and rejections:
\begin{align*}
 &p=0, \ q=0, \ {\bf r=0, \ s=0},\\
 &p=0, \ q=0, \ {\bf r=0},\ s=s, \ {\bf r_0 = \frac{-n}{\alpha-m-n}},\\
 &p=0, \ q=0, \ r = \frac{n\alpha - r_0(\alpha-m-n)}{(1+\alpha)(m+n)},\ {\bf s=0},\\
 &p=0, \ q=1, \ {\bf r=0, \ s=0}, \\
 &p=0, \ q=1, \ {\bf r=0},\ s=s, \ {\bf r_1 = \frac{-n}{\alpha-m-n}},\\
 &p=0, \ q=1, \ r = \frac{n\alpha - r_1(\alpha-m-n)}{(1+\alpha)(m+n)},\ {\bf s=0},\\
 &p=p, \ q=0, \ {\bf r=0, \ s=0}, \ \frac{r_0}{\lambda}=2, \\ 
 &p=p, \ q=1, \ {\bf r=0, \ s=0}, \ \frac{r_0}{\lambda}=1, \\
 &p=p, \ q=0, \ {\bf r=0}, \ s=s, \ \frac{r_0}{\lambda}=2, \ {\bf r_0 = \frac{-n}{\alpha-m-n}}, \\
 &p=p, \ q=1, \ {\bf r=0}, \ s=s, \ \frac{r_0}{\lambda}=1, \ {\bf r_1 = \frac{-n}{\alpha-m-n}}, \\
 &{\bf p=\#<0}, \ q=\frac{2(\alpha-m-n)}{1+m}, \ r=a_0, \ s=\frac{1+m+n}{1+\alpha} - \frac{n}{(1+\alpha)a_0},\\
 &p=\#, \ q=\#, \ r = \frac{2-n}{1-m-n}, {\bf s=0}.
\end{align*}

Coefficients Matrices $Mat_i$ for Linearized equations around $M_i$, $i=0,1,2,3$: 
\begin{align*}
 Mat_0 &= \begin{pmatrix}
          2 & 0 & 0 & 0 \\
          br_0 & 1 & 0 & 0\\
          \frac{r_0}{n}(\lambda r_0) & \frac{r_0}{n} & \frac{r_0}{n}\Big(\frac{\alpha-m-n}{\lambda(1+\alpha)} - \frac{n\alpha}{\lambda(1+\alpha)r_0}\Big) & \frac{r_0}{n}(\frac{\alpha r_0}{\lambda})\\
          s_0(\lambda r_0) & s_0 & s_0\Big(\frac{\alpha-m-n}{\lambda(1+\alpha)} + \frac{n}{\lambda(1+\alpha)r_0}\Big) & s_0(-\frac{r_0}{\lambda})
         \end{pmatrix}
\end{align*}
\begin{equation}
 \mu_{01}=2, \ \mu_{02}=1, \ \mu_{03}=\mu_0^+, \ \mu_{04}=\mu_0^-,
\end{equation}
where $\mu_0^\pm$ is the two roots of the 
$$ \mu^2 - \mu\Big(\frac{r_0}{n}\frac{\alpha-m-n}{\lambda(1+\alpha)} - \frac{r_0s_0}{\lambda} - \frac{\alpha}{\lambda(1+\alpha)}\Big) - \frac{r_0}{n}\frac{r_0s_0}{\lambda}(1+\alpha) = 0.$$
\begin{align*}
 \mu_0^+ &= \frac{A}{n} + D + \frac{BC}{A} + \cdots, \quad \mu_0^- = - \frac{BC}{A} + \cdots,
\end{align*}
where
\begin{align*}
 \begin{pmatrix} \frac{A}{n} & \frac{B}{n} \\ C & D \end{pmatrix} = 
 \begin{pmatrix}
 \frac{r_0}{n}\Big(\frac{\alpha-m-n}{\lambda(1+\alpha)} - \frac{n\alpha}{\lambda(1+\alpha)r_0}\Big) & \frac{r_0}{n}(\frac{\alpha r_0}{\lambda})\\
 s_0\Big(\frac{\alpha-m-n}{\lambda(1+\alpha)} + \frac{n}{\lambda(1+\alpha)r_0}\Big) & s_0(-\frac{r_0}{\lambda})
 \end{pmatrix}.
\end{align*}

Four eigenvectors are collected in the matrix:
\begin{align*}
 S_0&=
 \begin{pmatrix}
    1 & 0 & 0 & 0\\
    br_0 & 1 & 0 & 0\\
    y_1 & y_2 & 1 & y_3\\
    z_1 & z_2 & z_3 &1
 \end{pmatrix},
\end{align*}
where
\begin{align*}
 y_1&=-\frac{(\lambda+b)r_0}{\frac{\alpha-m-n}{\lambda(1+\alpha)} - \frac{2n}{r_0(1+\alpha)}\Big(1 + \frac{(\frac{1}{\lambda}+2)\frac{\alpha}{s}}{ \frac{1+\alpha}{\lambda}r + \frac{2}{s} }\Big) }\\
 y_2&=-\frac{1}{\frac{\alpha-m-n}{\lambda(1+\alpha)} - \frac{n}{r_0(1+\alpha)}\Big(1 + \frac{(\frac{1}{\lambda}+1)\frac{\alpha}{s}}{ \frac{1+\alpha}{\lambda}r + \frac{1}{s} }\Big) }\\
 y_3&=\frac{\mu_0^- +\frac{r_0s_0}{\lambda}}{s\Big(\frac{\alpha-m-n}{\lambda(1+\alpha)} + \frac{n}{\lambda(1+\alpha)r_0}\Big)}\\
 z_1&=n\bigg(\frac{\big(\frac{1}{\lambda}+2\big)\frac{1}{r}}{ \frac{1+\alpha}{\lambda}r + \frac{2}{s} }\bigg)y_1 \\
 z_2&=n\bigg(\frac{\big(\frac{1}{\lambda}+1\big)\frac{1}{r}}{ \frac{1+\alpha}{\lambda}r + \frac{1}{s} }\bigg)y_2 \\
 z_3&=n\bigg(\mu_0^+-\frac{r_0}{n}\Big(\frac{\alpha-m-n}{\lambda(1+\alpha)} - \frac{n\alpha}{\lambda(1+\alpha)r_0}\Big)\bigg)r_0(\frac{\alpha r_0}{\lambda})\bigg)
\end{align*}

\hrulefill
\begin{align*}
 Mat_1 &= \begin{pmatrix}
          -\frac{1+m+n}{\alpha-m-n} & 0 & 0 & 0\\
          (b-\lambda)r_1 & -1 & 0 & 0\\
          \frac{r_1}{n}(\lambda r_1) & \frac{r_1}{n} & \frac{r_1}{n}\Big(\frac{\alpha-m-n}{\lambda(1+\alpha)} - \frac{n\alpha}{\lambda(1+\alpha)r_1}\Big) & \frac{r_1}{n}(\frac{\alpha r_1}{\lambda})\\
          s_1(\lambda r_1) & s_1 & s_1\Big(\frac{\alpha-m-n}{\lambda(1+\alpha)} + \frac{n}{\lambda(1+\alpha)r_1}\Big) & s_1(-\frac{r_1}{\lambda})
         \end{pmatrix}
\end{align*}
\begin{equation}
 \mu_{11}=-\frac{1+m+n}{\alpha-m-n}, \ \mu_{12}=-1, \ \mu_{13}=\mu_1^+, \ \mu_{14}=\mu_1^-,
\end{equation}
where $\mu_1^\pm$ is the two roots of the 
$$ \mu^2 - \mu\Big(\frac{r_1}{n}\frac{\alpha-m-n}{\lambda(1+\alpha)} - \frac{r_1s_1}{\lambda} - \frac{\alpha}{\lambda(1+\alpha)}\Big) - \frac{r_1}{n}\frac{r_1s_1}{\lambda}(1+\alpha) = 0.$$
\begin{align*}
 \mu_1^+ &= \frac{A}{n} + D + \frac{BC}{A} + \cdots, \quad \mu_1^- = - \frac{BC}{A} + \cdots,
\end{align*}
where
\begin{align*}
 \begin{pmatrix} \frac{A}{n} & \frac{B}{n} \\ C & D \end{pmatrix} = 
 \begin{pmatrix}
 \frac{r_1}{n}\Big(\frac{\alpha-m-n}{\lambda(1+\alpha)} - \frac{n\alpha}{\lambda(1+\alpha)r_1}\Big) & \frac{r_1}{n}(\frac{\alpha r_1}{\lambda})\\
 s_1\Big(\frac{\alpha-m-n}{\lambda(1+\alpha)} + \frac{n}{\lambda(1+\alpha)r_1}\Big) & s_1(-\frac{r_1}{\lambda})
 \end{pmatrix}.
\end{align*}

{\bf case 1: $b=\lambda$}
Four eigenvectors are collected in the matrix:
\begin{align*}
 S_1&=
 \begin{pmatrix}
    1 & 0 & 0 & 0\\
    0 & 1 & 0 & 0\\
    y_1 & y_2 & 1 & y_3\\
    z_1 & z_2 & z_3 &1
 \end{pmatrix},
\end{align*}
where
\begin{align*}
 y_1&=-\frac{\lambda r_1}{\frac{\alpha-m-n}{\lambda(1+\alpha)} + n\frac{1+m+n}{\alpha-m-n}\frac{1}{r_0(1+\alpha)}\Big(1 + \frac{(\frac{1}{\lambda}-\frac{1+m+n}{\alpha-m-n})\frac{\alpha}{s}}{ \frac{1+\alpha}{\lambda}r -\frac{1+m+n}{\alpha-m-n} \frac{1}{s} }\Big) }\\
 y_2&=-\frac{1}{\frac{\alpha-m-n}{\lambda(1+\alpha)} + \frac{n}{r_1(1+\alpha)}\Big(1 + \frac{(\frac{1}{\lambda}-1)\frac{\alpha}{s}}{ \frac{1+\alpha}{\lambda}r - \frac{1}{s} }\Big) }\\
 y_3&=\frac{\mu_1^- +\frac{r_1s_1}{\lambda}}{s\Big(\frac{\alpha-m-n}{\lambda(1+\alpha)} + \frac{n}{\lambda(1+\alpha)r_1}\Big)}\\
 z_1&=n\bigg(\frac{\big(\frac{1}{\lambda}-\frac{1+m+n}{\alpha-m-n}\big)\frac{1}{r}}{ \frac{1+\alpha}{\lambda}r -\frac{1+m+n}{\alpha-m-n} \frac{1}{s} }\bigg)y_1 \\
 z_2&=n\bigg(\frac{\big(\frac{1}{\lambda}-1\big)\frac{1}{r}}{ \frac{1+\alpha}{\lambda}r - \frac{1}{s} }\bigg)y_2 \\
 z_3&=n\bigg(\mu_1^+-\frac{r_1}{n}\Big(\frac{\alpha-m-n}{\lambda(1+\alpha)} - \frac{n\alpha}{\lambda(1+\alpha)r_1}\Big)\bigg)r_1(\frac{\alpha r_1}{\lambda})\bigg)
\end{align*}
{\bf case 2: $b\ne\lambda$ and $-\frac{1+m+n}{\alpha-m-n}\ne -1$}
Four eigenvectors are collected in the matrix:
\begin{align*}
 S_1&=
 \begin{pmatrix}
    1 & 0 & 0 & 0\\
    \frac{(b-\lambda)r_1}{1-\frac{1+m+n}{\alpha-m-n}} & 1 & 0 & 0\\
    y_1 & y_2 & 1 & y_3\\
    z_1 & z_2 & z_3 &1
 \end{pmatrix},
\end{align*}
where
\begin{align*}
 y_1&=-\frac{\Big(\lambda + \frac{(b-\lambda)}{1-\frac{1+m+n}{\alpha-m-n}}\Big)r_1}{\frac{\alpha-m-n}{\lambda(1+\alpha)} + n\frac{1+m+n}{\alpha-m-n}\frac{1}{r_0(1+\alpha)}\Big(1 + \frac{(\frac{1}{\lambda}-\frac{1+m+n}{\alpha-m-n})\frac{\alpha}{s}}{ \frac{1+\alpha}{\lambda}r -\frac{1+m+n}{\alpha-m-n} \frac{1}{s} }\Big) }\\
 y_2&=-\frac{1}{\frac{\alpha-m-n}{\lambda(1+\alpha)} + \frac{n}{r_1(1+\alpha)}\Big(1 + \frac{(\frac{1}{\lambda}-1)\frac{\alpha}{s}}{ \frac{1+\alpha}{\lambda}r - \frac{1}{s} }\Big) }\\
 y_3&=\frac{\mu_1^- +\frac{r_1s_1}{\lambda}}{s\Big(\frac{\alpha-m-n}{\lambda(1+\alpha)} + \frac{n}{\lambda(1+\alpha)r_1}\Big)}\\
 z_1&=n\bigg(\frac{\big(\frac{1}{\lambda}-\frac{1+m+n}{\alpha-m-n}\big)\frac{1}{r}}{ \frac{1+\alpha}{\lambda}r -\frac{1+m+n}{\alpha-m-n} \frac{1}{s} }\bigg)y_1 \\
 z_2&=n\bigg(\frac{\big(\frac{1}{\lambda}-1\big)\frac{1}{r}}{ \frac{1+\alpha}{\lambda}r - \frac{1}{s} }\bigg)y_2 \\
 z_3&=n\bigg(\mu_1^+-\frac{r_1}{n}\Big(\frac{\alpha-m-n}{\lambda(1+\alpha)} - \frac{n\alpha}{\lambda(1+\alpha)r_1}\Big)\bigg)r_1(\frac{\alpha r_1}{\lambda})\bigg)
\end{align*}


{\bf case 3: $b\ne\lambda$ and $-\frac{1+m+n}{\alpha-m-n}= -1$}

Generalized eigenvector
\begin{align*}
NONO\\
 y_1 &= \frac{-\frac{1}{1+\alpha}\Big(\frac{n}{r}y_2 +\frac{\alpha}{s}z_2\Big) + br + 1}{\frac{\alpha-m-n}{\lambda(1+\alpha)}}\\
 z_1 &= \frac{-\Big(\frac{n}{r}y_2 +\frac{1}{s}z_2\Big)+ \frac{n}{r}\frac{1}{\lambda}y_1}{\frac{1+\alpha}{\lambda}r}
\end{align*}



% \begin{align*}
%  &X_{01} = \bigg( \Big( \frac{2n - \frac{\alpha-n}{\lambda}r_0^{1+n}}{\big({\lambda}+b\big) r_0^2}\Big) \;,\;\Big( \frac{2n - \frac{\alpha-n}{\lambda}r_0^{1+n}}{\big({\lambda}+b\big) r_0^2}\Big)br_0\;,\;1\bigg),\quad
%  X_{02} = \bigg(0, \Big(\frac{n- \frac{\alpha-n}{\lambda}r_0^{1+n}}{r_0}\Big), 1\bigg), \quad
%  X_{03} = (0,0,1),\\
%  &X_{11} = \bigg(  \Big(\frac{-n\frac{1+n}{\alpha-n} - \frac{\alpha-n}{\lambda}r_1^{1+n}}{\big(-\frac{1+n}{\alpha-n} \lambda +b\big) r_1^2}\Big)\Big(1-\frac{1+n}{\alpha-n}\Big) \;,\;\Big(\frac{-n\frac{1+n}{\alpha-n} - \frac{\alpha-n}{\lambda}r_1^{1+n}}{\big(-\frac{1+n}{\alpha-n} \lambda +b\big) r_1^2}\Big)(b-\lambda)r_1\;,\;1\bigg),\\
%  &X_{12} = \bigg(0, \Big(\frac{n- \frac{\alpha-n}{\lambda}r_0^{1+n}}{r_0}\Big), 1\bigg), \quad
%  X_{13} = (0,0,1),
% \end{align*}

\section{Characterization of the heteroclinic orbit}
\subsection{Asymptotic behavior of self-similar variables in $\xi$}
\pagebreak
\section{Existence via Geometric theory of singular perturbation}
In slow time scale $\eta$,
\begin{equation}
 \begin{aligned}
 \dot{p} &=p\Big(\frac{1}{\lambda}(r-a) + 2- \lambda p r -q\Big),\\
 \dot{q} &=q\Big(1 -\lambda p r -q\Big) + b p r,\\
 n\dot{r} &=r\Big(\frac{\alpha-m-n}{\lambda(1+\alpha)}(r-a) + \lambda pr + q +\frac{\alpha}{\lambda}r\big(s- \frac{1+m+n}{1+\alpha}\big) + \frac{n\alpha}{\lambda(1+\alpha)}\Big),\\
 \dot{s} &=s\Big(\frac{\alpha-m-n}{\lambda(1+\alpha)}(r-a) + \lambda pr + q - \frac{1}{\lambda}r\big(s- \frac{1+m+n}{1+\alpha}\big) - \frac{n}{\lambda(1+\alpha)}\Big).
 \end{aligned}
\end{equation}
In fast time scale $\tilde\eta=\frac{\eta}{n}$,
\begin{equation} \label{eq:fast}
 \begin{aligned}
 {p}' &=np\Big(\frac{1}{\lambda}(r-a) + 2- \lambda p r -q\Big),\\
 {q}' &=nq\Big(1 -\lambda p r -q\Big) + nb p r,\\
 {r}' &=r\Big(\frac{\alpha-m-n}{\lambda(1+\alpha)}(r-a) + \lambda pr + q +\frac{\alpha}{\lambda}r\big(s- \frac{1+m+n}{1+\alpha}\big) + \frac{n\alpha}{\lambda(1+\alpha)}\Big),\\
 {s}' &=ns\Big(\frac{\alpha-m-n}{\lambda(1+\alpha)}(r-a) + \lambda pr + q - \frac{1}{\lambda}r\big(s- \frac{1+m+n}{1+\alpha}\big) - \frac{n}{\lambda(1+\alpha)}\Big).
 \end{aligned}
\end{equation}
\eqref{eq:slow} at $n=0$:
\begin{equation}\label{eq:slow0}
 \begin{aligned}
 r &=\hat{r}(p,q,s,n=0) \triangleq \frac{ \frac{\alpha-m}{\lambda(1+\alpha)}a - q }{  \frac{\alpha-m}{\lambda(1+\alpha)} + \lambda p + \frac{\alpha}{\lambda}\big(s- \frac{1+m}{1+\alpha}\big)},\\% \quad \text{$=\hat{r}(0)$ for simplicity },\\
 \dot{p} &=p\Big(\frac{1}{\lambda}(\hat{r}-a) + 2- \lambda p \hat{r} -q\Big) & &= p\Big(\frac{D}{\lambda(1+\alpha)}(\hat{r}-a_0)\Big),\\
 \dot{q} &=q\Big(1 -\lambda p \hat{r} -q\Big) + b p \hat{r} & &=q\Big(\frac{\alpha-m}{\lambda(1+\alpha)}(\hat{r}-r_1)\Big) + b p \hat{r},\\
 \dot{s} &=s\Big(\frac{\alpha-m}{\lambda(1+\alpha)}(\hat{r}-a) + \lambda p\hat{r} + q - \frac{1}{\lambda}\hat{r}\big(s- \frac{1+m}{1+\alpha}\big)\Big) &&= -\frac{1+\alpha}{\lambda}\hat{r}s\big(s- \frac{1+m}{1+\alpha}\big).
 \end{aligned}
\end{equation}
\eqref{eq:fast} at $n=0$:
\begin{equation} \label{eq:fast0}
 \begin{aligned}
 {p}' &=0,\\
 {q}' &=0,\\
 {r}' &=r\Big(\frac{\alpha-m}{\lambda(1+\alpha)}(r-a) + \lambda pr + q +\frac{\alpha}{\lambda}r\big(s- \frac{1+m}{1+\alpha}\big)\Big),\\
 {y}' &=0.
 \end{aligned}
\end{equation}

% \subsection{Notion of Normally Hyperbolicity and the stable manifold theorem}
% $x\in \mathbb{R}^n$: slow variables; $y \in \mathbb{R}^m$: fast variables with respect to a small parameter $\epsilon$.
\begin{definition}[fast-slow case definition of normally hyperbolicity of invariant manifold]
A compact set $\Lambda$ is called normally hyperbolic if the $m\times m$ matrix $(D_y f)(p)$ of first partial derivatives with respect to the fast variables $y$ has no eigenvalues with zero real part for all $p \in \Lambda$.
\end{definition}
% 
% 
% \subsection{Notion of Hyperbolic Set and the (topological) stable manifold theorem}
% M.W. Hirsh, C.C. Pugh, {\it Stable manifolds and hyperbolic sets}, in Global Analysis \it{Proc. Symps. Pure Math.} {\bf 14}, Berkeley, Calif., 1968.
% \begin{definition}[Hyperbolic set]
% The invariant set $\Lambda \subset U$ is hyperbolic for the map $f: U \rightarrow M$ if $T_\Lambda M$ has a splitting (Whitney sum decomposition) $T_\Lambda M = E^u \oplus E^s$ satisfying:
% \begin{enumerate}
%  \item $E^u$ and $E^s$ are invariant under the bundle map $Tf$;
%  \item there exist constants $c>0$ and $0<\tau <1$ such that for all $n \in \mathbb{Z}_+$,
%  $$ max\{ ||Tf^n|_{E^s}||,~||Tf^{-n}|_{E^u}||\} < c\tau^n. $$
% \end{enumerate}
% \end{definition}
% 
% \begin{theorem}[Theorem 7.3 in HP68, Structural stability] Let $\Lambda \subset U$ be a compact hyperbolic set for the $C^1$ embedding $f: U\subset M \rightarrow M$. Given $\epsilon>0$ there is a compact neighborhood $V\subset U$ of $\Lambda$ and a neighborhood $\mathcal{N}$ of the map $f$ in $C^1(U,M)$ with the following properties: if $g_i\in N$ for $i=1,2$ then $g_i$ has a unique maximal hyperbolic set $\Lambda_i \subset V$ containing every invariant set of $g_i$ in $V$; and there is a unique homeomorphism $h_1: \Lambda_1 \rightarrow \Lambda_2$ such that $h_1 g_1 h_1^{-1} = g_2|_\Lambda$; and $d(h_1,1) \le \epsilon$. Moreover $h_1$ depends continuously on $(g_1,g_2) \in \mathcal{N}\times\mathcal{N}$.
% \end{theorem}


\subsection{Proof steps}

$3$-dimensional trapezoid $K$: ($\underbar{r}$ and $0<A<1$ is determined in the below later.)
\begin{align*}
 K &\triangleq \left\{ \: (p,q,s) \: | \:  p\ge0, ~~ q\ge0, ~~ \left|s-\frac{1+m}{1+\alpha}\right| \le A \frac{\alpha-m}{\alpha(1+\alpha)}, ~~ \hat{r}(p,q,s,0)\ge \underbar{r}\: \right\}.
\end{align*}
Take $\tilde{K} \supset\supset K$, but sufficiently tight.
 %\right. \left.\frac{\alpha-m}{\lambda(1+\alpha)}(\underbar{r}-a) + \lambda p\underbar{r} + q +\frac{\alpha}{\lambda}\underbar{r}\big(y- \frac{1+m}{1+\alpha}\big) \le 0.\right\},\\
\begin{align*}
 G_0 &\triangleq \left\{ \: (p,q,r,s) \: | \: r=\hat{r}(p,q,s,0), \quad (p,q,s)\in \tilde{K}. \: \right\}.
\end{align*}

The level-surface $\hat{r}(p,q,s,0) = \underbar{r}$ is an affine surface in $(p,q,s)$-space.
\begin{equation}
 \begin{aligned}
  \lambda \underbar{r}p + q +\frac{\alpha}{\lambda}\underbar{r}\big(s- \frac{1+m}{1+\alpha}\big) + \frac{\alpha-m}{\lambda(1+\alpha)}(\underbar{r}-a) = 0,\\
  \Longleftrightarrow \lambda \underbar{r}p + q-1 +\frac{\alpha}{\lambda}\underbar{r}\big(s- \frac{1+m}{1+\alpha}\big) + \frac{\alpha-m}{\lambda(1+\alpha)}(\underbar{r}-r_1) = 0
 \end{aligned}
\end{equation}


\begin{proposition} 
$G_0$ is normally hyperbolic with respect to the flow \eqref{eq:fast0}. 
\end{proposition}
\begin{proof}
 Let $f(p,q,r,s) \triangleq r\Big(\frac{\alpha-m}{\lambda(1+\alpha)}(r-a) + \lambda pr + q +\frac{\alpha}{\lambda}r\big(s- \frac{1+m}{1+\alpha}\big)\Big)$, the right hand side of $\eqref{eq:fast0}_3$. Then 
 \begin{align*}
 \frac{\partial f}{\partial r} &= \Big(\frac{\alpha-m}{\lambda(1+\alpha)}(\hat{r}-a) + \lambda p\hat{r} + q +\frac{\alpha}{\lambda}\hat{r}\big(s- \frac{1+m}{1+\alpha}\big)\Big) + \hat{r}\Big(\frac{\alpha-m}{\lambda(1+\alpha)} + \lambda p + \frac{\alpha}{\lambda}\big(s- \frac{1+m}{1+\alpha}\big)\Big)\\
 &= \hat{r}\Big(\frac{\alpha-m}{\lambda(1+\alpha)} + \frac{\alpha}{\lambda}\big(s- \frac{1+m}{1+\alpha}\big) + \lambda p\Big)\ge \underbar{r}\Big(A\underbar{r}\frac{\alpha-m}{\lambda(1+\alpha)} + \lambda p\Big)>0.
 \end{align*}
\end{proof}
\subsection{Reduced flow \eqref{eq:slow0} for $n=0$}
\begin{lemma}[$n=0$] 
The set $K$ is positively invariant to the flow \eqref{eq:slow0}.
% \begin{enumerate}
%  \item The set $K$ is positively invariant.
%  \item Let $T$ be a triangle $\{s=\frac{1+m}{1+\alpha}\} \cap K$. Then $T$ is invariant. Furthermore, $T\backslash B_\delta(M_0) \subset W^s(M_1)$.
%  \item $K\backslash B_\delta(M_0) \subset W^s(M_1)$ either.
% \end{enumerate}
\end{lemma}
\begin{proof}
 We calculate the inner product of the vector fields with inward normal vector on boundary.
 (1) on $p=0$ plane, $\nu = (1,0,0)$ and $X\cdot\nu = \dot{p}=0$;\\
 (2) on $q=0$ plane, $\nu = (0,1,0)$ and $X\cdot\nu=\dot{q} = bpr\ge0$,\\
 (3) on $\pm\left(s-\frac{1+m}{1+\alpha}\right) = A \frac{\alpha-m}{\alpha(1+\alpha)}$ plane, $\nu = (0,0,\mp1)$. First, note that $s>\frac{m}{\alpha}>0$ in $K$. Now $X\cdot\nu = \mp\Big(-\frac{1+\alpha}{\lambda}\hat{r}s\big(s- \frac{1+m}{1+\alpha}\big)\Big) = \frac{1+\alpha}{\lambda}\hat{r}sA \frac{\alpha-m}{\alpha(1+\alpha)}\ge\delta_0>0$.\\
 (4) on the affine plane $\hat{r}=\underbar{r}$, $\nu = (-\lambda \underbar{r}, -1,-\frac{\alpha}{\lambda}\underbar{r})$ and 
 \begin{align}
  -\lambda \underbar{r}\dot{p} -\dot{q}-\frac{\alpha}{\lambda}\underbar{r}\dot{s} &= -\lambda \underbar{r}p \Big(1-\lambda \underbar{r}p -q + \frac{1}{\lambda}(\underbar{r}-a)+1\Big) - q(1-\lambda \underbar{r}p -q\big) - \lambda \underbar{r}p\Big(\frac{b}{\lambda}\Big) + \frac{\alpha}{\lambda}\underbar{r}\frac{1+\alpha}{\lambda}\underbar{r}s\big(s- \frac{1+m}{1+\alpha}\big)\nonumber\\
  &= (1-\lambda \underbar{r}p -q)^2 - (1-\lambda \underbar{r}p -q) -\lambda \underbar{r}p\Big(\frac{1}{\lambda}(\underbar{r}-a)+1+\frac{b}{\lambda}\Big) + \frac{\alpha}{\lambda}\underbar{r}\frac{1+\alpha}{\lambda}\underbar{r}s\big(s- \frac{1+m}{1+\alpha}\big)\nonumber\\
  &= (1-\lambda \underbar{r}p -q)^2 - \frac{\alpha-m}{\lambda(1+\alpha)}(\underbar{r}-r_1) -\frac{\alpha}{\lambda}\underbar{r}\big(s- \frac{1+m}{1+\alpha}\big) -\underbar{r}p(\underbar{r}-1)+ \frac{\alpha}{\lambda}\underbar{r}\frac{1+\alpha}{\lambda}\underbar{r}s\big(s- \frac{1+m}{1+\alpha}\big)\nonumber\\
  &= (1-\lambda \underbar{r}p -q)^2 - \frac{\alpha-m}{\lambda(1+\alpha)}(\underbar{r}-r_1) -\underbar{r}p(\underbar{r}-1) + \frac{\alpha}{\lambda}\underbar{r}\big(s- \frac{1+m}{1+\alpha}\big)\Big(-1+\frac{1+\alpha}{\lambda}\underbar{s}\Big)\nonumber\\
  &= (1-\lambda \underbar{r}p -q)^2 - \frac{\alpha-m}{\lambda(1+\alpha)}(\underbar{r}-r_1) -\underbar{r}p(\underbar{r}-1) + \Big(\frac{\alpha}{\lambda}\underbar{r}\big(s- \frac{1+m}{1+\alpha}\big)\Big)^2\Big(\frac{1+\alpha}{\alpha}\Big) \nonumber\\
  &+ \Big(\frac{\alpha}{\lambda}\underbar{r}\big(s- \frac{1+m}{1+\alpha}\big)\Big)\Big(-1+ \frac{1+m}{\lambda}\Big)\nonumber\\
  &\ge (1-\lambda \underbar{r}p -q)^2 -\underbar{r}p(\underbar{r}-1) + \Big(\frac{\alpha}{\lambda}\underbar{r}\big(s- \frac{1+m}{1+\alpha}\big)\Big)^2\Big(\frac{1+\alpha}{\alpha}\Big) \nonumber\\
  &- \Big(\frac{\alpha-m}{\lambda(1+\alpha)}\Big)(\underbar{r}-r_1) - \Big(\frac{\alpha-m}{\lambda(1+\alpha)}\Big)\underbar{r}\Big|\frac{\alpha(1+\alpha)}{\alpha-m}\big(s- \frac{1+m}{1+\alpha}\big)\Big|\Big|-1+ \frac{1+m}{\lambda}\Big|\nonumber\\
  &\ge (1-\lambda \underbar{r}p -q)^2  -\underbar{r}p(\underbar{r}-1) + \Big(\frac{\alpha}{\lambda}\underbar{r}\big(s- \frac{1+m}{1+\alpha}\big)\Big)^2\Big(\frac{1+\alpha}{\alpha}\Big) \nonumber\\
  &- \Big(\frac{\alpha-m}{\lambda(1+\alpha)}\Big)\Big((1+A)\underbar{r}-r_1\Big) \ge \delta_1>0, \quad \text{if $0<\underbar{r} < \textrm{min}\,\left\{1, \frac{r_1}{1+A}\right\}$.} \label{eq:affine}
 \end{align}
\end{proof}
\begin{lemma}[$n=0$] 
$\textrm{int}\, K \subset W^s(M_1)$.
% \begin{enumerate}
%  \item The set $K$ is positively invariant.
%  \item Let $T$ be a triangle $\{s=\frac{1+m}{1+\alpha}\} \cap K$. Then $T$ is invariant. Furthermore, $T\backslash B_\delta(M_0) \subset W^s(M_1)$.
%  \item $K\backslash B_\delta(M_0) \subset W^s(M_1)$ either.
% \end{enumerate}
\end{lemma}
\begin{proof}
 First, we prove $\textrm{int}\, T \subset W^s(M_1)$, where $T$ is the triangle $\{s=\frac{1+m}{1+\alpha}\} \cap K$. Note that the plane $s=\frac{1+m}{1+\alpha}$ is invariant to \eqref{eq:slow0}. Let $x$ be an interior point of $T$. Since $T$ is positively invariant and thus the $\omega$-limit set of $x$ is in $T$ and non-empty, but it cannot contain the limit cycle because there is no critical point in interior of $T$. By Poincar\'e-Bendixson Theorem, the $\omega$-limit set of $x$ consists of critical points. However, $M_0|_T$ is an unstable node, so the $\omega$-limit set coincides with a single point $M_1$.
 
 Now, since 
 $$\frac{\big(s- \frac{1+m}{1+\alpha}\big)^{\cdot}}{\big(s- \frac{1+m}{1+\alpha}\big)} = -\frac{1+\alpha}{\lambda}\hat{r}s\le -A$$
 for the uniform constant $A>0$ in $K$, $s$ relaxes to the triangle $T$ as $\eta \rightarrow \infty$. If the orbit point is sufficiently close to the $T$, then it has to arrive at a point arbitrarily close to $M_1$ in a finite time because the point in $T$ does so and the time-$\eta$ flow map is $C^1$. By the stable manifold theorem at $M_1$, the orbit then converges to $M_1$ as $\eta \rightarrow \infty$.
\end{proof}
\begin{corollary}
 Every point in $W^u(M_0)$ emanated into $T$ is a heteroclinic orbit connecting $M_0$ to $M_1$.
\end{corollary}

\subsection{Persistence for $n>0$}
By the geometric singular perturbation theorem, for sufficiently small $n$, $\exists \hat{r}(p,q,s,n)$ defined on the $\tilde{K}$ such that
$$G_n\triangleq \{\:(p,q,r,s)\: | \: r=\hat{r}(p,q,s,n), \quad (p,q,s)\in \tilde{K}\:\}$$
is locally invariant to \eqref{eq:slow} and $\hat{r}$ is jointly smooth in $(p,q,s)$ and $n$.
We define the reduced problem,
\begin{equation} \label{eq:reduced}
 \begin{aligned}
 \dot{p} &=p\Big(\frac{1}{\lambda}(\hat{r}-a) + 2- \lambda p r -q\Big),\\
 \dot{q} &=q\Big(1 -\lambda p \hat{r} -q\Big) + b p \hat{r},\\
 \dot{s} &=s\Big(\frac{\alpha-m-n}{\lambda(1+\alpha)}(\hat{r}-a) + \lambda pr + q - \frac{1}{\lambda}\hat{r}\big(s- \frac{1+m+n}{1+\alpha}\big) - \frac{n}{\lambda(1+\alpha)}\Big).
 \end{aligned}
\end{equation}
Let $f_j(p,q,s,n)$, $j=1,2,3$ be the vector field of \eqref{eq:reduced}.
Due to the smoothness of $\hat{r}$ and $f_j$ in $n$, 
\begin{equation} \label{eq:uniform}
 \begin{aligned}
f_j(p,q,s,n) = f_j(p,q,s,0) +\mathcal{O}(n),\\
 \lambda \hat{r}(n)p + q +\frac{\alpha}{\lambda}\hat{r}(n)\big(s- \frac{1+m+n}{1+\alpha}\big) + \frac{\alpha-m-n}{\lambda(1+\alpha)}(\hat{r}(n)-a) = \mathcal{O}(n),\\
   \lambda \hat{r}(n)p + q-1 +\frac{\alpha}{\lambda}\hat{r}(n)\big(s- \frac{1+m+n}{1+\alpha}\big) + \frac{\alpha-m-n}{\lambda(1+\alpha)}(\hat{r}(n)-r_1) = \mathcal{O}(n),
 \end{aligned}
\end{equation}
uniformly in $K$.


\begin{lemma}[$n>0$]
$K$ is positively invariant to $\eqref{eq:reduced}$ for $n>0$ sufficiently small.
\end{lemma}
\begin{proof}
We check the inward normal component of the flow.
 On plane (1) $p=0$ and (2) $q=0$, inward normal component is nonnegative for the same reason as before. On plane (3) $\pm\left(s-\frac{1+m}{1+\alpha}\right) = A \frac{\alpha-m}{\alpha(1+\alpha)}$ and the affine plane (4), inward normal component is nonnegative provided $n$ is sufficiently small because \eqref{eq:uniform} holds and $\delta_0$ and $\delta_1$ are fixed constants away from $0$.
\end{proof}
 $s = \frac{1+m+n}{1+\alpha}$ plane is not anymore invariant plane and thus we cannot confine the flow on $2$-dimensional object. Instead, we analyze the flow in a thin slab $|s - \frac{1+m+n}{1+\alpha}| \le Cn$. Because it is $3$-dimensional, we cannot apply the Poincar\'e-Bendixson Theorem anymore. We directly analyze the flow here.

% \begin{claim} 
% $\exists C>0$ such that $K$ contracts into a thin slab $|s - \frac{1+m+n}{1+\alpha}| \le Cn$ as $\eta \rightarrow \infty$.
% \end{claim}
% \begin{proof}
%  Because of \eqref{eq:uniform}, we can write $\eqref{eq:reduced}_3$ as
%  $$ \dot{s} = -\frac{1+\alpha}{\lambda}\hat{r}(n)s\big(s- \frac{1+m+n}{1+\alpha}\big) + \mathcal{O}(n).$$
%  Then for each plane $\pm\big(s - \frac{1+m+n}{1+\alpha}\big) = A$, the flow is strictly inward provided $A>Cn$ for some $C>0$.
% \end{proof}
\begin{lemma}[$n>0$] \label{lem:3dstable}
$\textrm{int}\, K \subset W^s(M_1(n))$.
\end{lemma}
\begin{claim}
$\exists C>0$ that does not depend on $n$ and $(p,q,s)$ such that $K$ contracts into a $3$-dimensional trapezoid 
$$ Z\triangleq \left\{ \: (p,q,s) \: | \:  p\ge0, ~~ q\ge 1-Cn, ~~ \left|s-\frac{1+m}{1+\alpha}\right| \le Cn, ~~ \hat{r}(p,q,s,0)\ge \frac{r_1}{1+Cn}\: \right\}$$
\end{claim}
\begin{proof}
Note that the orbit cannot escape $K$ because of the Claim 3. 

First, We show that $K$ contracts into a thin slab $|s - \frac{1+m+n}{1+\alpha}| \le Cn$ as $\eta \rightarrow \infty$. 
Because of \eqref{eq:uniform}, we can re-write $\eqref{eq:reduced}_3$ as
 $$ \dot{s} = -\frac{1+\alpha}{\lambda}\hat{r}(n)s\big(s- \frac{1+m+n}{1+\alpha}\big) + \mathcal{O}(n).$$
Then for each plane $\pm\big(s - \frac{1+m+n}{1+\alpha}\big) = A\frac{\alpha-m}{\alpha(1+\alpha)}$, the flow is strictly toward the plane $s- \frac{1+m+n}{1+\alpha}=0$, provided $A\ge Cn$ for some $C>0$. 



For each of the affine plane $\hat{r}(p,q,s,0)=r$ restricted in this thin slab, the $\delta_1>0$ away from $0$ in \eqref{eq:affine} can be taken with $A=Cn$, or the flow \eqref{eq:slow0} for $n=0$ is strictly toward origin as long as $r < \frac{r_1}{1+A} \le \frac{r_1}{1+Cn}$. Since $\delta_1$ is away from $0$, if $n$ is sufficiently small, the same holds for the flow \eqref{eq:reduced} for $n>0$.

Now, consider the equation $\eqref{eq:reduced}_2$. Suppose that $\lambda\le b(n)$. Then so long as $q<1$, \\$\dot{q} \ge q(1-q)>0$. The proof for the claim is done.

Now, we are left with the remaining case $\lambda > b(n)$. First, that $\lambda > b(n)$ and $r_1(n)>0$ implies a few combinatorial consequences that
\begin{equation}\label{eq:combi}
 \begin{aligned}
  &\frac{1+m}{2(\alpha-m-n)} < \lambda < \frac{\alpha-m-n}{1+m+n}\Big(\frac{2+2\alpha-n}{1+\alpha}\Big),\\
  &a_0(n)-1 = \frac{2(1+\alpha-n)}{D} - 1 = \frac{1+m}{D} > 0,\\
  &a_0(n)-r_1(n)= \frac{n}{D} + \frac{(1+\alpha)(1+m+n)}{D(\alpha-m-n)}\lambda >\frac{n}{D} + \frac{(1+\alpha)(1+m+n)(1+m)}{2D(\alpha-m-n)^2}>0.
 \end{aligned}
\end{equation}
In order to proceed, we give a different proof for the claim that $int~T \subset W^s(M_1(0))$ for the flow \eqref{eq:slow0} when $n=0$. We prove that the orbits in $T$ is pushed by the two families of lines $p=M$ and $\hat{r}(p,q,\frac{1+m}{1+\alpha},0)=N$. Remember that $T$ is positively invariant.

For each of the contour line $\hat{r}(p,q,\frac{1+m}{1+\alpha},0)=N$, the normal component of the flow toward origin  is
\begin{align*}
 -(\lambda N \dot{p} + \dot{q}) &= -(1-\lambda Np -q)(\lambda Np+q) -pN(N-1) \\
 &= \Big(\frac{\alpha-m}{\lambda(1+\alpha)}\Big)^2 (N-r_1)(N-a) -pN(N-1) \le -\delta_3(N) <0,
\end{align*}
if $\textrm{max}\{1,r_1\} <N<a$. By \eqref{eq:combi}, we can take $\underbar{N} = a_0(0) - \frac{1}{2}\textrm{min}\left\{\frac{1+m}{D},\frac{(1+\alpha)(1+m)^2}{2D(\alpha-m)^2}\right\}=a0-\delta_4$. Therefore, $T$ contracts to the region $\hat{r}(p,q,\frac{1+m}{1+\alpha},0)\le\underbar{N}$.
Then, \eqref{eq:slow0} implies that $\dot{p} \le -\delta_5 p$, for some $\delta_5 >0$. Thus $p$ becomes arbitrarily small as $\eta \rightarrow \infty$. If $p$ is arbitrarily small, then $\dot{q} = q(1-q) -pr(\lambda q +b) > q(1-q) +\epsilon$, so $q$ increases arbitrarily close to $1$. Thus the orbits end up with being in the local stable manifold $W^s(M_1)$.

Now, we return to the reduced flow \eqref{eq:reduced} with $n>0$ in the thin slab. We can improve \eqref{eq:uniform} so that
\begin{equation} \label{eq:uniform2}
 \begin{aligned}
 \lambda \hat{r}(n)p + q + \frac{\alpha-m-n}{\lambda(1+\alpha)}(\hat{r}(n)-a(n)) &= \mathcal{O}(n),\\
   \lambda \hat{r}(n)p + q-1 + \frac{\alpha-m-n}{\lambda(1+\alpha)}(\hat{r}(n)-r_1(n)) &= \mathcal{O}(n), \quad \text{uniformly in the thin slab}.
 \end{aligned}
\end{equation}
Then $\eqref{eq:reduced}_1$ and $\eqref{eq:reduced}_2$ can be re-written as
\begin{equation} \label{eq:reduced_slab}
\begin{aligned}
 \dot{p} &= p\big(\frac{D}{\lambda(1+\alpha)}(\hat{r}(0)-a_0(0)) + \mathcal{O}(n)\big),\\
 \dot{q} &= q\big(\frac{\alpha-m}{\lambda(1+\alpha)}(\hat{r}(0)-r_1(0))\big) + bp\hat{r}(0) + \mathcal{O}(n).
\end{aligned}
\end{equation}
Having seen this and because $\delta_3$,$\delta_4$, and $\delta_5$ are all taken away from $0$, for sufficiently small $n$ that the thin slab is pushed by the two families of planes $\hat{r}(p,q,s,0)=N$ and $p=M$ and that $q$ increases close to $1$ remains to hold in the positively invariant thin slab. Thus the claim for the case $\lambda > b(n)$ follows.
\end{proof}
\begin{proof}[proof of Lemma \ref{lem:3dstable}]
 Let $B_\delta(M_1)$ be contained in a local stable manifold $W^s(M_1)$. For given $\delta$, choose $n$ sufficiently small so that the trapezoid $Z$ is contained in $B_\delta(M_1)$.
\end{proof}
\begin{corollary}
 Every point in $W^u(M_0(n))$ emanated into $T$ is a heteroclinic orbit connecting $M_0(n)$ to $M_1(n)$.
\end{corollary}

\begin{remark}[Non-normally hyperbolicity of $T$]
 Now, the $2$-dimensional unstable manifold $W^u(M_0(n))$ exists, whose counter part for $n=0$  is in the plane $T$. The $W^u(M_0(n))$ near $M_1(n)$ may not diffeomorphic to $W^u(M_0(0))$ near $M_1(0)$ (flat), if the $s$-directional eigenvalue is the least at $M_1$, which actually takes place.
\end{remark}


\pagebreak
% \begin{claim}
%  The flow \eqref{eq:slow0} in $K \backslash B_\delta(M_0)$ is structurally stable.
% \end{claim}
% \begin{proof}
%  We claim that $K \backslash B_\delta(M_0)$ is a hyperbolic set for the reduced flow \eqref{eq:slow0} as a whole $3$-dimensional set that has only stable bundle. Note that $K \backslash B_\delta(M_0)$ is positively invariant but is not invariant. We cut-off the \eqref{eq:slow0} near the boundary so that the set is invariant and maximal with respect to the cut-off flow.
% \end{proof}
% 
% 
% Now, consider the reduced problem for $n>0$.
% \begin{claim}
% \begin{enumerate}
%  \item Two dimensional $W^u(M_0(n))B_\delta(M_0) \subset K \subset W^s(M_1(n))$.
%  \item The heteroclinic orbit $\varphi^*(\eta)$ exists as a unstable fiber passing $M_0(n)$. More precisely
%     \begin{equation} \label{eq:estimate}
%         \varphi^*(\eta) \rightarrow M_1(n) \quad \text{as $\eta \rightarrow \infty$ and} \quad e^{2\eta}\big(\varphi^*(\eta) - M_0(n)\big) \rightarrow \kappa\vec{X}_{02} \quad \text{as $\eta \rightarrow -\infty$ for some $\kappa>0$}.
%     \end{equation} 
% \end{enumerate}
% \end{claim}





Noticing that $\displaystyle a_0(1+m+n)-c_0(1+\alpha)=n$ and that

\begin{enumerate}
 \item The equation \eqref{eq:th} can be rewritten in the form
 $$ \Big(\frac{1}{1+\alpha} \tth^{1+\alpha}\Big)_t = \frac{\tu^n}{\tg^n} \Big(\frac{1}{1+m+n} \tg^{1+m+n}\Big)_t$$
 and we expect, at least for the self-similar solutions, that
 $$ \frac{ \frac{1}{1+\alpha} \tth^{1+\alpha} }{ \frac{1}{1+m+n} \tg^{1+m+n} }  = 1 + \mathcal{O}(n) \overset{put}{=} s^n, \quad \text{for some $s$}. $$
 \item If so, we can define the variable 
 $$r = \Big(\Big(\frac{1+m+n}{1+\alpha}\Big)^{\frac{\alpha}{(1+\alpha)}}\tau \gamma^{\frac{\alpha-m-n}{1+\alpha}}\Big)^{\frac{1}{n}} = \frac{\tu}{\tg}\,s^{-\frac{\alpha}{1+\alpha}} \sim \mathcal{O}(1) $$
 and when $\alpha=0$, it reduces to $\frac{\tu}{\tg}$. The advantage of this definition to the $\frac{\tu}{\tg}$ is that the latter expression couples to the $\tth$ too whereas $r$ here does not. 
\end{enumerate}

\begin{align*}
 \frac{\ts\tu}{\tth} = \frac{\tg^{1+m+n}}{\tth^{1+\alpha}}\Big(\frac{\tu}{\tg}\Big)^{1+n} = \frac{1+m}{1+\alpha} \frac{1}{z}\,\Big(\frac{\tu}{\tg}\Big)^{1+n} =  \frac{1+m+n}{1+\alpha} \frac{1}{s^n}\,\Big(rs^{\frac{\alpha}{1+\alpha}}\Big)^{1+n}
 =\frac{1+m+n}{1+\alpha}\Big( r^{1+n} s^{\frac{\alpha-n}{1+\alpha}}\Big),
\end{align*}
\begin{align*}
n\frac{\dot{s}}{s} = \frac{\dot{z}}{z} = \frac{1+m+n}{\lambda}\Big(r^{1+n}s^{\frac{\alpha-n}{1+\alpha}} - rs^{\frac{\alpha}{1+\alpha}} \Big) + \frac{n}{\lambda} = \frac{1+m+n}{\lambda}\,rs^{\frac{\alpha}{1+\alpha}}\Big(r^{n}s^{\frac{-n}{1+\alpha}} - 1 \Big) + \frac{n}{\lambda}
\end{align*}

$(p,q,r,s)$-system:
\begin{align*}
 \frac{\dpp}{p}&=\left[\frac{1}{\lambda }\Big(\frac{\tu}{\tg}-a_0\Big)\right] & &-\left[d_1 + b\frac{\tv}{\ts} + \lambda\frac{\tu}{\tv}\frac{\tv}{\ts}\right]\\
 \frac{\dqq}{q}&=\left[b_1 +\frac{\tu}{\tv}\right] & &-\left[d_1 + b\frac{\tv}{\ts} + \lambda\frac{\tu}{\tv}\frac{\tv}{\ts}\right]\\
 n\frac{\drr}{r}&=\left[\frac{\alpha-m-n}{\lambda(1+\alpha) }\Big(\frac{\tu}{\tg}-a_0\Big)\right]& &+\left[d_1 + b\frac{\tv}{\ts} + \lambda\frac{\tu}{\tv}\frac{\tv}{\ts}\right]\\
 \frac{\dot{s}}{s} &= \frac{1+m+n}{\lambda}\,rs^{\frac{\alpha}{1+\alpha}}\Big(\frac{r^{n}s^{\frac{-n}{1+\alpha}} - 1}{n} \Big) + \frac{1}{\lambda}
\end{align*}

\begin{remark}
 \begin{enumerate}
  \item 
  \item Note that $s$ is not a fast variable. Even though $\displaystyle\frac{ \frac{1}{1+\alpha} \tth^{1+\alpha} }{ \frac{1}{1+m+n} \tg^{1+m+n} }$ relaxes to the manifold $1 + \mathcal{O}(n)$, the relaxation time is not of $\mathcal{O}(\frac{1}{n})$ but is of $\mathcal{O}(1)$.
  
 \end{enumerate}
\end{remark}



$(p,q,r,s)$-system:

\begin{equation}
\begin{aligned}
  {\dpp}&=p\bigg\{\Big[\frac{1}{\lambda }\Big(r-a_0\Big)\Big] -\Big[d_1 + q + \lambda p r\Big]\bigg\}\\
  {\dqq}&=q\bigg\{b_1-d_1 - q-\lambda p r\bigg\} +bpr,\\
 n{\drr}&=r\bigg\{\left[\frac{\alpha-m-n}{\lambda(1+\alpha) }\Big(\frac{\tu}{\tg}-a_0\Big)\right]& &+\left[d_1 + b\frac{\tv}{\ts} + \lambda\frac{\tu}{\tv}\frac{\tv}{\ts}\right]\bigg\}\\
 n\frac{\dot{s}}{s} &= \frac{1+m+n}{\lambda}\Big(r^{1+n}s^{\frac{\alpha-n}{1+\alpha}} - r \Big) + \frac{n}{\lambda}
\end{aligned}
\end{equation}

\begin{remark}
 In the case of the variable $r$, assuming $\tg, \ts,\tth \sim \mathcal{O}(1)$, the exponent $n$ is natural and the relaxation time scale $\drr \sim \mathcal{O}(\frac{1}{n})$. For the case of $\displaystyle\left(\frac{ \frac{1}{1+\alpha}\tth^{1+\alpha}}{ \frac{1}{1+m}\tg^{1+m} } -1 \right)$, there is no preferred relaxation time scale. For the time being, we do not specify the function $s(n,z-1)$ in the upcoming calculations.
\end{remark}
\begin{align*}
 \dot{s} =\frac{\partial s}{\partial (z-1)} \dot{z}
 &= \frac{1+m}{\lambda}\frac{\partial s}{\partial (z-1)} z\bigg\{-r - \frac{n}{D}+ \frac{r}{z}u^n\bigg\}\\
 &=\frac{1+m}{\lambda}\frac{\partial s}{\partial (z-1)} \bigg\{r(u^n-1) +r(1-z) -n\frac{z}{D}\bigg\}.
\end{align*}
We choose $s(n,z)$ such that
\begin{enumerate}
 \item As $(n,z-1) \rightarrow (0,0)$, $s(n,z-1) \rightarrow 0$, 
 \item For fixed $n$, the map $z \mapsto s$ is invertible, and for inverse $z=z(n,s)\rightarrow 1$ as $(n,s) \rightarrow (0,0)$.
\end{enumerate}
We set 
$$ s(n,z-1) = \frac{(z-1)^{\frac{1}{n}}}{n}, \quad \text{or} \quad z= 1+ns^n$$
and look for solutions of this form. Then we have


\begin{align*}
 \frac{\ts\tu}{\tth} &= \frac{1+m}{1+\alpha} \frac{r}{z}\,u^n = \frac{1+m}{1+\alpha} r + n\frac{1+m}{1+\alpha} \frac{r}{1+ns^n}\Big(\frac{u^n-1}{n}-s^n\Big),%\\
%  \frac{\tu}{\tv}\frac{\tv}{\ts}&=\frac{\ts}{\tv} \frac{\tg}{\ts} \frac{\tu}{\tg} \frac{\tv}{\ts} = pr.
\end{align*}





\begin{equation}
\begin{aligned}
  {\dpp}&=p\bigg\{\Big[\frac{1}{\lambda }\Big(r-a_0\Big)\Big] -\Big[d_1 + q + \lambda p r\Big]\bigg\}\\
  {\dqq}&=q\bigg\{b_1-d_1 + \lambda p r\bigg\} +bpr,\\
 n{\drr}&=r\bigg\{\Big[\frac{-m-n}{\lambda }\Big(r-a_0\Big)\Big]+\Big[d_1 + q + \lambda p r\Big]+\Big[\frac{\alpha}{\lambda }\Big(\frac{1+m}{1+\alpha}r-c_0\Big)\Big] + n\Big[\frac{\alpha}{\lambda }\frac{1+m}{1+\alpha} \frac{r}{1+ns^n}\Big(\frac{u^n-1}{n}-s^n\Big)\Big]\bigg\}\\
 n\dot{s}&=\frac{1+m}{\lambda}s^{1-n} \left\{r\frac{u^n-1}{n} - rs^n -\frac{1+ns^n}{D}\right\}.
\end{aligned}
\end{equation}

\pagebreak


\pagebreak

\section{Two-parameters family of self-similar shear banding solutions asymptotics}
\subsection{Asymptotic behavior of field variables in $t$ and $x$}




\end{document}