%%%%%%%%%%%%%%%%%%%%%%%%%%%%%%%%%%%%%%%%%%%%%%%
%
%    Self-Similar shear bands, Existence, Numerics, Asymptotics
%
%                                                      by
%
%                                       Min-Gi Lee
%
%                                          version Sep 2016
%
%
%%%%%%%%%%%%%%%%%%%%%%%%%%%%%%%%%%%%%%%%%%%%%%%
\documentclass[a4paper,11pt]{article}

\usepackage[margin=2cm]{geometry}
\usepackage{setspace}
%\onehalfspacing
\doublespacing
%\usepackage{authblk}
\usepackage{amsmath}
\usepackage{amssymb}
\usepackage{amsthm}
\usepackage{calrsfs}
% \usepackage[notcite,notref]{showkeys}

\usepackage{psfrag}
\usepackage{graphicx,subfigure}
\usepackage{color}
\def\red{\color{red}}
\def\blue{\color{blue}}
%\usepackage{verbatim}
% \usepackage{alltt}
%\usepackage{kotex}

\usepackage{enumerate}

%%%%%%%%%%%%%% MY DEFINITIONS %%%%%%%%%%%%%%%%%%%%%%%%%%%

\def\tr{\,\textrm{tr}\,}
\def\div{\,\textrm{div}\,}
\def\sgn{\,\textrm{sgn}\,}

\def\th{\tilde{h}}
\def\tx{\tilde{x}}
\def\tk{\tilde{\kappa}}


\def\bg{{\bar{\gamma}}}
\def\bv{{\bar{v}}}
\def\bth{{\bar{\theta}}}
\def\bs{{\bar{\sigma}}}
\def\bu{{\bar{u}}}
\def\bph{{\bar{\varphi}}}


\def\tg{{\tilde{\gamma}}}
\def\tv{{\tilde{v}}}
\def\tth{{\tilde{\theta}}}
\def\ts{{\tilde{\sigma}}}
\def\tu{{\tilde{u}}}
\def\tph{{\tilde{\varphi}}}

\def\dtg{{\dot{\tilde{\gamma}}}}
\def\dtv{{\dot{\tilde{v}}}}
\def\dtth{{\dot{\tilde{\theta}}}}
\def\dts{{\dot{\tilde{\sigma}}}}
\def\dtu{{\dot{\tilde{u}}}}
\def\dtph{{\dot{\tilde{\varphi}}}}

\def\dpp{\dot{p}}
\def\dqq{\dot{q}}
\def\drr{\dot{r}}
\def\dss{\dot{s}}

\def\ta{{\tilde{a}}}
\def\tb{{\tilde{b}}}
\def\tc{{\tilde{c}}}
\def\td{{\tilde{d}}}

\def\BO{{\mathcal{O}}}
\def\lio{{\mathcal{o}}}



\def\bx{\bar{x}}
\def\bm{\bar{\mathbf{m}}}
\def\K{\mathcal{K}}
\def\E{\mathcal{E}}
\def\del{\partial}
\def\eps{\varepsilon}

\newcommand{\tcr}{\textcolor{red}}
\newcommand{\tcb}{\textcolor{blue}}

\newcommand{\ubar}[1]{\text{\b{$#1$}}}
\newtheorem{theorem}{Theorem}
\newtheorem{lemma}{Lemma}[section]
\newtheorem{proposition}{Proposition}[section]
\newtheorem{corollary}{Corollary}[section]
\newtheorem{definition}{Definition}[section]
\newtheorem{claim}{Claim}

\theoremstyle{remark}
\newtheorem{remark}{Remark}[section]


%%%%%%%%%%%%%%%%%%%%%%%%%%%%%%%%%%%%%%%%%%%%%%%%%%%%%%%%%%
\begin{document}
\title{Localization in adiabatic shear flow \\via geometric theory of singular perturbations}
\author{Min-Gi Lee\footnotemark[1] \and Athanasios Tzavaras\footnotemark[2]}%\  \footnotemark[3]  \footnotemark[4]}
\date{}

\maketitle
\renewcommand{\thefootnote}{\fnsymbol{footnote}}
\footnotetext[1]{Min-Gi Lee \\ King Abdullah University of Science and Technology (KAUST), Computer, Electrical and Mathematical Sciences \& Engineering Division, KAUST, Thuwal, Saudi Arabia, \\ e-mail: mingi.lee@kaust.edu.sa}%{Computer, Electrical and Mathematical Sciences \& Engineering Division, King Abdullah University of Science and Technology (KAUST), Thuwal, Saudi Arabia}
\footnotetext[2]{Athanasios Tzavaras \\ King Abdullah University of Science and Technology (KAUST), Computer, Electrical and Mathematical Sciences \& Engineering Division, KAUST, Thuwal, Saudi Arabia,\\  e-mail: {athanasios.tzavaras@kaust.edu.sa}}
% \footnotetext[2]{Department of Mathematics and Applied Mathematics, University of Crete, Heraklion, Greece}
% \footnotetext[3]{Institute of Applied and Computational Mathematics, FORTH, Heraklion, Greece}
% \footnotetext[4]{Corresponding author : \texttt{athanasios.tzavaras@kaust.edu.sa}}
%\footnotetext[4]{Research supported by the King Abdullah University of Science and Technology (KAUST) }
\renewcommand{\thefootnote}{\arabic{footnote}}


\maketitle

\begin{abstract}
We study a localizing instability of metals occurring at a high speed shear motion. The imbalance of the thermal softening response of metals over the material can result in the formation of shear bands.
Adiabatic shear deformation of a thermo-visco-plastic material is considered to capture the phenomena and the existence of a two-parameter family of self-similar solutions are established. Hadamard instability is exhibited during the plastic deformation when the thermal softening response outweighs the strain hardening response. We understand the localizing instability as a result of the nonlinear competition between the Hadamard instability of the net softening and the viscosity of the rate dependency in the corresponding constitutive law. The existence is turned into a problem of constructing a heteroclinic orbit of an induced dymanical system. % by the self-similar ansatz and the sophisticatedly devised transformations. 
The dynamical system turns out to be high dimensional but have a fast-slow structure with respect to a small parameter. Exploiting the structure, geometric singular perturbation theory is applied to achieve the heteroclinic orbit. The heteroclinic orbit arises as a transversal intersection of two invariant manifolds in the phase space.
\end{abstract}

\tableofcontents

\section{Introduction}
Shear bands are narrow zones of intensely localized shear that are formed during the high speed shear motion of materials such as metals \cite{zener_effect_1944}. It often precedes ruptures and is one of the striking phenomena of the material failure. It is our main objective to capture this  phenomena in a simplest possible framework of one-dimensional thermo-visoco-plasticity. We describe a specimen on $xy$-plain that is in a shear motion in $y$-direction. Throughout this paper,
\begin{equation} \label{eq:vars}
\begin{aligned}
 \gamma(t,x) &: \text{shear strain}\\
 u(t,x)=\gamma_t &: \text{strain rate}\\
 v(t,x) &: \text{vertical velocity}\\
 \theta(t,x) &: \text{temperature}\\
 \tau(t,x) &: \text{shear stress}
\end{aligned}
\end{equation}
in $(t,x)\in \mathbb{R}^+ \times \mathbb{R}$, and equations describing them are

\hskip -1em
\begin{minipage}{0.5\linewidth}
\begin{equation} \label{intro-system0}
\begin{aligned}
 \gamma_t &= v_x, \quad \text{(kinematic compatibility)} \\	%\label{eq:g}\\
 v_t &= \tau_x, \quad \text{(momentum equation)} 	\\%\label{eq:v}\\
 \theta_t &= \tau v_x, \quad \text{(adiabatic energy equation)}	\\%\label{eq:th}\\
 \tau &=\theta^{-\alpha}\gamma^m (v_x)^n,\quad \text{(constitutive law)}				%\label{eq:tau}
\end{aligned}
\end{equation}
\end{minipage}
\qquad or equivalantly,
\begin{minipage}{0.3\linewidth}
\begin{equation} \label{intro-system1}
\begin{aligned}
 \gamma_t &= u,\\%~(\triangleq v_x), \\%\quad \text{(kinematic compatibility)} \\	%\label{eq:g}\\
 u_t &= \tau_{xx}, \\%\quad \text{(momentum conservation)} 	\\%\label{eq:v}\\
 \theta_t &= \tau u,\\% \quad \text{(energy conservation)}	\\%\label{eq:th}\\
 \tau &=\theta^{-\alpha}\gamma^m u^n.			%\label{eq:tau}
\end{aligned}
\end{equation}
\end{minipage}
\hfill

\eqref{intro-system0} and \eqref{intro-system1} are in non-dimensionalized forms and \cite{KT09} provides the derivation from the dimensional ones. We refer to \cite{clifton_rev_1990,shawki_shear_1989,wright_survey_2002} the review papers for mechanical perspective.

The constitutive law $\tau = \theta^{-\alpha}\gamma^m u^n$ characterizes, in the form of the power law, the nature of the material. $\alpha>0$ measures the degree of the thermal softening, $m>0$ measures that of strain hardening (or $m<0$ that of softening in cases of plastic flow), and $n>0$ measures that of the rate sensitivity. $n$ is assumed to be small, or the viscous contribution to the stress is assumed to be small. This model is the simple paradigm to capture the shear localization. The model, as well as the empirical power law $\eqref{intro-system0}_4$, has been appeared in early studies \cite{HN77} (in connection to necking) and in \cite{WF83,FM87}. \cite{tzavaras_plastic_1986,tzavaras_strain_1991,tzavaras_nonlinear_1992} studied the plastic flow from the mathematical point of view. In particular, the model with $m=0$ corresponds to the one for the rectilinear viscous fluid and the stability or instability results of the model accordingly to the parameters have been studied in \cite{bertsch_effect_1991, DH_1983, Tz_1986, Tz_1987, KOT14, KLT_HYP2016}. The model with $\alpha=0$ corresponds to the isothermal plasticity and the localizing self-similar solutions were obtained in \cite{LT16, KLT_2016}.

% From the mathematics literature,
%
%
%
%
% see \cite{tzavaras_plastic_1986,tzavaras_strain_1991,tzavaras_nonlinear_1992}, and \cite{bertsch_effect_1991, DH_1983, Tz_1986, Tz_1987, KOT14, KLT_HYP2016} (in the study of rectlinear viscous fluid), and in the study of its isothermal \cite{LT16, KLT_2016}.
% }

It is our main objective to prove the existence of a family of localizing solutions of \eqref{intro-system0} that is of the self-similar form
\begin{equation} \label{intro-sols}
\begin{aligned}
 \bg(t,x) &= (t+1)^a\Gamma\big((t+1)^\lambda x\big), & \bv(t,x) &= (t+1)^bV\big((t+1)^\lambda x\big), & \bth(t,x) &= (t+1)^c\Theta\big((t+1)^\lambda x\big),\\
 \bar{\tau}(t,x) &= (t+1)^d\Sigma\big((t+1)^\lambda x\big), & \bu(t,x) &= (t+1)^{a-1}U\big((t+1)^\lambda x\big).
\end{aligned}
\end{equation}
The exponents $a$, $b$, $c$, and $d$ are selected as in \eqref{eq:exponents} according to the scale invariance property, $\lambda>0$ accounts for the rate of localization, and $\xi=(t+1)^\lambda x$ is the self-similar variable. The idea of self-similar localizing solutions was first introduced in \cite{KOT14}.   One should notice that the exponent $\lambda$ is taken in opposite sign to the one in the typical parabolic problems where the defocussing behavior is expected.

The model \eqref{intro-system0} admits universally the {\it uniform shearing solution} regardless of the precise exponents $(\alpha,m,n)$. For a given constant $\gamma_0$ and the initial temperature $\theta_0(x)$, it consists of
\begin{equation} \label{intro:uss}
 \begin{aligned}
 \gamma &= t+\gamma_0, \quad v=x, \quad u=1, \quad \theta = \left( \frac{1+\alpha}{1+m} \big( (t+\gamma_0)^{1+m}-\gamma_0^{1+m}\big) + \theta_0^{1+\alpha}(x)\right)^{\frac{1}{1+\alpha}}, \\
 \tau&=\left( \frac{1+\alpha}{1+m} \big( (t+\gamma_0)^{1+m}-\gamma_0^{1+m}\big) + \theta_0^{1+\alpha}(x)\right)^{-\frac{\alpha}{1+\alpha}}(t+\gamma_0)^m.
 \end{aligned}
\end{equation}
This is a template solution that are contrasted with those exhibit localizing instability. In the latter, the growth of the strain is superlinear and localizes in a narrow zone.% in the same time.

Widely accepted explanation (\cite{shawki_shear_1989,clifton_rev_1990}) why such a localizing instability can take place is the following positive feedback scenario: the material keeps loading and
this mechanical working causes the increases in temperature. Due to the imperfectness of the deformation, the small non-uniformities in strain presences and this results in the non-uniformities in the heat production. When the speed of deformation is so high such that the time needed for the heat to diffuse out is not sufficient, then the process becomes effectively {\it adiabatic}, i.e., the heat produced at a spot accumulates. Metals typically thermally soften and strain hardens. For such a metal where the thermal softening outweighs the hardening response, the net response is softening and the spots that had larger deformation than surrounding becomes even easier to deform. This, in turn, gives rise to the severer non-uniformities in strain, triggering the positive feedback mechanism.

We conceive the instability of \eqref{intro:uss} by the loss of hyperbolicity in the system. For $n=0$, \eqref{intro-system0} is a first order system and we illustrated in Appendix \ref{append:hadamard} that along the uniform shearing solution, the hyperbolicity is lost if $-\alpha+m<0$. This foresees the catastrophic growth of oscillations in the initial value problem, which has coined the term {\it Hadamard Instability}.

As pointed out in \cite{KOT14}, however, the Hadamard Instability cannot be the complete accounts of the shear band phenomena. This is because what is observed in the process is the orderly development of localization \cite{zener_effect_1944} rather than the arbitrary growth of oscillations. As an opposing mechanism to the Hadamard Instability, the small viscosity ($n>0$) is suggested in the model, and the system is of parabolic-hyperbolic type. We should also note that, if $n$ is sufficiently large, then diffusive mechanism takes over so that the localization is suppressed, or \eqref{intro:uss} is expected to be globally asymptotically stable. The nonlinear stability of uniform shearing solution for the case $m=0$ with $-\alpha+n>0$ has been proved in \cite{DH_1983, Tz_1986, tzavaras_strain_1991, KT09}. This coincides with the threshold indicated from the study of the linearized problem \cite{FM87}. The linearized analysis has not been fully understood yet; analyzing linearized problem around the uniform shearing solutions bears a difficulty in that it appears as a non-autonomous system. \cite{KT09} studied the long time asymptotic effective equation for the general cases. The effective equation has a fourth order correction to the second order term that has a regularizing effect. In terms of second order term, the equation changes type from the parabolic to the backward parabolic as $-\alpha+m+n$ changes sign from positive value to negative value. As a matter of fact, it is circumstantial but we are led to focus on the regime $-\alpha+m+n<0$ for the localization to take place, where the Hadamard Instability and the diffusive mechanism are expectedly effect comparably.%

The first half of the tackling the problem is a formulation of the problem. We will turn the problem to that on the dynamical system. The series of techniques and transformations for the formulation were first introduced in \cite{KOT14}, and were successfully adapted to the strain independent model ($\tau=\varphi(\theta)u^n$) \cite{KLT_HYP2016}, and to the temperature independent model ($\tau=\varphi(\gamma)u^n$) \cite{LT16,KLT_2016}. Due to self-similarity ansatz, problem reduces to solving a system of ordinary differential equations
\begin{equation} \label{intro:ss-odes}
\begin{aligned}
 a \Gamma(\xi) + \lambda \xi \Gamma'(\xi) &= U(\xi), \\
 b V(\xi) + \lambda \xi V'(\xi) &= \Sigma'(\xi), \\
 c \Theta(\xi) + \lambda \xi \Theta'(\xi)&=\Sigma(\xi) U(\xi),\\
 \Sigma(\xi) &= \Theta(\xi)^{-\alpha} \Gamma(\xi)^m U(\xi)^n, \\
 V'(\xi)&=U(\xi),\\
 \Gamma(0)=\Gamma_0>0, \quad U(0)&=U_0>0, \quad \text{$\xi \in [0,\infty)$},
\end{aligned}
\end{equation}
where, we look for $(\Gamma,U,\Theta,\Sigma)$ that is even and $V$ that is odd. So that the symmetric extensions are regular, we impose
\begin{equation}
 V(0)=U'(0)=\Gamma'(0)=\Sigma'(0)=\Theta'(0)=0. \label{intro:bdry0}
\end{equation}

The coefficient $\xi$ in the highest order terms in \eqref{intro:ss-odes} tells that the system is  singular at $\xi=0$ and not autonomous. Such a singular system lacks general existence theory and case-by-case analyzes are required for its study. It turns out that the singularity in \eqref{intro:ss-odes} is resolved; we introduce series of nonlinear transformations to de-singularize \eqref{intro:ss-odes} discovering the proper existence theory of it. In its final form, we arrive at the following four-dimensional dynamical system on $(p,q,r,s)$
\begin{equation}\label{intro:slow}
 \begin{aligned}
 \dot{p} &=p\Big(\frac{1}{\lambda}(r-a) + 2- \lambda p r -q\Big), \\
 \dot{q} &=q\Big(1 -\lambda p r -q\Big) + b p r,\\
 n\dot{r} &=r\Big(\frac{\alpha-m-n}{\lambda(1+\alpha)}(r-a) + \lambda pr + q +\frac{\alpha}{\lambda}r\big(s- \frac{1+m+n}{1+\alpha}\big) + \frac{n\alpha}{\lambda(1+\alpha)}\Big),\\
 \dot{s} &=s\Big(\frac{\alpha-m-n}{\lambda(1+\alpha)}(r-a) + \lambda pr + q - \frac{1}{\lambda}r\big(s- \frac{1+m+n}{1+\alpha}\big) - \frac{n}{\lambda(1+\alpha)}\Big),
 \end{aligned}
\end{equation}
parametrized by $(\lambda,\alpha,m,n)$. \eqref{intro:bdry0} are transmitted to the corresponding asymptotic conditions for \eqref{intro:slow}. The problem looking for solutions of \eqref{intro:ss-odes} is turned into that looking for suitable heteroclinic orbits of \eqref{intro:slow}.

In the remaining half is the proof of the existence of the heteroclinic orbit. Having \eqref{intro:slow}, equilibrium points of \eqref{intro:slow} are completely sought in Section \ref{sec:equil} and the flow around them is explored in the phase space. The heteroclinic orbit we are to construct, among many others, is selected in Section \ref{sec:char}. This hypothesized heteroclinic orbit is constructed in Section \ref{sec:proof}.

Throughout our study, it is the smallness of $n$ that pertains to our analyzes. Its manifestations are twofold, one is singular and the other is regular. More precisely, the limiting case $n=0$ is the adiabatic inviscid problem and there, only two variables out of four plays the role. First, $\eqref{intro-system0}_3$ is re-written as
$$ \partial_t\Big(\frac{\theta^{1+\alpha}}{1+\alpha}\Big) = \big(\gamma_t^n\big)\partial_t\Big(\frac{\gamma^{1+m}}{1+m}\Big)$$
and at $n=0$, it says that modulo initial data one variable is expressed by the other, reducing the number of variables again by one. When $n>0$, while we cannot remove one of the variables, their co-evolving feature is expected to persist, or this feature is regularly perturbed. Second, the last equation $\tau=\theta^{-\alpha}\gamma^m(v_x)^n$ is a constraint equation that specifies a manifold on the function space of $(\gamma,v,\theta,\tau)$, which by solving the differential constraint equation reduces the number of variables by one. The constraint equation is singularly perturbed, at $n=0$ is an an algebraic equation $\tau=\theta^{-\alpha}\gamma^m$, while at $n>0$ is a differential equation. %Note that we can assign a far field condition at $x=\infty$ by having $n>0$.
At $n=0$, it does not uncover how $v$, that is responsible for the singular perturbation, is related to the rest of the variables until we solve \eqref{intro-system0} as a whole. In our study on the self-similar solutions, we understand the constraint equation by the one differentiated once, letting the constraint propagates by having the derivative vanish. The expressions for the derivatives of the variables are suitably substituted by the other equations in \eqref{intro-system0} to give rise to a suitable nontrivial constraint propagation equation. This procedure will be clarified in Section \ref{sec:proof}. It turns out that this propagation is a relaxation process towards a slow manifold that is perturbed from the one at $n=0$, which is the singularly perturbed nature. We capture this slow manifold in our study.

%that is perturbed from the one at the singular limit $n=0$.
%We let this identity be exact at a space time point $(t_0,x_0)$ and let this constraint propagate through the space-time by having its derivatives vanish. Later in the study on the self-similar solutions, space-time infinity ($\xi=\infty$) is chosen for the point of prescribed exact identity. It turns out that this propagation is a relaxation process towards a slow manifold that is perturbed from the one at the singular limit $n=0$. %This relaxation is singularly perturbed in a sense that if the starting point values of $(\gamma,v,\theta,\tau)|_{(t_0,x_0)}$ are off the manifold then layer is formed.

Aforementioned perturbative nature manifests itself in the $(p,q,r,s)$-system \eqref{intro:slow} by $(i)$ the presence of so called {\it fast-slow} structure with respect to the small parameter $n$ and $(ii)$ the appearance of the invariant manifold that is of two dimensional. {\it Geometric singular perturbation theory} is employed to conduct the Chapman-Enskog type reduction and to achieve the heteroclinic orbit. We will detail this in Section \ref{sec:proof}. More advanced theory of geometric singular perturbation theory can be found in \cite{fenichel_asymptotic_1974, fenichel_asymptotic_1977,fenichel_geometric_1979,HPS_1977,Sz1991}.

In the last section, we describe the emergence of the localization in the constructed two-parameters family of solutions via asymptotic behaviors scrutinized. The numerical computations of the solution, which have been described in \cite{KLT_HYP2016}, are supplemented again as well for its illustration. 


% It turned out that the self-similar solution is attained in the range
% \begin{equation} \label{intro-range}
% \begin{aligned}
% &0< \lambda < \frac{2(\alpha-m-n)}{1+m+n}\left(\frac{1+m}{1+m+n}\right) \\
% \Longleftrightarrow \quad &\frac{2(1+\alpha) -n}{D} < \frac{U_0}{\Gamma_0} < \frac{2(1+\alpha)}{1+m+n} -\frac{n}{D}\left( \frac{4(1+\alpha)(\alpha-m-n)}{(1+m+n)^2} +1\right),
% \end{aligned}
% \end{equation}
% where $\Gamma_0$ and $U_0$ parametrize the initial tip sizes at the origin in the strain and strain rate respectively. It is interesting to see that \eqref{intro-range} requires for the ratio $\frac{U_0}{\Gamma_0}$ to have not too small and not too big value.
% The onset of localization is read off from the solutions. We find, for instance, the finite difference in the rate of growth of the strain between the origin and the rest of the places as in the below.
% \begin{align*}
%   \gamma(t,0) &= (1+t)^{\frac{2+2\alpha-n}{D} + \frac{2+2\alpha}{D}\lambda}\Gamma(0), &
%  \gamma(t,x) &\sim t^{\frac{2+2\alpha-n}{D} - \frac{(1+\alpha)(1+m+n)}{D(\alpha-m-n)}\lambda}|x|^{-\frac{1+\alpha}{\alpha-m-n}}, \quad \text{as $t \rightarrow \infty$, $x\ne0$.}\\
%   \theta(t,0) &= (1+t)^{\frac{2(1+m)}{D} + \frac{2(1+m+n)}{D}\lambda}\Theta(0),&
%  \theta(t,x) &\sim t^{\frac{2(1+m)}{D} - \frac{(1+m+n)^2}{D(\alpha-m-n)}\lambda}|x|^{-\frac{1+m+n}{\alpha-m-n}}, \quad \text{as $t \rightarrow \infty$, $x\ne0$.}\\
%  u(t,0) &= (1+t)^{\frac{1+m}{D} + \frac{2+2\alpha}{D}\lambda}U(0),&
%  u(t,x) &\sim t^{\frac{1+m}{D} - \frac{(1+\alpha)(1+m+n)}{D(\alpha-m-n)}\lambda}|x|^{-\frac{1+\alpha}{\alpha-m-n}}, \quad \text{as $t \rightarrow \infty$, $x\ne0$.}\\
%  \sigma(t,0) &= (1+t)^{\frac{-2\alpha+2m+n}{D} + \frac{-2\alpha+2m+2n}{D}\lambda}\Sigma(0), &
%  \sigma(t,x) &\sim t^{\frac{-2\alpha+2m+n}{D} +\frac{1+m+n}{D}\lambda}|x|, \quad \text{as $t \rightarrow \infty$, $x\ne0$,}
% %  v_\infty(t)=(1+t)^{b}V_\infty = (1+t)^{\frac{1+m}{D} + \frac{1+m+n}{D}\lambda}V_\infty, \quad V_\infty \triangleq \lim_{\xi \rightarrow \infty} V(\xi) <\infty.
% \end{align*}
% and $v(t,x)$ at $t$ is an odd increasing function of $x$ that takes a transition from $-v_\infty(t)$ to $v_\infty(t)$, as $x$ runs from $-\infty$ to $\infty$, where $v_\infty(t)=(1+t)^{\frac{1+m}{D} + \frac{1+m+n}{D}\lambda}V_\infty$, for $V_\infty \triangleq \lim_{\xi \rightarrow \infty} V(\xi) <\infty.$
%
% the equivalent $(p,q,r,s)$-system \eqref{eq:slow} of \eqref{eq:ss-odes} that is autonomous and to turn the problem that seeks the self-similar solutions of \eqref{eq:ss-odes} into that seeks a heteroclinic orbit of \eqref{eq:slow}.
%
% \end{enumerate}

\section{Self-similar structure} \label{sec:scale}

\subsection{Scale invariance properties}
The system \eqref{intro-system0} admits a scale invariance property. Suppose $(\gamma,u,v,\theta,\sigma)$ is a solution. Then a rescaled version of it $(\gamma_\rho,u_\rho,v_\rho,\theta_\rho,\sigma_\rho)$ that is given by
\begin{equation}\label{eq:scale}
\begin{aligned}
 \gamma_\rho(t,x) &= \rho^a\gamma(\rho^{-1}t,\rho^\lambda x), &
 v_\rho(t,x) &= \rho^bv(\rho^{-1}t,\rho^\lambda x),\\
 \theta_\rho(t,x) &= \rho^c\theta(\rho^{-1}t,\rho^\lambda x), &
 \sigma_\rho(t,x) &= \rho^d\sigma(\rho^{-1}t,\rho^\lambda x),\\
 u_\rho(t,x) &= \rho^{b+\lambda}\gamma(\rho^{-1}t,\rho^\lambda x)
\end{aligned}
\end{equation}
is again a solution provided
\begin{equation} \label{eq:exponents}
\begin{aligned}
 a&= a_0 + a_1 \lambda=\frac{2+2\alpha-n}{D} + \frac{2+2\alpha}{D}\lambda, & b&=b_0 + b_1\lambda=\frac{1+m}{D} + \frac{1+m+n}{D}\lambda ,\\
 c&=c_0 + c_1\lambda=\frac{2(1+m)}{D} + \frac{2(1+m+n)}{D}\lambda, & d&=d_0 + d_1\lambda=\frac{-2\alpha + 2m +n}{D} + \frac{-2\alpha+2m+2n}{D}\lambda,
\end{aligned}
\end{equation}
for each $\lambda \in \mathbb{R}$, with the denominator $D = 1+2\alpha-m-n$. Throught this paper, the material parameters $(\alpha,m,n)$ run only in the ranges%are restricted in the following regime
\begin{equation}
 \begin{aligned}
  \alpha>0\quad&\text{(thermal softening)},\\
  m>-1 \quad&\text{(strain softening/hardening)}, \\%\label{eq:a1}\\
  n>0 \quad&\text{(strain rate sensitivity)},\\ %\label{eq:a2}\\
  -\alpha+m+n<0 \quad&\text{(net softening)}. \\%\label{eq:a3}\\
%   0< \lambda < \frac{2(\alpha-m-n)}{1+m+n}\left(\frac{1+m}{1+m+n}\right) \quad&\text{(localizing rate bound)}. %\label{eq:a4}
\end{aligned}\label{eq:paramrange}
\end{equation}
In this regime we note $D>1+\alpha>1$. Having $\lambda$ negative is the typical scaling for the dissipative problems. Since we are interested in the {\it localizing} phenomena, we look for the opposite self-similar scale $\lambda>0$, throughout this paper. %To have an upper bound for the localizing rate $\lambda$ will be reasoned in Section \ref{sec:equil}.

\subsection{Self-Similar solutions}
Motivated by the scale invariance property parametrized by $\lambda>0$, we look for the solutions of type
\begin{equation}\label{eq:ORItoCAP}
\begin{aligned}
 \gamma(t,x) &= t^a\Gamma(t^\lambda x), & v(t,x) &= t^b V(t^\lambda x), &\theta(t,x) &= t^c \Theta(t^\lambda x),\\
 \sigma(t,x) &= t^d \Sigma(t^\lambda x), & u(t,x) &= t^{b+\lambda} U(t^\lambda x)
\end{aligned}
\end{equation}
setting $\xi = t^\lambda x$. Plugging the ansatz into the system \eqref{intro-system0} gives us the system of ordinary differential and algebraic equations for $\big(\Gamma(\xi), V(\xi), \Theta(\xi), \Sigma(\xi), U(\xi)\big)$ to satisfy:

\begin{equation}
\begin{aligned}
 a \Gamma(\xi) + \lambda \xi \Gamma'(\xi) &= U(\xi),\\
 b V(\xi) + \lambda \xi V'(\xi) &= \Sigma'(\xi),\\
 c \Theta(\xi) + \lambda \xi \Theta'(\xi)&=\Sigma(\xi) U(\xi),\\
 \Sigma(\xi) &= \Theta(\xi)^{-\alpha} \Gamma(\xi)^m U(\xi)^n,\\
 V'(\xi)&=U(\xi).
\end{aligned} \label{eq:ss-odes}
\end{equation}
It is notable also that the uniform shear solution is attained as a self-similar solution at $\lambda = -\frac{1+m}{2(1+\alpha)}<0$ that is
\begin{equation*}
 \Gamma(\xi) = U(\xi)=U_0, \quad V(\xi) = U_0\xi, \quad  \Theta(\xi) = \Big( \frac{1+\alpha}{1+m} U_0^{1+m+n}\Big)^{\frac{1}{1+\alpha}}, \quad \Sigma(\xi) = \Big(\frac{1+\alpha}{1+m}\Big)^{\frac{-\alpha}{1+\alpha}} U_0^{\frac{-\alpha+m+n}{1+\alpha}}.
\end{equation*}
The system \eqref{eq:ss-odes} is non-autonomous and the coefficient $\xi$ in front of the highest derivative of $\Theta, V, \Gamma$ shows that the equations are singular at $\xi=0$.
%Existence or non-existence of the singular ordinary differential equations differs from cases to cases and requires case-by-case study. %and results depend on the characteristics of the problem.
It is our main work to de-singularize \eqref{eq:ss-odes} turning it into an autonomous one.

\section{Reduction to the construction of a heteroclinic orbit} \label{sec:formulation}
The goal of this section is to derive the equivalent $(p,q,r,s)$-system \eqref{eq:slow} of \eqref{eq:ss-odes} that is autonomous and to turn the problem that seeks the self-similar solutions of \eqref{eq:ss-odes} into that seeks a heteroclinic orbit of \eqref{eq:slow}. In the course, we devise a series of techniques to de-singularize the system \eqref{eq:ss-odes}. The series of techniques were first advanced in \cite{KOT14}, where the authors studied the fluid with temperature dependent viscous stress with exponential law $ \tau = \mu(\theta)u^n = e^{-\alpha\theta} u^n.$
It turned out that the series of techniques are successfully adapted in our study where we employ the power law for the thermo-visco-plastic solid % model whose stress law exhibits temperature softening
$ \tau = \theta^{-\alpha}\gamma^{m}u^n$ with $\alpha$, $m$, and $n$ in a range \eqref{eq:paramrange}.%, \quad \alpha>0, \quad m>-1, \quad n>0.$$

\subsection{De-singularization}
We begin by observing that if $\big(\Gamma(\xi), V(\xi), \Theta(\xi), \Sigma(\xi), U(\xi)\big)$ is a solution of \eqref{eq:ss-odes}, then so is \\$\big(\Gamma(-\xi), -V(-\xi), \Theta(-\xi), \Sigma(-\xi), U(-\xi)\big)$. From this fact, we look for self-similar profiles such that $\Gamma(\xi)$, $\Theta(\xi)$, $\Sigma(\xi)$, $U(\xi)$ are even functions of $\xi$, and $V(\xi)$ is an odd function of $\xi$. In doing so, we impose conditions
\begin{equation}
 V(0)=U'(0)=\Gamma'(0)=\Sigma'(0)=\Theta'(0)=0 \label{eq:bdry0}
\end{equation}
to have such extensions regular at origin. We regard \eqref{eq:ss-odes} as a boundary-value problem in the right half space $\xi \in [0,\infty)$ subject to the boundary conditions \eqref{eq:bdry0}. Because the system \eqref{eq:ss-odes} is singular, it is not clear in advance that how many conditions are needed to single out the solution. We will come to this point later in Section \ref{sec:char}.%, we also append the far field condition as $\xi \rightarrow \infty$ and yet another boundary conditions at $\xi=0$, detailing in the issue of fixing the unique solution, but for the moment we work upon \eqref{eq:bdry0}.

The system \eqref{eq:ss-odes} has its own scale invariance property: Provided $\big(\Gamma(\xi), V(\xi), \Theta(\xi), \Sigma(\xi), U(\xi)\big)$ is a solution of it, then the scaled version $\big(\Gamma_\rho(\xi), V_\rho(\xi), \Theta_\rho(\xi), \Sigma_\rho(\xi), U_\rho(\xi)\big)$ where
\begin{align*}
 \Gamma_\rho(\xi)&=\rho^{a_1}\Gamma(\rho\xi), & V_\rho(\xi)&=\rho^{b_1}V(\rho\xi), & \Theta_\rho(\xi)&=\rho^{c_1}\Theta(\rho\xi),\\
 \Sigma_\rho(\xi)&=\rho^{d_1}\Sigma(\rho\xi), & U_\rho(\xi)&=\rho^{b_1+1}U(\rho\xi)=\rho^{a_1}U(\rho\xi)
\end{align*}
is again a solution.

It is tempting to probe the monomials $$\big(\Gamma(\xi), V(\xi), \Theta(\xi), \Sigma(\xi), U(\xi)\big)=\big(A\xi^{-a_1}, B\xi^{-b_1},C\xi^{-c_1},D\xi^{-d_1},E\xi^{-a_1}\big)$$ obtained by setting $\rho=\xi^{-1}$ for a solution, forgetting the boundary conditions \eqref{eq:bdry0} for the moment. This did play the role of asymptotes of other solutions as $\xi \rightarrow \infty$ in \cite{KOT14}. However, in our study with the power law constitutive relation, the exponent $-(b_1+1)$ of $U(\xi)$ is less than $-1$ or it is integrable from $\infty$ to result in $V(\xi) = -\int_\xi^\infty U(\xi') \; d\xi' =V_\infty + B\xi^{-b_1}$, which ruins the simple monomial ansatz.

%
% where the monomials served as the proper asymptotes as $\xi \rightarrow \infty$. Whereas for our power law viscosity, as soon one attempts to do it, one realizes that the exponent $-(b_1+1)$ of $U(\xi)$ is less than $-1$ , or the leading order of $V(\xi) = -\int_\xi^\infty U(\xi') \; d\xi'$ is constant. This and leads to that. The same was observed for the strain-softening model equipped with a power law in KLT.

Nevertheless, the following multiplicative residuals to the monomials are the ones we work with, which give rise to a nice scaling property. For $(\bg,\bv,\bth,\bs,\tu)$ namely
\begin{equation} \label{eq:CAPtoBAR}
\begin{aligned}
 \bg(\xi)&=\xi^{a_1}\Gamma(\xi), &
 \bv(\xi)&=\xi^{b_1}V(\xi), &
 \bth(\xi)&=\xi^{c_1}\Theta(\xi), \\
 \bs(\xi)&=\xi^{d_1}\Sigma(\xi), &
 \bu(\xi)&=\xi^{b_1+1}U(\xi),
\end{aligned}
\end{equation}
the system they satisfy is
% These variables result in a nice property;
\begin{equation} \label{eq:barsys}
 \begin{aligned}
  a_0\bg + \lambda\xi\bg' &=\bu,\\
  b_0\bv + \lambda\xi\bv' &=-d_1 \bs + \xi\bs',\\
  c_0\bth+ \lambda\xi\bth'&=\bs\bu,\\
  \ts &=\bth^{-\alpha}\bg^m\bu^n,\\
  -b_1\bv+\xi\bv' &= \bu.
 \end{aligned}
\end{equation}
Now, introduce the new independent variable $\eta = \log\xi$ and define variables $(\tg,\tv,\tth,\ts,\tu)$ accordingly by
\begin{equation} \label{eq:BARtoTIL}
\begin{aligned}
 \tg(\log\xi)&=\bg(\xi), &
 \tv(\log\xi)&=\bv(\xi), &
 \tth(\log\xi)&=\bth(\xi), \\
 \ts(\log\xi)&=\bs(\xi), &
 \tu(\log\xi)&=\bu(\xi).
\end{aligned}
\end{equation}
Noticing that $\frac{d}{d\eta}\tg(\eta) = \xi \frac{d}{d\xi}\bg(\xi)$, we come to an autonomous system
\begin{equation} \label{eq:tildesys}
 \begin{aligned}
  a_0\tg + \lambda\dtg &=\tu,\\
  b_0\tv + \lambda\dtv &=-d_1 \ts + \dts,\\
  c_0\tth+ \lambda\dtth&=\ts\tu,\\
  \ts &=\tth^{-\alpha}\tg^m\tu^n.\\
  -b_1\tv+\dtv &= \tu,
 \end{aligned}
\end{equation}
We used the notation $\dot{f}=\frac{df}{d\eta}$.



\subsection{$(p,q,r,s)$-system derivation}
At \eqref{eq:tildesys} we arrived  an autonomous system. For this system, we observe that not all the variables equilibrate. To see this, suppose $\tu \rightarrow \tu_\infty\ge0$ as $\eta \rightarrow \infty$. Then from the last equation in \eqref{eq:tildesys}, we conclude that $\tv \rightarrow \infty$ and thus $\ts \rightarrow \infty$ either by the second equation. This raises difficulties in analyzing system and we can come up with proper variables all of which equilibrate simultaneously as $\eta \rightarrow \infty$ and $\eta \rightarrow -\infty$.

This can be heuristically done by looking at the system \eqref{eq:ss-odes} for $\big(\Gamma,V,\Theta,\Sigma,U)$. First, observe that
$$ f \sim \xi^\rho=e^{\rho\eta} \quad \text{as $\eta \rightarrow \infty$ (as $\xi \rightarrow \infty$) implies} \quad \frac{\dot{f}}{f} \rightarrow \rho,$$
$$ f \sim \xi^{\rho'}=e^{\rho'\eta} \quad \text{as $\eta \rightarrow -\infty$ (as $\xi \rightarrow 0$) implies} \quad \frac{\dot{f}}{f} \rightarrow \rho',$$
% $$ f \sim \xi^\rho'=e^{\rho'\eta} \quad \text{as $\eta \rightarrow \infty$ (as $\xi \rightarrow \infty$) implies} \quad \frac{\dot{f}}{f} \rightarrow \rho' \quad \text{as $\eta \rightarrow -\infty$ (as $\xi \rightarrow 0$) },$$
% $$ \frac{\dot{f}}{f} \rightarrow 0 \quad \text{as $\eta \rightarrow -\infty$ implies that } \quad f \rightarrow const. \quad \text{as $\xi \rightarrow 0$ and}$$
$$ \frac{\dot{(f/g)}}{f/g} = \frac{\dot{f}}{f} - \frac{\dot{g}}{g}. $$
This dictates that if the asymptotic behavior of the variables as $\eta \rightarrow \infty$ (resp. $\eta \rightarrow -\infty$) are known a priori, then the ratio of two variables that share the asymptotic order will equilibrate as $\eta \rightarrow \infty$ (resp. $\eta \rightarrow -\infty$). By looking at the system \eqref{eq:ss-odes} and presuming the polynomial asymptotics, we find the ratios
\begin{equation}\label{eq:pqrdef} 
 \begin{aligned}
  p \triangleq \frac{ \xi^{a_1} \Gamma(\xi)}{\xi^{d_1} \Sigma(\xi)}=\frac{\tg}{\ts}, \quad q \triangleq b\frac{ \xi^{b_1} V(\xi) }{ \xi^{d_1} \Sigma(\xi)}=b \frac{\tv}{\ts},  \quad r \triangleq \frac{ U(\xi) }{ \Gamma(\xi) } = \frac{\tu}{\tg}, \quad s \triangleq \frac{\Sigma(\xi)\Gamma(\xi)}{\Theta(\xi)} = \frac{\ts\tg}{\tth}
 \end{aligned}
\end{equation}
 work and $(p,q,r,s) \leftrightarrow (\tg,\tv, \tth,\ts)$ is a bijection.

% Define
% \begin{equation}\label{eq:pqrsdef}
%  \begin{aligned}
%   p &= \frac{\tg}{\ts}, & q&=b \frac{\tv}{\ts}, &  r &= \frac{\tu}{\tg}, & s&=\frac{\ts\tg}{\tth}.
%  \end{aligned}
% \end{equation}

With the calculation that is cumbersome but straightforward, we write
\begin{align*}
 \frac{\dpp}{p}&=\frac{\dtg}{\tg} - \frac{\dts}{\ts}& &=\left[\frac{1}{\lambda }\Big(\frac{\tu}{\tg}-a_0\Big)\right] & &-\left[d_1 + b\frac{\tv}{\ts} + \lambda\frac{\tu}{\tv}\frac{\tv}{\ts}\right]\\
 \frac{\dqq}{q}&=\frac{\dtv}{\tv} - \frac{\dts}{\ts}& &=\left[b_1 +\frac{\tu}{\tv}\right] & &-\left[d_1 + b\frac{\tv}{\ts} + \lambda\frac{\tu}{\tv}\frac{\tv}{\ts}\right]\\
 n\frac{\drr}{r}&=-(m+n)\frac{\dtg}{\tg}+\frac{\dts}{\ts} + \alpha\frac{\dtth}{\tth} & &=\left[\frac{-(m+n)}{\lambda}\Big(\frac{\tu}{\tg}-a_0\Big)\right]& &+\left[d_1 + b\frac{\tv}{\ts} + \lambda\frac{\tu}{\tv}\frac{\tv}{\ts}\right] + \left[\frac{\alpha}{\lambda }\Big(\frac{\ts\tu}{\tth}-c_0\Big)\right]\\
 \frac{\dot{s}}{s} &= \frac{\dtg}{\tg} + \frac{\dts}{\ts} - \frac{\dtth}{\tth} & &=\left[\frac{1}{\lambda }\Big(\frac{\tu}{\tg}-a_0\Big)\right] & &+\left[d_1 + b\frac{\tv}{\ts} + \lambda\frac{\tu}{\tv}\frac{\tv}{\ts}\right] -\left[\frac{1}{\lambda }\Big(\frac{\ts\tu}{\tth}-c_0\Big)\right].%\\
%  \dot{s} &=\frac{\partial s}{\partial (z-1)} \dot{z} &&= \frac{1+m}{\lambda}\frac{\partial s}{\partial (z-1)} \bigg\{z\big[-r - \frac{n}{D}\Big]+ ru^n\bigg\}.
\end{align*}
Noticing that
\begin{align*}
 \frac{\ts\tu}{\tth} = rs, \quad \frac{\tu}{\tv} = \frac{\ts}{\tv} \frac{\tg}{\ts} \frac{\tu}{\tg} = \frac{bpr}{q}, \quad \frac{\tu}{\tv} \frac{\tv}{\ts} = pr,
\end{align*}
we derive the $(p,q,r,s)$-system:
\begin{equation}\label{eq:slow} \tag{S}
 \begin{aligned}
 \dot{p} &=p\Big(\frac{1}{\lambda}(r-a) + 2- \lambda p r -q\Big),\\
 \dot{q} &=q\Big(1 -\lambda p r -q\Big) + b p r,\\
 n\dot{r} &=r\Big(\frac{\alpha-m-n}{\lambda(1+\alpha)}(r-a) + \lambda pr + q +\frac{\alpha}{\lambda}r\big(s- \frac{1+m+n}{1+\alpha}\big) + \frac{n\alpha}{\lambda(1+\alpha)}\Big),\\
 \dot{s} &=s\Big(\frac{\alpha-m-n}{\lambda(1+\alpha)}(r-a) + \lambda pr + q - \frac{1}{\lambda}r\big(s- \frac{1+m+n}{1+\alpha}\big) - \frac{n}{\lambda(1+\alpha)}\Big).
 \end{aligned}
\end{equation}
In the following a few sections, we analyze \eqref{eq:slow} as an autonomous system: We begin with sorting out the equilibrium points and conduct the linear stability. Most importantly, \eqref{eq:slow} possesses the {\it fast-slow} structure because of the small parameter $n$ in the left-hand-side of $\eqref{eq:slow}_3$; the dynamics of $r$ can be distinctively faster than those of the rest.

\section{Equilibria and their linear stability} \label{sec:equil}
The $(p,q,r,s)$-system \eqref{eq:slow} admits quite several equilibrium sets. They are listed in the Appendix \ref{append:lin}. Other than examining all of them, we first set forth our region of interest in a nonnegative sector $\{(p,q,r,s) \; | \; p\ge0, q\ge0, r\ge0, s\ge0 \}$. Moreover, we will be able to conceive the three dimensional submanifold $K$ by \eqref{eq:implicit} later that is positively invariant and $r$ and $s$ have strictly positive lower bound on it. %is contained in the narrower region $I \triangleq\left\{ \: (p,q,r,s) \: | \:  p,q\ge0, ~~ r,s\ge\delta>0\right\}$.
% \begin{align*}
%  \left\{ \: (p,q,r,s) \: | \:  p,q\ge0, ~~ r,s\ge\delta>0, ~~\left|s-\frac{1+m}{1+\alpha}\right| \le A \frac{\alpha-m}{\alpha(1+\alpha)}\: \right\}.
% \end{align*}
We postpone showing the positively invariance of $K$ until Section \ref{sec:proof} while we only examine two equilibrium points $M_0$ and $M_1$ that reside on $K$.

The reason why we reject $r,s\le0$ is as follows. If we transform back to the original variables, then we find
\begin{equation*}
 r(\eta)|_{\eta=t^\lambda x}=t\partial_t\log \gamma(t,x), \quad r(\eta)s(\eta)|_{\eta=t^\lambda x}=t\partial_t \log \theta(t,x).
\end{equation*}
%for the original variables $\theta(t,x)$ and $\gamma(t,x)$.
In case the variables $\theta(t,x)$ and $\gamma(t,x)$ have the polynomial asymptotic behaviors as $t \rightarrow \infty$ for a fixed $x$, above quantities pick the leading order exponents of the asymptotic expansions as $t \rightarrow \infty$, i.e., if $\gamma(t,x) = \mathcal{O}(t^\rho)$ as $t \rightarrow \infty$, then $t\partial_t\log \gamma(t,x) \rightarrow \rho$ as $t \rightarrow \infty$. Having $r,s>0$ is a reflection of our expectation that $\gamma(t,x) = \mathcal{O}(t^{\rho_1})$, $\theta(t,x) = \mathcal{O}(t^{\rho_2})$ for $\rho_1,\rho_2 >0$ as $t \rightarrow \infty$. Clearly, $t \rightarrow \infty$ when $\xi \rightarrow \infty$ for $x\ne0$. The reason is because the specimen keeps loading driven by the velocity assigned at far field in the problem configuration. Thus, regardless of the position $x$, we expect the strain $\gamma(t,x)$ always increases and does so to  $\infty$ as $t \rightarrow\infty$. On the same time, since the process is adiabatic, the heat produced at a point $x$ accumulates. Clearly energy keeps pumped into the domain from the far field driving velocity. Nevertheless the domain is unbounded in our study, we presume the temperature $\theta(t,x)$ always increases and does so to  $\infty$ as $t \rightarrow\infty$. The critical case where $r$ and $s$ can take $0$ will be quite subtle but we do not include those in our study.
% If the domain was bounded then clearly  but we expect the similar behavior. 

Now, we return to the equilibrium points in the region of interest.
Taking that interested orbits are confined in the region of positively invariance granted, we find only two equilibrium points relevant
\begin{align*}
 M_0 &= (0,0,r_0,s_0), & r_0 &=\frac{2+2\alpha-n}{D} + \frac{2+2\alpha}{D}\lambda, & s_0&=\frac{1+m+n}{1+\alpha} - \frac{n}{(1+\alpha)r_0},\\
 M_1 &= (0,1,r_1,s_1), & r_1 &= r_0-\frac{1+\alpha}{\alpha-m-n}\lambda, & s_1&=\frac{1+m+n}{1+\alpha} - \frac{n}{(1+\alpha)r_1}.
\end{align*}
Note that $r_1,s_1>0$ only under the constraint that %$\lambda$ is not arbitrarily large but
\begin{equation} \label{eq:lambda-range}
 0< \lambda < \frac{2(\alpha-m-n)}{1+m+n}\left(\frac{1+m}{1+m+n}\right).
\end{equation}

% \subsection{Linear stability of $M_0$ and $M_1$}
Here and hereafter, we denote the four eigenvalues and four eigenvectors of $M_i$, $i=0,1$  by $\mu_{ij}$ and $X_{ij}$ and $j=1,2,3,4$. %Appendix \ref{append:lin} contains the complete exposition of what this section discusses. 
\begin{itemize}
 \item $M_0$ is a saddle; it has three positive eigenvalues and one negative eigenvalue. 
 \begin{equation} \label{eq:eigM0}
  \mu_{01} = 2, \quad \mu_{02}=1, \quad \mu_{03}=\mu_0^+=\BO\Big(\frac{1}{n}\Big)>0, \quad \mu_{04}=\mu_0^{-}<0,
 \end{equation}
  where $\mu_0^\pm$ is respectively a positive and a negative solution of the quadratic equation
%  $$ \mu^2 - \mu\Big(\frac{r_0(1-s_0)}{n\lambda}-\frac{r_0s_0+1}{\lambda}\Big) - \frac{r_0^2s_0(\alpha-m-n)}{n\lambda^2}=0.$$
 $$ \Big(\mu - \frac{r_0}{n}\Big(\frac{1-s_0}{\lambda}-\frac{n}{\lambda r_0}\Big)\Big)\Big(\mu + \frac{s_0r_0}{\lambda}\Big) - \frac{s_0r_0}{n} \frac{1-s_0}{\lambda}\frac{\alpha r_0}{\lambda} = 0.$$
The leading orders of $\mu_0^\pm$ are given by
$$\mu_0^+ = \frac{\alpha-m}{n\lambda(1+\alpha)}\frac{2(1+\alpha)(1+\lambda)}{(1+2\alpha-m)}+\BO(1), \quad\mu_0^- = -\frac{1+m}{\lambda}\frac{2(1+\alpha)(1+\lambda)}{(1+2\alpha-m)}  + \BO(n).$$
Notice that the one of the positive eigenvalue $\mu_{03}$ is $\mathcal{O}( \frac{1}{n})$, which indicates the separably fast dynamics along the direction $X_{03}$. We will make use of this structure later.
% $$\mu_0^+ = \frac{r_0(1-s_0)}{n\lambda} + \frac{1}{\lambda}\frac{{r_0^2s_0}(\alpha-m-n)}{ {r_0(1-s_0)}-n(1+r_0s_0) } + \BO(n), \quad \mu_0^- = -\frac{1}{\lambda}\frac{{r_0^2s_0}(\alpha-m-n)}{ {r_0(1-s_0)}-n(1+r_0s_0) } + \BO(n).$$
 While the precise values of eigenvector components are presented in the Appendix \ref{append:lin}, we find the directions they point in the Figure \ref{fig:equilibria} for $n$ sufficiently small.
 \item $M_1$ is a saddle; it has one positive eigenvalue and three negative eigenvalues. 
\begin{equation} \label{eq:eigM1}
 \mu_{11}=-\frac{1+m+n}{\alpha-m-n}, \quad \mu_{12}=-1, \quad \mu_{13}=\mu_1^+=\BO\Big(\frac{1}{n}\Big)>0, \quad \mu_{14}=\mu_1^{-}<0,
\end{equation}
where $\mu_1^\pm$ is respectively a positive and a negative solution of the quadratic equation
 $$ \Big(\mu - \frac{r_1}{n}\Big(\frac{1-s_1}{\lambda}-\frac{n}{\lambda r_1}\Big)\Big)\Big(\mu + \frac{s_1r_1}{\lambda}\Big) - \frac{s_1r_1}{n} \frac{1-s_1}{\lambda}\frac{\alpha r_1}{\lambda} = 0$$
The leading orders of $\mu_1^\pm$ are given by
\begin{align*}
\mu_1^+ &= \frac{\alpha-m}{n\lambda(1+\alpha)}\Big(\frac{2(1+\alpha)(1+\lambda)}{(1+2\alpha-m) } - \frac{1+\alpha}{\alpha-m-n}\lambda\Big) + \BO(1), \\
\mu_1^- &= -\frac{1+m}{\lambda}\Big(\frac{2(1+\alpha)(1+\lambda)}{(1+2\alpha-m) } - \frac{1+\alpha}{\alpha-m-n}\lambda\Big) + \BO(n).
\end{align*}
Notice again that the positive eigenvalue $\mu_{13}$ is $\mathcal{O}( \frac{1}{n})$.

Differently from $M_0$, eigenvalues of $M_1$ have chances to be repeated. The exposition in the Appendix \ref{append:lin} specifies the possible combinations completely and the eigenvectors or the generalized eigenvectors are provided accordingly.
\end{itemize}
\begin{figure}
 \centering
  \psfrag{x0}{\scriptsize $M_0$}
  \psfrag{x1}{\scriptsize $M_1$}
  \psfrag{x2}{~~\scriptsize $1$}
  \psfrag{x3}{}
  \psfrag{p}{\scriptsize $p$}%=\frac{\gamma}{\sigma}$}
  \psfrag{q}{\scriptsize~~~$q$}%=n\frac{v}{\sigma}$}
  \psfrag{q*}{}%=\frac{2-n}{\lambda}$}
  \psfrag{r*0}{}%\hskip -15pt$r_0=1+\frac{2\lambda}{2-n}$}
  \psfrag{r*1}{}% \hskip -35pt$r_1=1-\frac{n\lambda}{(2-n)(1-n)}$}
  \psfrag{r*2}{}
  \subfigure[$pqr$-space]{
  \psfrag{r}{\scriptsize$r$}%=\big(\sigma\gamma^{(1-n)}\big)^{\frac{1}{n}}$}
  \includegraphics[width=6cm]{equilibriapqr.eps}\label{fig:eq1}
  }
  \quad \quad
  \subfigure[$pqs$-space]{
  \psfrag{r}{\scriptsize$s-\frac{1+m}{1+\alpha}$}%=\big(\sigma\gamma^{(1-n)}\big)^{\frac{1}{n}}$}
  \includegraphics[width=6cm]{equilibriapqs.eps}\label{fig:eq2}
  }
  \caption{Eigenvectors around $M_0$ and $M_1$ in $pqr$-space and in $pqs$-space respectively ($\mu_{11}\ne-1$ and $n\ll1$). } \label{fig:equilibria}
\end{figure}

\section{Characterization of the heteroclinic orbit} \label{sec:char}
As seen from the preceding section, $M_0$ has three dimensions for the unstable manifold and one for the stable manifold while $M_1$ has three dimensions for the stable manifold and one for the unstable manifold. Due to high dimensionality, it is rather not immediate to read off the behavior of the flow in the phase space. This section aims to develop a rough picture illustrating the flow mainly on the positive sector $p,q,r,s \ge0$. We discuss how the local stable and unstable manifolds extend globally and form a family of heteroclinic orbits. Our final goal in this section is to give a geometric characterization of the orbit we wanted to construct among those, which we will call $\chi(\eta)$ for the rest of this paper.

Recall that we determined one end point behavior, namely $\chi(\eta) \rightarrow M_1$ as $\eta \rightarrow \infty$ from the discussion on the incremental nature of the $\theta(t,x)$ and $\gamma(t,x)$ as $t \rightarrow \infty$. Now, we fix the asymptotic behavior of $\chi(\eta)$ as $\eta \rightarrow -\infty$. The boundary conditions \eqref{eq:bdry0} at $\xi=0$ does this task. Next proposition states how \eqref{eq:bdry0} are transmitted to asymptotic conditions for the $(p,q,r,s)$.

\begin{proposition} \label{prop1}
    Suppose $\big(\Gamma,V,\Theta,\Sigma,U\big)$ is a solution of \eqref{eq:ss-odes}, \eqref{eq:bdry0} that is smooth and bounded near $\xi=0$. Then the corresponding orbit defined by transformations \eqref{eq:CAPtoBAR}, \eqref{eq:BARtoTIL}, \eqref{eq:pqrdef} $\chi(\eta) = (p(\eta), q(\eta), r(\eta),s(\eta)) \rightarrow M_0$ as $\eta \rightarrow -\infty$. Furthermore, it meets $M_0$ along the direction of the first eigenvector $X_{01}$, i.e.,
    \begin{equation} \label{eq:alpha}
     e^{-2\eta}\big(\chi(\eta) - M_0 \big) \rightarrow \kappa X_{01}, \quad \text{for some constant $\kappa\ne0$ as $\eta \rightarrow -\infty$.}
    \end{equation}
\end{proposition}
\begin{remark} \label{rem:alpha}
  That the orbit meets $M_0$ along $X_{01}$ is nontrivial. $M_0$ has three dimensions of unstable subspaces and $\mu_{02}(=1)<\mu_{01}(=2)<\mu_{03}(=\BO(\frac{1}{n}))$. %The unstable manifolds of one, two, and three dimensions are separable from the fastest asymptotic rate. 
  Then orbits that meet $M_0$ in the direction of $X_{01}$ with rate $e^{2\eta}$ are all on the two dimensional manifold whose tangent space at $M_0$ is spanned by $X_{01}$ and $X_{03}$.
%
%
%   Any orbit $\varphi(\eta)$ in the unstable manifold of $M_0$ is asymptotically expanded by %is characterized by a constant triple $(\kappa_1,\kappa_2,\kappa_3)$ whose asymptotic expansion in the neighborhood of $M_0$ is given by
%  \begin{equation}\label{eq:alpha-expan}
%   \varphi(\eta) - M_0 = \kappa_1 e^{\mu_{01}\eta} + \kappa_2 e^{\mu_{02}\eta} + \kappa_3 e^{\mu_{03}\eta} + \text{higher-order terms as $\eta \rightarrow -\infty$}.
%  \end{equation}
%   The orbits escaping with asymptotic rate faster than $e^{\eta}$ can be separated out, namely the unique surface whose tangent space at $M_0$ is spanned by $X_{01}$ and $X_{03}$. All orbits except one on this surface meet $M_0$ in the direction of $X_{01}$ with rate $e^{2\eta}$; exceptional one is the $r$-axis.
%

%
%
%
%   Because $\mu_{02}(=1)<\mu_{01}(=2)<\mu_{03}(=\BO(\frac{1}{n}))$, the second term associated to the eigenvalue $\mu_{02}$ in the right-hand-side dominates other terms in the limit $\eta \rightarrow -\infty$ unless $\kappa_1=0$.
\end{remark}
\begin{proof}
Assuming the smoothness and boundedness of $\big(\Gamma,V,\Theta,\Sigma,U\big)$ in the neighborhood of $\xi=0$ and from the boundary conditions \eqref{eq:bdry0}, the derivatives of $\big(\Gamma,V,\Theta,\Sigma,U\big)$ evaluated at $\xi=0$ are obtained by differentiating the system \eqref{eq:ss-odes} repeatedly.
% First we check
% \begin{align*}
%  &\Gamma'(0)=U'(0)=\Sigma'(0)= \Big(\frac{U}{\Gamma}\Big)'(0)=\Big(\frac{\Sigma\Gamma}{\Theta}\Big)'(0)=0,\\
%  &\frac{U}{\Gamma}(0) = a, \quad \frac{\Sigma\Gamma}{\Theta}(0)=\frac{c}{a}.
% \end{align*}
Re-write \eqref{eq:ss-odes}
\begin{align*}
  a + \lambda\xi\frac{\Gamma'}{\Gamma} &= \frac{U}{\Gamma}, &
  c + \lambda\xi\frac{\Theta'}{\Theta} &= \frac{\Sigma\Gamma}{\Theta} \frac{U}{\Gamma},\\
  (b+\lambda)U  + \lambda \xi U'(\xi) &= \Sigma^{''} = \Big(\frac{\Sigma\Gamma}{\Theta} \frac{\Theta}{\Gamma}\Big)^{''}, &
  \frac{\Big(\frac{\Sigma\Gamma}{\Theta}\Big)^{''}}{\frac{\Sigma\Gamma}{\Theta}} &= (1+m+n)\frac{\Gamma^{''}}{\Gamma}-(1+\alpha) \frac{\Theta^{''}}{\Theta} + n \frac{ \big(\frac{U}{\Gamma}\big)^{''}}{\frac{U}{\Gamma}}
\end{align*}
from which albeit cumbersome we conclude
\begin{align*}
&\frac{U}{\Gamma}(0) = a = r_0,  & \Big(\frac{U}{\Gamma}\Big)'(0)&=0, & \Big(\frac{U}{\Gamma}\Big)^{''}(0) &= \frac{\Gamma(0)}{\Sigma(0)} \frac{-2(b+\lambda)r_0}{\frac{1-s_0}{\lambda}-\frac{n}{r_0}\Big(\frac{2}{s_0} + \frac{r_0}{\lambda}\Big)\left(\frac{ \frac{1}{\lambda}+2}{ \frac{1+\alpha}{\lambda}r_0 + \frac{2}{s_0}}\right)},\\
&\frac{\Sigma\Gamma}{\Theta}(0) = \frac{c}{a} = s_0,  & \Big(\frac{\Sigma\Gamma}{\Theta}\Big)'(0)&=0, &
\Big(\frac{\Sigma\Gamma}{\Theta}\Big)^{''}(0) &= \frac{n}{r_0} \left(\frac{ \frac{1}{\lambda}+2 }{ \frac{1+\alpha}{\lambda}r_0 + \frac{2}{s_0}}\right)\Big(\frac{U}{\Gamma}\Big)^{''}(0).
\end{align*}
Now, we consider the Taylor expansions of $p(\log\xi)$, $q(\log\xi)$, $r(\log\xi)$ and $s(\log\xi)$ at $\xi=0$ using above and \eqref{eq:bdry0}.
\begin{align*}
 p(\log\xi) &= \frac{ \tg }{\ts} = \frac{ \xi^{a_1} \Gamma(\xi)}{\xi^{d_1} \Sigma(\xi)} = \xi^2\frac{\Gamma(\xi)}{\Sigma(\xi)} = \xi^2\frac{\Gamma(0)}{\Sigma(0)} + o(\xi^2) \\
 %&= \xi^2\Big(\frac{U(0)}{\Phi(0)}\Big)^{-n}\Phi(0)^{1+\frac{\alpha-n}{1+\alpha}} + o(\xi^2),\\
 q(\log\xi) &= b\frac{\tv}{\ts} = b\frac{ \xi^{b_1} V(\xi) }{ \xi^{d_1} \Sigma(\xi)} = b\xi\frac{ V(\xi) }{ \Sigma(\xi)} = b\xi^2 \frac{U(0)}{\Sigma(0)}+ o(\xi^2)=\xi^2 ~br_0\frac{\Gamma(0)}{\Sigma(0)} + o(\xi^2) \\
 %&= \xi^2\Big(b\frac{U(0)}{\Phi(0)}\Big)\Big(\frac{U(0)}{\Phi(0)}\Big)^{-n}\Phi(0)^{1+\frac{\alpha-n}{1+\alpha}} + o(\xi^2),\\
 r(\log\xi) &= \frac{\tu}{ \tg } = \frac{ \xi^{1+b_1}U(\xi) }{ \xi^{a_1}\Gamma(\xi) } = \frac{ U(0) }{ \Gamma(0) }+ \xi \Big(\frac{U}{\Gamma}\Big)'(0) + \frac{1}{2}\xi^2\Big(\frac{U}{\Gamma}\Big)^{''}(0) + o(\xi^2)\\
  &=\frac{ U }{ \Gamma }(0) + \xi^2\frac{\Gamma(0)}{\Sigma(0)} \frac{-(b+\lambda)r_0}{\frac{1-s_0}{\lambda}-\frac{n}{r_0}\Big(\frac{2}{s_0} + \frac{r_0}{\lambda}\Big)\left(\frac{ \frac{1}{\lambda}+2}{ \frac{1+\alpha}{\lambda}r_0 + \frac{2}{s_0}}\right)} ,\\
 s(\log\xi) &= \frac{\ts\tg}{\tth} = \frac{ \xi^{a_1+d_1}\Sigma(\xi)\Gamma(\xi) }{\xi^{c_1} \Theta(\xi)} = \frac{ \Sigma\Gamma }{\Theta}(0) + \xi \Big(\frac{ \Sigma\Gamma }{\Theta}\Big)^{'}(0) + \frac{1}{2}\xi^2\Big(\frac{ \Sigma\Gamma }{\Theta}\Big)^{''}(0) + o(\xi^2)\\
 &=\frac{ \Sigma\Gamma }{\Theta}(0) + \xi^2 n \left(\frac{ \big(\frac{1}{\lambda}+2\big) \frac{1}{r_0} }{ \frac{1+\alpha}{\lambda}r_0 + \frac{2}{s_0}}\right)\frac{\Gamma(0)}{\Sigma(0)} \frac{-(b+\lambda)r_0}{\frac{1-s_0}{\lambda}-\frac{n}{r_0}\Big(\frac{2}{s_0} + \frac{r_0}{\lambda}\Big)\left(\frac{ \frac{1}{\lambda}+2}{ \frac{1+\alpha}{\lambda}r_0 + \frac{2}{s_0}}\right)}+ o(\xi^2).
\end{align*}
Therefore,
\begin{align*}
\chi(\log\xi)-M_0  = \big(p(\log\xi),q(\log\xi),r(\log\xi),s(\log\xi)\big) -M_0 =  \frac{\Gamma(0)}{\Sigma(0)}\xi^2 X_{01} + o(\xi^2),
\end{align*}
which is the \eqref{eq:alpha} for $\eta=\log\xi$.
\end{proof}
\begin{remark} \label{rem:signs}
For $n$ small enough, we find signs of second derivatives are definite: $\displaystyle \Big(\frac{U}{\Gamma}\Big)^{''}(0) <0$, $\displaystyle \Big(\frac{\Sigma\Gamma}{\Theta}\Big)^{''}(0) <0$, and
\begin{equation} \label{eq:second_der}
\begin{aligned}
\frac{\Gamma^{''}(0)}{\Gamma(0)} &= \frac{1}{2\lambda}\Big(\frac{U}{\Gamma}\Big)^{''}(0) < 0, &
\frac{\Theta^{''}(0)}{\Theta(0)} &= \frac{1}{2\lambda}\Big(s_0\Big(\frac{U}{\Gamma}\Big)^{''}(0) + r_0\Big(\frac{\Sigma\Gamma}{\Theta}\Big)^{''}(0)\Big)\Big)  < 0,\\
\frac{U^{''}(0)}{U(0)} &=\frac{\Gamma^{''}(0)}{\Gamma(0)} + \frac{ \big(\frac{U}{\Gamma}\big)^{''}(0)}{\frac{U}{\Gamma}(0)}< 0,&
\Sigma^{''}(0)&=(b+\lambda)U(0)>0.
\end{aligned}
\end{equation}
\end{remark}

% \subsection{Characterization of the heteroclinic orbit}
% For the cases $n>0$ but small, we hypothesize an invariant manifold $r=\hat{r}(p,q,s,n)$ that is called a {\it slow manifold} where the above three dimensional reduction is assumed. This is an invariant manifold of the suppressed vertical dynamics where the r-h-s of $\eqref{eq:slow}_3$ is kept small enough. Differently from the case $n=0$, orbits may leave the manifold but when it happens the dynamics is vertical, or the normal dynamics occurs much faster in the order of $\BO(\frac{1}{n})$. However, those orbits leave the manifold are not expected to return to form any heteroclinic orbit joining $M_0$ to $M_1$. For this reason, the study of vertical dynamics of \eqref{eq:slow} is not included in this paper. We focus only on the flow on the slow invariant manifold. Realization of this Chapman-Enskog type reduction is done in Section \ref{sec:proof} by geometric singular perturbation theory.
%
% Thanks to this reduction, the singularly perturbed problem \eqref{eq:slow} with respect to $n$ becomes the regularly perturbed problem on the reduced space. We expect qualitatively perturbed flow in the reduced space, namely
% $M_1$ is a stable node; $M_0$ is a saddle with two unstable dimensions and one stable dimension; two dimensional manifold $W^s(M_0)$ extends globally to form a family of heteroclinic orbits $W^s(M_0)\cap W^u(M_1)$; and $\chi(\eta)$ is the unique curve satisfying \eqref{eq:alpha} among them.

\subsection{A two-parameter family of heteroclinic orbits} \label{sec:twoparam}
Assuming the $\chi(\eta)$, this section is devoted to giving an interpretation of it in terms of the given data. By the data we refer to $\big(\Gamma(0),\Theta(0),\Sigma(0),U(0)\big)$, despite this is merely a choice. Since the system \eqref{eq:ss-odes} is singular it is not clear how many boundary conditions, for instances $\big(\Gamma(0),\Theta(0),\Sigma(0),U(0)\big)$, are independent. In the below, we clarify this issue.

The orbit curve satisfying \eqref{eq:alpha} is hypothesized to be one dimensional and achieved respectively for each parameter $(\lambda, \alpha, m,n)$. If $\chi(\eta)$ is the heteroclinic orbit, then so is the $\chi(\eta-\eta_0)$ for any $\eta_0\in \mathbb{R}$. In conclusion, in order to fix one heteroclinic orbit, the five parameters $\lambda$, $\eta_0$, $\alpha$, $m$, and $n$ are required. The latter three are in particular the material properties that respectively account for the thermal softening, strain hardening, and strain-rate hardening.

Other than those three, we associate the localization rate $\lambda$  and the translation factor $\eta_0$  to the physical data. Seen from \eqref{eq:ss-odes}, not all of them are independent but two out of six numbers $\lambda$, $\eta_0$, $\Gamma(0)$, $\Theta(0)$, $\Sigma(0)$, and $U(0)$ fixes the rest. We choose $\Gamma(0)$ and $U(0)$ for the primary parameters and specifies the rest in terms of them. $\Gamma(0)$ and $U(0)$ are the tip sizes of the strain and strain rate at $\xi=0$. To recap, we take a view that for each given triple of material parameters $(\alpha,m,n)$ there are two-parameters family of heteroclinic orbits by $\Gamma(0)$ and $U(0)$ .

As $r_0 = \frac{2(1+\alpha)-n}{D} + \frac{2(1+\alpha)}{D}\lambda = \frac{U(0)}{\Gamma(0)}$,
we put
\begin{equation} \label{eq:lambda}
 \lambda = \Big(\frac{U(0)}{\Gamma(0)} - \frac{2(1+\alpha)-n}{D}\Big)\frac{D}{2(1+\alpha)}.
\end{equation}
In particular, the restriction \eqref{eq:lambda-range} on $\lambda$ reads as the ratio condition
\begin{equation} \label{eq:restriction}
\begin{aligned}
 \frac{2(1+\alpha) -n}{D} < \frac{U(0)}{\Gamma(0)} &< \frac{2(1+\alpha) -n}{D} + \frac{D(\alpha-m-n)(1+m)}{(1+\alpha)(1+m+n)^2}\\
 &=\frac{2(1+\alpha)}{1+m+n} -\frac{n}{D}\left( \frac{4(1+\alpha)(\alpha-m-n)}{(1+m+n)^2} +1\right).
\end{aligned}
\end{equation}
Once $\lambda$ is fixed, straightforward calculation gives
$$\Theta(0) = \left(\frac{\Gamma(0)^m U(0)^{1+n}}{c}\right)^{\frac{1}{1+\alpha}}, \quad \Sigma(0) = c^{\frac{\alpha}{1+\alpha}}\Gamma(0)^{\frac{m}{1+\alpha}} U(0)^{-\frac{\alpha-n}{1+\alpha}}.$$

$\eta_0$ is defined by following procedure. Any orbit $\varphi(\eta)$ escaping $M_0$ in the direction $X_{01}$ is asymptotically expanded by %is characterized by a constant triple $(\kappa_1,\kappa_2,\kappa_3)$ whose asymptotic expansion in the neighborhood of $M_0$ is given by
 \begin{equation}\label{eq:alpha-expan}
  \varphi(\eta) - M_0 = \kappa_1 e^{\mu_{01}\eta} + \kappa_3 e^{\mu_{03}\eta} + \text{higher-order terms as $\eta \rightarrow -\infty$}.
 \end{equation}
Let $(\bar\kappa_1,\bar\kappa_3)$ be fixed and let $\bar\chi(\eta)$ be the heteroclinic orbit whose asymptotic expansion \eqref{eq:alpha-expan} is characterized by that. We look for $\chi(\eta) = \bar\chi(\eta-\eta_0)$. Then \eqref{eq:rapid} gives
$$\kappa_1 X_{01}=\lim_{\eta \rightarrow -\infty}\big(\chi(\eta) - M_0\big)e^{2\eta} = \lim_{\eta \rightarrow -\infty} \big(\bar\chi(\eta-\eta_0) - M_0\big)e^{2(\eta-\eta_0)}e^{2\eta_0} = e^{2\eta_0}\bar\kappa_1 X_{01}.$$
Thus $\eta_0 = \frac{1}{2}\log {\frac{\kappa_1}{\bar\kappa_1}}$ but from the proof of Proposition \ref{prop1}, we know that
$\kappa_1 = \frac{\Gamma(0)}{\Sigma(0)}$, or
\begin{equation}
 \eta_0 = \frac{1}{2}\log \left(\frac{\Gamma(0)}{\Sigma(0)\bar\kappa_1}\right).%\log \sqrt{\frac{\bar\kappa_1}{\frac{\Gamma(0)}{\Sigma(0)}}}.
\end{equation}
%The closer description of the $\big(\Gamma,V,\Theta,\Sigma,U\big)$ is supplemented in the last section.

% \section{Existence via Geometric theory of singular perturbation}
% In this section, we give a proof for the existence of the heteroclinic orbit hypothesized in the preceding section. It is accomplished by the two consecutive chunks of arguments, the geometric singular perturbation theory and the theorem of Poincar\'e-Bendixson on the positively invariant set.
%
% Considering $n$ as a small parameter, we take a point of view on the $(p,q,r)$-system regarding it a singularly perturbed problem. See the term $n\dot{r}$ in $\eqref{eq:pqrsys}_3$, which indicates that the evolution of $r$ is of faster time scale, and we refer to $r$ as a fast variable and the others $p$ and $q$ two slow variables.
%
% The geometric singular perturbation theory considers the dynamics that takes place near the zeroset of the right-hand-side of $\eqref{eq:pqrsys}_3$, where the time scale of the evolution of the fast variable possibly becomes comparable to that of slow variables. In particular for the critical case $n=0$, the orbits that are upon the zeroset are considered, on where the problem is essentially reduced to that of slow variables only and it becomes regularly perturbed problem. The upshot of the geometric singular perturbation theory is to enable continuing this reduction to $n>0$ provided $n$ is small.
%
% To be concrete and to get prepared to apply the geometric singular perturbation theory, in this section we examine two objects and one verification of the property of the latter: First, we examine the zeroset of the right-hand-side of $\eqref{eq:pqrsys}_3$ when $n=0$, which is given by the graph
% $$r=\frac{ \frac{\alpha c_0}{\lambda} - d_1 -q }{ \frac{\alpha}{\lambda} + \lambda p}\triangleq h(p,q;\lambda,\alpha,n=0) \quad \text{for $r>0$}.$$
% We decribe this graph in the phase space to probe an idea of developing arguments. A suitable compact piece of the graph is subjected to applying the theory and we refer to it as the {\it critical manifold}. This is our second object and we give the precise specification of this object. The critical manifold serves as the template invariant manifold when $n=0$ from which the perturbed invariant manifold when $n>0$ is disposed of.  Lastly, the {\it normally hyperbolicity} of the critical manifold is verified. This is in regard with the system \eqref{eq:pqr_fast} in the fast independent variable. By rescailing $\tilde\eta=\frac{\eta}{n}$ for $n>0$, we see the dynamics that occurs in the fast time scale. The critical case $n=0$ in this format describes the fast relaxation toward the critical manifold or escape from it. The critical manifold is the set of equilibrium for the \eqref{eq:fastn0} and is the center manifold of the individual equilibrium points in it. Normally hyperbolicity concerns the hyperbolicity in the rest of the dimensions and this is the key property that enables us to continue the invariant manifold for $n>0$.
% %In the first chunk, we exploit the fact that the $(p,q,r)$-system is of multiple time scale: Observe  the small parameter $n$ multiplied to $\dot{r}$ in \eqref{eq:pqrsys}, which makes the evolution of $r$ fast, i.e., if
% %\begin{equation}
% % f(p,q,r;\lambda,\alpha,n) = r\Big( \Big[\frac{\alpha-n}{\lambda(1+n)}\Big(r^{1+n}-c_0\Big)\Big]+\Big[d_1 + q + \lambda pr\Big]\Big).
% %\end{equation}
% %the right-hand-side of the equation on $r$, $\dot{r} \sim \frac{1}{n}$ away from the zero set of $f(p,q,r;\lambda,\alpha,n)$ When $n=0$, the orbit is strictly restricted on the zero set
% %\begin{equation}
% % Z \triangleq \{\,(p,q,r)\; | \; f(p,q,r,\lambda,\alpha,n=0)\, \} \label{eq:zeroset}
% %\end{equation}
% %and the problem is essentially reduced to the one of only slow variable $p$ and $q$, provided $r$ is solved from the algebraic equation \eqref{eq:zeroset}.
% %
% %The graph $\displaystyle r=\frac{ \frac{\alpha c_0}{\lambda} - d_1 -q }{ \frac{\alpha}{\lambda} + \lambda p}$.
% %is taken and is referred to as the critical manifold.
% %
% %
% %
% %In order to take the suitable compact piece of the zero set $Z$, we detail in the graph $\displaystyle r=\frac{ \frac{\alpha c_0}{\lambda} - d_1 -q }{ \frac{\alpha}{\lambda} + \lambda p}$.
% %
% %
% %
% %
% %
% %
% %\hrulefill
% %
% %
% %
% %
% %
% %
% %
% %
% %This exploits the fact that the (p,q,r)-system is of multiple time scale, i.e., the presence of the small parameter $n$ in front of  $\dot{r}$ indicates the time scale of $\dot{r}\sim \frac{1}{n}$ unless the right-hand-side of the equation is small enough to compensate.
% %
% %Let $f(p,q,r;\lambda,\alpha,n)$ be the right-hand-side of the equation on $r$, i.e.,
% %\begin{equation}
% % f(p,q,r,\lambda,\alpha,n) = r\Big( \Big[\frac{\alpha-n}{\lambda(1+n)}\Big(r^{1+n}-c_0\Big)\Big]+\Big[d_1 + q + \lambda pr\Big]\Big).
% %\end{equation}
% %The compact subset of the zero set of $f(p,q,r;\lambda,\alpha,n=0)$
% %$$ Z \triangleq \{\,(p,q,r)\; | \; f(p,q,r,\lambda,\alpha,n=0)\, \} $$
% %is taken and is referred to as the critical manifold.
% %
% %
% %\subsection{Reduction to the slow system}
%
% \subsection*{The graph $\displaystyle r=\frac{ \frac{\alpha c_0}{\lambda} - d_1 -q }{ \frac{\alpha}{\lambda} + \lambda p}$.}
% The equation of the graph can be written in the form
% \begin{equation}
%  q + \lambda {r}p + \frac{\alpha}{\lambda} \Big( r-r_0\Big)=0, \label{eq:level}
% \end{equation}
% where we observe that the level line $r=\bar{r}$ is the straight line in the phase space. On the $(p,q)$-plane, for $r$ in the range of $(0,r_0)$ the contour line crosses the first quadrant with the negative slope, intersecting $p$-axis and $q$-axis. When $r=r_0$ it is the line passing the origin and this point $(0,0,r_0) = M_0$. When $r=r_1$, the level line passes the $(0,1,r_1)=M_1$.
%
% \subsection*{Critical manifold}
% Inequality
% $$ \lambda < \frac{2(\alpha-n)}{(1+n)^2} $$
% prevents $r_1$ from being less than equal to $0$. Therefore, we always can take the value $0<\underbar{R}<r_1$. Having fixed the value $\underbar{R}$, we take the closed set $T$ that is the triangle in the first quadrant enclosed by $p$-axis, $q$-axis and the contour line $\underbar{R} = \bar{r}(p,q)$. Observe that $h\ge\underbar{R}>0$ on the graph if $(p,q) \in T$. Since $h$ is continuous in the neighborhood of $T$, we can take another closed set $K$ in the vicinity of $T$ on which $h$ is still away from $0$. We take the compact piece of the set $Z$ by
% \begin{equation}
%  G(\lambda,\alpha,n=0) \triangleq \Big\{\, (p,q,r) \;|\; (p,q) \in K, \text{ and } r=\frac{ \frac{\alpha c_0}{\lambda} - d_1 -q }{ \frac{\alpha}{\lambda} + \lambda p} \,\Big\} \subset Z
% \end{equation}
%
% \subsection*{Normally hyperbolicity}
% The system in {\it fast scale} with the independent variable $\tilde{\eta} = \eta/n$ is
% \begin{equation}\label{eq:pqr_fast} \tag*{($\tilde{P}$)}
% \begin{aligned}
%  p^\prime &=np\Big( \Big[\frac{1+\alpha}{1+n}\,\frac{1}{\lambda }\Big(r^{1+n}-c_0\Big)\Big] -\Big[d_1 + q + \lambda pr\Big]\Big), \\
%  q^\prime &=nq\Big(\Big[b_1 +\frac{bpr}{q}\Big] -\Big[d_1 + q + \lambda pr\Big]\Big), \\
%  r^\prime &=r\Big( \Big[\frac{\alpha-n}{\lambda(1+n)}\Big(r^{1+n}-c_0\Big)\Big]+\Big[d_1 + q + \lambda pr\Big]\Big)\triangleq f(p,q,r;\lambda,\alpha,n),
% \end{aligned}
% \end{equation}
% where we denoted $\displaystyle(\cdot)^\prime = \frac{d}{d\tilde{\eta}}(\cdot)$. In particular, the system $(\tilde{P})|_{n=0}$ reads
% \begin{align}
%  p^\prime =0, \quad q^\prime =0, \quad r^\prime=r\Big( \Big[\frac{\alpha}{\lambda}\Big(r-c_0\Big)\Big]+\Big[d_1 + q + \lambda pr\Big]\Big) = f(p,q,r;\lambda,\alpha,0). \label{eq:fastn0}
% \end{align}
%
% \begin{lemma} \label{lem:normal_hyper}
%  $G(\lambda,\alpha,0)$ is a normally hyperbolic invariant manifold with respect to the system $(\tilde{P})|_{n=0}$.
% \end{lemma}
% \begin{proof}
% To prove the normally hyperbolicity of the graph $G(\lambda,\alpha,n=0)$, we show that the coefficient matrix of the linearized system of $(\tilde{P})|_{n=0}$ around $G(\lambda,\alpha,n=0)$ has the eigenvalue $0$ exactly with the multiplicity $2$. Let $P$, $Q$, and $R$ be the perturbations of $p$, $q$, and $r$ respectively. The linearized equations after discarding terms higher than the first order are
% \begin{align*}
%  \begin{pmatrix} {P}^\prime\\ {Q}^\prime \\ {R}^\prime \end{pmatrix} =
%  \begin{pmatrix} 0 & 0& 0\\ 0 & 0 & 0\\ \lambda h^2 & h & ( \frac{\alpha}{ \lambda} + \lambda p )h \end{pmatrix} \begin{pmatrix} {P}\\ {Q} \\ {R} \end{pmatrix},
% \end{align*}
% where $h$ is a shorthand for $h(p,q;\lambda,\alpha,n=0)$. $( \frac{\alpha}{ \lambda} + \lambda p )h > 0$ because $\alpha>0$, $p\ge0$ and $h > \underbar{R}>0$ on the $G(\lambda,\alpha,n=0)$, which proves that $0$ is an eigenvalue with multiplicity $2$.
% \end{proof}
%
% \begin{proposition}
% For $n>0$ sufficiently small, there is a function $h(p,q;\lambda,\alpha,n) : T\subset \mathbb{R}^2 \mapsto \mathbb{R}$ such that
% \begin{equation} \tag*{(${R}$)} \label{eq:reduced}
% \begin{aligned}
%  \dot{p} &=p\Big(\frac{1}{ \lambda }\big(h(p,q;\lambda,\alpha,n) - \frac{2-n}{1+m-n}\big) - \frac{1-m+n}{1+m-n} + 1-q- \lambda p h(p,q;\lambda,\alpha,n)\Big),\\
%  \dot{q} &=q\Big(                                                                          1-q- \lambda p h(p,q;\lambda,\alpha,n)\Big) + b^{\lambda,m,n}ph(p,q;\lambda,\alpha,n),
% \end{aligned}
% \end{equation}
%
%
% \begin{enumerate}
%  \item $h$ is jointly smooth function of $p$, $q$, and $n$ in $T \times I$
%  \item The graph $r=h(p,q;\lambda,\alpha,n)$ is locally invariant, i.e., The orbit $(p,q,r)$
% \end{enumerate}
%
% \end{proposition}
%
%
% \subsubsection{Flow on the critical manifold : the case $m=1$}
%
% The marginal case $m=1$ provides closer detail.
% By substituting $h^{\lambda,1,0}(p,q)$ in place of $r$, the system is explicitly solved and
% %we can solve the system explicitly and the whole critical graph is completely characterized.
% the general solution on the graph is a family of parabolae $p=kq^2$ and $r=h^{\lambda,1,0}(p,q)$. This includes the two extremes $p=0$ and $q=0$, where $k$ takes $0$ and $\infty$ respectively. See Figure \ref{fig:hn0m1}. We focus on discussing two points: 1) In an effort to apprehend the flow of the rest of cases, we remark a few features for this marginal case, which in turn persist under the perturbation; and 2) we report features that do not persist too. These features do not play any role in our study, but this bifurcation is described here for clarity.
%
% We address the first point. Look at $M_0^{ \lambda,1,0}$ in Figure \ref{fig:hn0m1_b} surrounded by a family of parabolae in the neighborhood. Our interested direction $\vec{X}_{02}$ and the other $\vec{X}_{01}$ are annotated near $M_0^{ \lambda,1,0}$ by a dotted arrow. The family of parabolae is manifesting the fact that orbit curves meet $M_0^{ \lambda,1,0}$ tangentially to $\vec{X}_{01}$; one exception is the degenerate straight line that emanates in $\vec{X}_{02}$, which is depicted as the green one in Figure \ref{fig:hn0m1}, the target orbit. Another observation from the $pq$-plane is that the flow in the first quadrant far away from the origin is {\it inwards}. More precisely, as illustrated in Figure \ref{fig:hn0m1_b}, whenever $0<\underbar{R} < 1 = c^{\lambda,1,0}$ the flow on the contour line $\underbar{R} = h^{\lambda,1,0}$ is inwards. We make use of this observation in the proof of Section \ref{sec:proof_proof}.
%
% Now, we describe the bifurcation of this marginal case. The crucial difference is that $M_1^{\lambda,1,0}$ is replaced by a line of equilibria $h^{\lambda,1,0}(p,q) = c^{\lambda,1,0}=1$, which is the red line in Figure \ref{fig:hn0m1}. As a result, each of the parabolae emanated from $M_0^{\lambda,1,0}$ lands at a point among these equilibria. $\vec{X}_{02}$ is immersed on $q=0$ plane distinctively from all other cases and the target orbit in particular lands at the $q$-intercept of the line of equilibria. To compare this observation to the statement of Theorem \ref{thm:1}, the target orbit does not connect $M_0^{ \lambda,1,0}$ to $M_1^{ \lambda,1,0}$ but to this $q$-intercept. This observation does not spoil our proof in Section \ref{sec:proof_proof} because we assert the persistence of the critical manifold not the target orbit.

\section{Existence via Geometric theory of singular perturbations} \label{sec:proof}
This section is devoted to proving what has been hypothesized in the preceding section to capture the heteroclinic orbit $\chi(\eta)$ in the form of Theorem \ref{thm1} in the below. 
\begin{theorem} \label{thm1}
Let $\Lambda$ be a domain of the tuple $(\lambda,\alpha,m,n)\in\mathbb{R}^4$ defined by %.Suppose the three parameters $(\lambda,m,n)$ are such that
 \begin{align*}
  \alpha>0\quad&\text{(thermal softening)},\\
  m>-1 \quad&\text{(strain softening/hardening)}, \\%\label{eq:a1}\\
  n>0 \quad&\text{(strain rate sensitivity)},\\ %\label{eq:a2}\\
  -\alpha+m+n<0 \quad&\text{(net softening)}, \\%\label{eq:a3}\\
  0< \lambda < \frac{2(\alpha-m-n)}{1+m+n}\left(\frac{1+m}{1+m+n}\right) \quad&\text{(localizing rate bound)}. %\label{eq:a4}
\end{align*}
 For each $(\lambda,\alpha,m,0) \in \Lambda$, there is $n_0( \lambda,\alpha,m)$ such that for $n \in [0, n_0)$, $(\lambda,\alpha,m,n) \in \Lambda$ and the system $(*)^{\lambda,\alpha,m,n}$ admits a heteroclinic orbit $\chi^{\lambda,\alpha,m,n}(\eta)$ joining equilibrium $M_0^{\lambda,\alpha,m,n}$ to equilibrium $M_1^{\lambda,\alpha,m,n}$ with property that
    \begin{align} \label{eq:rapid}
%         &\chi(\eta) \rightarrow M_1 \quad \text{as $\eta \rightarrow \infty$ and} \\
        e^{-2\eta}\big(\chi^{\lambda,\alpha,m,n}(\eta) - M_0^{\lambda,\alpha,m,n}\big) \rightarrow \kappa X_{01}^{\lambda,\alpha,m,n} \quad \text{as $\eta \rightarrow -\infty$ for some $\kappa\ne0$}.
    \end{align}
\end{theorem}

The heteroclinic orbit $\chi^{\lambda,\alpha,m,n}(\eta)$ is achieved via {\it geometric singular perturbation theory}; the presence of the small parameter $n>0$ in the left-hand-side of $(*)_3$ elucidates the {\it fast-slow} structure of the system, having $r$ as a fast variable and others as slow variables. We will come with detail in the below. In the course of applying the theory, a few steps, we present the full details, may be taken granted to those working in this field, because the arguments there are rather well-known. Those readers may find the proof of the Theorem \ref{thm1} in Section \ref{sec:thmproof} rational. 

Recall that \eqref{eq:slow} accounts for a family of dynamical systems parametrized by $(\lambda,\alpha,m,n)$; the heteroclinic orbit will be achieved respectively for each admissible $(\lambda,\alpha,m,n)$. Nonetheless, to make notations simple we suppressed the dependencies on parameters $\lambda$, $\alpha$, and $m$ but exposed on $n$. %, we only expose the dependencies on $n$ when $n>0$.%; various quantities with suppressed indices are understood as the ones for $n=0$.


\subsection{Invariant manifold theory and geometric singular perturbation theory}
We quickly materialize the set of statements in the geometric singular perturbation theory that are in use. We follow definitions from \cite{fenichel_asymptotic_1977,fenichel_geometric_1979}. %Slightly different definitions are found in \cite{HPS_1977}. %
We are going to use the Theorem 12.2 in \cite{fenichel_geometric_1979} that is wrapped up in the form of \cite[Theorem 2.2]{Sz1991} via \cite[Theorem 3]{fenichel_asymptotic_1977}.

Let $r\ge2$ for the rest of the paper. Let $X$ be a $C^{r}$ vector field in $\mathbb{R}^d$. $\bar{\Lambda}=\Lambda \cup \partial \Lambda$ is a compact, connected $C^{r+1}$ manifold in $\mathbb{R}^d$. $F^t: \mathbb{R}^d \mapsto \mathbb{R}^d$ denotes the time $t$-map associated with the vector field $X$ and $DF^t$ denotes its differential. 

$\bar{\Lambda}$ is said to be overflowing invariant under $X$ if for every $m\in\bar{\Lambda}$ and $t\le0$, $F^t(m)\in \bar{\Lambda}$ and $X$ is pointing strictly outward on $\partial \Lambda$. $T \mathbb{R}^d|\Lambda$ denotes the tangent bundle of $\mathbb{R}^d$ along $\Lambda$ and $T\Lambda$ denotes the tangent bundle of $\Lambda$. A subbundle $E\subset T\mathbb{R}^d|\bar{\Lambda}$ is said to be negatively invariant if $E\subset DF^t(E)$ for all $t\le0$. Let $E\subset T\mathbb{R}^d|\bar{\Lambda}$ be a subbundle that is negatively invariant and is containing $T\Lambda$. With a given such $E$,  $T \mathbb{R}^d|M$ then splits into $T\mathbb{R}^d|\bar{\Lambda} =E\oplus E'= T\Lambda\oplus N\oplus E'$, where $N\subset E$ is any complement of $T\Lambda$ in $E$ and $E'\subset T\mathbb{R}^d|\bar{\Lambda}$ is any complement of $E$ in $T\mathbb{R}^d|\bar{\Lambda}$.  

With those introduced, we probe the separabilities of the three subbundles by their asymptotic growth rates backward in time: Let $m\in \bar{\Lambda}$, $t\le0$ and $v^0 \in T_m \Lambda$; $w^0\in N_m$; $x^0\in E'_m$; $v^t = DF^t(m)v^0$; $w^t = \pi^N DF^t(m)w^0$; $x^t = \pi^{E'}DF^t(m)x^0$,
% \begin{align*}
%  v^0 &\in T_m M, \quad w^0\in N_m, \quad x^0\in E'_m,\\
%  v^t &= DF^t(m)v^0, \quad w^t = \pi^N DF^t(m)w^0, \quad x^t = \pi^{E'}DF^t(m)x^0, \quad \text{where $\pi^N$ and $\pi^{E'}$ are bundle projections.}
% \end{align*}
where $\pi^N$ and $\pi^{E'}$ are bundle projections onto $N$ and $E'$ respectively.
Now we define five numbers for each $m\in \bar{\Lambda}$:
\begin{align*}
 \nu^s(m) &\triangleq \inf \Big\{\nu>0 \: : \: \frac{1}{|x^{-t}|} = o(\nu^t) \quad \text{as $t \rightarrow \infty$} \quad \forall x^0\in E'_m\Big\}. \quad \text{If $\nu^s(m)<1$, define}\\
 \sigma^s(m) &\triangleq \inf \Big\{\sigma>0 \: : \: |v^{-t}| = o(|x^{-t}|^\sigma) \quad \text{as $t \rightarrow \infty$} \quad \forall x^0\in E'_m, v^0\in T_m\Lambda\Big\}.
\end{align*}
Further, define
\begin{align*}
 \alpha^u(m) &\triangleq \inf \Big\{\alpha>0 \: : \: |w^{-t}| = o(\alpha^t) \quad \text{as $t \rightarrow \infty$}\quad \forall w^0\in N_m\Big\}. \quad \text{If $\alpha^u(m)<1$, define}\\
 \rho^u(m) &\triangleq \inf \Big\{\rho>0 \: : \: \frac{|w^{-t}|}{|v^{-t}|} = o(\rho^t) \quad \text{as $t \rightarrow \infty$} \quad \forall w^0\in N_m, v^0\in T_m\Lambda\Big\}.\quad \text{If $\rho^u(m)<1$, define}\\
 \tau^u(m) &\triangleq \inf \Big\{\tau>0 \: : \: |\hat{v}^{-t}| = o\left(\Big(\frac{|v^{-t}|}{|w^{-t}|}\Big)^{\tau}\right) \quad \text{as $t \rightarrow \infty$} \quad \forall w^0\in N_m, v^0\in T_m\Lambda,\hat{v}^0\in T_m\Lambda\Big\}.
\end{align*}

\begin{definition} \label{def:over}
Let $\bar{\Lambda}=\Lambda \cup \partial\Lambda$  an overflowing invariant manifold and $E$ a subbundle over it be given as above. We say an overflowing invariant manifold $\bar{\Lambda}$ satisfies assumptions \eqref{eq:A} and \eqref{eq:B}, $r'\le r-1$, with the subbundle $E$ if for all $m\in \bar{\Lambda}$ the rate numbers
\begin{align}
\nu^s(m)&<1, \quad \sigma^s(m)<\frac{1}{r}, \label{eq:A}\tag{$A_r$} \\
\alpha^u(m)&<1, \quad \rho^u(m)<1, \quad \tau^u(m)<\frac{1}{r'}. \label{eq:B}\tag{$B_{r'}$}
%\quad \text{for all $m\in \bar{\Lambda}$}.
\end{align}
\end{definition}
\begin{remark}
 \cite{fenichel_asymptotic_1977} considered more general assumptions and results and there $\rho_2$ and $\tau_2$ are additionally computed but \eqref{eq:A} and \eqref{eq:B} are stronger conditions to have them valid. %.implies the conditions for $\rho_2$ and $\tau_2$.
\end{remark}
\begin{definition}[Normally Hyperbolic Invariant Manifold] \label{def:nhim}
 Let $\Lambda$ be a compact manifold without boundary, invariant under $X$. Let $E^s$ and $E^u$ be subbundles containing $TM$ of $T \mathbb{R}^d|\Lambda$ such that $E^s + E^u = T \mathbb{R}^d|M$, $E^u$ is negatively invariant under $X$ and $E^s$ is so under $-X$. We say $\Lambda$ is $r$-normally hyperbolic if $\Lambda$ is an overflowing invariant manifold with a subbundle $E^u$ satisfying rate assumptions \eqref{eq:A} and $\Lambda$ is so with $E^s$ under $-X$. 
\end{definition}
The two notions for a certain invariant manifold in Definition \ref{def:over} and \ref{def:nhim} are for a certain type of persistence theorem under the perturbation of the vector field. Before we introduce the persistence theorem by Fenichel, we introduce the notion of the transversal intersection of two submanifolds in a phase space $\mathcal{M}$.
\begin{definition}[Transversal Intersection]  (\cite[Definition 3.1]{Sz1991})
 Let ${\mathcal{M}}_1$ and ${\mathcal{M}}_2$ be submanifolds of a manifold ${\mathcal{M}}$. The manifolds ${\mathcal{M}}_1$ and ${\mathcal{M}}_2$ intersect transversally at a point $m\in{\mathcal{M}}_1\cap {\mathcal{M}}_2$ iff 
 $$T_m{\mathcal{M}} =  T_m{\mathcal{M}}_1+T_m{\mathcal{M}}_2$$
 holds, where $T_m\mathcal{M}$ denotes the tangent space of the manifold $\mathcal{M}$ and similarly for $\mathcal{M}_1$ and $\mathcal{M}_2$.
\end{definition}

The persistence theorem is specialized for the dynamical system that has the {\it fast-slow} structure, 
\begin{equation} \label{eq:fast-slow}
 \left\{
 \begin{aligned}
  \dot{x}&=f(x,y,\epsilon),\\
  \epsilon\dot{y}&=g(x,y,\epsilon),
 \end{aligned}\right. \quad \text{where $\epsilon \in (-\epsilon_0,\epsilon_0), \epsilon_0>0$ small, $x\in \mathbb{R}^\ell$, $y\in \mathbb{R}^m$, $\ell+m=d$.}
\end{equation}
We say $x$ is a slow variable and $y$ is a fast variable. We assume $f$ and $g$ have enough smoothness in their domains of definitions. The meanings of the term are indicated by the following two limiting problems,
%     \hspace{3em} {Reduced Problem} \hspace{7em} Layer Problem $(\cdot)' = \frac{d}{d(t/\epsilon)} = \frac{d}{d\tilde{t}}.$
\begin{equation*} %\label{eq:reduced}
 \text{(Reduced Problem)}\quad\left\{
 \begin{aligned}
    \dot{x} &= f(x,y,0),\\
    0&= g(x,y,0),
 \end{aligned}\right. 
 \hspace{2.5em}
 \text{(Layer Problem)} \quad  
 \left\{
 \begin{aligned}
    x'&= 0,\\
    y'&= g(x,y,0), \quad (\cdot)' = \frac{d}{d(t/\epsilon)}.% = \frac{d}{d\tilde{t}}.
 \end{aligned}\right. 
\end{equation*}
% where the former describes the dynamics under the limiting assumption the fast variables $y$ have arrived equilibrium and $x$ evolves slowly, and the latter describes that of the fast variables $y$ relaxing towards equilibrium manifold keeping the slow variables $x$ unchanged. Formally the latter describes the initial layer behavior as its name indicates.

The zeroset $\mathcal{S}$ of $g(x,y,0)$ defines a manifold where the orbits of the Reduced problem are restricted. On the other hand, this manifold consists of equilibria of the Layer problem. We consider 
\begin{align*}
 \mathcal{S}&\subset \Big\{ (x,y)\:\Big|\: g(x,y,0)=0\Big\},\\
 \mathcal{S}_R&\subset \Big\{ (x,y)\in \mathcal{S} \:\Big|\: \text{$D_y g(x,y,0)$ has the full rank $m$}\Big\} \quad \text{open},\\
 \mathcal{S}_H&\subset \Big\{ (x,y)\in \mathcal{S}_R \:\Big|\: \text{all eigenvalues of $D_y g(x,y,0)$ have nontrivial real parts}\Big\}\quad \text{open}. 
\end{align*}
On $\mathcal{S}_R$, the equation $0=g(x,y,0)$ is locally solvable and we speak of the reduced vector field $X_R$ on slow variables. (See equation (7.8) in \cite{fenichel_geometric_1979}.) %A compact  $K \subset \mathcal{S}_H$ is normally hyperbolic to the Layer problem.

Next, we use the Fenichel's Theorems \cite[Theorem 12.2]{fenichel_geometric_1979} and \cite[Theorem 3]{fenichel_asymptotic_1977}.% and \cite[Theorem 2.2]{Szmolyan}. 
We omit the statements but it states the persistence properties of the compact branch $K\subset\mathcal{S}_H$ and its stable and unstable manifolds under the small perturbation. The upshot of the theorem is that any $\mathcal{N}\hookrightarrow K$ in the reduced phase space that is normally hyperbolic invariant under the reduced vector field $X_R$ persists under the perturbation in a suitable sense. The stable and unstable manifolds of $\mathcal{N}$ has a local lifting to the unreduced phase space and persist under the perturbation as well. In particular, as far as the perturbations of $\mathcal{N}$ and its unstable manifold are concerned, it is enough to have $\mathcal{N}$ overflowing invariant with its center-unstable bundle as in Definition \ref{def:over}.

Ingredients acquired from the Fenichel's Theorem enable one to compose a variety of geometric arguments, such as a transversal intersection, to achieve solutions. The one we use in this paper belongs to one of the simplest setting \cite[Theorem 3.1]{Sz1991}: We are going to take a simply connected branch of  $\mathcal{S}_H$ and its compact subset $K$. We pick $\mathcal{N}_0$ and $\mathcal{N}_1$ in $K$. The heteroclinic orbit is achieved by the transversal intersection of the unstable manifold of $\mathcal{N}_0$ and the stable manifold of $\mathcal{N}_1$ in $K$. This transversal intersection in $K$ then lifts to that in the unreduced phase space. (See \cite{Sz1991}.) %, when $\mathcal{N}_0$ and $\mathcal{N}_1$ are taken from the same branch $K$, the transversality in the reduced phase space constitutes the same in the unreduced phase space. 

\subsection{Singular orbits for the inviscid system with $n=0$}

Let us instantiate the perturbation theory for \eqref{eq:slow}. We take two normally hyperbolic manifolds %of the Reduced problem \eqref{eq:slow02}
$\mathcal{N}_0$ and $\mathcal{N}_1$, which are simply the equilibrium points $M_0$ and $M_1$. The goal of this section is to establish the transversal intersection of the $\mathcal{N}_0^u$, the unstable manifold of $\mathcal{N}_0$, and $\mathcal{N}^s_1$, the stable manifold of $\mathcal{N}_1$, in the $pqrs$-space. Unless otherwise explicitly mentioned, quantities without superscript $n$ are understood as ones for $n=0$. % As explained in \cite{Sz1991}, when $\mathcal{N}_0$ and $\mathcal{N}_1$ are taken from the same branch, the transversality in the reduced phase space constitutes the same in the original space.

We begin by writing the Reduced problem and the Layer problem for \eqref{eq:slow}. 
\begin{equation}\label{eq:slow0} \tag{R}
 \begin{aligned}
%  r &={r}(p,q,s,n=0) \triangleq \frac{ \frac{\alpha-m}{\lambda(1+\alpha)}a - q }{  \frac{\alpha-m}{\lambda(1+\alpha)} + \lambda p + \frac{\alpha}{\lambda}\big(s- \frac{1+m}{1+\alpha}\big)},\\% \quad \text{$={r}(0)$ for simplicity },\\
 \dot{p} &=p\Big(\frac{1}{\lambda}({r}-a) + 2- \lambda p {r} -q\Big),\\% & &\bigg(= p\Big(\frac{D}{\lambda(1+\alpha)}({r}-a_0) + \frac{\alpha}{\lambda}{r}\big(s- \frac{1+m}{1+\alpha}\big) \Big)\bigg),\\
 \dot{q} &=q\Big(1 -\lambda p {r} -q\Big) + b p {r},\\% & &\bigg(=q\Big(\frac{\alpha-m}{\lambda(1+\alpha)}({r}-r_1) + \frac{\alpha}{\lambda}{r}\big(s- \frac{1+m}{1+\alpha}\big) \Big) + b p {r}\bigg),\\
 0&=r\Big(\frac{\alpha-m}{\lambda(1+\alpha)}(r-a) + \lambda pr + q +\frac{\alpha}{\lambda}r\big(s- \frac{1+m}{1+\alpha}\big)\Big),\\
 \dot{s} &=s\Big(\frac{\alpha-m}{\lambda(1+\alpha)}({r}-a) + \lambda p{r} + q - \frac{1}{\lambda}{r}\big(s- \frac{1+m}{1+\alpha}\big)\Big),% & &\bigg(= -\frac{1+\alpha}{\lambda}{r}s\big(s- \frac{1+m}{1+\alpha}\big)\bigg).
 \end{aligned}
\end{equation}
is the Reduced problem and
\begin{equation} \label{eq:fast0} 
 \begin{aligned}
 {p}' &=0, \quad {q}' =0, \quad {r}' =r\Big(\frac{\alpha-m}{\lambda(1+\alpha)}(r-a) + \lambda pr + q +\frac{\alpha}{\lambda}r\big(s- \frac{1+m}{1+\alpha}\big)\Big), \quad{y}' =0,
 \end{aligned}
\end{equation}
is the Layer problem, where $(\cdot)'= \frac{d}{d\tilde{\eta}} = \frac{d}{d(\eta/n)}$ with the fast independent variable $\tilde{\eta}$. The zeroset of the %right-hand-side of the $\eqref{eq:slow0}_3$ 
$$g(p,q,r,s)\triangleq r\Big(\frac{\alpha-m}{\lambda(1+\alpha)}(r-a) + \lambda pr + q +\frac{\alpha}{\lambda}r\big(s- \frac{1+m}{1+\alpha}\big)\Big)$$
entirely consists of the equilibria of \eqref{eq:fast0}. We simply take one branch $\mathcal{S}_H$ and $K\subset\mathcal{S}_H$ compact, which will be called a {\it critical manifold}. We choose $K$ to be a graph $\big(D,\hat{r}(D)\big)$, where
\begin{equation*}
\hat{r}(p,q,s) = \frac{ \frac{\alpha-m}{\lambda(1+\alpha)}a - q }{  \frac{\alpha-m}{\lambda(1+\alpha)} + \lambda p + \frac{\alpha}{\lambda}\big(s- \frac{1+m}{1+\alpha}\big)}
\end{equation*}
or implicitly
\begin{equation}
\frac{\alpha-m}{\lambda(1+\alpha)}(\hat{r}-a) + \lambda p\hat{r} + q +\frac{\alpha}{\lambda}\hat{r}\big(s- \frac{1+m}{1+\alpha}\big)=0 \label{eq:implicit}
\end{equation}
and $D$ is a trapezoid in $pqs$-space
\begin{align*}
 D &\triangleq \left\{ \: (p,q,s) \: \Big| \:  p\ge-\epsilon, ~~ |q|\le2, ~~ \left|s-\frac{1+m}{1+\alpha}\right| \le \frac{1}{2}\min\left\{\frac{\alpha-m}{\alpha(1+\alpha)},\frac{1+m}{(1+\alpha)}\right\},\right.  \\
 &\left. \hat{r}(p,q,s)\ge \frac{1}{2}\min\{1,r_1\}\right\}.
\end{align*}
% \begin{align*}
%  K &\triangleq \left\{ \: (p,q,r,s) \: \big| \:  p\ge-\epsilon, ~~ |q|\le2, ~~ \left|s-\frac{1+m}{1+\alpha}\right| \le \frac{1}{2}\min\left\{\frac{\alpha-m}{\alpha(1+\alpha)},\frac{1+m}{(1+\alpha)}\right\}, \right. \\
%  &\left. \hat{r}(p,q,s)\ge \frac{1}{2}\min\{1,r_1\},~~ r=\hat{r}(p,q,s), ~~\hat{r}(p,q,s) = \frac{ \frac{\alpha-m}{\lambda(1+\alpha)}a - q }{  \frac{\alpha-m}{\lambda(1+\alpha)} + \lambda p + \frac{\alpha}{\lambda}\big(s- \frac{1+m}{1+\alpha}\big)}\: \right\}.
% \end{align*}
Now $K\triangleq\big(D,\hat{r}(D)\big)$. In particular, $K$ is chosen so that $M_0$ and $M_1$ are on $K$; $s$ and $r=\hat{r}(p,q,s)$ have positive lower bound on $K$. See the trapezoid $D$ in Figure \ref{fig:D}. $\epsilon$ will be taken sufficiently small later.


We notice that the implicit formula \eqref{eq:implicit} defines affine level sets of $\hat{r}$. In Figure \ref{fig:affine} are the level sets $\hat{r}(p,q,s)=R$ in the $pqs$-space, $0\le R\le a$. When $R=a$, it passes the origin, which is the equilibrium point $M_0$. As $R$ decreases the affine level sets sweep out the positive $p,q$ sector. 
\begin{figure}
 \centering
  \psfrag{p}{\scriptsize \hskip -2pt $p$}%=\frac{\gamma}{\sigma}$}
  \psfrag{q}{\scriptsize~~~$q$}%=n\frac{v}{\sigma}$}
  \psfrag{s}{\scriptsize $s-\frac{1+m}{1+\alpha}$}%=n\frac{v}{\sigma}$}
  \psfrag{x0}{\scriptsize \hskip -4pt$M_0$}
  \psfrag{x1}{\scriptsize $M_1$}
  \psfrag{G}{\scriptsize \hskip -85pt $\hat{r}(p,q,s)= \frac{1}{2}\min\{1,r_1\}$}
%   \psfrag{R1}{\scriptsize $\hat{r}(p,q,s)=a$}
%   \psfrag{R2}{\scriptsize $\hat{r}(p,q,s)=R_1$}
%   \psfrag{R3}{\scriptsize $\hat{r}(p,q,s)=R_2$}
  \includegraphics[width=5cm]{trapezoid.eps}
%   }
  \caption{The trapezoid $D$, the domain of the graph.} \label{fig:D}
\end{figure}
\begin{figure}
 \centering
  \psfrag{p}{\scriptsize $p$}%=\frac{\gamma}{\sigma}$}
  \psfrag{q}{\scriptsize~~~$q$}%=n\frac{v}{\sigma}$}
  \psfrag{s}{\scriptsize $s-\frac{1+m}{1+\alpha}$}%=n\frac{v}{\sigma}$}
  \psfrag{x0}{\scriptsize $M_0$}
  \psfrag{x1}{\scriptsize $M_1$}
  \psfrag{R1}{\scriptsize $\hat{r}(p,q,s)=a$}
  \psfrag{R2}{\scriptsize $\hat{r}(p,q,s)=R_1$}
  \psfrag{R3}{\scriptsize $\hat{r}(p,q,s)=R_2$}
%
%   \subfigure[Domain $G$ of a Graph]{
%   \includegraphics[width=5cm]{trapezoid.eps}\label{fig:flow0b}
%   }
%   \quad\quad
%   \subfigure[Affine level sets $\hat{r}(p,q,s)=R$ in $pqs$-space]{
  \includegraphics[width=6cm]{Affine.eps}
%   }
  \caption{Affine level sets $\hat{r}(p,q,s)=R$, $0\le R\le a$  in $pqs$-space} \label{fig:affine}
\end{figure}

Now, we verify $K\subset \mathcal{S}_H$. %the {\it Normally Hyperbolicity} of $K$. % upon the definition in \cite[p. 255]{Sz1991}.
% \begin{definition}[fast-slow case definition of normally hyperbolicity of invariant manifold \cite{K}]
% A compact set $\Lambda$ is called normally hyperbolic if the $m\times m$ matrix $(D_y f)(p)$ of first partial derivatives with respect to the fast variables $y$ has no eigenvalues with zero real part for all $p \in \Lambda$.
% \end{definition}
\begin{proposition}
$K\subset \mathcal{S}_H$, i.e., the partial jacobian $\frac{\partial g}{\partial r}(p,q,r,s)|_{r=\hat{r}(p,q,s)}$ is a nontrivial real number for all $(p,q,r,s)\in K$. %$K$ is normally hyperbolic with respect to the Layer problem \eqref{eq:fast0}.
\end{proposition}
\begin{proof}
 %Let $g(p,q,r,s) \triangleq r\Big(\frac{\alpha-m}{\lambda(1+\alpha)}(r-a) + \lambda pr + q +\frac{\alpha}{\lambda}r\big(s- \frac{1+m}{1+\alpha}\big)\Big)$, the right hand side of $\eqref{eq:fast0}_3$.  We need to show that $K\subset \mathcal{S}_H$, i.e., the partial jacobian $\frac{\partial g}{\partial r}$ is a nontrivial real number for all $m\in K$.
 \begin{align*}
 \left.\frac{\partial g}{\partial r}\right|_{K} &= \Big(\frac{\alpha-m}{\lambda(1+\alpha)}(\hat{r}-a) + \lambda p\hat{r} + q +\frac{\alpha}{\lambda}\hat{r}\big(s- \frac{1+m}{1+\alpha}\big)\Big) + \hat{r}\Big(\frac{\alpha-m}{\lambda(1+\alpha)} + \lambda p + \frac{\alpha}{\lambda}\big(s- \frac{1+m}{1+\alpha}\big)\Big)\\
 &= \hat{r}\Big(\frac{\alpha-m}{\lambda(1+\alpha)} + \frac{\alpha}{\lambda}\big(s- \frac{1+m}{1+\alpha}\big) + \lambda p\Big)\ge \frac{r_1}{2}\Big(\frac{\alpha-m}{2\lambda(1+\alpha)} - \lambda \epsilon\Big)>0.
 \end{align*}
%  Take $\epsilon$ so that the last inequality holds
\end{proof}

Let us rewrite \eqref{eq:slow0}, of the reduced vector field $X_R$, in an amenable form with the identity \eqref{eq:implicit}:
\begin{equation}\label{eq:slow02}
 \begin{aligned}
%   r &=\hat{r}(p,q,s,n=0) \triangleq \frac{ \frac{\alpha-m}{\lambda(1+\alpha)}a - q }{  \frac{\alpha-m}{\lambda(1+\alpha)} + \lambda p + \frac{\alpha}{\lambda}\big(s- \frac{1+m}{1+\alpha}\big)},\\% \quad \text{$=\hat{r}(0)$ for simplicity },\\
 \dot{p} &= p\Big(\frac{D}{\lambda(1+\alpha)}(\hat{r}-a_0) + \frac{\alpha}{\lambda}\hat{r}\big(s- \frac{1+m}{1+\alpha}\big) \Big),\\
 \dot{q} &=q\Big(1 -\lambda p \hat{r} -q\Big) + b p \hat{r},\\% & &\bigg(=q\Big(\frac{\alpha-m}{\lambda(1+\alpha)}(\hat{r}-r_1) + \frac{\alpha}{\lambda}\hat{r}\big(s- \frac{1+m}{1+\alpha}\big) \Big) + b p \hat{r}\bigg),\\
%  0&=r\Big(\frac{\alpha-m}{\lambda(1+\alpha)}(r-a) + \lambda pr + q +\frac{\alpha}{\lambda}r\big(s- \frac{1+m}{1+\alpha}\big)\Big),\\
 \dot{s} &= -\frac{1+\alpha}{\lambda}\hat{r}s\big(s- \frac{1+m}{1+\alpha}\big).
 \end{aligned}
\end{equation}

Having set forth the critical manifold $K$ and the reduced vector field $X_R$, the theorem of Fenichel holds in $K$. %Recall our task to make sure $\mathcal{N}_0^u$, the unstable manifold of $M_0$, intersects transversally $\mathcal{N}^s_1$, the stable manifold of $M_1$, in the $pqrs$-space. 
Let $W_0^u$ denotes the unstable manifold of $M_0$ of the reduced problem in $K$ and $W_1^s$ denotes similarly the stable manifold of $M_1$. As explained in \cite{Sz1991}, because $M_0$ and $M_1$ are taken from the same branch $K$, it reduces now to make sure $W_0^u$ intersects transversally $W_1^s$ in $K$.

Before we discuss further, we turn to the reduced linear stability of $M_0$ and $M_1$ in the $pqs$-space. We find that the expression in Section \ref{sec:equil} for the third eigenvalue $\mu_{03}$ and its eigenvector $X_{03}$ of $M_0$ ($\mu_{13}$ and $X_{13}$ respectively of $M_1$) does not make senses for $n=0$; the orbits are restricted on $K$. % only reduced three dimensions are available. 
For the case $n=0$, the direction of $X_{03}$ ($X_{13}$ respectively) is dropped out from a subject of discussion. Formulas in Section \ref{sec:equil} for the rest corresponding to the reduced space are valid for $n=0$. In the complementary three dimensional tangent space, we find that $M_1$ is a stable node and $M_0$ is a saddle that has two positive and one negative eigenvalues. Therefore, $W_0^u$ will be the two dimensional unstable manifold of $M_0$ and $W_1^s$ the three dimensional stable manifold of $M_1$.% Now we take closer look. We find that the unstable manifold $W^u(M_0)$ is the plane $s\equiv \frac{1+m}{1+\alpha}$ and extends globally. Seen from $\eqref{eq:slow0}_4$, this is a plane of attraction from the $s$-direction, which explains the remaining stable direction at $M_0$ as well. The two dimensional plane $s\equiv \frac{1+m}{1+\alpha}$ and its normal direction of attraction at last explains the total three dimensions of the stable manifold $W^s(M_1)$. In conclusion, $s\equiv\frac{1+m}{1+\alpha}$ forms a family of heteroclinic orbits joining $M_0$ to $M_1$ and in its normal $s$-direction flow is relaxing onto it in the course of attraction to the stable node $M_1$. Figure \ref{fig:flow0} is illustrating these features.

With those in mind, we find that for the inviscid problem with $n=0$, all the interesting flow occurs on the two-dimensional plane $s\equiv \frac{1+m}{1+\alpha}$ that is invariant. \eqref{eq:slow02} then decouples,
\begin{equation}\label{eq:slow03}
 \begin{aligned}
 \dot{p} &= p\Big(\frac{D}{\lambda(1+\alpha)}(\hat{r}-a_0)\Big),\\
 \dot{q} &= q\Big(1 -\lambda p \hat{r} -q\Big) + b p \hat{r}, 
 \end{aligned}
\end{equation}
where $\hat{r}=\hat{r}\big(p,q,\frac{1+m}{1+\alpha}\big)$. This two dimensional flow on the projected $pq$-plane around $M_0$ defines $W_0^u$ that is now contained in the plane $s\equiv \frac{1+m}{1+\alpha}$.
% This plane is invariant and tangent to the $W_0^u$ at $M_0$, and invariant is contained in the plane $s\equiv \frac{1+m}{1+\alpha}$  
By following Lemma, we completely characterize the flow on a triangle $T$ in the first quadrant of $pq$-plane. 
\begin{lemma} \label{lem:T}
 Let $T$ be the closed triangle on $s\equiv \frac{1+m}{1+\alpha}$ enclosed by $p=0$, $q=0$, and the level set $\hat{r}(p,q,\frac{1+m}{1+\alpha})= \frac{1}{2}\min\{1,r_1\}$ that is intersected by $s\equiv \frac{1+m}{1+\alpha}$. 
 Then $T\setminus M_0 \in W^s(M_1)$.
\end{lemma}
\begin{proof}
$T$ is a two dimensional compact positively invariant set: (1) on $p=0$, $\nu = (1,0)$ and $X_R\cdot\nu = \dot{p}=0$, where $X_R$ stands for the reduced vector field of \eqref{eq:slow02};
 (2) on $q=0$, $\nu = (0,1)$ and $X_R\cdot\nu=\dot{q} = bp\hat{r}\ge0$; lastly on the hypotenuse, let $\underbar{r}=\frac{1}{2}\min\{1,r_1\}$. The inward normal pointing the origin is $\nu = (-\lambda\underbar{r}, -1)$. We compute
  \begin{align}
  X_R\cdot\nu=-\lambda\underbar{r}\dot{p} -\dot{q}&= -\lambda \underbar{r}p \Big(1-\lambda \underbar{r}p -q + \frac{1}{\lambda}(\underbar{r}-a)+1\Big) - q(1-\lambda \underbar{r}p -q\big) - b \underbar{r}p \nonumber\\
  &= (1-\lambda \underbar{r}p -q)(-\lambda \underbar{r}p -q) -\underbar{r}p\Big((\underbar{r}-a)+\lambda+b\Big)\nonumber\\
  &= \left(\frac{\alpha-m}{\lambda(1+\alpha)}\right)^2(\underbar{r}-r_0)(\underbar{r}-r_1)+\underbar{r}p(1-\underbar{r})\ge \delta>0. \label{eq:affine}
 \end{align}

Having that shown, let $\Omega$ be an $\omega$-limit set of the orbit portrait from $x_0\in T\setminus M_0$. It is non-empty because $T$ is compact. It cannot contain $M_0$ because $M_0$ does not have a stable subspace. It cannot contain a periodic orbit because if so, there would be a fixed point in the interior of $T$ that is not true. It cannot contain a separatrix cycle because $T$ has only two fixed points $M_0$ and $M_1$ and again $M_0$ does not have a stable subspace.  By Poincar\'e-Bendixson Theorem, the $\omega$-limit set is $M_1$.
\end{proof}
In particular, let $\mathcal{F}^u_{M_0}\subset W^u_0$ be the most rapidly escaping orbit from $M_0$ satisfying \eqref{eq:rapid} that is characterizable by the Unstable manifold theorem for the hyperbolic fixed point. By Lemma \ref{lem:T}, $\mathcal{F}^u_{M_0}$ survives to end up arriving at $M_1$ and this gives the proof for $n=0$ of Theorem \ref{thm1}.
\begin{remark}[Neither of $T$ and $\chi(\eta)$ is normally hyperbolic]
Under the reduced vector field, neither $T$ nor $\chi(\eta)$ falls into the one in Definition \ref{def:over} or \ref{def:nhim}. Rate numbers are determined by the eigenvalues of $M_1$ that depends on the parameters and the violation of the assumptions does happen at $M_1$ in a valid range of parameters. If either of them were normally hyperbolic, then the heteroclinic $\chi^n(\eta)$ for $n>0$ small would have been obtained immediately by the theorem of Fenichel. Instead of working with those manifolds that lack persistences, we comprise $\chi(\eta)$ as a transversal intersection of the persistent manifolds $W_0^u$ and $W_1^s$.  % but in fact $T^n$ is homeomorphically attained by the same transversal intersection arguments. According to the sizes of eigenvalues of $M_1$, $T^n$ may develop the cusp at $M_1$. %This manifests the fact that $T$ is not a normally hyperbolic manifold.
%
%
% the $2$-dimensional unstable manifold $W^u(M_0^n)$ is attained. extended in $K$ is a posteriori attained. The counterpart for $n=0$ is the triangle $T$. We remark that $W^u(M_0(n))$ is merely homeomorphic to $T$ due to the possible cusp at $M_1$ according to the order of eigenvalues of $\mu_{11}$, $\mu_{12}$, and $\mu_{14}$. Despite the smoothness has nothing to do with our proof,  %
\end{remark}

% 
% % \begin{lemma}
% %  Stable manifold $F_R^{-t_0}$ for $t_0<\infty$ large enough contains an orbit point of $\chi(\eta)$ as an internal point.
% % \end{lemma}
% % \begin{proof}
% %  This is clear by Lemma \ref{lem:T}
% % \end{proof}
% 

% In this section, we prove the following proposition.
\begin{proposition} \label{prop:singular}
 Let $\mathcal{N}_0=M_0$, $\mathcal{N}_1=M_1$, $W^u_0=\mathcal{F}^u_{M_0}\subset W^u(M_0)$ the most rapidly escaping orbit from $M_0$ satisfying \eqref{eq:rapid}, $W^s_1=\Phi_{-t_0}(W^s_{loc}(M_1))$, the time $-t_0$ image of the local stable manifold of $M_1$ for large enough $t_0<\infty$. Then $\mathcal{F}^u_{M_0}$ intersects $W^s_1$ transversally in $pqs$-space.
\end{proposition}
\begin{proof}[Proof of Proposition \ref{prop:singular}]
% Let $\mathcal{F}^u_{M_0}\subset W^u(M_0)$ be the most rapidly escaping orbit from $M_0$ satisfying \eqref{eq:rapid} that is characterizable by the Unstable manifold theorem for the hyperbolic fixed point. By Lemma \ref{lem:T}, $\mathcal{F}^u_{M_0}$ survives to end up arriving at $M_1$. This proves the existence of the heteroclinic orbit for $n=0$.
%
% Now let 
 Let $W^s_{loc}(M_1)$ be the local three dimensional stable manifold of $M_1$ given by the Stable manifold theorem. Let $W^s_1=\Phi_{-t_0}(W^s_{loc}(M_1))$ be the time $-t_0$ image of $W^s_{loc}(M_1)$. For large enough $t_0<\infty$, by Lemma \ref{lem:T} the orbit point $m\in \mathcal{F}^u_{M_0}$ must be attained in $W^s_1$ as an interior point. Therefore the tangent space ${T}_m W^s_1$ is the whole of ${T}_m \mathbb{R}^3$. Then the intersection with $\mathcal{F}^u_{M_0}$ is trivially transversal.
\end{proof}

\subsection{Persistence for $n>0$} \label{sec:thmproof}

\begin{lemma} \label{lem:rapid}
 Let $\mathcal{N}_0=M_0$, $\mathcal{F}^u_{M_0}\subset W_0^u$ the most rapidly escaping orbit from $M_0$ satisfying \eqref{eq:rapid}. Then, for sufficiently small $n$, $\mathcal{F}^u_{M_0}$ perturbs in a $C^{r-1}$ manner to $\mathcal{F}^u_{M_0^n}$ the most rapidly escaping orbit from $M_0^n$ satisfying \eqref{eq:rapid}. 
\end{lemma}
\begin{proof}
 We pick an one dimensional orbit portraits in $W_0^1$ that are not most rapid, the line segment $q\in [- \frac{1}{2}, \frac{1}{2}]$ on $q$-axis, which we will denote $\bar{\Lambda}$. We claim that $\bar\Lambda$ is an overflowing invariant manifold as in Definition \ref{def:over} of the reduced problem. More precisely, it satisfies \eqref{eq:A} and \eqref{eq:B} with $r'=r-1$ and $E$ the tangent $pq$-plane. Then the most rapidly escaping orbit exists as an unstable family $\mathcal{F}^u_m$ with $m=M_0$ of the reduced problem. \cite[Theorem 3]{fenichel_asymptotic_1977} gives the persistence of the $C^{r-1}$ family $\mathcal{F}^u_m$ of the unstable manifold of $M_0$ under the perturbation.
 
 From \eqref{eq:slow03}, $\dot{q}=q(1-q)$ on $q$-axis, it is clear that $\bar\Lambda$ is overflowing invariant. Let $E$ be the $pq$-plane along $\bar{\Lambda}$ and $E'$ be the lines parallel to the $s$-axis. Then, $T \mathbb{R}^3|\bar\Lambda$ splits into three one dimensional bundles $T\Lambda\oplus N \oplus E'$.  %defined by the three coordinate basis in $pqs$-space, i.e., $\mathbf{e}_q\in T\Lambda$, $\mathbf{e}_p\in N$, and $\mathbf{e}_s\in E'$. 
 Asymptotic rate numbers are determined at $M_0$ by the eigenvalues of $M_0$. At $M_0$, $E'_{M_0}$ is the stable subspace with eigenvalue $-\mu_{04}$ and $N_{M_0}$ and $T_{M_0}\Lambda$ are the unstable ones with $\mu_{01}=2$ and $\mu_{02}=1$ respectively. From these, we compute 
 $$ \nu^s = e^{-\mu_{04}}, \quad\sigma^s = 0, \quad\alpha^u = e^{-2}, \quad\rho^u=e^{-1}, \quad\tau^u=0.$$
\end{proof}

Now Theorem \ref{thm1} follows exactly same way as in \cite[Theorem 3.1]{Sz1991} by the transversal intersection.

\begin{proof}[Proof of Theorem \ref{thm1}] 
%  By \cite[Theorem 3.1]{Sz1991}, In particular, $\mathcal{F}^u_{M_0}$ perturbs to $\mathcal{F}^u_{M_0^n}$, the most rapidly escaping orbit from $M_0^n$ by Lemma \ref{lem:rapid}
 By the theorem of Fenichel, $n_0$ can be taken sufficiently small so that for given $(\lambda,\alpha,m)$ if $n \in [0, n_0)$ then $(\lambda,\alpha,m,n) \in \Lambda$ and  the system $(*)^{\lambda,\alpha,m,n}$ admits a transversal heteroclinic orbit joining equilibrium $M_0^{n}$ to equilibrium $M_1^{n}$: $\mathcal{F}^u_{M_0}$ perturbs to $\mathcal{F}^u_{M_0^n}$ by Lemma \ref{lem:rapid} and $W_1^s$ perturbs to $W_1^{s,n}$ and the transversal intersection is stable under the perturbation. 
\end{proof}
\begin{figure}
 \centering
  \psfrag{X01}{\scriptsize$X_{02}$}
  \psfrag{X02}{\scriptsize$X_{01}$}
%   \psfrag{A}{$A$}
%   \psfrag{T}{$T$}
%   \psfrag{x0}{\scriptsize $M_0$}
%   \psfrag{x1}{\scriptsize $M_1$}
%   \psfrag{r0}{\scriptsize $\hat{r}(p,q,s,0)=r_0$}
%   \psfrag{r1}{\scriptsize $\hat{r}(p,q,s,0)=r_1$}
  \psfrag{p}{\scriptsize $p$}%=\frac{\gamma}{\sigma}$}
  \psfrag{q}{\scriptsize~~~$q$}%=n\frac{v}{\sigma}$}
  \psfrag{s}{\scriptsize $s-\frac{1+m}{1+\alpha}$}%=n\frac{v}{\sigma}$}
%   \subfigure[Affine level sets $\hat{r}(p,q,s)=R$ in $pqs$-space]{
%   \psfrag{r}{\scriptsize$r$}%=\big(\sigma\gamma^{(1-n)}\big)^{\frac{1}{n}}$}
  \psfrag{CC}{\scriptsize\hskip 5pt$\frac{\alpha-m}{\lambda(1+\alpha)}\big(-\frac{1}{\lambda},a\big)$}
  \includegraphics[width=8cm]{geom0.eps} \label{fig:flow0}
%   \flushleft
%   }
%   \quad \quad
%   \subfigure[Flow in $pqs$-space for $n=0$]{
%   \psfrag{r}{\scriptsize$s-\frac{1+m}{1+\alpha}$}%=\big(\sigma\gamma^{(1-n)}\big)^{\frac{1}{n}}$}
%   \includegraphics[width=6cm]{flow0pqs.eps}\label{fig:flow0b}
%   }
  \caption{The schematic sketch of the flow on the invariant plane $s=\frac{1+m}{1+\alpha}$.   On $s=\frac{1+m}{1+\alpha}$, $M_0$ is an unstable node and $M_1$ is a stable node. Two directions of unstable subspaces of $M_0$ are denoted by $X_{01}$ and $X_{02}$; straight lines emanated from the point $\frac{\alpha-m}{\lambda(1+\alpha)}\big(-\frac{1}{\lambda},a\big)$ are the level lines of $\hat{r}$ intersected with the plane $s=\frac{1+m}{1+\alpha}$; the curve in the fourth quadrant is the nullcline of the equation $\eqref{eq:slow02}_2$; the triangle $T$ is a 2-dimensional positively invariant set; the trapezoid $A$ is a 2-dimensional negatively invariant set; $\omega$-limit set of any point in $T$ is $M_1$; $\alpha$-limit set of any point in $A$ is $M_0$; in particular there is a heteroclinic orbit (green one) that is the most rapidly escaping orbit from $M_0$ through $X_{01}$. } 
\end{figure}


% \subsection{Existence of the heteroclinic orbit} \label{sec:proof_proof}
%

%
% \smallskip
% \noindent
% \begin{proof}
% %\mbox{}\\*\indent
% \noindent{\bf Reduction to the $(p,q)$-system.}
% \medskip
% For $(\lambda,\alpha)$, by the Theorem of Geometric singular perturbation theory, there exists $n_1$ such that if $n\in(0,n_1)$, the locally invariant manifold $G(\lambda,\alpha,n)$ with respect to the $(p,q,r)$-system with parameter $(\lambda,\alpha,n)$ exists. Moreover, the graph is again given by the graph $r=h(p,q;\lambda,\alpha,n)$, where the function has the definition on $D$ again. The function is, jointly with $n$, smooth in $(p,q,n) \in D\times(0,n_1)$. Taking smaller $n_1$ if necessary can achieve the inequality \eqref{eq:r1posineq}. We can take $n_1$ even smaller to ensure $h(p,q,n)>0$ in $D\times(0,n_1)$.
%
%
% Then, on the graph, $\big(p(\eta),q(\eta)\big)$ satisfies the planar system
% \begin{equation} \tag*{(${R}$)} \label{eq:reduced}
% \begin{aligned}
%  {\dpp}&=p\bigg\{\Big[\frac{1+\alpha}{1+n}\,\frac{1}{\lambda }\Big(h^{1+n}-c_0\Big)\Big] -\Big[d_1 + q + \lambda ph\Big]\bigg\},\\
%  {\dqq}&=q\Big(1-q-\lambda p h\Big) + bph,\\
% \end{aligned}
% \end{equation}
% where $h$ is an abbreviation for $h(p,q;\lambda,\alpha,n)$.
%
% \medskip
% \noindent{\bf Claim 1.}
% $M_0(\lambda,\alpha,n)$ and $M_1(\lambda,\alpha,n)$ are on the invariant manifold.
% \medskip
%
% The part of the geometric singular perturbation theory ensures that the hyperbolic equilibrium points that were on the critical manifold persists for $n>0$ and are on the subjected locally invariant manifold, dictating that the two equilibrium points $M_0(\lambda,\alpha,n)$ and $M_1(\lambda,m,n)$ are on the graph $r=h(p,q;\lambda,\alpha,n)$; the only exceptional case is when $\alpha=2n+1$, where $M_1$ fails to be hyperbolic. Nevertheless, $M_1$ is on the graph for the following reason. It is obvious from \eqref{eq:reduced} that $(\dot{p},\dot{q})=(0,0)$ at $(p,q)=(0,1)$. $\dot{r} = \frac{\partial h}{\partial p} \dot{p} + \frac{\partial h}{\partial q} \dot{q}$ has to be $0$ because the derivatives of $h(\lambda,\alpha,n)$ is close to the derivatives of $h(\lambda,\alpha,n=0)$ and the latters are bounded as seen from the graph formula. Hence the point $(0,1,h(p,q))$ is the equilibrium point and this must be $M_1$.
%
% Next, we analyze the planar system $(R)$. %We focus on two points $(0,0)$ and $(0,1)$, which are the two projections of $M_0$ and $M_1$ and thus are two equilibrium points of $(R)$.
%
% % \medskip
% % \noindent{\bf Claim 1.}
% % $(0,0)$ is an unstable node of $(R)$ and $(0,1)$ is an stable node of $(R)$.
% % \medskip
% %
% % Linear stability of $R$ around the $\tilde{M}_0$ and $\tilde{M}_1$ respectively gives the proof. Linearized system around $\tilde{M}_0$ with perturbations $P$ and $Q$ is
% % \begin{align*}
% %  \begin{pmatrix} \dot{P}\\ \dot{Q} \end{pmatrix} =
% %  \begin{pmatrix} 2 & 0 \\  br_0 & 1 \end{pmatrix} \begin{pmatrix} {P}\\ {Q} \end{pmatrix}
% % \end{align*}
% % and we see the two positive eigenvalues of the coefficient matrix. Linearized system around $\tilde{M}_1$ with perturbations $P$ and $Q$ is
% % \begin{align*}
% %  \begin{pmatrix} \dot{P}\\ \dot{Q} \end{pmatrix} =
% %  \begin{pmatrix} -\frac{1+n}{\alpha-n} & 0 \\  (b- \lambda)r_1 & -1 \end{pmatrix} \begin{pmatrix} {P}\\ {Q} \end{pmatrix},
% % \end{align*}
% % and we see the two negative eigenvalues of the coefficient matrix. The claim is shown.
% %
% \medskip
% \noindent{\bf Claim 2.}
% The triangle $T$  is positively invariant for the system $(R)$ with $(\lambda,\alpha,n=0)$.
% \medskip
%
% When $n=0$, the explicit formula for the graph is available, which makes $(R)$ even simpler,
% \begin{align*}
%  \dot{p} &= -\frac{D}{\alpha} p\Big(d_1+q+\lambda p h\Big),\\
%  \dot{q} &= q\Big(1-q-\lambda p h\Big) + b p h.
% \end{align*}
%
%
%
% We compute the inward normal component of $(\dot{p},\dot{q})$ on three sides of the $T$. On $p$-axis, the inward normal vector is $(0,1)$ and $(\dot{p},\dot{q})\cdot(0,1)=bpr\ge0$. On $q$-axis, the inward normal vector is $(1,0)$ and $(\dot{p},\dot{q})\cdot(1,0)=0$. Lastly for the hypotenuse, recall that it is the contour line $\underbar{R}=h(p,q,n=0)$, or $q + \lambda p\underbar{R} = \frac{\alpha}{\lambda}(r_0-\underbar{R})$. Define $\underbar{p}$ and $\underbar{q}$ to be the $p$-intercept and $q$-intercept of the contour line, or $\underbar{q} = \lambda \underbar{p}\underbar{R} = \frac{\alpha}{\lambda}(r_0-\underbar{R})$. Then $(-\underbar{q},-\underbar{p})$ is an inward normal vector on the contour line. We compute
% \begin{align*}
%     (\dot{p},\dot{q}) \cdot (-\underbar{q},-\underbar{p}) &= \frac{D}{\alpha}\underbar{q}p\Big(d_1 + q + \lambda p \underbar{R}\Big) -\underbar{p}\bigg\{q\Big(1-q-\lambda p \underbar{R}\Big) + b p \underbar{R}\bigg\}\\
%     &=\frac{D}{\alpha}\underbar{q}p\Big(d_1 +\underbar{q}\Big) -\underbar{p}\bigg\{\Big(\underbar{q}-\lambda p \underbar{R}\Big)\Big(1-\underbar{q}\Big) + b p \underbar{R}\bigg\}\\
%     &=-\underbar{p}\underbar{q}(1-\underbar{q}) + \underbar{q}p\Big(\frac{D}{\alpha}d_1 + \frac{D}{\alpha}\underbar{q} + (1-\underbar{q}) - \frac{b}{\lambda}\Big)\\
%     &=-\underbar{p}\underbar{q}(1-\underbar{q}) + \underbar{q}p\frac{1+\alpha}{\lambda}\Big(\frac{1}{1+\alpha}-\underbar{R}\Big)\\
%     &\ge -\underbar{p}\underbar{q}(1-\underbar{q}) \quad \text{since $\underbar{R} < \frac{1}{1+\alpha}$}\\
%     &\ge \delta_0 > 0.
% \end{align*}
% Thus the claim 2 is shown. In particular $\delta_0$ can be taken regardless of $n$.
%
% \medskip
% \noindent{\bf Claim 3.}
% The triangle $T$  is positively invariant for the system $(R)$ with $(\lambda,\alpha,n)$ provided $n$ is sufficiently small.
% \medskip
%
% Same reasonings apply on the $p$-axis and the $q$-axis, so it is sufficient to show that the flow is inward on the hypotenuse. Now the hypotenuse is not anymore a contour line of the function $h(p,q;\lambda,\alpha,n)$. We arrange terms of right-hand-sides of $(R)$ in the form
% \begin{align*}
%  {\dpp}&=p\bigg\{\Big[\frac{1+\alpha}{\lambda }\Big(\underbar{R}-c_0\Big)\Big] -\Big[d_1 + q + \lambda p\underbar{R}\Big]\bigg\} + \underbrace{p\bigg\{\Big[\frac{1+\alpha}{1+n}\,\frac{1}{\lambda }\Big(h^{1+n}-c_0\Big)\Big]-\Big[\frac{1+\alpha}{\lambda }\Big(\underbar{R}-c_0\Big)\Big] -\lambda p(h-\underbar{R})\Big]\bigg\}}_\text{$\triangleq g_1(p,q,n)$},\\
%  {\dqq}&=q\Big(1-q-\lambda p \underbar{R}\Big) + bp\underbar{R} + \underbrace{(-q\lambda p+b) (h-\underbar{R})}_\text{$\triangleq g_2(p,q,n)$}.
% \end{align*}
% Since $h$ is a smooth function of $n$ and $D$ is compact, provided $n$ is sufficiently small, we have an estimate
% \begin{equation}
%  |g_1(p,q,n)| + |g_2(p,q,n)| \le C_0 n, \quad \text{where $C_0$ does not depend on $p$, $q$, and $n$.}
% \end{equation}
% Therefore
% $$ (\dot{p},\dot{q}) \cdot(-\underbar{q},-\underbar{p}) \ge \delta_0 + C_0'n \quad \text{for another uniform constant $C_0'$}.$$
% Taking $n$ sufficiently small, say $n<n_0<n_1$, so that the last expression is positive gives the proof of the claim.
%
% \medskip
% Now, consider the orbit $\chi(\eta)$ emanates from $M_0$ that is on the graph $G(\lambda,\alpha,n)$ and in the same time through the direction $X_{02}$ (towards the first octant). This orbit segment exists due to the Stable(unstable) manifold theorem of the hyperbolic equilibrium point and it survives unless it leaves the locally invariant manifold $G(\lambda,\alpha,n)$ through the boundary.
%
% While the orbit is on the manifold, its projection $\tilde\chi(\eta)$ on $(p,q)$-plane satisfies \eqref{eq:reduced}. By Claim 2 and 3, the orbit $\tilde\chi(\eta)$ cannot escape the triangle $T$ and orbit exists for $\eta \in (-\infty,\infty)$ because the right-hand-sides of \eqref{eq:reduced} are bounded in $T$.
%
% Next, consider the $\omega$-limit set of the $\tilde\chi(\eta)$. It cannot contain the limit cycle since there is no equilibrium point interior of $T$. Then, the Poincar\'e-Bendixson Theorem implies that its $\omega$-limit point consists of equlibrium points. $(0,0)$ cannot be the one because it is unstable node, concluding that $\tilde\chi(\eta) \rightarrow (0,1)$ as $\eta \rightarrow \infty$. The lifting of the orbit in the three dimensional phase space by $r=h(p,q;\lambda,\alpha,n)$ is the desired heteroclinic orbit.
% \end{proof}

\section{Emergence of the localization}
% The dynamical system has a translation invariance, i.e., if $\chi(\eta)$ is a heteroclinic orbit, then so is a $\chi(\eta-\eta_0)$ for any $\eta_0\in \mathbb{R}$, implying that we have constructed in fact infinitely many heteroclinics. It turns out that all of them are relevant.  Besides of $\eta_0$, we have been using a few other parameters and this points to that we have constructed a family of solutions with a certain degrees of freedom. We precise relationships between parameters and the total number of independent parameters because not all of parameters are independent.
%
% To begin with, fix any referential heteroclinic orbit $Z(\eta)=\big(P(\eta),Q(\eta),R(\eta)\big)$, which is characterized by the unique constant $\bar{\kappa}$ in the expansion around the $M_0$ such that
% $$ Z(\eta) = \bar{\kappa}e^{2\eta}X_{02} + \text{higer-order terms as $\eta \rightarrow -\infty$.}$$%, with the associated constant $\bar{\kappa}$.}$$
% We set our heteroclinic orbit $\chi(\eta) = Z(\eta-\eta_0)$ and show how to fix $\eta_0$. From the proposition \ref{prop1},
% $$ \kappa = \Big(\frac{U(0)}{\Phi(0)}\Big)^{-n}\Phi(0)^{1+\frac{\alpha-n}{1+\alpha}}=\lim_{\eta \rightarrow -\infty} e^{-2\eta}\chi(\eta)  = \lim_{\eta \rightarrow -\infty} e^{-2\eta} Z(\eta-\eta_0) =\bar{\kappa}e^{-2\eta_0}.$$
% By treating $U(0)$ and $\Theta(0)$ as the primary parameters, $\eta_0$ is determined by
% \begin{equation}
%  \eta_0 = \frac{1}{2} \log\Big(\bar{\kappa} \Big(\frac{U(0)}{\Phi(0)}\Big)^{n}\Phi(0)^{-1-\frac{\alpha-n}{1+\alpha}}  \Big) = \frac{1}{2} \log\Big(\bar{\kappa} U(0)^n \Theta(0)^{-\alpha-\frac{1+\alpha}{1+n}}  \Big). \label{eq:eta0}
% \end{equation}
%
% Next, we check that
% $$ \frac{U(0)^{1+n}}{\Theta(0)^{1+\alpha}} = \lim_{\eta \rightarrow -\infty} r^{1+n} = c_0 = \frac{2}{D} + \frac{2+2n}{D}\lambda.$$
% Again, by treating $U(0)$ and $\Theta(0)$ as the primary parameters, we obtain
% \begin{equation}
%  \lambda = \frac{D}{2+2n}\frac{U(0)^{1+n}}{\Theta(0)^{1+\alpha}} - \frac{2}{2+2n}. \label{eq:lambda}
% \end{equation}
% Now, we find the ratio $\frac{U(0)^{1+n}}{\Theta(0)^{1+\alpha}}$ is restricted by \eqref{eq:r1posineq},
% \begin{equation} \label{eq:restriction}
%  \frac{2}{1+2\alpha-n} < \frac{U(0)^{1+n}}{\Theta(0)^{1+\alpha}} < \frac{2}{1+n}.
% \end{equation}
%
%
%
% The rest of the parameters $\alpha$ and $n$ are the material properties. In conclusion, for each of the material characterized by $\alpha$ and $n\ll1$, we have constructed the two-parameters family of heteroclinic orbits parameterized by $U(0)$ and $\Theta(0)$, where the localizing rate $\lambda$ and the translational factor $\eta_0$ are determined by \eqref{eq:lambda} and \eqref{eq:eta0} respectively, and the valid range of the ratio $\frac{U(0)^{1+n}}{\Theta(0)^{1+\alpha}}$ for the Theorem \ref{thm1} to apply is restricted by \eqref{eq:restriction}.


% \subsection{Asymptotic behavior of the solutions}
By transforming back with \eqref{eq:ORItoCAP}, \eqref{eq:CAPtoBAR}, \eqref{eq:BARtoTIL}, and \eqref{eq:pqrdef}, $\big(\Gamma(\xi),V(\xi),\Theta(\xi),\Sigma(\xi),U(\xi)\big)$ and \\$\big(\gamma(x,t),v(x,t),\theta(x,t),\sigma(x,t),u(x,t)\big)$ are recovered.
We replace $t \leftarrow t+1$ in the final expressions,
\begin{equation*}
\begin{aligned}
 \gamma(t,x) &= (t+1)^a\Gamma((t+1)^\lambda x), & v(t,x) &= (t+1)^b V((t+1)^\lambda x), &\theta(t,x) &= (t+1)^c \Theta((t+1)^\lambda x),\\
 \sigma(t,x) &= (t+1)^d \Sigma((t+1)^\lambda x), & u(t,x) &= (t+1)^{b+\lambda} U((t+1)^\lambda x),
\end{aligned}
\end{equation*}
so that we interpret $\big(\Gamma(\xi),V(\xi),\Theta(\xi),\Sigma(\xi),U(\xi)\big)=\big(\gamma(0,x),v(0,x),\theta(0,x),\sigma(0,x),u(0,x)\big)|_{x=\xi}$,  the initial states that give rise to the localization. For given material parameters $(\alpha, m, n)$, additionally two degrees of freedom fully accounts for the above two-parameters family of solutions. As explained in Section \ref{sec:twoparam}, the choices of $U(0)$ and $\Gamma(0)$ fix one solution and other boundary values and the localizing rate $\lambda$ are dependently decided. The valid ranges of $U(0)$ and $\Gamma(0)$ are such that the ratio $\frac{U(0)}{\Gamma(0)}$ is not too small and not too big, i.e.,
 $$\frac{2(1+\alpha) -n}{D} < \frac{U(0)}{\Gamma(0)} < \frac{2(1+\alpha)}{1+m+n} -\frac{n}{D}\left( \frac{4(1+\alpha)(\alpha-m-n)}{(1+m+n)^2} +1\right).$$
The localizing rate $\lambda$ is then determined by \eqref{eq:lambda} accordingly in the range
$$0< \lambda < \frac{2(\alpha-m-n)}{1+m+n}\left(\frac{1+m}{1+m+n}\right).$$

In the following sections, we examine characteristics of these two-parameters family of initial states and discuss the emergence of localization from them as time proceeds.

$\mathcal{S}$

\subsubsection{Properties of the self-similar solutions}
We state that each of the strain, strain-rate, and temperature has a small bump at the origin out of the asymptotically flat state, whose tip sizes are parameterized by $\Gamma(0)$ and $U(0)$. Accordingly, the velocity is an increasing function of $x$ that has a slightly steeper slope at origin than at the rest of the places; the stress is then the convex increasing function of $|x|$. These initial non-uniformities can be viewed as perturbations of the uniform shearing motion at a certain time.

% Now, we fix the two primary parameters $U(0)$ and $\Theta(0)$ and look into the
%
%
% In this section we contrast the initial non-uniformities $\big(\gamma(0,x),v(0,x),\theta(0,x),\sigma(0,x),u(0,x)\big)=\\ \big(\Gamma(\xi),V(\xi),\Theta(\xi),\Sigma(\xi),U(\xi)\big)|_{\xi=x}$ from the one snapshot at a certain time of the uniform shearing motion.
\begin{proposition} \label{prop:ss}
Let $\big(\Gamma(\xi),V(\xi),\Theta(\xi),\Sigma(\xi),U(\xi)\big)$ be the self-similar profiles that is defined by transformations of \eqref{eq:CAPtoBAR}, \eqref{eq:BARtoTIL}, and \eqref{eq:pqrdef} upon to the heteroclinic orbit $\chi(\eta)=\big(p(\eta),q(\eta),r(\eta),s(\eta)\big)$ constructed by Theorem \ref{thm1} in the valid range of the parameters $\Gamma(0)$ and $U(0)$ of \eqref{eq:restriction}. Then it follows that:
 \begin{enumerate}
  \item[(i)] The self-similar profile achieves the boundary condition at $\xi=0$,
    \begin{equation*}
    {V}(0) = \Gamma_\xi(0) = \Theta_\xi(0)=\Sigma_\xi(0) = {U}_\xi(0)=0, \quad \Gamma(0)=\Gamma_0, \quad U(0)=U_0.
  \end{equation*}
  \item[(ii)] Its asymptotic behavior as $\xi \rightarrow 0$ is given by
  \begin{equation} \label{eq:ss_asymp0}
  \begin{aligned}
    \Gamma(\xi) &= \Gamma_0 + \Gamma^{''}(0)\frac{\xi^2}{2} + o(\xi^2), & \Gamma^{''}(0)&<0,\\
    \Theta(\xi) &= \left(\frac{\Gamma_0^m U_0^{1+n}}{c}\right)^{\frac{1}{1+\alpha}} + \Theta^{''}(0)\frac{\xi^2}{2} + o(\xi^2), & \Theta^{''}(0)&<0,\\
    \Sigma(\xi) &= c^{\frac{\alpha}{1+\alpha}}\Gamma_0^{\frac{m}{1+\alpha}} U_0^{-\frac{\alpha-n}{1+\alpha}}+ \Sigma^{''}(0)\frac{\xi^2}{2} + o(\xi^2), & \Sigma^{''}(0)&>0, \\
    U(\xi) &= U_0 + U^{''}(0)\frac{\xi^2}{2} + o(\xi^2), & U^{''}(0)&<0,\\
    V(\xi) &= U_0\xi + U^{''}(0)\frac{\xi^3}{6} + o(\xi^3), & U^{''}(0)&<0.
  \end{aligned}
  \end{equation}
  \item[(iii)] Its asymptotic behavior as $\xi \rightarrow \infty$ is given by\\
  if $\mu_{11}\ne-1$ or $\mu_{11}=-1$ but $b=\lambda$,
  \begin{equation} \label{eq:ss_asymp1}
  \begin{aligned}
    \Gamma(\xi) &= \BO\big(\xi^{-\frac{1+\alpha}{\alpha-m-n}}), & V(\xi) &= \BO\big(1), &    \Theta(\xi) &= \BO\big(\xi^{-\frac{1+m+n}{\alpha-m-n}}),\\
   \Sigma(\xi) &= \BO\big(\xi), &   U(\xi) &= \BO\big(\xi^{-\frac{1+\alpha}{\alpha-m-n}})
  \end{aligned}
  \end{equation}
  otherwise
    \begin{equation} \label{eq:ss_asymp2}
  \begin{aligned}
    \Gamma(\xi) &= \BO\big(\xi^{-\frac{1+\alpha}{\alpha-m-n}}\big(\log\xi\big)^{\frac{1+\alpha}{D}}\big), & V(\xi) &= \BO\big(\big(\log\xi\big)^{-\frac{\alpha-m-n}{D}}\big), &    \Theta(\xi) &= \BO\big(\xi^{-\frac{1+m+n}{\alpha-m-n}}\big(\log\xi\big)^{\frac{1+m+n}{D}}\big),\\
   \Sigma(\xi) &= \BO\big(\xi\big(\log\xi\big)^{-\frac{\alpha-m-n}{D}}\big), &   U(\xi) &= \BO\big(\xi^{-\frac{1+\alpha}{\alpha-m-n}}\big(\log\xi\big)^{\frac{1+\alpha}{D}}\big)
  \end{aligned}
  \end{equation}
 \end{enumerate}

\end{proposition}
\begin{proof}
The proof of the Proposition \ref{prop1} and Remark \ref{rem:signs} contains $(i)$ and $(ii)$ and thus we are left to prove $(iii)$. In the similar fashion to \eqref{eq:alpha-expan}, any orbit $\psi(\eta)$ in the local stable manifold of $W^s(M_1)$ is characterized by the triple $(\kappa_1',\kappa_2',\kappa_3')$ in association with the asymptotic expansion
\begin{equation}
\begin{aligned}
 &\psi(\eta) -M_1\\
 &= \begin{cases} \kappa_1'e^{\mu_{11}\eta}X_{11} + \kappa_2'e^{\mu_{12}\eta}X_{12} + \kappa_4'e^{\mu_{14}\eta}X_{14} + \text{high order terms} & \text{if $\mu_{11}\ne-1$ or $\mu_{11}=-1$ but $b=\lambda$,}\\
 \kappa_1'\eta e^{\mu_{11}\eta}X_{11}' + \kappa_2'e^{\mu_{12}\eta}X_{12} + \kappa_4'e^{\mu_{14}\eta}X_{14} + \text{high order terms} & \text{otherwise }
 \end{cases}
\end{aligned}
\end{equation}
as $\xi \rightarrow \infty$. As it may be associated with the generalized eigenvector, we have separated cases as above.

Now, we have $q \rightarrow 1$, $r \rightarrow r_1$, $s \rightarrow s_1$ but $p \rightarrow 0$ and the leading order of $p$ is to be found. We look up  $X_{11}$ term in the above because $p$-component of the vectors $X_{12}$ and $X_{14}$ is $0$. Because the plane $p\equiv0$ is an invariant plane for a non-linear flow, triples of the form $(0,\kappa_2',\kappa_3')$ spans this invariant plane. Because our heteroclinic orbit $\chi(\eta)$ ventures out from the plane $p\equiv0$, $\kappa_1'$ of the $\chi(\eta)$ cannot be $0$.
This implies that the leading order of $p(\log\xi)$ is
$$p(\log\xi) = \begin{cases} \BO(\xi^{\mu_{11}}) & \text{if $\mu_{11}\ne-1$ or $\mu_{11}=-1$ but $b=\lambda$,}\\
 \BO(\xi^{\mu_{11}}\log\xi) & \text{otherwise}
 \end{cases}
 $$
as $\xi \rightarrow \infty$.

Asymptotics \eqref{eq:ss_asymp1} and \eqref{eq:ss_asymp2} are the straightforward calculations from the reconstruction formulas
\begin{align*}
 \tg&=p^{\frac{1+\alpha}{D}}r^{\frac{n}{D}}s^{\frac{\alpha}{D}}, & \tv &= \frac{1}{b} p^{-\frac{\alpha-m-n}{D}}qr^{\frac{n}{D}}s^{\frac{\alpha}{D}}, & \tth&=p^{\frac{1+m+n}{D}}r^{\frac{2n}{D}}s^{-\frac{1-m-n}{D}}, \\ \ts&=p^{-\frac{\alpha-m-n}{D}}r^{\frac{n}{D}}s^{\frac{\alpha}{D}},  & \tu&=p^{\frac{1+\alpha}{D}}r^{\frac{n}{D}+1}s^{\frac{\alpha}{D}},
\end{align*}
\eqref{eq:CAPtoBAR}, and \eqref{eq:BARtoTIL}.
%
% $p \sim \xi^{-\frac{1+n}{1+\alpha}}$
%
% $$ \tv \sim \xi^{-\frac{1+n}{D}}, \quad \tth \sim \xi^{-\frac{(1+n)^2}{(\alpha-n)D}}, \quad \ts \sim \xi^{\frac{1+n}{D}}, \quad \tu \sim \xi^{-\frac{(1+n)(1+\alpha)}{(\alpha-n)D}}$$
%
% $$ V \sim O(1), \quad \Theta \sim \xi^{-\frac{1+n}{\alpha-n}}, \quad \Sigma \sim \xi, \quad U \sim \xi^{-\frac{1+\alpha}{\alpha-n}}.$$
\end{proof}

\subsubsection{Emergence of localization}
As time proceeds, the initial states evolve out emerging the localization and this section is devoted to describing this localization: the deviation in the growth or decay rate at the origin from those at the rest of the places are contrasted, which is the emergence of the localization.

For its illustrations, we presented in the below the generic cases of $-\frac{1+m+n}{\alpha-m-n}\ne-1$; in non-generic cases we would have the logarithmic corrections according to the Proposition \ref{prop:ss}.
\begin{itemize}
 \item Strain : The strain keeps increasing at a fixed $x$ as time proceeds. However the growth at the origin is faster than that at the rest of the places,
\begin{align*}
 \gamma(t,0) &= (1+t)^{\frac{2+2\alpha-n}{D} + \frac{2+2\alpha}{D}\lambda}\Gamma(0), &
 \gamma(t,x) &\sim t^{\frac{2+2\alpha-n}{D} - \frac{(1+\alpha)(1+m+n)}{D(\alpha-m-n)}\lambda}|x|^{-\frac{1+\alpha}{\alpha-m-n}}, \quad \text{as $t \rightarrow \infty$, $x\ne0$.}
\end{align*}
Remember that the positivity of the growth rate $\frac{2+2\alpha-n}{D} - \frac{(1+\alpha)(1+m+n)}{D(\alpha-m-n)}\lambda$ was the ground for imposing the constraint \eqref{eq:lambda-range}.
\item Temperature : The temperature keeps increasing at a fixed $x$ as time proceeds. The growth at the origin is faster than that at the rest of the places,
\begin{align*}
 \theta(t,0) &= (1+t)^{\frac{2(1+m)}{D} + \frac{2(1+m+n)}{D}\lambda}\Theta(0),&
 \theta(t,x) &\sim t^{\frac{2(1+m)}{D} - \frac{(1+m+n)^2}{D(\alpha-m-n)}\lambda}|x|^{-\frac{1+m+n}{\alpha-m-n}}, \quad \text{as $t \rightarrow \infty$, $x\ne0$.}
\end{align*}
Again, the positivity of the growth rate $\frac{2(1+m)}{D} - \frac{(1+m+n)^2}{D(\alpha-m-n)}\lambda$ is from \eqref{eq:lambda-range}.
\item Strain rate : The growth rates of the strain-rate is by definition less by one to those of the strain, again illustrating the localization.
\begin{align*}
 u(t,0) &= (1+t)^{\frac{1+m}{D} + \frac{2+2\alpha}{D}\lambda}U(0),&
 u(t,x) &\sim t^{\frac{1+m}{D} - \frac{(1+\alpha)(1+m+n)}{D(\alpha-m-n)}\lambda}|x|^{-\frac{1+\alpha}{\alpha-m-n}}, \quad \text{as $t \rightarrow \infty$, $x\ne0$.}
\end{align*}
\item Stress : The stress keeps decreasing at a fixed $x$ as time proceeds. However, collapsing down of the stress at the origin is severer than that at the rest of the places,
\begin{align*}
 \sigma(t,0) &= (1+t)^{\frac{-2\alpha+2m+n}{D} + \frac{-2\alpha+2m+2n}{D}\lambda}\Sigma(0), &
 \sigma(t,x) &\sim t^{\frac{-2\alpha+2m+n}{D} +\frac{1+m+n}{D}\lambda}|x|, \quad \text{as $t \rightarrow \infty$, $x\ne0$,}
\end{align*}
noticing that $\frac{-2\alpha+2m+n}{D} +\frac{1+m+n}{D}\lambda\le-\frac{n}{1+m+n}$ in the valid range of the $\lambda$.


\item Velocity : The velocity is an odd function of $x$. At each time $t$, $v(t,x)$ is an increasing function of $x$ from $-v_\infty(t)$ to $v_\infty(t)$, as $x$ runs from $-\infty$ to $\infty$, where $v_\infty(t)\triangleq \lim_{x \rightarrow \infty} v(t,x)$. The velocity field is contrasted with the linear field of uniform shear motion. The self-similar scaling  $\xi=(1+t)^\lambda x$ implies that as time goes by the most of the transition takes place around the origin making the slope at origin steeper accordingly. It eventually forms a step-function type singularity at origin. Note that the asymptotic velocity $$v_\infty(t)=(1+t)^{b}V_\infty = (1+t)^{\frac{1+m}{D} + \frac{1+m+n}{D}\lambda}V_\infty, \quad V_\infty \triangleq \lim_{\xi \rightarrow \infty} V(\xi) <\infty.$$
This dictates that the particle of our solutions gets faster as it moves forward and we find the far field loading condition on the velocity different from that of uniform shearing motion, for the latter we have the constant velocity boundary condition. This deviation is a consequence of our simplifying assumption of self-similarity.
\end{itemize}

\begin{figure}[ht]
 \centering
%  \subfigure[{\red strain ($\gamma$)}]{
%  \includegraphics[scale=0.3]{strain_log} \label{fig:gamma}
%  }
 \subfigure[strain rate ($u$)]{
 \includegraphics[scale=0.3]{strain_rate_log} \label{fig:u}
 }
 \subfigure[temperature ($\theta$)]{
 \includegraphics[scale=0.3]{temperature_log} \label{fig:theta}
 }
 \subfigure[stress ($\sigma$)] {
 \includegraphics[scale=0.3]{stress_log} \label{fig:sigma}
 }
 \subfigure[velocity ($v$)] {
 \includegraphics[scale=0.3]{velocity} \label{fig:v}
 }
\caption{The localizing solution, for $\alpha=1.2$, $m=0$, $n=0.3$, $\lambda =0.5325$, and $\eta_0=-12.94$, sketched in the original variables $u, \ \theta, \ \sigma, \ v$.
All graphs except $v$ are in logarithmic scale.} \label{fig:computation}
\end{figure}

In Figure \ref{fig:computation} are the computational results of the original field variables $\big(\gamma(t,x)$, $u(t,x)$, $\theta(t,x)$, $\tau(t,x)$, $v(t,x)\big)$ that are computed from the heteroclinic orbit of \eqref{eq:slow} succeeded by transformations \eqref{eq:pqrdef}, \eqref{eq:BARtoTIL}, \eqref{eq:CAPtoBAR}, and \eqref{eq:ORItoCAP}. Parameters $\big(\alpha,m,n,\lambda)=(1.2,0,0.3,0.5325)$ was chosen for the computation. Before we examine them, we address one technical issue on the nontrivial choice of $m=0$ for its capturing. $M_0$ and $M_1$ are hyperbolic equilibrium points and their spectrum split into those with positive and negative real parts. While the Stable Manifold Theorem gives the separation of the corresponding stable and unstable manifolds, to achieve the same computationally is challenging task because the dynamical system is highly stiff in such a way that the presence of unstable subspace makes the computation destabilized. Since $M_0$ and $M_1$ both are saddle, computation in neither of time direction resolves this difficulty. 

While \eqref{intro-system0} bears this difficulty, with $m=0$, $\eqref{intro-system0}_2$-$\eqref{intro-system0}_4$ comprises a closed system, decoupling the first equation. The dimension is reduced by one in the corresponding dynamical system either (See \cite{KLT_HYP2016} for the precise formulation) and there, $M_0$ is an unstable node. By contrast to the saddle-saddle connection, in cases one of the equilibrium points is a stable or unstable {\it node}, solving the system in the stabilizing time direction produces numerically stable computations. We were able to compute the heteroclinic orbit backward in time, and to find Figure \ref{fig:computation}. Capturing the saddle-saddle connection will need more substantial considerations in numerical approach than that has been approached here.


Figures \ref{fig:computation} illustrates the emergence of localization by depicting the profiles at a few instances of time. Note that the vertical axes, except the one from Figure \ref{fig:v} for the velocity, are in logarithmic scale; the localization is significant. In Figure \ref{fig:u}, one sees the localization in strain rate; the initial profile is a small perturbation of the flat status whereas that at later time exhibits the localization at the origin. The same are seen in Figure \ref{fig:theta} %for the strain rate, and in Figure \ref{fig:theta} 
for the temperature as well. In the velocity profiles, it is clearly seen in Figure \ref{fig:v} the development of step function like singularity. On the other hand, stress collapses down to zero as seen from Figure \ref{fig:sigma}, with different rates at the origin and at the rest of the points respectively.








% \section*{Appendix A. Coefficient matrices for the linearized system} \label{append:linear}
% Using
% \begin{align*}
%  (r+R)^{1+n} = \begin{cases}
%                 R^{1+n} &\text{if $r=0$},\\
%                 r^{1+n}\Big(1+\frac{R}{r}\Big)^{1+n} = r^{1+n} + (1+n)r^nR + \mathcal{O}(\delta^2), & \text{if $r>\bar{r}>0$,}
%                \end{cases}
% \end{align*}
% the linearized system around the equilibrium point is:
%
% \noindent
% {\bf Cases $r=0$, $p=0$, $q=0$ or $q=1$}
% \begin{align*}
%  \dot{P} &=P\Big(-q-\frac{1+\alpha}{\lambda(1+n)} c_0 -d_1\Big),\\
%  \dot{Q} &=Q(1-q) -qQ,\\
%  \dot{R} &=\frac{R}{n}\Big(q-\frac{\alpha-n}{\lambda(1+n)} c_0 +d_1 \Big).
% \end{align*}
% \noindent
% {\bf Cases $\Big( \frac{\alpha-n}{\lambda(1+n)} r^{1+n} - \frac{\alpha-n}{\lambda(1+n)}c_0 + d_1 + q \Big)=0$, $p=0$, $q=0$ or $q=1$}
% \begin{align*}
%  \dot{P}&=P\Big( \frac{1+\alpha}{\lambda(1+n)} r^{1+n} - \frac{1+\alpha}{\lambda(1+n)} c_0 -d_1-q\Big) = P\Big(-\frac{D}{\alpha-n}(d_1+q)\Big),\\
%  \dot{Q}&=Q(1-q) +q(-Q-\lambda Pr) + bPr,\\
%  \dot{R}&=\frac{r}{n}\Big( \frac{\alpha-n}{\lambda} r^nR + Q + \lambda Pr\Big) + \frac{R}{n}\Big(\frac{\alpha-n}{\lambda(1+n)}r^{1+n}-\frac{\alpha-n}{\lambda(1+n)}r^{1+n}c_0 + d_1 +q\Big) = \frac{r}{n}\Big( \frac{\alpha-n}{\lambda} r^nR + Q + \lambda Pr\Big)
% \end{align*}
%
% Coefficients Matrices $Mat_i$ for Linearized equations around $M_i$, $i=0,1,2,3$:
% \begin{align*}
%  Mat_0 &= \begin{pmatrix}
%           -\frac{D}{\alpha-n}(d_1) & 0 & 0\\
%           br_0 & 1 & 0\\
%           \frac{\lambda r_0^2}{n} & \frac{r_0}{n} & \frac{\alpha-n}{n\lambda}r_0^{1+n}
%          \end{pmatrix}
%         = \begin{pmatrix}
%           2 & 0 & 0\\
%           br_0 & 1 & 0\\
%           \frac{\lambda r_0^2}{n} & \frac{r_0}{n} & \frac{\alpha-n}{n\lambda}r_0^{1+n}
%          \end{pmatrix}\\
%  Mat_1 &= \begin{pmatrix}
%           -\frac{D}{\alpha-n}(d_1+1) & 0 & 0\\
%           (b-\lambda)r_1 & -1 & 0\\
%           \frac{\lambda r_1^2}{n} & \frac{r_1}{n} & \frac{\alpha-n}{n\lambda}r_1^{1+n}
%          \end{pmatrix}
%         =\begin{pmatrix}
%           -\frac{1+n}{\alpha-n} & 0 & 0\\
%           (b-\lambda)r_1 & -1 & 0\\
%           \frac{\lambda r_1^2}{n} & \frac{r_1}{n} & \frac{\alpha-n}{n\lambda}r_1^{1+n}
%          \end{pmatrix}\\
%  Mat_2 &= \begin{pmatrix}
% 	  -1-\frac{1+\alpha}{\lambda(1+n)} c_0 -d_1 & 0 & 0\\
% 	  0 & -1 & 0\\
% 	  0 & 0 & \frac{1}{n}\Big(1-\frac{\alpha-n}{\lambda(1+n)} c_0 +d_1\Big)
%          \end{pmatrix}
%         = \begin{pmatrix}
% 	  -\frac{1+\alpha}{\lambda(1+n)} \Big(\frac{2}{D} + \frac{(1+n)^2}{D(1+\alpha)}\lambda\Big) & 0 & 0\\
% 	  0 & -1 & 0\\
% 	  0 & 0 & -\frac{\alpha-n}{\lambda n(1+n)}r_1^{1+n}
%          \end{pmatrix}\\
%  Mat_3 &= \begin{pmatrix}
% 	  -\frac{1+\alpha}{\lambda(1+n)} c_0 -d_1 & 0 & 0\\
% 	  0 & 1 & 0\\
% 	  0 & 0 & \frac{1}{n}\Big(-\frac{\alpha-n}{\lambda(1+n)} c_0 +d_1\Big)
%          \end{pmatrix}
% 	=\begin{pmatrix}
% 	  -\frac{1+\alpha}{\lambda(1+n)} \Big(\frac{2}{D} - \frac{2(\alpha-n)(1+n)}{D(1+\alpha)}\lambda\Big)& 0 & 0\\
% 	  0 & 1 & 0\\
% 	  0 & 0 & -\frac{\alpha-n}{\lambda n(1+n)}r_0^{1+n}
%          \end{pmatrix}
% \end{align*}
% The lower triangular matrix has the eigenvalues and eigenvectors such that
% \begin{align*}
%  MAT &= \begin{pmatrix}
%         A & 0 & 0\\
%         B & C & 0\\
%         D & E & F
%        \end{pmatrix}, \quad
%  \mu_1 = A, \quad\mu_2 = C, \quad\mu_3 = F,\\
%  v_1 &= \Big( \frac{ (A-C)(A-F) }{ D(A-C) + BE }, \frac{ B(A-F) }{ D(A-C) + BE }, 1), \quad  v_2 = (0, \frac{C-F}{E}, 1), \quad v_3 = (0,0,1).
% \end{align*}
% The eigenvectors in Section \ref{sec:equil} took the suitable normalization.
% \begin{align*}
%  &X_{01} = \bigg( \Big( \frac{2n - \frac{\alpha-n}{\lambda}r_0^{1+n}}{\big({\lambda}+b\big) r_0^2}\Big) \;,\;\Big( \frac{2n - \frac{\alpha-n}{\lambda}r_0^{1+n}}{\big({\lambda}+b\big) r_0^2}\Big)br_0\;,\;1\bigg),\quad
%  X_{02} = \bigg(0, \Big(\frac{n- \frac{\alpha-n}{\lambda}r_0^{1+n}}{r_0}\Big), 1\bigg), \quad
%  X_{03} = (0,0,1),\\
%  &X_{11} = \bigg(  \Big(\frac{-n\frac{1+n}{\alpha-n} - \frac{\alpha-n}{\lambda}r_1^{1+n}}{\big(-\frac{1+n}{\alpha-n} \lambda +b\big) r_1^2}\Big)\Big(1-\frac{1+n}{\alpha-n}\Big) \;,\;\Big(\frac{-n\frac{1+n}{\alpha-n} - \frac{\alpha-n}{\lambda}r_1^{1+n}}{\big(-\frac{1+n}{\alpha-n} \lambda +b\big) r_1^2}\Big)(b-\lambda)r_1\;,\;1\bigg),\\
%  &X_{12} = \bigg(0, \Big(\frac{n- \frac{\alpha-n}{\lambda}r_0^{1+n}}{r_0}\Big), 1\bigg), \quad
%  X_{13} = (0,0,1),
% \end{align*}
\appendix
\section{The loss of hyperbolicity} \label{append:hadamard}
We write \eqref{intro-system0} in a form of the first order transport equations.
\begin{equation} \label{eq:transport}
 \begin{pmatrix} \gamma_t \\ \theta_t \\ v_t \end{pmatrix} = \underbrace{
 \begin{pmatrix}
  0 & 0 & 1\\
  0 & 0 & \theta^{-\alpha}\gamma^m \\
  m\theta^{-\alpha}\gamma^{m-1} & -\alpha\theta^{-\alpha-1}\gamma^m & 0 \end{pmatrix}}_\text{$\triangleq B$}
  \begin{pmatrix} \gamma_x \\ \theta_x \\ v_x \end{pmatrix}.
\end{equation}
\eqref{eq:transport} is hyperbolic if $B$ has three real eigenvalues and three linearly independent eigenvectors.
\begin{align*}
 \det\big(B-\mu\textrm{I}\big) &= -\mu\big(\mu^2+\theta^{-\alpha}\gamma^{m-1}(\alpha \theta^{-\alpha-1}\gamma^{1+m} - m)\big)\\
 &=-\mu\big(\mu^2+\theta^{-\alpha}\gamma^{m-1}\Big(\frac{\alpha-m}{1+\alpha}+\alpha \Big(\frac{\gamma^{1+m}}{\theta^{1+\alpha}} - \frac{1+m}{1+\alpha}\Big)\Big)\\
 &=-\mu\big(\mu^2+\theta^{-\alpha}\gamma^{m-1}\Big(\frac{\alpha-m}{1+\alpha}+\alpha \frac{\gamma_0(x)^{1+m} - \frac{1+m}{1+\alpha}\theta_0(x)^{1+\alpha}}{\theta^{1+\alpha}}\Big),
%  \intertext{and for uniform shearing solution \eqref{intro:uss}}
%  &=-\mu\left(\mu^2+\theta_s^{-\alpha}\gamma_s^{m-1}\Big(\frac{\alpha-m}{1+\alpha}- \frac{\gamma_0^{m+1}}{\theta_s^{1+\alpha}} +\frac{1+m}{1+\alpha}\frac{\theta_0^{1+\alpha}}{\theta_s^{1+\alpha}}\Big)\right)
\end{align*}
where $\gamma_0(x)$ and $\theta_0(x)$ are initial values of $\gamma$ and $\theta$ respectively. If $\alpha-m>0$ and $\theta$ diverges as $t \rightarrow \infty$, \eqref{eq:transport} loses hyperbolicity in a finite time.
%
% but $\theta_s$ diverges as $t \rightarrow \infty$, or $\displaystyle - \frac{\gamma_0^{m+1}}{\theta_s^{1+\alpha}} +\frac{1+m}{1+\alpha}\frac{\theta_0^{1+\alpha}}{\theta_s^{1+\alpha}} \rightarrow 0$. If $\displaystyle \frac{\alpha-m}{1+\alpha}>0$, \eqref{eq:transport} loses hyperbolicity in a finite time.

\section{Other equilibria and the reasons of rejections}\label{append:equi_reject}
Values that gives the reason of rejection are underlined.
\begin{align*}
 &(1) & & p=0, \ q=0, \ \underline{ r=0, \ s=0},\\
 &(2) & & p=0, \ q=0, \ \underline{ r=0},\ s=s, \ \underline{ r_0 = \frac{-n}{\alpha-m-n}},\\
 &(3) & & p=0, \ q=0, \ r = \frac{n\alpha - r_0(\alpha-m-n)}{(1+\alpha)(m+n)},\ \underline{ s=0},\\
 &(4) & & p=0, \ q=1, \ \underline{ r=0, \ s=0}, \\
 &(5) & & p=0, \ q=1, \ \underline{ r=0},\ s=s, \ \underline{ r_1 = \frac{-n}{\alpha-m-n}},\\
 &(6) & & p=0, \ q=1, \ r = \frac{n\alpha - r_1(\alpha-m-n)}{(1+\alpha)(m+n)},\ \underline{ s=0},\\
 &(7) & & p=p, \ q=0, \ \underline{ r=0, \ s=0}, \ \frac{r_0}{\lambda}=2, \\
 &(8) & & p=p, \ q=1, \ \underline{ r=0, \ s=0}, \ \frac{r_0}{\lambda}=1, \\
 &(9) & & p=p, \ q=0, \ \underline{ r=0}, \ s=s, \ \frac{r_0}{\lambda}=2, \ \underline{ r_0 = \frac{-n}{\alpha-m-n}}, \\
 &(10) & & p=p, \ q=1, \ \underline{ r=0}, \ s=s, \ \frac{r_0}{\lambda}=1, \ \underline{ r_1 = \frac{-n}{\alpha-m-n}}, \\
 &(11) & & \underline{ p=-\frac{(\alpha-m-n)(1+m+n)}{(1+\alpha)(1+m)}<0}, \ q=\frac{2(\alpha-m-n)}{1+m}b, \ r=a_0, \ s=\frac{1+m+n}{1+\alpha} - \frac{n}{(1+\alpha)a_0},\\
 &(12) & & p=\left( \frac{2\alpha(1+m)}{D(1-m-n)} + \frac{2(\alpha-m-n)}{D}\lambda\right)\left(\frac{2\alpha(1+m)}{D(1-m-n)} - \frac{1+m+n}{D}\lambda\right)\frac{1-m-n}{\lambda(2-n)}\frac{1-m-n}{\lambda(1+m)}, \\
 & & &q=\left( \frac{2\alpha(1+m)}{D(1-m-n)} + \frac{2(\alpha-m-n)}{D}\lambda\right)\left(\frac{1+m}{D} + \frac{1+m+n}{D}\lambda\right)\frac{1-m-n}{\lambda(1+m)},\\
 & & &r = \frac{2-n}{1-m-n}, \ \underline{ s=0}.
\end{align*}

\section{Linearized problems around $M_0$ and $M_1$}\label{append:lin}
Coefficients matrix for the Linearized equations around $M_0$ is
\begin{align*}
 \begin{pmatrix}
          2 & 0 & 0 & 0 \\
          br_0 & 1 & 0 & 0\\
          \frac{r_0}{n}(\lambda r_0) & \frac{r_0}{n} & \frac{r_0}{n}\Big(\frac{\alpha-m-n}{\lambda(1+\alpha)} - \frac{n\alpha}{\lambda(1+\alpha)r_0}\Big) & \frac{r_0}{n}(\frac{\alpha r_0}{\lambda})\\
          s_0(\lambda r_0) & s_0 & s_0\Big(\frac{\alpha-m-n}{\lambda(1+\alpha)} + \frac{n}{\lambda(1+\alpha)r_0}\Big) & s_0(-\frac{r_0}{\lambda})
         \end{pmatrix}
        =\begin{pmatrix}
          2 & 0 & 0 & 0 \\
          br_0 & 1 & 0 & 0\\
          \frac{r_0}{n}(\lambda r_0) & \frac{r_0}{n} & \frac{r_0}{n}\frac{1}{\lambda}\Big(1-s_0-\frac{n}{r_0}\Big) & \frac{r_0}{n}(\frac{\alpha r_0}{\lambda})\\
          s_0(\lambda r_0) & s_0 & s_0\frac{1}{\lambda}(1-s_0) & s_0(-\frac{r_0}{\lambda})
         \end{pmatrix}
\end{align*}
Eigenvector $X_{0j}$ is collected in the matrix $S_0$ as $j$-th column vector, $j=1,2,3,4$.
\begin{equation} \label{eq:S0}
\begin{aligned}
 S_0&=
 \begin{pmatrix}
    1 & 0 & 0 & 0\\
    br_0 & 1 & 0 & 0\\
    y_1 & y_2 & 1 & y_4\\
    z_1 & z_2 & z_3 &1
 \end{pmatrix},
 \quad \quad
 \begin{array}{l}
\begin{pmatrix}
 y_1\\z_1
\end{pmatrix}
=-(\lambda+b)r_0\begin{pmatrix}
  \frac{ \frac{1+\alpha}{\lambda}r_0 + \frac{\mu_{01}}{s_0} }{ \Delta_1 }\\
  \frac{ \frac{n}{r_0}\big(\frac{1}{\lambda} + \mu_{01}\big) }{ \Delta_1 }
  \end{pmatrix},
  \quad
 \begin{pmatrix}
 y_2\\z_2
\end{pmatrix}
=-\begin{pmatrix}
  \frac{ \frac{1+\alpha}{\lambda}r_0 + \frac{\mu_{02}}{s_0} }{ \Delta_2 }\\
  \frac{ \frac{n}{r_0}\big(\frac{1}{\lambda} + \mu_{02}\big) }{ \Delta_2 }
  \end{pmatrix}\\
%    y_1=-\frac{(\lambda+b)r_0}{\frac{1-s_0}{\lambda} - nA}, \quad A=\frac{\big(\frac{r_0}{\lambda}+\frac{2}{s_0}\big)\big(\frac{1}{\lambda}+2\big)\frac{1}{r_0}}{ \frac{1+\alpha}{\lambda}r_0 + \frac{2}{s_0} }, \\
%  z_1=n\bigg(\frac{\big(\frac{1}{\lambda}+2\big)\frac{1}{r_0}}{ \frac{1+\alpha}{\lambda}r_0 + \frac{2}{s_0} }\bigg)y_1, \\
%  y_2=-\frac{1}{\frac{1-s_0}{\lambda} - nB }, \quad B=\frac{\big(\frac{r_0}{\lambda}+\frac{1}{s_0}\big)\big(\frac{1}{\lambda}+1\big)\frac{1}{r_0}}{ \frac{1+\alpha}{\lambda}r + \frac{1}{s_0} }\\
%  z_2=n\bigg(\frac{\big(\frac{1}{\lambda}+1\big)\frac{1}{r_0}}{ \frac{1+\alpha}{\lambda}r_0 + \frac{1}{s_0} }\bigg)y_2, \\
 z_3=n\bigg(\frac{\frac{1-s_0}{\lambda}}{\frac{n r_0}{\lambda} + \frac{n\mu_{0}^+}{s_0}}\bigg),\quad y_4=\frac{\frac{r_0}{\lambda}+\frac{\mu_0^-}{s_0}}{\frac{1-s_0}{\lambda}},
 \end{array}
\end{aligned}
\end{equation}
where $\Delta_1 = \frac{1-s_0}{\lambda}\big(\frac{1+\alpha}{\lambda}r_0 + \frac{\mu_{01}}{s_0}\big) -\frac{n}{r_0} \big( \frac{1}{\lambda} + \mu_{01}\big)\big(\frac{r_0}{\lambda} + \frac{\mu_{01}}{s_0}\big)$
%=\frac{-n}{r_0s_0}\det \left[\begin{pmatrix} \frac{r_0}{n}\big(\frac{1-s_0}{\lambda}-\frac{n}{\lambda r_0}\big) & \frac{r_0}{n}\frac{\alpha r_0}{\lambda}\\ s_0\frac{1-s_0}{\lambda} & -s_0\frac{r_0}{\lambda} \end{pmatrix} -\mu_{01}\textrm{I}\right]\ne0$
and $\Delta_2 = \frac{1-s_0}{\lambda}\big(\frac{1+\alpha}{\lambda}r_0 + \frac{\mu_{02}}{s_0}\big) -\frac{n}{r_0} \big( \frac{1}{\lambda} + \mu_{02}\big)\big(\frac{r_0}{\lambda} + \frac{\mu_{02}}{s_0}\big)$.
%=\frac{-n}{r_0s_0}\det \left[\begin{pmatrix} \frac{r_0}{n}\big(\frac{1-s_0}{\lambda}-\frac{n}{\lambda r_0}\big) & \frac{r_0}{n}\frac{\alpha r_0}{\lambda}\\ s_0\frac{1-s_0}{\lambda} & -s_0\frac{r_0}{\lambda} \end{pmatrix} -\mu_{02}\textrm{I}\right]\ne0$ respectively for the corresponding cases.
We find that $y_1,y_2,y_4<0$; $z_1,z_2,z_3 \sim\BO(n)$, provided $n$ is sufficiently small.

Next, coefficients matrix for the Linearized equations around $M_1$ is
\begin{align*}
 \begin{pmatrix}
          -\frac{1+m+n}{\alpha-m-n} & 0 & 0 & 0\\
          (b-\lambda)r_1 & -1 & 0 & 0\\
          \frac{r_1}{n}(\lambda r_1) & \frac{r_1}{n} & \frac{r_1}{n}\Big(\frac{\alpha-m-n}{\lambda(1+\alpha)} - \frac{n\alpha}{\lambda(1+\alpha)r_1}\Big) & \frac{r_1}{n}(\frac{\alpha r_1}{\lambda})\\
          s_1(\lambda r_1) & s_1 & s_1\Big(\frac{\alpha-m-n}{\lambda(1+\alpha)} + \frac{n}{\lambda(1+\alpha)r_1}\Big) & s_1(-\frac{r_1}{\lambda})
         \end{pmatrix}
         =\begin{pmatrix}
          -\frac{1+m+n}{\alpha-m-n} & 0 & 0 & 0\\
          (b-\lambda)r_1 & -1 & 0 & 0\\
          \frac{r_1}{n}(\lambda r_1) & \frac{r_1}{n} & \frac{r_1}{n}\frac{1}{\lambda}\Big(1-s_1-\frac{n}{r_1}\Big) & \frac{r_1}{n}(\frac{\alpha r_1}{\lambda})\\
          s_1(\lambda r_1) & s_1 & s_1\frac{1}{\lambda}(1-s_1) & s_1(-\frac{r_1}{\lambda})
         \end{pmatrix}
\end{align*}

%It turned out that the coefficient matrix for $M_1$ possibly possesses the Jordan block when $\mu_{11}$ and $\mu_{12}$ coincide. We first demonstrate the four eigenvectors for the generic case where $\mu_{11}\ne\mu_{12}$ and then specify the generalized eigenvectors for this special case.

The following exposition specifies all possible combinations but unless $\mu_{11}=\mu_{12}=-1$, the four linearly independent eigenvectors are attained, and when the exceptional case takes place the repeated eigenvalue $-1$ has one less geometric multiplicity than the algebraic multiplicity.% so we supplement one generalized eigenvector for the repeated eigenvalue $-1$.

As to the eigenvectors, notice first that the eigenvalues for $M_1$, differently from those for $M_0$, have chances to be repeated. While the exposition in the Appendix \ref{append:lin} specifies quite a few possible combinations, what is explained in the below is that unless $\mu_{11}=\mu_{12}=-1$, the four linearly independent eigenvectors are attained, and when the exceptional case takes place we will supplement precisely one generalized eigenvector for the repeated eigenvalue $-1$.

{\bf Case 1. $-\frac{1+m+n}{\alpha-m-n}\ne -1$; or $-\frac{1+m+n}{\alpha-m-n}= -1$ but $b=\lambda$. } This case yields the four linearly independent eigenvectors. The eigenvector $X_{1j}$ is collected in the matrix $S_1$ as $j$-th column vector, $j=1,2,3,4$, and in cases eigenvalues are repeated the corresponding eigenvectors are understood as a basis for the corresponding subspaces:
\begin{align*}
 S_1&=
 \begin{pmatrix}
    1 & 0 & 0 & 0\\
    x_1 & 1 & 0 & 0\\
    y_1 & y_2 & 1 & y_4\\
    z_1 & z_2 & z_3 &1
 \end{pmatrix}, \quad \quad
 \begin{array}{l}
  x_1=
 \begin{cases}
  \frac{(b-\lambda)r_1}{1+\mu_{11}} & \text{if $\mu_{11}\ne -1$,}\\
  0 & \text{otherwise,}
 \end{cases}\\
 z_3=n\bigg(\frac{\frac{1-s_1}{\lambda}}{\frac{n r_1}{\lambda} + \frac{n\mu_{1}^+}{s_1}}\bigg), \quad y_4=\frac{\frac{r_1}{\lambda}+\frac{\mu_1^-}{s_1}}{\frac{1-s_1}{\lambda}},\\
 \end{array}
\end{align*}
\begin{equation} \label{eq:S1-1}
\begin{aligned}
\begin{pmatrix}
 y_1\\z_1
\end{pmatrix}
=\begin{cases}
  -(\lambda r_1 + x_1)\begin{pmatrix}
  \frac{\lambda}{1-s_1}\\0
  \end{pmatrix} & \text{if $\mu_{14}=\mu_{11}$,}\\
  -(\lambda r_1 + x_1)
  \begin{pmatrix}
  \frac{ \frac{1+\alpha}{\lambda}r_1 + \frac{\mu_{11}}{s_1} }{ \Delta_3 }\\
  \frac{ \frac{n}{r_1}\big(\frac{1}{\lambda} + \mu_{11}\big) }{ \Delta_3 }
  \end{pmatrix} & \text{otherwise,}
 \end{cases}
 \quad
 \begin{pmatrix}
 y_2\\z_2
\end{pmatrix}
=\begin{cases}
 -\begin{pmatrix}
  \frac{\lambda}{1-s_1}\\0
  \end{pmatrix} & \text{if $\mu_{14}=\mu_{12}$,}\\
  -\begin{pmatrix}
  \frac{ \frac{1+\alpha}{\lambda}r_1 + \frac{\mu_{12}}{s_1} }{ \Delta_4 }\\
  \frac{ \frac{n}{r_1}\big(\frac{1}{\lambda} + \mu_{12}\big) }{ \Delta_4 }
  \end{pmatrix} & \text{otherwise,}
 \end{cases}
\end{aligned}
\end{equation}
where
\begin{align*}
 \Delta_3 &= \frac{1-s_1}{\lambda}\big(\frac{1+\alpha}{\lambda}r_1 + \frac{\mu_{11}}{s_1}\big) -\frac{n}{r_1} \big( \frac{1}{\lambda} + \mu_{11}\big)\big(\frac{r_1}{\lambda} + \frac{\mu_{11}}{s_1}\big)=\frac{-n}{r_1s_1}\det \left[\begin{pmatrix} \frac{r_1}{n}\big(\frac{1-s_1}{\lambda}-\frac{n}{\lambda r_1}\big) & \frac{r_1}{n}\frac{\alpha r_1}{\lambda}\\ s_1\frac{1-s_1}{\lambda} & -s_1\frac{r_1}{\lambda} \end{pmatrix} -\mu_{11}\textrm{I}\right]\ne0,\\
 \Delta_4 &= \frac{1-s_1}{\lambda}\big(\frac{1+\alpha}{\lambda}r_1 + \frac{\mu_{12}}{s_1}\big) -\frac{n}{r_1} \big( \frac{1}{\lambda} + \mu_{12}\big)\big(\frac{r_1}{\lambda} + \frac{\mu_{12}}{s_1}\big)=\frac{-n}{r_1s_1}\det \left[\begin{pmatrix} \frac{r_1}{n}\big(\frac{1-s_1}{\lambda}-\frac{n}{\lambda r_1}\big) & \frac{r_1}{n}\frac{\alpha r_1}{\lambda}\\ s_1\frac{1-s_1}{\lambda} & -s_1\frac{r_1}{\lambda} \end{pmatrix} -\mu_{12}\textrm{I}\right]\ne0
\end{align*}
respectively for the corresponding cases.
% $\Delta_3 = \frac{1-s_1}{\lambda}\big(\frac{1+\alpha}{\lambda}r_1 + \frac{\mu_{11}}{s_1}\big) -\frac{n}{r_1} \big( \frac{1}{\lambda} + \mu_{11}\big)\big(\frac{r_1}{\lambda} + \frac{\mu_{11}}{s_1}\big)=\frac{-n}{r_1s_1}\det \left[\begin{pmatrix} \frac{r_1}{n}\big(\frac{1-s_1}{\lambda}-\frac{n}{\lambda r_1}\big) & \frac{r_1}{n}\frac{\alpha r_1}{\lambda}\\ s_1\frac{1-s_1}{\lambda} & -s_1\frac{r_1}{\lambda} \end{pmatrix} -\mu_{11}\textrm{I}\right]\ne0$ and $\Delta_4 = \frac{1-s_1}{\lambda}\big(\frac{1+\alpha}{\lambda}r_1 + \frac{\mu_{12}}{s_1}\big) -\frac{n}{r_1} \big( \frac{1}{\lambda} + \mu_{12}\big)\big(\frac{r_1}{\lambda} + \frac{\mu_{12}}{s_1}\big)=\frac{-n}{r_1s_1}\det \left[\begin{pmatrix} \frac{r_1}{n}\big(\frac{1-s_1}{\lambda}-\frac{n}{\lambda r_1}\big) & \frac{r_1}{n}\frac{\alpha r_1}{\lambda}\\ s_1\frac{1-s_1}{\lambda} & -s_1\frac{r_1}{\lambda} \end{pmatrix} -\mu_{12}\textrm{I}\right]\ne0$

{\bf Case 2. $-\frac{1+m+n}{\alpha-m-n}= -1$ and $b\ne\lambda$: }
For this case $-1$ has one less geometric multiplicity than the algebraic multiplicity and we replace the first column of $S_1$ by the generalized eigenvector $\big(\frac{1}{(b-\lambda)r_1}, 0, y_1', z_1'\big)^T$, where
\begin{equation} \label{eq:S1-2}
\begin{aligned}
\begin{pmatrix}
 y_1'\\z_1'
\end{pmatrix}
=\begin{cases}
  \begin{pmatrix}
  -\frac{\lambda}{1-s_1}\big(\frac{\lambda}{b-\lambda} -\frac{n}{r_1}z_2\big)\\0
  \end{pmatrix} & \text{if $\mu_{14}=-1$,}\\
  -\frac{\lambda}{b-\lambda}
  \begin{pmatrix}
  \frac{ \frac{1+\alpha}{\lambda}r_1 + \frac{\mu_{11}}{s_1} }{ \Delta_3 }\\
  \frac{ \frac{n}{r_1}\big(\frac{1}{\lambda} + \mu_{11}\big) }{ \Delta_3 }
  \end{pmatrix} +
  \frac{n}{r_1}
  \begin{pmatrix}
  \frac{ y_2\big(\frac{r_1}{\lambda} + \frac{\mu_{11}}{s_1}\big) + z_2\frac{\alpha r_1}{\lambda} }{ \Delta_3 }\\
  \frac{ y_2\big(\frac{1-s_1}{\lambda}\big) + z_2\big(-\frac{1-s_1}{\lambda}+\frac{n}{r_1}\big(\frac{1}{\lambda}+\mu_{11}\big)\big) }{ \Delta_3 }
  \end{pmatrix} & \text{otherwise.}
 \end{cases}
\end{aligned}
\end{equation}

\begin{align*}
 0&=Mat_1 \begin{pmatrix} w\\x\\y\\z \end{pmatrix} -\mu \begin{pmatrix} w\\x\\y\\z \end{pmatrix}=
\begin{pmatrix}
 (\mu_{11}-\mu) w\\
 (b-\lambda)r_1 w +(\mu_{12}-\mu)x\\
 \frac{r_1}{n} \left[\lambda r_1 w + x + \big(\frac{1-s_1}{\lambda} - \frac{n}{r_1}\big(\frac{1}{\lambda}+\mu\big)\big)y + \frac{\alpha r_1}{\lambda}z\right]\\
 s_1 \left[ \lambda r_1 w + x + \big(\frac{1-s_1}{\lambda}\big)y -\big(\frac{r_1}{\lambda}+\frac{\mu}{s_1}\big)z\right]
 \end{pmatrix},\\
 A&\triangleq
 \begin{pmatrix}
 \frac{1-s_1}{\lambda} - \frac{n}{r_1}\big(\frac{1}{\lambda}+\mu\big) & \frac{\alpha r_1}{\lambda}\\
 \frac{1-s_1}{\lambda} & -\big(\frac{r_1}{\lambda}+\frac{\mu}{s_1}\big)
 \end{pmatrix}
 \begin{pmatrix} y\\z \end{pmatrix} = -(\lambda r_1 w +x)\begin{pmatrix} 1\\1\end{pmatrix}\\
 A^{-1} &= \frac{1}{\Delta}
 \begin{pmatrix} \big(\frac{r_1}{\lambda}+\frac{\mu}{s_1}\big) & \frac{\alpha r_1}{\lambda}\\
 \frac{1-s_1}{\lambda} & -\frac{1-s_1}{\lambda} + \frac{n}{r_1}\big(\frac{1}{\lambda}+\mu\big)
\end{pmatrix}, \quad
\Delta=\frac{1-s_1}{\lambda}\big( \frac{1+\alpha}{\lambda}r_1 + \frac{\mu}{s_1}\big) - \frac{n}{r_1}\big(\frac{1}{\lambda} +\mu\big)\big(\frac{r_1}{\lambda}+\frac{\mu}{s_1}\big)
\end{align*}



%
% \begin{align*}
%  y_1&=-\frac{(\lambda+b)r_0}{\frac{\alpha-m-n}{\lambda(1+\alpha)} - \frac{2n}{r_0(1+\alpha)}\Big(1 + \frac{(\frac{1}{\lambda}+2)\frac{\alpha}{s}}{ \frac{1+\alpha}{\lambda}r + \frac{2}{s} }\Big) }\\
%  y_2&=-\frac{1}{\frac{\alpha-m-n}{\lambda(1+\alpha)} - \frac{n}{r_0(1+\alpha)}\Big(1 + \frac{(\frac{1}{\lambda}+1)\frac{\alpha}{s}}{ \frac{1+\alpha}{\lambda}r + \frac{1}{s} }\Big) }\\
%  y_3&=\frac{\mu_0^- +\frac{r_0s_0}{\lambda}}{s\Big(\frac{\alpha-m-n}{\lambda(1+\alpha)} + \frac{n}{\lambda(1+\alpha)r_0}\Big)}\\
%  z_1&=n\bigg(\frac{\big(\frac{1}{\lambda}+2\big)\frac{1}{r}}{ \frac{1+\alpha}{\lambda}r + \frac{2}{s} }\bigg)y_1 \\
%  z_2&=n\bigg(\frac{\big(\frac{1}{\lambda}+1\big)\frac{1}{r}}{ \frac{1+\alpha}{\lambda}r + \frac{1}{s} }\bigg)y_2 \\
%  z_3&=n\bigg(\mu_0^+-\frac{r_0}{n}\Big(\frac{\alpha-m-n}{\lambda(1+\alpha)} - \frac{n\alpha}{\lambda(1+\alpha)r_0}\Big)\bigg)r_0(\frac{\alpha r_0}{\lambda})\bigg)
% \end{align*}



\begin{thebibliography}{10}

\bibitem{bertsch_effect_1991}
{\sc M.~Bertsch, L.~Peletier, and S.~Verduyn~Lunel},
The effect of temperature dependent viscosity on shear flow of  incompressible fluids,
{\it SIAM J. Math. Anal.} {\bf 22 } (1991), 328--343.

\bibitem{clifton_rev_1990}
{\sc R.J. Clifton},  High strain rate behaviour of metals,
% {\it Applied Mechanics Review}
{\it Appl. Mech. Rev.}
{\bf 43} (1990), S9-S22.

\bibitem{DH_1983}
{\sc C.M. Dafermos and L.~Hsiao},
Adiabatic shearing of incompressible fluids with temperature-dependent viscosity.
{\it Quart.  Applied Math.} {\bf 41} (1983), 45--58.

\bibitem{fenichel_persistence_1972}
{\sc N.~Fenichel},
Persistence and smoothness of invariant manifolds for  flows,
{\it Indiana Univ. Math. J.} {\bf 21} (1972) 193--226.

\bibitem{fenichel_asymptotic_1974}
{\sc N.~Fenichel},
Asymptotic stability with rate conditions,
{\it Indiana Univ. Math. J.} {\bf 23} (1974) 1109--1137.

\bibitem{fenichel_asymptotic_1977}
{\sc N.~Fenichel},
Asymptotic stability with rate conditions \textrm{II},
{\it Indiana Univ. Math. J.} {\bf 26} (1977) 81--93.

\bibitem{fenichel_geometric_1979}
{\sc N.~Fenichel},
Geometric singular perturbation theory for ordinary differential equations,
{\it J. Differ. Equations} {\bf 31} (1979), 53--98.

\bibitem{FM87}
{\sc C.~Fressengeas and A.~Molinari},
Instability and localization of plastic flow in shear at high strain rates,
{\it J.  Mech. Physics of Solids} {\bf 35} (1987), 185--211.

\bibitem{HPS_1977}
{\sc M.W. Hirsch, C.C. Pugh, and M. Shub},
{\it Invariant Manifolds}, LNM {\bf 583}, (Springer-Verlag, New York/Heidelberg/Berlin 1977)

\bibitem{HN77}
{\sc J.W.~Hutchinson and K.W.~Neale},
Influence of strain-rate sensitivity on necking under uniaxial tension,
{\it  Acta Metallurgica} {\bf 25} (1977), 839-846.

\bibitem{KOT14}
{\sc Th.~Katsaounis, J.~Olivier, and A.E.~Tzavaras},
Emergence of coherent localized structures in shear deformations of temperature dependent fluids,
{\it Archive for Rational Mechanics and Analysis} {\bf 224} (2017), 173--208.

\bibitem{KT09}
{\sc Th. Katsaounis and A.E.~Tzavaras},
Effective equations for localization and shear band formation,
{\it SIAM J. Appl. Math.}  {\bf 69} (2009), 1618--1643.

\bibitem{KLT_2016}
{\sc Th. Katsaounis, M.-G. Lee, and A.E. Tzavaras},
Localization in inelastic rate dependent shearing deformations,
{\it J. Mech. Phys. of Solids} {\bf 98} (2017), 106--125.

\bibitem{LT16}
{\sc M.-G.~Lee and A.E.~Tzavaras},
Existence of localizing solutions in plasticity via the geometric singular perturbation theory,
{\it Siam J. Appl. Dyn. Systems} {\bf 16} (2017), 337--360.

\bibitem{KLT_HYP2016}
{\sc M.-G. Lee, Th. Katsaounis, and A.E. Tzavaras},
Localization of Adiabatic Deformations in Thermoviscoplastic Materials, In Proceedings of the 16th International Conference on Hyperbolic Problems: Theory, Numerics, Applications (HYP2016), to appear.

%
% \bibitem{jones_geometric_1995}
% {\sc C.~K. R.~T. Jones},
% Geometric singular perturbation theory, in {\it Dynamical systems}, LNM {\bf 1609} (Springer Berlin Heidelberg 1995) 44--118.
%
%

%
% % \bibitem{perko_differential_2001}
% % {\sc L.~Perko},
% % {\it Differential equations and dynamical systems 3rd. ed.}, TAM {\bf 7} (Springer-Verlag New York 2001).
%

\bibitem{shawki_shear_1989}
{\sc T.G. Shawki and R.J. Clifton},
Shear band formation in thermal viscoplastic materials,
% {\it Mechanics of Materials}
{\it Mech. Mater.}
{\bf 8 } (1989), 13--43.

\bibitem{Sz1991}
{\sc P.~Szmolyan},
Transversal heteroclinic and homoclinic orbits in singular perturbation problems,
{\it J. Differ. Equations}
{\bf 92} (1991), 252--281.

\bibitem{Tz_1986}
{\sc A.E. Tzavaras},
Shearing of materials exhibiting thermal softening or temperature dependent viscosity,
{\em Quart.  Applied Math.} {\bf 44} (1986), 1--12.

\bibitem{Tz_1987}
{\sc A.E. Tzavaras},
Effect of thermal softening in shearing of strain-rate dependent materials.
{\em Archive for Rational Mechanics and Analysis}, {\bf 99} (1987), 349--374.

\bibitem{tzavaras_plastic_1986}
{\sc A.E. Tzavaras},
Plastic shearing of materials exhibiting strain hardening or strain softening,
% {\it Archive for Rational Mechanics and  Analysis}
{\it Arch. Ration. Mech. Anal.}
{\bf 94} (1986), 39--58.

\bibitem{tzavaras_strain_1991}
{\sc A.E. Tzavaras},
Strain softening in viscoelasticity of the rate type.
{\it J. Integral Equations Appl.} {\bf  3}  (1991), 195--238.

\bibitem{tzavaras_nonlinear_1992}
%\leavevmode\vrule height 2pt depth -1.6pt width 23pt,
{\sc A.E. Tzavaras},
Nonlinear analysis techniques for shear band formation at high strain-rates,
% {\it Applied Mechanics Reviews}
{\it Appl. Mech. Rev.}
{\bf  45} (1992), S82--S94.



%
% \bibitem{clifton_critical_1984}
% {\sc R.~J. Clifton, J.~Duffy, K.~A. Hartley, and T.~G. Shawki},
% On critical conditions for shear band formation at high strain rates.
% % {\it Scripta Metallurgica}
% {\it Scripta. Metall. Mater.}
% {\bf 18} (1984), 443--448.
%


%
% \bibitem{freistuhler_spectral_2002}
% {\sc H.~Freistühler and P.~Szmolyan},
% {Spectral stability of small shock waves},
% {\it Arch. Ration. Mech. Anal.}
% {\bf 164} (2002), 287--309.
% %   \href{http://dx.doi.org/10.1007/s00205-002-0215-8}{doi:\nolinkurl{10.1007/s00205-002-0215-8}},
% %   \url{http://dx.doi.org/10.1007/s00205-002-0215-8}.
% \bibitem{fressengeas_instability_1987}
% {\sc C.~Fressengeas, A.~Molinari},
% {Instability and localization of plastic flow in shear at high strain rates},
% {\it J. Mech. Phys. of Solids}
% {\bf 35} (1987), 185--211.
%
% \bibitem{gasser_geometric_1993}
% {\sc I.~Gasser and P.~Szmolyan},
% {A geometric singular perturbation analysis of detonation and deflagration waves},
% {\it {SIAM} J. Math. Anal.}
% {\bf 24} (1993), 968--986.
% %   \href{http://dx.doi.org/10.1137/0524058}{doi:\nolinkurl{10.1137/0524058}},
% %   \url{http://dx.doi.org/10.1137/0524058}.
% \bibitem{ghazaryan_traveling_2007}
% {\sc A.~Ghazaryan, P.~Gordon, and C.~K. R.~T. Jones},
% {Traveling waves in porous media combustion: uniqueness of waves for small thermal diffusivity},
% {\it J. Dyn. Differ. Equ.}
% {\bf 19} (2007), 951--966.
% %   \href{http://dx.doi.org/10.1007/s10884-007-9079-9}{doi:\nolinkurl{10.1007/s10884-007-9079-9}},
% %   \url{http://dx.doi.org/10.1007/s10884-007-9079-9}.
%
%

% \bibitem{MC_1987}
% {\sc A.~Molinari and R.~J. Clifton},
% Analytical characterization of shear localization in thermoviscoplastic materials,
% {\it Journal of Applied Mechanics}
% {\it J. Appl. Mech.}
% {\bf 54} (1987), 806--812.
%
%
% \bibitem{jones_geometric_1995}
% {\sc C.~K. R.~T. Jones},
% Geometric singular perturbation theory, in {\it Dynamical systems}, LNM {\bf 1609} (Springer Berlin Heidelberg 1995) 44--118.
%
%
%
%
%
%

%
% \bibitem{KUEHN_2015}
% {\sc C.~ Kuehn},
% {\it Multiple time scale dynamics}, Applied Mathematical Sciences, Vol. {\bf 191} (Springer Basel 2015).
%

%
% \bibitem{perko_differential_2001}
% {\sc L.~Perko},
% {\it Differential equations and dynamical systems 3rd. ed.}, TAM {\bf 7} (Springer-Verlag New York 2001).
%
%



%
% \bibitem{shawki_energy_1994}
% {\sc T.~G. Shawki},
% {An Energy Criterion for the Onset of Shear Localization in Thermal Viscoplastic Materials, Part II: Applications and Implications}, {\it ASME. J. Appl. Mech.}
% {\bf 61} (1994), 538--547.
%
%
% \bibitem{SS_2004}
% {\sc S. Schecter and P. Szmolyan}
% Composite waves in the Dafermos regularization.
% {\it J. Dynamics Diff. Equations} {\bf 16} (2004), 847-867.
%

%
% \bibitem{wiggins_normally_1994}
% {\sc S.~Wiggins},
% {\it Normally hyperbolic invariant manifolds in dynamical  systems}, AMS {\bf 105} (Springer-Verlag New York 1994).
%
\bibitem{wright_survey_2002}
{\sc T.W. Wright},
{\it The Physics and Mathematics of Shear Bands.} (Cambridge Univ. Press 2002).
%
% \bibitem{xiao_stability_2003}
% {\sc L.~Xiao-Biao and S.~ Schecter},
% {Stability of self-similar solutions of the {D}afermos regularization of a system of conservation laws},
% {SIAM J. Math. Anal.}
% {\bf 35} (2003), 884--921.

\bibitem{WF83}
{\sc F.H. Wu and L.B. Freund},
Deformation trapping due to thermoplastic instability in one-dimensional wave propagation,
{\it J. Mech. Phys. of Solids} {\bf  32} (1984), 119-132.

\bibitem{zener_effect_1944}
{\sc C.~Zener and J.~H. Hollomon},
Effect of strain rate upon plastic flow of steel,
% {\it  Journal of Applied Physics}
{\it J. Appl. Phys.}
{\bf 15} (1944), 22--32.

\end{thebibliography}
\end{document}

% \hrulefill
% \bibitem{dafermos_adiabatic_1983}
% {\sc C.~M. Dafermos and L.~Hsiao},
% Adiabatic shearing of incompressible fluids with temperature-dependent viscosity.
% {\it Quart.  Applied Math.} {\bf 41} (1983), 45--58.

% \bibitem{katsaounis_effective_2009}
% {\sc Th. Katsaounis and A.E.~Tzavaras},
%  Effective equations for localization and shear band formation,
%  {\it SIAM J. Appl. Math.}  {\bf 69} (2009), 1618--1643.

% \bibitem{schecter_undercomp_2002}
% {\sc S.~Schecter},
% {Undercompressive shock waves and the {D}afermos regularization},
% {\it Nonlinearity}
% {\bf 15} (2002), 1361--1377.
% \bibitem{deng_homoclinic_1990}
% {\sc B.~Deng},
% {Homoclinic bifurcations with nonhyperbolic equilibria},
% {\it {SIAM} J. Math. Anal.}
% {\bf 21} (1990),  693--720.
% %   \href{http://dx.doi.org/10.1137/0521037}{doi:\nolinkurl{10.1137/0521037}},
% %   \url{http://dx.doi.org/10.1137/0521037}.

%
% \bibitem{ghazaryan_travelling_2015}
% {\sc A.~Ghazaryan, V.~Manukian, and S.~Schecter},
% {Travelling waves in the holling-tanner model with weak diffusion},
% {\it Proc. R. Soc. A}
% {\bf 471} (2015), 20150045, 16.
% %   \href{http://dx.doi.org/10.1098/rspa.2015.0045}{doi:\nolinkurl{10.1098/rspa.2015.0045}},
% %   \url{http://dx.doi.org/10.1098/rspa.2015.0045}.
%
% \bibitem{gucwa_geometric_2009}
% {\sc I.~Gucwa and P.~Szmolyan},
% {Geometric singular perturbation analysis of an autocatalator model},
% {\it Discrete Contin. Dyn. Syst. Ser. S}
% {\bf 2} (2009), 783--806.
% %   \href{http://dx.doi.org/10.3934/dcdss.2009.2.783}{doi:\nolinkurl{10.3934/dcdss.2009.2.783}},
% %   \url{http://dx.doi.org/10.3934/dcdss.2009.2.783}.
%
% \bibitem{huber_geometric_2005}
% {\sc A.~Huber and P.~Szmolyan},
% {Geometric singular perturbation analysis of the yamada model},
% {\it {SIAM} J. Appl. Dyn. Syst.}
% {\bf 4} (2005), 607--648.
% %   \href{http://dx.doi.org/10.1137/040604820}{doi:\nolinkurl{10.1137/040604820}},
% %   \url{http://dx.doi.org/10.1137/040604820}.
% \bibitem{jones_construction_1991}
% {\sc C.~K. R.~T. Jones, N.~Kopell, and R.~Langer},
% {Construction of the {FitzHugh}-nagumo pulse using differential forms.} In: {\it Patterns and dynamics
%   in reactive media},
% IMA Volumes Math Appl 37, Springer, 1989, 101--115.
%
% \bibitem{jones_tracking_1994}
% {\sc C.~K. R.~T. Jones and N.~Kopell},
% {Tracking invariant manifolds with differential forms in singularly perturbed systems},
% {\it J. Differ. Equations}
% {\bf 108} (1994), 64--88.
% %   \href{http://dx.doi.org/10.1006/jdeq.1994.1025}{doi:\nolinkurl{10.1006/jdeq.1994.1025}},
% %   \url{http://dx.doi.org/10.1006/jdeq.1994.1025}.
% \bibitem{kaper_primer_2001}
% {\sc T.~J. Kaper and C.~K. R.~T. Jones},
% {A primer on the exchange lemma for fast-slow systems.} In: {\it Multiple-time-scale dynamical systems},
% IMA Volumes Math Appl 122, Springer, 1997, 65--87.
% \bibitem{katsaounis_localization_2011}
% {\sc Th. Katsaounis and A.E.~Tzavaras},
% Localization and shear bands in high strain-rate plasticity.
% In: {\it Nonlinear conservation laws and  applications}, A.~Bressan, G.-Q.~Chen, M.~Lewicka, D.~Wang, eds;
% IMA Volumes Math Appl 153, Springer, 2011, 365--377.
% \bibitem{popovic_geometric_2004}
% {\sc N.~Popović and P.~Szmolyan},
% {A geometric analysis of the lagerstrom model problem},
% {\it J. Differ. Equations}
% {\bf 199} (2004), 290--325.
% %   \href{http://dx.doi.org/10.1016/j.jde.2003.08.004}{doi:\nolinkurl{10.1016/j.jde.2003.08.004}},
% %   \url{http://dx.doi.org/10.1016/j.jde.2003.08.004}.
% \bibitem{bates_existence_1997}
% {\sc P.~W. Bates, P.~C. Fife, R.~A. Gardner, and C.~K. R.~T. Jones},
% {The existence of travelling wave solutions of a generalized phase-field model},
% {\it {SIAM} J. Math. Anal.}
% {\bf 28} (1997), 60--93.
% %   \href{http://dx.doi.org/10.1137/S0036141095283820}{doi:\nolinkurl{10.1137/S0036141095283820}},
% %   \url{http://dx.doi.org/10.1137/S0036141095283820}.
%
% \bibitem{beck_electrical_2008}
% {\sc M.~Beck, C.~K. R.~T. Jones, D.~Schaeffer, and M.~Wechselberger},
% {Electrical waves in a one-dimensional model of cardiac tissue},
% {\it {SIAM} J. Appl. Dyn. Syst.}
% {\bf 7} (2008),  1558--1581.
% %   \href{http://dx.doi.org/10.1137/070709980}{doi:\nolinkurl{10.1137/070709980}},
% %   \url{http://dx.doi.org/10.1137/070709980}.

% \bibitem{schecter_exchange_2008}
% {\sc S.~Schecter},
% {Exchange lemmas. i. deng's lemma},
% {\it J. Differ. Equations}
% {\bf 245} (2008), 392--410.
% %   \href{http://dx.doi.org/10.1016/j.jde.2007.08.011}{doi:\nolinkurl{10.1016/j.jde.2007.08.011}},
% %   \url{http://dx.doi.org/10.1016/j.jde.2007.08.011}.
%
% \bibitem{schecter_exchange_2008-1}
% {\sc S.~Schecter}, {Exchange lemmas. {ii}. general exchange lemma},
% {\it J. Differ. Equations}
% {\bf 245} (2008), 411--441.
% %   \href{http://dx.doi.org/10.1016/j.jde.2007.10.021}{doi:\nolinkurl{10.1016/j.jde.2007.10.021}},
% %   \url{http://dx.doi.org/10.1016/j.jde.2007.10.021}.

% \bibitem{wright_stress_1987}
% {\sc Thomas~W. Wright and John~W. Walter},
% On stress collapse in adiabatic shear bands,
% {\it J. Mech. Phys. of Solids} {\bf 35} (1987),
%  701--720.
%
% \bibitem{WF83}
% {\sc F.H. Wu and L.B. Freund},
% Deformation trapping due to thermoplastic instability in one-dimensional wave propagation,
% {\it J. Mech. Phys. of Solids} {\bf  32} (1984), 119-132..
%
% \bibitem{AKS87}
% L.~Anand, K.H.~Kim and T.G.~Shawki,
% Onset of shear localization in viscoplastic solids,
% {\it J. Mech. Phys. Solids}
% {\bf 35} (1987), 407-429.
%
% \bibitem{BC04}
% {\sc Th.~ Baxevanis and N.~Charalambakis},
% The role of material non-homogeneities on the formation and evolution of strain non-uniformities in thermoviscoplastic shearing,
% {\it Quart. Appl. Math.} {\bf 62} (2004), . 97-116.
%
% \bibitem{baxevanis_adaptive_2010}
% {\sc Th~H. Baxevanis, Th~Katsaounis, and A.~E. Tzavaras},
% Adaptive finite element computations of shear band formation,
%   {\it Math. Models  Methods Appl. Sci.} {\bf 20}  (2010),  423--448.
%
% \bibitem{BD02}
% {\sc T.J.~Burns and M.A.~Davies},
% On repeated adiabatic shear band formation during high speed machining,
% {\it International Journal of Plasticity} {\bf 18 } (2002),  507-530.
%
% \bibitem{CB99}
% {\sc L.~ Chen and R.C.~Batra },
% The asymptotic structure of a shear band in mode-II deformations.
% {\it International Journal of Engineering Science} {\bf 37} (1999),  895-919.
%
% \bibitem{estep_2001}
% {\sc Donald~J Estep, Sjoerd M~Verduyn Lunel, and Roy~D Williams},
% {Analysis of Shear Layers in a Fluid with Temperature-Dependent Viscosity},
%  {\it  J. Comp. Physics}  {\bf 173} (2001), 17--60.
%
% \bibitem{KCS85}
% {\sc R.W. Klopp. R.J. Clifton, and T.G. Shawki},
% Pressure-shear impact and the dynamic viscoplastic response of metals,
% {\it Mechanics of Materials} {\bf 4} (1985), 375-385.

% \bibliography{dynamical}
\end{thebibliography}
\end{document}


\end{thebibliography}

\end{document}	

\pagebreak

\begin{proof}

