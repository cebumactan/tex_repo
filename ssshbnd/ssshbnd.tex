%%%%%%%%%%%%%%%%%%%%%%%%%%%%%%%%%%%%%%%%%%%%%%%
%
%   Localization in adiabatic shear flow via geometric theory of singular perturbations
%
%                                                      by
%
%                                       Min-Gi Lee, Th. Katsaounis and A.E. Tzavaras
%
%                                          version Sep 2016
%
%
%%%%%%%%%%%%%%%%%%%%%%%%%%%%%%%%%%%%%%%%%%%%%%%
\documentclass[a4paper,11pt]{article}

\usepackage[margin=2cm]{geometry}
%\usepackage{setspace}
%\onehalfspacing
%\doublespacing
%\usepackage{authblk}
\usepackage{amsmath}
\usepackage{amssymb}
\usepackage{amsthm}
\usepackage{calrsfs}
\usepackage[notcite,notref]{showkeys}

\usepackage{psfrag}
\usepackage{graphicx,subfigure}
\usepackage{color}
\def\red{\color{red}}
\def\blue{\color{blue}}
%\usepackage{verbatim}
% \usepackage{alltt}
%\usepackage{kotex}



\usepackage{enumerate}

%%%%%%%%%%%%%% MY DEFINITIONS %%%%%%%%%%%%%%%%%%%%%%%%%%%

\def\tr{\,\textrm{tr}\,}
\def\div{\,\textrm{div}\,}
\def\sgn{\,\textrm{sgn}\,}

\def\th{\tilde{h}}
\def\tx{\tilde{x}}
\def\tk{\tilde{\kappa}}


\def\bg{{\bar{\gamma}}}
\def\bv{{\bar{v}}}
\def\bth{{\bar{\theta}}}
\def\bs{{\bar{\sigma}}}
\def\bu{{\bar{u}}}
\def\bph{{\bar{\varphi}}}


\def\tg{{\tilde{\gamma}}}
\def\tv{{\tilde{v}}}
\def\tth{{\tilde{\theta}}}
\def\ts{{\tilde{\sigma}}}
\def\tu{{\tilde{u}}}
\def\tph{{\tilde{\varphi}}}

\def\dtg{{\dot{\tilde{\gamma}}}}
\def\dtv{{\dot{\tilde{v}}}}
\def\dtth{{\dot{\tilde{\theta}}}}
\def\dts{{\dot{\tilde{\sigma}}}}
\def\dtu{{\dot{\tilde{u}}}}
\def\dtph{{\dot{\tilde{\varphi}}}}

\def\dpp{\dot{p}}
\def\dqq{\dot{q}}
\def\drr{\dot{r}}
\def\dss{\dot{s}}

\def\ta{{\tilde{a}}}
\def\tb{{\tilde{b}}}
\def\tc{{\tilde{c}}}
\def\td{{\tilde{d}}}

\def\BO{{\mathcal{O}}}
\def\lio{{\mathcal{o}}}



\def\bx{\bar{x}}
\def\bm{\bar{\mathbf{m}}}
\def\K{\mathcal{K}}
\def\E{\mathcal{E}}
\def\del{\partial}
\def\eps{\varepsilon}

\newcommand{\tcr}{\textcolor{red}}
\newcommand{\tcb}{\textcolor{blue}}

\newcommand{\ubar}[1]{\text{\b{$#1$}}}
\newtheorem{theorem}{Theorem}
\newtheorem{lemma}{Lemma}[section]
\newtheorem{proposition}{Proposition}[section]
\newtheorem{corollary}{Corollary}[section]
\newtheorem{definition}{Definition}[section]
\newtheorem{claim}{Claim}

\theoremstyle{remark}
\newtheorem{remark}{Remark}[section]


%%%%%%%%%%%%%%%%%%%%%%%%%%%%%%%%%%%%%%%%%%%%%%%%%%%%%%%%%%
\begin{document}
\title{Localization in adiabatic shear flow \\via geometric theory of singular perturbations}
\author{ Min-Gi Lee\footnotemark[1] \and Theodoros Katsaounis\footnotemark[2] \footnotemark[3]  \and Athanasios E. Tzavaras\footnotemark[4]} %\ \footnotemark[3]  % \footnotemark[4]}
\date{}

\maketitle
\renewcommand{\thefootnote}{\fnsymbol{footnote}}
\footnotetext[1]{Min-Gi Lee \\ King Abdullah University of Science and Technology (KAUST), Computer, Electrical and Mathematical Sciences \& Engineering Division, KAUST, Thuwal, Saudi Arabia,  e-mail: mingi.lee@kaust.edu.sa}%{Computer, Electrical and Mathematical Sciences \& Engineering Division, King Abdullah University of Science and Technology (KAUST), Thuwal, Saudi Arabia}
\footnotetext[2]{Theodoros Katsaounis \\ King Abdullah University of Science and Technology (KAUST), Computer, Electrical and Mathematical Sciences \& Engineering Division, KAUST, Thuwal, Saudi Arabia, e-mail: mingi.lee@kaust.edu.sa}%{Computer, Electrical and Mathematical Sciences \& Engineering Division, King Abdullah University of Science and Technology (KAUST), Thuwal, Saudi Arabia}
\footnotetext[3]{Institute of Applied and Computational Mathematics, FORTH, Heraklion, Greece}
\footnotetext[4]{Athanasios E. Tzavaras \\ King Abdullah University of Science and Technology (KAUST), Computer, Electrical and Mathematical Sciences \& Engineering Division, KAUST, Thuwal, Saudi Arabia,  e-mail: {athanasios.tzavaras@kaust.edu.sa}}
% \footnotetext[2]{Department of Mathematics and Applied Mathematics, University of Crete, Heraklion, Greece}
%\footnotetext[3]{Institute of Applied and Computational Mathematics, FORTH, Heraklion, Greece}
% \footnotetext[4]{Corresponding author : \texttt{athanasios.tzavaras@kaust.edu.sa}}
%\footnotetext[4]{Research supported by the King Abdullah University of Science and Technology (KAUST) }
\renewcommand{\thefootnote}{\arabic{footnote}}


\maketitle


%baseline for working
\baselineskip=18pt

%baseline for submission
%\baselineskip=14pt

\begin{abstract}
We study localization occurring during high speed shear deformations of metals leading to the formation of shear bands.
The localization instability results from the competition among Hadamard instability (caused by softening response) and the stabilizing effects of strain-rate hardening.  We consider a hyperbolic-parabolic system that expresses the above mechanism and construct
self-similar solutions of localizing type that arise as the outcome of the above competition.
The existence of self-similar solutions is turned, via a series of transformations, 
 into a problem of constructing a heteroclinic orbit for an induced dynamical system.
The dynamical system is four dimensional but has a fast-slow structure 
with respect to a small parameter capturing the strength of strain-rate hardening.  
 Geometric singular perturbation theory is applied to construct the heteroclinic orbit as a transversal intersection of two invariant manifolds in the phase space.
\end{abstract}

%\tableofcontents

\section{Introduction}
Shear bands are narrow zones of intensely localized shear that are formed during the high speed shear motion of metals \cite{zener_effect_1944, clifton_rev_1990}. 
They often precede rupture and are one of the striking instances of material instability leading to failure. 
Our main objective is to capture this phenomenon in the simplest possible meaningful framework of one-dimensional thermo-visoco-plasticity. 




\section{Description of the problem and statement of the main results}
{\red new section}


A specimen located in the $xy$-plain undergoes shear motion in the $y$-direction. The motion is described by the (plastic) shear strain
$\gamma(x,t)$, the strain rate $u(x,t) = \gamma_t (x,t)$, the velocity $v(x,t)$ in the shear direction, the temperature $\theta(x,t)$ and the shear stress
$\sigma (x,t)$ all defined in $(t,x)\in \mathbb{R}^+ \times \mathbb{R}$. The shear motion of the thermoviscoplastic material is described by the equations 

\begin{equation} \label{intro-system0}
\begin{aligned}
 \gamma_t &= v_x, && \text{(kinematic compatibility)} \\	%\label{eq:g}\\
 v_t &= \sigma_x, && \text{(momentum equation)} 	\\%\label{eq:v}\\
 \theta_t &= \sigma v_x, && \text{(adiabatic energy equation)}	\\%\label{eq:th}\\
\sigma &= \theta^{-\alpha}\gamma^m (v_x)^n,&& \text{(constitutive law)}				%\label{eq:tau}
\end{aligned}
\end{equation}
They lead to the hyperbolic-parabolic system:
\begin{equation} 
\label{intro-system1}
\begin{aligned}
u_t &= \big ( \theta^{-\alpha}\gamma^m u^n \big )_{xx} , \\%\quad \text{(momentum conservation)} 	\\%\label{eq:v}\\
 \gamma_t &= u,\\%~(\triangleq v_x), \\%\quad \text{(kinematic compatibility)} \\	%\label{eq:g}\\
 \theta_t &= \theta^{-\alpha}\gamma^m u^{n+1} .\\% \quad \text{(energy conservation)}	\\%\label{eq:th}\\
% \sigma &=\theta^{-\alpha}\gamma^m u^n.			%\label{eq:tau}
\end{aligned}
\end{equation}
We refer to \cite{KT09} for a derivation of the non-dimensional forms \eqref{intro-system0} and \eqref{intro-system1} and to \cite{clifton_rev_1990,shawki_shear_1989,wright_survey_2002} for further information on the mechanical aspects
of the model.

The constitutive law $\sigma = \theta^{-\alpha}\gamma^m u^n$ characterizes, in the form of an empirical power law, the nature of the material. The parameter $\alpha>0$ 
measures the degree of the thermal softening, $m>0$ measures that of strain hardening (or $m<0$ in case of a softening plastic flow), and $n>0$ measures the degree of
strain-rate hardening and is assumed to be small, $n \ll 1$. The system \eqref{intro-system0} captures the simplest mechanism proposed for shear localization in
high-speed deformations of metals \cite{zener_effect_1944, clifton_rev_1990}, and an (isothermal) variant appears in early studies of necking \cite{HN77}. There is an extensive mechanics literature on the problem,  {\it e.g.} \cite{shawki_shear_1989,clifton_rev_1990, wright_survey_2002}, and early mathematical results 
appear in \cite{tzavaras_plastic_1986, tzavaras_nonlinear_1992} and \cite{Tz_1987} where a proof of shear band formation appears. 
The model with $m=0$ corresponds to shear of a viscous temperature-dependent fluid and stability or instability results of the model appear in \cite{bertsch_effect_1991, DH_1983, Tz_1986, Tz_1987, KOT14}. The model with $\alpha=0$ corresponds to isothermal plasticity and localizing self-similar solutions were obtained for this model in \cite{LT16, KLT_2016}. The above studies concern simplified cases of \eqref{intro-system1} with a reduced number of equations; by contrast, here we consider the full model.


% From the mathematics literature,
%
%
%
%
% see \cite{tzavaras_plastic_1986,tzavaras_strain_1991,tzavaras_nonlinear_1992}, and \cite{bertsch_effect_1991, DH_1983, Tz_1986, Tz_1987, KOT14, KLT_HYP2016} (in the study of rectlinear viscous fluid), and in the study of its isothermal \cite{LT16, KLT_2016}.
% }

Our objective is to construct a family of localizing solutions for \eqref{intro-system0} of self-similar form
\begin{equation} \label{intro-sols}
\begin{aligned}
 \bg(t,x) &= (t+1)^a\Gamma\big((t+1)^\lambda x\big), & \bv(t,x) &= (t+1)^bV\big((t+1)^\lambda x\big), & \bth(t,x) &= (t+1)^c\Theta\big((t+1)^\lambda x\big),\\
 \bar{\sigma}(t,x) &= (t+1)^d\Sigma\big((t+1)^\lambda x\big), & \bu(t,x) &= (t+1)^{a-1}U\big((t+1)^\lambda x\big), 
\end{aligned}
\end{equation}
with exponents $a$, $b$, $c$, $d$ as in \eqref{eq:exponents} exploiting scale invariance properties of \eqref{intro-system1}. The parameter $\lambda>0$ accounts for the rate of localization, and $\xi=(t+1)^\lambda x$ stands for the self-similar variable. 
The idea of self-similar localizing solutions was first introduced in \cite{KOT14}.   Note that the exponent $\lambda$ is taken of the opposite sign to that typical for parabolic problems, thus leading to localizing solutions.

The model \eqref{intro-system0} admits a class of {\it uniform shearing solutions} independently of the exponents $(\alpha,m,n)$, emerging from constant initial
data $\gamma_0$ and $\theta_0$, and  reading
\begin{equation} \label{intro:uss}
 \begin{aligned}
 \gamma = t+\gamma_0, \quad v&=x, \quad u=1, \quad \theta = \left( \frac{1+\alpha}{1+m} \big( (t+\gamma_0)^{1+m}-\gamma_0^{1+m}\big) + \theta_0^{1+\alpha}\right)^{\frac{1}{1+\alpha}}, \\
 \sigma&=\left( \frac{1+\alpha}{1+m} \big( (t+\gamma_0)^{1+m}-\gamma_0^{1+m}\big) + \theta_0^{1+\alpha} \right)^{-\frac{\alpha}{1+\alpha}}(t+\gamma_0)^m.
 \end{aligned}
\end{equation}
Uniform shear should be contrasted to the solutions that will be constructed here which exhibit localizing instability.
 In the latter case, the growth of the strain is superlinear at the origin and the $x-t$-profile localizes in a narrow zone.% in the same time.

%Widely accepted explanation (\cite{shawki_shear_1989,clifton_rev_1990}) why such a localizing instability can take place is the following positive feedback scenario: the material keeps loading and
%this mechanical working causes the increases in temperature. Due to the imperfectness of the deformation, the small non-uniformities in strain presences and this results in the non-uniformities in the heat production. When the speed of deformation is so high such that the time needed for the heat to diffuse out is not sufficient, then the process becomes effectively {\it adiabatic}, i.e., the heat produced at a spot accumulates. Metals typically thermally soften and strain hardens. For such a metal where the thermal softening outweighs the hardening response, the net response is softening and the spots that had larger deformation than surrounding becomes even easier to deform. This, in turn, gives rise to the severer non-uniformities in strain, triggering the positive feedback mechanism.

Note that, for $n=0$, \eqref{intro-system0} is a first order system. The due cause for the instability of \eqref{intro:uss} is the loss of hyperbolicity of the system occurring when $n=0$.  As illustrated in Appendix \ref{append:hadamard}, along the uniform shearing solution the model is an initial value problem of elliptic equations past a critical threshold for the parameter range $-\alpha+m<0$. This leads one to expect catastrophic (exponential) growth of oscillations in the initial value problem, what has been termed {\it Hadamard Instability}. However, as pointed out in \cite{KOT14},  Hadamard Instability cannot by itself explain the formation of shear bands, 
as what is observed in the process is the orderly development of localization \cite{zener_effect_1944} rather than the arbitrary growth of oscillations. The regularizing effect of small viscosity ($0 < n \ll 1$) competes with the Hadamard instability.
% Opposed to Hadamard Instability, lies the regularizing effect of small viscosity ($0 < n \ll 1$).
When $n$ is sufficiently large, the diffusive mechanism dominates and localization is suppressed, see \cite{DH_1983, Tz_1986, tzavaras_nonlinear_1992}. 
A conjecture as to where the threshold of instability occurs is provided by the asymptotic analysis in  \cite{KT09}, devising
an effective equation that changes type along a threshold from forward to backward parabolic.
 It leads to expect instability when $-\alpha+m+n$ changes sign from positive to negative value. Indeed,
this conjecture will be validated here and localising solutions are constructed in the regime $-\alpha+m+n<0$.

This article is organized as follows: 
Sections \ref{sec:scale} and  \ref{sec:formulation} deal with the formulation of the problem. A series of techniques and transformations were devised in
 \cite{KOT14} and successfully adapted to the strain independent model ($\sigma=\varphi(\theta)u^n$) \cite{KLT_HYP2016}, 
 and to the temperature independent model ($\sigma=\varphi(\gamma)u^n$) \cite{LT16,KLT_2016}. 
 They are furher adapted to the larger sytem of equations here. Applying the {\it ansatz} \eqref{intro-sols} leads to the system of 
 singular ordinary differential equations
\begin{equation} \label{intro:ss-odes}
\begin{aligned}
 a \Gamma(\xi) + \lambda \xi \Gamma'(\xi) &= U(\xi), \\
 b V(\xi) + \lambda \xi V'(\xi) &= \Sigma'(\xi), \\
 c \Theta(\xi) + \lambda \xi \Theta'(\xi)&=\Sigma(\xi) U(\xi),\\
 \Sigma(\xi) &= \Theta(\xi)^{-\alpha} \Gamma(\xi)^m U(\xi)^n, \\
 V'(\xi)&=U(\xi),\\
 \Gamma(0)=\Gamma_0>0, \quad U(0)&=U_0>0, \quad \text{$\xi \in [0,\infty)$},
\end{aligned}
\end{equation}
where, we look for $(\Gamma,\Theta)$ that is even, and $V$ that is odd (therefore $U$ and $\Sigma$ are even). We impose
\begin{equation}
 V(0)=U'(0)=\Gamma'(0)=\Sigma'(0)=\Theta'(0)=0. \label{intro:bdry0}
\end{equation}
so that the symmetric extensions are regular for the self-similar profiles.

The system \eqref{intro:ss-odes} is singular (at $\xi=0$) and non-autonomous and as such it does not fit under a general existence theory.
Nevertheless,  the singularity in \eqref{intro:ss-odes} can be resolved and the system desingularized. Furthermore, upon introducing a series of nonlinear transformations,  the construction of profiles for \eqref{intro:ss-odes} is turned to the
construction of a heteroclinic orbit for the four-dimensional dynamical system for $(p,q,r,s)$ 
\begin{equation}\label{intro:slow}\tag{S}
 \begin{aligned}
 \dot{p} &=p\Big(\frac{1}{\lambda}(r-a) + 2- \lambda p r -q\Big), \\
 \dot{q} &=q\Big(1 -\lambda p r -q\Big) + b p r,\\
 n\dot{r} &=r\Big(\frac{\alpha-m-n}{\lambda(1+\alpha)}(r-a) + \lambda pr + q +\frac{\alpha}{\lambda}r\big(s- \frac{1+m+n}{1+\alpha}\big) + \frac{n\alpha}{\lambda(1+\alpha)}\Big),\\
 \dot{s} &=s\Big(\frac{\alpha-m-n}{\lambda(1+\alpha)}(r-a) + \lambda pr + q - \frac{1}{\lambda}r\big(s- \frac{1+m+n}{1+\alpha}\big) - \frac{n}{\lambda(1+\alpha)}\Big),
 \end{aligned}
\end{equation}
parametrized by $(\lambda,\alpha,m,n)$. The initial conditions \eqref{intro:bdry0} are transmitted to asymptotic conditions for the heteroclinic
as $\eta(=\log\xi) \to -\infty$ while the behavior as $\eta \to \infty$ captures the asymptotic behavior of the profiles.


In Section \ref{sec:equil} we give two relevant equilibria and in the following Sections \ref{sec:char} and \ref{sec:proof}, we turn to the existence of the heteroclinic orbit. In Section  \ref{sec:equil} are listed
the equilibria of  \eqref{intro:slow} together with the linearized flow around them in phase space. We single out the heteroclinic orbit by imposing the asymptotic conditions based on our expectation and \eqref{intro:bdry0} at two end points of the orbit in Section \ref{sec:char}.

The heteroclinic is constructed in Section \ref{sec:proof}.
The smallness of $n$ is exploited throughout the analysis. Its implications are twofold, one is singular and the other is regular. 
Note that we expect that in the limit $n=0$ only two variables (out of the original four) will play a role.
First, $\eqref{intro-system0}_3$ can be re-written as
$$ \partial_t\Big(\frac{\theta^{1+\alpha}}{1+\alpha}\Big) = \big(\gamma_t^n\big)\partial_t\Big(\frac{\gamma^{1+m}}{1+m}\Big) \, , $$
from which one can see that the equation at $n=0$ is integrable and we can eliminate one variable modulo initial data, specifically $\frac{\theta^{1+\alpha}}{1+\alpha} = \frac{\gamma^{1+m}}{1+m} + f(x)$ for some $f(x)$. For $n>0$, while we cannot anymore integrate and write one variable in terms of the other, the relationship turns out to be persistent for small $n$.
Second, the constitutive law 
$\sigma=\theta^{-\alpha}\gamma^m(v_x)^n$ defines the variable $\sigma$. At $n=0$, this becomes an algebraic equation and if further $\frac{\theta^{1+\alpha}}{1+\alpha} = \frac{\gamma^{1+m}}{1+m}$, we have $\sigma\gamma^{\tfrac{\alpha-m}{1+\alpha}}=\left(\tfrac{1+\alpha}{1+m}\right)^{\tfrac{-\alpha}{1+\alpha}} $. As the manifold appears as a level set of a nonlinear function of $\gamma$ and $\sigma$, it is convenient to have $\sigma$ in the prolonged state space of 4-tuple $(\gamma,v,\theta,\sigma)$ and make uses of the nonlinear function in analyzing the phase space. At $n>0$ the algebraic equation turns into a differential equation and, restricted to the study of self-similar solutions, the propagation of the constitutive law as a constraint equation turns out to be a relaxation process towards a slow manifold. The corresponding slow manifold for $n>0$ will be captured in Section \ref{sec:singpert}. 

The above mechanism is made apparent in the $(p,q,r,s)$-system \eqref{intro:slow} since for small $n$ \eqref{intro:slow} can be viewed as a {\it fast-slow} system with a three dimensional manifold that again has a nested two dimensional invariant manifold within it. The fast-slow structure gives the persistence of the three dimensional slow manifold for small $n$. The finer persistence properties that are needed further are provided by {\it Geometric singular perturbation theory}. To achieve the heteroclinic orbit by applying the geometric singular perturbation theory is done with detail in Section \ref{sec:proof}. More advanced results on Geometric singular perturbation theory can be found in \cite{fenichel_asymptotic_1974, fenichel_asymptotic_1977,fenichel_geometric_1979,HPS_1977,Sz1991,wiggins_normally_1994,Jones_1995,KUEHN_2015}.
In section \ref{sec:localization}, we describe the emergence of localization in the constructed two-parameter family of solutions via asymptotic analysis. The numerical computations of the solution, which have been described in \cite{KLT_HYP2016}, are provided again here for illustrative purposes. 

The emerging self-similar solutions depend on two parameters $(U(0), \Gamma (0))$ describing the initial nonuniformity; the rate of localization $\lambda$ is 
in turn determined by \eqref{eq:lambda} and, due to the construction necessities, has to obey the bound \eqref{eq:lambda}.
They provide an example of instabilty resulting to localization. To our knowldge, they are the first instance depicting such behavior for
a sufficiently broad model \eqref{intro-system0} or \eqref{intro-system1} that includes the key contributing factors 
of thermal softening, strain hardening and strain-rate hardening, that entering in the 
basic mechanism of shear band formation proposed by Zener and Hollomon \cite{zener_effect_1944} and Clifton \cite{clifton_rev_1990}. 
They complement the original proof of shear band formation \cite{Tz_1987}, where shear band formation is shown but in a set-up where energy is supplied via the boundary. 
Some of the key predictions of stress-collapse are common in both results, but the present has the conceptual 
advantage to capture the emergence of localization as the combined result
of Hadamard instability with small viscosity effects.

%It is not clear at this point how the self-similar solutions will play a role in a composite deformation
%or will interact with boundary effects. This will be the subject of a future work. It is however, notable that the two independent parameters that enter in
%the solution the size of the nonuniformity $(\Gamma_0, U_0)$ and the rate $\lambda$ are interconnected through




\vfil\eject

\section{Self-similar solutions} \label{sec:scale}

%\subsection{Scale invariance properties}
The system \eqref{intro-system1} has a scale invariance property: If $(\gamma,u,v,\theta,\sigma)$ is a solution, then the rescaled function 
$(\gamma_\rho,u_\rho,v_\rho,\theta_\rho,\sigma_\rho)$ defined by
\begin{equation}\label{eq:scale}
\begin{aligned}
 \gamma_\rho(t,x) &= \rho^a\gamma(\rho^{-1}t,\rho^\lambda x), &
 v_\rho(t,x) &= \rho^bv(\rho^{-1}t,\rho^\lambda x),\\
 \theta_\rho(t,x) &= \rho^c\theta(\rho^{-1}t,\rho^\lambda x), &
 \sigma_\rho(t,x) &= \rho^d\sigma(\rho^{-1}t,\rho^\lambda x),\\
 u_\rho(t,x) &= \rho^{b+\lambda}\gamma(\rho^{-1}t,\rho^\lambda x)
\end{aligned}
\end{equation}
is again a solution, provided
\begin{equation} \label{eq:exponents}
\begin{aligned}
 a&:= a_0 + a_1 \lambda=\frac{2+2\alpha-n}{D} + \frac{2(1 + \alpha)}{D}\lambda, & b&:=b_0 + b_1\lambda=\frac{1+m}{D} + \frac{1+m+n}{D}\lambda ,\\
 c&:=c_0 + c_1\lambda=\frac{2(1+m)}{D} + \frac{2(1+m+n)}{D}\lambda, & d&:=d_0 + d_1\lambda=\frac{-2\alpha + 2m +n}{D} + \frac{2(-\alpha+m+n)}{D}\lambda,
\end{aligned}
\end{equation}
for each $\lambda \in \mathbb{R}$, with the denominator 
\begin{equation}
\label{defD}
D = 1+2\alpha-m-n \, .
\end{equation}
Throught this paper, the material parameters $(\alpha,m,n)$ run only through the ranges
\begin{equation}
 \begin{aligned}
  \alpha>0\quad&\text{(thermal softening)},\\
  m>-1 \quad&\text{(strain softening/hardening)}, \\%\label{eq:a1}\\
  n>0 \quad&\text{(strain rate sensitivity)},\\ %\label{eq:a2}\\
  -\alpha+m+n<0 \quad&\text{(net softening)}. \\%\label{eq:a3}\\
%   0< \lambda < \frac{2(\alpha-m-n)}{1+m+n}\left(\frac{1+m}{1+m+n}\right) \quad&\text{(localizing rate bound)}. %\label{eq:a4}
\end{aligned}\label{eq:paramrange}
\end{equation}
that are associated to physical properties of the problem at hand (in parentheses).
In this regime we note $D>1+\alpha>1$. Having $\lambda$ negative is the typical scaling used in parabolic problems and capturing the effect
of diffusion.
Here, we are interested in the opposite behavior, of localization, and throughout this work we restrict attention to the opposite range $\lambda>0$. 
%To have an upper bound for the localizing rate $\lambda$ will be reasoned in Section \ref{sec:equil}.

%\subsection{Self-Similar solutions}
Motivated by the scale invariance property, we seek for $\lambda > 0$ solutions of the form
\begin{equation}\label{eq:ORItoCAP}
\begin{aligned}
 \gamma(t,x) &= t^a\Gamma(t^\lambda x), & v(t,x) &= t^b V(t^\lambda x), &\theta(t,x) &= t^c \Theta(t^\lambda x),\\
 \sigma(t,x) &= t^d \Sigma(t^\lambda x), & u(t,x) &= t^{b+\lambda} U(t^\lambda x) \, , 
\end{aligned}
\end{equation}
setting $\xi = t^\lambda x$. Plugging the ansatz into the system \eqref{intro-system0} gives the system of ordinary differential and algebraic equations for $\big(\Gamma(\xi), V(\xi), \Theta(\xi), \Sigma(\xi), U(\xi)\big)$ :

\begin{equation}
\begin{aligned}
 a \Gamma(\xi) + \lambda \xi \Gamma'(\xi) &= U(\xi),\\
 b V(\xi) + \lambda \xi V'(\xi) &= \Sigma'(\xi),\\
 c \Theta(\xi) + \lambda \xi \Theta'(\xi)&=\Sigma(\xi) U(\xi),\\
 \Sigma(\xi) &= \Theta(\xi)^{-\alpha} \Gamma(\xi)^m U(\xi)^n,\\
 V'(\xi)&=U(\xi).
\end{aligned} \label{eq:ss-odes}
\end{equation}
Note that the uniform shear solution is attained as a self-similar solution at $\lambda = -\frac{1+m}{2(1+\alpha)}<0$, with
\begin{equation*}
 \Gamma(\xi) = U(\xi)=U_0, \quad V(\xi) = U_0\xi, \quad  \Theta(\xi) = \Big( \frac{1+\alpha}{1+m} U_0^{1+m+n}\Big)^{\frac{1}{1+\alpha}}, \quad \Sigma(\xi) = \Big(\frac{1+\alpha}{1+m}\Big)^{\frac{-\alpha}{1+\alpha}} U_0^{\frac{-\alpha+m+n}{1+\alpha}}.
\end{equation*}
The system \eqref{eq:ss-odes} is non-autonomous and  singular at $\xi=0$.
%Existence or non-existence of the singular ordinary differential equations differs from cases to cases and requires case-by-case study. %and results depend on the characteristics of the problem.
In the sequel, we de-singularize \eqref{eq:ss-odes} and turn it into an autonomous system.



\vfil\eject

\section{Reduction to the construction of a heteroclinic orbit} \label{sec:formulation}
The goal of this section is to derive an equivalent system \eqref{eq:slow} to \eqref{eq:ss-odes} that is autonomous and to turn the problem of
constructing profiles for \eqref{eq:ss-odes} to the construction of a heteroclinic orbit for \eqref{eq:slow}. We employ techniques from \cite{KOT14}, where
a temperature dependent viscous stress with exponential law $ \sigma = \mu(\theta)u^n = e^{-\alpha\theta} u^n$ is studied. However, the present
analysis and especially the construction of the heteroclinic orbit is quite more complicated here.


\subsection{De-singularization}
Observe that if $\big(\Gamma(\xi), V(\xi), \Theta(\xi), \Sigma(\xi), U(\xi)\big)$ solves \eqref{eq:ss-odes}, then 
$\big(\Gamma(-\xi), -V(-\xi), \Theta(-\xi), \Sigma(-\xi), U(-\xi)\big)$ is also a solution. 
Using this symmetry, we will look for self-similar profiles such that $\Gamma(\xi)$, $\Theta(\xi)$, $\Sigma(\xi)$, $U(\xi)$ are even functions of $\xi$, and $V(\xi)$ is an odd function of $\xi$. Accordingly, we impose the conditions
\begin{equation}
 V(0)=U'(0)=\Gamma'(0)=\Sigma'(0)=\Theta'(0)=0 \label{eq:bdry0}
\end{equation}
at the origin. We regard \eqref{eq:ss-odes} as a boundary-value problem in the right half-line $\xi \in [0,\infty)$ subject to the boundary conditions \eqref{eq:bdry0}. Because the system \eqref{eq:ss-odes} is singular, it is not clear in advance how many conditions are needed to single out the solution. We will come to this point later in Section \ref{sec:char}.%, we also append the far field condition as $\xi \rightarrow \infty$ and yet another boundary conditions at $\xi=0$, detailing in the issue of fixing the unique solution, but for the moment we work upon \eqref{eq:bdry0}.

The system \eqref{eq:ss-odes} is itself scale invariant: Given a solution $\big(\Gamma(\xi), V(\xi), \Theta(\xi), \Sigma(\xi), U(\xi)\big)$ the rescaled
function $\big(\Gamma_\rho(\xi), V_\rho(\xi), \Theta_\rho(\xi), \Sigma_\rho(\xi), U_\rho(\xi)\big)$ defined by
\begin{equation}
\label{selfsimilardef2}
\begin{aligned}
 \Gamma_\rho(\xi)&=\rho^{a_1}\Gamma(\rho\xi), & V_\rho(\xi)&=\rho^{b_1}V(\rho\xi), & \Theta_\rho(\xi)&=\rho^{c_1}\Theta(\rho\xi),\\
 \Sigma_\rho(\xi)&=\rho^{d_1}\Sigma(\rho\xi), & U_\rho(\xi)&=\rho^{b_1+1}U(\rho\xi)=\rho^{a_1}U(\rho\xi)
\end{aligned}
\end{equation}
is again a solution. 
A particular self-similar solution of of this type would have the form
 $$\big(\Gamma(\xi), V(\xi), \Theta(\xi), \Sigma(\xi), U(\xi)\big)=\big(A\xi^{-a_1}, B\xi^{-b_1},C\xi^{-c_1},D\xi^{-d_1},E\xi^{-a_1}\big)$$ 
with $A, B, C, D, E$ appropriate constants. Such a solution is singular at $\xi =0$ and fails to fulfill \eqref{eq:bdry0}. This suggests considering the change of variables
\begin{equation} \label{eq:CAPtoBAR}
\begin{aligned}
 \bg(\xi)&=\xi^{a_1}\Gamma(\xi), &
 \bv(\xi)&=\xi^{b_1}V(\xi), &
 \bth(\xi)&=\xi^{c_1}\Theta(\xi), \\
 \bs(\xi)&=\xi^{d_1}\Sigma(\xi), &
 \bu(\xi)&=\xi^{b_1+1}U(\xi) ,
\end{aligned}
\end{equation}
with $a_1, b_1, c_1$ and $d_1$ as in \eqref{eq:exponents}, in order to de-singularize the problem.
After some cumbersome, but straightforward calculation, we find that $(\bg,\bv,\bth,\bs,\bu)$ satisfies
% These variables result in a nice property;
\begin{equation} \label{eq:barsys}
 \begin{aligned}
  a_0\bg + \lambda\xi\bg' &=\bu,\\
  b_0\bv + \lambda\xi\bv' &=-d_1 \bs + \xi\bs',\\
  c_0\bth+ \lambda\xi\bth'&=\bs\bu,\\
  \ts &=\bth^{-\alpha}\bg^m\bu^n,\\
  -b_1\bv+\xi\bv' &= \bu.
 \end{aligned}
\end{equation}
Next, introduce a new independent variable $\eta = \log\xi$ and define $(\tg,\tv,\tth,\ts,\tu)$ by
\begin{equation} \label{eq:BARtoTIL}
\begin{aligned}
 \tg(\log\xi)&=\bg(\xi), &
 \tv(\log\xi)&=\bv(\xi), &
 \tth(\log\xi)&=\bth(\xi), \\
 \ts(\log\xi)&=\bs(\xi), &
 \tu(\log\xi)&=\bu(\xi).
\end{aligned}
\end{equation}
Noticing that $\frac{d}{d\eta}\tg(\eta) = \xi \frac{d}{d\xi}\bg(\xi)$, we obtain an autonomous system
\begin{equation} \label{eq:tildesys}
 \begin{aligned}
  a_0\tg + \lambda\dtg &=\tu,\\
  b_0\tv + \lambda\dtv &=-d_1 \ts + \dts,\\
  c_0\tth+ \lambda\dtth&=\ts\tu,\\
  \ts &=\tth^{-\alpha}\tg^m\tu^n.\\
  -b_1\tv+\dtv &= \tu,
 \end{aligned}
\end{equation}
where the notation $\dot{f}=\frac{df}{d\eta}$ is used.

The system \eqref{eq:tildesys} is autonomous and one might attempt to consider its equilibria. 
However, it is easy to conclude that we cannot expect a heteroclinic that  tends to equilibria of \eqref{eq:tildesys}.
Indeed, suppose $\tu \rightarrow \tu_\infty\ge 0$ as $\eta \rightarrow \infty$. 
Then from the last equation in \eqref{eq:tildesys}, 
we conclude that $\tv \rightarrow \infty$. % and thus also $\ts \rightarrow \infty$ by the second equation. 
This suggests to enlarge the scope and consider solutions that grow as polynomials (or faster) at infinities.




\subsection{The $(p,q,r,s)$-system derivation}
Next, we attempt to come up with a new choice of variables that tend to equilibria as $\eta \rightarrow \pm \infty$ and accommodate
orbits that have power behavior at infinities. We rewrite \eqref{eq:tildesys} in the form
\begin{equation}
 \label{eq:tildesys2}
\begin{aligned}
\frac{d}{d\eta}{(\ln{\tg})}  &=  \tfrac{1}{\lambda} \big (- a_0 +  \frac{\tu}{\tg} \big ),
\\
 \frac{d}{d\eta}{(\ln{\tv})}  &=  {\red b_1 + \frac{\tu}{\tv} , }
\\
\frac{d}{d\eta}{(\ln{\tth})} &=   \tfrac{1}{\lambda} \big (- c_0 +  \frac{\ts \tu}{\tth} \big ), 
\\
\frac{d}{d\eta}{( {\red \ln{\ts} } )} &= d_1 + b \frac{\tv}{\ts} +  {\red  \lambda \frac{\tu}{\ts}  }
\end{aligned}
\end{equation}
and view it as describing the evolution of $(\tg,\tv,\tth,\ts)$ with $\tu$ determined by $\tu = \left ( \frac{\ts}{ \tth^{-\alpha} \tg^m} \right )^\frac{1}{n}$.

This leads us to define
\begin{equation}\label{eq:pqrdef} 
 \begin{aligned}
  p :=\frac{\tg}{\ts}, \quad q :=b \frac{\tv}{\ts},  \quad r = \frac{\tu}{\tg} = \left ( \frac{\ts}{ \tth^{-\alpha} \tg^{m+n}} \right )^\frac{1}{n}  , \quad s := \frac{\ts\tg}{\tth} \, .
 \end{aligned}
\end{equation}
The transformation $(p,q,r,s) \leftrightarrow (\tg,\tv, \tth,\ts)$ is a bijection in the first quadrant with the inverse determined by
$$
\tg = p^\frac{1+\alpha}{D} s^\frac{\alpha}{D} r^\frac{n}{D} \, \quad \tth = p^\frac{1+m+n}{D} s^\frac{m+n-1}{D} r^\frac{2n}{D}
$$
and then
$$
\sigma = \frac{1}{\tg} p \, , \quad v = \frac{1}{b} \sigma \, q
$$
Using \eqref{eq:tildesys2} and \eqref{eq:pqrdef}, we write
\begin{align*}
 \frac{\dpp}{p}&=\frac{\dtg}{\tg} - \frac{\dts}{\ts}& &=\left[\frac{1}{\lambda }\Big(\frac{\tu}{\tg}-a_0\Big)\right] & &-\left[d_1 + b\frac{\tv}{\ts} + 
 \lambda   {\red  \frac{\tu}{\tg} \frac{\tg}{\ts}   }  \right]
 \\
 \frac{\dqq}{q}&=\frac{\dtv}{\tv} - \frac{\dts}{\ts}& &=\left[b_1 +\frac{\tu}{\tv}\right] & &-\left[d_1 + b\frac{\tv}{\ts} + \lambda  {\red  \frac{\tu}{\tg} \frac{\tg}{\ts}   }   \right]
 \\
 n\frac{\drr}{r}&=-(m+n)\frac{\dtg}{\tg}+\frac{\dts}{\ts} + \alpha\frac{\dtth}{\tth} & &=\left[\frac{-(m+n)}{\lambda}\Big(\frac{\tu}{\tg}-a_0\Big)\right]& &+\left[d_1 + b\frac{\tv}{\ts} + \lambda  {\red  \frac{\tu}{\tg} \frac{\tg}{\ts} }  \right] + \left[\frac{\alpha}{\lambda }\Big(\frac{\ts\tu}{\tth}-c_0\Big)\right]\\
 \frac{\dot{s}}{s} &= \frac{\dtg}{\tg} + \frac{\dts}{\ts} - \frac{\dtth}{\tth} & &=\left[\frac{1}{\lambda }\Big(\frac{\tu}{\tg}-a_0\Big)\right] & &+\left[d_1 + b\frac{\tv}{\ts} 
 + \lambda  {\red  \frac{\tu}{\tg} \frac{\tg}{\ts} }  \right] -\left[\frac{1}{\lambda }\Big(\frac{\ts\tu}{\tth}-c_0\Big)\right].%\\
%  \dot{s} &=\frac{\partial s}{\partial (z-1)} \dot{z} &&= \frac{1+m}{\lambda}\frac{\partial s}{\partial (z-1)} \bigg\{z\big[-r - \frac{n}{D}\Big]+ ru^n\bigg\}.
\end{align*}
We note that
\begin{align*}
 \frac{\ts\tu}{\tth} = rs, \quad \frac{\tu}{\tv} = \frac{bpr}{q}, \quad \frac{\tu}{\ts} = pr,
\end{align*}
and using \eqref{eq:exponents} and \eqref{defD}, after a cumbersome but straightforward calculation, we derive the $(p,q,r,s)$-system:
\begin{equation}\label{eq:slow} \tag{S}
 \begin{aligned}
 \dot{p} &=p\Big(\frac{1}{\lambda}(r-a) + 2- \lambda p r -q\Big),\\
 \dot{q} &=q\Big(1 -\lambda p r -q\Big) + b p r,\\
 n\dot{r} &=r\Big(\frac{\alpha-m-n}{\lambda(1+\alpha)}(r-a) + \lambda pr + q +\frac{\alpha}{\lambda}r\big(s- \frac{1+m+n}{1+\alpha}\big) + \frac{n\alpha}{\lambda(1+\alpha)}\Big),\\
 \dot{s} &=s\Big(\frac{\alpha-m-n}{\lambda(1+\alpha)}(r-a) + \lambda pr + q - \frac{1}{\lambda}r\big(s- \frac{1+m+n}{1+\alpha}\big) - \frac{n}{\lambda(1+\alpha)}\Big).
 \end{aligned}
\end{equation}
In the sequel, we analyze \eqref{eq:slow} as an autonomous system: We begin with sorting its equilibria and analyzing their linear stability. Most importantly, \eqref{eq:slow} possesses the {\it fast-slow} structure because of the small parameter $n$ in the left-hand-side of $\eqref{eq:slow}_3$; the dynamics of $r$ can be distinctively faster than those of the other variables.



\vfil\eject

\section{Equilibria and their linear stability} \label{sec:equil}

System \eqref{eq:slow} admits several equilibria listed in the Appendix \ref{append:lin}. Our region of interest is the sector $\{(p,q,r,s) \; | \; p\ge0, q\ge0, r>0, s>0 \}$. That $p,q\ge0$ comes from the requirement that $\tg,\tv,\ts\ge0$. The reason we reject the region $r,s\le0$ stems from mechanical considerations: If we transform back to the original variables, then we find
\begin{equation*}
 r(\eta)|_{\eta=t^\lambda x}=t\partial_t\log \gamma(t,x), \quad r(\eta)s(\eta)|_{\eta=t^\lambda x}=t\partial_t \log \theta(t,x).
\end{equation*}
%for the original variables $\theta(t,x)$ and $\gamma(t,x)$.
Shear band initiation is related to conditions of loading where both the plastic strain and the temperature are increasing. This motivates to restrict on the region $r > 0$, $s > 0$. While all of the equilibrium points of \eqref{eq:slow} are listed in the Appendix \ref{append:lin}, it shows that at most two equilibria can reside in this region:
\begin{align*}
 M_0 &= (0,0,r_0,s_0), & r_0 &=\frac{2+2\alpha-n}{D} + \frac{2+2\alpha}{D}\lambda, & s_0&=\frac{1+m+n}{1+\alpha} - \frac{n}{(1+\alpha)r_0},\\
 M_1 &= (0,1,r_1,s_1), & r_1 &= r_0-\frac{1+\alpha}{\alpha-m-n}\lambda, & s_1&=\frac{1+m+n}{1+\alpha} - \frac{n}{(1+\alpha)r_1}.
\end{align*}
We find $r_0>0$ and $r_0s_0 = \frac{2(1+m)}{D} + \frac{2(1+m+n)}{D}\lambda>0$, or $r_0, s_0>0$. Thus $M_0$ always resides in the region. However, $M_1$ can be out of the region $r>0$, $s>0$ if $\lambda$ is large enough. Note that $r_1,s_1>0$ only if $\frac{1+m+n}{1+\alpha}r_1 > \frac{n}{(1+\alpha)}$. This reads
$$\frac{1+m+n}{1+\alpha}\Big(\frac{2+2\alpha-n}{D} - \frac{(1+\alpha)(1+m+n)}{D(\alpha-m-n)}\lambda\Big) > \frac{n}{(1+\alpha)},$$
% $\lambda$ is not arbitrarily large but in the range %in this region only under the constraint that
and thus $M_1$ resides in the region only under the constraint
\begin{equation} \label{eq:lambda-range}
 0< \lambda < \frac{2(\alpha-m-n)}{1+m+n}\left(\frac{1+m}{1+m+n}\right).
\end{equation}
Henceforth, we restrict attention to rates $\lambda$ satisfying \eqref{eq:lambda-range}.




% \eqref{eq:slow} admits several equilibria listed in the Appendix \ref{append:lin}. Our region of interest is the sector $\{(p,q,r,s) \; | \; p\ge0, q\ge0, r\ge0, s\ge0 \}$. We will establish later, in section \ref{sec:proof}, 
% that the three dimensional submanifold $K$ defined by \eqref{eq:implicit} 
% is positively invariant and $r$ and $s$ have strictly positive lower bounds on it. 
% %is contained in the narrower region $I \triangleq\left\{ \: (p,q,r,s) \: | \:  p,q\ge0, ~~ r,s\ge\delta>0\right\}$.
% % \begin{align*}
% %  \left\{ \: (p,q,r,s) \: | \:  p,q\ge0, ~~ r,s\ge\delta>0, ~~\left|s-\frac{1+m}{1+\alpha}\right| \le A \frac{\alpha-m}{\alpha(1+\alpha)}\: \right\}.
% % \end{align*}
% For the moment we examine the two equilibrium points $M_0$ and $M_1$ that reside on $K$.
% 
% The reason we reject the region $r,s\le0$ stems from mechanical considerations: If we transform back to the original variables, then we find
% \begin{equation*}
%  r(\eta)|_{\eta=t^\lambda x}=t\partial_t\log \gamma(t,x), \quad r(\eta)s(\eta)|_{\eta=t^\lambda x}=t\partial_t \log \theta(t,x).
% \end{equation*}
% %for the original variables $\theta(t,x)$ and $\gamma(t,x)$.
% Shear band initiation is related to conditions of loading where both the plastic strain and the temperature are increasing. This motivates to
% restrict on the region $r > 0$, $s > 0$.
% 
% There are two equilibria that lie in this region:
% \begin{align*}
%  M_0 &= (0,0,r_0,s_0), & r_0 &=\frac{2+2\alpha-n}{D} + \frac{2+2\alpha}{D}\lambda, & s_0&=\frac{1+m+n}{1+\alpha} - \frac{n}{(1+\alpha)r_0},\\
%  M_1 &= (0,1,r_1,s_1), & r_1 &= r_0-\frac{1+\alpha}{\alpha-m-n}\lambda, & s_1&=\frac{1+m+n}{1+\alpha} - \frac{n}{(1+\alpha)r_1}.
% \end{align*}
% We find $r_0$ and $r_0s_0 = c = \frac{2(1+m)}{D} + \frac{2(1+m+n)}{D}\lambda$ strictly positive, or $r_0, s_0>0$. However $r_1,s_1$ become negative for large $\lambda$. We find $r_1,s_1>0$ only under the constraint that %$\lambda$ is not arbitrarily large but
% \begin{equation} \label{eq:lambda-range}
%  0< \lambda < \frac{2(\alpha-m-n)}{1+m+n}\left(\frac{1+m}{1+m+n}\right).
% \end{equation}

% \subsection{Linear stability of $M_0$ and $M_1$}
We denote the four eigenvalues and four eigenvectors of the vector field linearized at $M_i$, $i=0,1$,  by $\mu_{ij}$ and $X_{ij}$ with $j=1,2,3,4$. %Appendix \ref{append:lin} contains the complete exposition of what this section discusses. 
\begin{itemize}
 \item $M_0$ is a saddle; the matrix of the linearized vector field at $M_0$ has three positive eigenvalues and one negative eigenvalue. 
 \begin{equation} \label{eq:eigM0}
  \mu_{01} = 2, \quad \mu_{02}=1, \quad \mu_{03}=\mu_0^+=\BO\Big(\frac{1}{n}\Big)>0, \quad \mu_{04}=\mu_0^{-}<0,
 \end{equation}
  where $\mu_0^\pm$ are respectively a positive and a negative solution of the quadratic equation
%  $$ \mu^2 - \mu\Big(\frac{r_0(1-s_0)}{n\lambda}-\frac{r_0s_0+1}{\lambda}\Big) - \frac{r_0^2s_0(\alpha-m-n)}{n\lambda^2}=0.$$
 $$ \Big(\mu - \frac{r_0}{n}\Big(\frac{1-s_0}{\lambda}-\frac{n}{\lambda r_0}\Big)\Big)\Big(\mu + \frac{s_0r_0}{\lambda}\Big) -
 \frac{s_0r_0(1-s_0)(\alpha r_0)}{n\lambda^2} = 0.$$%\frac{s_0r_0}{n} \frac{1-s_0}{\lambda}\frac{\alpha r_0}{\lambda} = 0.$$
The leading orders of $\mu_0^\pm$ are given by
% $$\mu_0^+ = \frac{\alpha-m}{n\lambda(1+\alpha)}\frac{2(1+\alpha)(1+\lambda)}{(1+2\alpha-m)}+\BO(1), \quad\mu_0^- = -\frac{1+m}{\lambda}\frac{2(1+\alpha)(1+\lambda)}{(1+2\alpha-m)}  + \BO(n).$$
$$\mu_0^+ = \frac{2(\alpha-m)(1+\alpha)(1+\lambda)}{n\lambda(1+\alpha)(1+2\alpha-m)}+\BO(1), \quad\mu_0^- = -\frac{2(1+m)(1+\alpha)(1+\lambda)}{\lambda(1+2\alpha-m)}  + \BO(n).$$
Notice that the one of the positive eigenvalue $\mu_{03}$ is $\mathcal{O}( \frac{1}{n})$, which indicates the separably fast dynamics along the direction $X_{03}$. We will make use of this structure later.
% $$\mu_0^+ = \frac{r_0(1-s_0)}{n\lambda} + \frac{1}{\lambda}\frac{{r_0^2s_0}(\alpha-m-n)}{ {r_0(1-s_0)}-n(1+r_0s_0) } + \BO(n), \quad \mu_0^- = -\frac{1}{\lambda}\frac{{r_0^2s_0}(\alpha-m-n)}{ {r_0(1-s_0)}-n(1+r_0s_0) } + \BO(n).$$
The precise values of the eigenvector components are presented in the Appendix \ref{append:lin}, the directions of the eigenvectors are
pointed out in Fig. \ref{fig:equilibria} for $n$ sufficiently small.
 \item $M_1$ is a saddle; the matrix of the linearized vector field at $M_1$ has one positive eigenvalue and three negative eigenvalues. 
\begin{equation} \label{eq:eigM1}
 \mu_{11}=-\frac{1+m+n}{\alpha-m-n}, \quad \mu_{12}=-1, \quad \mu_{13}=\mu_1^+=\BO\Big(\frac{1}{n}\Big)>0, \quad \mu_{14}=\mu_1^{-}<0,
\end{equation}
where $\mu_1^\pm$ is respectively a positive and a negative solution of the quadratic equation
 $$ \Big(\mu - \frac{r_1}{n}\Big(\frac{1-s_1}{\lambda}-\frac{n}{\lambda r_1}\Big)\Big)\Big(\mu + \frac{s_1r_1}{\lambda}\Big) - %\frac{s_1r_1}{n} \frac{1-s_1}{\lambda}\frac{\alpha r_1}{\lambda} = 0$$
\frac{s_1r_1(1-s_1)(\alpha r_1)}{n\lambda^2} = 0.$$
The leading orders of $\mu_1^\pm$ are given by
\begin{align*}
\mu_1^+ &= \frac{\alpha-m}{n\lambda(1+\alpha)}\Big(\frac{2(1+\alpha)(1+\lambda)}{(1+2\alpha-m) } - \frac{1+\alpha}{\alpha-m-n}\lambda\Big) + \BO(1), \\
\mu_1^- &= -\frac{1+m}{\lambda}\Big(\frac{2(1+\alpha)(1+\lambda)}{(1+2\alpha-m) } - \frac{1+\alpha}{\alpha-m-n}\lambda\Big) + \BO(n).
\end{align*}
Notice again that the positive eigenvalue $\mu_{13}$ is $\mathcal{O}( \frac{1}{n})$.

In constrast to what happens at $M_0$, eigenvalues of of the linearized vector field at $M_1$ may be repeated. The exposition in the Appendix \ref{append:lin} specifies the possible combinations completely and the eigenvectors or the generalized eigenvectors are provided accordingly.
\end{itemize}
\begin{figure}
 \centering
  \psfrag{x0}{\scriptsize $M_0$}
  \psfrag{x1}{\scriptsize $M_1$}
  \psfrag{x2}{~~\scriptsize $1$}
  \psfrag{x3}{}
  \psfrag{p}{\scriptsize $p$}%=\frac{\gamma}{\sigma}$}
  \psfrag{q}{\scriptsize~~~$q$}%=n\frac{v}{\sigma}$}
  \psfrag{q*}{}%=\frac{2-n}{\lambda}$}
  \psfrag{r*0}{}%\hskip -15pt$r_0=1+\frac{2\lambda}{2-n}$}
  \psfrag{r*1}{}% \hskip -35pt$r_1=1-\frac{n\lambda}{(2-n)(1-n)}$}
  \psfrag{r*2}{}
  \subfigure[$pqr$-space]{
  \psfrag{r}{\scriptsize$r$}%=\big(\sigma\gamma^{(1-n)}\big)^{\frac{1}{n}}$}
  \includegraphics[width=6cm]{equilibriapqr.eps}\label{fig:eq1}
  }
  \quad \quad
  \subfigure[$pqs$-space]{
  \psfrag{r}{\scriptsize$s-\frac{1+m}{1+\alpha}$}%=\big(\sigma\gamma^{(1-n)}\big)^{\frac{1}{n}}$}
  \includegraphics[width=6cm]{equilibriapqs.eps}\label{fig:eq2}
  }
  \caption{Eigenvectors around $M_0$ and $M_1$ in $pqr$-space and in $pqs$-space respectively ($\mu_{11}\ne-1$ and $n\ll1$). } \label{fig:equilibria}
\end{figure}

\section{Characterization of the heteroclinic orbit} \label{sec:char}
The equilibrium $M_0$ has three dimensions for the unstable manifold and one for the stable manifold while the equilibrium $M_1$ has three dimensions for the 
stable manifold and one for the unstable manifold. Due to the high dimensionality, it is not immediate to read off the behavior 
of the flow in phase space. This section aims to develop a rough picture illustrating the flow on the positive sector $p,q,r,s \ge0$. 
We discuss how the local stable and unstable manifolds extend globally to intersect each other. 

Understanding the flow is then linked to our objective to characeterize the heteroclinic orbit we search for. The heteroclinic orbit of our interests will be called $\chi(\eta)$ for the rest of this paper. We will clarify the transmission of the conditions \eqref{eq:bdry0} imposed at $\xi=0$ to the one for system \eqref{eq:slow} and this gives one characterization condition at the one end point as $\eta(=\log\xi) \rightarrow -\infty$. For the other end point as $\eta \rightarrow \infty$, we project our expectation that the variables $p$, $q$, $r$ and $s$ equilibrate to a bounded state as $\eta \rightarrow \infty$. This in particular is sufficient to have the varables $\tg$, $\tv$, $\tth$, and $\ts$ grow at most exponentially as seen from \eqref{eq:tildesys2}. Recalling back, if $\chi(\eta)$ equilibrate as $\eta \rightarrow \infty$ then the properties of strict increase (in time) of  $\theta(t,x)$ and $\gamma(t,x)$ restricts the asymptotic state in sector $r>0$ and $s>0$ and we find $M_1$ the only equilibrium at this range. This characterizes the behavior at the other end point, namely $\chi(\eta) \rightarrow M_1$ as $\eta \rightarrow \infty$. The orbits those do not equilibrate may be studied with physical relevance as well but we do not study further on those orbits in this paper. 

These two end points behavior are interpreted geometrically as well: The end point behavior as $\eta \rightarrow -\infty$ specifies a nontrivial submanifold of the unstable manifold of $M_0$ on which the heteroclinic orbit emanates out. This submanifold intersects the stable manifold of $M_1$ and the intersection of these two manifold is the heteroclinic orbit we search for. 

Now, we begin clarifying what has been suggested. The following proposition states how \eqref{eq:bdry0} are transmitted to asymptotic conditions on $(p,q,r,s)$.

% Now, we study the asymptotic behavior of $\chi(\eta)$ as $\eta \rightarrow -\infty$ in terms of the boundary conditions \eqref{eq:bdry0} at $\xi=0$.
\begin{proposition} \label{prop1}
    Suppose $\big(\Gamma,V,\Theta,\Sigma,U\big)$ is a smooth solution of \eqref{eq:ss-odes}, \eqref{eq:bdry0} in the half-line $\xi\ge0$, all components of which at $\xi=0$ are positive and bounded. Then the corresponding orbit defined by transformations \eqref{eq:CAPtoBAR}, \eqref{eq:BARtoTIL}, \eqref{eq:pqrdef}, $\chi(\eta) = (p(\eta), q(\eta), r(\eta),s(\eta)) \rightarrow M_0$ as $\eta \rightarrow -\infty$. Furthermore, it tends to $M_0$ along the direction of the first eigenvector $X_{01}$, i.e.,
    \begin{equation} \label{eq:alpha}
     e^{-2\eta}\big(\chi(\eta) - M_0 \big) \rightarrow \kappa X_{01}, \quad \text{for some constant $\kappa>0$ as $\eta \rightarrow -\infty$.}
    \end{equation}
\end{proposition}

\begin{remark} \label{rem:alpha}
  That the orbit approches $M_0$ tangent to $X_{01}$ as $\eta \rightarrow -\infty$ is nontrivial. $M_0$ has three-dimensional unstable manifold and $\mu_{02}(=1)<\mu_{01}(=2)<\mu_{03}(=\BO(\frac{1}{n}))$. %The unstable manifolds of one, two, and three dimensions are separable from the fastest asymptotic rate. 
  So orbits emanating from $M_0$ tangent to $X_{01}$ all lie on a two dimensional manifold that at $M_0$ tangent to the plane spanned by $X_{01}$ and $X_{03}$. The two dimensional submanifold will be referred to as the Strongly unstable manifold of $M_0$.
\end{remark}

\begin{proof}
Assuming smoothness and boundedness of $\big(\Gamma,V,\Theta,\Sigma,U\big)$ in the neighborhood of $\xi=0$ and from the boundary conditions \eqref{eq:bdry0}, the derivatives of $\big(\Gamma,V,\Theta,\Sigma,U\big)$ evaluated at $\xi=0$ are obtained by differentiating the system \eqref{eq:ss-odes} repeatedly.
% First we check
% \begin{align*}
%  &\Gamma'(0)=U'(0)=\Sigma'(0)= \Big(\frac{U}{\Gamma}\Big)'(0)=\Big(\frac{\Sigma\Gamma}{\Theta}\Big)'(0)=0,\\
%  &\frac{U}{\Gamma}(0) = a, \quad \frac{\Sigma\Gamma}{\Theta}(0)=\frac{c}{a}.
% \end{align*}
Re-write \eqref{eq:ss-odes}
\begin{align*}
  a + \lambda\xi\frac{\Gamma'}{\Gamma} &= \frac{U}{\Gamma}, &
  c + \lambda\xi\frac{\Theta'}{\Theta} &= \frac{\Sigma\Gamma}{\Theta} \frac{U}{\Gamma},\\
  (b+\lambda)U  + \lambda \xi U'(\xi) &= \Sigma^{''} = \Big(\frac{\Sigma\Gamma}{\Theta} \frac{\Theta}{\Gamma}\Big)^{''}, &
  \frac{\Big(\frac{\Sigma\Gamma}{\Theta}\Big)^{''}}{\frac{\Sigma\Gamma}{\Theta}} &= (1+m+n)\frac{\Gamma^{''}}{\Gamma}-(1+\alpha) \frac{\Theta^{''}}{\Theta} + n \frac{ \big(\frac{U}{\Gamma}\big)^{''}}{\frac{U}{\Gamma}},
\end{align*}
from where after a computation we conclude
\begin{align*}
&\frac{U}{\Gamma}(0) = a = r_0,  & \Big(\frac{U}{\Gamma}\Big)'(0)&=0, & \Big(\frac{U}{\Gamma}\Big)^{''}(0) &= \frac{\Gamma(0)}{\Sigma(0)} \frac{-2(b+\lambda)r_0}{\frac{1-s_0}{\lambda}-\frac{n}{r_0}\Big(\frac{2}{s_0} + \frac{r_0}{\lambda}\Big)\left(\frac{ \frac{1}{\lambda}+2}{ \frac{1+\alpha}{\lambda}r_0 + \frac{2}{s_0}}\right)},\\
&\frac{\Sigma\Gamma}{\Theta}(0) = \frac{c}{a} = s_0,  & \Big(\frac{\Sigma\Gamma}{\Theta}\Big)'(0)&=0, &
\Big(\frac{\Sigma\Gamma}{\Theta}\Big)^{''}(0) &= \frac{n}{r_0} \left(\frac{ \frac{1}{\lambda}+2 }{ \frac{1+\alpha}{\lambda}r_0 + \frac{2}{s_0}}\right)\Big(\frac{U}{\Gamma}\Big)^{''}(0).
\end{align*}
Now, we consider the Taylor expansions of $p(\log\xi)$, $q(\log\xi)$, $r(\log\xi)$ and $s(\log\xi)$ at $\xi=0$ using the above and \eqref{eq:bdry0}.
\begin{align*}
 p(\log\xi) &= \frac{ \tg }{\ts} = \frac{ \xi^{a_1} \Gamma(\xi)}{\xi^{d_1} \Sigma(\xi)} = \xi^2\frac{\Gamma(\xi)}{\Sigma(\xi)} = \xi^2\frac{\Gamma(0)}{\Sigma(0)} + o(\xi^2) \\
 %&= \xi^2\Big(\frac{U(0)}{\Phi(0)}\Big)^{-n}\Phi(0)^{1+\frac{\alpha-n}{1+\alpha}} + o(\xi^2),\\
 q(\log\xi) &= b\frac{\tv}{\ts} = b\frac{ \xi^{b_1} V(\xi) }{ \xi^{d_1} \Sigma(\xi)} = b\xi\frac{ V(\xi) }{ \Sigma(\xi)} = b\xi^2 \frac{U(0)}{\Sigma(0)}+ o(\xi^2)=\xi^2 ~br_0\frac{\Gamma(0)}{\Sigma(0)} + o(\xi^2) \\
 %&= \xi^2\Big(b\frac{U(0)}{\Phi(0)}\Big)\Big(\frac{U(0)}{\Phi(0)}\Big)^{-n}\Phi(0)^{1+\frac{\alpha-n}{1+\alpha}} + o(\xi^2),\\
 r(\log\xi) &= \frac{\tu}{ \tg } = \frac{ \xi^{1+b_1}U(\xi) }{ \xi^{a_1}\Gamma(\xi) } = \frac{ U(0) }{ \Gamma(0) }+ \xi \Big(\frac{U}{\Gamma}\Big)'(0) + \frac{1}{2}\xi^2\Big(\frac{U}{\Gamma}\Big)^{''}(0) + o(\xi^2)\\
  &=\frac{ U }{ \Gamma }(0) + \xi^2\frac{\Gamma(0)}{\Sigma(0)} \frac{-(b+\lambda)r_0}{\frac{1-s_0}{\lambda}-\frac{n}{r_0}\Big(\frac{2}{s_0} + \frac{r_0}{\lambda}\Big)\left(\frac{ \frac{1}{\lambda}+2}{ \frac{1+\alpha}{\lambda}r_0 + \frac{2}{s_0}}\right)} ,\\
 s(\log\xi) &= \frac{\ts\tg}{\tth} = \frac{ \xi^{a_1+d_1}\Sigma(\xi)\Gamma(\xi) }{\xi^{c_1} \Theta(\xi)} = \frac{ \Sigma\Gamma }{\Theta}(0) + \xi \Big(\frac{ \Sigma\Gamma }{\Theta}\Big)^{'}(0) + \frac{1}{2}\xi^2\Big(\frac{ \Sigma\Gamma }{\Theta}\Big)^{''}(0) + o(\xi^2)\\
 &=\frac{ \Sigma\Gamma }{\Theta}(0) + \xi^2 n \left(\frac{ \big(\frac{1}{\lambda}+2\big) \frac{1}{r_0} }{ \frac{1+\alpha}{\lambda}r_0 + \frac{2}{s_0}}\right)\frac{\Gamma(0)}{\Sigma(0)} \frac{-(b+\lambda)r_0}{\frac{1-s_0}{\lambda}-\frac{n}{r_0}\Big(\frac{2}{s_0} + \frac{r_0}{\lambda}\Big)\left(\frac{ \frac{1}{\lambda}+2}{ \frac{1+\alpha}{\lambda}r_0 + \frac{2}{s_0}}\right)}+ o(\xi^2).
\end{align*}
Therefore,
\begin{align*}
\chi(\log\xi)-M_0  = \big(p(\log\xi),q(\log\xi),r(\log\xi),s(\log\xi)\big) -M_0 =  \frac{\Gamma(0)}{\Sigma(0)}\xi^2 X_{01} + o(\xi^2),
\end{align*}
which is the \eqref{eq:alpha} for $\eta=\log\xi$.
\end{proof}
\begin{remark} \label{rem:signs}
For $n$ small enough, we find signs of second derivatives are definite: $\displaystyle \Big(\frac{U}{\Gamma}\Big)^{''}(0) <0$, $\displaystyle \Big(\frac{\Sigma\Gamma}{\Theta}\Big)^{''}(0) <0$, and
\begin{equation} \label{eq:second_der}
\begin{aligned}
\frac{\Gamma^{''}(0)}{\Gamma(0)} &= \frac{1}{2\lambda}\Big(\frac{U}{\Gamma}\Big)^{''}(0) < 0, &
\frac{\Theta^{''}(0)}{\Theta(0)} &= \frac{1}{2\lambda}\Big(s_0\Big(\frac{U}{\Gamma}\Big)^{''}(0) + r_0\Big(\frac{\Sigma\Gamma}{\Theta}\Big)^{''}(0)\Big)\Big)  < 0,\\
\frac{U^{''}(0)}{U(0)} &=\frac{\Gamma^{''}(0)}{\Gamma(0)} + \frac{ \big(\frac{U}{\Gamma}\big)^{''}(0)}{\frac{U}{\Gamma}(0)}< 0,&
\Sigma^{''}(0)&=(b+\lambda)U(0)>0.
\end{aligned}
\end{equation}
\end{remark}


\subsection{A two-parameter family of heteroclinic orbits} \label{sec:twoparam}
Assuming the existence of the heteroclinic $\chi(\eta)$, this section is devoted to giving an interpretation of it in terms of the given data. By  data we refer to 
$\big(\Gamma(0),\Theta(0),\Sigma(0),U(0)\big)$. Since the system \eqref{eq:ss-odes} is singular it is not clear how many boundary conditions, 
for  $\big(\Gamma(0),\Theta(0),\Sigma(0),U(0)\big)$, are independent. Below, we clarify this issue.

The orbit satisfying \eqref{eq:alpha} is hypothesized to be one dimensional and achieved for each set of parameters $(\lambda, \alpha, m,n)$. 
If $\chi(\eta)$ is a heteroclinic orbit, then so is the reparametrization $\chi(\eta-\eta_0)$ for any $\eta_0\in \mathbb{R}$. In conclusion, in order to fix one heteroclinic orbit,  five parameters $\lambda$, $\eta_0$, $\alpha$, $m$, and $n$ are required. 
The latter three are the material properties that account for thermal softening, strain hardening, and strain-rate hardening.

Other than those three, we associate the localization rate $\lambda$  and the translation factor $\eta_0$  to the physical data. Due to the singular nature of \eqref{eq:ss-odes}, not all of them are independent but two out of six numbers $\lambda$, $\eta_0$, $\Gamma(0)$, $\Theta(0)$, $\Sigma(0)$, and $U(0)$ fixes the rest. In what follows, we choose $\Gamma(0)$ and $U(0)$ as the primary parameters and specify the rest in terms of them. $\Gamma(0)$ and $U(0)$ are the tip sizes of the strain and strain rate at $\xi=0$. To recap, we take a view that for each given triple of material parameters $(\alpha,m,n)$ there is a two-parameters family of heteroclinic orbits by $\Gamma(0)$ and $U(0)$ .

As $r_0 = \frac{2(1+\alpha)-n}{D} + \frac{2(1+\alpha)}{D}\lambda = \frac{U(0)}{\Gamma(0)}$,
we put
\begin{equation} \label{eq:lambda}
 \lambda = \Big(\frac{U(0)}{\Gamma(0)} - \frac{2(1+\alpha)-n}{D}\Big)\frac{D}{2(1+\alpha)}.
\end{equation}
In particular, the restriction \eqref{eq:lambda-range} on $\lambda$ reads as the ratio condition
\begin{equation} \label{eq:restriction}
\begin{aligned}
 \frac{2(1+\alpha) -n}{D} < \frac{U(0)}{\Gamma(0)} &< \frac{2(1+\alpha) -n}{D} + \frac{4(1+\alpha)(\alpha-m-n)(1+m)}{D(1+m+n)^2}\\
 &=\frac{2(1+\alpha)}{1+m+n} -\frac{n}{D}\left( \frac{4(1+\alpha)(\alpha-m-n)}{(1+m+n)^2} +1\right).
\end{aligned}
\end{equation}
Once $\lambda$ is fixed, a straightforward calculation gives
$$\Theta(0) = c^{-\frac{1}{1+\alpha}}\Gamma(0)^{\frac{m}{1+\alpha}} U(0)^{\frac{1+n}{1+\alpha}}, \quad \Sigma(0) = c^{\frac{\alpha}{1+\alpha}}\Gamma(0)^{\frac{m}{1+\alpha}} U(0)^{-\frac{\alpha-n}{1+\alpha}}.$$

$\eta_0$ is defined by the following procedure:  Any orbit $\varphi(\eta)$ escaping $M_0$ in the direction $X_{01}$ is characterized by the two constants $\kappa_1$ and $\kappa_3$ that accounts for the asymptotic expansion %is characterized by a constant triple $(\kappa_1,\kappa_2,\kappa_3)$ whose asymptotic expansion in the neighborhood of $M_0$ is given by
 \begin{equation}\label{eq:alpha-expan}
  \varphi(\eta) - M_0 = \kappa_1 e^{\mu_{01}\eta} X_{01} + \kappa_3 e^{\mu_{03}\eta}X_{03} + \text{higher-order terms as $\eta \rightarrow -\infty$}.
 \end{equation}
Let $(\bar\kappa_1,\bar\kappa_3)$ be fixed and let $\bar\chi(\eta)$ be the heteroclinic orbit characterized by the two numbers. We look for $\chi(\eta) = \bar\chi(\eta-\eta_0)$. Then \eqref{eq:alpha} gives
$$\kappa_1 X_{01}=\lim_{\eta \rightarrow -\infty}\big(\chi(\eta) - M_0\big)e^{-2\eta} = \lim_{\eta \rightarrow -\infty} \big(\bar\chi(\eta-\eta_0) - M_0\big)e^{-2(\eta-\eta_0)}e^{-2\eta_0} = e^{-2\eta_0}\bar\kappa_1 X_{01}.$$
Thus $\eta_0 = \frac{1}{2}\log {\frac{\bar\kappa_1}{\kappa_1}}$ but from the proof of Proposition \ref{prop1}, we know that
$\kappa_1 = \frac{\Gamma(0)}{\Sigma(0)}$, or
\begin{equation}
 \eta_0 = \frac{1}{2}\log \left(\frac{\Sigma(0)}{\Gamma(0)}\bar\kappa_1\right).%\log \sqrt{\frac{\bar\kappa_1}{\frac{\Gamma(0)}{\Sigma(0)}}}.
\end{equation}
%The closer description of the $\big(\Gamma,V,\Theta,\Sigma,U\big)$ is supplemented in the last section.

% \section{Existence via Geometric theory of singular perturbation}
% In this section, we give a proof for the existence of the heteroclinic orbit hypothesized in the preceding section. It is accomplished by the two consecutive chunks of arguments, the geometric singular perturbation theory and the theorem of Poincar\'e-Bendixson on the positively invariant set.
%
% Considering $n$ as a small parameter, we take a point of view on the $(p,q,r)$-system regarding it a singularly perturbed problem. See the term $n\dot{r}$ in $\eqref{eq:pqrsys}_3$, which indicates that the evolution of $r$ is of faster time scale, and we refer to $r$ as a fast variable and the others $p$ and $q$ two slow variables.
%
% The geometric singular perturbation theory considers the dynamics that takes place near the zeroset of the right-hand-side of $\eqref{eq:pqrsys}_3$, where the time scale of the evolution of the fast variable possibly becomes comparable to that of slow variables. In particular for the critical case $n=0$, the orbits that are upon the zeroset are considered, on where the problem is essentially reduced to that of slow variables only and it becomes regularly perturbed problem. The upshot of the geometric singular perturbation theory is to enable continuing this reduction to $n>0$ provided $n$ is small.
%
% To be concrete and to get prepared to apply the geometric singular perturbation theory, in this section we examine two objects and one verification of the property of the latter: First, we examine the zeroset of the right-hand-side of $\eqref{eq:pqrsys}_3$ when $n=0$, which is given by the graph
% $$r=\frac{ \frac{\alpha c_0}{\lambda} - d_1 -q }{ \frac{\alpha}{\lambda} + \lambda p}\triangleq h(p,q;\lambda,\alpha,n=0) \quad \text{for $r>0$}.$$
% We decribe this graph in the phase space to probe an idea of developing arguments. A suitable compact piece of the graph is subjected to applying the theory and we refer to it as the {\it critical manifold}. This is our second object and we give the precise specification of this object. The critical manifold serves as the template invariant manifold when $n=0$ from which the perturbed invariant manifold when $n>0$ is disposed of.  Lastly, the {\it normally hyperbolicity} of the critical manifold is verified. This is in regard with the system \eqref{eq:pqr_fast} in the fast independent variable. By rescailing $\tilde\eta=\frac{\eta}{n}$ for $n>0$, we see the dynamics that occurs in the fast time scale. The critical case $n=0$ in this format describes the fast relaxation toward the critical manifold or escape from it. The critical manifold is the set of equilibrium for the \eqref{eq:fastn0} and is the center manifold of the individual equilibrium points in it. Normally hyperbolicity concerns the hyperbolicity in the rest of the dimensions and this is the key property that enables us to continue the invariant manifold for $n>0$.
% %In the first chunk, we exploit the fact that the $(p,q,r)$-system is of multiple time scale: Observe  the small parameter $n$ multiplied to $\dot{r}$ in \eqref{eq:pqrsys}, which makes the evolution of $r$ fast, i.e., if
% %\begin{equation}
% % f(p,q,r;\lambda,\alpha,n) = r\Big( \Big[\frac{\alpha-n}{\lambda(1+n)}\Big(r^{1+n}-c_0\Big)\Big]+\Big[d_1 + q + \lambda pr\Big]\Big).
% %\end{equation}
% %the right-hand-side of the equation on $r$, $\dot{r} \sim \frac{1}{n}$ away from the zero set of $f(p,q,r;\lambda,\alpha,n)$ When $n=0$, the orbit is strictly restricted on the zero set
% %\begin{equation}
% % Z \triangleq \{\,(p,q,r)\; | \; f(p,q,r,\lambda,\alpha,n=0)\, \} \label{eq:zeroset}
% %\end{equation}
% %and the problem is essentially reduced to the one of only slow variable $p$ and $q$, provided $r$ is solved from the algebraic equation \eqref{eq:zeroset}.
% %
% %The graph $\displaystyle r=\frac{ \frac{\alpha c_0}{\lambda} - d_1 -q }{ \frac{\alpha}{\lambda} + \lambda p}$.
% %is taken and is referred to as the critical manifold.
% %
% %
% %
% %In order to take the suitable compact piece of the zero set $Z$, we detail in the graph $\displaystyle r=\frac{ \frac{\alpha c_0}{\lambda} - d_1 -q }{ \frac{\alpha}{\lambda} + \lambda p}$.
% %
% %
% %
% %
% %
% %
% %\hrulefill
% %
% %
% %
% %
% %
% %
% %
% %
% %This exploits the fact that the (p,q,r)-system is of multiple time scale, i.e., the presence of the small parameter $n$ in front of  $\dot{r}$ indicates the time scale of $\dot{r}\sim \frac{1}{n}$ unless the right-hand-side of the equation is small enough to compensate.
% %
% %Let $f(p,q,r;\lambda,\alpha,n)$ be the right-hand-side of the equation on $r$, i.e.,
% %\begin{equation}
% % f(p,q,r,\lambda,\alpha,n) = r\Big( \Big[\frac{\alpha-n}{\lambda(1+n)}\Big(r^{1+n}-c_0\Big)\Big]+\Big[d_1 + q + \lambda pr\Big]\Big).
% %\end{equation}
% %The compact subset of the zero set of $f(p,q,r;\lambda,\alpha,n=0)$
% %$$ Z \triangleq \{\,(p,q,r)\; | \; f(p,q,r,\lambda,\alpha,n=0)\, \} $$
% %is taken and is referred to as the critical manifold.
% %
% %
% %\subsection{Reduction to the slow system}
%
% \subsection*{The graph $\displaystyle r=\frac{ \frac{\alpha c_0}{\lambda} - d_1 -q }{ \frac{\alpha}{\lambda} + \lambda p}$.}
% The equation of the graph can be written in the form
% \begin{equation}
%  q + \lambda {r}p + \frac{\alpha}{\lambda} \Big( r-r_0\Big)=0, \label{eq:level}
% \end{equation}
% where we observe that the level line $r=\bar{r}$ is the straight line in the phase space. On the $(p,q)$-plane, for $r$ in the range of $(0,r_0)$ the contour line crosses the first quadrant with the negative slope, intersecting $p$-axis and $q$-axis. When $r=r_0$ it is the line passing the origin and this point $(0,0,r_0) = M_0$. When $r=r_1$, the level line passes the $(0,1,r_1)=M_1$.
%
% \subsection*{Critical manifold}
% Inequality
% $$ \lambda < \frac{2(\alpha-n)}{(1+n)^2} $$
% prevents $r_1$ from being less than equal to $0$. Therefore, we always can take the value $0<\underbar{R}<r_1$. Having fixed the value $\underbar{R}$, we take the closed set $T$ that is the triangle in the first quadrant enclosed by $p$-axis, $q$-axis and the contour line $\underbar{R} = \bar{r}(p,q)$. Observe that $h\ge\underbar{R}>0$ on the graph if $(p,q) \in T$. Since $h$ is continuous in the neighborhood of $T$, we can take another closed set $K$ in the vicinity of $T$ on which $h$ is still away from $0$. We take the compact piece of the set $Z$ by
% \begin{equation}
%  G(\lambda,\alpha,n=0) \triangleq \Big\{\, (p,q,r) \;|\; (p,q) \in K, \text{ and } r=\frac{ \frac{\alpha c_0}{\lambda} - d_1 -q }{ \frac{\alpha}{\lambda} + \lambda p} \,\Big\} \subset Z
% \end{equation}
%
% \subsection*{Normally hyperbolicity}
% The system in {\it fast scale} with the independent variable $\tilde{\eta} = \eta/n$ is
% \begin{equation}\label{eq:pqr_fast} \tag*{($\tilde{P}$)}
% \begin{aligned}
%  p^\prime &=np\Big( \Big[\frac{1+\alpha}{1+n}\,\frac{1}{\lambda }\Big(r^{1+n}-c_0\Big)\Big] -\Big[d_1 + q + \lambda pr\Big]\Big), \\
%  q^\prime &=nq\Big(\Big[b_1 +\frac{bpr}{q}\Big] -\Big[d_1 + q + \lambda pr\Big]\Big), \\
%  r^\prime &=r\Big( \Big[\frac{\alpha-n}{\lambda(1+n)}\Big(r^{1+n}-c_0\Big)\Big]+\Big[d_1 + q + \lambda pr\Big]\Big)\triangleq f(p,q,r;\lambda,\alpha,n),
% \end{aligned}
% \end{equation}
% where we denoted $\displaystyle(\cdot)^\prime = \frac{d}{d\tilde{\eta}}(\cdot)$. In particular, the system $(\tilde{P})|_{n=0}$ reads
% \begin{align}
%  p^\prime =0, \quad q^\prime =0, \quad r^\prime=r\Big( \Big[\frac{\alpha}{\lambda}\Big(r-c_0\Big)\Big]+\Big[d_1 + q + \lambda pr\Big]\Big) = f(p,q,r;\lambda,\alpha,0). \label{eq:fastn0}
% \end{align}
%
% \begin{lemma} \label{lem:normal_hyper}
%  $G(\lambda,\alpha,0)$ is a normally hyperbolic invariant manifold with respect to the system $(\tilde{P})|_{n=0}$.
% \end{lemma}
% \begin{proof}
% To prove the normally hyperbolicity of the graph $G(\lambda,\alpha,n=0)$, we show that the coefficient matrix of the linearized system of $(\tilde{P})|_{n=0}$ around $G(\lambda,\alpha,n=0)$ has the eigenvalue $0$ exactly with the multiplicity $2$. Let $P$, $Q$, and $R$ be the perturbations of $p$, $q$, and $r$ respectively. The linearized equations after discarding terms higher than the first order are
% \begin{align*}
%  \begin{pmatrix} {P}^\prime\\ {Q}^\prime \\ {R}^\prime \end{pmatrix} =
%  \begin{pmatrix} 0 & 0& 0\\ 0 & 0 & 0\\ \lambda h^2 & h & ( \frac{\alpha}{ \lambda} + \lambda p )h \end{pmatrix} \begin{pmatrix} {P}\\ {Q} \\ {R} \end{pmatrix},
% \end{align*}
% where $h$ is a shorthand for $h(p,q;\lambda,\alpha,n=0)$. $( \frac{\alpha}{ \lambda} + \lambda p )h > 0$ because $\alpha>0$, $p\ge0$ and $h > \underbar{R}>0$ on the $G(\lambda,\alpha,n=0)$, which proves that $0$ is an eigenvalue with multiplicity $2$.
% \end{proof}
%
% \begin{proposition}
% For $n>0$ sufficiently small, there is a function $h(p,q;\lambda,\alpha,n) : T\subset \mathbb{R}^2 \mapsto \mathbb{R}$ such that
% \begin{equation} \tag*{(${R}$)} \label{eq:reduced}
% \begin{aligned}
%  \dot{p} &=p\Big(\frac{1}{ \lambda }\big(h(p,q;\lambda,\alpha,n) - \frac{2-n}{1+m-n}\big) - \frac{1-m+n}{1+m-n} + 1-q- \lambda p h(p,q;\lambda,\alpha,n)\Big),\\
%  \dot{q} &=q\Big(                                                                          1-q- \lambda p h(p,q;\lambda,\alpha,n)\Big) + b^{\lambda,m,n}ph(p,q;\lambda,\alpha,n),
% \end{aligned}
% \end{equation}
%
%
% \begin{enumerate}
%  \item $h$ is jointly smooth function of $p$, $q$, and $n$ in $T \times I$
%  \item The graph $r=h(p,q;\lambda,\alpha,n)$ is locally invariant, i.e., The orbit $(p,q,r)$
% \end{enumerate}
%
% \end{proposition}
%
%
% \subsubsection{Flow on the critical manifold : the case $m=1$}
%
% The marginal case $m=1$ provides closer detail.
% By substituting $h^{\lambda,1,0}(p,q)$ in place of $r$, the system is explicitly solved and
% %we can solve the system explicitly and the whole critical graph is completely characterized.
% the general solution on the graph is a family of parabolae $p=kq^2$ and $r=h^{\lambda,1,0}(p,q)$. This includes the two extremes $p=0$ and $q=0$, where $k$ takes $0$ and $\infty$ respectively. See Fig. \ref{fig:hn0m1}. We focus on discussing two points: 1) In an effort to apprehend the flow of the rest of cases, we remark a few features for this marginal case, which in turn persist under the perturbation; and 2) we report features that do not persist too. These features do not play any role in our study, but this bifurcation is described here for clarity.
%
% We address the first point. Look at $M_0^{ \lambda,1,0}$ in Fig. \ref{fig:hn0m1_b} surrounded by a family of parabolae in the neighborhood. Our interested direction $\vec{X}_{02}$ and the other $\vec{X}_{01}$ are annotated near $M_0^{ \lambda,1,0}$ by a dotted arrow. The family of parabolae is manifesting the fact that orbit curves meet $M_0^{ \lambda,1,0}$ tangentially to $\vec{X}_{01}$; one exception is the degenerate straight line that emanates in $\vec{X}_{02}$, which is depicted as the green one in Fig. \ref{fig:hn0m1}, the target orbit. Another observation from the $pq$-plane is that the flow in the first quadrant far away from the origin is {\it inwards}. More precisely, as illustrated in Fig. \ref{fig:hn0m1_b}, whenever $0<\underbar{R} < 1 = c^{\lambda,1,0}$ the flow on the contour line $\underbar{R} = h^{\lambda,1,0}$ is inwards. We make use of this observation in the proof of Section \ref{sec:proof_proof}.
%
% Now, we describe the bifurcation of this marginal case. The crucial difference is that $M_1^{\lambda,1,0}$ is replaced by a line of equilibria $h^{\lambda,1,0}(p,q) = c^{\lambda,1,0}=1$, which is the red line in Fig. \ref{fig:hn0m1}. As a result, each of the parabolae emanated from $M_0^{\lambda,1,0}$ lands at a point among these equilibria. $\vec{X}_{02}$ is immersed on $q=0$ plane distinctively from all other cases and the target orbit in particular lands at the $q$-intercept of the line of equilibria. To compare this observation to the statement of Theorem \ref{thm:1}, the target orbit does not connect $M_0^{ \lambda,1,0}$ to $M_1^{ \lambda,1,0}$ but to this $q$-intercept. This observation does not spoil our proof in Section \ref{sec:proof_proof} because we assert the persistence of the critical manifold not the target orbit.



\vfil\eject

\section{Existence via Geometric theory of singular perturbations} \label{sec:proof}
This section is devoted to proving the existence of a heteroclinic orbit $\chi(\eta)$ with limiting behavior as determined in the preceding section.

\begin{theorem} \label{thm1}
Let $H$ be a domain for the tuple $(\lambda,\alpha,m,n)\in\mathbb{R}^4$ defined by restricting $(\alpha,m,n)$ to take values in the range
\eqref{eq:paramrange} and $\lambda$ to satsfy the bound
 \begin{align*}
%  \alpha>0\quad&\text{(thermal softening)},\\
%  m>-1 \quad&\text{(strain softening/hardening)}, \\%\label{eq:a1}\\
%  n>0 \quad&\text{(strain rate sensitivity)},\\ %\label{eq:a2}\\
%  -\alpha+m+n<0 \quad&\text{(net softening)}, \\%\label{eq:a3}\\
  0< \lambda < \frac{2(\alpha-m-n)}{1+m+n}\left(\frac{1+m}{1+m+n}\right) \quad&\text{(localizing rate bound)}. %\label{eq:a4}
\end{align*}
 For $(\lambda,\alpha,m,0) \in H$, there is $n_0( \lambda,\alpha,m)$ such that for $n \in [0, n_0)$ and $(\lambda,\alpha,m,n) \in H$ System \eqref{eq:slow}  admits a heteroclinic orbit $\chi^{\lambda,\alpha,m,n}(\eta)$ joining equilibrium $M_0^{\lambda,\alpha,m,n}$ to equilibrium $M_1^{\lambda,\alpha,m,n}$ with the property that
    \begin{align} \label{eq:rapid}
%         &\chi(\eta) \rightarrow M_1 \quad \text{as $\eta \rightarrow \infty$ and} \\
        e^{-2\eta}\big(\chi^{\lambda,\alpha,m,n}(\eta) - M_0^{\lambda,\alpha,m,n}\big) \rightarrow \kappa X_{01}^{\lambda,\alpha,m,n} \quad \text{as $\eta \rightarrow -\infty$ for some $\kappa\ne0$}.
    \end{align}
\end{theorem}

The heteroclinic orbit $\chi^{\lambda,\alpha,m,n}(\eta)$ is achieved via {\it geometric singular perturbation theory}; the presence of the small parameter $n>0$ in the left-hand-side of \eqref{eq:slow}$_3$ elucidates the {\it fast-slow} structure of the system, having $r$ as a fast variable and the rest as slow variables.  
In the interest of the reader, we present some preliminary steps in detail. Experts with Geometric theory of singular
perturbations may wish to proceed directly to Sections \ref{sec:singorb}, \ref{sec:thmproof}. 

Recall that \eqref{eq:slow} accounts for a family of dynamical systems parametrized by $(\lambda,\alpha,m,n)$; the heteroclinic orbit will be achieved respectively for each admissible $(\lambda,\alpha,m,n)$. To simplify notations we suppress the dependence on $\lambda$, $\alpha$, and $m$ but retain
the dependence on $n$.

\subsection{Invariant manifold theory and geometric singular perturbation theory}\label{sec:singpert}
We quickly state some rudiments of the geometric singular perturbation theory that are in use, following definitions from \cite{fenichel_asymptotic_1977,fenichel_geometric_1979}. %Slightly different definitions are found in \cite{HPS_1977}. %
We will use \cite[Theorem 12.2]{fenichel_geometric_1979}, which is wrapped up in the form of \cite[Theorem 2.2]{Sz1991}, and \cite[Theorem 3.1]{Sz1991}.

Let $r\ge2$ and let $X$ be a $C^{r}$ vector field in $\mathbb{R}^d$. $\bar{\Lambda}=\Lambda \cup \partial \Lambda$ is a compact, connected $C^{r+1}$ manifold in $\mathbb{R}^d$. $F^t: \mathbb{R}^d \mapsto \mathbb{R}^d$ denotes the time $t$-map associated with the vector field $X$ and $DF^t$ denotes its differential. 

$\bar{\Lambda}$ is said to be overflowing invariant under $X$ if for every $m\in\bar{\Lambda}$ and $t\le0$, $F^t(m)\in \bar{\Lambda}$ and $X$ is pointing strictly outward on $\partial \Lambda$. $T \mathbb{R}^d|\Lambda$ denotes the tangent bundle of $\mathbb{R}^d$ along $\Lambda$ and $T\Lambda$ denotes the tangent bundle of $\Lambda$. A subbundle $E\subset T\mathbb{R}^d|\bar{\Lambda}$ is said to be negatively invariant if $E\supset DF^t(E)$ for all $t\le0$. Let $E\subset T\mathbb{R}^d|\bar{\Lambda}$ be a subbundle that is negatively invariant and contains $T\Lambda$. With a given such $E$,  $T \mathbb{R}^d|\bar\Lambda$ then splits into $T\mathbb{R}^d|\bar{\Lambda} =E\oplus E'= T\Lambda\oplus N\oplus E'$, where $N\subset E$ is any complement of $T\Lambda$ in $E$ and $E'\subset T\mathbb{R}^d|\bar{\Lambda}$ is any complement of $E$ in $T\mathbb{R}^d|\bar{\Lambda}$.  

With those introduced, we probe the separabilities of the three subbundles by their asymptotic growth rates backward in time: Let $m\in \bar{\Lambda}$, $t\le0$ and $v^0 \in T_m \Lambda$; $w^0\in N_m$; $x^0\in E'_m$; $v^t = DF^t(m)v^0$; $w^t = \pi^N DF^t(m)w^0$; $x^t = \pi^{E'}DF^t(m)x^0$,
% \begin{align*}
%  v^0 &\in T_m M, \quad w^0\in N_m, \quad x^0\in E'_m,\\
%  v^t &= DF^t(m)v^0, \quad w^t = \pi^N DF^t(m)w^0, \quad x^t = \pi^{E'}DF^t(m)x^0, \quad \text{where $\pi^N$ and $\pi^{E'}$ are bundle projections.}
% \end{align*}
where $\pi^N$ and $\pi^{E'}$ are bundle projections onto $N$ and $E'$ respectively.
Now we define at most five rates for each $m\in \bar{\Lambda}$:
\begin{align*}
 \nu^s(m) &\triangleq \inf \Big\{\nu>0 \: : \: \frac{1}{|x^{-t}|} = o(\nu^t) \quad \text{as $t \rightarrow \infty$} \quad \forall x^0\in E'_m\Big\}, \quad \text{if $\nu^s(m)<1$, define}\\
 \sigma^s(m) &\triangleq \inf \Big\{\sigma>0 \: : \: |v^{-t}| = o(|x^{-t}|^\sigma) \quad \text{as $t \rightarrow \infty$} \quad \forall x^0\in E'_m, v^0\in T_m\Lambda\Big\},
\end{align*}
\begin{align*}
 \alpha^u(m) &\triangleq \inf \Big\{\alpha>0 \: : \: |w^{-t}| = o(\alpha^t) \quad \text{as $t \rightarrow \infty$}\quad \forall w^0\in N_m\Big\}, \quad \text{if $\alpha^u(m)<1$, define}\\
 \rho^u(m) &\triangleq \inf \Big\{\rho>0 \: : \: \frac{|w^{-t}|}{|v^{-t}|} = o(\rho^t) \quad \text{as $t \rightarrow \infty$} \quad \forall w^0\in N_m, v^0\in T_m\Lambda\Big\},\quad \text{if $\rho^u(m)<1$, define}\\
 \tau^u(m) &\triangleq \inf \Big\{\tau>0 \: : \: |\hat{v}^{-t}| = o\left(\Big(\frac{|v^{-t}|}{|w^{-t}|}\Big)^{\tau}\right) \quad \text{as $t \rightarrow \infty$} \quad \forall w^0\in N_m, v^0\in T_m\Lambda,\hat{v}^0\in T_m\Lambda\Big\}.
\end{align*}

\begin{definition} \label{def:over}
Let $\bar{\Lambda}=\Lambda \cup \partial\Lambda$  an overflowing invariant manifold and $E$ a subbundle over it be given as above. We say an overflowing invariant manifold $\bar{\Lambda}$ satisfies assumptions \eqref{eq:A} and \eqref{eq:B}, $r'\le r-1$, with the subbundle $E$ if for all $m\in \bar{\Lambda}$ the type numbers
\begin{align}
\nu^s(m)&<1, \quad \sigma^s(m)<\frac{1}{r}, \label{eq:A}\tag{$A_r$}   \\
\alpha^u(m)&<1, \quad \rho^u(m)<1, \quad \tau^u(m)<\frac{1}{r'}. \label{eq:B}\tag{$B_{r'}$}  
%\quad \text{for all $m\in \bar{\Lambda}$}.
\end{align}
\end{definition}
\begin{remark}
 \cite{fenichel_asymptotic_1977} considered more general rate assumptions without \eqref{eq:A} but requiring two additional rates $\rho_2<1$ and $\tau_2<\frac{1}{r'}$. \eqref{eq:A} and \eqref{eq:B} are stronger conditions to have them valid. %.implies the conditions for $\rho_2$ and $\tau_2$.
\end{remark}
\begin{definition}[Normally Hyperbolic Invariant Manifold] \label{def:nhim}
 Let $\Lambda$ be a compact manifold without boundary, invariant under $X$. Let $E^s$ and $E^u$ be subbundles of $T \mathbb{R}^d|\Lambda$ such that $E^s + E^u = T \mathbb{R}^d|\Lambda$, $E^s\cap E^u=T\Lambda$, $E^u$ is negatively invariant under $X$ and $E^s$ is so under $-X$. We say $\Lambda$ is $r$-normally hyperbolic if $\Lambda$ is an overflowing invariant manifold with a subbundle $E^u$ satisfying the rate assumptions \eqref{eq:A} and $\Lambda$ is so with $E^s$ under $-X$. 
\end{definition}
The two notions for a certain invariant manifold in Definition \ref{def:over} and \ref{def:nhim} are for a certain type of persistence theorem under the perturbation of the vector field. Before we introduce the persistence theorem by Fenichel, we introduce the notion of the transversal intersection of two submanifolds in a phase space $\mathcal{M}$.
\begin{definition}[Transversal Intersection]  (\cite[Definition 3.1]{Sz1991})
 Let ${\mathcal{M}}_1$ and ${\mathcal{M}}_2$ be submanifolds of a manifold ${\mathcal{M}}$. The manifolds ${\mathcal{M}}_1$ and ${\mathcal{M}}_2$ intersect transversally at a point $m\in{\mathcal{M}}_1\cap {\mathcal{M}}_2$ iff 
 $$T_m{\mathcal{M}} =  T_m{\mathcal{M}}_1+T_m{\mathcal{M}}_2$$
 holds, where $T_m\mathcal{M}$ denotes the tangent space of the manifold $\mathcal{M}$ and similarly for $\mathcal{M}_1$ and $\mathcal{M}_2$.
\end{definition}

The persistence theorem is specialized for the dynamical system that has the {\it fast-slow} structure, 
\begin{equation} \label{eq:fast-slow}
 \left\{
 \begin{aligned}
  \dot{x}&=f(x,y,\epsilon),\\
  \epsilon\dot{y}&=g(x,y,\epsilon),
 \end{aligned}\right. \quad \text{where $\epsilon \in (-\epsilon_0,\epsilon_0), \epsilon_0>0$ small, $x\in \mathbb{R}^\ell$, $y\in \mathbb{R}^k$, $\ell+k=d$.}
\end{equation}
We say $x$ is a slow variable and $y$ is a fast variable. We assume $f$ and $g$ have enough smoothness in their domains of definitions. The meanings of the term are indicated by the following two limiting problems,
%     \hspace{3em} {Reduced Problem} \hspace{7em} Layer Problem $(\cdot)' = \frac{d}{d(t/\epsilon)} = \frac{d}{d\tilde{t}}.$
\begin{equation*} %\label{eq:reduced}
 \text{(Reduced Problem)}\quad\left\{
 \begin{aligned}
    \dot{x} &= f(x,y,0),\\
    0&= g(x,y,0),
 \end{aligned}\right. 
 \hspace{2.5em}
 \text{(Layer Problem)} \quad  
 \left\{
 \begin{aligned}
    x'&= 0,\\
    y'&= g(x,y,0), \quad (\cdot)' = \frac{d}{d(t/\epsilon)}.% = \frac{d}{d\tilde{t}}.
 \end{aligned}\right. 
\end{equation*}
% where the former describes the dynamics under the limiting assumption the fast variables $y$ have arrived equilibrium and $x$ evolves slowly, and the latter describes that of the fast variables $y$ relaxing towards equilibrium manifold keeping the slow variables $x$ unchanged. Formally the latter describes the initial layer behavior as its name indicates.

The zeroset $\mathcal{S}$ of $g(x,y,0)$ defines a manifold where the orbits of the Reduced problem are restricted. This is in general not realized as a graph but can have many branches. This manifold consists of equilibria of the Layer problem. We consider 
\begin{align*}
 \mathcal{S}&\subset \Big\{ (x,y)\:\Big|\: g(x,y,0)=0\Big\},\\
 \mathcal{S}_R&\subset \Big\{ (x,y)\in \mathcal{S} \:\Big|\: \text{$D_y g(x,y,0)$ has the full rank $k$}\Big\} \quad \text{open},\\
 \mathcal{S}_H&\subset \Big\{ (x,y)\in \mathcal{S}_R \:\Big|\: \text{all eigenvalues of $D_y g(x,y,0)$ have nontrivial real parts}\Big\}\quad \text{open}. 
\end{align*}
On $\mathcal{S}_R$, the equation $0=g(x,y,0)$ is locally solvable and we speak of the reduced vector field $X_R$ on slow variables. (See equation (7.8) in \cite{fenichel_geometric_1979}.) %A compact  $K \subset \mathcal{S}_H$ is normally hyperbolic to the Layer problem.

Next, we use Fenichel's Theorems in the form of \cite[Theorem 2.2]{Sz1991}. In particular this extended version with overflowing invariant manifold needs \cite[Theorem 3]{fenichel_asymptotic_1977}. % and \cite[Theorem 2.2]{Szmolyan}. 
We omit the statements, but the result states the persistence properties of the compact set $K\subset\mathcal{S}_H$ and its stable and unstable manifolds under a small perturbation. The upshot of the theorem is that any $\mathcal{N}\hookrightarrow K$ in the reduced phase space that is normally hyperbolic invariant under the reduced vector field $X_R$ persists under the perturbation in a suitable sense. The stable and unstable manifolds of $\mathcal{N}$ has local liftings to the unreduced phase space and persist under the perturbation as well. In particular, as far as the perturbations of $\mathcal{N}$ and its unstable manifold are concerned, it is enough to have $\mathcal{N}$ overflowing invariant as in Definition \ref{def:over} with its center-unstable bundle.

Ingredients acquired from the Fenichel's Theorem enable one to compose a variety of geometric arguments, such as a transversal intersection, to achieve solutions. The one used here belongs to one of the simplest settings \cite[Theorem 3.1]{Sz1991}: We take a simply connected branch of  $\mathcal{S}_H$ and a compact subset $K\subset \mathcal{S}_H$ will be chosen in the Section \ref{sec:choice}. We pick $\mathcal{N}_0$ and $\mathcal{N}_1$ in $K$. The heteroclinic orbit is achieved by the transversal intersection of the unstable manifold of $\mathcal{N}_0$ and the stable manifold of $\mathcal{N}_1$ in $K$. This transversal intersection in $K$ then lifts to that in the unreduced phase space. (See \cite{Sz1991}.) %, when $\mathcal{N}_0$ and $\mathcal{N}_1$ are taken from the same branch $K$, the transversality in the reduced phase space constitutes the same in the unreduced phase space. 

\subsection{Singular orbits for the inviscid system with $n=0$}\label{sec:singorb}

Let us instantiate the perturbation theory for \eqref{eq:slow}. We take two normally hyperbolic manifolds %of the Reduced problem \eqref{eq:slow02}
$\mathcal{N}_0$ and $\mathcal{N}_1$, which are simply the equilibrium points $M_0$ and $M_1$. The goal of this section is to establish the transversal intersection of the $\mathcal{N}_0^u$, the unstable manifold of $\mathcal{N}_0$ in $pqrs$-space, and $\mathcal{N}^s_1$, the stable manifold of $\mathcal{N}_1$.  

Bunch of symbols, following \cite{Sz1991}, are introduced here. %that applies for the entire of this paper. 
$K\subset\mathcal{S}_H$ denotes a compact critical manifold we choose later. The reduced problem is defined in $K$ and $X_R$ denotes the reduced vector field. $W_0^u$ denotes the reduced unstable manifold of $\mathcal{N}_0$ and $W_1^s$ denotes the reduced stable manifold of $\mathcal{N}_1$ restricted in $K$. The reduced vector field restricted in $K$ may as well have a finer invariant manifold $\mathcal{N}' \hookrightarrow$ that is normally hyperbolic to it. As usual $\mathcal{N'}$, $W^u(\mathcal{N}')$, and $W^s(\mathcal{N}')$ will be the manifold, its unstable, and stable manifold embedded in $K$. Inside of $W^u(\mathcal{N}')$ are foliations $\mathcal{F}^u_x$ (See \cite[Theorem 12.2]{fenichel_geometric_1979}), meaning that a foliation that passes $x\in W^u(\mathcal{N}')$. $\mathcal{F}^s_x$ is similarly for $W^s(\mathcal{N}')$.
All objects for $n>0$ that are persistantly continued by the invariant manifold theory are set by the superscript $n$, for examples $K^n$, $N_0^n$, $N_0^{u,n}$, $W_0^{u,n}$, $M_0^n$, $\cdots$. 

Having notations introduced, we we write the Reduced problem and the Layer problem for \eqref{eq:slow} :
\begin{equation}\label{eq:slow0} \tag{R}
 \begin{aligned}
%  r &={r}(p,q,s,n=0) \triangleq \frac{ \frac{\alpha-m}{\lambda(1+\alpha)}a - q }{  \frac{\alpha-m}{\lambda(1+\alpha)} + \lambda p + \frac{\alpha}{\lambda}\big(s- \frac{1+m}{1+\alpha}\big)},\\% \quad \text{$={r}(0)$ for simplicity },\\
 \dot{p} &=p\Big(\frac{1}{\lambda}({r}-a) + 2- \lambda p {r} -q\Big),\\% & &\bigg(= p\Big(\frac{D}{\lambda(1+\alpha)}({r}-a_0) + \frac{\alpha}{\lambda}{r}\big(s- \frac{1+m}{1+\alpha}\big) \Big)\bigg),\\
 \dot{q} &=q\Big(1 -\lambda p {r} -q\Big) + b p {r},\\% & &\bigg(=q\Big(\frac{\alpha-m}{\lambda(1+\alpha)}({r}-r_1) + \frac{\alpha}{\lambda}{r}\big(s- \frac{1+m}{1+\alpha}\big) \Big) + b p {r}\bigg),\\
 0&=r\Big(\frac{\alpha-m}{\lambda(1+\alpha)}(r-a) + \lambda pr + q +\frac{\alpha}{\lambda}r\big(s- \frac{1+m}{1+\alpha}\big)\Big),\\
 \dot{s} &=s\Big(\frac{\alpha-m}{\lambda(1+\alpha)}({r}-a) + \lambda p{r} + q - \frac{1}{\lambda}{r}\big(s- \frac{1+m}{1+\alpha}\big)\Big),% & &\bigg(= -\frac{1+\alpha}{\lambda}{r}s\big(s- \frac{1+m}{1+\alpha}\big)\bigg).
 \end{aligned}
\end{equation}
is the Reduced problem while
\begin{equation} \label{eq:fast0} 
 \begin{aligned}
 {p}' &=0, \quad {q}' =0, \quad {r}' =r\Big(\frac{\alpha-m}{\lambda(1+\alpha)}(r-a) + \lambda pr + q +\frac{\alpha}{\lambda}r\big(s- \frac{1+m}{1+\alpha}\big)\Big), {\red  \quad{s}' =0 },
 \end{aligned}
\end{equation}
is the Layer problem, where $(\cdot)'= \frac{d}{d\tilde{\eta}} = \frac{d}{d(\eta/n)}$ denotes differentiation with respect to the fast independent variable $\tilde{\eta}$. The zero-set of
\begin{equation}
g(p,q,r,s)\triangleq r\Big(\frac{\alpha-m}{\lambda(1+\alpha)}(r-a) + \lambda pr + q +\frac{\alpha}{\lambda}r\big(s- \frac{1+m}{1+\alpha}\big)\Big)\label{eq:zeroset} 
\end{equation}
consists entirely of the equilibria of \eqref{eq:fast0}. %We simply take one branch $\mathcal{S}_H$ and $K\subset\mathcal{S}_H$ compact, which will be called a {\it critical manifold}. 

\subsubsection{Choice of the critical manifold $K$} \label{sec:choice}
The algebraic equation $g(p,q,r,s)=0$ specifies the three dimensional hypersurfaces. In particular, away from $r\equiv0$ plane, one obtains the hypersurface by the graph. The function $\hat{r}(p,q,r)$ can be set to be
\begin{equation*}
\hat{r}(p,q,s) = \frac{ \frac{\alpha-m}{\lambda(1+\alpha)}a - q }{  \frac{\alpha-m}{\lambda(1+\alpha)} + \lambda p + \frac{\alpha}{\lambda}\big(s- \frac{1+m}{1+\alpha}\big)}
\end{equation*}
or implicitly
\begin{equation}
\frac{\alpha-m}{\lambda(1+\alpha)}(\hat{r}-a) + \lambda p\hat{r} + q +\frac{\alpha}{\lambda}\hat{r}\big(s- \frac{1+m}{1+\alpha}\big)=0. \label{eq:implicit}
\end{equation}

\begin{figure}[ht]
 \centering
  \psfrag{p}{\scriptsize $p$}%=\frac{\gamma}{\sigma}$}
  \psfrag{q}{\scriptsize~~~$q$}%=n\frac{v}{\sigma}$}
  \psfrag{s}{\scriptsize $s-\frac{1+m}{1+\alpha}$}%=n\frac{v}{\sigma}$}
  \psfrag{x0}{\scriptsize $M_0$}
  \psfrag{x1}{\scriptsize $M_1$}
  \psfrag{R1}{\scriptsize $\hat{r}(p,q,s)=a$}
  \psfrag{R2}{\scriptsize $\hat{r}(p,q,s)=R_1$}
  \psfrag{R3}{\scriptsize $\hat{r}(p,q,s)=R_2$}
%
%   \subfigure[Domain $G$ of a Graph]{
%   \includegraphics[width=5cm]{trapezoid.eps}\label{fig:flow0b}
%   }
%   \quad\quad
%   \subfigure[Affine level sets $\hat{r}(p,q,s)=R$ in $pqs$-space]{
  \includegraphics[width=6cm]{Affine.eps}
%   }
  \caption{Affine level sets $\hat{r}(p,q,s)=R$, $0\le R\le a$  in $pqs$-space} \label{fig:affine}
\end{figure}

Level sets of $\hat{r}$ depicted in $pqs$-space envisions the hypersurface. Level sets are checked to be affine, from the implicit formula \eqref{eq:implicit}. Fig. \ref{fig:affine} illustrates a few marked affine surfaces of $\hat{r}(p,q,s)=R$, in the range $0\le R\le a$. When $R=a(=r_0)$, it passes the origin, which is the equilibrium point $M_0$. As $R$ decreases the affine level sets sweep out the positive $p,q$ sector. The surface meets the other equilibrium $M_1$ when $R=r_1$. $R$ then further decreases until it touches the $r\equiv0$ plane.




The critical manifold $K$ is then chosen in considerations of the function $\hat{r}(p,q,r)$. We choose the domain of $\hat{r}$ as a certain trapezoid in $pqs$-space as in the below, and then $K\triangleq\big(D,\hat{r}(D)\big)$.
\begin{align*}
 D &\triangleq \left\{ \: (p,q,s) \: \Big| \:  p\ge-\epsilon, ~~ |q|\le2, ~~ \left|s-\frac{1+m}{1+\alpha}\right| \le \frac{1}{2}\min\left\{\frac{\alpha-m}{\alpha(1+\alpha)},\frac{1+m}{(1+\alpha)}\right\},\right.  \\
 &\left. \hat{r}(p,q,s)\ge \frac{1}{2}\min\{1,r_1\}\right\}.
\end{align*}
$\epsilon$ will be taken sufficiently small right below.
% \begin{align*}
%  K &\triangleq \left\{ \: (p,q,r,s) \: \big| \:  p\ge-\epsilon, ~~ |q|\le2, ~~ \left|s-\frac{1+m}{1+\alpha}\right| \le \frac{1}{2}\min\left\{\frac{\alpha-m}{\alpha(1+\alpha)},\frac{1+m}{(1+\alpha)}\right\}, \right. \\
%  &\left. \hat{r}(p,q,s)\ge \frac{1}{2}\min\{1,r_1\},~~ r=\hat{r}(p,q,s), ~~\hat{r}(p,q,s) = \frac{ \frac{\alpha-m}{\lambda(1+\alpha)}a - q }{  \frac{\alpha-m}{\lambda(1+\alpha)} + \lambda p + \frac{\alpha}{\lambda}\big(s- \frac{1+m}{1+\alpha}\big)}\: \right\}.
% \end{align*}
\begin{figure}[ht]
 \centering
  \psfrag{p}{\scriptsize \hskip -2pt $p$}%=\frac{\gamma}{\sigma}$}
  \psfrag{q}{\scriptsize~~~$q$}%=n\frac{v}{\sigma}$}
  \psfrag{s}{\scriptsize $s-\frac{1+m}{1+\alpha}$}%=n\frac{v}{\sigma}$}
  \psfrag{x0}{\scriptsize \hskip -4pt$M_0$}
  \psfrag{x1}{\scriptsize $M_1$}
  \psfrag{K}{\scriptsize \hskip -85pt $\hat{r}(p,q,s)= \frac{1}{2}\min\{1,r_1\}$}
%   \psfrag{R1}{\scriptsize $\hat{r}(p,q,s)=a$}
%   \psfrag{R2}{\scriptsize $\hat{r}(p,q,s)=R_1$}
%   \psfrag{R3}{\scriptsize $\hat{r}(p,q,s)=R_2$}
  \includegraphics[width=5cm]{trapezoid.eps}
%   }
  \caption{The trapezoid $D$, the domain of the graph.} \label{fig:D}
\end{figure}
In particular, $K$ is chosen so that $(i)$ $M_0$ and $M_1$ are on $K$; $(ii)$ $s$ and $r=\hat{r}(p,q,s)$ have positive lower bound on $K$. See the trapezoid $D$ in Fig. \ref{fig:D}. 

Now, we verify $K\subset \mathcal{S}_H$. 
\begin{proposition}
$K\subset \mathcal{S}_H$, i.e., the partial jacobian $\frac{\partial g}{\partial r}(p,q,r,s)|_{r=\hat{r}(p,q,s)} >0 $ for all $(p,q,r,s)\in K$. %$K$ is normally hyperbolic with respect to the Layer problem \eqref{eq:fast0}.
\end{proposition}
\begin{proof}
 %Let $g(p,q,r,s) \triangleq r\Big(\frac{\alpha-m}{\lambda(1+\alpha)}(r-a) + \lambda pr + q +\frac{\alpha}{\lambda}r\big(s- \frac{1+m}{1+\alpha}\big)\Big)$, the right hand side of $\eqref{eq:fast0}_3$.  We need to show that $K\subset \mathcal{S}_H$, i.e., the partial jacobian $\frac{\partial g}{\partial r}$ is a nontrivial real number for all $m\in K$.
 \begin{align*}
 \left.\frac{\partial g}{\partial r}\right|_{K} &= \Big(\frac{\alpha-m}{\lambda(1+\alpha)}(\hat{r}-a) + \lambda p\hat{r} + q +\frac{\alpha}{\lambda}\hat{r}\big(s- \frac{1+m}{1+\alpha}\big)\Big) + \hat{r}\Big(\frac{\alpha-m}{\lambda(1+\alpha)} + \lambda p + \frac{\alpha}{\lambda}\big(s- \frac{1+m}{1+\alpha}\big)\Big)\\
 &= \hat{r}\Big(\frac{\alpha-m}{\lambda(1+\alpha)} + \lambda p + \frac{\alpha}{\lambda}\big(s- \frac{1+m}{1+\alpha}\big) \Big)\ge \frac{1}{2}\min\{1,r_1\}\Big(\frac{\alpha-m}{2\lambda(1+\alpha)} - \lambda \epsilon\Big).
 \end{align*}
 Taking $\epsilon < \frac{\alpha-m}{4\lambda^2(1+\alpha)}$, independently of $n$, is enough.
\end{proof}
\subsubsection{Nested invariant manifold structures in $K$}
The flow \eqref{eq:slow0}, strictly restricted on $K$, is further analyzed. The three dimensional flow
\begin{equation}\label{eq:slow02} \tag{$\text{R}^\prime$}
 \begin{aligned}
%   r &=\hat{r}(p,q,s,n=0) \triangleq \frac{ \frac{\alpha-m}{\lambda(1+\alpha)}a - q }{  \frac{\alpha-m}{\lambda(1+\alpha)} + \lambda p + \frac{\alpha}{\lambda}\big(s- \frac{1+m}{1+\alpha}\big)},\\% \quad \text{$=\hat{r}(0)$ for simplicity },\\
 \dot{p} &= p\Big(\frac{D}{\lambda(1+\alpha)}(\hat{r}-a_0) + \frac{\alpha}{\lambda}\hat{r}\big(s- \frac{1+m}{1+\alpha}\big) \Big),\\
 \dot{q} &=q\Big(1 -\lambda p \hat{r} -q\Big) + b p \hat{r},\\% & &\bigg(=q\Big(\frac{\alpha-m}{\lambda(1+\alpha)}(\hat{r}-r_1) + \frac{\alpha}{\lambda}\hat{r}\big(s- \frac{1+m}{1+\alpha}\big) \Big) + b p \hat{r}\bigg),\\
%  0&=r\Big(\frac{\alpha-m}{\lambda(1+\alpha)}(r-a) + \lambda pr + q +\frac{\alpha}{\lambda}r\big(s- \frac{1+m}{1+\alpha}\big)\Big),\\
 \dot{s} &= -\frac{1+\alpha}{\lambda}\hat{r}s\Big(s- \frac{1+m}{1+\alpha}\Big).
 \end{aligned}
\end{equation}
appended by the other dimension with the component $r=\hat{r}(p,q,s)$ is the flow of \eqref{eq:slow0}. 

It is necessary to reveal a few finer invariant structures of the reduced flow \eqref{eq:slow02}, where the invariant manifold theory is equally applied and thus several of which will be persistent as well. This will be crucial ingredients of our arguments. To summarize, in this three dimensional reduced space $K$, we see embeddings 
$$ M_0, M_1 \quad \hookrightarrow \quad \text{$q$-axis} \quad \hookrightarrow \quad \text{$s\equiv\tfrac{1+m}{1+\alpha}$ plane} \quad \hookrightarrow \quad K$$
which consist of manifolds all of those are invariant under the flow \eqref{eq:slow02}. Indeed, on the plane $s\equiv\tfrac{1+m}{1+\alpha}$, \eqref{eq:slow02} decouples,
\begin{equation}\label{eq:slow03}
 \begin{aligned}
 \dot{p} &= p\Big(\frac{D}{\lambda(1+\alpha)}(\hat{r}-a_0)\Big),\\
 \dot{q} &= q\Big(1 -\lambda p \hat{r} -q\Big) + b p \hat{r}, 
 \end{aligned}
\end{equation}
where $\hat{r}=\hat{r}\big(p,q,\frac{1+m}{1+\alpha}\big)$. If further $p\equiv0$, yet another invariant line is seen and importantly this line contains the equilibrium points $M_0$ and $M_1$.

\begin{figure}[ht] 
 \centering
  \psfrag{p}{$p$}
  \psfrag{q}{$q$}
  \psfrag{s}{\hskip -2em $s-\tfrac{1+m}{1+\alpha}$}
  \psfrag{N2}{\scriptsize $N'$: segment in $q$-axis}
  \psfrag{F2}{\scriptsize normal foliations}% $\mathcal{F}(N_2)$}
  \psfrag{M0}{$M_0$}
  \psfrag{M1}{$M_1$}
  \psfrag{F0}{}%{\tiny \hskip -10em $M_0\xhookrightarrow{} N_2\xhookrightarrow{} \underbrace{N_2\oplus \mathcal{F}(N_2)}_{\triangleq N_1} $}
  \psfrag{F1}{}%{\tiny $M_1\xhookrightarrow{} W^s(M_1)\xhookrightarrow{} W^s(M_1)\oplus \mathcal{F}(N_0)$} 
 \includegraphics[height=5cm]{Hetero_Foliation} 
  \caption{Nested invariant manifold structures} \label{fig:HF} 
\end{figure}
Fig.s \ref{fig:HF} well-illustrate the rest of the program. The justification of the following descriptions will be the subject of the next section. $M_1$ is a stable node; the three dimensional volume surrounding $M_1$ in Fig. \ref{fig:HF} depicts its stable manifold. $M_0$ is a saddle; $M_0$ has two unstable dimensions in $s\equiv\tfrac{1+m}{1+\alpha}$, and has one stable dimension in its oblique direction. The vector aligned to $q$-axis and the one to the green orbit are two eigenvectors for the unstable dimensions. This explains how our heteroclinic orbit (the green one) appears in $K$. %The one to the $q$-axis always has the smaller eigenvalue $1$ to the other eigenvalue $2$. It is common idea in the invariant manifold theory to conceive the rapid orbit with greater eigenvalue as one of the foliations onto the slow orbit with smaller eigenvalue. This shall be clarified in the below.

Not all of the manifolds appeared are {\it normally hyperbolic} to the reduced vector field \eqref{eq:slow02}: $M_0$ and $M_1$ are hyperbolic equilibrium points; $\mathcal{N}'$, a segment of $q$-axis (the blue portion in Fig. \ref{fig:HF}) will be identified as an overflowing manifold satisfying the rate assumptions \eqref{eq:A} and \eqref{eq:B}; however, $s\equiv \tfrac{1+m}{1+\alpha}$ plane in general turned out not to fulfill the necessary rate assumptions. 



\subsubsection{Flow of \eqref{eq:slow02} completely analyzed.}%on $s\equiv\tfrac{1+m}{1+ \alpha}$}

In the phase space $K$ particularized by the marked invariant manifolds, the flow of \eqref{eq:slow02} is completely analyzable. We understand the overall flow by the one restricted on the invariant plane $s\equiv\tfrac{1+m}{1+\alpha}$ and the stable relaxation of the ambient space toward it. See Fig. \ref{fig:n0pqs}. Inside of the plane $s\equiv\tfrac{1+m}{1+\alpha}$ is characterizable by the planar dynamical systems theory.



\begin{figure}[ht] 
 \centering
%  \subfigure[Flow on and around $s\equiv\tfrac{1+m}{1+ \alpha}$ seen in $pqr$-space]{
 \psfrag{p}{$p$}
 \psfrag{q}{$q$}
 \psfrag{s}{$s-\tfrac{1+m}{1+ \alpha}$}
 \psfrag{x0}{\hskip -2pt $M_0$}
 \psfrag{x1}{\hskip -4pt $M_1$}
 \psfrag{B}{}
 \includegraphics[width=4cm]{flow0pqs0} 
%  }
 \caption{Flow on and around $s\equiv\tfrac{1+m}{1+ \alpha}$} \label{fig:n0pqs}
\end{figure}

\begin{remark}
That the invariant plane $s\equiv\tfrac{1+m}{1+ \alpha}$ has the one stable dimension in $K$ somehow falsely suggests that the plane might be normally hyperbolic. The rate assumption demands the strictly larger rate behavior in the normal direction and as a matter of fact, the rates are computed according to the negative eigenvalues at $M_1$ and in general the rate along the normal direction is not faster than those along the tangential directions. As consequences, we do not assert its persistence. 
\end{remark}

We turn to the reduced linear stability of $M_0$ and $M_1$ in $pqs$-space. To summarize, the expressions in Section \ref{sec:equil} still make senses for $n=0$, with the exceptions that those for the third eigenvalue $\mu_{03}$ and the third eigenvector $X_{03}$ of $M_0$ ($\mu_{13}$ and $X_{13}$ respectively of $M_1$) are not in use. Indeed when $n=0$, the flow is restricted on the three dimensional set $K$. %In the reduced phase space $K$, 

With those in mind, the following Lemma via the planar dynamical systems theory completely characterizes the flow on a triangle $T$ specified below in the plane. %first quadrant of the $pq$-plane. 

\begin{figure}[ht]
 \centering
  \psfrag{X01}{\scriptsize$X_{02}$}
  \psfrag{X02}{\scriptsize$X_{01}$}
%   \psfrag{A}{$A$}
%   \psfrag{T}{$T$}
%   \psfrag{x0}{\scriptsize $M_0$}
%   \psfrag{x1}{\scriptsize $M_1$}
%   \psfrag{r0}{\scriptsize $\hat{r}(p,q,s,0)=r_0$}
%   \psfrag{r1}{\scriptsize $\hat{r}(p,q,s,0)=r_1$}
  \psfrag{p}{\scriptsize $p$}%=\frac{\gamma}{\sigma}$}
  \psfrag{q}{\scriptsize~~~$q$}%=n\frac{v}{\sigma}$}
  \psfrag{s}{\scriptsize $s-\frac{1+m}{1+\alpha}$}%=n\frac{v}{\sigma}$}
%   \subfigure[Affine level sets $\hat{r}(p,q,s)=R$ in $pqs$-space]{
%   \psfrag{r}{\scriptsize$r$}%=\big(\sigma\gamma^{(1-n)}\big)^{\frac{1}{n}}$}
  \psfrag{CC}{\scriptsize\hskip 5pt$\frac{\alpha-m}{\lambda(1+\alpha)}\big(-\frac{1}{\lambda},a\big)$}
  \includegraphics[width=8cm]{geom0.eps} \label{fig:flow0}
%   \flushleft
%   }
%   \quad \quad
%   \subfigure[Flow in $pqs$-space for $n=0$]{
%   \psfrag{r}{\scriptsize$s-\frac{1+m}{1+\alpha}$}%=\big(\sigma\gamma^{(1-n)}\big)^{\frac{1}{n}}$}
%   \includegraphics[width=6cm]{flow0pqs.eps}\label{fig:flow0b}
%   }
  \caption{The schematic sketch of the flow on the invariant plane $s=\frac{1+m}{1+\alpha}$.   On $s=\frac{1+m}{1+\alpha}$, $M_0$ is an unstable node and $M_1$ is a stable node. Two directions of unstable subspaces of $M_0$ are denoted by $X_{01}$ and $X_{02}$; the straight lines emanating from the 
  point $\frac{\alpha-m}{\lambda(1+\alpha)}\big(-\frac{1}{\lambda},a\big)$ are the intersections of the level sets of $\hat{r}$ 
  with the plane $s=\frac{1+m}{1+\alpha}$; the curve in the fourth quadrant is the nullcline of the equation $\eqref{eq:slow02}_2$; the triangle $T$ is a 2-dimensional positively invariant set; the trapezoid $A$ is a 2-dimensional negatively invariant set; the $\omega$-limit set of any point in $T$ is $M_1$; 
  the $\alpha$-limit set of any point in $A$ is $M_0$; in particular there is a heteroclinic orbit (green one) that emanates from $M_0$ through $X_{01}$, the strongly unstable manifold of $M_0$.} 
\end{figure}

\begin{lemma} \label{lem:T}
 Let $T$ be the closed triangle on $s\equiv \frac{1+m}{1+\alpha}$ enclosed by $p=0$, $q=0$, and the level set $\hat{r}(p,q,\frac{1+m}{1+\alpha})= \frac{1}{2}\min\{1,r_1\}$ that is intersected by $s\equiv \frac{1+m}{1+\alpha}$. 
 Then $T\setminus M_0 \subset W^s(M_1)$.
\end{lemma}
\begin{proof}
$T$ is a two dimensional compact positively invariant set: (1) on $p=0$, $\nu = (1,0)$ and $X_R\cdot\nobreak\nu = \dot{p}=0$, where $X_R$ stands for the reduced vector field of \eqref{eq:slow02};
 (2) on $q=0$, $\nu = (0,1)$ and $X_R\cdot\nobreak\nu=\dot{q} = bp\hat{r}\ge0$ ($b$ in \eqref{eq:exponents} is always positive); lastly on the hypotenuse, let $\underbar{r}=\frac{1}{2}\min\{1,r_1\}$. The inward normal pointing the origin is $\nu = (-\lambda\underbar{r}, -1)$. We compute
  \begin{align}
  X_R\cdot\nu=-\lambda\underbar{r}\dot{p} -\dot{q}&= -\lambda \underbar{r}p \Big(1-\lambda \underbar{r}p -q + \frac{1}{\lambda}(\underbar{r}-a)+1\Big) - q(1-\lambda \underbar{r}p -q\big) - b \underbar{r}p \nonumber\\
  &= (1-\lambda \underbar{r}p -q)(-\lambda \underbar{r}p -q) -\underbar{r}p\Big((\underbar{r}-a)+\lambda+b\Big)\nonumber\\
  &= \left(\frac{\alpha-m}{\lambda(1+\alpha)}\right)^2(\underbar{r}-r_0)(\underbar{r}-r_1)+\underbar{r}p(1-\underbar{r})\ge \delta>0. \label{eq:affine}
 \end{align}

Having that shown, let $\Omega$ be an $\omega$-limit set of the orbit portrait from $x_0\in T\setminus M_0$. It is non-empty because $T$ is compact. It cannot contain $M_0$, because for the flow restricted in $T$, $M_0$ does not have a stable manifold. It cannot contain a periodic orbit because if so, there would be a fixed point in the interior of $T$ and this is not true. 
It cannot contain a separatrix cycle because $T$ has only two fixed points $M_0$ and $M_1$ and again $M_0$ does not have a stable manifold.  By Poincar\'e-Bendixson Theorem, the $\omega$-limit set is $M_1$.
\end{proof}


Now we are able to say the following: We call $\mathcal{F}^u_{M_0}\subset W^u_0$ the strongly unstable manifold of $M_0$ satisfying \eqref{eq:rapid} (thee green one in Fig. \ref{fig:HF}) that is characterizable by the Unstable manifold theorem for the hyperbolic fixed point. That $\mathcal{F}^u_{M_0}$ survives to end up arriving at $M_1$ follows by Lemma \ref{lem:T}, and this gives the proof for $n=0$ of Theorem \ref{thm1}. The following proposition phrases that better geometrically: the one dimensional manifold $\mathcal{F}^u_{M_0}\subset W^u_0$ intersects the three dimensional manifold $W_1^s(=W^s(M_1))$ transversally (see Fig. \ref{fig:HF}).

\begin{proposition} \label{prop:singular}
 Let $\mathcal{N}_0=M_0$, $\mathcal{N}_1=M_1$, $\mathcal{F}^u_{M_0}\subset W^u_0$ the strongly unstable manifold of $M_0$ satisfying \eqref{eq:rapid}, $W^s_1=\Phi_{-t_0}(W^s_{loc}(M_1))$, the time $-t_0$ image of the local stable manifold of $M_1$ for large enough $t_0<\infty$. Then $\mathcal{F}^u_{M_0}$ intersects $W^s_1$ transversally in $pqs$-space.
\end{proposition}
\begin{proof}[Proof of Proposition \ref{prop:singular}]
% Let $\mathcal{F}^u_{M_0}\subset W^u(M_0)$ be the most rapidly escaping orbit from $M_0$ satisfying \eqref{eq:rapid} that is characterizable by the Unstable manifold theorem for the hyperbolic fixed point. By Lemma \ref{lem:T}, $\mathcal{F}^u_{M_0}$ survives to end up arriving at $M_1$. This proves the existence of the heteroclinic orbit for $n=0$.
%
% Now let  Let $W^s_{loc}(M_1)$ be the local three dimensional stable manifold of $M_1$ given by the Stable manifold theorem. Let $W^s_1=\Phi_{-t_0}(W^s_{loc}(M_1))$ be the time $-t_0$ image of $W^s_{loc}(M_1)$. 
For large enough $t_0<\infty$, by Lemma \ref{lem:T} the orbit point $x\in \mathcal{F}^u_{M_0}$ must be attained in $W^s_1$ as an interior point. Therefore the tangent space ${T}_x W^s_1$ is the whole of ${T}_x \mathbb{R}^3$. Then the intersection with $\mathcal{F}^u_{M_0}$ is trivially transversal.
\end{proof}

\subsection{Persistence for $n>0$} \label{sec:thmproof}
Having set forth the critical manifold $K$ in $\mathcal{S}_H$ and the reduced vector field $X_R$ on $K$, the theorem of Fenichel holds in $K$; the family $K^n$ of {\it slow manifolds} persistently exist provided $n$ is sufficiently small. Now we show the finer hyperbolic structure of $\mathcal{F}^u_{M_0} \hookrightarrow K$. %It is common idea in invariant manifold theory to characterize the strongly unstable manifold as the unstable manifold of the complementary part in $W^u(M_0)$.

\begin{lemma} \label{lem:rapid}
 Let $\mathcal{N}_0=M_0$, $\mathcal{F}^u_{M_0}\subset W_0^u$ the strongly unstable manifold of $M_0$ satisfying \eqref{eq:rapid}. Then, for sufficiently small $n$, $\mathcal{F}^u_{M_0}$ perturbs in a $C^{r-1}$ manner to $\mathcal{F}^{u,n}_{M_0^n}$ the strongly unstable manifold of $M_0^n$ satisfying \eqref{eq:rapid}. 
\end{lemma}
\begin{proof}
 We pick the line segment $q\in [- \frac{1}{2}, \frac{1}{2}]$ on $q$-axis, which we will denote ${\mathcal{N}}'$. This is an one dimensional orbit portraits in $W_0^u$ that is not strongly unstable. We claim that ${\mathcal{N}}'$ is an overflowing invariant manifold as in Definition \ref{def:over} of the reduced problem. More precisely, it satisfies \eqref{eq:A} and \eqref{eq:B} with $r'=r-1$ and $E$ the tangent $pq$-plane. 
 
 From \eqref{eq:slow03}, $\dot{q}=q(1-q)$ on $q$-axis, it is clear that ${\mathcal{N}}'$ is overflowing invariant. Let $E$ be $pq$-plane along ${\mathcal{N}}'$ and $E'$ be the lines parallel to the $s$-axis. Then, $T \mathbb{R}^3|{\mathcal{N}}'$ splits into three one dimensional bundles $T{\mathcal{N}}'\oplus N \oplus E'$ with $N$ complementary to $T{\mathcal{N}}'$ in $E$ such that $N_{M_0}$ is parallel to $X_{01}$.  %defined by the three coordinate basis in $pqs$-space, i.e., $\mathbf{e}_q\in T\Lambda$, $\mathbf{e}_p\in N$, and $\mathbf{e}_s\in E'$. 
 Asymptotic rates are determined at $M_0$ by the eigenvalues of $M_0$. At $M_0$, $E'_{M_0}$ is the stable subspace with eigenvalue $-\mu_{04}$ and $N_{M_0}$ and $T_{M_0}{\mathcal{N}}'$ are the unstable ones with $\mu_{01}=2$ and $\mu_{02}=1$ respectively. From these, we compute 
 $$ \nu^s = e^{-\mu_{04}}, \quad\sigma^s = 0, \quad\alpha^u = e^{-2}, \quad\rho^u=e^{-1}, \quad\tau^u=0.$$

 Therefore, the strongly unstable manifold exists as an unstable $C^{r-1}$ family of manifolds $\mathcal{F}^u_x$ with $x=M_0$ of the reduced problem by \cite[Theorem 3]{fenichel_asymptotic_1977} and it is persistent in a basepoint-wise manner under the perturbation. 
\end{proof}
Persistence of the stable manifold $W_1^s (= W^s(M_1))$ is the classical stable manifold theorem. Now Theorem \ref{thm1} follows in the same way as in \cite[Theorem 3.1]{Sz1991} by the transversal intersection.

\begin{proof}[Proof of Theorem \ref{thm1}] 
%  By \cite[Theorem 3.1]{Sz1991}, In particular, $\mathcal{F}^u_{M_0}$ perturbs to $\mathcal{F}^u_{M_0^n}$, the most rapidly escaping orbit from $M_0^n$ by Lemma \ref{lem:rapid}
 By the theorem of Fenichel, $n_0$ can be taken sufficiently small so that for given $(\lambda,\alpha,m,0)\in H$ if $n \in [0, n_0)$ then $(\lambda,\alpha,m,n) \in H$ and  the system  \eqref{eq:slow} admits a transversal heteroclinic orbit joining equilibrium $M_0^{n}$ to equilibrium $M_1^{n}$: $\mathcal{F}^u_{M_0}$ perturbs to $\mathcal{F}^{u,n}_{M_0^n}$ by Lemma \ref{lem:rapid} and $W_1^s$ perturbs to $W_1^{s,n}$ and the transversal intersection is stable under the perturbation. 
\end{proof}



\vfil\eject
\section{Numerical computation of the heteroclinic orbits}
{\red new section}

We compute here the .....





\vfil\eject
\section{Emergence of localization}
\label{sec:localization}
% The dynamical system has a translation invariance, i.e., if $\chi(\eta)$ is a heteroclinic orbit, then so is a $\chi(\eta-\eta_0)$ for any $\eta_0\in \mathbb{R}$, implying that we have constructed in fact infinitely many heteroclinics. It turns out that all of them are relevant.  Besides of $\eta_0$, we have been using a few other parameters and this points to that we have constructed a family of solutions with a certain degrees of freedom. We precise relationships between parameters and the total number of independent parameters because not all of parameters are independent.
%
% To begin with, fix any referential heteroclinic orbit $Z(\eta)=\big(P(\eta),Q(\eta),R(\eta)\big)$, which is characterized by the unique constant $\bar{\kappa}$ in the expansion around the $M_0$ such that
% $$ Z(\eta) = \bar{\kappa}e^{2\eta}X_{02} + \text{higer-order terms as $\eta \rightarrow -\infty$.}$$%, with the associated constant $\bar{\kappa}$.}$$
% We set our heteroclinic orbit $\chi(\eta) = Z(\eta-\eta_0)$ and show how to fix $\eta_0$. From the proposition \ref{prop1},
% $$ \kappa = \Big(\frac{U(0)}{\Phi(0)}\Big)^{-n}\Phi(0)^{1+\frac{\alpha-n}{1+\alpha}}=\lim_{\eta \rightarrow -\infty} e^{-2\eta}\chi(\eta)  = \lim_{\eta \rightarrow -\infty} e^{-2\eta} Z(\eta-\eta_0) =\bar{\kappa}e^{-2\eta_0}.$$
% By treating $U(0)$ and $\Theta(0)$ as the primary parameters, $\eta_0$ is determined by
% \begin{equation}
%  \eta_0 = \frac{1}{2} \log\Big(\bar{\kappa} \Big(\frac{U(0)}{\Phi(0)}\Big)^{n}\Phi(0)^{-1-\frac{\alpha-n}{1+\alpha}}  \Big) = \frac{1}{2} \log\Big(\bar{\kappa} U(0)^n \Theta(0)^{-\alpha-\frac{1+\alpha}{1+n}}  \Big). \label{eq:eta0}
% \end{equation}
%
% Next, we check that
% $$ \frac{U(0)^{1+n}}{\Theta(0)^{1+\alpha}} = \lim_{\eta \rightarrow -\infty} r^{1+n} = c_0 = \frac{2}{D} + \frac{2+2n}{D}\lambda.$$
% Again, by treating $U(0)$ and $\Theta(0)$ as the primary parameters, we obtain
% \begin{equation}
%  \lambda = \frac{D}{2+2n}\frac{U(0)^{1+n}}{\Theta(0)^{1+\alpha}} - \frac{2}{2+2n}. \label{eq:lambda}
% \end{equation}
% Now, we find the ratio $\frac{U(0)^{1+n}}{\Theta(0)^{1+\alpha}}$ is restricted by \eqref{eq:r1posineq},
% \begin{equation} \label{eq:restriction}
%  \frac{2}{1+2\alpha-n} < \frac{U(0)^{1+n}}{\Theta(0)^{1+\alpha}} < \frac{2}{1+n}.
% \end{equation}
%
%
%
% The rest of the parameters $\alpha$ and $n$ are the material properties. In conclusion, for each of the material characterized by $\alpha$ and $n\ll1$, we have constructed the two-parameters family of heteroclinic orbits parameterized by $U(0)$ and $\Theta(0)$, where the localizing rate $\lambda$ and the translational factor $\eta_0$ are determined by \eqref{eq:lambda} and \eqref{eq:eta0} respectively, and the valid range of the ratio $\frac{U(0)^{1+n}}{\Theta(0)^{1+\alpha}}$ for the Theorem \ref{thm1} to apply is restricted by \eqref{eq:restriction}.


% \subsection{Asymptotic behavior of the solutions}
By transforming back with \eqref{eq:ORItoCAP}, \eqref{eq:CAPtoBAR}, \eqref{eq:BARtoTIL}, and \eqref{eq:pqrdef}, $\big(\Gamma(\xi),V(\xi),\Theta(\xi),\Sigma(\xi),U(\xi)\big)$ and \\$\big(\gamma(x,t),v(x,t),\theta(x,t),\sigma(x,t),u(x,t)\big)$ are recovered.
We replace $t \leftarrow t+1$ in the final expressions,
\begin{equation*}
\begin{aligned}
 \gamma(t,x) &= (t+1)^a\Gamma((t+1)^\lambda x), & v(t,x) &= (t+1)^b V((t+1)^\lambda x), &\theta(t,x) &= (t+1)^c \Theta((t+1)^\lambda x),\\
 \sigma(t,x) &= (t+1)^d \Sigma((t+1)^\lambda x), & u(t,x) &= (t+1)^{b+\lambda} U((t+1)^\lambda x),
\end{aligned}
\end{equation*}
so that we interpret $\big(\Gamma(\xi),V(\xi),\Theta(\xi),\Sigma(\xi),U(\xi)\big)=\big(\gamma(0,x),v(0,x),\theta(0,x),\sigma(0,x),u(0,x)\big)|_{x=\xi}$,  the initial states that give rise to the localization. For given material parameters $(\alpha, m, n)$, additionally two degrees of freedom fully accounts for the above two-parameters family of solutions. As explained in Section \ref{sec:twoparam}, the choices of $U(0)$ and $\Gamma(0)$ fix one solution and other boundary values and the localizing rate $\lambda$ are dependently decided. The valid ranges of $U(0)$ and $\Gamma(0)$ are such that the ratio $\frac{U(0)}{\Gamma(0)}$ is not too small and not too big, specifically,
 $$\frac{2(1+\alpha) -n}{D} < \frac{U(0)}{\Gamma(0)} < \frac{2(1+\alpha) -n}{D} + \frac{4(1+\alpha)(\alpha-m-n)(1+m)}{D(1+m+n)^2}.$$
The localizing rate $\lambda$ is then determined by \eqref{eq:lambda} accordingly in the range
$$0< \lambda < \frac{2(\alpha-m-n)}{1+m+n}\left(\frac{1+m}{1+m+n}\right).$$

In the following sections, we examine characteristics of these two-parameters family of initial states and discuss the emergence of localization from them as time proceeds.

\subsection{Properties of the self-similar solutions}
We state that each of the strain, strain-rate, and temperature has a small bump at the origin out of the asymptotically flat state, whose tip sizes are parameterized by $\Gamma(0)$ and $U(0)$. Accordingly, the velocity is an increasing function of $x$ that has a slightly steeper slope at origin than at the rest of the places; the stress is then the convex increasing function of $|x|$. These initial non-uniformities can be viewed as perturbations of the uniform shearing motion at a certain time.

% Now, we fix the two primary parameters $U(0)$ and $\Theta(0)$ and look into the
%
%
% In this section we contrast the initial non-uniformities $\big(\gamma(0,x),v(0,x),\theta(0,x),\sigma(0,x),u(0,x)\big)=\\ \big(\Gamma(\xi),V(\xi),\Theta(\xi),\Sigma(\xi),U(\xi)\big)|_{\xi=x}$ from the one snapshot at a certain time of the uniform shearing motion.
\begin{proposition} \label{prop:ss}
Let $\big(\Gamma(\xi),V(\xi),\Theta(\xi),\Sigma(\xi),U(\xi)\big)$ be the self-similar profiles that is defined by transformations of \eqref{eq:CAPtoBAR}, \eqref{eq:BARtoTIL}, and \eqref{eq:pqrdef} upon to the heteroclinic orbit $\chi(\eta)=\big(p(\eta),q(\eta),r(\eta),s(\eta)\big)$ constructed by Theorem \ref{thm1} in the valid range of the parameters $\Gamma(0)=\Gamma_0$ and $U(0)=U_0$ satisfying \eqref{eq:restriction}. Then it follows that:
 \begin{enumerate}
  \item[(i)] The self-similar profile achieves the boundary condition at $\xi=0$,
    \begin{equation*}
    {V}(0) = \Gamma_\xi(0) = \Theta_\xi(0)=\Sigma_\xi(0) = {U}_\xi(0)=0, \quad \Gamma(0)=\Gamma_0, \quad U(0)=U_0.
  \end{equation*}
  \item[(ii)] Its asymptotic behavior as $\xi \rightarrow 0$ is given by
  \begin{equation} \label{eq:ss_asymp0}
  \begin{aligned}
    \Gamma(\xi) -\Gamma_0 &= \Gamma^{''}(0)\frac{\xi^2}{2} + o(\xi^2), & \Gamma^{''}(0)&<0,\\
    \Theta(\xi) - c^{-\frac{1}{1+\alpha}}\Gamma_0^{\frac{m}{1+\alpha}} U_0^{\frac{1+n}{1+\alpha}} &= \Theta^{''}(0)\frac{\xi^2}{2} + o(\xi^2), & \Theta^{''}(0)&<0,\\
    \Sigma(\xi) - c^{\frac{\alpha}{1+\alpha}}\Gamma_0^{\frac{m}{1+\alpha}} U_0^{-\frac{\alpha-n}{1+\alpha}} &= \Sigma^{''}(0)\frac{\xi^2}{2} + o(\xi^2), & \Sigma^{''}(0)&>0, \\
    U(\xi) - U_0 &= U^{''}(0)\frac{\xi^2}{2} + o(\xi^2), & U^{''}(0)&<0,\\
    V(\xi) - U_0\xi &= U^{''}(0)\frac{\xi^3}{6} + o(\xi^3), & U^{''}(0)&<0.
  \end{aligned}
  \end{equation}
  \item[(iii)] Its asymptotic behavior as $\xi \rightarrow \infty$ is given by\\
  if $\mu_{11}\ne-1$, or $\mu_{11}=-1$ but $b=\lambda$,
  \begin{equation} \label{eq:ss_asymp1}
  \begin{aligned}
    \Gamma(\xi) &= \BO\big(\xi^{-\frac{1+\alpha}{\alpha-m-n}}), & V(\xi) &= \BO\big(1), &    \Theta(\xi) &= \BO\big(\xi^{-\frac{1+m+n}{\alpha-m-n}}),\\
   \Sigma(\xi) &= \BO\big(\xi), &   U(\xi) &= \BO\big(\xi^{-\frac{1+\alpha}{\alpha-m-n}})
  \end{aligned}
  \end{equation}
  otherwise
    \begin{equation} \label{eq:ss_asymp2}
  \begin{aligned}
    \Gamma(\xi) &= \BO\big(\xi^{-\frac{1+\alpha}{\alpha-m-n}}\big(\log\xi\big)^{\frac{1+\alpha}{D}}\big), & V(\xi) &= \BO\big(\big(\log\xi\big)^{-\frac{\alpha-m-n}{D}}\big),
    \\ 
        \Theta(\xi) &= \BO\big(\xi^{-\frac{1+m+n}{\alpha-m-n}}\big(\log\xi\big)^{\frac{1+m+n}{D}}\big),\\
   \Sigma(\xi) &= \BO\big(\xi\big(\log\xi\big)^{-\frac{\alpha-m-n}{D}}\big), &   U(\xi) &= \BO\big(\xi^{-\frac{1+\alpha}{\alpha-m-n}}\big(\log\xi\big)^{\frac{1+\alpha}{D}}\big)
  \end{aligned}
  \end{equation}
 \end{enumerate}

\end{proposition}
\begin{proof}
The proof of the Proposition \ref{prop1} and Remark \ref{rem:signs} contains $(i)$ and $(ii)$ and thus we are left to prove $(iii)$. In the similar fashion to \eqref{eq:alpha-expan}, any orbit $\psi(\eta)$ in the local stable manifold of $W^s(M_1)$ is characterized by the triple $(\kappa_1',\kappa_2',\kappa_3')$ in association with the asymptotic expansion
\begin{equation}
\begin{aligned}
 &\psi(\eta) -M_1\\
 &= \begin{cases} \kappa_1'e^{\mu_{11}\eta}X_{11} + \kappa_2'e^{\mu_{12}\eta}X_{12} + \kappa_4'e^{\mu_{14}\eta}X_{14} + \text{high order terms} & \text{if $\mu_{11}\ne-1$, or $\mu_{11}=-1$ but $b=\lambda$,}\\
 \kappa_1'\eta e^{\mu_{11}\eta}X_{11}' + \kappa_2'e^{\mu_{12}\eta}X_{12} + \kappa_4'e^{\mu_{14}\eta}X_{14} + \text{high order terms} & \text{otherwise }
 \end{cases}
\end{aligned}
\end{equation}
as $\xi \rightarrow \infty$. As it may be associated with the generalized eigenvector, we have separated cases as above.

Now, we have $q \rightarrow 1$, $r \rightarrow r_1$, $s \rightarrow s_1$ but $p \rightarrow 0$ and the leading order of $p$ is to be found. We can determinee the coefficient of $X_{11}$, above, because the $p$-component of the vectors $X_{12}$ and $X_{14}$ is $0$. Because the plane $p\equiv0$ is an invariant plane for a non-linear flow, triples of the form $(0,\kappa_2',\kappa_3')$ spans this invariant plane. Because our heteroclinic orbit $\chi(\eta)$ ventures out from the plane $p\equiv0$, $\kappa_1'$ of the $\chi(\eta)$ cannot be $0$.
This implies that the leading order of $p(\log\xi)$ is
$$p(\log\xi) = \begin{cases} \BO(\xi^{\mu_{11}}) & \text{if $\mu_{11}\ne-1$ or $\mu_{11}=-1$ but $b=\lambda$,}\\
 \BO(\xi^{\mu_{11}}\log\xi) & \text{otherwise}
 \end{cases}
 $$
as $\xi \rightarrow \infty$.

Asymptotics \eqref{eq:ss_asymp1} and \eqref{eq:ss_asymp2} are the straightforward calculations from the reconstruction formulas
\begin{align*}
 \tg&=p^{\frac{1+\alpha}{D}}r^{\frac{n}{D}}s^{\frac{\alpha}{D}}, & \tv &= \frac{1}{b} p^{-\frac{\alpha-m-n}{D}}qr^{\frac{n}{D}}s^{\frac{\alpha}{D}}, & \tth&=p^{\frac{1+m+n}{D}}r^{\frac{2n}{D}}s^{-\frac{1-m-n}{D}}, \\ \ts&=p^{-\frac{\alpha-m-n}{D}}r^{\frac{n}{D}}s^{\frac{\alpha}{D}},  & \tu&=p^{\frac{1+\alpha}{D}}r^{\frac{n}{D}+1}s^{\frac{\alpha}{D}},
\end{align*}
\eqref{eq:CAPtoBAR}, and \eqref{eq:BARtoTIL}.
%
% $p \sim \xi^{-\frac{1+n}{1+\alpha}}$
%
% $$ \tv \sim \xi^{-\frac{1+n}{D}}, \quad \tth \sim \xi^{-\frac{(1+n)^2}{(\alpha-n)D}}, \quad \ts \sim \xi^{\frac{1+n}{D}}, \quad \tu \sim \xi^{-\frac{(1+n)(1+\alpha)}{(\alpha-n)D}}$$
%
% $$ V \sim O(1), \quad \Theta \sim \xi^{-\frac{1+n}{\alpha-n}}, \quad \Sigma \sim \xi, \quad U \sim \xi^{-\frac{1+\alpha}{\alpha-n}}.$$
\end{proof}

\subsection{Emergence of localization}
Emergence of localization 

As time proceeds, the initial states evolve out emerging the localization and this section is devoted to describing this localization: the deviation in the growth or decay rate at the origin from those at the rest of the places are contrasted, which is the emergence of the localization.

{\red For illustration}, we present below the generic cases of $-\frac{1+m+n}{\alpha-m-n}\ne-1$; in non-generic cases we would have the logarithmic corrections according to Proposition \ref{prop:ss}.
\begin{itemize}
 \item Strain : The strain keeps increasing at a fixed $x$ as time proceeds. However the growth at the origin is faster than that at the rest of the places,
\begin{align*}
 \gamma(t,0) &= (1+t)^{\frac{2+2\alpha-n}{D} + \frac{2+2\alpha}{D}\lambda}\Gamma(0), &
 \gamma(t,x) &\sim t^{\frac{2+2\alpha-n}{D} - \frac{(1+\alpha)(1+m+n)}{D(\alpha-m-n)}\lambda}|x|^{-\frac{1+\alpha}{\alpha-m-n}}, \quad \text{as $t \rightarrow \infty$, $x\ne0$.}
\end{align*}
Remember that the positivity of the growth rate $\frac{2+2\alpha-n}{D} - \frac{(1+\alpha)(1+m+n)}{D(\alpha-m-n)}\lambda$ was the ground for imposing the constraint \eqref{eq:lambda-range}.
\item Temperature : The temperature keeps increasing at a fixed $x$ as time proceeds. The growth at the origin is faster than that at the rest of the places,
\begin{align*}
 \theta(t,0) &= (1+t)^{\frac{2(1+m)}{D} + \frac{2(1+m+n)}{D}\lambda}\Theta(0),&
 \theta(t,x) &\sim t^{\frac{2(1+m)}{D} - \frac{(1+m+n)^2}{D(\alpha-m-n)}\lambda}|x|^{-\frac{1+m+n}{\alpha-m-n}}, \quad \text{as $t \rightarrow \infty$, $x\ne0$.}
\end{align*}
Again, the positivity of the growth rate $\frac{2(1+m)}{D} - \frac{(1+m+n)^2}{D(\alpha-m-n)}\lambda$ is from \eqref{eq:lambda-range}.
\item Strain rate : The growth rates of the strain-rate is by definition less by one to those of the strain, again illustrating the localization.
\begin{align*}
 u(t,0) &= (1+t)^{\frac{1+m}{D} + \frac{2+2\alpha}{D}\lambda}U(0),&
 u(t,x) &\sim t^{\frac{1+m}{D} - \frac{(1+\alpha)(1+m+n)}{D(\alpha-m-n)}\lambda}|x|^{-\frac{1+\alpha}{\alpha-m-n}}, \quad \text{as $t \rightarrow \infty$, $x\ne0$.}
\end{align*}
\item Stress : The stress keeps decreasing at a fixed $x$ as time proceeds. However, collapsing down of the stress at the origin is severer than that at the rest of the places,
\begin{align*}
 \sigma(t,0) &= (1+t)^{\frac{-2\alpha+2m+n}{D} + \frac{-2\alpha+2m+2n}{D}\lambda}\Sigma(0), &
 \sigma(t,x) &\sim t^{\frac{-2\alpha+2m+n}{D} +\frac{1+m+n}{D}\lambda}|x|, \quad \text{as $t \rightarrow \infty$, $x\ne0$,}
\end{align*}
noticing that $\frac{-2\alpha+2m+n}{D} +\frac{1+m+n}{D}\lambda\le-\frac{n}{1+m+n}$ in the valid range of the $\lambda$.


\item Velocity : The velocity is an odd function of $x$. At each time $t$, $v(t,x)$ is an increasing function of $x$ from $-v_\infty(t)$ to $v_\infty(t)$, as $x$ runs from $-\infty$ to $\infty$, where $v_\infty(t)\triangleq \lim_{x \rightarrow \infty} v(t,x)$. The velocity field is contrasted with the linear field of uniform shear motion. The self-similar scaling  $\xi=(1+t)^\lambda x$ implies that as time goes by the most of the transition takes place around the origin making the slope at origin steeper accordingly. It eventually forms a step-function type singularity at origin. Note that the asymptotic velocity $$v_\infty(t)=(1+t)^{b}V_\infty = (1+t)^{\frac{1+m}{D} + \frac{1+m+n}{D}\lambda}V_\infty, \quad V_\infty \triangleq \lim_{\xi \rightarrow \infty} V(\xi) <\infty.$$
This dictates that the particle of our solutions gets faster as it moves forward and we find the far field loading condition on the velocity different from that of uniform shearing motion, for the latter we have the constant velocity boundary condition. This deviation is a consequence of our simplifying assumption of self-similarity.
\end{itemize}

\begin{figure}[ht]
 \centering
%  \subfigure[{\red strain ($\gamma$)}]{
%  \includegraphics[scale=0.3]{strain_log} \label{fig:gamma}
%  }
 \subfigure[strain rate ($u$)]{
 \includegraphics[scale=0.3]{strain_rate_log} \label{fig:u}
 }
 \subfigure[temperature ($\theta$)]{
 \includegraphics[scale=0.3]{temperature_log} \label{fig:theta}
 }
 \subfigure[stress ($\sigma$)] {
 \includegraphics[scale=0.3]{stress_log} \label{fig:sigma}
 }
 \subfigure[velocity ($v$)] {
 \includegraphics[scale=0.3]{velocity} \label{fig:v}
 }
\caption{The localizing solution, for $\alpha=1.2$, $m=0$, $n=0.3$, $\lambda =0.5325$, and $\eta_0=-12.94$, sketched in the original variables $u, \ \theta, \ \sigma, \ v$.
All graphs except $v$ are in logarithmic scale.} \label{fig:computation}
\end{figure}

In Fig. \ref{fig:computation} are the computational results of the original field variables $\big(\gamma(t,x)$, $u(t,x)$, $\theta(t,x)$, $\sigma(t,x)$, $v(t,x)\big)$ that are computed from the heteroclinic orbit of \eqref{eq:slow} succeeded by transformations \eqref{eq:pqrdef}, \eqref{eq:BARtoTIL}, \eqref{eq:CAPtoBAR}, and \eqref{eq:ORItoCAP}. Parameters $\big(\alpha,m,n,\lambda)=(1.2,0,0.3,0.5325)$ was chosen for the computation. Before we examine them, we address one technical issue on the nontrivial choice of $m=0$ for its capturing. $M_0$ and $M_1$ are hyperbolic equilibrium points and their spectrum split into those with positive and negative real parts. While the Stable Manifold Theorem gives the separation of the corresponding stable and unstable manifolds, to achieve the same computationally is challenging task because the dynamical system is highly stiff in such a way that the presence of unstable subspace makes the computation destabilized. Since $M_0$ and $M_1$ both are saddle, computation in neither of time direction resolves this difficulty. 

While \eqref{intro-system0} bears this difficulty, with $m=0$, $\eqref{intro-system0}_2$-$\eqref{intro-system0}_4$ comprises a closed system, decoupling the first equation. The dimension is reduced by one in the corresponding dynamical system either (See \cite{KLT_HYP2016} for the precise formulation) and there, $M_0$ is an unstable node. By contrast to the saddle-saddle connection, in cases one of the equilibrium points is a stable or unstable {\it node}, solving the system in the stabilizing time direction produces numerically stable computations. We were able to compute the heteroclinic orbit backward in time, and to find Fig. \ref{fig:computation}. Capturing the saddle-saddle connection will need more substantial considerations in numerical approach than that has been approached here.


Fig.s \ref{fig:computation} illustrates the emergence of localization by depicting the profiles at a few instances of time. The vertical axes, except for the in Fig. \ref{fig:v} for the velocity, are in logarithmic scale. In Fig. \ref{fig:u}, one sees the localization in strain rate; the initial profile is a small perturbation of
a constant, at later time localization occurs at the origin. The same is seen in Fig. \ref{fig:theta} 
for the temperature and in  Fig. \ref{fig:v}, which eventually looks like a step function. On the other hand, stress collapses down to zero as seen from Fig. \ref{fig:sigma}, with different rates at the origin and at the rest of the points respectively.








% \section*{Appendix A. Coefficient matrices for the linearized system} \label{append:linear}
% Using
% \begin{align*}
%  (r+R)^{1+n} = \begin{cases}
%                 R^{1+n} &\text{if $r=0$},\\
%                 r^{1+n}\Big(1+\frac{R}{r}\Big)^{1+n} = r^{1+n} + (1+n)r^nR + \mathcal{O}(\delta^2), & \text{if $r>\bar{r}>0$,}
%                \end{cases}
% \end{align*}
% the linearized system around the equilibrium point is:
%
% \noindent
% {\bf Cases $r=0$, $p=0$, $q=0$ or $q=1$}
% \begin{align*}
%  \dot{P} &=P\Big(-q-\frac{1+\alpha}{\lambda(1+n)} c_0 -d_1\Big),\\
%  \dot{Q} &=Q(1-q) -qQ,\\
%  \dot{R} &=\frac{R}{n}\Big(q-\frac{\alpha-n}{\lambda(1+n)} c_0 +d_1 \Big).
% \end{align*}
% \noindent
% {\bf Cases $\Big( \frac{\alpha-n}{\lambda(1+n)} r^{1+n} - \frac{\alpha-n}{\lambda(1+n)}c_0 + d_1 + q \Big)=0$, $p=0$, $q=0$ or $q=1$}
% \begin{align*}
%  \dot{P}&=P\Big( \frac{1+\alpha}{\lambda(1+n)} r^{1+n} - \frac{1+\alpha}{\lambda(1+n)} c_0 -d_1-q\Big) = P\Big(-\frac{D}{\alpha-n}(d_1+q)\Big),\\
%  \dot{Q}&=Q(1-q) +q(-Q-\lambda Pr) + bPr,\\
%  \dot{R}&=\frac{r}{n}\Big( \frac{\alpha-n}{\lambda} r^nR + Q + \lambda Pr\Big) + \frac{R}{n}\Big(\frac{\alpha-n}{\lambda(1+n)}r^{1+n}-\frac{\alpha-n}{\lambda(1+n)}r^{1+n}c_0 + d_1 +q\Big) = \frac{r}{n}\Big( \frac{\alpha-n}{\lambda} r^nR + Q + \lambda Pr\Big)
% \end{align*}
%
% Coefficients Matrices $Mat_i$ for Linearized equations around $M_i$, $i=0,1,2,3$:
% \begin{align*}
%  Mat_0 &= \begin{pmatrix}
%           -\frac{D}{\alpha-n}(d_1) & 0 & 0\\
%           br_0 & 1 & 0\\
%           \frac{\lambda r_0^2}{n} & \frac{r_0}{n} & \frac{\alpha-n}{n\lambda}r_0^{1+n}
%          \end{pmatrix}
%         = \begin{pmatrix}
%           2 & 0 & 0\\
%           br_0 & 1 & 0\\
%           \frac{\lambda r_0^2}{n} & \frac{r_0}{n} & \frac{\alpha-n}{n\lambda}r_0^{1+n}
%          \end{pmatrix}\\
%  Mat_1 &= \begin{pmatrix}
%           -\frac{D}{\alpha-n}(d_1+1) & 0 & 0\\
%           (b-\lambda)r_1 & -1 & 0\\
%           \frac{\lambda r_1^2}{n} & \frac{r_1}{n} & \frac{\alpha-n}{n\lambda}r_1^{1+n}
%          \end{pmatrix}
%         =\begin{pmatrix}
%           -\frac{1+n}{\alpha-n} & 0 & 0\\
%           (b-\lambda)r_1 & -1 & 0\\
%           \frac{\lambda r_1^2}{n} & \frac{r_1}{n} & \frac{\alpha-n}{n\lambda}r_1^{1+n}
%          \end{pmatrix}\\
%  Mat_2 &= \begin{pmatrix}
% 	  -1-\frac{1+\alpha}{\lambda(1+n)} c_0 -d_1 & 0 & 0\\
% 	  0 & -1 & 0\\
% 	  0 & 0 & \frac{1}{n}\Big(1-\frac{\alpha-n}{\lambda(1+n)} c_0 +d_1\Big)
%          \end{pmatrix}
%         = \begin{pmatrix}
% 	  -\frac{1+\alpha}{\lambda(1+n)} \Big(\frac{2}{D} + \frac{(1+n)^2}{D(1+\alpha)}\lambda\Big) & 0 & 0\\
% 	  0 & -1 & 0\\
% 	  0 & 0 & -\frac{\alpha-n}{\lambda n(1+n)}r_1^{1+n}
%          \end{pmatrix}\\
%  Mat_3 &= \begin{pmatrix}
% 	  -\frac{1+\alpha}{\lambda(1+n)} c_0 -d_1 & 0 & 0\\
% 	  0 & 1 & 0\\
% 	  0 & 0 & \frac{1}{n}\Big(-\frac{\alpha-n}{\lambda(1+n)} c_0 +d_1\Big)
%          \end{pmatrix}
% 	=\begin{pmatrix}
% 	  -\frac{1+\alpha}{\lambda(1+n)} \Big(\frac{2}{D} - \frac{2(\alpha-n)(1+n)}{D(1+\alpha)}\lambda\Big)& 0 & 0\\
% 	  0 & 1 & 0\\
% 	  0 & 0 & -\frac{\alpha-n}{\lambda n(1+n)}r_0^{1+n}
%          \end{pmatrix}
% \end{align*}
% The lower triangular matrix has the eigenvalues and eigenvectors such that
% \begin{align*}
%  MAT &= \begin{pmatrix}
%         A & 0 & 0\\
%         B & C & 0\\
%         D & E & F
%        \end{pmatrix}, \quad
%  \mu_1 = A, \quad\mu_2 = C, \quad\mu_3 = F,\\
%  v_1 &= \Big( \frac{ (A-C)(A-F) }{ D(A-C) + BE }, \frac{ B(A-F) }{ D(A-C) + BE }, 1), \quad  v_2 = (0, \frac{C-F}{E}, 1), \quad v_3 = (0,0,1).
% \end{align*}
% The eigenvectors in Section \ref{sec:equil} took the suitable normalization.
% \begin{align*}
%  &X_{01} = \bigg( \Big( \frac{2n - \frac{\alpha-n}{\lambda}r_0^{1+n}}{\big({\lambda}+b\big) r_0^2}\Big) \;,\;\Big( \frac{2n - \frac{\alpha-n}{\lambda}r_0^{1+n}}{\big({\lambda}+b\big) r_0^2}\Big)br_0\;,\;1\bigg),\quad
%  X_{02} = \bigg(0, \Big(\frac{n- \frac{\alpha-n}{\lambda}r_0^{1+n}}{r_0}\Big), 1\bigg), \quad
%  X_{03} = (0,0,1),\\
%  &X_{11} = \bigg(  \Big(\frac{-n\frac{1+n}{\alpha-n} - \frac{\alpha-n}{\lambda}r_1^{1+n}}{\big(-\frac{1+n}{\alpha-n} \lambda +b\big) r_1^2}\Big)\Big(1-\frac{1+n}{\alpha-n}\Big) \;,\;\Big(\frac{-n\frac{1+n}{\alpha-n} - \frac{\alpha-n}{\lambda}r_1^{1+n}}{\big(-\frac{1+n}{\alpha-n} \lambda +b\big) r_1^2}\Big)(b-\lambda)r_1\;,\;1\bigg),\\
%  &X_{12} = \bigg(0, \Big(\frac{n- \frac{\alpha-n}{\lambda}r_0^{1+n}}{r_0}\Big), 1\bigg), \quad
%  X_{13} = (0,0,1),
% \end{align*}


\vfil\eject


\appendix
\section{The loss of hyperbolicity} \label{append:hadamard}
Consider the system \eqref{intro-system0} for $n=0$ when the viscoplastic effects are turned off. It is rewritten in the form of  first order transport equations,
\begin{equation} \label{eq:transport}
 \begin{pmatrix} \gamma_t \\ \theta_t \\ v_t \end{pmatrix} = \underbrace{
 \begin{pmatrix}
  0 & 0 & 1\\
  0 & 0 & \theta^{-\alpha}\gamma^m \\
  m\theta^{-\alpha}\gamma^{m-1} & -\alpha\theta^{-\alpha-1}\gamma^m & 0 \end{pmatrix}}_\text{$\triangleq B$}
  \begin{pmatrix} \gamma_x \\ \theta_x \\ v_x \end{pmatrix}.
\end{equation}
\eqref{eq:transport} is hyperbolic if $B$ has three real eigenvalues and three linearly independent eigenvectors.
\begin{align*}
 \det\big(B-\mu\textrm{I}\big) &= -\mu\big(\mu^2+\theta^{-\alpha}\gamma^{m-1}(\alpha \theta^{-\alpha-1}\gamma^{1+m} - m)\big)\\
 &=-\mu\big(\mu^2+\theta^{-\alpha}\gamma^{m-1}\Big(\frac{\alpha-m}{1+\alpha}+\alpha \Big(\frac{\gamma^{1+m}}{\theta^{1+\alpha}} - \frac{1+m}{1+\alpha}\Big)\Big)\\
 &=-\mu\big(\mu^2+\theta^{-\alpha}\gamma^{m-1}\Big(\frac{\alpha-m}{1+\alpha}+\alpha \frac{\gamma_0(x)^{1+m} - \frac{1+m}{1+\alpha}\theta_0(x)^{1+\alpha}}{\theta^{1+\alpha}}\Big),
\end{align*}
where $\gamma_0(x)$ and $\theta_0(x)$ are initial values of $\gamma$ and $\theta$ respectively. If $\alpha-m>0$ and $\theta$ increases
along the evolution then \eqref{eq:transport} loses hyperbolicity in a finite time. In particular, this 
happens along the uniform shearing solutions \eqref{intro:uss}  as $t \rightarrow \infty$.


\section{Complete list of equilibria}\label{append:equi_reject}
Complete list of equilibria of System \eqref{eq:slow} are listed below. Underlined the reader will find the components that show the point lies outside the region
of interest and leads to rejection of them.
\allowdisplaybreaks
\begin{align*}
 &(1) & & p=0, \ q=0, \ \underline{ r=0, \ s=0},\\
 &(2) & & p=0, \ q=0, \ \underline{ r=0},\ s=s, \ \underline{ r_0 = \frac{-n}{\alpha-m-n}},\\
 &(3) & & p=0, \ q=0, \ r = \frac{n\alpha - r_0(\alpha-m-n)}{(1+\alpha)(m+n)},\ \underline{ s=0},\\
 &(4) & & p=0, \ q=1, \ \underline{ r=0, \ s=0}, \\
 &(5) & & p=0, \ q=1, \ \underline{ r=0},\ s=s, \ \underline{ r_1 = \frac{-n}{\alpha-m-n}},\\
 &(6) & & p=0, \ q=1, \ r = \frac{n\alpha - r_1(\alpha-m-n)}{(1+\alpha)(m+n)},\ \underline{ s=0},\\
 &(7) & & p=p, \ q=0, \ \underline{ r=0, \ s=0}, \ \frac{r_0}{\lambda}=2, \\
 &(8) & & p=p, \ q=1, \ \underline{ r=0, \ s=0}, \ \frac{r_0}{\lambda}=1, \\
 &(9) & & p=p, \ q=0, \ \underline{ r=0}, \ s=s, \ \frac{r_0}{\lambda}=2, \ \underline{ r_0 = \frac{-n}{\alpha-m-n}}, \\
 &(10) & & p=p, \ q=1, \ \underline{ r=0}, \ s=s, \ \frac{r_0}{\lambda}=1, \ \underline{ r_1 = \frac{-n}{\alpha-m-n}}, \\
 &(11) & & \underline{ p=-\frac{(\alpha-m-n)(1+m+n)}{(1+\alpha)(1+m)}<0}, \ q=\frac{2(\alpha-m-n)}{1+m}b, \ r=a_0, \ s=\frac{1+m+n}{1+\alpha} - \frac{n}{(1+\alpha)a_0},\\
 &(12) & & p=\Big ( \frac{2\alpha(1+m)}{D(1-m-n)} + \frac{2(\alpha-m-n)}{D}\lambda\Big)\Big(\frac{2\alpha(1+m)}{D(1-m-n)} - \frac{1+m+n}{D}\lambda\Big)\frac{1-m-n}{\lambda(2-n)}\frac{1-m-n}{\lambda(1+m)}, \\
 & & &q=\left( \frac{2\alpha(1+m)}{D(1-m-n)} + \frac{2(\alpha-m-n)}{D}\lambda\right)\left(\frac{1+m}{D} + \frac{1+m+n}{D}\lambda\right)\frac{1-m-n}{\lambda(1+m)},\\
 & & &r = \frac{2-n}{1-m-n}, \ \underline{ s=0}.
\end{align*}
\section{Linearized problems around $M_0$ and $M_1$}\label{append:lin}
The coefficient matrix for the linearized system around the equlibrium $M_0$ is
\begin{align*}
 \begin{pmatrix}
          2 & 0 & 0 & 0 \\
          br_0 & 1 & 0 & 0\\
          \frac{r_0}{n}(\lambda r_0) & \frac{r_0}{n} & \frac{r_0}{n}\Big(\frac{\alpha-m-n}{\lambda(1+\alpha)} - \frac{n\alpha}{\lambda(1+\alpha)r_0}\Big) & \frac{r_0}{n}(\frac{\alpha r_0}{\lambda})\\
          s_0(\lambda r_0) & s_0 & s_0\Big(\frac{\alpha-m-n}{\lambda(1+\alpha)} + \frac{n}{\lambda(1+\alpha)r_0}\Big) & s_0(-\frac{r_0}{\lambda})
         \end{pmatrix}
        =\begin{pmatrix}
          2 & 0 & 0 & 0 \\
          br_0 & 1 & 0 & 0\\
          \frac{r_0}{n}(\lambda r_0) & \frac{r_0}{n} & \frac{r_0}{n}\frac{1}{\lambda}\Big(1-s_0-\frac{n}{r_0}\Big) & \frac{r_0}{n}(\frac{\alpha r_0}{\lambda})\\
          s_0(\lambda r_0) & s_0 & s_0\frac{1}{\lambda}(1-s_0) & s_0(-\frac{r_0}{\lambda})
         \end{pmatrix}
\end{align*}
The corresponding eigenvectors $X_{0j}$ are collected in the matrix $S_0$ as $j$-th column vector, $j=1,2,3,4$.
\begin{equation} \label{eq:S0}
\begin{aligned}
 S_0&=
 \begin{pmatrix}
    1 & 0 & 0 & 0\\
    br_0 & 1 & 0 & 0\\
    y_1 & y_2 & 1 & y_4\\
    z_1 & z_2 & z_3 &1
 \end{pmatrix},
 \quad \quad
 \begin{array}{l}
\begin{pmatrix}
 y_1\\z_1
\end{pmatrix}
=-(\lambda+b)r_0\begin{pmatrix}
  \frac{ \frac{1+\alpha}{\lambda}r_0 + \frac{\mu_{01}}{s_0} }{ \Delta_1 }\\
  \frac{ \frac{n}{r_0}\big(\frac{1}{\lambda} + \mu_{01}\big) }{ \Delta_1 }
  \end{pmatrix},
  \quad
 \begin{pmatrix}
 y_2\\z_2
\end{pmatrix}
=-\begin{pmatrix}
  \frac{ \frac{1+\alpha}{\lambda}r_0 + \frac{\mu_{02}}{s_0} }{ \Delta_2 }\\
  \frac{ \frac{n}{r_0}\big(\frac{1}{\lambda} + \mu_{02}\big) }{ \Delta_2 }
  \end{pmatrix}\\
%    y_1=-\frac{(\lambda+b)r_0}{\frac{1-s_0}{\lambda} - nA}, \quad A=\frac{\big(\frac{r_0}{\lambda}+\frac{2}{s_0}\big)\big(\frac{1}{\lambda}+2\big)\frac{1}{r_0}}{ \frac{1+\alpha}{\lambda}r_0 + \frac{2}{s_0} }, \\
%  z_1=n\bigg(\frac{\big(\frac{1}{\lambda}+2\big)\frac{1}{r_0}}{ \frac{1+\alpha}{\lambda}r_0 + \frac{2}{s_0} }\bigg)y_1, \\
%  y_2=-\frac{1}{\frac{1-s_0}{\lambda} - nB }, \quad B=\frac{\big(\frac{r_0}{\lambda}+\frac{1}{s_0}\big)\big(\frac{1}{\lambda}+1\big)\frac{1}{r_0}}{ \frac{1+\alpha}{\lambda}r + \frac{1}{s_0} }\\
%  z_2=n\bigg(\frac{\big(\frac{1}{\lambda}+1\big)\frac{1}{r_0}}{ \frac{1+\alpha}{\lambda}r_0 + \frac{1}{s_0} }\bigg)y_2, \\
 z_3=n\bigg(\frac{\frac{1-s_0}{\lambda}}{\frac{n r_0}{\lambda} + \frac{n\mu_{0}^+}{s_0}}\bigg),\quad y_4=\frac{\frac{r_0}{\lambda}+\frac{\mu_0^-}{s_0}}{\frac{1-s_0}{\lambda}},
 \end{array}
\end{aligned}
\end{equation}
where $\Delta_1 = \frac{1-s_0}{\lambda}\big(\frac{1+\alpha}{\lambda}r_0 + \frac{\mu_{01}}{s_0}\big) -\frac{n}{r_0} \big( \frac{1}{\lambda} + \mu_{01}\big)\big(\frac{r_0}{\lambda} + \frac{\mu_{01}}{s_0}\big)$
%=\frac{-n}{r_0s_0}\det \left[\begin{pmatrix} \frac{r_0}{n}\big(\frac{1-s_0}{\lambda}-\frac{n}{\lambda r_0}\big) & \frac{r_0}{n}\frac{\alpha r_0}{\lambda}\\ s_0\frac{1-s_0}{\lambda} & -s_0\frac{r_0}{\lambda} \end{pmatrix} -\mu_{01}\textrm{I}\right]\ne0$
and $\Delta_2 = \frac{1-s_0}{\lambda}\big(\frac{1+\alpha}{\lambda}r_0 + \frac{\mu_{02}}{s_0}\big) -\frac{n}{r_0} \big( \frac{1}{\lambda} + \mu_{02}\big)\big(\frac{r_0}{\lambda} + \frac{\mu_{02}}{s_0}\big)$.
%=\frac{-n}{r_0s_0}\det \left[\begin{pmatrix} \frac{r_0}{n}\big(\frac{1-s_0}{\lambda}-\frac{n}{\lambda r_0}\big) & \frac{r_0}{n}\frac{\alpha r_0}{\lambda}\\ s_0\frac{1-s_0}{\lambda} & -s_0\frac{r_0}{\lambda} \end{pmatrix} -\mu_{02}\textrm{I}\right]\ne0$ respectively for the corresponding cases.
We find that $y_1,y_2,y_4<0$; $z_1,z_2,z_3 \sim\BO(n)$, provided $n$ is sufficiently small.

Next, the coefficient matrix for the linearized equations around $M_1$ is
\begin{align*}
 \begin{pmatrix}
          -\frac{1+m+n}{\alpha-m-n} & 0 & 0 & 0\\
          (b-\lambda)r_1 & -1 & 0 & 0\\
          \frac{r_1}{n}(\lambda r_1) & \frac{r_1}{n} & \frac{r_1}{n}\Big(\frac{\alpha-m-n}{\lambda(1+\alpha)} - \frac{n\alpha}{\lambda(1+\alpha)r_1}\Big) & \frac{r_1}{n}(\frac{\alpha r_1}{\lambda})\\
          s_1(\lambda r_1) & s_1 & s_1\Big(\frac{\alpha-m-n}{\lambda(1+\alpha)} + \frac{n}{\lambda(1+\alpha)r_1}\Big) & s_1(-\frac{r_1}{\lambda})
         \end{pmatrix}
         =\begin{pmatrix}
          -\frac{1+m+n}{\alpha-m-n} & 0 & 0 & 0\\
          (b-\lambda)r_1 & -1 & 0 & 0\\
          \frac{r_1}{n}(\lambda r_1) & \frac{r_1}{n} & \frac{r_1}{n}\frac{1}{\lambda}\Big(1-s_1-\frac{n}{r_1}\Big) & \frac{r_1}{n}(\frac{\alpha r_1}{\lambda})\\
          s_1(\lambda r_1) & s_1 & s_1\frac{1}{\lambda}(1-s_1) & s_1(-\frac{r_1}{\lambda})
         \end{pmatrix}
\end{align*}

%It turned out that the coefficient matrix for $M_1$ possibly possesses the Jordan block when $\mu_{11}$ and $\mu_{12}$ coincide. We first demonstrate the four eigenvectors for the generic case where $\mu_{11}\ne\mu_{12}$ and then specify the generalized eigenvectors for this special case.

The following exposition specifies all possible combinations. Except for the case $\mu_{11}=\mu_{12}=-1$, four linearly independent eigenvectors are attained. 
When the exceptional case occurs the repeated eigenvalue $-1$ has one less geometric multiplicity than the algebraic multiplicity.% so we supplement one generalized eigenvector for the repeated eigenvalue $-1$.

As to the eigenvectors, notice first that the eigenvalues for $M_1$, differently from those for $M_0$, have chances to be repeated. While the exposition in
Appendix \ref{append:lin} specifies quite a few possible combinations, 
what is explained below is that unless $\mu_{11}=\mu_{12}=-1$, the four linearly independent eigenvectors are attained, and when the exceptional case takes place we will supplement precisely one generalized eigenvector for the repeated eigenvalue $-1$.

{\bf Case 1. $-\frac{1+m+n}{\alpha-m-n}\ne -1$; or $-\frac{1+m+n}{\alpha-m-n}= -1$ but $b=\lambda$. } This case yields the four linearly independent eigenvectors. The eigenvector $X_{1j}$ is collected in the matrix $S_1$ as $j$-th column vector, $j=1,2,3,4$, and in cases eigenvalues are repeated the corresponding eigenvectors are understood as a basis for the corresponding subspaces:
\begin{align*}
 S_1&=
 \begin{pmatrix}
    1 & 0 & 0 & 0\\
    x_1 & 1 & 0 & 0\\
    y_1 & y_2 & 1 & y_4\\
    z_1 & z_2 & z_3 &1
 \end{pmatrix}, \quad \quad
 \begin{array}{l}
  x_1=
 \begin{cases}
  \frac{(b-\lambda)r_1}{1+\mu_{11}} & \text{if $\mu_{11}\ne -1$,}\\
  0 & \text{otherwise,}
 \end{cases}\\
 z_3=n\bigg(\frac{\frac{1-s_1}{\lambda}}{\frac{n r_1}{\lambda} + \frac{n\mu_{1}^+}{s_1}}\bigg), \quad y_4=\frac{\frac{r_1}{\lambda}+\frac{\mu_1^-}{s_1}}{\frac{1-s_1}{\lambda}},\\
 \end{array}
\end{align*}
\begin{equation} \label{eq:S1-1}
\begin{aligned}
\begin{pmatrix}
 y_1\\z_1
\end{pmatrix}
=\begin{cases}
  -(\lambda r_1 + x_1)\begin{pmatrix}
  \frac{\lambda}{1-s_1}\\0
  \end{pmatrix} & \text{if $\mu_{14}=\mu_{11}$,}\\
  -(\lambda r_1 + x_1)
  \begin{pmatrix}
  \frac{ \frac{1+\alpha}{\lambda}r_1 + \frac{\mu_{11}}{s_1} }{ \Delta_3 }\\
  \frac{ \frac{n}{r_1}\big(\frac{1}{\lambda} + \mu_{11}\big) }{ \Delta_3 }
  \end{pmatrix} & \text{otherwise,}
 \end{cases}
 \quad
 \begin{pmatrix}
 y_2\\z_2
\end{pmatrix}
=\begin{cases}
 -\begin{pmatrix}
  \frac{\lambda}{1-s_1}\\0
  \end{pmatrix} & \text{if $\mu_{14}=\mu_{12}$,}\\
  -\begin{pmatrix}
  \frac{ \frac{1+\alpha}{\lambda}r_1 + \frac{\mu_{12}}{s_1} }{ \Delta_4 }\\
  \frac{ \frac{n}{r_1}\big(\frac{1}{\lambda} + \mu_{12}\big) }{ \Delta_4 }
  \end{pmatrix} & \text{otherwise,}
 \end{cases}
\end{aligned}
\end{equation}
where
\begin{align*}
 \Delta_3 &= \frac{1-s_1}{\lambda}\big(\frac{1+\alpha}{\lambda}r_1 + \frac{\mu_{11}}{s_1}\big) -\frac{n}{r_1} \big( \frac{1}{\lambda} + \mu_{11}\big)\big(\frac{r_1}{\lambda} + \frac{\mu_{11}}{s_1}\big)
 \\
 &=\frac{-n}{r_1s_1}\det \left[\begin{pmatrix} \frac{r_1}{n}\big(\frac{1-s_1}{\lambda}-\frac{n}{\lambda r_1}\big) & \frac{r_1}{n}\frac{\alpha r_1}{\lambda}\\ s_1\frac{1-s_1}{\lambda} & -s_1\frac{r_1}{\lambda} \end{pmatrix} -\mu_{11}\textrm{I}\right]\ne0,
 \\
 \Delta_4 &= \frac{1-s_1}{\lambda}\big(\frac{1+\alpha}{\lambda}r_1 + \frac{\mu_{12}}{s_1}\big) -\frac{n}{r_1} \big( \frac{1}{\lambda} + \mu_{12}\big)\big(\frac{r_1}{\lambda} + \frac{\mu_{12}}{s_1}\big)
 \\
 &=\frac{-n}{r_1s_1}\det \left[\begin{pmatrix} \frac{r_1}{n}\big(\frac{1-s_1}{\lambda}-\frac{n}{\lambda r_1}\big) & \frac{r_1}{n}\frac{\alpha r_1}{\lambda}\\ s_1\frac{1-s_1}{\lambda} & -s_1\frac{r_1}{\lambda} \end{pmatrix} -\mu_{12}\textrm{I}\right]\ne0
\end{align*}
respectively for the corresponding cases.
% $\Delta_3 = \frac{1-s_1}{\lambda}\big(\frac{1+\alpha}{\lambda}r_1 + \frac{\mu_{11}}{s_1}\big) -\frac{n}{r_1} \big( \frac{1}{\lambda} + \mu_{11}\big)\big(\frac{r_1}{\lambda} + \frac{\mu_{11}}{s_1}\big)=\frac{-n}{r_1s_1}\det \left[\begin{pmatrix} \frac{r_1}{n}\big(\frac{1-s_1}{\lambda}-\frac{n}{\lambda r_1}\big) & \frac{r_1}{n}\frac{\alpha r_1}{\lambda}\\ s_1\frac{1-s_1}{\lambda} & -s_1\frac{r_1}{\lambda} \end{pmatrix} -\mu_{11}\textrm{I}\right]\ne0$ and $\Delta_4 = \frac{1-s_1}{\lambda}\big(\frac{1+\alpha}{\lambda}r_1 + \frac{\mu_{12}}{s_1}\big) -\frac{n}{r_1} \big( \frac{1}{\lambda} + \mu_{12}\big)\big(\frac{r_1}{\lambda} + \frac{\mu_{12}}{s_1}\big)=\frac{-n}{r_1s_1}\det \left[\begin{pmatrix} \frac{r_1}{n}\big(\frac{1-s_1}{\lambda}-\frac{n}{\lambda r_1}\big) & \frac{r_1}{n}\frac{\alpha r_1}{\lambda}\\ s_1\frac{1-s_1}{\lambda} & -s_1\frac{r_1}{\lambda} \end{pmatrix} -\mu_{12}\textrm{I}\right]\ne0$

{\bf Case 2. $-\frac{1+m+n}{\alpha-m-n}= -1$ and $b\ne\lambda$: }
For this case $-1$ has one less geometric multiplicity than the algebraic multiplicity and we replace the first column of $S_1$ by the generalized eigenvector $\big(\frac{1}{(b-\lambda)r_1}, 0, y_1', z_1'\big)^T$, where
\begin{equation} \label{eq:S1-2}
\begin{aligned}
\begin{pmatrix}
 y_1'\\z_1'
\end{pmatrix}
=\begin{cases}
  \begin{pmatrix}
  -\frac{\lambda}{1-s_1}\big(\frac{\lambda}{b-\lambda} -\frac{n}{r_1}z_2\big)\\0
  \end{pmatrix} & \text{if $\mu_{14}=-1$,}\\
  -\frac{\lambda}{b-\lambda}
  \begin{pmatrix}
  \frac{ \frac{1+\alpha}{\lambda}r_1 + \frac{\mu_{11}}{s_1} }{ \Delta_3 }\\
  \frac{ \frac{n}{r_1}\big(\frac{1}{\lambda} + \mu_{11}\big) }{ \Delta_3 }
  \end{pmatrix} +
  \frac{n}{r_1}
  \begin{pmatrix}
  \frac{ y_2\big(\frac{r_1}{\lambda} + \frac{\mu_{11}}{s_1}\big) + z_2\frac{\alpha r_1}{\lambda} }{ \Delta_3 }\\
  \frac{ y_2\big(\frac{1-s_1}{\lambda}\big) + z_2\big(-\frac{1-s_1}{\lambda}+\frac{n}{r_1}\big(\frac{1}{\lambda}+\mu_{11}\big)\big) }{ \Delta_3 }
  \end{pmatrix} & \text{otherwise.}
 \end{cases}
\end{aligned}
\end{equation}

\begin{align*}
 0&=Mat_1 \begin{pmatrix} w\\x\\y\\z \end{pmatrix} -\mu \begin{pmatrix} w\\x\\y\\z \end{pmatrix}=
\begin{pmatrix}
 (\mu_{11}-\mu) w\\
 (b-\lambda)r_1 w +(\mu_{12}-\mu)x\\
 \frac{r_1}{n} \left[\lambda r_1 w + x + \big(\frac{1-s_1}{\lambda} - \frac{n}{r_1}\big(\frac{1}{\lambda}+\mu\big)\big)y + \frac{\alpha r_1}{\lambda}z\right]\\
 s_1 \left[ \lambda r_1 w + x + \big(\frac{1-s_1}{\lambda}\big)y -\big(\frac{r_1}{\lambda}+\frac{\mu}{s_1}\big)z\right]
 \end{pmatrix},\\
 A&\triangleq
 \begin{pmatrix}
 \frac{1-s_1}{\lambda} - \frac{n}{r_1}\big(\frac{1}{\lambda}+\mu\big) & \frac{\alpha r_1}{\lambda}\\
 \frac{1-s_1}{\lambda} & -\big(\frac{r_1}{\lambda}+\frac{\mu}{s_1}\big)
 \end{pmatrix}
 \begin{pmatrix} y\\z \end{pmatrix} = -(\lambda r_1 w +x)\begin{pmatrix} 1\\1\end{pmatrix}\\
 A^{-1} &= \frac{1}{\Delta}
 \begin{pmatrix} \big(\frac{r_1}{\lambda}+\frac{\mu}{s_1}\big) & \frac{\alpha r_1}{\lambda}\\
 \frac{1-s_1}{\lambda} & -\frac{1-s_1}{\lambda} + \frac{n}{r_1}\big(\frac{1}{\lambda}+\mu\big)
\end{pmatrix}, \quad
\Delta=\frac{1-s_1}{\lambda}\big( \frac{1+\alpha}{\lambda}r_1 + \frac{\mu}{s_1}\big) - \frac{n}{r_1}\big(\frac{1}{\lambda} +\mu\big)\big(\frac{r_1}{\lambda}+\frac{\mu}{s_1}\big)
\end{align*}



%
% \begin{align*}
%  y_1&=-\frac{(\lambda+b)r_0}{\frac{\alpha-m-n}{\lambda(1+\alpha)} - \frac{2n}{r_0(1+\alpha)}\Big(1 + \frac{(\frac{1}{\lambda}+2)\frac{\alpha}{s}}{ \frac{1+\alpha}{\lambda}r + \frac{2}{s} }\Big) }\\
%  y_2&=-\frac{1}{\frac{\alpha-m-n}{\lambda(1+\alpha)} - \frac{n}{r_0(1+\alpha)}\Big(1 + \frac{(\frac{1}{\lambda}+1)\frac{\alpha}{s}}{ \frac{1+\alpha}{\lambda}r + \frac{1}{s} }\Big) }\\
%  y_3&=\frac{\mu_0^- +\frac{r_0s_0}{\lambda}}{s\Big(\frac{\alpha-m-n}{\lambda(1+\alpha)} + \frac{n}{\lambda(1+\alpha)r_0}\Big)}\\
%  z_1&=n\bigg(\frac{\big(\frac{1}{\lambda}+2\big)\frac{1}{r}}{ \frac{1+\alpha}{\lambda}r + \frac{2}{s} }\bigg)y_1 \\
%  z_2&=n\bigg(\frac{\big(\frac{1}{\lambda}+1\big)\frac{1}{r}}{ \frac{1+\alpha}{\lambda}r + \frac{1}{s} }\bigg)y_2 \\
%  z_3&=n\bigg(\mu_0^+-\frac{r_0}{n}\Big(\frac{\alpha-m-n}{\lambda(1+\alpha)} - \frac{n\alpha}{\lambda(1+\alpha)r_0}\Big)\bigg)r_0(\frac{\alpha r_0}{\lambda})\bigg)
% \end{align*}


\vfil\eject
\begin{thebibliography}{10}

\bibitem{bertsch_effect_1991}
{\sc M.~Bertsch, L.~Peletier, and S.~Verduyn~Lunel},
The effect of temperature dependent viscosity on shear flow of  incompressible fluids,
{\it SIAM J. Math. Anal.} {\bf 22 } (1991), 328--343.

\bibitem{clifton_rev_1990}
{\sc R.J. Clifton},  High strain rate behaviour of metals,
% {\it Applied Mechanics Review}
{\it Appl. Mech. Rev.}
{\bf 43} (1990), S9-S22.

\bibitem{DH_1983}
{\sc C.M. Dafermos and L.~Hsiao},
Adiabatic shearing of incompressible fluids with temperature-dependent viscosity.
{\it Quart.  Applied Math.} {\bf 41} (1983), 45--58.

% \bibitem{fenichel_persistence_1972}
% {\sc N.~Fenichel},
% Persistence and smoothness of invariant manifolds for  flows,
% {\it Indiana Univ. Math. J.} {\bf 21} (1972) 193--226.

\bibitem{fenichel_asymptotic_1974}
{\sc N.~Fenichel},
Asymptotic stability with rate conditions,
{\it Indiana Univ. Math. J.} {\bf 23} (1974) 1109--1137.

\bibitem{fenichel_asymptotic_1977}
{\sc N.~Fenichel},
Asymptotic stability with rate conditions \textrm{II},
{\it Indiana Univ. Math. J.} {\bf 26} (1977) 81--93.

\bibitem{fenichel_geometric_1979}
{\sc N.~Fenichel},
Geometric singular perturbation theory for ordinary differential equations,
{\it J. Differ. Equations} {\bf 31} (1979), 53--98.

%\bibitem{FM87}
%{\sc C.~Fressengeas and A.~Molinari},
%Instability and localization of plastic flow in shear at high strain rates,
%{\it J.  Mech. Physics of Solids} {\bf 35} (1987), 185--211.

\bibitem{HPS_1977}
{\sc M.W. Hirsch, C.C. Pugh, and M. Shub},
{\it Invariant Manifolds}, LNM {\bf 583}, (Springer-Verlag, New York/Heidelberg/Berlin 1977)

\bibitem{HN77}
{\sc J.W.~Hutchinson and K.W.~Neale},
Influence of strain-rate sensitivity on necking under uniaxial tension,
{\it  Acta Metallurgica} {\bf 25} (1977), 839-846.

\bibitem{KOT14}
{\sc Th.~Katsaounis, J.~Olivier, and A.E.~Tzavaras},
Emergence of coherent localized structures in shear deformations of temperature dependent fluids,
{\it Archive for Rational Mechanics and Analysis} {\bf 224} (2017), 173--208.

\bibitem{KT09}
{\sc Th. Katsaounis and A.E.~Tzavaras},
Effective equations for localization and shear band formation,
{\it SIAM J. Appl. Math.}  {\bf 69} (2009), 1618--1643.

\bibitem{KLT_2016}
{\sc Th. Katsaounis, M.-G. Lee, and A.E. Tzavaras},
Localization in inelastic rate dependent shearing deformations,
{\it J. Mech. Phys. of Solids} {\bf 98} (2017), 106--125.

{\blue
\bibitem{Jones_1995}
{\sc Christopher K. R. T.  Jones}, 
Geometric singular perturbation theory. 
Dynamical systems (Montecatini Terme, 1994), pp 44Ð118, {\it Lecture Notes in Math.}, 1609, Springer, Berlin, 1995.
}

\bibitem{KUEHN_2015}
{\sc C.~ Kuehn}, 
{\it Multiple time scale dynamics}, Applied Mathematical Sciences, Vol. {\bf 191} (Springer Basel 2015).

\bibitem{LT16}
{\sc M.-G.~Lee and A.E.~Tzavaras},
Existence of localizing solutions in plasticity via the geometric singular perturbation theory,
{\it Siam J. Appl. Dyn. Systems} {\bf 16} (2017), 337--360.

\bibitem{KLT_HYP2016}
{\sc M.-G. Lee, Th. Katsaounis, and A.E. Tzavaras},
Localization of Adiabatic Deformations in Thermoviscoplastic Materials, In Proceedings of the 16th International Conference on Hyperbolic Problems: Theory, Numerics, Applications (HYP2016), to appear.

%
% \bibitem{jones_geometric_1995}
% {\sc C.~K. R.~T. Jones},
% Geometric singular perturbation theory, in {\it Dynamical systems}, LNM {\bf 1609} (Springer Berlin Heidelberg 1995) 44--118.
%
%

%
% % \bibitem{perko_differential_2001}
% % {\sc L.~Perko},
% % {\it Differential equations and dynamical systems 3rd. ed.}, TAM {\bf 7} (Springer-Verlag New York 2001).
%

\bibitem{shawki_shear_1989}
{\sc T.G. Shawki and R.J. Clifton},
Shear band formation in thermal viscoplastic materials,
% {\it Mechanics of Materials}
{\it Mech. Mater.}
{\bf 8 } (1989), 13--43.

\bibitem{Sz1991}
{\sc P.~Szmolyan},
Transversal heteroclinic and homoclinic orbits in singular perturbation problems,
{\it J. Differ. Equations}
{\bf 92} (1991), 252--281.

\bibitem{Tz_1986}
{\sc A.E. Tzavaras},
Shearing of materials exhibiting thermal softening or temperature dependent viscosity,
{\em Quart.  Applied Math.} {\bf 44} (1986), 1--12.

\bibitem{Tz_1987}
{\sc A.E. Tzavaras},
Effect of thermal softening in shearing of strain-rate dependent materials.
{\em Archive for Rational Mechanics and Analysis}, {\bf 99} (1987), 349--374.

\bibitem{tzavaras_plastic_1986}
{\sc A.E. Tzavaras},
Plastic shearing of materials exhibiting strain hardening or strain softening,
% {\it Archive for Rational Mechanics and  Analysis}
{\it Arch. Ration. Mech. Anal.}
{\bf 94} (1986), 39--58.

%\bibitem{tzavaras_strain_1991}
%{\sc A.E. Tzavaras},
%Strain softening in viscoelasticity of the rate type.
%{\it J. Integral Equations Appl.} {\bf  3}  (1991), 195--238.

\bibitem{tzavaras_nonlinear_1992}
%\leavevmode\vrule height 2pt depth -1.6pt width 23pt,
{\sc A.E. Tzavaras},
Nonlinear analysis techniques for shear band formation at high strain-rates,
% {\it Applied Mechanics Reviews}
{\it Appl. Mech. Rev.}
{\bf  45} (1992), S82--S94.



%
% \bibitem{clifton_critical_1984}
% {\sc R.~J. Clifton, J.~Duffy, K.~A. Hartley, and T.~G. Shawki},
% On critical conditions for shear band formation at high strain rates.
% % {\it Scripta Metallurgica}
% {\it Scripta. Metall. Mater.}
% {\bf 18} (1984), 443--448.
%


%
% \bibitem{freistuhler_spectral_2002}
% {\sc H.~Freistühler and P.~Szmolyan},
% {Spectral stability of small shock waves},
% {\it Arch. Ration. Mech. Anal.}
% {\bf 164} (2002), 287--309.
% %   \href{http://dx.doi.org/10.1007/s00205-002-0215-8}{doi:\nolinkurl{10.1007/s00205-002-0215-8}},
% %   \url{http://dx.doi.org/10.1007/s00205-002-0215-8}.
% \bibitem{fressengeas_instability_1987}
% {\sc C.~Fressengeas, A.~Molinari},
% {Instability and localization of plastic flow in shear at high strain rates},
% {\it J. Mech. Phys. of Solids}
% {\bf 35} (1987), 185--211.
%
% \bibitem{gasser_geometric_1993}
% {\sc I.~Gasser and P.~Szmolyan},
% {A geometric singular perturbation analysis of detonation and deflagration waves},
% {\it {SIAM} J. Math. Anal.}
% {\bf 24} (1993), 968--986.
% %   \href{http://dx.doi.org/10.1137/0524058}{doi:\nolinkurl{10.1137/0524058}},
% %   \url{http://dx.doi.org/10.1137/0524058}.
% \bibitem{ghazaryan_traveling_2007}
% {\sc A.~Ghazaryan, P.~Gordon, and C.~K. R.~T. Jones},
% {Traveling waves in porous media combustion: uniqueness of waves for small thermal diffusivity},
% {\it J. Dyn. Differ. Equ.}
% {\bf 19} (2007), 951--966.
% %   \href{http://dx.doi.org/10.1007/s10884-007-9079-9}{doi:\nolinkurl{10.1007/s10884-007-9079-9}},
% %   \url{http://dx.doi.org/10.1007/s10884-007-9079-9}.
%
%

% \bibitem{MC_1987}
% {\sc A.~Molinari and R.~J. Clifton},
% Analytical characterization of shear localization in thermoviscoplastic materials,
% {\it Journal of Applied Mechanics}
% {\it J. Appl. Mech.}
% {\bf 54} (1987), 806--812.
%
%
% \bibitem{jones_geometric_1995}
% {\sc C.~K. R.~T. Jones},
% Geometric singular perturbation theory, in {\it Dynamical systems}, LNM {\bf 1609} (Springer Berlin Heidelberg 1995) 44--118.
%
%
%
%
%
%

%
% \bibitem{KUEHN_2015}
% {\sc C.~ Kuehn},
% {\it Multiple time scale dynamics}, Applied Mathematical Sciences, Vol. {\bf 191} (Springer Basel 2015).
%

%
% \bibitem{perko_differential_2001}
% {\sc L.~Perko},
% {\it Differential equations and dynamical systems 3rd. ed.}, TAM {\bf 7} (Springer-Verlag New York 2001).
%
%



%
% \bibitem{shawki_energy_1994}
% {\sc T.~G. Shawki},
% {An Energy Criterion for the Onset of Shear Localization in Thermal Viscoplastic Materials, Part II: Applications and Implications}, {\it ASME. J. Appl. Mech.}
% {\bf 61} (1994), 538--547.
%
%
% \bibitem{SS_2004}
% {\sc S. Schecter and P. Szmolyan}
% Composite waves in the Dafermos regularization.
% {\it J. Dynamics Diff. Equations} {\bf 16} (2004), 847-867.
%

%
% \bibitem{wiggins_normally_1994}
% {\sc S.~Wiggins},
% {\it Normally hyperbolic invariant manifolds in dynamical  systems}, AMS {\bf 105} (Springer-Verlag New York 1994).
%
\bibitem{wiggins_normally_1994}
{\sc S.~Wiggins}, 
{\it Normally hyperbolic invariant manifolds in dynamical  systems}, AMS {\bf 105} (Springer-Verlag New York 1994).

\bibitem{wright_survey_2002}
{\sc T.W. Wright},
{\it The Physics and Mathematics of Shear Bands.} (Cambridge Univ. Press 2002).
%
% \bibitem{xiao_stability_2003}
% {\sc L.~Xiao-Biao and S.~ Schecter},
% {Stability of self-similar solutions of the {D}afermos regularization of a system of conservation laws},
% {SIAM J. Math. Anal.}
% {\bf 35} (2003), 884--921.

%\bibitem{WF83}
%{\sc F.H. Wu and L.B. Freund},
%Deformation trapping due to thermoplastic instability in one-dimensional wave propagation,
%{\it J. Mech. Phys. of Solids} {\bf  32} (1984), 119-132.

\bibitem{zener_effect_1944}
{\sc C.~Zener and J.~H. Hollomon},
Effect of strain rate upon plastic flow of steel,
% {\it  Journal of Applied Physics}
{\it J. Appl. Phys.}
{\bf 15} (1944), 22--32.

\end{thebibliography}
\end{document}

% \hrulefill
% \bibitem{dafermos_adiabatic_1983}
% {\sc C.~M. Dafermos and L.~Hsiao},
% Adiabatic shearing of incompressible fluids with temperature-dependent viscosity.
% {\it Quart.  Applied Math.} {\bf 41} (1983), 45--58.

% \bibitem{katsaounis_effective_2009}
% {\sc Th. Katsaounis and A.E.~Tzavaras},
%  Effective equations for localization and shear band formation,
%  {\it SIAM J. Appl. Math.}  {\bf 69} (2009), 1618--1643.

% \bibitem{schecter_undercomp_2002}
% {\sc S.~Schecter},
% {Undercompressive shock waves and the {D}afermos regularization},
% {\it Nonlinearity}
% {\bf 15} (2002), 1361--1377.
% \bibitem{deng_homoclinic_1990}
% {\sc B.~Deng},
% {Homoclinic bifurcations with nonhyperbolic equilibria},
% {\it {SIAM} J. Math. Anal.}
% {\bf 21} (1990),  693--720.
% %   \href{http://dx.doi.org/10.1137/0521037}{doi:\nolinkurl{10.1137/0521037}},
% %   \url{http://dx.doi.org/10.1137/0521037}.

%
% \bibitem{ghazaryan_travelling_2015}
% {\sc A.~Ghazaryan, V.~Manukian, and S.~Schecter},
% {Travelling waves in the holling-tanner model with weak diffusion},
% {\it Proc. R. Soc. A}
% {\bf 471} (2015), 20150045, 16.
% %   \href{http://dx.doi.org/10.1098/rspa.2015.0045}{doi:\nolinkurl{10.1098/rspa.2015.0045}},
% %   \url{http://dx.doi.org/10.1098/rspa.2015.0045}.
%
% \bibitem{gucwa_geometric_2009}
% {\sc I.~Gucwa and P.~Szmolyan},
% {Geometric singular perturbation analysis of an autocatalator model},
% {\it Discrete Contin. Dyn. Syst. Ser. S}
% {\bf 2} (2009), 783--806.
% %   \href{http://dx.doi.org/10.3934/dcdss.2009.2.783}{doi:\nolinkurl{10.3934/dcdss.2009.2.783}},
% %   \url{http://dx.doi.org/10.3934/dcdss.2009.2.783}.
%
% \bibitem{huber_geometric_2005}
% {\sc A.~Huber and P.~Szmolyan},
% {Geometric singular perturbation analysis of the yamada model},
% {\it {SIAM} J. Appl. Dyn. Syst.}
% {\bf 4} (2005), 607--648.
% %   \href{http://dx.doi.org/10.1137/040604820}{doi:\nolinkurl{10.1137/040604820}},
% %   \url{http://dx.doi.org/10.1137/040604820}.
% \bibitem{jones_construction_1991}
% {\sc C.~K. R.~T. Jones, N.~Kopell, and R.~Langer},
% {Construction of the {FitzHugh}-nagumo pulse using differential forms.} In: {\it Patterns and dynamics
%   in reactive media},
% IMA Volumes Math Appl 37, Springer, 1989, 101--115.
%
% \bibitem{jones_tracking_1994}
% {\sc C.~K. R.~T. Jones and N.~Kopell},
% {Tracking invariant manifolds with differential forms in singularly perturbed systems},
% {\it J. Differ. Equations}
% {\bf 108} (1994), 64--88.
% %   \href{http://dx.doi.org/10.1006/jdeq.1994.1025}{doi:\nolinkurl{10.1006/jdeq.1994.1025}},
% %   \url{http://dx.doi.org/10.1006/jdeq.1994.1025}.
% \bibitem{kaper_primer_2001}
% {\sc T.~J. Kaper and C.~K. R.~T. Jones},
% {A primer on the exchange lemma for fast-slow systems.} In: {\it Multiple-time-scale dynamical systems},
% IMA Volumes Math Appl 122, Springer, 1997, 65--87.
% \bibitem{katsaounis_localization_2011}
% {\sc Th. Katsaounis and A.E.~Tzavaras},
% Localization and shear bands in high strain-rate plasticity.
% In: {\it Nonlinear conservation laws and  applications}, A.~Bressan, G.-Q.~Chen, M.~Lewicka, D.~Wang, eds;
% IMA Volumes Math Appl 153, Springer, 2011, 365--377.
% \bibitem{popovic_geometric_2004}
% {\sc N.~Popović and P.~Szmolyan},
% {A geometric analysis of the lagerstrom model problem},
% {\it J. Differ. Equations}
% {\bf 199} (2004), 290--325.
% %   \href{http://dx.doi.org/10.1016/j.jde.2003.08.004}{doi:\nolinkurl{10.1016/j.jde.2003.08.004}},
% %   \url{http://dx.doi.org/10.1016/j.jde.2003.08.004}.
% \bibitem{bates_existence_1997}
% {\sc P.~W. Bates, P.~C. Fife, R.~A. Gardner, and C.~K. R.~T. Jones},
% {The existence of travelling wave solutions of a generalized phase-field model},
% {\it {SIAM} J. Math. Anal.}
% {\bf 28} (1997), 60--93.
% %   \href{http://dx.doi.org/10.1137/S0036141095283820}{doi:\nolinkurl{10.1137/S0036141095283820}},
% %   \url{http://dx.doi.org/10.1137/S0036141095283820}.
%
% \bibitem{beck_electrical_2008}
% {\sc M.~Beck, C.~K. R.~T. Jones, D.~Schaeffer, and M.~Wechselberger},
% {Electrical waves in a one-dimensional model of cardiac tissue},
% {\it {SIAM} J. Appl. Dyn. Syst.}
% {\bf 7} (2008),  1558--1581.
% %   \href{http://dx.doi.org/10.1137/070709980}{doi:\nolinkurl{10.1137/070709980}},
% %   \url{http://dx.doi.org/10.1137/070709980}.

% \bibitem{schecter_exchange_2008}
% {\sc S.~Schecter},
% {Exchange lemmas. i. deng's lemma},
% {\it J. Differ. Equations}
% {\bf 245} (2008), 392--410.
% %   \href{http://dx.doi.org/10.1016/j.jde.2007.08.011}{doi:\nolinkurl{10.1016/j.jde.2007.08.011}},
% %   \url{http://dx.doi.org/10.1016/j.jde.2007.08.011}.
%
% \bibitem{schecter_exchange_2008-1}
% {\sc S.~Schecter}, {Exchange lemmas. {ii}. general exchange lemma},
% {\it J. Differ. Equations}
% {\bf 245} (2008), 411--441.
% %   \href{http://dx.doi.org/10.1016/j.jde.2007.10.021}{doi:\nolinkurl{10.1016/j.jde.2007.10.021}},
% %   \url{http://dx.doi.org/10.1016/j.jde.2007.10.021}.

% \bibitem{wright_stress_1987}
% {\sc Thomas~W. Wright and John~W. Walter},
% On stress collapse in adiabatic shear bands,
% {\it J. Mech. Phys. of Solids} {\bf 35} (1987),
%  701--720.
%
% \bibitem{WF83}
% {\sc F.H. Wu and L.B. Freund},
% Deformation trapping due to thermoplastic instability in one-dimensional wave propagation,
% {\it J. Mech. Phys. of Solids} {\bf  32} (1984), 119-132..
%
% \bibitem{AKS87}
% L.~Anand, K.H.~Kim and T.G.~Shawki,
% Onset of shear localization in viscoplastic solids,
% {\it J. Mech. Phys. Solids}
% {\bf 35} (1987), 407-429.
%
% \bibitem{BC04}
% {\sc Th.~ Baxevanis and N.~Charalambakis},
% The role of material non-homogeneities on the formation and evolution of strain non-uniformities in thermoviscoplastic shearing,
% {\it Quart. Appl. Math.} {\bf 62} (2004), . 97-116.
%
% \bibitem{baxevanis_adaptive_2010}
% {\sc Th~H. Baxevanis, Th~Katsaounis, and A.~E. Tzavaras},
% Adaptive finite element computations of shear band formation,
%   {\it Math. Models  Methods Appl. Sci.} {\bf 20}  (2010),  423--448.
%
% \bibitem{BD02}
% {\sc T.J.~Burns and M.A.~Davies},
% On repeated adiabatic shear band formation during high speed machining,
% {\it International Journal of Plasticity} {\bf 18 } (2002),  507-530.
%
% \bibitem{CB99}
% {\sc L.~ Chen and R.C.~Batra },
% The asymptotic structure of a shear band in mode-II deformations.
% {\it International Journal of Engineering Science} {\bf 37} (1999),  895-919.
%
% \bibitem{estep_2001}
% {\sc Donald~J Estep, Sjoerd M~Verduyn Lunel, and Roy~D Williams},
% {Analysis of Shear Layers in a Fluid with Temperature-Dependent Viscosity},
%  {\it  J. Comp. Physics}  {\bf 173} (2001), 17--60.
%
% \bibitem{KCS85}
% {\sc R.W. Klopp. R.J. Clifton, and T.G. Shawki},
% Pressure-shear impact and the dynamic viscoplastic response of metals,
% {\it Mechanics of Materials} {\bf 4} (1985), 375-385.

% \bibliography{dynamical}
\end{thebibliography}
\end{document}


\end{thebibliography}

\end{document}	

\pagebreak