%------------------------------------------------------------------------------
% Beginning of journal.tex
%------------------------------------------------------------------------------
%
% AMS-LaTeX version 2 sample file for journals, based on amsart.cls.
%
%        ***     DO NOT USE THIS FILE AS A STARTER.      ***
%        ***  USE THE JOURNAL-SPECIFIC *.TEMPLATE FILE.  ***
%
% Replace amsart by the documentclass for the target journal, e.g., tran-l.
%
\documentclass{amsart}
 \usepackage[notref,notcite]{showkeys}
\usepackage{color}
\def\red{\color{red}}
\def\blue{\color{blue}}

\def\tbr{{\blue{to be revised}}}
\def\bG{\bar{\Gamma}}
\def\bS{\bar{\Sigma}}
\def\bV{\bar{V}}
\def\bU{U_{0}}

\def\tg{\tilde{\gamma}}
\def\ts{\tilde{\sigma}}
\def\tv{\tilde{v}}
\def\tu{\tilde{u}}
\def\tit{\tilde{t}}

\def\del{\partial}
\def\eps{\varepsilon}


\def\lo{\mathcal{o}}

\newcommand{\tcr}{\textcolor{red}}
\newcommand{\tcb}{\textcolor{blue}}
\newcommand{\ubar}[1]{\text{\b{$#1$}}}


\newtheorem{theorem}{Theorem}[section]
\newtheorem{lemma}[theorem]{Lemma}

\theoremstyle{definition}
\newtheorem{definition}[theorem]{Definition}
\newtheorem{example}[theorem]{Example}
\newtheorem{xca}[theorem]{Exercise}
\newtheorem{corollary}[theorem]{Corollary}
\newtheorem{proposition}[theorem]{Proposition}
\theoremstyle{remark}
\newtheorem{remark}[theorem]{Remark}

\numberwithin{equation}{section}

%    Absolute value notation
\newcommand{\abs}[1]{\lvert#1\rvert}

%    Blank box placeholder for figures (to avoid requiring any
%    particular graphics capabilities for printing this document).
\newcommand{\blankbox}[2]{%
  \parbox{\columnwidth}{\centering
%    Set fboxsep to 0 so that the actual size of the box will match the
%    given measurements more closely.
    \setlength{\fboxsep}{0pt}%
    \fbox{\raisebox{0pt}[#2]{\hspace{#1}}}%
  }%
}

\begin{document}
 
\title{Infinitely many segregated vector solutions of Schrodinger system} 



%    Information for second author
\author{Ohsang Kwon}
\address{Department of Mathematics, Chungbuk National University, Cheongju, South Korea}
\email{ohsangkwon@chungbuk.ac.kr}
\thanks{O. Kwon is supported by }

\author{Min-Gi Lee}
\address{Department of Mathematics, Kyungpook National University, Daegu, South Korea}
\email{leem@knu.ac.kr}
\thanks{M.-G Lee is supported by }

\author{Youngae Lee}
\address{Department of Mathematical Sciences, College of Natural Sciences,  Ulsan National Institute of Science and Technology (UNIST), South Korea}
\email{youngaelee@unist.ac.kr}
\thanks{ Y. Lee is supported by the National Research Foundation of Korea(NRF) grant funded by the Korea government(MSIT) (No. NRF-2018R1C1B6003403). }

%    General info
\subjclass[2020]{Primary 35J50, 35Q55, 35B40 ; Secondary  35B45, 35J40}

\date{\today}

%\dedicatory{This paper is dedicated to our advisors.}

\keywords{Coupled Schrodinger system, segregation, vector solution, distribution of bump, energy expansion}

\begin{abstract}
 
\end{abstract}

\maketitle

%\section*{This is an unnumbered first-level section head}
%This is an example of an unnumbered first-level heading.

%% The correct journal style for \specialsection is all uppercase; a known bug
%% in amsart.cls prevents this, so input must be uppercase until it is fixed.
%\specialsection*{This is a Special Section Head}
%\specialsection*{THIS IS A SPECIAL SECTION HEAD}
%This is an example of a special section head%
%%%%%%%%%%%%%%%%%%%%%%%%%%%%%%%%%%%%%%%%%%%%%%%%%%%%%%%%%%%%%%%%%%%%%%%%
%\footnote{Here is an example of a footnote. Notice that this footnote
%text is running on so that it can stand as an example of how a footnote
%with separate paragraphs should be written.
%\par
%And here is the beginning of the second paragraph.}%
%%%%%%%%%%%%%%%%%%%%%%%%%%%%%%%%%%%%%%%%%%%%%%%%%%%%%%%%%%%%%%%%%%%%%%%%

\section{Introduction} \label{sec:intro}
 


We consider the following system of Schr\"odinger equations
\begin{equation}\left.\begin{cases}\label{system}
 -\Delta U + \lambda U = \alpha_0 U^3+ \beta UV^2\\
 -\Delta V + \mu(y) V = \alpha_1 V^3+\beta U^2V 
\end{cases}\right. \text{in} \quad \mathbb{R}^N, \ N=2,  3.\end{equation}
where  $\lambda$, $\alpha_0$, $\alpha_1>0$ are positive constants,    $\beta \in \mathbb{R}$ is the  coupling constant, and $\mu: \mathbb{R}^N \rightarrow \mathbb{R}$ is a potential function.  Our objective is to construct infinitely many solution of \eqref{system} where the radial symmetry is broken exhibiting many peaks over a circle and at the origin.

This work answers a question among many problems from exploring the radial symmetry, or else breaking of it, of the solutions of non-linear Schr\"odinger equations. By the work of  Gidas et. al. \cite{GNN} via the moving plane method, the following radial symmetry result is well-known, i.e., for the scalar Schr\"odinger equation
\begin{equation} \label{single}
 -\Delta U+ \mu(|y|) U = |U|^{p-1}U \quad \text{in $ \mathbb{R}^N$} %\quad 1<p < \frac{N+2}{N-2}}
\end{equation}
if the potential function $\mu$ is radial and non-decreasing in the radial variable, then the solution of \eqref{single} must be radial. Thus, the solutions with interesting patterns breaking the radial symmetry, such as solutions with many peaks, can only exist after violating the assumptions of this problem. Further, it is also an interesting question if the cross interactions between species, where one considers then the system of coupled Schr\"odinger equations, provide different circumstances in nature so that non-radial solutions exist.

For the scalar Schr\"odinger equation, by omitting the non-decreasing condition, i.e., $\mu$ is merely radial, such an example was presented by Wei and Yan \cite{wei_yan_2014}. They considered a $\mu$ that possesses the prescribed asymptotic behavior
\begin{align} \label{asymp0}
 \mu(|y|) = \mu_0 + \frac{a}{|y|^m} + \mathcal{O}\left(\frac{1}{|y|^{m+\theta}}\right) \quad \text{for some $\mu_0,a,\theta>0$ and $m>1$ as $|y| \rightarrow \infty$}.
\end{align}
The constructed solution $u$ has segregated $k$ peaks over a circle. More specifically, $u$ peaks about $k$ points $x_i=(z_i,\mathbf{0})$, $i=1,\cdots,k$ where
$$z_i= \left( R \cos\left(\frac{2(i-1)\pi}{k}\right), ~R \sin\left(\frac{2(i-1)\pi}{k}\right)\right)\in \mathbb{R}^2 \quad \text{for $i=1,\cdots,k$}$$
and the radius $R$ is sufficiently large dependently on the number of peaks $k$. They showed that there are infinitely many such solutions for every $k$ larger than a certain $k_0$. Cerami, et.al. \cite{cerami_passaseo_solimini_2015} showed that such a result holds when the radial symmetry of $\mu$ is also broken. 

Also, it has been actively studied if radial symmetry result of scalar case persists for the problems of coupled Schr\"odinger systems. If $n$-species are interacting, we then employ $n$ potential functions. We adopt a naive classification by behaviors of potentials here, we will speak of a potential that is a constant function, an important case among non-decreasing potentials, or a potential that is an increasing function, and so on. In regard with this, for the prescribed asymptotic behaviors similar to \eqref{asymp0},  asymptotically $a\le 0$ corresponds to the non-decreasing potential while $a>0$ corresponds to the decreasing potential.

Having those said, we remark a few relevant works. Peng and Wang \cite{peng_wang_2013} considered two species Schr\"odinger system 
\begin{equation*}\begin{cases}
 -\Delta U+ P(|x|)U = \alpha_0 U^3 + \beta V^2 U,\\
 -\Delta V+ Q(|x|)V = \beta V^3 + \alpha_0 U^2 V,
\end{cases}\end{equation*}
and constructed various interesting solutions. They generated solutions of patterns where two species peak in a synchronized manner at shared sites or in a segregated manner at respective sites over a circle. The assumption on the potential function were such that one of $P(\cdot)$ and $Q(\cdot)$ violates the non-decreasing condition, or some other sophisticated conditions were supposed. See Theorem 1.1 and Theorem 1.2 in \cite{peng_wang_2013}. Long, et. al. \cite{long_tang_yang_2020} considered a problem where the two potentials are decreasing exponentially and obtained the result. The problems even without radial symmetry of potentials are studied by Ao and Wei \cite{ao_wei_2014} and Ao, et. al. \cite{ao_wang_yao_2016}. The case with nonlinearity other than $p=3$ were studied by Zheng \cite{zheng_2017}. Recently, non-radial solutions have been constructed in the three species problem where the potentials are all positive constant, thus non-decreasing, by Peng, et. al. \cite{peng_wang_wang_2019}. Their results clearly show that it is possible to have non-radial patterns in the system case even with radial and non-increasing potentials. % by contrast to the scalar case \eqref{single}. 
The results can be contrasted to those of Wei and Yan \cite{wei_yan_2014}: Not infinitely many solutions are constructed and results follow under a certain conditions on the constants.


Lin and Peng \cite{lin_peng_2014} advanced a question into another direction. They sought for a different pattern. In their solution, while one species peaks at many points over a circle as before, the other species peaks at the origin. They constructed solutions of this patterns in a model where the coupling terms are linear.

Our work in this paper can be summerized as follows. Continuing the work of Lin and Peng  \cite{lin_peng_2014} we seek for a solution of the type where one species has a peak at the origin but we work with the two species model where the coupling terms are non-linear as seen in \eqref{system}. Also, we employ one potential to be a constant function while the other potential $\mu(|y|)$ violates the non-decreasing condition. We though mention that we have made a small improvement from the work of Wei and Yan \cite{wei_yan_2014} and others that the asymptotic behavior of $\mu$ is assumed to satisfy the following with $m> \frac{1}{2}$:
\begin{equation}\tag{$A$}\label{A}
 \begin{aligned}
  &1. \quad \mu(y) = \mu_0 + \frac{a}{|y|^m} + o\left(\frac{1}{|y|^m}\right) \quad \text{for some $\mu_0,a>0$ and $m> \frac{1}{2}$ as $|y| \rightarrow \infty$.}\\
  &2. \quad \text{$\mu$ is bounded and $\mu(y)\ge \tilde{\mu}_0 > 0$ for all $y$.}
 \end{aligned}
\end{equation}
We now state our main theorem. The fraction number $f_0$ in the statement will be fixed in section \ref{sec:result}.
\begin{theorem} \label{mainthm} Suppose \eqref{A} is satisfied with $m > \frac{1}{2}$ and $\left|\frac{\beta}{\mu_0}\right| < f_0$. Then $\exists k_0$ such that for $k\ge k_0$ there exists a solution of the form $\Big(U_0 +u, \displaystyle \sum_{i=1}^k U_i + v\Big)$ of \eqref{system} where for some small constant $\epsilon>0$ $ \|u\|_{H^1},  \|v\|_{H^1} \le k^{-m -\epsilon + \frac{1}{2}}$.
\end{theorem}
\begin{remark}
  Using dilation arguments, it is sufficient to give a proof of Theorem \ref{mainthm} under the assumption $\mu_0=1$ in \eqref{A} for the  case $|\beta| \le f_0$. For the rest of the paper, $\mu_0=1$.
\end{remark}
The paper is organized in the following way: In next section, we introduce a few notions. We prove the main theorem in the section \ref{sec:result}.

\section{Notations} \label{notions}

We first let $U_{0}$ and $V_{0}$ be respectively the unique  {positive} radial solutions (see Kwong \cite{Kwong}) of
\begin{equation}\label{u0v0}
 -\Delta U_{0} + \lambda U_{0} - \alpha_0 U_{0}^3 = 0, \quad \text{and} \quad -\Delta V_{0} + V_{0} - \alpha_1 V_{0}^3 = 0,
\end{equation}
whose maxima occur at the origin. Abusing notations, $U_{0}$ and $V_{0}$ are functions of radial variable $r$. Throughout this paper, $M$ will denote the constant so that
\begin{equation}\label{exp}U_{0}(|y|) \le Me^{-\sqrt{\lambda}|y|}|y|^{-\left(\frac{N-1}{2}\right)}, \quad \text{and} \quad V_{0}(|y|) \le Me^{-|y|}|y|^{-\left(\frac{N-1}{2}\right)}.\end{equation}


We fix an integer $k>0$, which will be the number of bumps away from the origin for the second component in \ref{system}, and let \[V_{i}(y):=V_{0}(|y-x_i|)\quad\mbox{for}\  i=1,\cdots,k.\] For the points
$$z_i= \left( R \cos\left(\frac{2(i-1)\pi}{k}\right), ~R \sin\left(\frac{2(i-1)\pi}{k}\right)\right) \quad \text{for $i=1,\cdots,k$,}$$
$x_i = z_i$ in case $N=2$, and $x_i = (z_i,0)$ in case $N=3$.  Here, $R>0$ will be chosen sufficiently large later, dependently on $k$. It is also convenient to subdivide the plane $\mathbb{R}^2$ into $k$ sectors
$$\Omega^\prime_i := \left\{z \in \mathbb{R}^2~\Big|~ z\cdot z_i \ge  {R}|z|\cos\left(\frac{\pi}{k}\right)\right\} \quad i=1,\cdots,k$$
and we set $\Omega_i = \Omega^\prime_i$ if $N=2$ and $\Omega_i = \Omega^\prime_i \times \mathbb{R}$ if $N=3$.




On $H^1( \mathbb{R}^N)$, we introduce inner products
$$ \left<u,v\right>_0:= \int_{ \mathbb{R}^N} \nabla u\cdot\nabla v + \lambda uv \quad \text{and} \quad \left<u,v\right>_1:= \int_{ \mathbb{R}^N} \nabla u\cdot\nabla v + \mu(y) uv.$$ and two norms $\|\cdot\|_0:=\sqrt{\left<\cdot,\cdot\right>_0}$, and $\|\cdot\|_1:=\sqrt{\left<\cdot,\cdot\right>_1}$. We fix a closed subspace $H_s$ of $H^1( \mathbb{R}^N)$ possesing the following symmetry. Let $Q$ be a couter-clock-wise rotating matrix of an angle $\frac{2\pi}{k}$ for the first two coordinates. We define 
\begin{equation*}
 H_s:= \left\{ u \in H^1 (\mathbb{R}^N) ~\Big|~ \text{$u(Q^iy) = u(y)$ for $i=1,\cdots,k$ and $u$ is even in $y_n$, $n=2,\cdots,N$.}\right\}
\end{equation*}
Nextly, we define a  closed subspace $E:= H_s \times E_1$, where
\begin{equation}
 \begin{aligned} 
    E_1:= \left\{ v \in H_s ~\Big|~  \sum_{i=1}^k\int_{\mathbb{R}^N}  V_{i}^2 \frac{\partial V_{i}}{\partial R} \,v~dy = 0 \right\},
 \end{aligned}
\end{equation}
 and we equip $E$ with the norm \[\|(u,v)\|:= \mbox{max} \left\{ \| u\|_0, \| v\|_1\right\}.\] We look for a solution of the form $\big(U_0 +u,  {\sum^k_{i=1}} V_{i} + v\big)$ where $(u,v)\in E$ is the perturbations  with small norm.


\section{Results} \label{sec:result}
Our solution scheme is based on the observations below. Suppose $\Big(U_0 +u, \displaystyle \sum_{i=1}^k U_i + v\Big)$ is a solution of \eqref{system}, or $(u,v)$ formally solves
\begin{equation*}
 \left\{
 \begin{aligned}
 &-\Delta u + \lambda u - 3\alpha_0U_0^2 u =g_0(u,v),  \\
 &-\Delta v + \mu(y) v - 3\alpha_1\left(\sum_{i=1}^k V_{i}\right)^2 v =g_1(u,v),  
 \end{aligned}
 \right.
\end{equation*}
where
\begin{equation*}
 \left\{
 \begin{aligned}
 & g_0(u,v):=3\alpha_0U_{ {0}} u^2 + \alpha_0u^3 + \beta(U_0+u)\left(\sum_{i=1}^k V_{i} +v\right)^2,\\
 & g_1(u,v):= 3\alpha_1\left(\sum_{i=1}^k V_{i}\right)v^2 + \alpha_1v^3 + \beta\left(U_0 +u\right)^2\left(\sum_{i=1}^k V_{i} +v\right)  \\
 & \quad \quad \quad \quad \quad \quad \quad - (\mu-1)\left(\sum_{i=1}^k V_{i}\right)+ \alpha_1\left\{ \left(\sum_{i=1}^k V_{i}\right)^3 - \sum_{i=1}^k V_{i}^3 \right\}.
 \end{aligned}
 \right.
\end{equation*}
Based on these observations, for a fixed $u \in H_s$, we consider the following linear functional $\ell_u$ on $H_s$:
$$\ell_u[\varphi] := \int_{ \mathbb{R}^N}\left( \nabla u\cdot\nabla\varphi + \lambda u \varphi - 3\alpha_0 U_0^2 u\varphi\right)dy,$$
that is obviously bounded. This in turn, via Riesz representation theorem, define the linear operator $L_0: H_s\rightarrow H_s$ by the defining relation
\begin{equation}\label{l0}\left<L_0(u),\varphi\right>_0 = \ell_u[\varphi].\end{equation}
For a fixed $v \in E_1$,
$$\tilde\ell_v[\phi]:=\int_{ \mathbb{R}^N}\left( \nabla v\cdot\nabla\phi + \mu(y) v \phi - 3\alpha_1 \left(\sum_{i=1}^k V_{i}\right)^2 v\phi\right)dy,$$
and the linear operator $L_1: E_1 \rightarrow E_1$ is defined by the relation
\begin{equation}\label{l1}\left<L_1(v),\phi\right>_1 = \tilde\ell_v[\phi].\end{equation}

On the other hand, for $N=2$ or $3$, by the Sovolev embeddings and calculations we see in the proof of Proposition \ref{Fixed}, $g_0(u,v)$ and $g_1(u,v)$ {defines the bounded linear functionals {$G_0$ and $G_1$,} respectively on $H_s$ and $E_1$} 
\begin{align*}
 G_0(u,v)[\varphi]:=\int_{ \mathbb{R}^N} g_0(u,v)\varphi \quad \text{for $\varphi \in H_s$}, \quad
 G_1(u,v)[\phi]:=\int_{ \mathbb{R}^N} g_1(u,v)\phi \quad \text{for $\phi \in E_1$}.
\end{align*}
Applying Riesz representation theorem, there exist $\Gamma_0(u,v) \in E_0$, $\Gamma_1(u,v) \in E_1$ with
$$\left<\Gamma_0(u,v),\varphi\right>_0 = G_0(u,v)[\varphi] \quad \text{for $\varphi \in E_0$}, \quad \left<\Gamma_1(u,v),\phi\right>_1 = G_1(u,v)[\phi] \quad \text{for $\phi \in E_1$}.$$
Combined with the inverses $L_0^{-1}$ and $L_1^{-1}$ on $E_0$ and $E_1$ respectively (See Lemma \ref{inv0}), the map
$$(u,v) \mapsto (\bar{u},\bar{v})=\left(L_0^{-1}\Big(\Gamma_0(u,v)\Big), ~L_1^{-1}\Big(\Gamma_1(u,v)\Big)\right)$$
defines an operator $F: E \rightarrow E$. If $(u,v)$ is any fixed point of $F$ then it holds that
\begin{align*}
 &\left(L_0(u),L_1(v)\right) = \left(\Gamma_0(u,v),\Gamma_1(u,v)\right)\end{align*}
 if and only if 
 \begin{align*}
 \left\{ 
 \begin{aligned}
  &\int_{\mathbb{R}^N} \nabla u \cdot \nabla \varphi + \lambda u\varphi - 3\alpha_0U_0^2 u\varphi = \int_{\mathbb{R}^N} g_0(u,v)\varphi \quad\mbox{for all}\  \varphi\in H_s,\\
  &\int_{\mathbb{R}^N} \nabla v \cdot \nabla \phi + \mu(y) v\phi - 3\alpha_1\left(\sum_{i=1}^k V_{i}\right)^2v\phi = \int_{\mathbb{R}^N} g_1(u,v)\phi \quad \mbox{for all}\ \phi \in E_1.
 \end{aligned}
 \right.
\end{align*}

Now we state that the kernel of operators $-\Delta +(1+3U_0^2)\textrm{Id}$,  $-\Delta +(1+3V_0^2)\textrm{Id}$ in $H_s$ is null, indeed,  the following results of the invertibilities of $L_0$ and $L_1$  are known.
\begin{lemma} \cite{Kwong},  \cite[Lemma 2.1]{wei_yan_2014}  \label{inv0} There are constants $\rho_0, \rho_1>0$ satisfying
\begin{align*}
\rho_0 \|\varphi\|_0 &\le  \|L_0 \varphi\|_0  \le (\rho_0)^{-1}\|\varphi\|_0  \quad \text{for all $\varphi \in H_s$},\ \mbox{and} \\
\rho_1 \|\phi\|_1 &\le  \|L_1 \phi\|_1 \le (\rho_1)^{-1}\|\phi\|_1 \quad \text{for all $\phi \in E_1$}
\end{align*}
\end{lemma}



 
 



Nextly, we  establish the uniform bound of the sum $\displaystyle \sum_{i=1}^k V_{i}$ independently of the number of bumps $k$, provided that $\frac{R}{k}$ is sufficiently large. We provide the proof of a tweaked version of Lemma A.1 in \cite{wei_yan_2014}.
\begin{lemma}\label{ksum}  \cite[Lemma A.1]{wei_yan_2014} Let $\eta \in (0,1)$ and suppose $\frac{R}{k}$ is such that $e^{-\frac{2\eta R}{k}} \le \frac{1}{2}$. Then for $y \in{\Omega}_1$, we have 
$$\displaystyle \sum_{i=2}^k V_{i}(y) \le 4Me^{-\frac{2\eta R}{k}} e^{-(1-\eta)|y-x_1|} \quad\text{and} \quad \displaystyle \sum_{i=1}^k V_{i}(y) \le 4Me^{-(1-\eta)|y-x_1|}.$$
\end{lemma}
\begin{proof}

For $y\in \Omega_1$, it holds that $|y-x_i|\ge |y-x_1|$ and that $|y-x_1| \ge |x_i-x_1|-|y-x_i|$, resulting in $|y-x_i|\ge \frac{1}{2}|x_i-x_1|$. Together with \eqref{exp}, we have that
$$\sum_{i\ge 2} V_{i}(y) \le \sum_{i\ge 2} Me^{-(1-\eta)|y-x_i|} e^{-\eta|y-x_i|}
\le Me^{-(1-\eta)|y-x_1|}\sum_{i\ge 2} e^{-\frac{\eta}{2}|x_i-x_1|} .$$
We let $\theta_i = \frac{2\pi(i-1)}{k}$. Then $ \frac{|x_i-x_1|}{2} = R\left|\sin\left(\frac{\theta_i}{2}\right)\right|$ and using that $ \sin\left(\frac{\theta_i}{2}\right) \ge \frac{\theta_i}{\pi}$ for $0\le\theta_i \le \pi$, we see that  if the common ratio $e^{-\frac{2\eta R}{k}} \le \frac{1}{2}$, then
\begin{align}\label{sum2}
 \sum_{i\ge 2} e^{-\frac{\eta}{2}|x_i-x_1|} &\le 2\sum_{0<\theta_i \le \pi} e^{-\frac{\eta R}{\pi}\theta_i} \le 4 e^{-\frac{2\eta R}{k}}.
\end{align}
Assertions then straightforwardly follow.
 \end{proof}
In view of Lemma \ref{ksum}, there exists  $\gamma_0 = \gamma_0(M)$ with $\displaystyle \max\left\{U_0^2 , \left(\sum_{i=1}^k V_{i}\right)^2\right\} \le \gamma_0.$
We fix a number $f_0$ such that \begin{equation}\label{f0}0< f_0 < \frac{1}{\gamma_0},\end{equation}
and let \begin{equation}\label{sk}{S_k=\left[\left(\frac{m}{2\pi} -\delta_0\right)k\log k,\left(\frac{m}{2\pi}+\delta_0\right)k\log k\right],}\end{equation}
{where $\delta_0>0$ is a small constant, independent of $k$.}

\begin{proposition} \label{Fixed} $\exists k_0>0$ such that for $k\ge k_0$ and $R \in \Big[\frac{3m}{4}k\log k, \frac{5m}{4}k\log k\Big]$, the map $F$ has a fixed point $(u_*,v_*) \in E$. For some $\epsilon>0$ $\|(u_*,v_*)\| \le k^{\red -m -\epsilon+ \frac{1}{2}}.$
 \end{proposition}
\begin{proof}
1. Let $B \subset E$ be the ball with $\|(u,v)\| \le k^{-m + \frac{1}{2}}$. We first show that for $k$ sufficiently large and $R \in \Big[\frac{3m}{4}k\log k, \frac{5m}{4}k\log k\Big]$, $F(B) \subset B$.

Let $(\bar u, \bar v) := F(u,v)$. Then,
 \begin{align*}
  \|\bar u\|_0 &= \left\|L_0^{-1}\Big(\Gamma_0(u,v)\Big)\right\|_0 \le \frac{1}{\rho_0}\left\|\Gamma_0(u,v)\right\|_0 = \frac{1}{\rho_0} \sup_{\|\varphi\|_0=1} \left| \int_{ \mathbb{R}^N} g_0(u,v) \varphi \right|\\
  &\le  {\frac{1}{\rho_0}} \sup_{\|\varphi\|_0=1} \Bigg[   {\int_{ \mathbb{R}^N} c_0}|\varphi| \left( |u|^2 +|v|^2 |u^3| + |uv| + |uv^2|\right) + \left|  {2}\beta v U_0\left(\sum_{i=1}^k V_{i}\right)\varphi\right|dy\\
  & +  {\int_{ \mathbb{R}^N} }  \left|\beta U_0\left(\sum_{i=1}^k V_{i}\right)^2\varphi\right| +  \left|\beta u\left(\sum_{i=1}^k V_{i}\right)^2\varphi\right|dy\Bigg],
 \end{align*}
 {where a constant $c_0>0$ depends only on $\alpha_0, \beta, \lambda$. }
For $y\in \Omega_1$, we have that
 \begin{align*}
  U_0\left(\sum_{i=1}^k V_{i}\right) &\le C_0(M)e^{-\sqrt{\lambda}|y|}e^{-(1-\eta)|y-x_1|} \le C_1(M) e^{-\gamma R} \quad \text{for some small constant $\gamma>0$,}
\end{align*} where $C_0(M), C_1(M)>0$ are constants independent of $k$. 
Moreover, by the symmetry we have $\left\|U_0\left(\sum_{i=1}^k V_{i}\right)\right\|_{L^\infty(\mathbb{R}^N)} \le C(M) e^{-\gamma R}$ for some constant $C(M)>0$, independent of $k$.  We also see that 
 \begin{align*}
  U_0\left(\sum U_i\right) &\le Ce^{-\sqrt{\lambda}|y|}e^{-(1-\eta)|y-x_1|} \le C(M) e^{-\gamma R} \quad \text{for some small $\gamma>0$}
 \intertext{
and by the symmetry we have $\left\|U_0\left(\sum U_i\right)\right\|_\infty \le C(M) e^{-\gamma R}$. Also }
% \begin{align*}
 \int_{ \mathbb{R}^N}  U_0\left(\sum U_i\right)^2\varphi &\le C\sqrt{k}\left\|U_0\left(\sum U_i\right)\right\|_{L^2(\Omega_1)}\|\varphi\|_{L^2({ \mathbb{R}^N})}  \\
 &\le C\sqrt{k} \left\|e^{-\sqrt{\lambda}|y| - (1-\eta)|y-x_1|}\right|_{L^2(\Omega_1)}\\
 &\le C \sqrt{k} e^{-\gamma R}\|\varphi\|_{L^2({ \mathbb{R}^N})} \quad \text{for some small $\gamma>0$. Finally}\\
\int u\left(\sum U_i\right)^2\varphi &\le \gamma_0(M) \|u\|_{L^2({ \mathbb{R}^N})}\|\varphi\|_{L^2({ \mathbb{R}^N})}.
 \end{align*}
Therefore, we have that
\begin{equation} \label{p1}
 \|\bar u\|_0 \le C \left (k^{-2m+1} + (\sqrt{k} + 1)e^{-\gamma R}\right) + |\beta|\gamma_0(M)k^{-m + \frac{1}{2}}.
\end{equation}
Now,
 \begin{align*}
  \|\bar v\|_0 &= \left\|L_1^{-1}\Big(\Gamma_1(u,v)\Big)\right\|_1 \le \frac{1}{\rho_1}\left\|\Gamma_1(u,v)\right\|_1 = \frac{1}{\rho_1} \sup_{\|\phi\|_1=1} \left| \int_{ \mathbb{R}^N} g_1(u,v) \phi \right|\\
  &\le  {\frac{1}{\rho_1}}\sup_{\|\phi\|_1=1} \Bigg[  {\int_{ \mathbb{R}^N} c_1} |\phi|\left( |u^2| + |v^2|+ |v^3| + |uv| +|u^2v|\right)  \\
  &+ \int_{\mathbb{R}^N} \left| {2} \beta u U_0\left(\sum_{i=1}^k V_{i}\right)\phi\right| + \left|\beta U_0^2\left(\sum_{i=1}^k V_{i}\right)\phi\right| + \left|(\mu(y)-1)\left(\sum_{i=1}^k V_{i}\right)\phi\right|dy\\
  &  + \int_{\mathbb{R}^N}\left|\left(\sum_{i=1}^k V_{i}\right)^3 - \sum_{i=1}^k V_{i}^3 \right| |\phi| + \left|\beta v U_0^2  \phi\right| \Bigg]dy,
 \end{align*}
 where a constant $c_1$ depends only on $\alpha_1, \beta$. 
It is possible by \eqref{A} to choose large $k_0$ such that if $k\ge k_0$, then for $R \in S_k$, it holds that
$$ |\mu - 1| \le \frac{2a}{|y|^m} \quad \text{for $|y|\ge \frac{R}{2}$}.$$
\begin{align*}
 &\int_{\mathbb{R}^N} \left|(\mu(y)-1)\left(\sum_{i=1}^k V_{i}\right)\phi\right| dy \le \|\phi\|_{L^2(\mathbb{R}^N)} \sqrt{k}\left\| (\mu -1)\left(\sum_{i=1}^k V_{i}\right)\right\|_{L^2(\Omega_1)}\\
 &\le \|\phi\|_{L^2(\mathbb{R}^N)} \sqrt{k}\left(\left\| (\mu -1)\left(\sum_{i=1}^k V_{i}\right)\right\|_{L^2(\Omega_1 \cap B_{\frac{R}{2}(x_i)})} +  \left\| (\mu -1)\left(\sum_{i=1}^k V_{i}\right)\right\|_{L^2(\Omega_1 \setminus  B_{\frac{R}{2}(x_i)})}\right)\\
 &\le C_1\|\phi\|_{L^2(\mathbb{R}^N)} \sqrt{k} \left\| (\mu -1)\right\|_{L^\infty( B_{\frac{R}{2}(x_i)} )} \left\|e^{-(1-\eta)|y-x_1|}\right\|_{L^2(\Omega_1)} + C\|\phi\|_{L^2(\mathbb{R}^N)} \sqrt{k} e^{-\gamma R}\\
 &\le C_2\|\phi\|_{L^2(\mathbb{R}^N)} \left( \frac{\sqrt{k}}{R^m} + \sqrt{k}e^{-\gamma R}\right),
\end{align*}where $C_1, C_2>0$ are constants independent of $k$. 
Also,  we see that
\begin{align*}
 \left(\sum_{i=1}^k V_{i}\right)^3 - \sum_{i=1}^k V_{i}^3 &= \left[ V_{x_1} \sum_{i\in \hat{1}} V_{i} + U_2 \sum_{i\in \hat{2}} V_{i} + \cdots + U_k \sum_{i\in \hat{k}} V_{i} \right] \left(\sum_{i=1}^k V_{i}\right) \\
 & \le (V_{x_1} + \sqrt{\gamma_0}) \left(\sum_{i\ge 2} V_{i}\right)\left(\sum_{i=1}^k V_{i}\right) \le 2\gamma_0\left(\sum_{i\ge 2} V_{i}\right),\end{align*}
 where $\hat{\ell} = \{1,2,\cdots,\ell-1,\ell+1,\cdots,k\}$. Hence \begin{equation}\begin{aligned}\label{sumdiff}
  \int \left(\left(\sum_{i=1}^k V_{i}\right)^3 - \sum_{i=1}^k V_{i}^3 \right)\phi &\le \|\phi\|_{L^2(\mathbb{R}^N)} \sqrt{k}\left\|\left(\sum_{i=1}^k V_{i}\right)^3 - \sum_{i=1}^k V_{i}^3 \right\|_{L^2(\Omega_1)} \\
 &\le C\sqrt{k}e^{-\frac{2\eta R}{k}} \left\|e^{-(1-\eta)|y - x_1|}\right\|_{L^2(\Omega_1)} \le C\sqrt{k}e^{-\frac{2\eta R}{k}},
 \end{aligned}\end{equation}where $C>0$ is a constant independent of $k$. 
Therefore, we have that
\begin{equation} \label{p2}
 \|\bar v\|_1 \le C\left( k^{-2m+1} + (\sqrt{k}+1)e^{-\gamma R} + \frac{\sqrt{k}}{R^m} + \sqrt{k}e^{-\frac{2\eta R}{k}} \right) + |\beta|\gamma_0(M)k^{-m+ \frac{1}{2}}.
\end{equation}
Combining \eqref{p1} and \eqref{p2} with a suitable choice of $\eta \in (0,1)$ and using  the assumption $|\beta|< f_0<\frac{1}{\gamma_0}$, we have
that $$\|(u,v)\| \le k^{-m + \frac{1}{2}}.$$


2. Assertions then follows if we show that $F$ is a contraction in the ball $B$. For $(u_1,v_1)\in B$ and $(u_2,v_2) \in B$, let $(\bar{u}_1, \bar{v}_1) = F(u_1,v_1)$ and $(\bar{u}_2, \bar{v}_2) = F(u_2,v_2)$. Since we have 
\begin{align*}
&g_0(u_1,v_1) - g_0(u_2,v_2)\\
&= {(u_1 - u_2)}\left\{ 3\alpha_0U_0(u_1+u_2) + \alpha_0\big(u_1^2+u_1u_2 + u_2^2\big) + \beta \left(\sum_{i=1}^k V_{i}\right)^2 + 2\beta \left(\sum_{i=1}^k V_{i}\right) v_1 + \beta v_1^2\right\}\\
&+  {(v_1-v_2)}\beta\left\{ 2U_0 \left(\sum_{i=1}^k V_{i}\right) + U_0(v_1+v_2) + 2\left( \sum_{i=1}^k V_{i}\right)u_2 + u_2(v_1+v_2)\right\},\end{align*}and
\begin{align*}
&g_1(u_1,v_1) - g_2(u_2,v_2)\\
&= {(u_1 - u_2)}\beta\left\{ 2U_0\left(\sum_{i=1}^k V_{i}\right) + 2U_0v_1 + \left(\sum_{i=1}^k V_{i}\right)(u_1+u_2) +v_1(u_1+u_2)\right\}\\
&+  {(v_1-v_2)}\left\{ 3\alpha_1\left(\sum_{i=1}^k V_{i}\right)(v_1+v_2) + \alpha_1\big(v_1^2 + v_1v_2 + v_2^2\big) + \beta U_0^2 + 2\beta U_0u_2 + \beta u_2^2\right\},
\end{align*}
calculations in the first part of the proof show that
\begin{align*}
 \|(\bar{u}_1 - \bar{u}_2, \bar{v}_1 - \bar{v}_2)\| &\le \|(u_1 - u_2,v_1-v_2)\| \left( |\beta|\gamma_0(M) + Ck^{-m+ \frac{1}{2}}\right). \\
\end{align*}
Therefore there exists $k_0$  so large that for $k \ge k_0$, $\left( |\beta|\gamma_0(M) + Ck^{-m+ \frac{1}{2}}\right) < 1$.

\end{proof}




 
 We define
\begin{equation}
\begin{aligned}\label{energy}
I(U, V)=& \frac{1}{2} \int_{\mathbb{R}^{N}}\left(|\nabla U|^{2}+\lambda U^{2}+|\nabla V|^{2}+\mu(y) V^{2}\right) d y \\
&-\frac{1}{4} \int_{\mathbb{R}^{N}}\left(\alpha_0 U^{4}+\alpha_1 V^{4}+2 \beta U^{2} V^{2}\right) d y
\end{aligned}
\end{equation}
We note that $I \in C^{2}\left(H_{s} \times H_{s}\right)$ and a critical point of $I$ is a solution of \eqref{energy}. 
In view of Proposition \ref{Fixed}, there is a Lagrange multiplier $\Lambda_{R}$ satisfying
\begin{equation}
\left\{\begin{array}{l}
-\Delta U_{R}+\lambda U_{R}=\alpha_0 U_{R}^{3}+\beta V_{R}^{2} U_{R} \\
-\Delta V_{R}+V_{R}=\alpha_1 V_{R}^{3}+\beta U_{R}^{2} V_{R}+\Lambda_{R} \sum_{i=1}^{k} V_{i}^{2} \frac{\partial V_{i}}{\partial R}
\end{array}\right.
\end{equation}
where $\left(U_{R}, V_{R}\right)=\left(U_{0}+u_{R}, \sum_{i=1}^{k} V_{i}+v_{R}\right)$. Let
\begin{equation}\label{def_F}
F(R)=I(U_R,V_R)=I\left(U_{0}+u_{R}, \sum_{i=1}^{k} V_{i}+v_{R}\right).\end{equation} Multiplying the second equation by $\frac{\partial V_{i}}{\partial R}$
and integrating in $\mathbb{R}^{N}$, we can see that $F^{\prime}(R)=0$ implies that the constant $\Lambda_{R}$ in $(9)$ vanishes, and thus $\left(U_{R}, V_{R}\right)$ is a solution to our main equation \eqref{system}. Therefore, in order to complete the proof of Theorem \ref{mainthm}, it is enough to show that the maximization problem $\max _{R \in S_{k}} F(R)$ is achieved by an interior point of $S_{k}$.  We note that
\begin{equation}\begin{aligned}\label{FR}
F(R)&=I\left(U_{0}, \sum_{i=1}^{k} V_{i}\right)+l\left(u_{R}, v_{R}\right)+\frac{1}{2}<L\left(u_{R}, v_{R}\right),\left(u_{R}, v_{R}\right)>\\&+R\left(u_{R}, v_{R}\right)
\end{aligned}\end{equation}
where
\begin{equation}
\begin{aligned}
l(u, v)=& \int_{\mathbb{R}^{2}}(\mu(y)-1)\left(\sum_{i=1}^{k} V_{i}\right) v d y  +\int_{\mathbb{R}^{2}}\alpha_1\left(\sum_{i=1}^{k} V_{i}^{3}-\left(\sum_{i=1}^{k} V_{i}\right)^{3}\right) v d y \\
&-\beta \int_{\mathbb{R}^{N}}\left(U_{0}^{2}\left(\sum_{i=1}^{k} V_{i}\right) v+U_{0}\left(\sum_{i=1}^{k} V_{i}\right)^{2} u\right) d y,
\end{aligned}
\end{equation}
$L(u, v)=\left(L_{0}(u), L_{1}(v)\right)$  is a bounded linear operator from $H_s \times E_{1}$ to $H_s \times E_{1}$ (see \eqref{l0} and \eqref{l1} for the definition of $L_0$, $L_1$), and
\begin{equation}
\begin{aligned}
R(u, v)=&-\frac{1}{4}\int_{\mathbb{R}^{N}}\left(4\alpha_0 U_{0} u^{3}+\alpha_0 u^{4}+4\alpha_1\left(\sum_{i=1}^{k} V_{i}\right) v^{3}+ \alpha_1 v^{4}\right) d y \\
&-\frac{\beta}{2} \int_{\mathbb{R}^{N}}\left(2U_{0} u v^{2}+2\left(\sum_{i=1}^{k} V_{i}\right) u^{2} v+u^2v^2\right) d y
\\
&-\frac{\beta}{2} \int_{\mathbb{R}^{N}}\left(U_{0}^{2} v^2+4U_{0}\left(\sum_{i=1}^{k} V_{i}\right) uv+\left(\sum_{i=1}^{k} V_{i}\right)^2u^2\right) d y.
\end{aligned}
\end{equation}
 Since $l(u, v)$ is a bounded linear functional in $H_s \times E_{1}$, there is an $\bar{l}_{k} \in H_s \times E_{1}$ satisfying $l(u, v)=<\bar{l}_{k},(u, v)>$ for any $(u, v) \in H_s \times E_{1}$. Now we estimate $\bar{l}_{k}$ in the following lemma.
\begin{lemma}\label{lem_lin}   There is a constant $C>0$, independent of $k$, satisfying $\left\|\bar{l}_{k}\right\| \leq C k^{-m+1}$.\end{lemma}
\begin{proof}In view of $\left(A_{1}\right)$, we see that
\begin{equation}\label{e1}
\int_{\mathbb{R}^{2}}|\mu(y)-1| \sum_{i=1}^{k}\left|V_{i}\right||v| d y \leq k O\left(R^{-m}\right)\|v\|_{1} \leq O\left(k^{-m+1}\|v\|_{1}\right)
\end{equation}
By the similar estimation in \eqref{sumdiff}, we also get
\begin{equation}\label{e2}
\int_{\mathbb{R}^{2}}\left|\sum_{i=1}^{k} V_{i}^{3}-\left(\sum_{i=1}^{k} V_{i}\right)^{3}\right||v| d y \leq O\left(\sqrt{k} e^{-\frac{2 \eta R}{k}}\|v\|_{1}\right).
\end{equation}
From the exponential decay in \eqref{exp}, we also get
\begin{equation}\begin{aligned}\label{e3}
&  \int_{\mathbb{R}^{N}}\left|U_{0}^{2}\left(\sum_{i=1}^{k} V_{i}\right) v+U_{0}\left(\sum_{i=1}^{k} V_{i}\right)^{2} u\right|d y\\&=O\left(e^{-(1-\eta)\min\{\sqrt{\lambda},1\} R}(\|u\|_0+\|v\|_{1})\right).\end{aligned}
\end{equation}
By using the above estimations \eqref{e1}-\eqref{e3}, we obtain Lemma \ref{lem_lin}.
\end{proof}
 \begin{lemma}\label{lemma_mu}If  $k\ge k_0\ge 4$ and $R \in S_k$, then \[\int_{\Omega_{1}}(\mu(|y|)-1) V_1^{2}dy=\frac{a}{\left|x_{1}\right|^{m}}\int_{\mathbb{R}^N}V_0^2+o\left(\frac{1}{(k\ln k)^m}\right).\]  \end{lemma}
\begin{proof}We  note that $B_{\frac{2|x_1|}{k}}(x_1)\subseteq\Omega_1$  by using $|x_1|=R$, the definition of $\Omega_1$, and  $\sin x\ge \frac{2}{\pi}x$ for $x\in [0,\frac{\pi}{2}]$.
Then for $\eta\in(0,1)$,  we see from \eqref{A} that
\begin{equation}
\begin{aligned}\label{mu1}
&\int_{\Omega_1}\left(\mu\left(\left|y\right|\right)-1\right) V_1^{2}dy\\&=\int_{B_{\frac{2|x_1|}{k}}(0)}\left(\mu\left(\left|y+x_{1}\right|\right)-1\right) V_0^{2}dy+O\left(e^{-(1-\eta)\frac{4|x_1|}{k}}\right) \\
&=\int_{B_{\frac{2R}{k}}(0)}\left(\frac{a}{\left|y+x_{1}\right|^{m}}+o\left(\frac{1}{\left|y+x_{1}\right|^{m}}\right)\right) V_0^{2}dy+O\left(e^{-(1-\eta)\frac{4|x_1|}{k}}\right),
\end{aligned}
\end{equation}
here we used $\frac{2R}{k}<R$ if $k\ge k_0\ge 4$. Moreover, we see that for any $n>0$,
\begin{equation}\label{mu2}
\frac{1}{\left|y+x_{1}\right|^{{n}}}=\frac{1}{\left|x_{1}\right|^{{n}}}\left(1+O\left(\frac{|y|}{\left|x_{1}\right|}\right)\right), \quad y \in B_{\frac{2|x_1|}{k}}(0)
\end{equation}
Thus,
\begin{equation}\label{mu3}
\int_{B_{\frac{2|x_1|}{k}}(0)} \frac{1}{\left|y+x_{1}\right|^{{n}}} V_0^{2}dy=\frac{1}{\left|x_{1}\right|^{{n}}} \int_{\mathbb{R}^{N}} V_0^{2}dy+O\left(\frac{1}{\left|x_{1}\right|^{{n}+1}}+e^{-(1-\eta)\frac{4|x_1|}{k}}\right)
\end{equation}In view of \eqref{mu1}-\eqref{mu3} and $R\in S_k$, we obtain Lemma \ref{lemma_mu}.
\end{proof}
\begin{proposition}\label{A.3}If  $k\ge k_0\ge 4$ and $R \in S_k$, then 
\begin{equation*}
I\left(U_0, \sum_{i=1}^k V_{i}\right)=A_0+k\left(A_1+\frac{A_2}{R^{m}}-A_3 e^{-\frac{2R\pi}{k}}\left(\frac{k}{R}\right)^{\frac{N-1}{2}}+o\left(\frac{1}{(k\ln k)^{m}}\right)\right),
\end{equation*}
where   $A_0=\frac{\alpha_0}{4} \int_{\mathbb{R}^{N}} U_0^{4}dy,$ $
A_1=\frac{\alpha_1}{4} \int_{\mathbb{R}^{N}} V_0^{4}dy,$   $A_2=\frac{a}{2} \int_{\mathbb{R}^{N}} V_0^{2}dy,$ and $A_{3}>0$ is a positive constant, independent of $k$.\end{proposition}
\begin{proof} In view of the definition of $I$ in \eqref{energy} and the exponential decay in \eqref{exp}, we have 
\begin{equation}
\begin{aligned}\label{f1}
&I\left(U_0,  \sum_{i=1}^k V_{i}\right)\\&= \frac{1}{2} \int_{\mathbb{R}^{N}}\left(|\nabla U_0|^{2}+\lambda U_0^{2}-\frac{\alpha_0}{2} U_0^{4}\right)dy-\frac{\beta}{2}\int_{\mathbb{R}^N}U_0^{2} \left( \sum_{i=1}^k V_{i}\right)^{2} d y
\\&+
 \frac{1}{2} \int_{\mathbb{R}^{N}}\left(\left|\nabla\left( \sum_{i=1}^k V_{i}\right)\right|^{2}+\mu(y) \left( \sum_{i=1}^k V_{i}\right)^{2} -\frac{\alpha_1}{2} \left( \sum_{i=1}^k V_{i}\right)^{4}\right) d y
\\&= \frac{\alpha_0}{4} \int_{\mathbb{R}^{N}} U_0^{4}dy+
 \frac{1}{2} \int_{\mathbb{R}^{N}}\left(\mu(y)-1\right) \left( \sum_{i=1}^k V_{i}\right)^{2}  d y
\\&+
 \frac{1}{2} \int_{\mathbb{R}^{N}}\left(\left|\nabla\left( \sum_{i=1}^k V_{i}\right)\right|^{2}+ \left( \sum_{i=1}^k V_{i}\right)^{2} -\frac{\alpha_1}{2} \left( \sum_{i=1}^k V_{i}\right)^{4}\right) d y 
 \\&+O(e^{-2(1-\eta)\min\{1,\sqrt{\lambda}\}R}).
 \end{aligned}
\end{equation}In view of  the symmetry and the equation \eqref{u0v0}, we see that
\begin{equation}
\begin{aligned}\label{f2}
&\int_{\mathbb{R}^{N}}\left(\left|\nabla \left( \sum_{i=1}^k V_{i}\right)\right|^{2}+\left( \sum_{i=1}^k V_{i}\right)^{2}\right)dy\\&=\alpha_1\sum_{j=1}^{k} \sum_{i=1}^{k} \int_{\mathbb{R}^{N}} V_j^{3} V_i dy 
=\alpha_1k \left(\int_{\mathbb{R}^{N}} V_0^{4}dy+ \sum_{i=2}^{k} \int_{\mathbb{R}^{N}} V_1^{3} V_idy\right)
\end{aligned}
\end{equation}
We recall
\begin{equation}
\Omega_{j}=\left\{y=\left(y^{\prime}, y^{\prime \prime}\right) \in \mathbb{R}^{2} \times \mathbb{R}^{N-2}:\left\langle\frac{y^{\prime}}{\left|y^{\prime}\right|}, \frac{z_{j}}{\left|z_{j}\right|}\right\rangle \geq \cos \frac{\pi}{k}\right\}, \quad j=1, \ldots, k
\end{equation}
It follows from Lemma \ref{ksum}  and  Lemma \ref{lemma_mu} that
\begin{equation}
\begin{aligned}\label{f3}
&\int_{\mathbb{R}^{N}}(\mu(|y|)-1) \left( \sum_{i=1}^k V_{i}\right)^{2}dy=k \int_{\Omega_{1}}(\mu(|y|)-1) \left( \sum_{i=1}^k V_{i}\right)^{2}dy \\
&=k \int_{\Omega_{1}}(\mu(|y|)-1)\left(V_1+O\left(e^{-\frac{2\eta R}{k}} e^{-(1-\eta)\left|y-x_{1}\right|}\right)\right)^{2}dy \\
&=k \int_{\Omega_{1}}(\mu(|y|)-1) V_1^{2}dy+k O\left(\int_{\Omega_{1}}|\mu(|y|)-1| e^{-\frac{2\eta R}{k}} e^{-(1-\eta)\left|y-x_{1}\right|}dy\right) \\
&=k\left(\frac{2A_2}{\left|x_{1}\right|^{m}}+o\left(\frac{1}{(k\ln k)^m}\right)\right).
\end{aligned}
\end{equation}
For any $y \in \Omega_{1}$,
\begin{equation}
\left( \sum_{i=1}^k V_{i}\right)^{4}=V_1^{4}+4 V_1^{3} \sum_{j=2}^{k} V_j+O\left(V_1^{2}\left(\sum_{j=2}^{k} V_j\right)^{2}+V_1\left(\sum_{j=2}^{k} V_j\right)^{3}\right).
\end{equation}
In view of Lemma \ref{ksum}, we have
\begin{equation*}
\begin{aligned}
&V_1\left(\sum_{j=2}^{k} V_j\right)^{3}+V_1^{2}\left(\sum_{j=2}^{k} V_j\right)^{2} 
  \leq 16M^2  e^{-\frac{2\eta R}{k}} e^{-2(1-\eta)\left|y-x_{1}\right|} V_1 \left( \sum_{j=2}^{k} V_j \right).
\end{aligned}
\end{equation*}
Since $R\in S_k$, the estimation in \eqref{sum2} and \eqref{exp} imply 
\begin{equation}
\begin{aligned}\label{f4}
&\int_{\mathbb{R}^{N}} \left( \sum_{i=1}^k V_{i}\right)^{4}dy=k \int_{\Omega_{1}} \left( \sum_{i=1}^k V_{i}\right)^{4}dy \\
&=k \left\{\int_{\Omega_{1}}\left(V_1^{4}+4 \sum_{i=2}^{k} V_1^{3} V_i\right)dy+O\left(\sum_{i=2}^{k} e^{-   \frac{2\eta R}{k}} e^{-\left|x_{1}-x_{i}\right|}\right) \right\}\\
&=k\left\{\int_{\mathbb{R}^{N}}\left( V_1^{4}+4 \sum_{i=2}^{k}  V_1^{3} V_i\right)dy+o\left( \frac{1}{(k\ln k)^m}\right)\right\}.
\end{aligned}
\end{equation}
By using \eqref{f1}-\eqref{f4} and the definition of $A_i$, $i=0,1,2$, we get that 
\begin{equation}
\begin{aligned}\label{f5}
&I\left(U_0, \left( \sum_{i=1}^k V_{i}\right)\right)  
\\& = A_0+ 
 k\left(A_1+\frac{A_2}{\left|x_{1}\right|^{m}}-\frac{\alpha_1}{2} \left( \sum_{i=2}^{k} \int_{\mathbb{R}^{N}} V_1^{3} V_idy\right)+o\left(\frac{1}{(k\ln k)^m}\right)\right).
\end{aligned}
\end{equation}
Moreover, the estimation \eqref{exp}  and \eqref{sum2} imply that there are constants $\sigma\in(0,\frac{1}{2})$ and ${B}_{0}>0$ satisfying 
\begin{equation}
\begin{aligned}\label{es1}
&\sum_{i=2}^{k} \int_{\mathbb{R}^{N}} V_1^{3} V_idy
\\&=\sum_{i=2}^{k}\left( \int_{B_{\sqrt{|x_1-x_i|}}(0)} +\int_{\mathbb{R}^{N}\setminus B_{\sqrt{|x_1-x_i|}}(0)} \right)\left(V_0^{3}(y) V_0(y-x_i+x_1)\right)dy
\\&={B}_{0} \sum_{i=2}^{k} \left(e^{-\left|x_{1}-x_{i}\right|}|x_1-x_i|^{-\frac{N-1}{2}}\right)+O\left(\sum_{i=2}^{k} e^{-(1+\sigma)\left|x_{1}-x_{i}\right|}\right).\end{aligned}
\end{equation}
Note that $|x_1-x_i|=2R \left|\sin\left(\frac{\pi (i-1)}{k}\right)\right|\le \frac{2R\pi(i-1)}{k}$. In view of Taylor expansion of $\sin x$, we have $\sin x= x\left(1+O(x^2)\right)$ if $|x|<1$. Since $k\ge k_0\ge 4$, we see that   
\begin{equation}\label{es2}\begin{aligned}  e^{-\left|x_{1}-x_{2}\right|}|x_1-x_2|^{-\frac{N-1}{2}}= e^{-\frac{2R\pi}{k}\left(1+O\left(\frac{\pi^2}{k^2}\right)\right)}\left|\frac{2R\pi}{k}\left(1+O\left(\frac{\pi^2}{k^2}\right)\right)\right|^{-\frac{N-1}{2}}.\end{aligned}\end{equation}
Moreover, by using the estimation \eqref{sum2}, we have 
\begin{equation}\label{es3}\begin{aligned}&\sum_{i=3}^{k} \left(e^{-\left|x_{1}-x_{i}\right|}|x_1-x_i|^{-\frac{N-1}{2}}\right)+\sum_{i=2}^{k} e^{-(1+\sigma)\left|x_{1}-x_{i}\right|}\\& \le 4 \left(e^{-\frac{8|x_1|}{k}} + e^{-\frac{4(1+\sigma)|x_1|}{k}}\right).\end{aligned}\end{equation} 
Therefore,  the above estimations \eqref{es1}-\eqref{es3} imply that there is a constant $B_1>0$ satisfying  \begin{equation}
\begin{aligned}
&\sum_{i=2}^{k} \int_{\mathbb{R}^{N}} V_1^{3} V_idy = {B}_{1} e^{-\frac{2R\pi}{k}}\left(\frac{k}{R}\right)^{\frac{N-1}{2}}+O\left(e^{-\frac{4(1+\sigma)\left|x_{1}\right|}{k}}\right),
\end{aligned}
\end{equation}and thus complete the proof of  Proposition \ref{A.3}.
\end{proof}
Now we are ready to prove Theorem \ref{mainthm}.  
\begin{proof}[Proof of Theorem \ref{mainthm}]
In view of \eqref{FR} and the definition of $\bar{l}_k$, we see that 
\begin{equation}\begin{aligned}\label{FR2}
F(R)&=I\left(U_{0}, \sum_{i=1}^{k} V_{i}\right)+<\bar{l}_k,(u_{R}, v_{R})>+\frac{1}{2}<L\left(u_{R}, v_{R}\right),\left(u_{R}, v_{R}\right)>\\&+R\left(u_{R}, v_{R}\right).\end{aligned}
\end{equation}
By using Lemma \ref{lem_lin} and Proposition \ref{Fixed}, we have   $\left\|\bar{l}_{k}\right\| \leq C k^{-m+1}$, $\|L\|\le C$, and $\|(u_R,v_R)\| \le k^{-m -\epsilon+ \frac{1}{2}}$ for some constant $C>0$, independent of $k$. 
Together with \eqref{exp} and Proposition \ref{A.3}, we get that 
\begin{equation}\begin{aligned}\label{FR}
F(R)&= A_0+ 
 k\left(A_1+\frac{A_2}{R^{m}}-A_3e^{-\frac{2R\pi}{k}}\left(\frac{k}{R}\right)^{\frac{N-1}{2}}+o\left(\frac{1}{(k\ln k)^m}\right)\right).
\end{aligned}
\end{equation}
Since  the function $g(R):=\frac{A_2}{R^{m}}-A_3e^{-\frac{2R\pi}{k}}\left(\frac{k}{R}\right)^{\frac{N-1}{2}}$ has a maximum point in the interior of $S_k$,    $\max _{R \in S_{k}} F(R)$ is achieved by an interior point $R_0$ of $S_{k}$. Therefore,  we can conclude that  $(U_{R_0}, V_{R_0})$ is a solution of \eqref{system}, and thus complete the proof of Theorem \ref{mainthm}.
\end{proof}

%{\red
%To do list:
%\begin{enumerate}
% \item reference {Wang and Zhao \cite{wang_zhao_2017}}, kwong
% \item Lemma 3.1
% \item Lemma 3.2
% \item reduction and interval $S_k$
% \item proof of main theorem
% \item title
% \item abstract
% \item revision of section 3
%
%\end{enumerate}
%}



\bibliographystyle{amsplain}
\begin{thebibliography}{10}

\bibitem{ao_wang_yao_2016}
Weiwei Ao, Liping Wang, and Wei Yao, \emph{Infinitely many solutions for
  nonlinear {S}chr\"{o}dinger system with non-symmetric potentials}, Commun.
  Pure Appl. Anal. \textbf{15} (2016), no.~3, 965--989. 

\bibitem{ao_wei_2014}
Weiwei Ao and Juncheng Wei, \emph{Infinitely many positive solutions for
  nonlinear equations with non-symmetric potentials}, Calc. Var. Partial
  Differential Equations \textbf{51} (2014), no.~3-4, 761--798. 

\bibitem{cerami_passaseo_solimini_2015}
Giovanna Cerami, Donato Passaseo, and Sergio Solimini, \emph{Nonlinear scalar
  field equations: existence of a positive solution with infinitely many
  bumps}, Ann. Inst. H. Poincar\'{e} Anal. Non Lin\'{e}aire \textbf{32} (2015),
  no.~1, 23--40. 

\bibitem{GNN}
B.~Gidas, Wei~Ming Ni, and L.~Nirenberg, \emph{Symmetry of positive solutions
  of nonlinear elliptic equations in {${\bf R}^{n}$}}, Mathematical analysis
  and applications, {P}art {A}, Adv. in Math. Suppl. Stud., vol.~7, Academic
  Press, New York-London, 1981, pp.~369--402. 

\bibitem{Kwong}
Man~Kam Kwong, \emph{Uniqueness of positive solutions of {$\Delta u-u+u^p=0$}
  in {${\bf R}^n$}}, Arch. Rational Mech. Anal. \textbf{105} (1989), no.~3,
  243--266. 

\bibitem{lin_peng_2014}
Chang-Shou Lin and Shuangjie Peng, \emph{Segregated vector solutions for
  linearly coupled nonlinear {S}chr\"{o}dinger systems}, Indiana Univ. Math. J.
  \textbf{63} (2014), no.~4, 939--967. 

\bibitem{long_tang_yang_2020}
Wei Long, Zhongwei Tang, and Sudan Yang, \emph{Many synchronized vector
  solutions for a {B}ose-{E}instein system}, Proc. Roy. Soc. Edinburgh Sect. A
  \textbf{150} (2020), no.~6, 3293--3320. 

\bibitem{peng_wang_wang_2019}
Shuangjie Peng, Qingfang Wang, and Zhi-Qiang Wang, \emph{On coupled nonlinear
  {S}chr\"{o}dinger systems with mixed couplings}, Trans. Amer. Math. Soc.
  \textbf{371} (2019), no.~11, 7559--7583. 

\bibitem{peng_wang_2013}
Shuangjie Peng and Zhi-qiang Wang, \emph{Segregated and synchronized vector
  solutions for nonlinear {S}chr\"{o}dinger systems}, Arch. Ration. Mech. Anal.
  \textbf{208} (2013), no.~1, 305--339. 

\bibitem{wang_zhao_2017}
Liping Wang and Chunyi Zhao, \emph{Infinitely many solutions for nonlinear
  {S}chr\"{o}dinger equations with slow decaying of potential}, Discrete
  Contin. Dyn. Syst. \textbf{37} (2017), no.~3, 1707--1731. 

\bibitem{wei_yan_2014}
Juncheng Wei and Shusen Yan, \emph{Infinitely many positive solutions for the
  nonlinear {S}chr\"{o}dinger equations in {$\mathbb R^N$}}, Calc. Var. Partial
  Differential Equations \textbf{37} (2010), no.~3-4, 423--439. 

\bibitem{zheng_2017}
Lvzhou Zheng, \emph{Segregated vector solutions for the nonlinear
  {S}chr\"{o}dinger systems in {$\mathbb{R}^3$}}, Mediterr. J. Math. \textbf{14}
  (2017), no.~3, Paper No. 107, 21. 


\end{thebibliography}

\end{document}

%------------------------------------------------------------------------------
% End of journal.tex
%------------------------------------------------------------------------------
