%just for
% LaTeX Curriculum Vitae Template
%
% Copyright (C) 2004-2009 Jason Blevins <jrblevin@sdf.lonestar.org>
% http://jblevins.org/projects/cv-template/
%
% You may use use this document as a template to create your own CV
% and you may redistribute the source code freely. No attribution is
% required in any resulting documents. I do ask that you please leave
% this notice and the above URL in the source code if you choose to
% redistribute this file.

\documentclass[letterpaper]{article}

\usepackage{hyperref}
\usepackage{geometry}

% Comment the following lines to use the default Computer Modern font
% instead of the Palatino font provided by the mathpazo package.
% Remove the 'osf' bit if you don't like the old style figures.
%\usepackage[T1]{fontenc}
%%\usepackage[sc,osf]{mathpazo}
%\usepackage[sc]{mathpazo}
% Set your name here

\def\name{Min-Gi Lee}

% Replace this with a link to your CV if you like, or set it empty
% (as in \def\footerlink{}) to remove the link in the footer:
\def\footerlink{http://amath.kaist.ac.kr/pde_lab/}

% The following metadata will show up in the PDF properties
\hypersetup{
  colorlinks = true,
  urlcolor = black,
  pdfauthor = {\name},
  pdfkeywords = {economics, statistics, mathematics},
  pdftitle = {\name: Curriculum Vitae},
  pdfsubject = {Curriculum Vitae},
  pdfpagemode = UseNone
}

\geometry{
  body={6.5in, 8.5in},
  left=1.0in,
  top=1.25in
}

% Customize page headers
\pagestyle{myheadings}
\markright{\name}
\thispagestyle{empty}

% Custom section fonts
\usepackage{sectsty}
\sectionfont{\rmfamily\mdseries\Large}
\subsectionfont{\rmfamily\mdseries\itshape\large}

% Other possible font commands include:
% \ttfamily for teletype,
% \sffamily for sans serif,
% \bfseries for bold,
% \scshape for small caps,
% \normalsize, \large, \Large, \LARGE sizes.

% Don't indent paragraphs.
\setlength\parindent{0em}

% Make lists without bullets
\renewenvironment{itemize}{
  \begin{list}{}{
    \setlength{\leftmargin}{1.5em}
  }
}{
  \end{list}
}

\begin{document}

% Place name at left
{\huge \name}

% Alternatively, print name centered and bold:
%\centerline{\huge \bf \name}

\vspace{0.25in}

\begin{minipage}{0.45\linewidth}
    %4700 King Abdullah University of Science \& Technology\\
    KAUST\\
    Computer, Electrical and Mathematical \\Science and Engineering\\
    Thuwal 23955-6900\\
    Kingdom of Saudi Arabia
%  \href{http://www.kaist.ac.kr/}{KAIST} \\
%  Department of Mathematical Science \\
%  291 Daehak-ro, Yuseong-gu \\
%  Daejeon, 305-701, Korea
\end{minipage}
\begin{minipage}{0.45\linewidth}
  \vfil
  \begin{tabular}{ll}
    {}\\
    Phone: & +966-12-808-6142 \\
    %Fax: &  (919) 962-5678 \\
    Email: & \href{mailto:mingi.lee@kaust.edu.sa}{\tt mingi.lee@kaust.edu.sa} \\
    {}\\
    %Born on &Sep. 2, 1981 in Seoul, Korea.
    %Homepage: & \href{http://amath.kaist.ac.kr/pde_lab/members/MinGiLee}{\tt http://amath.kaist.ac.kr/}\\
    %& {\tt pde\_lab/members/MinGiLee}
%    Phone: & 82-42-350-2797 \\
%    %Fax: &  (919) 962-5678 \\
%    Email: & \href{mailto:mg.lee@kaist.ac.kr}{\tt mg.lee@kaist.ac.kr} \\
%    Homepage: & \href{http://amath.kaist.ac.kr/pde_lab/members/MinGiLee}{\tt http://amath.kaist.ac.kr/}\\
%    & {\tt pde\_lab/members/MinGiLee}
  \end{tabular}
\end{minipage}

% \bigskip
% \quad Born on Sep. 2, 1981 in Seoul, Korea.
% \section*{Personal}

% \begin{itemize}
% \item 
% \end{itemize}

\section*{Education}

\begin{itemize}
  \item Ph.D. Department of Mathematical Science, KAIST, 2014.
  \item M.S. Department of Mathematical Science, KAIST, 2009.
  \item B.S. Department of Mathematics, KAIST, 2007.
  \item \hskip 3em Minor degree from Department of Electrical Engineering.
%  \item B.S. Department of Mathematics, KAIST, 17th, Aug. 2007.
%  \item M.S. Department of Mathematical Science, KAIST, 21st, Aug. 2009.
%  \item Ph.D. Department of Mathematical Science, KAIST, 22nd, Aug. 2014.
\end{itemize}


\section*{Employment}
\begin{itemize}
  \item 2014$\sim$, Postdoctoral Fellow, King Abdullah University of Science \& Technology
\end{itemize}
%\begin{itemize}
%\item Stanford University 1927--1931.
%\item Columbia University 1931--1946.
%\item University of North Carolina, 1946--1973.
%\end{itemize}

\section*{Research Interests}
\begin{itemize}
%\item Partial Differential Equations, Numerical Simulations,
\item Continuum Mechanics, Plastic deformations, Thermo-mechanical problem,
\item Inverse Problems.
%\item Quantum Field Theory and Relativistic Field Equations.
\end{itemize}

\section*{Reference Letters}
\begin{enumerate}
\item Yong-Jung Kim %\item Partial Differential Equations, Numerical Simulations,

\quad
\begin{minipage}{0.7\linewidth}
    Professor,\\
    Department of Mathematical Sciences, KAIST\\
    291 Daehak-ro, Yuseong-gu, Daejeon, 34141, Korea\\
    email : yongkim@kaist.edu.sa\\
    phone : +82-42-350-2739\\
    homepage : \href{http://amath.kaist.ac.kr/pde_lab/members/YongJungKim/}{\tt\url{http://amath.kaist.ac.kr/pde_lab/members/YongJungKim/}}
%  Department of Mathematical Science \\
%  291 Daehak-ro, Yuseong-gu \\
%  Daejeon, 305-701, Korea
\end{minipage}
\item Marshall Slemrod

\quad
\begin{minipage}{0.7\linewidth}
    Professor Emeriti,\\
    Department of Mathematics, UW-Madison\\
    480 Lincoln Drive\\
    Madison, Wi  53706\\
    email : slemrod@math.wisc.edu\\
    phone : +1-608-263-4283 
    
    \bigskip
    Consultant,\\
    Department of Mathematics, Weizmann Institute of Science\\
    234 Herzl Street, Rehovot 7610001, Israel\\
    phone : +972-8-934-4310
\end{minipage} 
\newpage
\item Athanasios Tzavaras

\quad
\begin{minipage}{0.7\linewidth}
    Professor,\\
    Applied Mathematics and Computational Science Division\\
    King Abdullah University of Science and Technology (KAUST)\\
    Thuwal, 23955-6900\\
    Saudi Arabia\\
    email : athanasios.tzavaras@kaust.edu.sa\\
    phone : +966-12-808-0699 \\
    homepage : \href{http://www.tem.uoc.gr/~thanos.tzavaras/}{\tt\url{http://www.tem.uoc.gr/~thanos.tzavaras/}}
\end{minipage}
\item Hyeonbae Kang

\quad
\begin{minipage}{0.7\linewidth}
    Jungseok Chair Professor,\\
    Department of Mathematics,   Inha University\\
    Incheon 402-751, S. Korea\\
    email : hbkang@inha.ac.kr\\
    phone : +82-32-860-7622\\
    homepage : \href{http://math.inha.ac.kr/~hbkang }{\tt\url{http://math.inha.ac.kr/~hbkang }} 

\end{minipage} 
\end{enumerate}


%\section*{List of Publications}
\section*{List of Publications and Preprints}
%\section*{Papers and Preprints}
%\subsection*{Journal Articles}

\begin{enumerate}
\item Min-Gi Lee, Athanasios Tzavaras, Existence of localizing solutions in plasticity via the geometric theory of singular perturbations, {\it Siam J. Appl. Dyn. Systems}, to appear.
\item Theodoros Katsaounis, Min-Gi Lee, Athanasios Tzavaras, Localization in inelastic rate dependent shearing deformations, {\it J. Mech. Phys. of Solids} 98 (2017), 106--125.
\item Yong-Jung Kim, Min-Gi Lee, Existence and uniqueness in anisotropic conductivity reconstruction with Faraday's law, {\it submitted to J. Differential Equations}.
\item Min-Gi Lee, Min-Su Ko, Yong-Jung Kim, Orthotropic conductivity reconstruction with virtual resistive network and Faraday's law, {\it Math. Methods Appl. Sci.} 39 (2016), 1183-1196.% \href{http://dx.doi.org/10.1002/mma.3564}{doi:10.1002/mma.3564}.
\item Yong-Jung Kim, Min-Gi Lee, Well-posedness of the conductivity reconstruction from an interior current density in terms of Schauder theory, {\it Quart. Appl. Math.} 73 (2015), no.3, 419-433.
%\end{enumerate}
%
%\subsection*{Preprints}
%\begin{enumerate}
\item Yong-Jung Kim, Min-Gi Lee and Marshall Slemrod, Thermal creep of a rarefied gas on the basis of non-linear Korteweg-theory, {\it Arch. Ration. Mech. Anal.} 214 (2015), no.2, 353-379.
\item Min-Gi Lee, Yong-Jung Kim, Min-Su Ko, Virtual Resistive Network and Conductivity Reconstruction with Faraday's law, {\it Inverse Problems.} 30 (2014), no. 12, 125009-125029.
%\item Min-Gi Lee, A modification on the Ivanenko-Landau-K\"ahler equation. (preprint) \href{http://amath.kaist.ac.kr/pde_lab/members/MinGiLee/pdfs/003ilk.pdf}{[pdf]}
\item Tae Hwi Lee, Hyun Soo Nam, Min-Gi Lee, Yong-Jung Kim, Eung Je Woo, and Oh In Kwon, Reconstruction of Conductivity Using Dual Loop Method with One Injection Current in MREIT {\it Phys. Med. Biol.} 55 (2010), no.24, 7523-7539.
\item Min-Gi Lee, Network approach to conductivity recovery, (2009, Master thesis, KAIST). \href{http://amath.kaist.ac.kr/pde_lab/members/MinGiLee/pdfs/masterthesis.pdf>}{[pdf]}
\item Min-Gi Lee, Well-posedness in anisotropic electrical conductivity reconstruction (2014, Ph.D. thesis, KAIST). \href{http://amath.kaist.ac.kr/pde_lab/members/MinGiLee/pdfs/phdthesis.pdf>}{[pdf]}
\end{enumerate}

%\subsection*{Preprints}
%\begin{enumerate}
%\item Min-Gi Lee, Yong-Jung Kim, Min-Su Ko, Orthotropic conductivity reconstruction with virtual resistive network and Faraday's law, {\it Inverse Problems}, (submitted). \href{http://amath.kaist.ac.kr/papers/Kim/37.pdf}{[pdf]}
%\item Min-Gi Lee, Numerical computation of Dirac equation without spectrum degeneracy. (preprint) \href{http://amath.kaist.ac.kr/pde_lab/members/MinGiLee/pdfs/003ilk.pdf}{[pdf]}
%%\item Yong Jung Kim, Min-Gi Lee, Well-posedness of the conductivity reconstruction with interior current data and virtual resistive networks, \href{http://amath.kaist.ac.kr/pde_lab/members/MinGiLee/pdfs/24Isotropic.pdf}{[pdf]}
%%\item Min-Gi Lee, A Dirac-K\"{a}hler equation with one dirac component in Minkowski space and non-doubling lattice fermion with non-local chiral symmetries, \href{http://amath.kaist.ac.kr/pde_lab/members/MinGiLee/pdfs/manuscript1218.pdf}{[pdf]}
%\end{enumerate}

\section*{Grants}
\begin{itemize}
  \item 2011$\sim$2014, National Junior Research Fellowship.%, 60,000,000 korean won/year.
\end{itemize}

\section*{Skills and Experiences}
\begin{itemize}
  \item Capable of writing codes in C, Python, and Matlab.
  \item Have experience to implement high order Finite element methods.
  \item Have taken a mini-course on Finite volume methods, KAUST, SA, 2016.
\end{itemize}


\section*{List of Talks}
%\subsection*{Upcoming}
%\begin{itemize}
%
%\end{itemize}
\subsection*{Presentation}
\begin{itemize}
  \item {Existence of localizing solutions in plasticity via the geometric singular perturbation theory}
    \begin{itemize}
      \item Universit\"at Stuttgart, Stuttgart, Germany, August 22nd, 2016
    \end{itemize}  
  \item {Localization out of the competition of rate dependence and Hadamard instability}
    \begin{itemize}
      \item XVI International Conference on Hyperbolic Problems Theory, Numerics, Applications. Aachen, Germany, August 1-5, 2016
    \end{itemize}  
  \item {A class of solutions that appear in a model for shear band formation}
    \begin{itemize}
      \item NIMS, Daejeon, Korea January 4-8, 2016
    \end{itemize}
  \item {A coherent shear localizing solution that appears in a model for shear band formation}
    \begin{itemize}
      \item Society of Engineering Science 52nd Annual Technical Meeting in Texas A\&M, US, September 26-28, 2015
    \end{itemize}
  \item {Emergence of coherent localized structures in shear deformation}
    \begin{itemize}
      \item Non-linear evolutions : kinetic equations and defect dynamics in Hausdorff center for mathematics, Bonn, Germany, July 13-17, 2015
    \end{itemize}
  \item {Numerical computation of Dirac equation without spectrum degeneracy}
    \begin{itemize}
      \item The 9th East Asian Conference on PDE (poster session) in Nara, July 28-31, 2014
    \end{itemize}
    \item {Thermal creep of a rarefied gas on the basis of non-linear Korteweg-theory}
    \begin{itemize}
      \item Workshop in nonlinear pdes in UNIST, Ulsan, May 30-31, 2014
    \end{itemize}
    \item {Well-posedness in anisotropic conductivity reconstruction}
    \begin{itemize}
      \item HYKE seminar in SNU, Seoul, May 30, 2014
    \end{itemize}
  \item {Well-posedness of anisotropic conductivity reconstruction problem with internal current density data in two dimensions}
    \begin{itemize}
      \item Applied Inverse Problem Conference 2013 in KAIST, Daejeon, July 1-5, 2013
    \end{itemize}
  \item {Analysis of equation in Homing Guidance Loop considering first order dynamic lag}
    \begin{itemize}
      \item Korean Agency for Defense Development in Daejeon, September 20, 2012
    \end{itemize}
  \item {Analysis of equation in Homing Guidance Loop using confluent hypergeometry Kummer functions}
    \begin{itemize}
      \item Korean Agency for Defense Development in Daejeon, January 31, 2012
    \end{itemize}
  \item {Reconstruction of Anisotropic Conductivity}
    \begin{itemize}
      \item KSIAM Annual meeting in Jeju, November 25-27, 2011
    \end{itemize}
  \item {On some remarks of conductivity recovery problem}
    \begin{itemize}
      \item Analysis \& Applied mathematics seminar, KAIST, February 9, 2011
    \end{itemize}
  \item {Uniqueness of anisotropic conductivity recovery with current density}
    \begin{itemize}
      \item Workshop for Young Mathematicians in Korea, in Postech, January 29, 2011
    \end{itemize}
  \item {A direct method for a conductivity recovery using a resistive network}
    \begin{itemize}
      \item KSIAM 2008 Annual meeting, in Suanbo, November 28-29, 2008
    \end{itemize}
\end{itemize}

\section*{Teaching}
\subsection*{Advisory service}
\begin{itemize}
  \item Via Visiting Students Research Program, KAUST, SA May $\sim$ September, 2016 on {\it On 1-D Euler-Korteweg flow driven by temperature gradient.}
\end{itemize}
\subsection*{Assistant}
\begin{itemize}
  \item Calculus 1, KAIST, Spring, 2008
  \item Freshmen Design, KAIST, Autumn, 2008
%   \begin{itemize}
%     \item took place in english.
%   \end{itemize}
\end{itemize}




\section*{Visitings, Academic Activities}
\begin{itemize}
  \item Institute Mittag-Leffler, Djursholm, Sweden, October 24-31, 2016.
  \item Universit\"at Stuttgart, Stuttgart, Germany, August 14-27, 2016.
  \item ECNU Center for PDE, Shanghai, China, April 10-21, 2011.
\end{itemize}

\section*{Research Projects Participated as a graduate students in KAIST.}
\begin{itemize}
  \item 2010.03 $\sim$ 2012.02, Virtual Resistive Network \& a Development of an Anisotropic Conductivity Reconstruction Method for a Medical Imaging, Korea Science and Engineering Foundation, CRA
  \item 2010.09 $\sim$ 2012.12, Homing Guidance Loop Design and Analysis Based on Confluent Hypergeometric Kummer Differential Equation, Korean Agency for Defense Development, RA
\end{itemize}



\section*{Schools Attended}
%
\begin{itemize}
  \item Korea PDE School \#4, NIMS, Febrary 10-14, 2014
  \item International School on Numerical Relativity and Gravitational Waves, APCTP, August 3-10, 2013
  \item Korea PDE School \#3, KAIST, January 7-11, 2013
%  \item POSTECH Summer Workshop on Application and Analysis on PDEs, Postech, June, 20-22, 2012
  \item Korea PDE School \#2, NIMS, January 9-13, 2012
%  \item The 8-th East Asia PDE Conference, Postech, December, 18-22, 2011
%  \item Conference on Kinetic Theory and Related Fields, Postech, June 22-24, 2011
  \item Intensive Lecture Series on PDE, Postech, June 20-21, 2011
  \item Korea PDE School \#1, NIMS, February 22-26,2011
  \item Workshop for Young Mathematicians in Korea, Postech, January 27-29, 2011
%  \item The 3rd workshop on PDEs of Mathematical Physics in CBNU, January 6-7, 2011
%  \item Conference for Elliptic and Parabolic PDEs in KAIST, November 5-7, 2009
%  \item NIMS Thematic Workshop on Conservation Laws, Plasma and Related Fields December in KAIST October 21-23, 2010
%  \item Conference On Nonlinear PDE's in Postech, October 5-8, 2010
%  \item KAIST-NIMS International Workshop on ��Nonlinear Partial Differential Equations��: theory, application \& numerical computation, Theme of the year: Nonlinear Conservation Laws, April 29-May 2, 2010
%  \item NIMS Hot Topics Workshop on Kinetic Theory and Fluid Dynamics in SNU, October 22-24, 2009
%  \item Asian Inverse Problem conference in Hanbat Univ., August 17-20, 2009
%  \item KSIAM 2009 Spring Conference in SNU, May 29-30 2009
%  \item NIMS Workshop and Minicourse "Mathematical Analysis, Numerics and Applications in fluid and gas dynamics" in KAIST, Jan. 19-Feb. 3, 2009
%  \item KSIAM 2008 Annual meeting, in Suanbo, November 28-29, 2008
%  \item KSIAM 2008 Spring Conference in Postech, May 30-31 2008
\end{itemize}


%
%\section*{Internal Seminars}
%\begin{itemize}
%  \item seminars on inverse problems, KAIST, Summer, 2009
%%  \item Krylov's book : Lectures on Elliptic and Parabolic equation in Holder space
%\end{itemize}

%\subsection*{Proceedings}

%\begin{itemize}
%\item A generalized T-Test and measure of multivariate dispersion,
%  Proc. Second Berkeley Symposium of Mathematical Statistics and
%  Probability, 1951.
%\end{itemize}

\bigskip

% Footer
\begin{center}
  \begin{footnotesize}
    Last updated: \today \\
    %\href{\footerlink}{\texttt{\footerlink}}
  \end{footnotesize}
\end{center}

\end{document}
