\documentclass[a4paper,11pt]{article}

\usepackage[margin=2cm]{geometry}
\usepackage{setspace}
%\onehalfspacing
% \doublespacing
%\usepackage{authblk}
\usepackage{amsmath}
\usepackage{amssymb}
\usepackage{amsthm}

\usepackage[notcite,notref]{showkeys}

% \usepackage{psfrag}
% \usepackage{graphicx,subfigure}
\usepackage{color}
\def\red{\color{red}}
\def\blue{\color{blue}}
%\usepackage{verbatim}
% \usepackage{alltt}
%\usepackage{kotex}

\usepackage{enumerate}
\usepackage[hidelinks]{hyperref}

%%%%%%%%%%%%%% MY DEFINITIONS %%%%%%%%%%%%%%%%%%%%%%%%%%%

\def\tr{\,\textrm{tr}\,}
\def\div{\,\textrm{div}\,}
\def\sgn{\,\textrm{sgn}\,}
\def\cof{\,\textrm{cof}\,}
\def\det{\,\textrm{det}\,}


\def\th{\tilde{h}}
\def\tx{\tilde{x}}
\def\tk{\tilde{\kappa}}


\def\bg{{\bar{\gamma}}}
\def\bv{{\bar{v}}}
\def\bth{{\bar{\theta}}}
\def\bs{{\bar{\sigma}}}
\def\bu{{\bar{u}}}
\def\bph{{\bar{\varphi}}}


\def\tg{{\tilde{\gamma}}}
\def\tv{{\tilde{v}}}
\def\tth{{\tilde{\theta}}}
\def\ts{{\tilde{\sigma}}}
\def\tu{{\tilde{u}}}
\def\tph{{\tilde{\varphi}}}

\def\dtg{{\dot{\tilde{\gamma}}}}
\def\dtv{{\dot{\tilde{v}}}}
\def\dtth{{\dot{\tilde{\theta}}}}
\def\dts{{\dot{\tilde{\sigma}}}}
\def\dtu{{\dot{\tilde{u}}}}
\def\dtph{{\dot{\tilde{\varphi}}}}

\def\dpp{\dot{p}}
\def\dqq{\dot{q}}
\def\drr{\dot{r}}
\def\dss{\dot{s}}

\def\ta{{\tilde{a}}}
\def\tb{{\tilde{b}}}
\def\tc{{\tilde{c}}}
\def\td{{\tilde{d}}}

\def\BO{{\mathcal{O}}}
\def\lio{{\mathcal{o}}}



\def\bx{\bar{x}}
\def\bm{\bar{\mathbf{m}}}
\def\K{\mathcal{K}}
\def\E{\mathcal{E}}
\def\H{\mathcal{H}}
\def\del{\partial}
\def\eps{\varepsilon}

\def\F{\mathbf{F}}

\newcommand{\tcr}{\textcolor{red}}
\newcommand{\tcb}{\textcolor{blue}}

\newcommand{\ubar}[1]{\text{\b{$#1$}}}
\newtheorem{theorem}{Theorem}[section]
\newtheorem{lemma}{Lemma}[section]
\newtheorem{proposition}{Proposition}[section]
\newtheorem{definition}{Definition}[section]
\newtheorem{remark}{Remark}[section]

%%%%%%%%%%%%%%%%%%%%%%%%%%%%%%%%%%%%%%%%%%%%%%%%%%%%%%%%%%
\begin{document}
\title{List of Publications}
\author{Min-Gi Lee\footnotemark[1]}
\date{}

\maketitle
\renewcommand{\thefootnote}{\fnsymbol{footnote}}
\footnotetext[1]{King Abdullah University of Science and Technology (KAUST),\\ Computer, Electrical and Mathematical Sciences \& Engineering Division (CEMSE),\\ Thuwal 23955-6900, Saudi Arabia, \\ mingi.lee@kaust.edu.sa}%; \href{https://cebumactan.github.io/ming-lee}{https://cebumactan.github.io/ming-lee}}%Computer, Electrical and Mathematical Sciences \& Engineering Division(CEMSE),\\ King Abdullah University of Science and Technology (KAUST), Thuwal, Saudi Arabia; mingi.lee@kaust.edu.sa; \href{https://cebumactan.github.io/ming-lee}}
% \footnotetext[2]{Department of Mathematics and Applied Mathematics, University of Crete, Heraklion, Greece}
% \footnotetext[3]{Institute of Applied and Computational Mathematics, FORTH, Heraklion, Greece}
% \footnotetext[4]{Corresponding author : \texttt{athanasios.tzavaras@kaust.edu.sa}}
%\footnotetext[4]{Research supported by the King Abdullah University of Science and Technology (KAUST) }
\renewcommand{\thefootnote}{\arabic{footnote}}


\maketitle

\begin{enumerate}
\item Min-Gi Lee, Athanasios Tzavaras, Existence of localizing solutions in plasticity via the geometric theory of singular perturbations, {\it Siam J. Appl. Dyn. Systems}, to appear. (arXiv:1608.00198).
\item Theodoros Katsaounis, Min-Gi Lee, Athanasios Tzavaras, Localization in inelastic rate dependent shearing deformations, {\it J. Mech. Phys. of Solids} 98 (2017), 106--125.
\item Min-Gi Lee, Min-Su Ko, Yong-Jung Kim, Orthotropic conductivity reconstruction with virtual resistive network and Faraday's law, {\it Math. Methods Appl. Sci.} 39 (2016), 1183-1196.% \href{http://dx.doi.org/10.1002/mma.3564}{doi:10.1002/mma.3564}.
\item Yong-Jung Kim, Min-Gi Lee, Well-posedness of the conductivity reconstruction from an interior current density in terms of Schauder theory, {\it Quart. Appl. Math.} 73 (2015), no.3, 419-433.
%\end{enumerate}
%
%\begin{enumerate}
\item Yong-Jung Kim, Min-Gi Lee and Marshall Slemrod, Thermal creep of a rarefied gas on the basis of non-linear Korteweg-theory, {\it Arch. Ration. Mech. Anal.} 214 (2015), no.2, 353-379.
\item Min-Gi Lee, Yong-Jung Kim, Min-Su Ko, Virtual Resistive Network and Conductivity Reconstruction with Faraday's law, {\it Inverse Problems.} 30 (2014), no. 12, 125009-125029.
%\item Min-Gi Lee, A modification on the Ivanenko-Landau-K\"ahler equation. (preprint) \href{http://amath.kaist.ac.kr/pde_lab/members/MinGiLee/pdfs/003ilk.pdf}{[pdf]}
\item Tae Hwi Lee, Hyun Soo Nam, Min-Gi Lee, Yong-Jung Kim, Eung Je Woo, and Oh In Kwon, Reconstruction of Conductivity Using Dual Loop Method with One Injection Current in MREIT {\it Phys. Med. Biol.} 55 (2010), no.24, 7523-7539.
\subsection*{Preprints}
\item Min-Gi Lee, Theodoros Katsaounis and Athanasios E. Tzavaras, Localization of Adiabatic Deformations in Thermoviscoplastic Materials, In Proceedings of the 16th International Conference on Hyperbolic Problems: Theory, Numerics, Applications (HYP2016), (preprint).
\item Yong-Jung Kim, Min-Gi Lee, Existence and uniqueness in anisotropic conductivity reconstruction with Faraday's law, {\it submitted to J. Differential Equations}.

\subsection*{Theses}
\item Min-Gi Lee, Network approach to conductivity recovery, (2009, Master thesis, KAIST). %\href{http://amath.kaist.ac.kr/pde_lab/members/MinGiLee/pdfs/masterthesis.pdf>}{[pdf]}
\item Min-Gi Lee, Well-posedness in anisotropic electrical conductivity reconstruction (2014, Ph.D. thesis, KAIST). %\href{http://amath.kaist.ac.kr/pde_lab/members/MinGiLee/pdfs/phdthesis.pdf>}{[pdf]}
\end{enumerate}

\end{document}